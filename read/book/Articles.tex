%!TEX program = xelatex
\documentclass[UTF8, 12pt, a4paper]{ctexrep}

\usepackage{graphicx} % 插图 \includegraphics[scale=0.6]{fullscreen.png}
\graphicspath{{image/}}

\usepackage[left=1.25in,right=1.25in,top=1in,bottom=1in]{geometry}

\usepackage{fontspec}
% 西文字体
\setmainfont{Ysabeau Office}
% 设置中文正文(衬线)、无衬线、等宽字体
%\setCJKmainfont{FZYouSongS 508R}
\setCJKmainfont{LXGW WenKai Mono}
%\setCJKmainfont{方正博雅方刊宋b.ttf}

\setlength{\parskip}{12pt} % 段落间隔是 \lineskip 和 \parskip 之和,这里设置 \parskip 的值是为了增加段落的间隔

%{\kaishu 楷书}{\fangsong 仿宋}
%\textbf{粗体}
%\textit{斜体}
%\tiny \scriptsize \footnotesize \small \normalsize \large \Large \LARGE \huge \Huge {\Huge huge}
%\raggedleft{右对齐} \centering{居中}

%\setcounter{secnumdepth}{0} % section 不带编号
\pagestyle{empty} % 停止文档编号
\ctexset{section={format+=\centering}}

%——
%~\\:一行空白  \\[行距]:可加入任意间距的空白行 [xpt]
%\footnote{脚注}

%\begin{scriptsize} \noindent 缩小不缩进 \end{scriptsize} 

\usepackage{ulem} % 下划线修正
%\uline{下划线}
\usepackage{CJKfntef} % 多种汉字样式下划线
%\CJKsout{汉字下划删除线}
%\CJKunderwave{汉字下划波浪线}
%\CJKunderdblline{汉字下划双线}
%\CJKunderdot{汉字下划加点}
\usepackage {enumerate} %列表编号

\usepackage{tocbibind}
\usepackage[bookmarks=true,colorlinks,linkcolor=black]{hyperref}
\usepackage{bookmark}

%\mbox{无边框盒子}
%\fbox{带边框盒子}
%\setlength{\fboxsep}{1em} \framebox[1.2\width][l]{带边框盒子,对齐参数 c(中),l(左)、r(右)、s(分散),默认居中}

%\begin{verse} 无缩进引用 \end{verse}
%\begin{quotation} 首行缩进引用 \end{quotation}

% AucTex
% C-c C-c 对主文档执行各种命令
% C-c C-v 预览 PDF
% C-c C-s 添加章节
% C-c C-e 添加各类环境

%\begin{figure}[htbp]
%\centering\includegraphics[scale=0.9]{20191231.png}
%\end{figure}

% \footnote{}

\begin{document}

\newpage

\chapter{走向资本主义还是走向社会主义 列夫·托洛茨基 1930/04/25}

〔说明〕原文作于普林基波岛,载于《反对派公报》第11期。

俄文链接:https://www.marxists.org/russkij/trotsky/works/trotm273.html

\begin{quotation}
    \small {
    过去遗留下来的城乡矛盾不可能在十几年以内消除。相反,工人国家击退了敌人,开始认真发展工业,此时这种矛盾还会变得更加激烈。由于总人口增长,以及年轻一代农民渴望独立自主,经济分化同时也以前所未有的速度发展。工业和文化的发展要不可避免地牺牲农村,这种发展很快,快到激发了农民的新利益和新需求,但它又很慢,不足以满足全体农民的这些利益与需求。因此,城市和乡村之间的矛盾已经到了极其尖锐的新阶段。孤立而落后的小农经济看不到希望,这一点仍然是矛盾的基础。

    事实证明,地主土地和落后农具组合在一起,这种条件下的生产“合作社”在经济层面完全无法持续。相反,一些大产业建立在机械化和切合实际的轮作等基础之上,它们则撑过了1905年和接下来几年的动荡,直到十月革命它们才被收归国有。当然,那时的耕作是发生在地主的土地上,但是危险之处在于,现在是人为地,也就是过早地组建了大型集体农庄,几十、几百个农民用着同样的工具,单个农民的劳动淹没在其他农民的劳动当中;由于个体积极性的丧失,土地的耕作水平甚至可能会比个体农户还要低。

    这一切可以实现,也会得到实现。但必须要正确地“计算时机”——如果做不到这一点,军事行动就会失败,经济行动也是如此。如果国际国内环境有利,农业的物质技术条件就可以在大约10-15年内得到根本性的变革,并且为集体化提供稳固的生产基础。然而,我们和这种情况之间相差的这些年份或许都足够敌人推翻苏维埃政权好几次了……

    如果说官僚主义中派的生涯就是在它们之间随机应变,那么波拿巴主义的国家机器就是要公开地切断一切传统的联系,其中甚至包括党的联系,并且作为专权的“调停人”,“自由”地在阶级之间周旋。斯大林主义是在无意识地、但也更加危险地为波拿巴主义做准备。
    }
\end{quotation}

\newpage

\section{自由主义者和孟什维克的观点}

俄罗斯的自由主义在侨居海外的这些年里没有增长半点智慧,它把一切新的经济形式,特别是集体化,都看成是在回归农奴制。最近,司徒卢威在某个地方歇斯底里地叫嚷说,俄罗斯已经回到了17世纪,只不过没有上帝。哪怕司徒卢威的评价正确无误,革命也是正确的。在旧统治阶级开明的领导下,从17世纪到20世纪的农业并没有向前走多远,所以回归也不会退得太远。无论如何,把农民从上帝那里解放出来也就是让他摆脱了一个重大的障碍。上帝是17世纪农民财务清单上必不可少的补充,祂和破犁驽马一起,组成了庄稼人的三件套——这实在是不幸。只有借助机械发动机和电力才能战胜它们。这项任务还在前面,但它会得到解决的。

自由主义假装看不见苏维埃制度巨大的经济成就,看不见能够证明“社会主义具有无可比拟的优越性”的实验证据。面对世界历史上前所未有的工业发展速度,这些来自被打倒阶级的经济学大家们只能避而不谈。而孟什维克这群资产阶级的应声虫则拿“对农民的极度剥削”当借口,反对这种发展速度。而对另一些问题,比如说,为什么英国人在印度对农民进行的剥削达不到同样的效果,为什么印度本土和英国的工业发展速度都不曾达到过接近苏联的水平,他们则完全不做讨论。麦克唐纳\footnote{詹姆斯·拉姆齐·麦克唐纳(1866—1937),英国工党创建人和领袖之一。1906-1909年任独立工党主席。1911—1914年和1922—1931年任工党议会党团主席。推行机会主义政策,鼓吹阶级合作和将资本主义逐渐演化为社会主义的理论。后世的工党多把他批评为伙同敌对党派把工党拖入深渊的叛徒。——译注}在印度枪杀了想要自主生活的工人和农民,为什么不去问问他怎么看印度的发展速度呢?不过,他们可要靠麦克唐纳和米勒来养活,恐怕不见得能“询问”这种问题……

自由派和孟什维克引用农奴制和阿拉克切耶夫制度\footnote{阿拉克切耶夫制度(Аракчеевщина)是1815—1825年俄国推行的警察专制政策和军屯制度,主要发起人是阿拉克切耶夫伯爵。在文献中,这个词经常用于负面描述,表现19世纪上半叶俄罗斯帝国平民自由受到的重大限制,以及国家行政的强势特征。——译注}的举动是典型的反动论证,是在反对历史进程当中的一切新发展。对于这种臆想的“回归”过去,黑格尔老人在他的“三段式”中给出了一个哲学公式:正题、反题、合题。想要让反题(即资本主义)永远延续的阶级会不可避免地把合题的第一步(即社会主义)看成是单纯的回归正题(即农奴制)。刽子手加利费\footnote{加斯东·亚历山大·奥古斯特,加利费侯爵(1830—1909)是一位法国将领,因参与镇压巴黎公社而闻名,被称为“公社的刽子手”。——译注}队伍当中的经济学家和哲学家就曾经指责巴黎公社有反动倾向,说它想让现代社会退回到中世纪的公社。对此,马克思这样写道:

“……新的历史创举通常遭到的命运就是被误认为是对旧的、甚至已经过时的社会生活形式的抄袭,只要它们稍微与这些形式有点相似。”(《法兰西内战》)今天的资产阶级批评家并没能想出什么新办法。况且要去哪里找新办法呢?俄罗斯自由主义和俄罗斯“民主制度”的“意识形态”不过是在剽窃,而且还来得极其迟缓。难怪司徒卢威32年前就这样写过:“越往东走,资产阶级就越虚弱、越卑鄙。”历史给这段话补充了一句:“它的民主制也是如此。”

司徒卢威正在重提他1893年的口号:“让我们向资本主义学习!”不同的是,这个口号在四十年前毕竟还是意味着向前走,而现在则是在向后退。要知道沙皇俄国向资本主义学习,主要的成果就是十月革命。和那个著名的俄罗斯谚语\footnote{“学习的根脉苦涩,但它的果实甜蜜。”(Корень учения горек, да плод его сладок.)——译注}相反,对于“老师”来说,学习的根是甜的,而果实却非常苦涩。那么,在恢复资本主义的时候,要怎样才能保证不结出这种“果实”呢?在这方面,俄罗斯的资产阶级发现,那些文明国家的“富足”非常有问题(而且完全无法持续),除此之外他们在国外就没有半点收获了。但问题就在于此:尽管新国家里充满了旧国家的罪恶,但它向资本主义学习却完全不是旧国家历史的重复。十月革命是世界资产阶级在其最薄弱环节发生的崩溃。他们幻想俄国能在十月革命之后回到和平的资本主义,这是最离奇、最愚钝的空想;确保资本主义在中国和印度和平发展都要“容易”得多。顺便说一下,现在后者的权力正由第二国际掌握着\footnote{印度国民大会党是社会党国际的正式成员,同时麦克唐纳领导的英国工党正在执政。托洛茨基写作此文时,由甘地领导的食盐进军运动正在进行中。——译注}。那就试试看吧,先生们!我们把话说在前面:你们办不到。正是因为短暂地向资本主义学习,中国和印度都在朝着各自的十月革命前进。这就是世界发展的辩证法,不可能从里面跳出来。

孟什维主义希望迅速“解决这个双重任务,让国家产业适应国家的实际经济发展水平,同时为这种适应创造政治和法律前提”。这种狡猾的说法是为了恢复资产阶级政权。“政治和法律”前提肯定指的是资产阶级民主制。资产阶级的孟什维主义说:“工厂归你们,而我们要像在德国和英国那样,有机会担任议员、市长、部长和策尔基贝尔\footnote{卡尔·弗里德里希·策尔基贝尔(1878—1961)是德国社会民主党政治家。在魏玛共和国时期,他先是担任科隆的警察局长(1922—1926),然后调往柏林(1926—1929),最后在多特蒙德(1930—1933)。策尔基贝尔对1929年暴力镇压柏林的五一示威活动负有政治责任,30余名示威者和无辜群众在警察行动中被杀,这一事件在历史上被称为“血腥五月”(Blutmai)。——译注}。”这就是“双重任务”。1917年,掌权的孟什维主义试图保护资产阶级、抵抗十月革命,资产阶级却不信任它的保护,去找科尔尼洛夫帮忙。现在,孟什维主义又提议用“民主的”方式清算十月革命,为资产阶级扫清道路。但是试图复辟资本主义的人明白,以“渐进”的方式回归资本主义是不切实际的。只有通过多年的内战、对这个由苏维埃政权从废墟里振兴起来的国家造成新的破坏,资产阶级反革命才能(如果真的能的话)实现它的目标。

第二版的俄罗斯资本主义绝不会简单地延续和发展革命前,或者更确切地说,战前的资本主义:这不仅是因为它们中间有漫长的间隔,里面充满了战争和革命,还因为世界资本主义,也就是俄罗斯资本主义的主人,在这个时期经历了最为深刻的崩溃和变革。金融资本变强了很多,而世界则狭窄了很多。现在俄罗斯的资本主义只能是亚洲式的、卖身为奴的殖民地资本主义。俄国的商业、工业和银行业资产阶级当时救出了自己的动产资本,所以在这些年里完全融入了外国的资本主义体系。对于“真正”“严肃”的复辟者来说,俄国资产阶级的复辟正是从外部对俄国进行殖民剥削的机会。在中国,外国资本通过买办,也就是中国的代理人来进行支配,买办们则倚靠世界帝国主义掠夺本国人民,从而大发横财。在俄国复辟资本主义会创造出一种化学意义上纯净的俄罗斯买办文化,它的“政治和法律”前提则会以邓尼金和蒋介石的模式为样板。当然,这一切会和上帝以及斯拉夫花体字\footnote{斯拉夫花体字(вязь)是一种古代的装饰性西里尔字体,字母之间相互连接,形成连续的装饰。在13世纪的南斯拉夫纪念碑上就已经出现了这种字体,14世纪末到15世纪初在东斯拉夫和瓦拉几亚地区也出现了这种字体,在16世纪的俄罗斯,也就是伊凡四世统治时期,花体字得到了重大发展,但之后逐渐废弃。典型的使用场合是中世纪的圣像画题词和书面文献写作,也作为图书的装饰手段。——译注}(也就是杀人犯们的“灵魂”所需要的各种东西)一起出现。

这种富丽堂皇能持续多久?将要摆在复辟面前的不仅有工人的问题,还有农民的问题。在斯托雷平的领导下,农场主阶层相当成功地分离了出来,与它相联系的则是极其严重的无产化、赤贫化,以及乡村各种社会症结的高度激化,从而让1917年农民战争的爆发规模不可阻挡。除了斯托雷平的办法以外,资产阶级和社会民主党找不出别的路,而且在资本主义的基础上也不可能有其他道路。只不过,他们现在面对的不是1200万到1500万户农民,而是2500万户。把资本主义阶层从他们中间分离出来就会引起无产化和赤贫化,规模之大足以让1917年之前的那些过程都黯然失色。即使反革命拒绝复辟地主阶级——它能拒绝吗?——土地问题这个死过两次的幽灵也会立刻出现在它面前。毕竟,就算是在几乎没有地主阶层的中国,土地问题所蕴含的爆炸性力量也不亚于印度的水平。再说一次:哪怕俄罗斯的资本主义处于一个稍高一些的阶段,它的发展也会是中国式的发展。要解决孟什维主义的“双重任务”,这就是唯一可能的办法。

结论很明确:除了它所开辟的社会主义前途以外,苏维埃制度还是唯一能在目前的世界条件下维持俄罗斯民族独立的制度,虽然它既没有萨罗夫的塞拉芬,也没有ѣ(ять)这个字母\footnote{萨罗夫的塞拉芬(1794—1833),本名普罗霍尔·伊西多维奇·莫什宁,是俄罗斯正教会最富盛名的圣人之一,在正教会与公教会中都受敬拜。信徒普遍将他看做19世纪诸长老中最伟大者;ѣ(ять)是一个早期西里尔字母,在1917年的俄文正字法改革中被废除。——译注}。

\section{新环境下的旧矛盾}

要理解苏联所经历的主要困难有什么意义,就必须牢记:虽然十月革命带来的断裂造成了灾难性的深远影响,但目前的经济发展——哪怕它经历了深刻的改变——仍然是革命前和战前基本进程的延续。尽管自由主义和社会民主主义把希望完全寄托在过去上(资本主义、二月革命、民主制度),但他们对当前经济制度的批判却完全以“忽略现在和过去的继承关系”为基础。他们描绘的情况让人感觉城市和乡村之间的矛盾好像是十月革命造成的,而实际上正是这种矛盾把无产阶级起义和土地变革结合起来,为革命的胜利创造了机会。

苏联的农村危机是落后小农经济的危机。为了让重要的农业繁荣、稳固、得到拯救,统治阶级使尽了浑身解数:1861年所谓的“解放”\footnote{指沙皇亚历山大二世推行的废除农奴制改革。——译注},1905年革命中斯托雷平的反革命立法,最后还有1917年双重政权时期的政策。但这一切都无济于事。

在全球金融资本的压力下,俄罗斯资本主义得以加速发展,而对于被带入市场环境的落后俄罗斯农民来说,这种发展极大地激化了他们扩大土地面积的渴望。正是资本主义自己让前资本主义农民“平分土地”的“梦想”壮大到不可遏止的地步。用资本主义农场主经济的路线来对抗农民的渴望,这类尝试从构思上来说很现实,但“仅仅”因为资本主义的总体发展速度和农民阶级向农场主演化的速度不协调,它便惨遭失败了。让沙皇俄国服从于世界市场和金融资本,以及随之而来的,把市场、财政和军事负担压在农民身上,这段路转瞬之间就走完了;而把可靠的户主分离出来、组成一个农场主阶层,这件事却是以“龟速”完成的。为了解决这种不协调,1907年到1917年间的资产阶级和地主反革命可谓是绞尽了脑汁。

总之,在过往的全部历史当中,各种盘根错节的混乱都落在了土地之上,要想把土地关系里的这些东西肃清干净,只有革命的土地国有化才是唯一可能的措施。国有化就是要把所有,或者说几乎所有的土地转给农民。但是,凭着从过去继承来的农业工具和方法,把土地转让给农民就意味着土地经济的进一步分化,因此也就是在为新的农业危机做准备。

过去遗留下来的城乡矛盾不可能在十几年以内消除。相反,工人国家击退了敌人,开始认真发展工业,此时这种矛盾还会变得更加激烈。由于总人口增长,以及年轻一代农民渴望独立自主,经济分化同时也以前所未有的速度发展。工业和文化的发展要不可避免地牺牲农村,这种发展很快,快到激发了农民的新利益和新需求,但它又很慢,不足以满足全体农民的这些利益与需求。因此,城市和乡村之间的矛盾已经到了极其尖锐的新阶段。孤立而落后的小农经济看不到希望,这一点仍然是矛盾的基础。

那么,和革命前的状况相比,它有什么不同之处呢?可以说是天差地别。

首先,没有大土地私有制了,农民阶级不能通过吞并统治阶级的土地来扩大土地面积,从而在经济困境(或者更准确地说,2500万个经济困境)中找到出路。这条路线对国家未来的前途是最有好处的,但它已经到头了。因此,农民阶级被迫去找寻其它的道路。

其次——这个区别同样重要——领导国家的这个政府无论犯下过怎样的错误,都会尽其所能地提高农民的物质和精神水平。工人阶级的利益也是往这个方向走的,尽管革命社会的结构发生了种种变化,但它仍然是国家的统治阶级。

从这个广泛的、历史的,总而言之也是唯一正确的观点来看,自由主义者所声称的“整个集体化就是赤裸裸的暴力产物”是再纯粹不过的谬论。农民靠旧办法来利用革命的土地资源,这就导致了土地的高度分割,所以,对小块土地进行整合,也就是说把它们合并成更大的经济区划,成了关乎农民阶级生死存亡的问题。

在从前的历史时代里,农民反对土地压迫的斗争有几种方式:要么起义;要么涌向那些未开垦的土地,形成一股强大的殖民潮流;要么投向形形色色的教派,教派则用天堂的空旷奖赏农民,让他安于地上的逼仄。

马克思曾经说过,农民不只有偏见,也有理性。在他的整个历史当中,这两种品质以不同的组合方式出现。而一旦越过某条界限,农民那源于生活的现实主义就会被一些荒谬绝伦的迷信给困住。在农民经济走投无路的现实面前,“理性”显得越无能为力,“偏见”就越是蓬勃发展。

在更高的历史阶段上,农民阶级的理性和偏见也以一种崭新的形式和不同的比例在全盘集体化当中得到了体现。12年的革命当中,新经济政策取代战时共产主义,以及新经济政策的个别阶段相互交替,这就促使农民产生了这样的想法:要摆脱贫困和落后,就得走某些新的路;只不过这些路还没有经过检验,它们的好处还没有得到证明。1923-28年间,政府的政策让农村的上层专注于扩大和改进个体经济,下层则因此迷失了方向。这一次,城乡矛盾的性质是面包罢工\footnote{1927年,农村生产者因为担心爆发战争而限制了粮食供应,导致1928年粮食和饲料价格多次上涨、农产品价格失衡、国营商业和集市贸易中的零售价格上涨;1929年2月起,城市的面包供应过渡到票证制度。面包罢工是苏联大规模集体化的动因之一。——译注}。政府突然改变方向,闩上了市场的大门,又把集体化的大门敞开了。农民一下子就冲了进去,他们怀揣新的希望,理性和偏见结合在一起;少数人怀有觉悟,但与此同时大多数人是盲从的,他们随大流加入运动。结果政府被打得措手不及,而且——唉!——就其本身来说,它在这次事件中引起的偏见要远远多过理性。“全联盟”层面显现出了可怕的冒进现象,而事后聪明的领导层则试图把这个“全联盟”的大冒进兑成地方的一连串小冒进,以此为自己开脱。为此,中央委员会的秘书处保存了一大批提前录好的唱片:从州级到区级单位的录音都有。

\section{冒进的本质是什么?}

斯大林写了一篇《答集体农庄庄员同志们》\footnote{参见诸夏怀斯社版《斯大林全集》第12卷第147页。——译注},凭良心讲,这篇文章相当冗长而且无知得吓人。他在文中谈到,“有些人”错误地对待中农,“还有些人”没领会集体农庄的规章(顺便一说,这规章是在各种冒进现象之后颁布的)——然后又说明智的领导层因此感到多么痛心。他写的这些东西都很有意思,有些地方甚至显得令人感动。尽管如此,斯大林根本没有提到,那40\%的农民(绝不“退却”!——结果三月的时候,斯大林就把60\%集体化率的目标往下调了二十个百分点)要怎么管理这巨大的农产业——他不谈工具和器材,而只有这些东西才能让产业的大规模发挥应有的作用,对于它们的社会形式,他更是只字不提。

无论农民的“单干精神”多么伟大,他也会在无可争辩的经济事实面前退缩。哪怕是资本主义国家里的农民合作社发展历史也能证明这一点。正是因为生产过程具有分散性,所以需要把贸易和信贷功能社会化。1905年革命之后,合作社吸引了沙皇俄国的几百万农户。但这种合作社的范围限于采购、销售、信贷和储蓄,而绝不包含生产职能。生产仍然保持分散的原因不在于农民的心理属性,而在于他的生产工具和生产方法:它们正是单干作风的基础。

分散的农业经济毫无出路,所以筹备了集体化;但在官僚的三板斧催促下,集体化推进的速度快得出乎预料,暴露出了生产资料和集体化规模之间的巨大矛盾。此时他们创立了一种挽回局面的理论,说这些以原始农具为基础的大型集体农庄应该被看成是社会主义性质的手工工场。这听起来很深奥,但哪怕是死啃书本的呆子也知道,给一个东西重新取名字并不会改变它的本质。

要为农业“手工工场”辩护,集体农庄的“集体化”形式并不是理由,而只能说用工场手工业的办法耕作土地在生产方面还有利可图。我们仍然要问,那为什么过去的发展没发现这种好处?

当然,用抽象的统计计算不难证明:农具就算再简陋,集体化也能带来好处。然而,虽然现在这种想法千篇一律地在演讲、文章和通告中重复出现,它却仔细做好防备,避免和活生生的经验当面对质。在各类集体当中,农民的大家庭能最“自然”地匹配落后庄稼汉的农具,但恰恰是这种家庭在十月革命后经历了最为剧烈的瓦解。现在又认真地考虑在这同一种生产力基础上让一群彼此陌生的家庭组成一个稳固的集体,这怎么可能呢?

建立在简陋农具之上的大规模生产合作社已经经过了历史的检验:它们以劳役制\footnote{劳役制(отработка)指1861年改革后的一种农业制度,农民为了偿付地租、债款或者欠粮,自带农具和牲畜为地主劳作。——译注}为基础,在地主的领地里开展。怎么样呢?总的来说,这些产业的水平甚至比个体农户还要低。1905年革命之后,这些“劳役制”的产业大量停办了,农民银行则把土地分成单独的小块并出售给农民。事实证明,地主土地和落后农具组合在一起,这种条件下的生产“合作社”在经济层面完全无法持续。相反,一些大产业建立在机械化和切合实际的轮作等基础之上,它们则撑过了1905年和接下来几年的动荡,直到十月革命它们才被收归国有。当然,那时的耕作是发生在地主的土地上,但是危险之处在于,现在是人为地,也就是过早地组建了大型集体农庄,几十、几百个农民用着同样的工具,单个农民的劳动淹没在其他农民的劳动当中;由于个体积极性的丧失,土地的耕作水平甚至可能会比个体农户还要低。

建立在简单农具组合之上的集体农庄也是一种社会主义形式的农产业,这就像是在说建立在劳役制之上的地主经济是一种资本主义形式的大产业一样。这就对“社会主义手工工场”的想法做出了无情的裁决。

布哈林在“理论”方面肆意添枝加叶,以此来代替集体农庄的物质基础,他解释说,由于农业发展速度大大落后于工业,“唯一可行的出路就是对农业进行社会主义改造”。由此可见,搞全盘集体化不是把它当成农业生产关系发展的物质准备阶段,而是把它看作摆脱困境的唯一“出路”。可以说,驱使他提出这个问题的是纯粹的行政目的论。

当然,布哈林说国内正在发生的进程不是简单地折返回“战时共产主义”形式,这一点是对的。它完全不是回到过去。毫无疑问,目前的转折中有着非常伟大的因素,能够为未来打下基础。但是一切的问题都在于比例,在于合乎实际的相互关系。除了为社会主义未来打下基础之外,转折中还包含着最为直接,而且是最为致命的危险。布哈林只不过顺便地谈了一下这些危险:“由于集体农庄和国营农场的发展,对复杂机械、拖拉机和联合收割机以及人工化肥等物品的巨大需求正在超越供给水平,这里的‘剪刀差’还在增长,而且速度相当快。”这段惊人的字句仅仅是插在一篇热情洋溢的文章里,没得出任何进一步的结论。而与此同时,地基和屋顶之间增长的“剪刀差”只能预示一件事情,那就是倾覆。

布哈林提出,为集体化农业准备一个有计划的开端非常重要,各区集体农庄同工业和地方苏维埃机构也有紧密的联系,此时他说道:“接下来克服官僚主义的过程中,我们要把它消灭在胚胎阶段。”是的,胚胎阶段。但要是把胚胎当成婴儿,或者把婴儿当成少年,那就麻烦了。由于其技术基础不合理,集体农庄正在不可避免地创造一群最为糟糕的、寄生虫式的经济官僚。尽管在国家管理方面,历史上的农民不止一次地充当过各种官僚主义的消极支持者,但他们完全无法忍受直接经济领域的官僚主义。不要忘记这一点。

布哈林重复说,集体化必然会改造农民的天性。这毫无疑问。但要做到这一点,需要的不是他们的“想法”,而是拖拉机、圆盘犁和联合收割机。在生产过程当中空谈理想和精神从来都没有成功过。当然,虽然现在拖拉机的数量完全可以忽略不计,按照计划,这个数量一定会增长得越来越快;但是,不可能在未来的拖拉机上建起今天的集体农庄。何况拖拉机还需要燃料。要在大片幅员辽阔的地带合理地保障燃料供应,这对生产、组织和运输都是巨大的挑战。但是就算有了拖拉机和燃料,它们本身也毫无价值;最为重要的技术成果和总体素养水平环环相扣,组成一根完整的链条,而它们只有成为这链条的一部分,才能发挥力量。这一切可以实现,也会得到实现。但必须要正确地“计算时机”——如果做不到这一点,军事行动就会失败,经济行动也是如此。如果国际国内环境有利,农业的物质技术条件就可以在大约10-15年内得到根本性的变革,并且为集体化提供稳固的生产基础。然而,我们和这种情况之间相差的这些年份或许都足够敌人推翻苏维埃政权好几次了……

但是——唉!——从布哈林那儿什么也得不到:他这次用左脚蹬开现实,一路狂奔,在形而上学和投机倒把的高空驰骋。我们现在已经开始担心,恐怕布哈林又要为被斯大林砸烂的碗碟负责。不过,我们感兴趣的并不是布哈林。

世界资产阶级的报刊,至少是其中最有远见、能够深谋远虑地进行一些挑拨的报刊,它们在全盘集体化进行到最紧张的时候不停地重复说,这一次没有退路:要么把试验进行到底,要么苏维埃独裁政权就会覆灭,并且在它们看来,“把试验进行到底”就完全等同于覆灭。而另一边,苏联的官方报刊打从战役一开始就鼓吹说攻势连续不断,而不曾回头去检验一下。斯大林则直接号召贫农起来“无情地粉碎”富农……把它当成一个阶级那样。只有左翼反对派带来了不和谐的声音,它从去年秋天开始就警告说,速度不协调引起的风潮当中隐藏着危险的因素,会在不久的将来导致不可避免的危机。一系列事件很快就表明,只有大资本的报刊和共产主义左派报刊这对站在两极的对头知道自己想要什么。农村战线上的攻势很快就暴露出了自身的矛盾,又一下子把这些矛盾激化到极其尖锐的地步。于是开始做这样的事:揭露冒进现象、放松退出集体农庄的限制、事实上中断了去富农化运动,等等。与此同时,又严厉禁止把这一切退却叫做退却。没人知道明天会发生什么。

不过这个时候,做事还是必须要有始有终。如果执政的党不这么做,发展的自发过程就会踩着专政制度的背脊去完成它。对“计划”进行的修改来得越早、越广泛、越大胆——或者更确切地说,越早提出集体思考得出的计划,并且用它来应对危险的“成功”引起的混乱——纠正已经犯下的错误时痛苦就越少,缓解城乡发展中极其尖锐的失衡现象时成效就越可靠,同时也更能赢得新的期限,让它们接近逐渐成熟的欧洲革命“时刻”。

最糟糕的是,目前的退却杂乱无章,而且被官僚的胡扯和俏皮话掩盖了起来。党陷入了惊慌,但又保持着沉默。主要的危险就在于此。

\section{只有党才能找到出路}

从前的资产阶级取得政权后,在各种党派和潮流的不断斗争中掌管社会的前途,而这种斗争经常采取的形式则是内战。尽管无产阶级比起资产阶级要一致得多,但这种一致还远远算不上绝对。工人官僚不仅会成为无产阶级对其他阶级施加影响的工具,同时也会成为其他阶级对无产阶级施加影响的工具。错综复杂的世界关系在此处发挥着作用,而归根到底它才是决定性的因素。总的来说这就足以解释,为什么在无产阶级革命的基础上,执政党内可能出现而且确实出现了深刻的分歧,并且以派别的面貌将分歧呈现出来。单纯地禁止不能将其消除。

斗争不可避免,而因为它不仅要以专政制度为基础来进行,也是为了维护专政制度的利益,所以它的形式必须要能让培养出正确政治路线的成本降到最低。斯大林主义的官僚曾试图完全摆脱党的存在所带来的政治成本,然而事实证明,官僚自己造成的政治曲折才带来了最大的成本。这种曲折同机关部门的制度密不可分,它不受党的监督,而且每次自己犯下错误都一下子跳到一边,不去处理后果。如果以为无产阶级专政就有权无数次地走上弯路,那就会非常致命。不,这种历史威望是有限的。

党的代表大会已经两年半没有召开了,在此期间政策却急剧转变了好几次——而且是在最根本的问题上。如今,召开代表大会成了违背上层意愿的事情,领导机关觉得它不是解决内部困境的办法,而是个烦人的障碍和明显的危险。在内战年代,代表大会每年都会召开,有时甚至一年召开两次,而现在,在和平时期,在社会主义工业取得毋庸置疑的成就之后,在领导层保证说“农民一定会转向社会主义一边”之后,党内生活的紧张气氛竟然到了如此难以忍受的地步,连代表大会都成了累赘、疑难和危险,这种情况要怎么解释呢?

当然可以反驳说,现在主要的敌人不是国内的资产阶级,而是在战后得到巩固的世界资产阶级。这说得没错。但是,国内的社会主义基础实际上也加强了,所以来自外部的危险完全不能解释制度的官僚化。有最广泛、最充分、最自由的民主制度做基础,社会主义社会能够同外敌作战。国内制度的不断恶化必然有国内的原因,外部压力只能与国内的阶级关系联系起来。

一些人用“有必要同内部的敌人作斗争”这样的理由来为党体制的恶化辩解开脱,而这样一来他们也就默认了“近年来力量对比的变化对无产阶级及其政党不利”这种假设。但难不成现在的富农比内战期间包括富农的整个资产阶级还要危险吗?当时的旧统治阶级可还没有失去自信,它指望布尔什维主义会迅速垮台,而且还有自己的军队。这种假设好像和显而易见的事实相矛盾。无论如何,整个官方理论都完全不能容纳它,因为官方理论在四周看见的情况只有一个,那就是社会主义部门在不断加强并且取代资本主义部门。

人们越来越难解释,为什么现在只要与领导层,也就是与军事化的斯大林派别产生任何分歧,只要有任何进行批评的尝试、提出任何上层没预见到的建议,都会立刻导致有组织的大迫害,而且过程像哑剧一样静默;为什么之后还有“理论”层面的清算,搞得就像是红色教授学院来的懒惰执事和诵经士在办安魂弥撒。

如果认定,党目前的体制是唯一一个可能实现的体制,认定它的演变合情合理、无法抵抗,这就是认定党,还有革命,会不可避免地走向死亡。假如他们要完全把党的代表大会扔到一边去,并且宣布,比如说,大会将“在需要的时候”召开,这要费很大功夫吗?这会给现在的体制带来什么新麻烦吗?几乎完全不会。但这个机关被迫要寻求别人对自己的赞同,所以不能不摆起另一副面孔。官僚需要一个超级仲裁人,并且它推举的这个角色最符合它自我保护的本能。斯大林主义,这种在政党层面为波拿巴主义做准备的派别,其本质就在于此。

党内的两种极端倾向反映或者说表达了小资产阶级和无产阶级的两种路线,如果说官僚主义中派的生涯就是在它们之间随机应变,那么波拿巴主义的国家机器就是要公开地切断一切传统的联系,其中甚至包括党的联系,并且作为专权的“调停人”,“自由”地在阶级之间周旋。斯大林主义是在无意识地、但也更加危险地为波拿巴主义做准备。必须要理解这一点。也是时候理解这一点了。

是什么因素在经济取得成功的情况下依然让政治局势恶化,并且使得专政制度陷入过度紧张呢?这种因素有两类:其一根植在群众中,其二则根植在专政的机关中。

市侩的庸人经常说,十月革命是群众“幻想”的产物。封建主义和资本主义都没能让群众养成用唯物主义思想理解历史的精神,如果从这种意义上来说,上面的说法是对的。但是幻想和幻想是不同的。如果没有爱国主义的幻想、没有社会民主党在维持这种幻想时所起的主要作用,掠夺人类、让人类流干鲜血的帝国主义战争就不可能发生。群众对十月革命的幻想是一种夸大的希望,想要迅速地改变自己的命运。但是,难道迄今为止的历史上有哪次伟大事件里没有这种创造性的幻想么?

然而毫无疑问的是,革命的实际进程磨灭了群众的幻想,从而降低了群众在1917年给予领导的党的额外信用。当然,作为回报,对历史进程真正力量的经验和理解也增进了。但不可忽视的是,幻想的丧失速度比理论认识的积累速度要快得多。从革命阶级本身发生的心理变化中寻找原因的话,这就是过去的反革命能够成功的主要原因之一。

另一个危险因素在于专政机器的蜕化。官僚已经养成了许多统治阶级的特征,而且这也是很大一部分劳动群众对他们的看法。官僚为了自保而进行的斗争压制群众的思想生活,故意把毫不革命的新幻想塞给他们,并且,对于正在发生的事情,迟迟不用切合实际的理解来取代已经丧失的幻想。从马克思主义的观点来看,苏联的官僚机构很明显不可能成为新的统治阶级。它的特殊化和它社会作用的上升(其形式是任意发号施令)会不可避免地导致专政制度的危机,而危机的结果要么是让革命在更高的基础上复兴,要么是让资产阶级社会得以复辟。正是因为人人都能感觉到(尽管只有少数人清楚地理解)这二选一的抉择正在迫近,现在的制度才陷入了极度的紧张。

毫无疑问的是,“在单独一国建设社会主义”的全部矛盾也会在官僚主义的发展中体现出来。换句话说,即便有正确的领导,官僚主义迟早也会造成威胁。然而,一切问题都在于期限和时间。允许世界资本主义,尤其是欧洲资本主义继续存在哪怕是几十年时间,就意味着接受了“苏维埃政权必然会垮台”这件事,而且,国家机器的前波拿巴主义蜕化将为公然推动反革命和发起政变做好准备,不管这政变是热月式的还是马上就是波拿巴式的。我们眼前必须始终保有这种预期,这样才能在正在发生的事件中正确地确定自己的方位。我重复一遍,全部的问题都在于期限,但是这个期限并不能预先知晓,因为它是由有生力量的斗争决定的。要是德国革命和中国革命没有遭到可耻的灾难性失败,今天整个世界的情况就会是另一番面貌。可见,我们从客观条件出发,每次都会重新回到领导层的问题上。关键不在于一个人或者一群人(尽管他们的问题并非不重要),而在于领导层和党的相互关系,在于党和阶级的相互关系。

正是从这个视角出发,才有了联共和共产国际的制度问题。我们得知,反对派中的一些摇摆成员(奥库德扎瓦\footnote{可能指米哈伊尔·斯捷潘诺维奇·奥库德扎瓦(1883—1937)或尼古拉·斯捷潘诺维奇·奥库德扎瓦(1891—1937),格鲁吉亚族革命者。兄弟二人都长期参与反对派活动,并在大清洗中被枪决;他们的弟弟,沙夫拉·斯捷潘诺维奇·奥库德扎瓦也因此受牵连,于同年被枪决。——译注}等人)提出了一套新的理论:更健康的制度一定会自行从斯大林现在的“左翼”政策当中“长出来”。这种乐观主义的宿命论是对马克思主义最恶劣的讽刺。现在的领导层不是白纸一张,它有它自己的历史,这历史又同它的“总路线”紧密相连、不可分割。斯大林主义领导层的历史就是犯下空前错误、并且因此造成国际无产阶级溃败的历史。目前领导层的“左”转完全是由它昨天右倾路线带来的后果造成的。转向越是猛烈,官僚的钳制就越是凶狠,这样一来才不会让党看清昨天和今天之间的矛盾。

党务机器毁灭性的僵化不只是客观矛盾的产物,还是这个领导层的具体历史的结果,而且上述的矛盾也是通过领导层折射出来的。通过对上下层人员的人为淘汰选择,这个领导层把过去所有的错误都定型下来,也为未来所有的错误打好了基础。而最重要的是,这个领导层已经为它自己向波拿巴主义的进一步蜕化埋下了伏笔。这条道路上隐藏着主要的、最直接的、最强烈的危险,威胁到十月革命的存亡。

折向左边绝对不代表中派领导层能够通过官僚的内部努力把自己转变成马克思主义的领导层。折向左边的含义完全不是这样,而是:不管是在客观条件上,还是在工人阶级不敢明说的情绪当中,都暗含着对热月路线的深刻抵制——如果不公开掀起反革命动乱,过渡到这条路线就仍然是件不可想象的事情。虽然扼住了党的咽喉,但领导层还做不到不回头去看它,因为,尽管寂静无声、模糊不清,阶级力量的预警和提醒还是在党内流动着。问题讨论、思想斗争、会议和代表大会被党内的密探、电话监听和书信检查取代了,但是阶级压力也影响着这些秘密的“阴险”办法。这就是说,向左转的根源,以及转向力度如此强烈的原因都在领导层以外,后者不过是在让这转向显得考虑不周、保守落后和无法依靠。

领导层不承认、不了解自己犯下的错误和罪行,在事件的压力下绕着自己的轴心来回旋转,又在新的道路上堆砌新的错误,但仅仅因此就对它忍气吞声——那这就是连官僚的程度都达不到的庸人,而绝不是什么革命者。但也许真的像拉狄克、季诺维也夫、加米涅夫、斯米尔加和其他退休的思想家们、这些懦弱的山羊们诉苦时说的那样,真的“没有别的办法”了?他们的诉苦只有一个意思:革命——反正已经死了,那还不如和“人民”一起:众人在一起,死也不可怕(на миру и смерть красна)。而我们同这些腐朽的情绪没有半点关系。

现在这个党虽然并不是作为一个党而存在的,但还是能默默地让领导层扭转180度。没有任何地方说过,也没有任何人证明过这样一个观点:这样一个党,哪怕它具备必不可少的主动性,也不能对过去走过的路进行集体检验,由此进行深度的力量重组,进而让自己得到复兴。在历史上,比共产党笨拙得多、僵化得多的机关也能一次次地发现,深刻的内部危机让它们有能力得以恢复和更新。这一点,也只有这一点,才是我们在国内和国际层面面临的问题。

反对派的观点完全不同于奥库德扎瓦同志等人自负的形而上学,因为反对派的观点要以鲜活的方向斗争为前提,因而也就要求左翼反对派拿出最高的积极性。只有在政治上破产的人才会在关键时刻把责任推卸给事物的客观进程,并且在安慰的预言当中找寻出路。盲从心态和尾巴主义,要描绘堕落和蜕化时期的特点,可找不到比这两个词更好的形容了。布尔什维主义是从反对它们的斗争中发展起来的,左翼反对派则是在延续布尔什维主义的历史路线。它的责任不是溶入中派,而是要在整个体系中更加积极主动。

\newpage
\chapter{毛泽东的新民主主义论再评价 胡绳 1998}

出处:中共党史研究 出版日期:1999第3期

\begin{quotation}
    \footnotesize {
    民粹主义思想是可能从小农经济中自发产生的,在这种思想影响下,革命就会发生破坏工商业和城市的举动。试看太平天国的历史,那时的人们当然不知道什么民粹派,但太平天国明确地提出了,实际上也执行了一套自己的政策。他们在打到南京后,用现在的话说,一进城就“共了产”,把所有商铺的财产都分门别类地收归军队所有,实际上就是消灭社会上存在的工商业。太平天国也主张实行一种农业上的绝对平均主义,按照人口来计算,平均给每一个人分多少土地,规定每一家人能有多少财产,能养几只鸡等,多余的财产都要交公。虽然太平天国的这套主张没有也不可能完全实行,但可以看到,这就是在中国农民中自发产生的民粹主义思想。所以,在中国革命完全胜利、全国解放的前夕,毛主席特别提出绝对平均主义的问题,提醒全党注意,而且加上一个“农业社会主义”的称呼。

    梁启超等人的错误,不在于说中国现在还不能实行社会主义,而在于认为既然不能马上实行社会主义,就不需要社会主义者,不需要社会主义思想,不需要成立共产党,大家都应当一心一意奔资本主义。为反驳这种观点,显然要有较多的马克思主义才行。否则,只是斥责资本主义的罪恶,声讨资本主义的落后性,藉以论证中国应该立刻实行社会主义,那就实际上染上了民粹主义的色彩。

    李立三、王明认为,革命一发展到大城市,资产阶级民主革命就要立刻直接转变为无产阶级社会主义革命。李立三在那时说过的这句话可谓典型:“革命胜利的开始,革命政权建立的开始就是革命转变的开始,中间决不会有丝毫的间隔。”30年代前期这种“左”倾机会主义并不表示对小农经济的崇拜,这的确是和民粹主义不同的,但可以说,它的基本性质是类似于民粹主义的。因为它以为可以在经济很落后的情况下,即普遍存在着小农经济的情况下一下子将民主革命转到社会主义革命去。毛主席在党的七大联系着批评民粹主义时说,我们的同志对消灭资本主义急得很,在这方面是过急了。他所批评的这种急性病,是从30年代遗留下来的。

    1944年毛主席给博古的那封有名的信。在这封信里他说:“简单言之,新民主主义社会的基础是机器,不是手工。我们现在还没有获得机器,所以我们还没有胜利……现在的农村是暂时的根据地,不是也不能是整个中国民主社会的主要基础。由农业基础到工业基础,正是我们革命的任务。”
    }
\end{quotation}

\newpage

今天我来参加“毛泽东、邓小平与马克思主义中国化”理论研讨会,虽然做了些准备,但没有成文的稿子。题目倒是有一个,叫作《毛泽东的新民主主义论再评价》。所谓再评价,当然不是说,要推翻过去的评价。新民主主义理论是完全符合国情的指导中国民主主义革命阶段的理论,已在实践中得到完全证实,是马克思主义普遍原理和中国革命的具体实际相结合的完满典型。但我觉得,学术界过去对毛泽东新民主主义论的评价还有些不够,它有一个方面的内容还未得到充分重视,因此有再评价的必要。

1

为讨论现在要谈的问题,我想从毛泽东是不是带有民粹主义的思想说起。这个问题,恐怕是由研究毛泽东的美国人首先提出来的。他们认为,毛泽东的民粹主义思想很严重。对他们的这种看法,国内的学者有人同意,有人反对,有人说要分析研究。我认为,至少在毛泽东一生最辉煌的时期之一,即大体在民主革命时期的1939年到1949年,毛泽东不但没有丝毫染上民粹主义的思想,而且是坚决地反对民粹主义的。他不仅仅在口头上反对,而且从理论上和实践上鲜明地、坚定地反对民粹主义。甚至可以说,虽然过去我们党内有些同志表示反对民粹主义,但从理论和实践两方面坚定地、透彻地反对民粹主义,毛泽东是我们党内的第一人。

首先,我想引用毛主席讲过的几段话,这些话可以表明,毛主席为什么要反对民粹主义,为什么他认为在中国社会,在中国党内反对民粹主义有很重要的意义。

1945年,毛泽东在党的七大发表《论联合政府》的重要报告,他在此前的六届七中全会上对《论联合政府》作的说明中指出:“报告中……着重说明民主革命,指出只有经过民主主义,才能到达社会主义,这是马克思主义的天经地义。这就将我们同民粹主义区别开来,民粹主义在中国与我们党内的影响是很广大的。”\footnote{《毛泽东文集》第3卷,人民出版社1996年版,第275页。}

七大开会时,因为已经把《论联合政府》印成书面报告发给大家,所以毛主席没有再照本子念,而是作了一个口头报告,来解释书面报告的主要内容。其中提到“民粹派的思想”。他说:“这种思想,在农民出身的党员占多数的党内是会长期存在的。所谓民粹主义,就是要直接由封建经济发展到社会主义经济,中间不经过发展资本主义的阶段。俄国的民粹派就是这样。”\footnote{《毛泽东在七大的报告和讲话集》,中央文献出版社1995年版,第126页。}接着,毛主席指出,民粹派最后变成了反革命的社会革命党。他说:“他们‘左’的要命,要更快地搞社会主义,不发展资本主义。结果呢,他们变成了反革命。布尔什维克就不是这样……俄国在十月革命胜利以后,还有一个时期让资本主义作为部分经济而存在,而且还是很大的一部分……一直到第二个五年计划时,才把城市的中小资本家与乡村的富农消灭。我们的同志对消灭资本主义急得很……我们的同志在这方面是太急了。”\footnote{《毛泽东文集》第3卷,第323页。}

毛主席这段话牵涉到俄国革命时的一些问题。社会革命党在俄国革命史中的地位与作用,学术界有不同看法,这里不作讨论。列宁的新经济政策实行不久,就被斯大林中止了,现在学术界对这个问题有很多看法,有人提出,新经济政策是不是停止得过早?对这问题,也不作讨论。我只想指出,毛主席讲这些话的意思是说,我们不要像俄国的民粹派,对消灭资本主义不能急,太急了不行。

民粹主义思想的内容,在毛主席这段话里也交待清楚了。大家知道,民粹主义是19世纪末年在俄国出现的一种思潮,列宁、普列汉诺夫花大力批评它。它的基本特征就是毛主席说的,主张不经过资本主义,直接从封建经济,也就是从小农经济发展到社会主义。表面上看起来,民粹主义者非常反对资本主义,热心于社会主义,但实际上他们的这种主张是行不通的、错误的。毛主席的以上两段话都指出,民粹主义在中国有很大的影响。当然,并不是说俄国民粹派的思想直接影响到中国共产党。那时也许我们党内很多同志并不知道俄国民粹派。说民粹派思想在中国有很大影响,主要是由于中国革命在广大农村内进行,大多数党员是农民出身,在这种特定的环境下,党内容易产生实质上类似于俄国民粹主义的倾向。

我介绍的第三段话,是1948年4月1日毛主席在晋绥干部会议的讲话中说:“现在农村中流行的一种破坏工商业、在分配土地问题上主张绝对平均主义的思想(是一种农业社会主义的思想),它的性质是反动的、落后的、倒退的。我们必须批判这种思想。”\footnote{《毛泽东文集》第4卷,人民出版社1991年版,第1314页。引文中用括号包起来的短语“是一种农业社会主义的思想”,在1960年《毛泽东选集》第4卷初版中已删去,但原来是有的。}那时,革命快要取得全国的胜利,我们开始进入中等的和大的城市。新华社在同年7月发表了一个关于“农业社会主义”的问答。在引用了毛主席的上述这段话(其中有“是一种农业社会主义的思想”这个表述)以后,新华社解释说:“毛主席在这里所说的农业社会主义思想,是指在小农经济基础上产生出来的一种平均主义思想。抱有这种思想的人们……以为把整个社会经济都改造为划一的'平均的'小农经济,就是实行社会主义,而可以避免资本主义的发展。”新华社的回答还引用了历史上的例子,指出帝俄时代的民粹派和中国的太平天国的人们,大都是抱有这一类的思想的。新华社还解释说:“在新民主主义的国家内,土地改革后农民中一定程度的阶级分化仍然是不可避免的……社会主义不是依靠小生产可以建设起来的,而是必须依靠社会化的大生产,首先是工业的大生产来从事建设。”“要达到社会主义,实现社会主义的工业和农业,必须经过新民主主义经济一个时期的发展,在新民主主义社会中大量地发展公私近代化工业”\footnote{有几种文献数据汇编中可以看到新华社这个问答的全文。如中国人民解放军政治学院党史教研室编《中共党史参考数据》第11册,165-168页。}。这个“问答”用新华社的名义来解释毛主席的讲话,很显然,是传达了当时中央的意旨。毛主席的讲话和新华社的“问答”,在各路解放大军纷纷进入城市时起了很大的积极的影响。

从上面所引的毛主席的三段话,可以看出,民粹主义思想是可能从小农经济中自发产生的,在这种思想影响下,革命就会发生破坏工商业和城市的举动。试看太平天国的历史,那时的人们当然不知道什么民粹派,但太平天国明确地提出了,实际上也执行了一套自己的政策。他们在打到南京后,用现在的话说,一进城就“共了产”,把所有商铺的财产都分门别类地收归军队所有,实际上就是消灭社会上存在的工商业。太平天国也主张实行一种农业上的绝对平均主义,按照人口来计算,平均给每一个人分多少土地,规定每一家人能有多少财产,能养几只鸡等,多余的财产都要交公。虽然太平天国的这套主张没有也不可能完全实行,但可以看到,这就是在中国农民中自发产生的民粹主义思想。所以,在中国革命完全胜利、全国解放的前夕,毛主席特别提出绝对平均主义的问题,提醒全党注意,而且加上一个“农业社会主义”的称呼。后来编《毛泽东选集》的时候,为什么删去这个称呼,未听到过权威的解释。大概是因为“反对农业社会主义”很容易在字面上引起误解,使人以为是反对农业的社会主义改造。

确实,到了革命大军进城时,绝对平均主义成为极需十分注意的问题。如毛主席所说,农村中流行着一种破坏工商业、分配土地主张绝对平均的思潮。在旧中国,私营工商业差不多都和土地剥削有联系。当城市不在我们手里的时候,地主避到城里,农民就没有办法了。当时城市已经解放,如果农民纷纷进城分地主财产,就可能导致破坏工商业、破坏城市的后果。在分配土地上不能采取绝对平均主义,在城市实行绝对平均主义,那就更加有害了。看来农业社会主义这个概念比绝对平均主义含义更多,突出地显示了要不要发展工业的问题,或者更准确地说,要不要工业、农业和其他方面的现代化的问题。

这里说的是全国解放时的情况。那么在此以前,在农村打游击,搞农村革命根据地的时期,这个问题是不是就没有那么重要呢?不,这个问题同样十分重要。中国革命中有许多问题要从中国的实际出发,用马克思主义基本原理去解决。其中一个重要问题,就是资本主义和社会主义的关系问题,也就是用民粹主义思想还是根据马克思主义理论去处理这个关系的问题。在长期的中国革命进程中,党在这个问题上有没有沾染一些民粹主义气味?是不是能杜绝民粹主义?是不是因避免民粹主义而走向相反的另一极端?需要认真研究。实践证明,这个问题如果解决得好,中国革命就顺利;解决得不好,就会产生各种各样的问题。中国共产党历史的发展清楚地说明了这一点。

2

中国革命说得长远一些,可以一直追溯到戊戌变法、辛亥革命。那时候,中国先进的人们要解决中国的问题,使中国前进,走什么道路呢?他们认为,应以西方国家为师,走资本主义道路。在那时,这种主张是唯一进步的主张,如果能使中国从半殖民地半封建的境地,变成独立的资本主义国家,当然是个很大的进步。

经过第一次世界大战,资本主义制度的弱点在世界上充分暴露出来,大不吃香了。十月革命以后,中国出现了社会主义思潮。这时候,中国共产党还没有成立,只有一些共产主义小组,有一些学习马克思主义、信仰社会主义的青年。这时期有过一场论战,以梁启超、张东荪为代表的一些人反对讲社会主义,起来跟他们辩论的是陈独秀、李达等人。有的党史书讲那时有三大论战,其中就有这次关于社会主义的论战。一般的叙述是说,社会主义是新生力量,论战一展开,社会主义论者就把资本主义论打垮了,“得胜回朝”,好像容易得很。但是仔细看一下论战的材料,并不是那么简单。梁启超、张东荪大致是说,中国现在太穷,很弱,受列强的压迫,经济十分落后,在这种情况下怎么能实行社会主义呢?他们说,中国大多数人是农民、游民,工人很少,还不配讲社会主义革命。所以当务之急应该是发展工业,这就只能靠走资本主义道路。社会主义思想、社会主义的党现在不需要,且等将来才有用。陈独秀等人立即起而反驳。这些为卫护社会主义而斗争的先进分子的勇气可嘉,但他们还没有真正从中国的情况出发进行研究,因而只是说,资本主义在世界上已经没落了,社会主义如日中天,再讲实行资本主义就太落后了。发展工业,不必靠资本主义,直接实行社会主义,更能发展工业。这样说,并不能真正驳倒对方。驳梁启超等人恐怕应该这么说:中国现在的确还很落后,不能马上实行社会主义,要先解决当前最迫切的反帝反封建问题。解决了这个问题,才能发展经济,以后才有可能实行社会主义。可是反帝反封建靠谁呢?要靠人民大众。这个革命,中国资产阶级领导不了,必须由无产阶级来领导,因此社会主义思想在革命中能起指导作用,以社会主义为宗旨的党在革命中有重要地位,这个党先搞民主革命,而以社会主义为远大目标,等等。当然,不能责怪陈独秀等人,那时他们不可能说出这一套来,这一套是经过后来长期的革命经验才能说出来的。梁启超等人的错误,不在于说中国现在还不能实行社会主义,而在于认为既然不能马上实行社会主义,就不需要社会主义者,不需要社会主义思想,不需要成立共产党,大家都应当一心一意奔资本主义。为反驳这种观点,显然要有较多的马克思主义才行。否则,只是斥责资本主义的罪恶,声讨资本主义的落后性,藉以论证中国应该立刻实行社会主义,那就实际上染上了民粹主义的色彩。

1921年中国共产党成立后,大家很快对马克思主义有了较进一步的认识。这时候来了俄国和共产国际的“教员”。他们满肚子都是马克思主义,就向中国共产党人说,你们现在还年轻,力量还太小,现在还不能独立干什么,要和国民党合作,先搞民主革命,反帝反封建,将来才能自己搞社会主义。中国共产党人开始并不相信,不愿意照这样做,后来觉得老大哥的话不错,就实行了第一次国共合作。中国共产党人也就开始懂得了,在中国,为社会主义而奋斗,必须先搞资产阶级民主革命。

可是,当时俄国和共产国际的人对中国的实际情况都不真正了解。他们教条主义地看问题,认为既然是资产阶级民主革命,革命胜利后就一定是建立资产阶级的政权。共产党现在只能去帮助国民党,等到资产阶级夺取政权了,共产党才有自己的“戏”,现在只是帮忙、跑腿,甚至于当苦力。其实,在半殖民地半封建的中国并没有能够比较彻底地进行民主革命的资产阶级。来中国的这些俄国人以及共产国际的专家,误以为蒋介石、汪精卫的国民党代表着革命的资产阶级。结果国共合作还没有取得全国政权,蒋介石、汪精卫已经投降帝国主义和封建主义,和他们并肩协力,反过来屠杀共产党人了。党史书说,党在这时期犯了右倾机会主义错误,这个错误在中国党内应由陈独秀负责,也是在共产国际和俄国人的指挥棒下造成的。这种右倾机会主义思想从根本上是出于这样的估计:既然中国是搞资产阶级民主革命,革命的领导权应属于资产阶级,政权也主要由资产阶级执掌。这种想法固然避免了民粹主义,却走到了和民粹主义绝对相反的另一错误极端,是和中国实际不符合的,既导致革命惨痛失败,也解决不了社会主义与资本主义在中国的关系问题。

年轻的中国共产党经过这番大挫折后,又站起来重新开始革命。在这种情况下,很容易出现“左”的情绪,不甘心再搞资产阶级民主革命,以为不如干脆搞社会主义革命。这时候,俄国的老大哥又起了一点教员的作用。在莫斯科召开的中共六大上,斯大林亲自出面,指出中国现在还是要搞资产阶级民主革命,革命有高潮、低潮,现在是低潮,以后还会出现高潮。老大哥对马克思主义毕竟是懂得多一点,说服了中国共产党人。中国共产党的力量在20年代后期和30年代前期又发展了起来。

这时发生了两个问题。当时的革命主要是党在农村中组织农民武装、打游击战、开辟根据地。共产国际和俄国领导人历来看不起这种农村斗争,认为共产党长期陷在农村中是没有前途的,甚至会改变自己的性质。那么怎样才能快些进城呢?这是一个问题。另一个问题是中国的资产阶级似乎已全部离开以至反对革命,那么资产阶级民主革命怎么搞下去呢?共产党的目的是社会主义,现在的资产阶级民主革命和将来的社会主义革命究竟怎样衔接起来呢?脱离中国实际的共产国际和俄国的领导人解决不了这两个问题,他们很自然地倾向于争取赶快进城,并且尽早地实现革命的转变--从资产阶级民主革命转变到社会主义革命。第二次国内战争时期以李立三、王明为代表的“左”倾冒险主义、教条主义,就是受共产国际和俄国人的影响而产生的。

这里提供一个材料为例。1930年6月,共产国际执行委员会政治秘书处关于中国问题的决议说:“中国革命运动的新高涨已经成为不可争辩的事实。”“应该加紧全部力量去发展政治罢工,准备在一切或几个工业中心的政治大罢工。”“中国革命的当前阶段是带着(他们不明确说“是”,而说“带着”--引者)资产阶级民主主义的性质”,但“和普通的资产阶级民主革命不同”,是“因为工人与农民是在直接同资产阶级的斗争中去实行资产阶级民主阶段的任务”。“它在取得胜利时就开辟社会主义发展的前途”。“中国的民主专政将不得不一贯到底地没收中外资本家的企业,不得不实行很重大的社会主义性质的步骤”。这些话很明显地是混淆了两个革命阶段,在民主革命阶段就要实行重大的社会主义性质的步骤。这个决议还说:“中国革命转变为社会主义革命的期限,将比按照俄国1905年革命条件所预料的,要大大缩短”。“中国国内的经济条件提出非资本主义进化的必要”,也就是“苏维埃的、非资本主义的、社会主义的发展道路”\footnote{《共产国际有关中国革命的文献资料》,中国社会科学出版社1982年版,第92页。}。这些话显然在当时中共党内很容易煽起一种“左”的情绪,以至形成“左”的路线。

从当时“左”的路线中,可以很明显地看出有这样两个特点:一是在资产阶级民主革命时期,不只是反帝反封建,还要一般地反资本主义。资本主义在哪里呢?主要在大城市。李立三、王明路线的领导,都承认我们现在要搞资产阶级民主革命,而且只好先在农村搞武装斗争,但是在那样落后的农村中谈什么反资本主义,未免太可笑了。所以他们都不顾条件是否成熟,急于攻打大城市,以便进一步结合工人阶级反对资产阶级。二是在反帝反封建的同时,反对任何“中间势力”。中间势力是什么呢?实际上主要就是民族资产阶级的力量。李立三、王明认为,革命一发展到大城市,资产阶级民主革命就要立刻直接转变为无产阶级社会主义革命。李立三在那时说过的这句话可谓典型:“革命胜利的开始,革命政权建立的开始就是革命转变的开始,中间决不会有丝毫的间隔。”\footnote{李立三1930年5月发表的论文中语,见《中共党史教学参考数据》第1册,人民出版社1978版,第273页。}而且那时的立三认为,只要拿下一两个省,就是“革命胜利的开始”。30年代前期“左”倾错误的结果,大家知道,几乎葬送了中国红军和中国革命。

30年代前期这种“左”倾机会主义,人们很少讲它和民粹主义的关系。这种“左”倾论调并不表示对小农经济的崇拜,这的确是和民粹主义不同的,但可以说,它的基本性质是类似于民粹主义的。因为它以为可以在经济很落后的情况下,即普遍存在着小农经济的情况下一下子将民主革命转到社会主义革命去。毛主席在党的七大联系着批评民粹主义时说,我们的同志对消灭资本主义急得很,在这方面是过急了。他所批评的这种急性病,是从30年代遗留下来的。

3

遵义会议纠正了党的“左”倾错误。以毛泽东为核心的党中央成功地领导党进入抗日战争。这里讲一点个人的经验。抗日战争开始时,我刚刚参加共产党,当然完全拥护党采取国共合作的方针,争取抗日战争的胜利。这是当时头等的大事。但是心里不能不怀疑,现在同国民党合作,抗日胜利后中国到底变成什么样子,怎样变成我们所要的社会主义?实在不懂。我就请教一些老同志,和他们讨论。老同志说,咱们当然是要搞社会主义,抗日战争的结果将是非资本主义的前途。讲非资本主义的前途,这很好。抗日战争胜利后,当然应当不是资本主义前途。但非资本主义是什么,就是社会主义吧?可为什么不说社会主义,而说非资本主义?搞不清楚。一些老同志也解释不了。

毛主席在抗日战争初期的1939年12月发表《中国革命和中国共产党》,又在1940年1月发表《新民主主义论》。提出和解释新民主主义的这两篇论文有十分伟大的意义,解决了前两个时期的经验仍说不清楚的问题。他说,中国资产阶级民主革命取得胜利,包括抗日战争胜利后,不能变成资产阶级专政的资本主义社会,但也不能马上变成社会主义社会。这个社会里有社会主义因素,但又有资本主义因素。无产阶级在资产阶级民主革命(包括抗日战争)中要争夺领导权,以至掌握领导权。这就不是旧民主主义革命,而是新民主主义革命。他在中国革命的历史上,第一次提出新民主主义革命的概念,指出新民主主义革命“虽然按其社会性质,基本上依然还是资产阶级民主主义的,它的客观要求,是为资本主义的发展扫清道路﹔然而这种革命……是新的、被无产阶级领导的、以在第一阶段上建立新民主主义的社会和建立各个革命阶级联合专政的国家为目的的革命。因此,这种革命又恰是为社会主义的发展扫清更广大的道路。”\footnote{《毛泽东选集》第2卷,人民出版社1991年版,第668页。}

毛泽东的新民主主义论一提出来,使人们眼界豁然开朗,一下子清楚了,明确了。他的新民主主义理论,确是为中国革命当前任务和它的前途作出了科学的、符合实际的、易于了解的论断。

1945年,毛主席在写七大报告--《论联合政府》时,把《新民主主义论》中的许多观点更进一步往前推进了。《论联合政府》中说:“中国也不可能、因此就不应该企图建立一个纯粹民族资产阶级的旧式民主专政的国家”。“在中国的现阶段,在中国人民的任务还是反对民族压迫和封建压迫,在中国社会经济的必要条件还不具备时,中国人民也不可能实行社会主义的国家制度。”\footnote{《毛泽东选集》第3卷,人民出版社1991年版,第1055页。}这个思想和《新民主主义论》的提法是一贯的。《论联合政府》里有一段很有名的话:“有些人不了解共产党人为什么不但不怕资本主义,反而在一定的条件下提倡它的发展。我们的回答是这样简单:拿资本主义的某种发展去代替外国帝国主义和本国封建主义的压迫,不但是一个进步,而且是一个不可避免的过程。它不但有利于资产阶级,同时也有利于无产阶级”,后来编《毛泽东选集》的时候,在这里又加了一个短语,“或者说更有利于无产阶级”。接着说:“现在的中国是多了一个外国的帝国主义和一个本国的封建主义,而不是多了一个本国的资本主义,相反地,我们的资本主义是太少了。”\footnote{《毛泽东选集》第3卷,第1060页。}他特别申明:我们中国共产党人是根据自己对于马克思主义的社会发展规律的认识,来明确地认识这一点的\footnote{《毛泽东选集》第3卷,第1060页。}。后来在编《毛泽东选集》的时候,毛主席把过去的有些文章做了若干文字修改,可是这段话没有改动,甚至还如上所述加了一个短语。

毛主席关于七大的其他一些讲话,过去人们只能在档案馆里查阅,现在有了中央文献研究室编辑的《毛泽东文集》和《毛泽东在七大的报告和讲话集》,都可以看到了。那时,毛主席在解说七大的报告时说:“这个报告与《新民主主义论》不同的,是确定了需要资本主义的广大发展……资本主义的广大发展在新民主主义政权下是无害有益的”\footnote{《毛泽东文集》第3卷,第275页。}。他在七大的口头报告里又说:“在我的报告里,对资本主义问题已经有所发挥,比较充分地肯定了它。这有什么好处呢?是有好处的。我是在这样的条件下肯定的,就是孙中山所说的'不能操纵国民之生计'的资本主义。至于操纵国民生计的大地主、大银行家、大买办,那是不包括在里面的。”“我们这样肯定要广泛地发展资本主义,是只有好处,没有坏处的。对于这个问题,在我们党内有些人相当长的时间里搞不清楚,存在一种民粹派的思想。”\footnote{《毛泽东文集》第322-323页。}

这里,毛主席又联系到民粹主义问题。他的意思是,如果不承认只有经过民主主义才能到达社会主义,不承认新民主主义政权下还需要资本主义的广大发展,那就和民粹主义区别不开了。那么,毛主席的这种观点是不是马克思主义的呢?是不是马克思主义,不能光找书本,拿马克思主义的书来核实一下有没有这个话。那是不行的。马克思主义书本里面,没有讲中国搞新民主主义,还可以发展资本主义,在什么条件下应该发展资本主义等等。因此,是不是马克思主义,应该看是否基于马克思主义的立场、观点和方法而完全符合于中国的实际状况和需要。革命在全国胜利以前,我们党对中国国情有一个估计,具体数字不一定完全准确,但总的估计是正确的。这就是认为在抗日战争前夜(那是旧中国经济发展最高的时候),全国范围内现代性的工业大约只占国民经济的百分之十左右,农业、手工业占百分之九十左右。毛主席1949年3月在党的七届二中全会报告中指出上述基本国情后说:这是“在中国革命的时期内和在革命胜利以后一个相当长的时期内一切问题的基本出发点”\footnote{《毛泽东选集》第4卷,第1430页。}。党的新民主主义的方针以及对资本主义的政策,是从这种具体情况出发的。由此得出的对私人资本主义的结论是:“在革命胜利以后一个相当长的时期内,还需要尽可能地利用城乡私人资本主义的积极性,以利于国民经济的向前发展。”\footnote{《毛泽东选集》第4卷,第1431页。}在马克思主义老祖宗的书本里是找不到这些的。

毛主席还从抗日战争的经验得出一个现成书本上没有,令教条主义者吃惊的结论。他说:“无产阶级是可以领导资产阶级的。我们要按实际办事,不是按书本办事,而王明则反对无产阶级领导资产阶级,说列宁没有讲过。”\footnote{《毛泽东文集》第3卷,第74页。}

4

研究新民主主义论是不是马克思主义的,还可以看一下1944年毛主席给博古的那封有名的信。在这封信里他说:“新民主主义社会的基础是工厂(社会生产,公营的与私营的)与合作社(变工队在内),不是分散的个体经济。分散的个体经济--家庭农业与家庭手工业是封建社会的基础,不是民主社会(旧民主、新民主、社会主义,一概在内)的基础,这是马克思主义区别于民粹主义的地方。”这里又一次讲到民粹主义。可见,毛主席那时候经常想到要防止民粹主义的问题。信上接着说,“简单言之,新民主主义社会的基础是机器,不是手工。我们现在还没有获得机器,所以我们还没有胜利……现在的农村是暂时的根据地,不是也不能是整个中国民主社会的主要基础。由农业基础到工业基础,正是我们革命的任务。”\footnote{《毛泽东书信选集》,人民出版社1983年版,第238-239页。}这里的观点非常鲜明。有人以为毛主席出身农村,就对农村偏爱,他不是这样的。

当时在根据地(解放区)建立的社会,还说不上是完全的新民主主义社会,因为“新民主主义社会的基础是机器”。毛主席确实是结合中国实际情况运用了马克思主义。马克思主义认为,经过资本主义的发展才可能到社会主义,因为只有在资本主义创造的生产力的基础上才能建立社会主义。但并不是一定要经过资产阶级统治的那种资本主义社会。马克思1881年在给一个俄国友人写的信稿中说:俄国农村公社“能够不通过资本主义生产的一切可怕的波折,而吸收它(资本主义)的一切肯定的成就”\footnote{《马克思恩格斯全集》第19卷,第431页。}。马克思对传统的俄国农村公社的这个指望后来并没有实现。但由此可见,马克思并不认为“资本主义的一切可怕的波折”是不可避免的,他认为社会主义必须吸取资本主义的“一切肯定的成就”。正是在这点上,民粹主义和马克思主义不兼容。毛泽东的新民主主义论就包含着在中国的具体条件下,如何利用资本主义以发展社会主义的内容,指出了一条不经过资产阶级专政的资本主义社会,避免那种“可怕的波折”但又吸收资本主义的一切“肯定的成就”的路子。

马克思主义对旧世界的批判和对旧世界崩溃以后的预见,都是科学的。但是,在马克思、恩格斯年轻的时候,曾经把形势估计得过于乐观,认为那时已经到了资本主义崩溃的时候了。在19世纪40年代,他们过高地估计了西欧资本主义陷于崩溃的形势。恩格斯晚年说:“当时欧洲大陆经济发展的状况还远没有成熟到可以导致铲除资本主义生产方式的程度”\footnote{恩格斯:《法兰西阶级斗争导言》,《马克思恩格斯全集》第22卷,第595、597页。}。所谓是否成熟到可以铲除资本主义生产方式的程度,这要根据事实,根据客观的经济社会条件,而不能凭主观的愿望。历史还告诉我们,无产阶级和它的先锋队无产阶级政党是不是有力量夺取政权,这和就经济状况说铲除资本主义生产方式的条件是否已经成熟,两者不一定是一回事。并不一定是到了铲除它的条件已经完全成熟的时候,无产阶级才有可能夺得政权﹔当然也不一定是无产阶级夺取了政权时,铲除资本主义的条件已经成熟。无产阶级政党能否取得胜利,掌握政权,与各种国内国际条件有关,只要形势有利,就应该紧紧抓住时机,毫不放松,夺取胜利,中国共产党就是这样做的。毛主席的伟大就在这里,不仅提出了新民主主义论,而且领导全党和全军在新民主主义革命的紧要关头大胆地跃进,取得了胜利。小平同志说,如果没有毛主席,我们革命的胜利可能要晚几十年。在胜利前,毛主席又早已清醒地看到,中国革命将在资本主义不是太多而是太少的情况下取得胜利,因此对新民主主义社会有了种种设想,而对从民主革命到社会主义革命的过渡,采取十分慎重的态度。

在无产阶级政党取得政权后,还不具备全面地实现社会主义社会的条件和可能,怎么办呢?那就要经过迂回的道路。

绝不能因为社会主义革命的条件还没有成熟,就等着而不去夺取政权。无产阶级革命家要抓住机遇夺取革命的胜利,然后再在无产阶级政权下补生产力和其他文化条件的课。这就是中国革命必须要走新民主主义道路的道理。这就是新民主主义论主张在革命胜利、无产阶级夺得政权后一个相当长的时期内,一切有利于国民经济的资本主义成分应允许其存在和发展的根据。

这究竟是否符合于马克思主义呢?可以引用马克思主义的一句名言,“无论哪一个社会形态,在它们所能容纳的全部生产力发挥出来以前,是决不会灭亡的﹔而新的更高的生产关系,在它存在的物质条件在旧社会的胎胞里成熟以前,是决不会出现的。”\footnote{《马克思恩格斯选集》第2卷,人民出版社1995版,第33页。}马克思在《〈政治经济学批判〉序言》里的这句话,说的是马克思主义历史唯物主义的一个基本论点,一个原理。马克思主义是无产阶级的革命理论,是要以社会主义社会代替资本主义社会,以社会主义世界代替资本主义世界,就这意义,可以说这是“兴社灭资”论。但如果脱离上述的这个原理而只知“兴社灭资”,那就可能并不是马克思主义的科学社会主义,而是空想社会主义、冒险主义、民粹主义,或别的什么。

毛泽东的新民主主义理论在指导中国革命取得胜利过程中,有两个方面最能显示它的特点:完全是从中国实际出发,是马克思主义原有的书本上没有的,但又的确符合马克思主义原理。一个方面是农民问题。这就是,在农村党领导组织农民武装,建立农村革命根据地,在战争和革命中提高农民水平(可以说,使他们逐渐摆脱自发的民粹主义、农业社会主义),以农村包围城市,武装夺取政权,最后取得全国胜利。另一个方面就是资本主义问题。和“左”倾机会主义在民主革命过程中就反对一切资产阶级不同,毛主席的理论是,要区别民族资产阶级即中等资产阶级和大资产阶级。对前者要采取慎重的政策,不是一概打倒,一律反对。解放战争时党的三大经济纲领,一个是土地改革,一个是没收官僚资本,还有一个就是保护民族工商业。在革命取得全国胜利的时候,党从农村进入城市,毛主席和党中央在强调全心全意依靠工人阶级的同时,十分注意团结一切民主党派,团结主要是中等资产阶级的社会力量,同他们合作共事,一起建立民主共和国。就因为中间阶层虽然有其弱点,但在中国是最有文化的,并且有一些资本和办工业的本领,是可以影响很多人的。如果不团结这个力量,那么我们纵然进了城,也将在城市中站不稳脚﹔而且很可能助长我们队伍中的民粹主义、农业社会主义的倾向,那就更有遭致失败的危险。新民主主义革命之所以能取得胜利,就因为有工人阶级、共产党的领导,团结、动员了中国人口中占最大多数的农民,还组成了一个由参加这个革命的一切人组成的统一战线。毛主席当时说:这个“统一战线是十分广大的,这里包括了工人、农民、独立劳动者、自由职业者、知识分子、民族资产阶级以及从地主阶级分裂出来的一部分开明绅士”\footnote{《毛泽东选集》第4卷,第1313页。}。依靠包含上述两个特点的新民主主义理论的指导,才可能有共产党领导的、以工农联盟为基础的这样广大的统一战线,也才可能取得并保持民主革命的胜利。

5

也许有人说,在新民主主义革命完成以后,新民主主义理论也没有什么现实价值,而只可供历史的回顾了。我以为不是这样的。这个理论不仅对新民主主义革命有指导作用,而且有助于我们考虑建国以后的一些问题,以至今天我们研究社会主义初级阶段理论和实行改革开放政策时,也还可以从中得到某些启发。

建国之初,我们党对何时和如何进行社会主义革命的问题采取了非常慎重的态度。在召开人民政协制定《共同纲领》时,有的民主人士建议要在共同纲领里提到社会主义,我们党的领导人都说暂不提。毛泽东和刘少奇、周恩来等领导人那时涉及这个问题的讲话,都是按新民主主义论的精神讲的。例如,1952年10月周恩来说:“毛主席的方针是稳步前进,三年恢复,十年、二十年发展。发展新民主主义经济可能要十年、二十年”\footnote{《周恩来经济文选》,中央文献出版社1993年版,第122页。}。这里,我讲一件自己经历的事情。1952年开展“三反”、“五反”运动后,当时很流行的《学习》杂志上有篇文章写了“给资本主义敲了丧钟”这样的话。毛主席看到后,大加批评,说现在怎么能给资本主义敲丧钟?还远不到这时候!

到了1953年,事情有了改变。那时说,社会主义革命从1949年已经开始。其实,这个说法不大能够服人。毫无疑问,人民共和国从一开始已经通过没收官僚资本,有了国营经济,这就是有了社会主义经济因素。而且共产党取得政权就是个重大的社会主义因素。但是,这跟社会主义革命是两回事情。建国时,在《共同纲领》中不提社会主义,当时认为必须有相当严重的社会主义步骤,才是社会主义时期的开始,在这以前,是新民主主义时期。1953年,宣布全面发动对农业、手工业的个体经济和私人资本主义工商业的社会主义改造,这确实是社会主义的重大步骤。而原来党内的共识是,这种步骤要在三年准备,十年建设之后才采取的。应该说,在建国后三年党的方针有了明显的改变。

我们党的方针政策,不是在任何情况下一成不变的,是应该随着形势的变化而变化的。那么从40年代初到50年代初,中国发生了什么变化呢?翻天覆地的显著变化就是人民共和国建立,共产党掌握了政权。随着这个变化,社会生活的各方面当然都发生了很多变化。但是所有这些变化都还没有能改变农业生产和农村经济落后的状况(农民使用的几乎还是两千年前老祖宗使用的工具,自然经济和半自然经济支配着农村),还没有能从根本上改变现代性工业生产只占国民经济中很小的百分比的情况。这也就是说,使得毛主席1945年在七大讲的,资本主义在中国不是太多,而是太少的情势并没有发生根本变化。

讲中国的资本主义不是太多,而是太少,这是马克思主义的科学语言。用普通常识的眼光来看,既然最后要消灭资本主义,那么似乎应该是资本主义愈少愈好,国为愈少就愈容易消灭。过去有很多人是这样想的。孙中山就这样看,他说过,趁资本主义还少,甚至还没有的时候,赶快搞社会主义革命,这样做,比以后资本主义多了时再搞社会主义革命容易得多。这是他在20世纪初讲的话。我们不能责备孙中山。他不可能懂得,资本主义不能只被看作一种罪恶,它能为社会主义提供必要的物质准备。马克思主义者是从客观社会发展规律来认识资本主义的历史作用。如果认为趁资本主义还少,还没有发展起来,就可以马上过渡到社会主义,这就是倒向民粹主义,而离开马克思主义。

拿1949年-1953年和1945年相比,资本主义恐怕并不是更多一点,而是更少了一点。官僚资本的很大部分跑到外国去了,还有一部分被带到台湾去了(这当然对我们建立社会主义国营经济不利)。很多民族资本家为了避免战祸,也由于对国民党政权崩溃后的形势看不准,把他们的资产(或其中的一部分)转移到国外和香港。经过八年抗日战争和三年国内战争,1949年时中国资本主义经济的总量,看来也不会比1936年时较多,而是更少些。到解放以后,由于人民共和国实行公私兼顾的政策,私营工商业有过一个发展的“黄金时期”,但也不可能发展得那么快。

1953年党宣布过渡时期的总路线。表述这条总路线的标准语言是:“从中华人民共和国成立到社会主义改造基本完成,这是一个过渡时期。党在这个过渡时期的总路线和总任务,是要在一个相当长的时期内,逐步实现国家的社会主义工业化,并逐步实现国家对农业、对手工业和对资本主义工商业的社会主义改造”。所谓相当长的时期,是指人民共和国最初三年的经济恢复时期和这以后的三个五年计划时期,共18年。按这些叙述,过渡时期的结束是以社会主义改造基本完成为标志。但事实上,社会主义改造的速度出乎人意料。在1953年后三年,1956年已经“完成”了对农业、手工业、资本主义工商业的社会主义改造。至于社会主义工业化,第一个五年计划(1953-1957)经济建设取得了很好的成就,五年间农业总产值增长了25\%,年均增长4.5\%,工业总产值增长了128.6\%,年均增长18\%。可是因为起点很低,虽然速度快,但距离社会主义工业化完成当然还遥远得很。所以毛泽东1956年在《论十大关系》中说,中国的特点是“一穷二白”,“穷就是没有多少工业,农业也不发达,白就是一张白纸,文化水平科学水平都不高”。当时人们称社会主义工业化为总路线的“主体”,而称社会主义改造为“两翼”。主体与两翼不像预计那样平行地发展,其原因是什么,后果会怎样,特别是社会主义改造的飞速完成,是符合实际的要求呢,还是主要依靠政权力量人为地促成?这本来是值得考虑的问题。当时的领导人并不明确地认为作为两翼的“三改”过于超前,但确认工业化落后,于是1958年搞“大跃进”,想用群众运动的方式把工业化一下子搞上去,结果没有成功。同时,“两翼”似乎根本不考虑是否与“主体”相适应而仍继续猛进。最显著的是农业方面,1958年全部农村都跃进到了人民公社化。那时社会上有人民公社是进入共产主义的“金桥”的说法,虽然这是一部分基层干部的创造,但这种认识和那时的领导思想不无关系。可以说,领导思想失之毫厘,民粹主义的思想就在下面大为膨胀。当农业生产力没有任何显著提高,国家的工业化正在发端的时候,认为从人民公社就能够进入共产主义,这是什么思想?只能说这种思想在实质上属于民粹主义的范畴,和马克思主义距离很远。党的领导,首先是毛主席很快地发现这个问题,采取步骤纠正“共产风”,并对农村人民公社的有些方面作了些调整。但是始终没有充分的事实能够说明,人民公社制度是与中国农村实际相适应的,是有益于提高农业生产力、发展农村经济的。事实证明的恰好相反,但人民公社制度直到80年代初才终于被取消。

在人民公社存在的20多年间,与人民公社制度相联系的种种混乱思想,如“割资本主义尾巴”、“穷过渡”一直扰乱人心,在实践中起坏作用。这里只说一下“穷过渡”。那时人们所说的过渡有种种层次:或过渡到大队所有制,或过渡到社有制,或过渡到全民所有制,或过渡到共产主义,总之,是过渡到被认为是社会主义更高一级的台阶上。其所以能过渡,不是因为生产力发展,不是因为富,而是因为穷,是“趁穷过渡”。这种穷过渡的思想,当然只能使人联想到民粹主义。

中国在新民主主义革命胜利,“推翻了三座大山”以后,“一穷二白”就是中国落后的根源。尽管不妨把一穷二白形容为一张纯洁的白纸,但在上面绝不可能任意画出最新最美的图画。

6

毛泽东的新民主主义理论,也能帮助我们深入理解提出社会主义初级阶段理论的重要意义,深入理解近20年来改革开放的必要性和有关政策的正确性,并且从历史的、理论的角度深入认识为什么要把发展生产力摆在首要地位。

历史不会绝对地重复,但是在革命进程中,的确有时候像列宁所说的,人们不得不一再重复做一件事情。列宁在十月革命后通过实践看到,想用一个冲锋就进入社会主义是不可能的,他就毅然地改变方针,实行革命的退却,开始新经济政策。列宁在1922年,即十月革命后五年曾这样说:“在一个小农国家里,要奠定社会主义经济基础,不可能“不犯错误,不实行退却,不一再重新做那还没有完成和做得不对的事情”,因此要求共产党人“一直保持着有机体的活力和灵活性,准备再一次‘从头开始’最困难的任务”\footnote{《列宁选集》第4卷,人民出版社1972年版,第597页。}。

新中国建立后的差不多前30年,回顾起来有许多毛病、错误,甚至于有些是根本不应该那样做的﹔但我们毕竟还是取得了许多成就,而且更重要的是积累了许多过去没有的经验。经过30年间的曲折道路,无论如何工业水平比1949年高得多了,但是仍然远不能说是摆脱了一穷二白的境地。那30年间得到的经验中最重要的一条,是决不能不顾生产力发展的水平而追求社会主义生产关系的提高,这种提高不但不是真正的提高,而且只会对生产力的发展和社会的进步起阻碍作用。这条经验是马克思主义区别于民粹主义的要害。1978年的十一届三中全会后,党接受了30年来的经验,也在30年来的成就的基础上重新开始中国的社会主义伟大事业。当然不是回到1949年。只就经济上说,第一,社会主义全民所有制经济的力量大为增加﹔第二,解放前的私营资本主义已经消失﹔第三,农民从合作化到公社化的经验中,学到了应该如何做和不应该如何做。形势发生了这样大的变化,因此我们不可能重新走新民主主义道路。

在十一届三中全会后实行开放政策时,党的文件中往往特别标明,开放的对象包括资本主义发达的国家\footnote{例如1984年十二届三中全会通过的《中共中央关于经济体制改革的决定》中说:“必须吸收和借鉴当今世界各国包括资本主义发达国家的一切反映现代化生产规律的先进管理方法”。}。党在收回香港、澳门的政策和准备施行于台湾的和平统一政策中,都明确宣布要按“一国两制”的原则办事,即在这几个地区继续实行资本主义制度。国家以宪法肯定我国今后将长期处于社会主义初级阶段,因而也就要长期坚持公有制为主体、多种所有制经济共同发展的基本经济制度,长期坚持按劳分配为主体、多种分配方式并存的分配制度,从而非公有制经济成为我国社会主义市场经济的重要组成部分。非公有制经济包括私营经济和个体经济,在过去是作为资本主义和资本主义尾巴而被排斥的。私营经济中的大部分(也许有些是例外),在性质上同旧社会中的私人资本主义经济相同,但并不完全相同而有自己的特色。因为它是在共产党领导的社会主义国家中产生的,它是社会主义市场经济中的一个组成部分,它受到国家的引导、监督和管理。具有资本主义性质的这部分私营经济,与其他非公有制经济一样,其合法的权利和利益受到国家的保护。所有这些都是积极地利用资本主义,以促进社会主义发展的政策。如果按民粹主义的思路,这些政策是不可设想的。民粹主义思路名为憎恶和厌弃资本主义,实为害怕资本主义,躲避资本主义。我们要坚持社会主义公有制在我国经济制度中为主体,以保证不致走向与民粹主义相反的另一极端,即走向资本主义社会,同时也有必要注意防止类似于民粹主义的偏向,即防止以为好像不需要再把发展生产力摆在首要地位,以比较落后的生产力,就可以进入社会主义的比较高级的阶段的倾向﹔防止急于消灭资本主义,而不知道充分利用资本主义的必要性的倾向。

那么,为建设社会主义而利用资本主义,会不会遇到什么风险呢?应该说,不能绝对排除各种大的小的风险。做任何新的事情,走任何新的路子,都不可能没有一点风险,没有一点副作用。一帆风顺,坐着不动,打瞌睡就可以到达彼岸去,这种事情是没有的。我们只有用马克思主义观察中国的现实,按照实际情况,大胆地提出新的路子,防止和克服可能发生的种种风险。马克思主义的道路从来不是一条平坦笔直的、绝对平稳安全的、毫无风险的路。

所以,我认为,重新学习、认识毛主席关于新民主主义的理论,以及他用马克思主义原理正确处理在中国历史条件下资本主义和社会主义关系的完整理论,对于我们今天理解邓小平理论,理解改革开放和社会主义初级阶段的理论及其方针政策是很有帮助的。

\newpage
\chapter{十月革命四周年 列宁 1921/10/14}
\newpage

10月25日(11月7日)的四周年快到了。

这个伟大的日子离开我们愈远,俄国无产阶级革命的意义就愈明显,我们对自己工作的整个实际经验也就思考得愈深刻。

这种意义和这种经验可以极其简要地(当然是极不充分极不精确地)说明如下。

俄国革命直接的迫切的任务是资产阶级民主性的任务:打倒中世纪制度的残余,彻底肃清这些残余,扫除俄国的这种野蛮现象、这种耻辱、这种严重妨碍我国一切文化发展和一切进步的障碍。

我们有权引以自豪的是,从对人民群众的深远影响来看,我们所做的这种清除工作比125年多以前的法国大革命要坚决、迅速、大胆、有效、广泛和深刻得多。

不论是无政府主义者还是小资产阶级民主派(即孟什维克和社会革命党人,他们是国际上这一社会阶层的俄国代表)在资产阶级民主革命和社会主义革命(即无产阶级革命)的关系问题上,过去和现在都讲了不知多少胡涂话。四年来的事实已经完全证实,我们在这一点上对马克思主义的理解和对以往革命经验的估计是正确的。我们比谁都更彻底地进行了资产阶级民主革命。我们完全是自觉地、坚定地和一往直前地向着社会主义革命迈进,我们知道社会主义革命和资产阶级民主革命之间并没有隔着一道万里长城,我们知道只有斗争才能决定我们(最终)能够前进多远,能够完成无限崇高的任务中的哪一部分,巩固我们胜利中的那一部分。这过些时候就会见分晓。其实现在我们已经看到,在对社会进行社会主义改造的事业中,对一个满目疮痍、苦难深重的落后国家来说,我们已经做了很多很多工作。

可是,我不准备多谈我国革命的资产阶级民主主义内容。马克思主义者应当懂得这一内容指什么。为了说明问题,我们举几个明显的例子。

我国革命的资产阶级民主主义内容,指的是消灭俄国社会关系(秩序、制度)中的中世纪制度,农奴制度,封建制度。

到1917年,俄国农奴制度究竟还有哪些主要表现、残余或遗迹呢?还有君主制、等级制、土地占有制、土地使用权、妇女地位、宗教和民族压迫。试从这些“奥吉亚斯的牛圈”──顺便说一下,一切先进国家在125年和250年前以至更早以前(英国在1649年)完成它们的资产阶级民主革命时,都在很大程度上留下了没有打扫干凈的奥吉亚斯的牛圈──试从这些奥吉亚斯的牛圈拿出任何一间来,你们都会看到,我们已经把它打扫得干干凈凈。从1917年10月25日(11月7日)到解散立宪会议(1918年1月5日)这十来个星期里,我们在这方面所做的工作,比资产阶级的民主派和自由派(立宪民主党)以及小资产阶级民主派(孟什维克和社会革命党人)在他们执政的八个月里所做的要多千百倍。

这些胆小鬼、空谈家、妄自尊大的纳尔苏修斯和哈姆雷特总是挥舞纸剑,可是连君主制都没有消灭!我们却把全部君主制垃圾比任何人任何时候都更干凈地扫除了。我们没有让等级制这个古老的建筑留下一砖一瓦(英、法、德这些最先进的国家至今还没有消除等级制的遗迹!)。等级制的老根,即封建制度和农奴制度在土地占有制方面的残余,也被我们彻底铲除了。伟大十月革命的土地改革“最终”会有怎样的结果,这个问题“可以争论”(国外有足够的著作家、立宪民主党人、孟什维克和社会革命党人来争论这个问题)。我们现在不愿把时间花在这些争论上,因为我们正在用斗争来解决这种争论以及与此有关的许多争论。然而有一件事实是无可争辩的:小资产阶级民主派与保持农奴制传统的地主“妥协了”八个月,而我们在几星期内就把这些地主连同他们的一切传统都从俄国的土地上彻底扫除了。

就拿宗教、妇女的毫无权利或非俄罗斯民族的被压迫和不平等地位来说吧。这些都是资产阶级民主革命的问题。小资产阶级民主派这些鄙俗之徒在这些问题上空谈了八个月。世界上没有一个最先进的国家按照资产阶级民主方针彻底地解决了这些问题。而在我国,这些问题已由十月革命后颁布的法律彻底地解决了。我们一向在认真地同宗教进行斗争。我们让一切非俄罗斯民族成立了自己的共和国或自治区。在我们俄国,妇女无权或少权这种卑鄙、丑恶、可耻的现象,这种农奴制和中世纪制度的可恶的残余已经没有了,而这种现象却在世界各国无一例外被自私自利的资产阶级和愚蠢的吓怕了的小资产阶级重新恢复了。

这都是资产阶级民主革命的内容。在150年和250年以前,这一革命(如果就同一类型的每一民族形式来说,可以说是这些革命)的先进领袖们曾向人民许愿,说要使人类排除中世纪的特权,排除妇女的不平等地位,排除国家对这种或那种宗教(即“宗教思想”、“宗教信仰”)的种种优待,排除民族权利的不平等。许了愿,但没有兑现。他们是不可能兑现的,障碍在于要“尊重”……“神圣的私有制”。在我国无产阶级革命中,就不存在这种对倍加可恶的中世纪制度和对“神圣的私有制”的可恶的“尊重”。

但是,要巩固俄国各族人民所取得的资产阶级民主革命的成果,我们就应当继续前进,而我们也确实前进了。我们把资产阶级民主革命的问题作为我们主要的和真正的工作即无产阶级革命的、社会主义的工作的“副产品”顺便解决了。我们一向说,改良是革命的阶级斗争的副产品。我们不仅说过并且还用事实证明过,资产阶级民主改造是无产阶级革命即社会主义革命的副产品。顺便提一下,所有考茨基、希法亭、马尔托夫、切尔诺夫、希尔奎特、龙格、麦克唐纳、屠拉梯之流以及“第二半”马克思主义的其它英雄们,都不能理解资产阶级民主革命和无产阶级社会主义革命之间的这种相互关系。前一革命可以转变为后一革命。后一革命可以顺便解决前一革命的问题。后一革命可以巩固前一革命的事业。斗争,只有斗争,才能决定后一革命能比前一革命超出多远。

苏维埃制度就是由一种革命发展为另一种革命的明证或表现之一。苏维埃制度是供工人和农民享受的最高限度的民主制,同时它又意味着与资产阶级民主制的决裂,意味着具有世界历史意义的新型民主制即无产阶级民主制或无产阶级专政的产生。

让垂死的资产阶级和依附于它的小资产阶级民主派的猪狗们用数不清的诅咒、谩骂、嘲笑来攻击我们在建设我们苏维埃制度中的失利和错误吧。我们一分钟也没有忘记,我们过去和现在确实有很多的失利和错误。在缔造前所未有的新型国家制度这种全世界历史上新的事业中,难道能没有失利和错误吗?我们一定要百折不挠地努力纠正这些失利和错误,改变我们对苏维埃原则的实际运用远未达到尽善尽美的状况。但是我们有权自豪,而且我们确实很自豪,因为我们有幸能够开始建设苏维埃国家,从而开创全世界历史的新时代,由一个新阶级实行统治的时代。这个阶级在一切资本主义国家里是受压制的,如今却到处都在走向新的生活,去战胜资产阶级,建立无产阶级专政,使人类摆脱资本的桎梏和帝国主义战争。

关于帝国主义战争,关于金融资本所实行的目前左右着全世界的国际政策(这种政策必然会引起新的帝国主义战争,必然会导致极少数“先进”强国变本加厉地压迫、抢劫、掠夺和扼杀各落后的弱小民族)的问题,从1914年起就成为世界各国全部政策中的基本问题。这是一个有关千百万人生死存亡的问题。这关系到在我们眼看着资产阶级正准备的、从资本主义中产生出来的下一次帝国主义战争中是否会有2000万人死亡(而在1914─1918年的大战和附加的、至今还没有结束的“小”战中是1000万人死亡),在这一不可避免的(如果有资本主义存在)未来战争中是否会有6000万人残废(而在1914─1918年是3000万人残废)。在这个问题上,我们的十月革命也开辟了世界历史的新纪元。资产阶级的奴仆和应声虫社会革命党人、孟什维克以及全世界所有的假“社会主义”的小资产阶级民主派,都嘲笑“变帝国主义战争为国内战争”这个口号。其实这个口号是唯一的真理,虽然听起来令人不愉快、粗暴、赤裸裸、无情,的确如此,但同无数极其精巧的沙文主义与和平主义谎言相比,终究是一个真理。这些谎言被戳穿了。布雷斯特和约被揭露了。比布雷斯特和约更糟糕的凡尔赛和约的作用和后果,一天比一天更加无情地被揭露出来。千百万人都在思考着昨天战争的起因和行将到来的明天战争的问题,他们愈来愈清楚地、明确地、必然地认识到一个严峻的真理:不经过布尔什维克的斗争和布尔什维克的革命,就不能摆脱帝国主义战争以及必然会产生这种战争的帝国主义世界(如果我们还用老的正字法,我就会在这里写上两个含义不同的“”),就不能摆脱这个地狱。

让资产阶级和和平主义者、将军和市侩、资本家和庸人、一切基督教徒及第二国际和第二半国际的所有骑士们疯狂地咒骂这个革命吧。不管他们怎样不停地泄愤、造谣和诽谤,都不能抹杀一个具有世界历史意义的事实──千百年来奴隶们第一次公开地提出了这样的口号来回答奴隶主之间的战争:变奴隶主之间的分赃战争为各国奴隶反对各国奴隶主的战争。

这个口号千百年来第一次由一种模糊渺茫的期望变成了明确的政治纲领,变成了千百万被压迫者在无产阶级领导下进行的实际斗争,变成了无产阶级的第一次胜利,变成了消灭战争的第一次胜利,变成了全世界工人联盟对各国资产阶级联盟的第一次胜利,而资产阶级无论是和是战,无非都是牺牲资本奴隶的利益,牺牲雇佣工人的利益,牺牲农民的利益,牺牲劳动人民的利益。

这第一次胜利还不是最终的胜利。这次胜利是我国十月革命经历了空前的艰难、困苦和磨难,经历了很多重大的失败和错误以后取得的。难道一个落后国家的人民不经过失败和错误就能战胜世界上最强大最先进的国家所进行的帝国主义战争吗?我们不怕承认自己的错误,我们将冷静地看待这些错误,以便学会改正这些错误。但事实总是事实:用奴隶反对一切奴隶主的革命来“回答”奴隶主之间的战争的诺言,千百年来第一次得到了彻底的实现……并且还在克服一切困难继续得到实现。

我们已经开始了这一事业。至于哪一个国家的无产者在什么时候、在什么期间把这一事业进行到底,这个问题并不重要。重要的是,坚冰已经打破,航路已经开通,道路已经指明。

“保卫祖国”即保卫日本反对美国侵略、或保卫美国反对日本侵略、或保卫法国反对英国侵略如此等等的各国资本家先生们,请继续玩弄你们伪善的把戏吧!第二国际和第二半国际的骑士先生们以及全世界所有和平主义的市侩庸人,请继续用新的“巴塞尔宣言”(仿照1912年巴塞尔宣言的式样)来“敷衍”反对帝国主义战争的斗争手段的问题吧!第一次的布尔什维克革命使地球上一亿人首先摆脱了帝国主义战争和帝国主义世界。以后的革命一定会使全人类摆脱这种战争和这个世界。

我们最后的一项事业,也是最重要最困难而又远远没有完成的事业,就是经济建设,就是在破坏了的封建基地和半破坏的资本主义基地上为新的社会主义大厦奠定经济基础。在这一最重要最困难的事业中,我们遭受的失败最多,犯的错误最多。开始这样一个全世界从未有过的事业,难道能没有失败没有错误吗?但是,我们已经开始了这一事业。我们正在进行这一事业。我们现在正用“新经济政策”来纠正我们的许多错误,我们正在学习怎样在一个小农国家里进一步建设社会主义大厦而不犯这些错误。

困难是巨大的。我们已经习惯同巨大的困难作斗争。我们的敌人把我们叫作“硬骨头”和“碰硬政策”的代表不是没有道理的。但是我们也学会了──至少是在一定程度上学会了革命所必需的另一种艺术:灵活机动,善于根据客观条件的变化而迅速急剧地改变自己的策略,如果原先的道路在当前这个时期证明不合适,走不通,就选择另一条道路来达到我们的目的。

我们为热情的浪潮所激励,我们首先激发了人民的一般政治热情,然后又激发了他们的军事热情,我们曾计划依靠这种热情直接实现与一般政治任务和军事任务同样伟大的经济任务。我们计划(说我们计划欠周地设想也许较确切)用无产阶级国家直接下命令的办法在一个小农国家里按共产主义原则来调整国家的产品生产和分配。现实生活说明我们错了。为了作好向共产主义过渡的准备(通过多年的工作来准备),需要经过国家资本主义和社会主义这些过渡阶段。不能直接凭热情,而要借助于伟大革命所产生的热情,靠个人利益,靠同个人利益的结合,靠经济核算,在这个小农国家里先建立起牢固的桥梁,通过国家资本主义走向社会主义;否则你们就不能到达共产主义,否则你们就不能把千百万人引导到共产主义。现实生活就是这样告诉我们的。革命发展的客观进程就是这样告诉我们的。

三四年来我们稍稍学会了实行急剧的转变(在需要急剧转变的时候),现在我们开始勤奋、细心、刻苦地(虽然还不够勤奋,不够细心,不够刻苦)学习实行一种新的转变,学习实行“新经济政策”。无产阶级国家必须成为一个谨慎、勤勉、能干的“业主”,成为一个精明的批发商,否则,就不能使这个小农国家在经济上站稳脚跟。现在,在我们和资本主义的(暂时还是资本主义的)西方并存的条件下,没有其它道路可以过渡到共产主义。批发商这类经济界人物同共产主义似乎有天壤之别。但正是这类矛盾在实际生活中能把人们从小农经济经过国家资本主义引导到社会主义。同个人利益结合,能够提高生产;我们首先需要和绝对需要的是增加生产。批发商业在经济上把千百万小农联合起来,引起他们经营的兴趣,把他们联系起来,把他们引导到更高的阶段:实现生产中各种形式的联系和联合。我们已经开始对经济政策作必要的改变。我们在这方面已经有了某些成就,虽然是不大的、局部的成就,但毕竟是确定无疑的成就。我们就要从这门新“学科”的预备班毕业了。只要坚定地、顽强地学下去,用实际经验来检验我们迈出的每一步,不怕已经开始的工作一改再改,不怕纠正我们的错误,仔细领会这些错误的意义,我们就一定会升到更高的班级。我们一定会修完整个“课程”,尽管世界经济和世界政治的情况使这一课程的学习比我们预计的时间要长得多,困难要多得多。不管过渡时期的苦难如灾荒、饥荒和经济破坏多么深重,我们决不气馁,一定要把我们的事业进行到最后胜利。

\newpage
\chapter{毛泽东}
\newpage

\section{时局估量和红军行动问题 1930/1/5}

〔说明〕此文为《星星之火,可以燎原》的未删减版。原载《抗战以前选集》,1944年由中共中央北方局选编出版,韶山毛泽东图书馆。
录入所据的版本中,“彭德怀式的流动游击政策”被改为“×××式的流动游击政策”,经读者指出,改回。此致感谢!


林彪同志:

新年已经到来几天了,你的信我还没有回答。一则因为有些事情忙着,二则也因为我到底写点什么给你呢?有什么好一点的东西可以贡献给你呢?搜索我的枯肠,没有想出一点什么适当的东西来,因此也就拖延着。现在我想得一点东西了,虽然不知道到底于你的情况切合不切合,但我这点材料实是现今斗争中一个重要的问题,即使于你的个别情况不切合,仍是一般紧要的问题,所以我就把它提出来。

我要提出的是什么问题呢?就是对于时局的估量和伴随而来的我们的行动问题。我从前颇感觉、至今还有些感觉你对于时局的估量是比较的悲观。去年五月十八日晚上瑞金的会议席上,你这个观点最明显。我知道你相信革命高潮不可避免的要到来,但你不相信革命高潮有迅速到来的可能,因此在行动上你不赞成一年争取江西的计划,而只赞成闽粤赣交界三区域的游击;同时在三区域也没有建立赤色政权的深刻的观念,因之也就没有由这种赤色政权的深入与扩大去促进全国革命高潮的深刻的观念。由你相信彭德怀式的流动游击政策一点看来,似乎你认为在距离革命高潮尚远的时期做建立政权的艰苦工作为徒劳,而有用比较轻便的流动游击方式去扩大政治影响,等到全国各地争取群众的工作做好了,或做到某个地步了,然后来一个全国暴动,那时把红军的力量加上去,就成为全国形势的大革命。你的这种全国范围的、包括一切地方的、先争取群众后建立政权的理论,我觉得是于中国革命不适合的。你的这种理论的来源,据我的观察,主要是没有把中国是一个帝国主义最后阶段中互相争夺的半殖民地一件事认清楚。如果认清了中国是一个帝国主义最后阶段中互相争夺的半殖民地,则一,就会明白全个世界里头何以只有中国有这种统治阶级混战的怪事,而且何以混战一天激烈一天,一天扩大一天,何以始终不能有统一的政权。二,就会明白农民问题意义的严重,因之,也就明白农村暴动何以有现在这样的全国形势的发展。三,就会明白工农政权口号之绝对的正确。四,就会明白相应于全个世界中只有中国有统治阶级混战的一件怪事而产生出来的另外一件怪事,即红军与游击队的存在与发展,以及伴随红军与游击队而来的,成长于四围白色政权中的小块红色政权(苏维埃)之存在与发展(中国以外无此怪事)。五,也就会明白红军游击队及苏维埃区域之发展,它是半殖民地农民斗争的最高形式,也就是半殖民地农民斗争必然走向的形式。六,也就会明白无疑义的它(红军与农民苏维埃)是半殖民地无产阶级斗争最重要的同盟力量(无产阶级要走上去领导它),无疑义的它是促进全国革命高潮的重要因素。七,也就会明白单纯的流动游击政策是不能达到促进全国革命高潮的任务,而朱毛式、贺龙式、李文林式、方志敏\footnote{方志敏(一八九九——一九三五),江西弋阳人,赣东北革命根据地和红军第十军的主要创建人。一九二二年加入中国社会主义青年团,一九二四年加入中国共产党,曾被增补为中国共产党第六届中央委员会委员。一九二八年一月,在江西的弋阳、横峰一带发动农民举行武装起义。一九二八年至一九三三年,领导起义的农民坚持游击战争,实行土地革命,建立红色政权,逐步地将农村革命根据地扩大到江西东北部和福建北部、安徽南部、浙江西部,将地方游击队发展为正规红军。一九三四年十一月,带领红军第十军团向皖南进军,继续执行抗日先遣队北上的任务。一九三五年一月,在同国民党军队作战中被捕。同年八月,在南昌英勇牺牲。}式之有根据地的,有计划地建设政权的,红军游击队与广大农民群众紧密地配合着组织着从斗争中训练着的,深入土地革命的,扩大武装组织从乡暴动队、区赤卫大队、县赤卫总队、地方红军以至于超地方红军的,政权发展是波浪式向前扩大的政策,是无疑义地正确的。必须这样,才能树立对全国革命群众的信仰,如苏俄之于全世界然;必须这样,才能给统治阶级以甚大的困难,动摇其基础而促进其内部的分解;也必须这样,才能真正的创造红军,成为将来大革命的重要工具之一。总而言之,必须这样,才能促进革命的高潮。

我现在再要说一说我所感觉得的你对于时局估量比较悲观的原因。你的估量我觉得恰是党内革命急性病派的估量的对面。犯着革命急性病的同志们是看大了主观的力量\footnote{这里所说的“主观力量”,是指有组织的革命力量。},而看小了客观的力量,这种估量多半从唯心观点出发,结果无疑的要走上盲动主义的错误道路。你没有这种错误,但你似乎有另一方面缺点,就是把主观力量看得小一些,把客观力量看得大一些,这亦是一种不切当的估量,又必然要产生另一方面的坏结果。你承认主观力量之弱与客观力量之强,但你似乎没有认识下面的那些要点:

(一)中国革命的主观力量虽弱,但立足于中国脆弱的社会经济组织之上的统治阶级的一切组织(政权、武装、党派、组织等)也是弱的。这样就可以解释西欧各国革命的主观力量虽然比中国革命主观力量要强得多,但因为他们的统治阶级的力量比中国统治的力量更要强大得许多倍,所以他们仍然不能即时爆发革命,中国革命的主观力量虽弱,但因为客观力量也是弱的,所以中国革命之走向高潮一定比西欧快。

(二)大革命失败后革命的主观力量的确大为削弱,剩下的一点小小的主观力量,若据形式上看,自然要使同志们(作这样看法的同志们)发生悲观的念头,但若从实质上看便大大不然。这里用得着中国的一句老话:“星星之火,可以燎原”。即是说现在虽只有一点小小的力量,但它的发展是很快的,它在中国的环境里不仅是具备了发展的可能性,〈简〉直是具备了发展的必然性,这在五卅运动及其后的大革命运动已得了充分的证明。我们看事决然的是要看他的实质,而把它的形式只看作入门的向导,一进了门就要抓住它的实质,而把那做向导的形式抛在一边,这才是科学的可靠的而且含了革命意义的分析方法。

(三)对客观力量的估量亦然,也决然不可只看它的形式,要去看它的实质。当湘赣边界割据的初期,有少数同志在当时湖南省委的不正确估量之下,真正相信湖南省委的话,把阶级敌人看的一钱不值,到现在还传为笑话的“十分动摇”“恐慌万状”两句话,就是那时(前年五月至六月)湖南省委估量湖南的统治者鲁涤平\footnote{鲁涤平(一八八七——一九三五),湖南宁乡人。一九二八年时任国民党湖南省政府主席。}的形容词。在这种估量之下,就必然要产生政治上的盲动主义。但到了前年十一月至去年二月(蒋桂战争\footnote{指一九二九年三四月间蒋介石和广西军阀李宗仁、白崇禧之间的战争。}未爆发前)约四个月间,最大的第三次会剿\footnote{一九二八年七月至十一月,江西、湖南两省的国民党军队两次“会剿”井冈山革命根据地失败后,又于同年底至一九二九年初调集湖南、江西两省共六个旅的兵力,对井冈山革命根据地发动第三次“会剿”。毛泽东等周密地研究了粉碎敌人“会剿”的计划,决定红军第四军主力转入外线打击敌人,以红四军的一部配合红五军留守井冈山。经过内外线的艰苦转战,红军开辟了赣南、闽西革命根据地,曾经被敌人一度侵占的井冈山革命根据地也得到了恢复和发展。}临到了井冈山的时候,一部份同志又有“红旗到底打得多久”的怀疑出来了。其实那时英、美、日在中国的斗争已经走到十分露骨的地步,蒋桂冯的混战的式子业已形成,实质是反革命潮流开始下落,革命潮流开始复兴的时候。但那时不但红军及地方党内有一种悲观的心理,就是中央那时亦不免为那种形式上的客观情况所迷惑,而发出了悲观的论调;二月七日中央来信\footnote{指中共中央一九二九年二月七日给红军第四军前敌委员会的信。本文中引录的一九二九年四月五日红军第四军前敌委员会给中央的信上,曾大略地摘出该信的内容,主要是关于当时形势的估计和红军的行动策略问题。中央的这封信所提出的意见是不适当的,所以前委在给中央的信中提出了不同的意见。}就是代表那时候党内悲观分析的证据。

(四)现时的客观情况,还是容易给只观察形式不观察实质的同志们以迷惑,特别是我们在红军工作的人,一遇到败仗,或四面围困,或强敌跟追的时候,往往不自觉地把这种一时的特殊的小的环境,一般化扩大化起来,仿佛全国全世界的形势概属未可乐观,而革命胜利前途殊属渺茫得很。所以有这种抛弃实质的观察,是原因于他对一般悄况的实质未曾科学地了解到。如问中国革命高潮是否快要到来,只有详细地去察看引起革命高潮的各种矛盾是否向前发展才能决定。如果我们确切认识了国际上帝国主义相互间、帝国主义与殖民地间、帝国主义与无产阶级间的矛盾是发展了,因而帝国主义争夺中国的需要就更迫切:帝国主义争夺一迫切,帝国主义与整个中国的矛盾和帝国主义者相互间的矛盾,就同时在中国境内发展起来,因此就造成中国统治阶级间的一日扩大一日、一日激烈一日的混战——中国统治阶级间的矛盾,就越益发展起来;伴随统治者间的矛盾——军阀混战而来的赋税之无情的加重,就促令广大的负担赋税者与统治者间的矛盾日益发展;伴随帝国主义与中国资本主义的矛盾,即中国资产阶级得不到帝国主义的让步,就即刻发展了中国资产阶级与中国工人阶级之间的矛盾,即中国资产阶级不得不加重对工人阶级的剥削;伴随于帝国主义商品侵略,商业资本剥蚀,与赋税负担加重等,对于地主阶级的矛盾,使地主阶级与农民的矛盾越益深刻化,即地租与利钱的剥削越益加重;为了外货的压迫,工农广大群众消费力的枯竭和政府赋税加重,使国货商人及独立小生产者,日及于破产之途;为了无限制增加军队于粮饷不足的条件之下及战争之日多一日,使得士兵群众天天在饥寒奔走伤亡的惨痛中;为了国家赋税加重,地主租息加重及战祸日广一日,造成了普遍全国的灾荒与匪祸,使广大的农民及城市贫民走到求生不得的道路;因无钱开学,使在学学生有失学之忧;因生产落后,使毕业学生无就业之望;认识了以上这些矛盾,就知道中国是怎样的在一种皇皇不可终日的局面之下,怎样的在一种无政府状态之下,就知道反帝反军阀反地主的革命高潮,是怎样的不可避免而且是很快的要到来。中国是全国都布满了干柴,很迅速的就要燃成烈火;“星火燎原”的话,正是现时局面的适当形容词。只要看一看各地工人罢工、农民暴动、士兵哗变、商人罢市,学生罢课之全国形势的发展,就知道已经不仅是“星星之火”,而距“燎原”的时期,是毫无疑义的不远的了。

上面的话的大意,在去年四月五日前委给中央的信中就已经表明出来了。那封信上说道:

“中央此信(指二月九日来信)对客观形势和主观力量的估量,都太悲观了。三次进剿井冈山表示了反革命的最高潮,然至此为止,往后便是反革命潮流逐渐低落,革命潮流逐渐升涨。党的战斗力组织力虽然弱到如中央所云,但在反革命潮流逐渐低落的形势之下,恢复一定很快,党内干部份子的消极态度也会迅速消灭。群众是一定归向我们的;屠杀主义固然是为渊驱鱼,改良主义也再不能号召群众了。群众对国民党的幻想一定很快的消灭。在将来形势之下,什么党派都不能和共产党争群众的。六次大会\footnote{中国共产党第六次全国代表大会于一九二八年六月十八日至七月十一日在莫斯科举行。会上,瞿秋白作了《中国革命与共产党》的报告,周恩来作了组织问题和军事问题的报告,刘伯承作了军事问题的补充报告。会议通过了政治、苏维埃政权组织、土地、农民等问题决议案和军事工作草案。这次大会肯定了中国社会仍旧是半殖民地半封建社会,中国当时的革命依然是资产阶级民主革命,指出了当时的政治形势是在两个高潮之间和革命发展是不平衡的,党在当时的总任务不是进攻,而是争取群众。会议在批判右倾机会主义的同时,特别指出了当时党内最主要的危险倾向是脱离群众的盲动主义、军事冒险主义和命令主义。这次大会的主要方面是正确的,但也有缺点和错误。它对于中间阶级的两面性和反动势力的内部矛盾缺乏正确的估计和适当的政策;对于大革命失败后党所需要的策略上的有秩序的退却,对于农村根据地的重要性和民主革命的长期性,也缺乏必要的认识。}指示的政治路线和组织路线是十分对的:革命的现时阶段是民权主义而不是社会主义,党的目前任务是争取群众而不是马上武装暴动。但革命的发展是很快的,武装暴动的宣传和准备应该采取积极的精神。在大混乱的现局之下,只有积极口号积极精神才能领导群众,党的战斗力的恢复也一定要在这种积极精神之下才有可能。我们感觉党在从前犯了盲动主义的错误,现在却在一些地方颇有取消主义的倾向了。……无产阶级领导是革命胜利的唯一关键,党的无产阶级基础之建立,中心区域产业支部之创造,是目前党在组织方面的最大任务。但同时农村斗争的发展,小区域苏维埃的建立,红军之创造与扩大,亦是帮助城市斗争,促进革命潮流高涨的条件。所以抛弃城市斗争,沈溺于农村游击主义是最大的错误,但畏惧农民势力发展,以为将超过工人的领导而不利于革命,如果党员中有这种意见,我们以为也是错误的。因为半殖民地中国的革命,只有农民斗争不得工人领导而失败,没有农民斗争发展超过工人势力而不利于革命本身的”。

这封信对红军行动策略问题有如下之答复:

“中央要我们将队伍分得很小,散向农村中,朱、毛离开队伍,隐匿大的目标,目的在保存红军和发动群众。这是一种理想。以连或营为单位,单独行动,分散在农村中,用游击的战术发动群众避免目标,我们从前年冬天(一九二七)就计划起,而且多次实行都是失败的。因为:(一)红军多不是本地人,与地方赤卫队来历不同。(二)分小则领导机关不健全,恶劣环境应付不来容易失败。(三)容易被敌人各个击破。(四)愈是恶劣环境愈须集中,领导者愈须坚决奋斗,方能团结内部应付敌人。只有在好的环境里才好分兵游击,领导者也不如在恶劣环境时之刻不能离。……”

这一段的缺点是:所举不能分兵的理由,都是消极的,这是很不够的。兵力集中的积极理由应该是:集中了才能打破大一点的敌人,才能占领城池。打破了大一点的敌人,占领了城池,才能发动大范围的群众,建立几个县份联在一块的政权。这样才能耸动远近的视听(所谓扩大政治影响),才能于促进革命高潮上发生些实际的效力。如我们前年干的湘赣边界政权,去年干的闽西政权\footnote{指福建西部长汀、龙岩、永定、上杭等县的工农民主政权,它是红军第四军主力一九二九年离开井冈山进入福建时新开辟的革命根据地。},都是这种兵力集中政策的结果。这是大的原则。至于也有分兵的时候没有呢?有的。前委给中央的信上说了红军的游击战术,那里面包括了近距离的分兵。大要如下:

“我们三年来从斗争中所得的战术,真是和古今中外的战术都不同。用我们的战术,群众斗争的发动是一天一天广大的,任何强大的敌人是奈何我们不得的。我们的战术就是游击的战术。大要说来是:‘分兵以发动群众,集中以应付敌人’。‘敌进我退,敌驻我扰,敌疲我打,敌退我追’。‘固定区域的割据。用波浪式的推进政策。强敌跟追,用盘旋式的打圈子政策’。‘很短的时间,很好的方法,发动很大的群众’。这种战术正如打网,要随时打开,又要随时收拢。打开以争取群众,收拢以应付敌人。三年以来都是用的这种战术’。”

这里所谓“打开”,就指近距离的,或如湘赣边界第一次打下永新时,二十九团与三十一团之永新境内的分兵;第三次打下永新时二十八团往安福边境,二十九团往莲花,三十一团往吉安边界的分兵;又如去年四月至五月之赣南各县分兵,七月之闽西各县分兵,都是适例。至于远距离的分兵,则要在好一点的环境和在比较健全的领导机关两个条件之下才有可能。因为分兵的目的,是为了更能争取群众,更能深入土地革命和建立政权,更能扩大红军和地方武装。若不能这到这些目的,甚至反因分兵而遭失败,削弱了红军势力,如前年八月湘赣边界分兵打郴州一样,则不如不分的好。如果具备了上述两个条件,那是无疑地应该分兵的,因为在这两个条件下分兵比集中更有利。至于在严重环境下为保存实力避免目标集中而分兵,此点我在原则上是反对的,前头所引前委给中央的信内业已说明。此外,将来是否有因为经济情况不许可集中而应该分兵工作的时候呢?那也或许会有,但我不能对此下一肯定的断语,因为我们还没有这种情况的具体经验。

中央二月来信的精神是不好的,这封信给了四军党内小部份同志以不良影响,即如你也似乎受了些影响。中央那时还有一个通告谓蒋桂战争不一定会爆发。但从此以后中央的估量和指示,大体说来都完全是对的了。对于那个估量不适当的通告(其实只通告内一部份),中央已发了一个通告去更正。对于红军这一信,虽没有形式的更正,但后来的指示,就完全没有那些悲观的精神了,对红军行动的主张也完全与前委的主张一致了。但中央那个信给一部份同志的不良影响是仍存在的。前委覆中央那个信虽然是与中央来信同时在党内发表了;但对于这一部份同志似乎没有发生很大的影响,因为中央那个信恰合了这一部份同志的脾胃,而中央后头许多对于时局估量的正确指示,或反不为这部份同志所注意,注意了或仍不能把从前的那个印象洗干净。因此,我觉得就在现时仍有对此问题加以解释的必要。

关于一年争取江西的计划,也是去年四月前委向中央提出的,后来又在雩都有一次决定。当时指出的理由见之于给中央信上的,现录如下:

“蒋桂部队在九江一带彼此逼近,大战爆发即在眼前。国民党统治从此瓦解,革命高潮很快的会到来。在这种局面之下来布置工作,我们觉得南方数省中粤湘两省买办地主的军力太大,湖南则更因党的盲动主义政策的错误,党内党外群众几乎尽失。闽赣浙三省则另成一种形势。第一,三省军力最弱。浙江只有蒋伯成〔诚〕\footnote{蒋伯诚,当时任国民党浙江省防军司令。}的少数省防军。福建五部虽有十四团,但郭旅\footnote{指国民党福建省防军暂编第二混成旅旅长郭凤鸣。}已被击破;陈卢两部\footnote{指福建的着匪陈国辉和卢兴邦,他们的部队在一九二六年被国民党政府收编。}均土匪军,战力甚低;陆战队两旅在沿海从前并未打过仗,战力必不大;只有张贞\footnote{张贞,当时任国民党军暂编第一师师长。}比较能打,但据福建省委分析张亦只有两团好的;且现完全是无政府,不统一。江西朱培德、熊式辉两部\footnote{朱培德(一八八九——一九三七),云南盐兴(今禄丰县)人。当时任国民党江西省政府主席。熊式辉(一八九三——一九七四),江西安义人。当时任国民党江西省政府委员、第五师师长。}共有十六团比闽浙军力为强,然比起湖南来就差得多。第二,三省的盲动主义错误比较少。除浙江情况我们不大明了外,江西福建两省党和群众的基础,都比湖南好些。以江西论,赣北之德安、修水、铜鼓尚有相当基础。赣西宁冈、永新、莲花、遂川党和赤卫队的势力是依然仍在的;赣南的希望更是很大,吉安、永新、兴国等县的红军第二第四团有日益发展之势;方志敏的红军并未消灭。这样就造成了向南昌包围的形势。我们建议中央在国民党军阀长期战争期间,我们要和蒋桂两派争取江西,同时兼及闽西、浙西。在三省扩大红军的数量,造成群众的割据,以一年为期完成此计划。此一年中,要在上海、无锡、宁波、杭州、福州,厦门等处建立无产阶级的斗争基础,使能领导浙赣闽三省的农民斗争。江西省委必须健全,南昌、九江、吉安及南浔路的职工基础须努力建立起来。”

上面一年为期争取江西的话,不对的是机械地规定着一年为期。至于争取江西,在我的意识中除开江西的本身条件之外,还包含有全国革命高潮快要到来的意义,因为如果不相信革命高潮快要到来,便决不能得到一年争取江西的结论。那个建议的缺点就是不该机械地说为一年,因此,影响到革命高潮快要到来的所谓“快要”,也不免伴上了一些机械性和急燥性。但你不相信一年争取江西,则是由于你之过高估量客观力量和过低估量主观力量,由此不相信革命高潮之快要到来,由此而得到的结论。至于江西主客观条件是很值得注意的。除主观条件仍如前头所说,没有新的意见增加外,客观条件现在可以明白指出的有三点:一是江西的经济主要是封建残余即地租剥削的经济,商业资产阶级势力较小,而地主的武装在南方各省中又比哪一省都有力。二是江西没有本省的军队,自来都是外省军队前往驻防。外来军队“剿共”“剿匪”,情形不熟,又远非本省军队之关系切身,而往往不很热心。三是距离帝国主义干涉的影响比较小一点,不比广东接近香港,差不多什么都要受英国的支配。我们懂得了这三点,就可以解释为什么江西的农村暴动比哪一省要普遍,红军游击队比哪一省要多了。

我要对你说的话大致已经说完了。扯开了话匣,说的未免太多。但我觉得我们的讨论问题是有益的,我们讨论的这个问题果然正确地解决了,影响到红军的行动实在不小,所以我很高兴的写了这一篇。

末了还有两点须要作个申明。一是所谓革命高潮快要到来的“快要”二字作何解释,这点是许多同志的共同问题。马克思主义者不是算命先生,未来的发展和变化,只应该也只能说出个大的方向,不应该也不能机械地规定时日。但我所说的中国革命高潮快要到来,决不是如有些人所谓“有到来之可能”之完全没有行动意义的,可望而不可即的一种空的东西。它是站在地平线上遥望海中已经看得桅杆尖头了的一支航船,它是立于高山之岭远看东方光芒四射喷薄欲出的一轮朝日,它是燥动于母腹中的快要成熟了的一个婴儿。二是我说你欲用流动游击方式去扩大政治影响,不是说你有单纯军事观点和流寇思想。你显然没有此二者,因为二者完全没有争取群众的观念,你则是主张“放手争取群众”的,你不但主张,而且是在实际做的。我所不赞成你的是指你缺乏建立政权的深刻的观念,因之对于争取群众促进革命高潮的任务,就必然不能如你心头所想的完满地达到。我这封信所要说的主要目的就在这一点。

我的不对的地方,请你指正。

毛泽东于上杭古田

\section{实践论 1937/7}

\centerline{论认识和实践的关系——知和行的关系}

* 在中国共产党内,曾经有一部分教条主义的同志长期拒绝中国革命的经验,否认“马克思主义不是教条而是行动的指南”这个真理,而只生吞活剥马克思主义书籍中的只言片语,去吓唬人们。还有另一部分经验主义的同志长期拘守于自身的片断经验,不了解理论对于革命实践的重要性,看不见革命的全局,虽然也是辛苦地——但却是盲目地在工作。这两类同志的错误思想,特别是教条主义思想,曾经在一九三一年至一九三四年使得中国革命受了极大的损失,而教条主义者却是披着马克思主义的外衣迷惑了广大的同志。毛泽东的《实践论》,是为着用马克思主义的认识论观点去揭露党内的教条主义和经验主义——特别是教条主义这些主观主义的错误而写的。因为重点是揭露看轻实践的教条主义这种主观主义,故题为《实践论》。毛泽东曾以这篇论文的观点在延安的抗日军事政治大学作过讲演。

马克思以前的唯物论,离开人的社会性,离开人的历史发展,去观察认识问题,因此不能了解认识对社会实践的依赖关系,即认识对生产和阶级斗争的依赖关系。

首先,马克思主义者认为人类的生产活动是最基本的实践活动,是决定其它一切活动的东西。人的认识,主要地依赖于物质的生产活动,逐渐地了解自然的现象、自然的性质、自然的规律性、人和自然的关系;而且经过生产活动,也在各种不同程度上逐渐地认识了人和人的一定的相互关系。一切这些知识,离开生产活动是不能得到的。在没有阶级的社会中,每个人以社会一员的资格,同其它社会成员协力,结成一定的生产关系,从事生产活动,以解决人类物质生活问题。在各种阶级的社会中,各阶级的社会成员,则又以各种不同的方式,结成一定的生产关系,从事生产活动,以解决人类物质生活问题。这是人的认识发展的基本来源。

人的社会实践,不限于生产活动一种形式,还有多种其它的形式,阶级斗争,政治生活,科学和艺术的活动,总之社会实际生活的一切领域都是社会的人所参加的。因此,人的认识,在物质生活以外,还从政治生活文化生活中(与物质生活密切联系),在各种不同程度上,知道人和人的各种关系。其中,尤以各种形式的阶级斗争,给予人的认识发展以深刻的影响。在阶级社会中,每一个人都在一定的阶级地位中生活,各种思想无不打上阶级的烙印。

马克思主义者认为人类社会的生产活动,是一步又一步地由低级向高级发展,因此,人们的认识,不论对于自然界方面,对于社会方面,也都是一步又一步地由低级向高级发展,即由浅入深,由片面到更多的方面。在很长的历史时期内,大家对于社会的历史只能限于片面的了解,这一方面是由于剥削阶级的偏见经常歪曲社会的历史,另方面,则由于生产规模的狭小,限制了人们的眼界。人们能够对于社会历史的发展作全面的历史的了解,把对于社会的认识变成了科学,这只是到了伴随巨大生产力——大工业而出现近代无产阶级的时候,这就是马克思主义的科学。

马克思主义者认为,只有人们的社会实践,才是人们对于外界认识的真理性的标准。实际的情形是这样的,只有在社会实践过程中(物质生产过程中,阶级斗争过程中,科学实验过程中),人们达到了思想中所预想的结果时,人们的认识才被证实了。人们要想得到工作的胜利即得到预想的结果,一定要使自己的思想合于客观外界的规律性,如果不合,就会在实践中失败。人们经过失败之后,也就从失败取得教训,改正自己的思想使之适合于外界的规律性,人们就能变失败为胜利,所谓“失败者成功之母”,“吃一堑长一智”,就是这个道理。辩证唯物论的认识论把实践提到第一的地位,认为人的认识一点也不能离开实践,排斥一切否认实践重要性、使认识离开实践的错误理论。列宁这样说过:“实践高于(理论的)认识,因为它不但有普遍性的品格,而且还有直接现实性的品格。”\footnote{见列宁《黑格尔〈逻辑学〉一书摘要》。新的译文是:“实践高于(理论的)认识,因为它不仅具有普遍性的品格,而且还具有直接现实性的品格。”(《列宁全集》第55卷,人民出版社1990年版,第183页)}马克思主义的哲学辩证唯物论有两个最显着的特点:一个是它的阶级性,公然申明辩证唯物论是为无产阶级服务的;再一个是它的实践性,强调理论对于实践的依赖关系,理论的基础是实践,又转过来为实践服务。判定认识或理论之是否真理,不是依主观上觉得如何而定,而是依客观上社会实践的结果如何而定。真理的标准只能是社会的实践。实践的观点是辩证唯物论的认识论之第一的和基本的观点\footnote{参见马克思《关于费尔巴哈的提纲》(《马克思恩格斯选集》第1卷,人民出版社1972年版,第16—19页)和列宁《唯物主义和经验批判主义》第二章第六节(《列宁全集》第18卷,人民出版社1988年版,第144页)。}。

然而人的认识究竟怎样从实践发生,而又服务于实践呢?这只要看一看认识的发展过程就会明了的。

原来人在实践过程中,开始只是看到过程中各个事物的现象方面,看到各个事物的片面,看到各个事物之间的外部联系。例如有些外面的人们到延安来考察,头一二天,他们看到了延安的地形、街道、屋宇,接触了许多的人,参加了宴会、晚会和群众大会,听到了各种说话,看到了各种文件,这些就是事物的现象,事物的各个片面以及这些事物的外部联系。这叫做认识的感性阶段,就是感觉和印象的阶段。也就是延安这些各别的事物作用于考察团先生们的感官,引起了他们的感觉,在他们的脑子中生起了许多的印象,以及这些印象间的大概的外部的联系,这是认识的第一个阶段。在这个阶段中,人们还不能造成深刻的概念,作出合乎论理(即合乎逻辑)的结论。

社会实践的继续,使人们在实践中引起感觉和印象的东西反复了多次,于是在人们的脑子里生起了一个认识过程中的突变(即飞跃),产生了概念。概念这种东西已经不是事物的现象,不是事物的各个片面,不是它们的外部联系,而是抓着了事物的本质,事物的全体,事物的内部联系了。概念同感觉,不但是数量上的差别,而且有了性质上的差别。循此继进,使用判断和推理的方法,就可产生出合乎论理的结论来。《三国演义》上所谓“眉头一皱计上心来”,我们普通说话所谓“让我想一想”,就是人在脑子中运用概念以作判断和推理的工夫。这是认识的第二个阶段。外来的考察团先生们在他们集合了各种材料,加上他们“想了一想”之后,他们就能够作出“共产党的抗日民族统一战线的政策是彻底的、诚恳的和真实的”这样一个判断了。在他们作出这个判断之后,如果他们对于团结救国也是真实的的话,那末他们就能够进一步作出这样的结论:“抗日民族统一战线是能够成功的。”这个概念、判断和推理的阶段,在人们对于一个事物的整个认识过程中是更重要的阶段,也就是理性认识的阶段。认识的真正任务在于经过感觉而到达于思维,到达于逐步了解客观事物的内部矛盾,了解它的规律性,了解这一过程和那一过程间的内部联系,即到达于论理的认识。重复地说,论理的认识所以和感性的认识不同,是因为感性的认识是属于事物之片面的、现象的、外部联系的东西,论理的认识则推进了一大步,到达了事物的全体的、本质的、内部联系的东西,到达了暴露周围世界的内在的矛盾,因而能在周围世界的总体上,在周围世界一切方面的内部联系上去把握周围世界的发展。

这种基于实践的由浅入深的辩证唯物论的关于认识发展过程的理论,在马克思主义以前,是没有一个人这样解决过的。马克思主义的唯物论,第一次正确地解决了这个问题,唯物地而且辩证地指出了认识的深化的运动,指出了社会的人在他们的生产和阶级斗争的复杂的、经常反复的实践中,由感性认识到论理认识的推移的运动。列宁说过:“物质的抽象,自然规律的抽象,价值的抽象以及其它等等,一句话,一切科学的(正确的、郑重的、非瞎说的)抽象,都更深刻、更正确、更完全地反映着自然。”\footnote{见列宁《黑格尔〈逻辑学〉一书摘要》(《列宁全集》第55卷,人民出版社1990年版,第142页)。}马克思列宁主义认为:认识过程中两个阶段的特性,在低级阶段,认识表现为感性的,在高级阶段,认识表现为论理的,但任何阶段,都是统一的认识过程中的阶段。感性和理性二者的性质不同,但又不是互相分离的,它们在实践的基础上统一起来了。我们的实践证明:感觉到了的东西,我们不能立刻理解它,只有理解了的东西才更深刻地感觉它。感觉只解决现象问题,理论才解决本质问题。这些问题的解决,一点也不能离开实践。无论何人要认识什么事物,除了同那个事物接触,即生活于(实践于)那个事物的环境中,是没有法子解决的。不能在封建社会就预先认识资本主义社会的规律,因为资本主义还未出现,还无这种实践。马克思主义只能是资本主义社会的产物。马克思不能在自由资本主义时代就预先具体地认识帝国主义时代的某些特异的规律,因为帝国主义这个资本主义最后阶段还未到来,还无这种实践,只有列宁和斯大林才能担当此项任务。马克思、恩格斯、列宁、斯大林之所以能够作出他们的理论,除了他们的天才条件之外,主要地是他们亲自参加了当时的阶级斗争和科学实验的实践,没有这后一个条件,任何天才也是不能成功的。“秀才不出门,全知天下事”,在技术不发达的古代只是一句空话,在技术发达的现代虽然可以实现这句话,然而真正亲知的是天下实践着的人,那些人在他们的实践中间取得了“知”,经过文字和技术的传达而到达于“秀才”之手,秀才乃能间接地“知天下事”。如果要直接地认识某种或某些事物,便只有亲身参加于变革现实、变革某种或某些事物的实践的斗争中,才能触到那种或那些事物的现象,也只有在亲身参加变革现实的实践的斗争中,才能暴露那种或那些事物的本质而理解它们。这是任何人实际上走着的认识路程,不过有些人故意歪曲地说些反对的话罢了。世上最可笑的是那些“知识里手”\footnote{里手,湖南方言,内行的意思。},有了道听途说的一知半解,便自封为“天下第一”,适足见其不自量而已。知识的问题是一个科学问题,来不得半点的虚伪和骄傲,决定地需要的倒是其反面——诚实和谦逊的态度。你要有知识,你就得参加变革现实的实践。你要知道梨子的滋味,你就得变革梨子,亲口吃一吃。你要知道原子的组织同性质,你就得实行物理学和化学的实验,变革原子的情况。你要知道革命的理论和方法,你就得参加革命。一切真知都是从直接经验发源的。但人不能事事直接经验,事实上多数的知识都是间接经验的东西,这就是一切古代的和外域的知识。这些知识在古人在外人是直接经验的东西,如果在古人外人直接经验时是符合于列宁所说的条件“科学的抽象”,是科学地反映了客观的事物,那末这些知识是可靠的,否则就是不可靠的。所以,一个人的知识,不外直接经验的和间接经验的两部分。而且在我为间接经验者,在人则仍为直接经验。因此,就知识的总体说来,无论何种知识都是不能离开直接经验的。任何知识的来源,在于人的肉体感官对客观外界的感觉,否认了这个感觉,否认了直接经验,否认亲自参加变革现实的实践,他就不是唯物论者。“知识里手”之所以可笑,原因就是在这个地方。中国人有一句老话:“不入虎穴,焉得虎子。”这句话对于人们的实践是真理,对于认识论也是真理。离开实践的认识是不可能的。

为了明了基于变革现实的实践而产生的辩证唯物论的认识运动——认识的逐渐深化的运动,下面再举出几个具体的例子。

无产阶级对于资本主义社会的认识,在其实践的初期——破坏机器和自发斗争时期,他们还只在感性认识的阶段,只认识资本主义各个现象的片面及其外部的联系。这时,他们还是一个所谓“自在的阶级”。但是到了他们实践的第二个时期——有意识有组织的经济斗争和政治斗争的时期,由于实践,由于长期斗争的经验,经过马克思、恩格斯用科学的方法把这种种经验总结起来,产生了马克思主义的理论,用以教育无产阶级,这样就使无产阶级理解了资本主义社会的本质,理解了社会阶级的剥削关系,理解了无产阶级的历史任务,这时他们就变成了一个“自为的阶级”。

中国人民对于帝国主义的认识也是这样。第一阶段是表面的感性的认识阶段,表现在太平天国运动和义和团运动等笼统的排外主义的斗争上\footnote{一九五一年三月二十七日,毛泽东在致李达的信中说:“《实践论》中将太平天国放在排外主义一起说不妥,出选集时拟加修改,此处暂仍照原。”}。第二阶段才进到理性的认识阶段,看出了帝国主义内部和外部的各种矛盾,并看出了帝国主义联合中国买办阶级和封建阶级以压榨中国人民大众的实质,这种认识是从一九一九年五四运动\footnote{五四运动是一九一九年五月四日发生的反帝反封建的爱国运动。当时,第一次世界大战刚刚结束,英、美、法、日、意等战胜国在巴黎召开对德和会,决定由日本继承德国在中国山东的特权。中国是参加对德宣战的战胜国之一,但北洋军阀政府却准备接受这个决定。五月四日,北京学生游行示威,反对帝国主义的这一无理决定和北洋军阀政府的妥协。这次运动迅速地获得了全国人民的响应,到六月三日以后,发展成为有工人阶级、城市小资产阶级和民族资产阶级参加的广大群众性的反帝反封建的爱国运动。五四运动也是反对封建文化的新文化运动。以一九一五年《青年杂志》(后改名《新青年》)创刊为起点的新文化运动,竖起“民主”和“科学”的旗帜,反对旧道德,提倡新道德,反对旧文学,提倡新文学。五四运动中的先进分子接受了马克思主义,使新文化运动发展成为马克思主义思想运动,他们致力于马克思主义同中国工人运动相结合,在思想上和干部上准备了中国共产党的成立。}前后才开始的。

我们再来看战争。战争的领导者,如果他们是一些没有战争经验的人,对于一个具体的战争(例如我们过去十年的土地革命战争)的深刻的指导规律,在开始阶段是不了解的。他们在开始阶段只是身历了许多作战的经验,而且败仗是打得很多的。然而由于这些经验(胜仗,特别是败仗的经验),使他们能够理解贯串整个战争的内部的东西,即那个具体战争的规律性,懂得了战略和战术,因而能够有把握地去指导战争。此时,如果改换一个无经验的人去指导,又会要在吃了一些败仗之后(有了经验之后)才能理会战争的正确的规律。

常常听到一些同志在不能勇敢接受工作任务时说出来的一句话:没有把握。为什么没有把握呢?因为他对于这项工作的内容和环境没有规律性的了解,或者他从来就没有接触过这类工作,或者接触得不多,因而无从谈到这类工作的规律性。及至把工作的情况和环境给以详细分析之后,他就觉得比较地有了把握,愿意去做这项工作。如果这个人在这项工作中经过了一个时期,他有了这项工作的经验了,而他又是一个肯虚心体察情况的人,不是一个主观地、片面地、表面地看问题的人,他就能够自己做出应该怎样进行工作的结论,他的工作勇气也就可以大大地提高了。只有那些主观地、片面地和表面地看问题的人,跑到一个地方,不问环境的情况,不看事情的全体(事情的历史和全部现状),也不触到事情的本质(事情的性质及此一事情和其它事情的内部联系),就自以为是地发号施令起来,这样的人是没有不跌交子的。

由此看来,认识的过程,第一步,是开始接触外界事情,属于感觉的阶段。第二步,是综合感觉的材料加以整理和改造,属于概念、判断和推理的阶段。只有感觉的材料十分丰富(不是零碎不全)和合于实际(不是错觉),才能根据这样的材料造出正确的概念和论理来。

这里有两个要点必须着重指明。第一个,在前面已经说过的,这里再重复说一说,就是理性认识依赖于感性认识的问题。如果以为理性认识可以不从感性认识得来,他就是一个唯心论者。哲学史上有所谓“唯理论”一派,就是只承认理性的实在性,不承认经验的实在性,以为只有理性靠得住,而感觉的经验是靠不住的,这一派的错误在于颠倒了事实。理性的东西所以靠得住,正是由于它来源于感性,否则理性的东西就成了无源之水,无本之木,而只是主观自生的靠不住的东西了。从认识过程的秩序说来,感觉经验是第一的东西,我们强调社会实践在认识过程中的意义,就在于只有社会实践才能使人的认识开始发生,开始从客观外界得到感觉经验。一个闭目塞听、同客观外界根本绝缘的人,是无所谓认识的。认识开始于经验——这就是认识论的唯物论。

第二是认识有待于深化,认识的感性阶段有待于发展到理性阶段——这就是认识论的辩证法\footnote{参见列宁《黑格尔〈逻辑学〉一书摘要》:“要理解,就必须从经验开始理解、研究,从经验上升到一般。”(《列宁全集》第55卷,人民出版社1990年版,第175页)}。如果以为认识可以停顿在低级的感性阶段,以为只有感性认识可靠,而理性认识是靠不住的,这便是重复了历史上的“经验论”的错误。这种理论的错误,在于不知道感觉材料固然是客观外界某些真实性的反映(我这里不来说经验只是所谓内省体验的那种唯心的经验论),但它们仅是片面的和表面的东西,这种反映是不完全的,是没有反映事物本质的。要完全地反映整个的事物,反映事物的本质,反映事物的内部规律性,就必须经过思考作用,将丰富的感觉材料加以去粗取精、去伪存真、由此及彼、由表及里的改造制作工夫,造成概念和理论的系统,就必须从感性认识跃进到理性认识。这种改造过的认识,不是更空虚了更不可靠了的认识,相反,只要是在认识过程中根据于实践基础而科学地改造过的东西,正如列宁所说乃是更深刻、更正确、更完全地反映客观事物的东西。庸俗的事务主义家不是这样,他们尊重经验而看轻理论,因而不能通观客观过程的全体,缺乏明确的方针,没有远大的前途,沾沾自喜于一得之功和一孔之见。这种人如果指导革命,就会引导革命走上碰壁的地步。

理性认识依赖于感性认识,感性认识有待于发展到理性认识,这就是辩证唯物论的认识论。哲学上的“唯理论”和“经验论”都不懂得认识的历史性或辩证性,虽然各有片面的真理(对于唯物的唯理论和经验论而言,非指唯心的唯理论和经验论),但在认识论的全体上则都是错误的。由感性到理性之辩证唯物论的认识运动,对于一个小的认识过程(例如对于一个事物或一件工作的认识)是如此,对于一个大的认识过程(例如对于一个社会或一个革命的认识)也是如此。

然而认识运动至此还没有完结。辩证唯物论的认识运动,如果只到理性认识为止,那末还只说到问题的一半。而且对于马克思主义的哲学说来,还只说到非十分重要的那一半。马克思主义的哲学认为十分重要的问题,不在于懂得了客观世界的规律性,因而能够解释世界,而在于拿了这种对于客观规律性的认识去能动地改造世界。在马克思主义看来,理论是重要的,它的重要性充分地表现在列宁说过的一句话:“没有革命的理论,就不会有革命的运动。”\footnote{见列宁《俄国社会民主党人的任务》(《列宁全集》第2卷,人民出版社1984年版,第443页);并见列宁《怎么办?》第一章第四节(《列宁全集》第6卷,人民出版社1986年版,第23页)。}然而马克思主义看重理论,正是,也仅仅是,因为它能够指导行动。如果有了正确的理论,只是把它空谈一阵,束之高阁,并不实行,那末,这种理论再好也是没有意义的。认识从实践始,经过实践得到了理论的认识,还须再回到实践去。认识的能动作用,不但表现于从感性的认识到理性的认识之能动的飞跃,更重要的还须表现于从理性的认识到革命的实践这一个飞跃。抓着了世界的规律性的认识,必须把它再回到改造世界的实践中去,再用到生产的实践、革命的阶级斗争和民族斗争的实践以及科学实验的实践中去。这就是检验理论和发展理论的过程,是整个认识过程的继续。理论的东西之是否符合于客观真理性这个问题,在前面说的由感性到理性之认识运动中是没有完全解决的,也不能完全解决的。要完全地解决这个问题,只有把理性的认识再回到社会实践中去,应用理论于实践,看它是否能够达到预想的目的。许多自然科学理论之所以被称为真理,不但在于自然科学家们创立这些学说的时候,而且在于为尔后的科学实践所证实的时候。马克思列宁主义之所以被称为真理,也不但在于马克思、恩格斯、列宁、斯大林等人科学地构成这些学说的时候,而且在于为尔后革命的阶级斗争和民族斗争的实践所证实的时候。辩证唯物论之所以为普遍真理,在于经过无论什么人的实践都不能逃出它的范围。人类认识的历史告诉我们,许多理论的真理性是不完全的,经过实践的检验而纠正了它们的不完全性。许多理论是错误的,经过实践的检验而纠正其错误。所谓实践是真理的标准,所谓“生活、实践底观点,应该是认识论底首先的和基本的观点”\footnote{见列宁《唯物主义和经验批判主义》第二章第六节(《列宁全集》第18卷,人民出版社1988年版,第144页)。},理由就在这个地方。斯大林说得好:“理论若不和革命实践联系起来,就会变成无对象的理论,同样,实践若不以革命理论为指南,就会变成盲目的实践。”\footnote{见斯大林《论列宁主义基础》第三部分《理论》。新的译文是:“离开革命实践的理论是空洞的理论,而不以革命理论为指南的实践是盲目的实践。”(《斯大林选集》上卷,人民出版社1979年版,第199—200页)}

说到这里,认识运动就算完成了吗?我们的答复是完成了,又没有完成。社会的人们投身于变革在某一发展阶段内的某一客观过程的实践中(不论是关于变革某一自然过程的实践,或变革某一社会过程的实践),由于客观过程的反映和主观能动性的作用,使得人们的认识由感性的推移到了理性的,造成了大体上相应于该客观过程的法则性的思想、理论、计划或方案,然后再应用这种思想、理论、计划或方案于该同一客观过程的实践,如果能够实现预想的目的,即将预定的思想、理论、计划、方案在该同一过程的实践中变为事实,或者大体上变为事实,那末,对于这一具体过程的认识运动算是完成了。例如,在变革自然的过程中,某一工程计划的实现,某一科学假想的证实,某一器物的制成,某一农产的收获,在变革社会过程中某一罢工的胜利,某一战争的胜利,某一教育计划的实现,都算实现了预想的目的。然而一般地说来,不论在变革自然或变革社会的实践中,人们原定的思想、理论、计划、方案,毫无改变地实现出来的事,是很少的。这是因为从事变革现实的人们,常常受着许多的限制,不但常常受着科学条件和技术条件的限制,而且也受着客观过程的发展及其表现程度的限制(客观过程的方面及本质尚未充分暴露)。在这种情形之下,由于实践中发现前所未料的情况,因而部分地改变思想、理论、计划、方案的事是常有的,全部地改变的事也是有的。即是说,原定的思想、理论、计划、方案,部分地或全部地不合于实际,部分错了或全部错了的事,都是有的。许多时候须反复失败过多次,才能纠正错误的认识,才能到达于和客观过程的规律性相符合,因而才能够变主观的东西为客观的东西,即在实践中得到预想的结果。但是不管怎样,到了这种时候,人们对于在某一发展阶段内的某一客观过程的认识运动,算是完成了。

然而对于过程的推移而言,人们的认识运动是没有完成的。任何过程,不论是属于自然界的和属于社会的,由于内部的矛盾和斗争,都是向前推移向前发展的,人们的认识运动也应跟着推移和发展。依社会运动来说,真正的革命的指导者,不但在于当自己的思想、理论、计划、方案有错误时须得善于改正,如同上面已经说到的,而且在于当某一客观过程已经从某一发展阶段向另一发展阶段推移转变的时候,须得善于使自己和参加革命的一切人员在主观认识上也跟着推移转变,即是要使新的革命任务和新的工作方案的提出,适合于新的情况的变化。革命时期情况的变化是很急速的,如果革命党人的认识不能随之而急速变化,就不能引导革命走向胜利。

然而思想落后于实际的事是常有的,这是因为人的认识受了许多社会条件的限制的缘故。我们反对革命队伍中的顽固派,他们的思想不能随变化了的客观情况而前进,在历史上表现为右倾机会主义。这些人看不出矛盾的斗争已将客观过程推向前进了,而他们的认识仍然停止在旧阶段。一切顽固党的思想都有这样的特征。他们的思想离开了社会的实践,他们不能站在社会车轮的前头充任向导的工作,他们只知跟在车子后面怨恨车子走得太快了,企图把它向后拉,开倒车。

我们也反对“左”翼空谈主义。他们的思想超过客观过程的一定发展阶段,有些把幻想看作真理,有些则把仅在将来有现实可能性的理想,勉强地放在现时来做,离开了当前大多数人的实践,离开了当前的现实性,在行动上表现为冒险主义。

唯心论和机械唯物论,机会主义和冒险主义,都是以主观和客观相分裂,以认识和实践相脱离为特征的。以科学的社会实践为特征的马克思列宁主义的认识论,不能不坚决反对这些错误思想。马克思主义者承认,在绝对的总的宇宙发展过程中,各个具体过程的发展都是相对的,因而在绝对真理的长河中,人们对于在各个一定发展阶段上的具体过程的认识只具有相对的真理性。无数相对的真理之总和,就是绝对的真理\footnote{参见列宁《唯物主义和经验批判主义》第二章第五节。原文是:“人类思维按其本性是能够给我们提供并且正在提供由相对真理的总和所构成的绝对真理的。”(《列宁全集》第18卷,人民出版社1988年版,第135页)}。客观过程的发展是充满着矛盾和斗争的发展,人的认识运动的发展也是充满着矛盾和斗争的发展。一切客观世界的辩证法的运动,都或先或后地能够反映到人的认识中来。社会实践中的发生、发展和消灭的过程是无穷的,人的认识的发生、发展和消灭的过程也是无穷的。根据于一定的思想、理论、计划、方案以从事于变革客观现实的实践,一次又一次地向前,人们对于客观现实的认识也就一次又一次地深化。客观现实世界的变化运动永远没有完结,人们在实践中对于真理的认识也就永远没有完结。马克思列宁主义并没有结束真理,而是在实践中不断地开辟认识真理的道路。我们的结论是主观和客观、理论和实践、知和行的具体的历史的统一,反对一切离开具体历史的“左”的或右的错误思想。

社会的发展到了今天的时代,正确地认识世界和改造世界的责任,已经历史地落在无产阶级及其政党的肩上。这种根据科学认识而定下来的改造世界的实践过程,在世界、在中国均已到达了一个历史的时节——自有历史以来未曾有过的重大时节,这就是整个儿地推翻世界和中国的黑暗面,把它们转变过来成为前所未有的光明世界。无产阶级和革命人民改造世界的斗争,包括实现下述的任务:改造客观世界,也改造自己的主观世界——改造自己的认识能力,改造主观世界同客观世界的关系。地球上已经有一部分实行了这种改造,这就是苏联。他们还正在促进这种改造过程。中国人民和世界人民也都正在或将要通过这样的改造过程。所谓被改造的客观世界,其中包括了一切反对改造的人们,他们的被改造,须要通过强迫的阶段,然后才能进入自觉的阶段。世界到了全人类都自觉地改造自己和改造世界的时候,那就是世界的共产主义时代。

通过实践而发现真理,又通过实践而证实真理和发展真理。从感性认识而能动地发展到理性认识,又从理性认识而能动地指导革命实践,改造主观世界和客观世界。实践、认识、再实践、再认识,这种形式,循环往复以至无穷,而实践和认识之每一循环的内容,都比较地进到了高一级的程度。这就是辩证唯物论的全部认识论,这就是辩证唯物论的知行统一观。

\section{矛盾论 1937/8}

* 这篇哲学论文,是毛泽东继《实践论》之后,为了同一的目的,即为了克服存在于中国共产党内的严重的教条主义思想而写的,曾在延安的抗日军事政治大学作过讲演。在收入本书第一版的时候,作者作了部分的补充、删节和修改。

事物的矛盾法则,即对立统一的法则,是唯物辩证法的最根本的法则。列宁说:“就本来的意义讲,辩证法是研究对象的本质自身中的矛盾。”\footnote{见列宁《黑格尔〈哲学史讲演录〉一书摘要》(《列宁全集》第55卷,人民出版社1990年版,第213页)。}列宁常称这个法则为辩证法的本质,又称之为辩证法的核心\footnote{参见列宁《谈谈辩证法问题》:“统一物之分为两个部分以及对它的矛盾着的部分的认识……,是辩证法的实质(是辩证法的‘本质’之一,是它的基本的特点或特征之一,甚至可说是它的最基本的特点或特征)。”并参见《黑格尔〈逻辑学〉一书摘要》中关于“辩证法的要素”部分:“可以把辩证法简要地规定为关于对立面的统一的学说。这样就会抓住辩证法的核心,可是这需要说明和发挥。”(《列宁全集》第55卷,人民出版社1990年版,第305、192页)}。因此,我们在研究这个法则时,不得不涉及广泛的方面,不得不涉及许多的哲学问题。如果我们将这些问题都弄清楚了,我们就在根本上懂得了唯物辩证法。这些问题是:两种宇宙观;矛盾的普遍性;矛盾的特殊性;主要的矛盾和主要的矛盾方面;矛盾诸方面的同一性和斗争性;对抗在矛盾中的地位。

苏联哲学界在最近数年中批判了德波林学派\footnote{德波林(一八八一——一九六三),苏联哲学家。一九二九年当选为苏联科学院院士。三十年代初,苏联哲学界发动对德波林学派的批判,认为他们犯了理论脱离实践、哲学脱离政治等唯心主义性质的错误。}的唯心论,这件事引起了我们的极大的兴趣。德波林的唯心论在中国共产党内发生了极坏的影响,我们党内的教条主义思想不能说和这个学派的作风没有关系。因此,我们现在的哲学研究工作,应当以扫除教条主义思想为主要的目标。

一、两种宇宙观

在人类的认识史中,从来就有关于宇宙发展法则的两种见解,一种是形而上学的见解,一种是辩证法的见解,形成了互相对立的两种宇宙观。列宁说:“对于发展(进化)所持的两种基本的(或两种可能的?或两种在历史上常见的?)观点是:(一)认为发展是减少和增加,是重复;(二)认为发展是对立的统一(统一物分成为两个互相排斥的对立,而两个对立又互相关联着)。”\footnote{见列宁《谈谈辩证法问题》。新的译文是:“有两种基本的(或两种可能的?或两种在历史上常见的?)发展(进化)观点:认为发展是减少和增加,是重复;以及认为发展是对立面的统一(统一物之分为两个互相排斥的对立面以及它们之间的相互关系)。”(《列宁全集》第55卷,人民出版社1990年版,第306页)}列宁说的就是这两种不同的宇宙观。

形而上学,亦称玄学。这种思想,无论在中国,在欧洲,在一个很长的历史时间内,是属于唯心论的宇宙观,并在人们的思想中占了统治的地位。在欧洲,资产阶级初期的唯物论,也是形而上学的。由于欧洲许多国家的社会经济情况进到了资本主义高度发展的阶段,生产力、阶级斗争和科学均发展到了历史上未有过的水平,工业无产阶级成为历史发展的最伟大的动力,因而产生了马克思主义的唯物辩证法的宇宙观。于是,在资产阶级那里,除了公开的极端露骨的反动的唯心论之外,还出现了庸俗的进化论,出来对抗唯物辩证法。

所谓形而上学的或庸俗进化论的宇宙观,就是用孤立的、静止的和片面的观点去看世界。这种宇宙观把世界一切事物,一切事物的形态和种类,都看成是永远彼此孤立和永远不变化的。如果说有变化,也只是数量的增减和场所的变更。而这种增减和变更的原因,不在事物的内部而在事物的外部,即是由于外力的推动。形而上学家认为,世界上各种不同事物和事物的特性,从它们一开始存在的时候就是如此。后来的变化,不过是数量上的扩大或缩小。他们认为一种事物永远只能反复地产生为同样的事物,而不能变化为另一种不同的事物。在形而上学家看来,资本主义的剥削,资本主义的竞争,资本主义社会的个人主义思想等,就是在古代的奴隶社会里,甚至在原始社会里,都可以找得出来,而且会要永远不变地存在下去。说到社会发展的原因,他们就用社会外部的地理、气候等条件去说明。他们简单地从事物外部去找发展的原因,否认唯物辩证法所主张的事物因内部矛盾引起发展的学说。因此,他们不能解释事物的质的多样性,不能解释一种质变为他种质的现象。这种思想,在欧洲,在十七世纪和十八世纪是机械唯物论,在十九世纪末和二十世纪初则有庸俗进化论。在中国,则有所谓“天不变,道亦不变”\footnote{见《汉书•董仲舒传》。董仲舒(公元前一七九——前一○四)是孔子学派在西汉的主要代表,他曾经对汉武帝说:“道之大原出于天,天不变,道亦不变。”“道”是中国古代哲学家的通用语,它的意义是“道路”或“道理”,可作“法则”或“规律”解说。}的形而上学的思想,曾经长期地为腐朽了的封建统治阶级所拥护。近百年来输入了欧洲的机械唯物论和庸俗进化论,则为资产阶级所拥护。

和形而上学的宇宙观相反,唯物辩证法的宇宙观主张从事物的内部、从一事物对他事物的关系去研究事物的发展,即把事物的发展看做是事物内部的必然的自己的运动,而每一事物的运动都和它的周围其它事物互相联系着和互相影响着。事物发展的根本原因,不是在事物的外部而是在事物的内部,在于事物内部的矛盾性。任何事物内部都有这种矛盾性,因此引起了事物的运动和发展。事物内部的这种矛盾性是事物发展的根本原因,一事物和他事物的互相联系和互相影响则是事物发展的第二位的原因。这样,唯物辩证法就有力地反对了形而上学的机械唯物论和庸俗进化论的外因论或被动论。这是清楚的,单纯的外部原因只能引起事物的机械的运动,即范围的大小,数量的增减,不能说明事物何以有性质上的千差万别及其互相变化。事实上,即使是外力推动的机械运动,也要通过事物内部的矛盾性。植物和动物的单纯的增长,数量的发展,主要地也是由于内部矛盾所引起的。同样,社会的发展,主要地不是由于外因而是由于内因。许多国家在差不多一样的地理和气候的条件下,它们发展的差异性和不平衡性,非常之大。同一个国家吧,在地理和气候并没有变化的情形下,社会的变化却是很大的。帝国主义的俄国变为社会主义的苏联,封建的闭关锁国的日本变为帝国主义的日本,这些国家的地理和气候并没有变化。长期地被封建制度统治的中国,近百年来发生了很大的变化,现在正在变化到一个自由解放的新中国的方向去,中国的地理和气候并没有变化。整个地球及地球各部分的地理和气候也是变化着的,但以它们的变化和社会的变化相比较,则显得很微小,前者是以若干万年为单位而显现其变化的,后者则在几千年、几百年、几十年、甚至几年或几个月(在革命时期)内就显现其变化了。按照唯物辩证法的观点,自然界的变化,主要地是由于自然界内部矛盾的发展。社会的变化,主要地是由于社会内部矛盾的发展,即生产力和生产关系的矛盾,阶级之间的矛盾,新旧之间的矛盾,由于这些矛盾的发展,推动了社会的前进,推动了新旧社会的代谢。唯物辩证法是否排除外部的原因呢?并不排除。唯物辩证法认为外因是变化的条件,内因是变化的根据,外因通过内因而起作用。鸡蛋因得适当的温度而变化为鸡子,但温度不能使石头变为鸡子,因为二者的根据是不同的。各国人民之间的互相影响是时常存在的。在资本主义时代,特别是在帝国主义和无产阶级革命的时代,各国在政治上、经济上和文化上的互相影响和互相激动,是极其巨大的。十月社会主义革命不只是开创了俄国历史的新纪元,而且开创了世界历史的新纪元,影响到世界各国内部的变化,同样地而且还特别深刻地影响到中国内部的变化,但是这种变化是通过了各国内部和中国内部自己的规律性而起的。两军相争,一胜一败,所以胜败,皆决于内因。胜者或因其强,或因其指挥无误,败者或因其弱,或因其指挥失宜,外因通过内因而引起作用。一九二七年中国大资产阶级战败了无产阶级,是通过中国无产阶级内部的(中国共产党内部的)机会主义而起作用的。当着我们清算了这种机会主义的时候,中国革命就重新发展了。后来,中国革命又受了敌人的严重的打击,是因为我们党内产生了冒险主义。当着我们清算了这种冒险主义的时候,我们的事业就又重新发展了。由此看来,一个政党要引导革命到胜利,必须依靠自己政治路线的正确和组织上的巩固。

辩证法的宇宙观,不论在中国,在欧洲,在古代就产生了。但是古代的辩证法带着自发的朴素的性质,根据当时的社会历史条件,还不可能有完备的理论,因而不能完全解释宇宙,后来就被形而上学所代替。生活在十八世纪末和十九世纪初期的德国著名哲学家黑格尔,对于辩证法曾经给了很重要的贡献,但是他的辩证法却是唯心的辩证法。直到无产阶级运动的伟大的活动家马克思和恩格斯综合了人类认识史的积极的成果,特别是批判地吸取了黑格尔的辩证法的合理的部分,创造了辩证唯物论和历史唯物论这个伟大的理论,才在人类认识史上起了一个空前的大革命。后来,经过列宁和斯大林,又发展了这个伟大的理论。这个理论一经传到中国来,就在中国思想界引起了极大的变化。

这个辩证法的宇宙观,主要地就是教导人们要善于去观察和分析各种事物的矛盾的运动,并根据这种分析,指出解决矛盾的方法。因此,具体地了解事物矛盾这一个法则,对于我们是非常重要的。

二、矛盾的普遍性

为了叙述的便利起见,我在这里先说矛盾的普遍性,再说矛盾的特殊性。这是因为马克思主义的伟大的创造者和继承者马克思、恩格斯、列宁、斯大林他们发现了唯物辩证法的宇宙观,已经把唯物辩证法应用在人类历史的分析和自然历史的分析的许多方面,应用在社会的变革和自然的变革(例如在苏联)的许多方面,获得了极其伟大的成功,矛盾的普遍性已经被很多人所承认,因此,关于这个问题只需要很少的话就可以说明白;而关于矛盾的特殊性的问题,则还有很多的同志,特别是教条主义者,弄不清楚。他们不了解矛盾的普遍性即寓于矛盾的特殊性之中。他们也不了解研究当前具体事物的矛盾的特殊性,对于我们指导革命实践的发展有何等重要的意义。因此,关于矛盾的特殊性的问题应当着重地加以研究,并用足够的篇幅加以说明。为了这个缘故,当着我们分析事物矛盾的法则的时候,我们就先来分析矛盾的普遍性的问题,然后再着重地分析矛盾的特殊性的问题,最后仍归到矛盾的普遍性的问题。

矛盾的普遍性或绝对性这个问题有两方面的意义。其一是说,矛盾存在于一切事物的发展过程中;其二是说,每一事物的发展过程中存在着自始至终的矛盾运动。

恩格斯说:“运动本身就是矛盾。”\footnote{见恩格斯《反杜林论》第一编第十二节《辩证法。量和质》(《马克思恩格斯选集》第3卷,人民出版社1972年版,第160页)。}列宁对于对立统一法则所下的定义,说它就是“承认(发现)自然界(精神和社会两者也在内)的一切现象和过程都含有互相矛盾、互相排斥、互相对立的趋向”\footnote{见列宁《谈谈辩证法问题》。新的译文是:“承认(发现)自然界的(也包括精神的和社会的)一切现象和过程具有矛盾着的、相互排斥的、对立的倾向。”(《列宁全集》第55卷,人民出版社1990年版,第306页)}。这些意见是对的吗?是对的。一切事物中包含的矛盾方面的相互依赖和相互斗争,决定一切事物的生命,推动一切事物的发展。没有什么事物是不包含矛盾的,没有矛盾就没有世界。

矛盾是简单的运动形式(例如机械性的运动)的基础,更是复杂的运动形式的基础。

恩格斯这样说明过矛盾的普遍性:“如果简单的机械的移动本身包含着矛盾,那末,物质的更高的运动形式,特别是有机生命及其发展,就更加包含着矛盾。……生命首先就在于:生物在每一个瞬间是它自身,但却又是别的什么。所以,生命也是存在于物体和过程本身中的不断地自行产生并自行解决的矛盾;这一矛盾一停止,生命亦即停止,于是死就来到。同样,我们看到了,在思维的范围以内我们也不能避免矛盾,并且我们看到了,例如,人的内部无限的认识能力与此种认识能力仅在外部被局限的而且认识上也被局限的个别人们身上的实际的实现二者之间的矛盾,是在人类世代的无穷的——至少对于我们,实际上是无穷的——连续系列之中,是在无穷的前进运动之中解决的。”

“高等数学的主要基础之一,就是矛盾……”

“就是初等数学,也充满着矛盾。……”\footnote{以上所引恩格斯的三段话,均见恩格斯《反杜林论》第一编第十二节《辩证法。量和质》。其中第二段“高等数学的主要基础之一,就是矛盾……”,《反杜林论》中的原文是:“我们已经提到,高等数学的主要基础之一是这样一个矛盾:在一定条件下直线和曲线应当是一回事。高等数学还有另一个矛盾:在我们眼前相交的线,只要离开交点五六厘米,就应当认为是平行的、即使无限延长也不会相交的线。可是,高等数学利用这些和其它一些更加尖锐的矛盾获得了不仅是正确的、而且是初等数学所完全不能达到的成果。”(《马克思恩格斯选集》第3卷,人民出版社1972年版,第160—161页)}

列宁也这样说明过矛盾的普遍性:“在数学中,正和负,微分和积分。

在力学中,作用和反作用。

在物理学中,阳电和阴电。

在化学中,原子的化合和分解。

在社会科学中,阶级斗争。”\footnote{见列宁《谈谈辩证法问题》(《列宁全集》第55卷,人民出版社1990年版,第305—306页)。}

战争中的攻守,进退,胜败,都是矛盾着的现象。失去一方,他方就不存在。双方斗争而又联结,组成了战争的总体,推动了战争的发展,解决了战争的问题。

人的概念的每一差异,都应把它看作是客观矛盾的反映。客观矛盾反映入主观的思想,组成了概念的矛盾运动,推动了思想的发展,不断地解决了人们的思想问题。

党内不同思想的对立和斗争是经常发生的,这是社会的阶级矛盾和新旧事物的矛盾在党内的反映。党内如果没有矛盾和解决矛盾的思想斗争,党的生命也就停止了。

由此看来,不论是简单的运动形式,或复杂的运动形式,不论是客观现象,或思想现象,矛盾是普遍地存在着,矛盾存在于一切过程中,这一点已经弄清楚了。但是每一过程的开始阶段,是否也有矛盾存在呢?是否每一事物的发展过程具有自始至终的矛盾运动呢?

从苏联哲学界批判德波林学派的文章中看出,德波林学派有这样一种见解,他们认为矛盾不是一开始就在过程中出现,须待过程发展到一定的阶段才出现。那末,在那一时间以前,过程发展的原因不是由于内部的原因,而是由于外部的原因了。这样,德波林回到形而上学的外因论和机械论去了。拿这种见解去分析具体的问题,他们就看见在苏联条件下富农和一般农民之间只有差异,并无矛盾,完全同意了布哈林的意见。在分析法国革命时,他们就认为在革命前,工农资产阶级合组的第三等级中,也只有差异,并无矛盾。德波林学派这类见解是反马克思主义的。他们不知道世界上的每一差异中就已经包含着矛盾,差异就是矛盾。劳资之间,从两阶级发生的时候起,就是互相矛盾的,仅仅还没有激化而已。工农之间,即使在苏联的社会条件下,也有差异,它们的差异就是矛盾,仅仅不会激化成为对抗,不取阶级斗争的形态,不同于劳资间的矛盾;它们在社会主义建设中形成巩固的联盟,并在由社会主义走向共产主义的发展过程中逐渐地解决这个矛盾。这是矛盾的差别性的问题,不是矛盾的有无的问题。矛盾是普遍的、绝对的,存在于事物发展的一切过程中,又贯串于一切过程的始终。

新过程的发生是什么呢?这是旧的统一和组成此统一的对立成分让位于新的统一和组成此统一的对立成分,于是新过程就代替旧过程而发生。旧过程完结了,新过程发生了。新过程又包含着新矛盾,开始它自己的矛盾发展史。

事物发展过程的自始至终的矛盾运动,列宁指出马克思在《资本论》中模范地作了这样的分析。这是研究任何事物发展过程所必须应用的方法。列宁自己也正确地应用了它,贯彻于他的全部著作中。

“马克思在《资本论》中,首先分析的是资产阶级社会(商品社会)里最简单的、最普通的、最基本的、最常见的、最平常的、碰到亿万次的关系——商品交换。这一分析在这个最简单的现象之中(资产阶级社会的这个‘细胞’之中)暴露了现代社会的一切矛盾(以及一切矛盾的胚芽)。往后的叙述又向我们表明了这些矛盾和这个社会各个部分总和的自始至终的发展(增长与运动两者)。”

列宁说了上面的话之后,接着说道:“这应该是一般辩证法的……叙述(以及研究)方法。”\footnote{见列宁《谈谈辩证法问题》(《列宁全集》第55卷,人民出版社1990年版,第307页)。}

中国共产党人必须学会这个方法,才能正确地分析中国革命的历史和现状,并推断革命的将来。

三、矛盾的特殊性

矛盾存在于一切事物发展的过程中,矛盾贯串于每一事物发展过程的始终,这是矛盾的普遍性和绝对性,前面已经说过了。现在来说矛盾的特殊性和相对性。

这个问题,应从几种情形中去研究。

首先是各种物质运动形式中的矛盾,都带特殊性。人的认识物质,就是认识物质的运动形式,因为除了运动的物质以外,世界上什么也没有,而物质的运动则必取一定的形式。对于物质的每一种运动形式,必须注意它和其它各种运动形式的共同点。但是,尤其重要的,成为我们认识事物的基础的东西,则是必须注意它的特殊点,就是说,注意它和其它运动形式的质的区别。只有注意了这一点,才有可能区别事物。任何运动形式,其内部都包含着本身特殊的矛盾。这种特殊的矛盾,就构成一事物区别于他事物的特殊的本质。这就是世界上诸种事物所以有千差万别的内在的原因,或者叫做根据。自然界存在着许多的运动形式,机械运动、发声、发光、发热、电流、化分、化合等等都是。所有这些物质的运动形式,都是互相依存的,又是本质上互相区别的。每一物质的运动形式所具有的特殊的本质,为它自己的特殊的矛盾所规定。这种情形,不但在自然界中存在着,在社会现象和思想现象中也是同样地存在着。每一种社会形式和思想形式,都有它的特殊的矛盾和特殊的本质。

科学研究的区分,就是根据科学对象所具有的特殊的矛盾性。因此,对于某一现象的领域所特有的某一种矛盾的研究,就构成某一门科学的对象。例如,数学中的正数和负数,机械学中的作用和反作用,物理学中的阴电和阳电,化学中的化分和化合,社会科学中的生产力和生产关系、阶级和阶级的互相斗争,军事学中的攻击和防御,哲学中的唯心论和唯物论、形而上学观和辩证法观等等,都是因为具有特殊的矛盾和特殊的本质,才构成了不同的科学研究的对象。固然,如果不认识矛盾的普遍性,就无从发现事物运动发展的普遍的原因或普遍的根据;但是,如果不研究矛盾的特殊性,就无从确定一事物不同于他事物的特殊的本质,就无从发现事物运动发展的特殊的原因,或特殊的根据,也就无从辨别事物,无从区分科学研究的领域。

就人类认识运动的秩序说来,总是由认识个别的和特殊的事物,逐步地扩大到认识一般的事物。人们总是首先认识了许多不同事物的特殊的本质,然后才有可能更进一步地进行概括工作,认识诸种事物的共同的本质。当着人们已经认识了这种共同的本质以后,就以这种共同的认识为指导,继续地向着尚未研究过的或者尚未深入地研究过的各种具体的事物进行研究,找出其特殊的本质,这样才可以补充、丰富和发展这种共同的本质的认识,而使这种共同的本质的认识不致变成枯槁的和僵死的东西。这是两个认识的过程:一个是由特殊到一般,一个是由一般到特殊。人类的认识总是这样循环往复地进行的,而每一次的循环(只要是严格地按照科学的方法)都可能使人类的认识提高一步,使人类的认识不断地深化。我们的教条主义者在这个问题上的错误,就是,一方面,不懂得必须研究矛盾的特殊性,认识各别事物的特殊的本质,才有可能充分地认识矛盾的普遍性,充分地认识诸种事物的共同的本质;另一方面,不懂得在我们认识了事物的共同的本质以后,还必须继续研究那些尚未深入地研究过的或者新冒出来的具体的事物。我们的教条主义者是懒汉,他们拒绝对于具体事物做任何艰苦的研究工作,他们把一般真理看成是凭空出现的东西,把它变成为人们所不能够捉摸的纯粹抽象的公式,完全否认了并且颠倒了这个人类认识真理的正常秩序。他们也不懂得人类认识的两个过程的互相联结——由特殊到一般,又由一般到特殊,他们完全不懂得马克思主义的认识论。

不但要研究每一个大系统的物质运动形式的特殊的矛盾性及其所规定的本质,而且要研究每一个物质运动形式在其发展长途中的每一个过程的特殊的矛盾及其本质。一切运动形式的每一个实在的非臆造的发展过程内,都是不同质的。我们的研究工作必须着重这一点,而且必须从这一点开始。

不同质的矛盾,只有用不同质的方法才能解决。例如,无产阶级和资产阶级的矛盾,用社会主义革命的方法去解决;人民大众和封建制度的矛盾,用民主革命的方法去解决;殖民地和帝国主义的矛盾,用民族革命战争的方法去解决;在社会主义社会中工人阶级和农民阶级的矛盾,用农业集体化和农业机械化的方法去解决;共产党内的矛盾,用批评和自我批评的方法去解决;社会和自然的矛盾,用发展生产力的方法去解决。过程变化,旧过程和旧矛盾消灭,新过程和新矛盾发生,解决矛盾的方法也因之而不同。俄国的二月革命和十月革命所解决的矛盾及其所用以解决矛盾的方法是根本上不相同的。用不同的方法去解决不同的矛盾,这是马克思列宁主义者必须严格地遵守的一个原则。教条主义者不遵守这个原则,他们不了解诸种革命情况的区别,因而也不了解应当用不同的方法去解决不同的矛盾,而只是千篇一律地使用一种自以为不可改变的公式到处硬套,这就只能使革命遭受挫折,或者将本来做得好的事情弄得很坏。

为要暴露事物发展过程中的矛盾在其总体上、在其相互联结上的特殊性,就是说暴露事物发展过程的本质,就必须暴露过程中矛盾各方面的特殊性,否则暴露过程的本质成为不可能,这也是我们作研究工作时必须十分注意的。

一个大的事物,在其发展过程中,包含着许多的矛盾。例如,在中国资产阶级民主革命过程中,有中国社会各被压迫阶级和帝国主义的矛盾,有人民大众和封建制度的矛盾,有无产阶级和资产阶级的矛盾,有农民及城市小资产阶级和资产阶级的矛盾,有各个反动的统治集团之间的矛盾等等,情形是非常复杂的。这些矛盾,不但各各有其特殊性,不能一律看待,而且每一矛盾的两方面,又各各有其特点,也是不能一律看待的。我们从事中国革命的人,不但要在各个矛盾的总体上,即矛盾的相互联结上,了解其特殊性,而且只有从矛盾的各个方面着手研究,才有可能了解其总体。所谓了解矛盾的各个方面,就是了解它们每一方面各占何等特定的地位,各用何种具体形式和对方发生互相依存又互相矛盾的关系,在互相依存又互相矛盾中,以及依存破裂后,又各用何种具体的方法和对方作斗争。研究这些问题,是十分重要的事情。列宁说:马克思主义的最本质的东西,马克思主义的活的灵魂,就在于具体地分析具体的情况。就是说的这个意思。我们的教条主义者违背列宁的指示,从来不用脑筋具体地分析任何事物,做起文章或演说来,总是空洞无物的八股调,在我们党内造成了一种极坏的作风。

研究问题,忌带主观性、片面性和表面性。所谓主观性,就是不知道客观地看问题,也就是不知道用唯物的观点去看问题。这一点,我在《实践论》一文中已经说过了。所谓片面性,就是不知道全面地看问题。例如:只了解中国一方、不了解日本一方,只了解共产党一方、不了解国民党一方,只了解无产阶级一方、不了解资产阶级一方,只了解农民一方、不了解地主一方,只了解顺利情形一方、不了解困难情形一方,只了解过去一方、不了解将来一方,只了解个体一方、不了解总体一方,只了解缺点一方、不了解成绩一方,只了解原告一方、不了解被告一方,只了解革命的秘密工作一方、不了解革命的公开工作一方,如此等等。一句话,不了解矛盾各方的特点。这就叫做片面地看问题。或者叫做只看见局部,不看见全体,只看见树木,不看见森林。这样,是不能找出解决矛盾的方法的,是不能完成革命任务的,是不能做好所任工作的,是不能正确地发展党内的思想斗争的。孙子论军事说:“知彼知己,百战不殆。”\footnote{见《孙子·谋攻》。}他说的是作战的双方。唐朝人魏征说过:“兼听则明,偏信则暗。”\footnote{魏征(五八○——六四三),唐代初期的政治活动家和历史学家。本文引语见《资治通鉴》卷一百九十二。}也懂得片面性不对。可是我们的同志看问题,往往带片面性,这样的人就往往碰钉子。《水浒传》上宋江三打祝家庄,两次都因情况不明,方法不对,打了败仗。后来改变方法,从调查情形入手,于是熟悉了盘陀路,拆散了李家庄、扈家庄和祝家庄的联盟,并且布置了藏在敌人营盘里的伏兵,用了和外国故事中所说木马计相像的方法,第三次就打了胜仗。《水浒传》上有很多唯物辩证法的事例,这个三打祝家庄,算是最好的一个。列宁说:“要真正地认识对象,就必须把握和研究它的一切方面、一切联系和‘媒介’。我们决不会完全地作到这一点,可是要求全面性,将使我们防止错误,防止僵化。”\footnote{见列宁《再论工会、目前局势及托洛茨基同志和布哈林同志的错误》。新的译文是:“要真正地认识事物,就必须把握住、研究清楚它的一切方面、一切联系和‘中介’。我们永远也不会完全做到这一点,但是,全面性这一要求可以使我们防止犯错误和防止僵化。”(《列宁全集》第40卷,人民出版社1986年版,第291页)}我们应该记得他的话。表面性,是对矛盾总体和矛盾各方的特点都不去看,否认深入事物里面精细地研究矛盾特点的必要,仅仅站在那里远远地望一望,粗枝大叶地看到一点矛盾的形相,就想动手去解决矛盾(答复问题、解决纠纷、处理工作、指挥战争)。这样的做法,没有不出乱子的。中国的教条主义和经验主义的同志们所以犯错误,就是因为他们看事物的方法是主观的、片面的和表面的。片面性、表面性也是主观性,因为一切客观事物本来是互相联系的和具有内部规律的,人们不去如实地反映这些情况,而只是片面地或表面地去看它们,不认识事物的互相联系,不认识事物的内部规律,所以这种方法是主观主义的。

不但事物发展的全过程中的矛盾运动,在其相互联结上,在其各方情况上,我们必须注意其特点,而且在过程发展的各个阶段中,也有其特点,也必须注意。

事物发展过程的根本矛盾及为此根本矛盾所规定的过程的本质,非到过程完结之日,是不会消灭的;但是事物发展的长过程中的各个发展的阶段,情形又往往互相区别。这是因为事物发展过程的根本矛盾的性质和过程的本质虽然没有变化,但是根本矛盾在长过程中的各个发展阶段上采取了逐渐激化的形式。并且,被根本矛盾所规定或影响的许多大小矛盾中,有些是激化了,有些是暂时地或局部地解决了,或者缓和了,又有些是发生了,因此,过程就显出阶段性来。如果人们不去注意事物发展过程中的阶段性,人们就不能适当地处理事物的矛盾。

例如,自由竞争时代的资本主义发展为帝国主义,这时,无产阶级和资产阶级这两个根本矛盾着的阶级的性质和这个社会的资本主义的本质,并没有变化;但是,两阶级的矛盾激化了,独占资本和自由资本之间的矛盾发生了,宗主国和殖民地的矛盾激化了,各资本主义国家间的矛盾即由各国发展不平衡的状态而引起的矛盾特别尖锐地表现出来了,因此形成了资本主义的特殊阶段,形成了帝国主义阶段。列宁主义之所以成为帝国主义和无产阶级革命时代的马克思主义,就是因为列宁和斯大林正确地说明了这些矛盾,并正确地作出了解决这些矛盾的无产阶级革命的理论和策略。

拿从辛亥革命开始的中国资产阶级民主革命过程的情形来看,也有了若干特殊阶段。特别是在资产阶级领导时期的革命和在无产阶级领导时期的革命,区别为两个很大不同的历史阶段。这就是:由于无产阶级的领导,根本地改变了革命的面貌,引出了阶级关系的新调度,农民革命的大发动,反帝国主义和反封建主义的革命彻底性,由民主革命转变到社会主义革命的可能性,等等。所有这些,都是在资产阶级领导革命时期不可能出现的。虽然整个过程中根本矛盾的性质,过程之反帝反封建的民主革命的性质(其反面是半殖民地半封建的性质),并没有变化,但是,在这长时间中,经过了辛亥革命失败和北洋军阀统治,第一次民族统一战线的建立和一九二四年至一九二七年的革命,统一战线破裂和资产阶级转入反革命,新的军阀战争,土地革命战争,第二次民族统一战线建立和抗日战争等等大事变,二十多年间经过了几个发展阶段。在这些阶段中,包含着有些矛盾激化了(例如土地革命战争和日本侵入东北四省),有些矛盾部分地或暂时地解决了(例如北洋军阀的被消灭,我们没收了地主的土地),有些矛盾重新发生了(例如新军阀之间的斗争,南方各革命根据地丧失后地主又重新收回土地)等等特殊的情形。

研究事物发展过程中的各个发展阶段上的矛盾的特殊性,不但必须在其联结上、在其总体上去看,而且必须从各个阶段中矛盾的各个方面去看。

例如国共两党。国民党方面,在第一次统一战线时期,因为它实行了孙中山的联俄、联共、援助工农的三大政策,所以它是革命的、有朝气的,它是各阶级的民主革命的联盟。一九二七年以后,国民党变到了与此相反的方面,成了地主和大资产阶级的反动集团。一九三六年十二月西安事变后又开始向停止内战、联合共产党共同反对日本帝国主义这个方面转变。这就是国民党在三个阶段上的特点。形成这些特点,当然有种种的原因。中国共产党方面,在第一次统一战线时期,它是幼年的党,它英勇地领导了一九二四年至一九二七年的革命;但在对于革命的性质、任务和方法的认识方面,却表现了它的幼年性,因此在这次革命的后期所发生的陈独秀主义能够起作用,使这次革命遭受了失败。一九二七年以后,它又英勇地领导了土地革命战争,创立了革命的军队和革命的根据地,但是它也犯过冒险主义的错误,使军队和根据地都受了很大的损失。一九三五年以后,它又纠正了冒险主义的错误,领导了新的抗日的统一战线,这个伟大的斗争现在正在发展。在这个阶段上,共产党是一个经过了两次革命的考验、有了丰富的经验的党。这些就是中国共产党在三个阶段上的特点。形成这些特点也有种种的原因。不研究这些特点,就不能了解两党在各个发展阶段上的特殊的相互关系:统一战线的建立,统一战线的破裂,再一个统一战线的建立。而要研究两党的种种特点,更根本的就必须研究这两党的阶级基础以及因此在各个时期所形成的它们和其它方面的矛盾的对立。例如,国民党在它第一次联合共产党的时期,一方面有和国外帝国主义的矛盾,因而它反对帝国主义;另一方面有和国内人民大众的矛盾,它在口头上虽然允许给予劳动人民以许多的利益,但在实际上则只给予很少的利益,或者简直什么也不给。在它进行反共战争的时期,则和帝国主义、封建主义合作反对人民大众,一笔勾销了人民大众原来在革命中所争得的一切利益,激化了它和人民大众的矛盾。现在抗日时期,国民党和日本帝国主义有矛盾,它一面要联合共产党,同时它对共产党和国内人民并不放松其斗争和压迫。共产党则无论在哪一时期,均和人民大众站在一道,反对帝国主义和封建主义;但在现在的抗日时期,由于国民党表示抗日,它对国民党和国内封建势力,也就采取了缓和的政策。由于这些情况,所以或者造成了两党的联合,或者造成了两党的斗争,而且即使在两党联合的时期也有又联合又斗争的复杂的情况。如果我们不去研究这些矛盾方面的特点,我们就不但不能了解这两个党各各和其它方面的关系,也不能了解两党之间的相互关系。

由此看来,不论研究何种矛盾的特性——各个物质运动形式的矛盾,各个运动形式在各个发展过程中的矛盾,各个发展过程的矛盾的各方面,各个发展过程在其各个发展阶段上的矛盾以及各个发展阶段上的矛盾的各方面,研究所有这些矛盾的特性,都不能带主观随意性,必须对它们实行具体的分析。离开具体的分析,就不能认识任何矛盾的特性。我们必须时刻记得列宁的话:对于具体的事物作具体的分析。

这种具体的分析,马克思、恩格斯首先给了我们以很好的模范。

当马克思、恩格斯把这事物矛盾的法则应用到社会历史过程的研究的时候,他们看出生产力和生产关系之间的矛盾,看出剥削阶级和被剥削阶级之间的矛盾以及由于这些矛盾所产生的经济基础和政治及思想等上层建筑之间的矛盾,而这些矛盾如何不可避免地会在各种不同的阶级社会中,引出各种不同的社会革命。

马克思把这一法则应用到资本主义社会经济结构的研究的时候,他看出这一社会的基本矛盾在于生产的社会性和占有制的私人性之间的矛盾。这个矛盾表现于在各别企业中的生产的有组织性和在全社会中的生产的无组织性之间的矛盾。这个矛盾的阶级表现则是资产阶级和无产阶级之间的矛盾。

由于事物范围的极其广大,发展的无限性,所以,在一定场合为普遍性的东西,而在另一一定场合则变为特殊性。反之,在一定场合为特殊性的东西,而在另一一定场合则变为普遍性。资本主义制度所包含的生产社会化和生产资料私人占有制的矛盾,是所有有资本主义的存在和发展的各国所共有的东西,对于资本主义说来,这是矛盾的普遍性。但是资本主义的这种矛盾,乃是一般阶级社会发展在一定历史阶段上的东西,对于一般阶级社会中的生产力和生产关系的矛盾说来,这是矛盾的特殊性。然而,当着马克思把资本主义社会这一切矛盾的特殊性解剖出来之后,同时也就更进一步地、更充分地、更完全地把一般阶级社会中这个生产力和生产关系的矛盾的普遍性阐发出来了。

由于特殊的事物是和普遍的事物联结的,由于每一个事物内部不但包含了矛盾的特殊性,而且包含了矛盾的普遍性,普遍性即存在于特殊性之中,所以,当着我们研究一定事物的时候,就应当去发现这两方面及其互相联结,发现一事物内部的特殊性和普遍性的两方面及其互相联结,发现一事物和它以外的许多事物的互相联结。斯大林在他的名著《论列宁主义基础》一书中说明列宁主义的历史根源的时候,他分析了列宁主义所由产生的国际环境,分析了在帝国主义条件下已经发展到极点的资本主义的诸矛盾,以及这些矛盾使无产阶级革命成为直接实践的问题,并造成了直接冲击资本主义的良好的条件。不但如此,他又分析了为什么俄国成为列宁主义的策源地,分析了沙皇俄国当时是帝国主义一切矛盾的集合点以及俄国无产阶级所以能够成为国际的革命无产阶级的先锋队的原因。这样,斯大林分析了帝国主义的矛盾的普遍性,说明列宁主义是帝国主义和无产阶级革命时代的马克思主义;又分析了沙俄帝国主义在这一般矛盾中所具有的特殊性,说明俄国成了无产阶级革命理论和策略的故乡,而在这种特殊性中间就包含了矛盾的普遍性。斯大林的这种分析,给我们提供了认识矛盾的特殊性和普遍性及其互相联结的模范。

马克思和恩格斯,同样地列宁和斯大林,他们对于应用辩证法到客观现象的研究的时候,总是指导人们不要带上任何的主观随意性,而必须从客观的实际运动所包含的具体的条件,去看出这些现象中的具体的矛盾、矛盾各方面的具体的地位以及矛盾的具体的相互关系。我们的教条主义者因为没有这种研究态度,所以弄得一无是处。我们必须以教条主义的失败为鉴戒,学会这种研究态度,舍此没有第二种研究法。

矛盾的普遍性和矛盾的特殊性的关系,就是矛盾的共性和个性的关系。其共性是矛盾存在于一切过程中,并贯串于一切过程的始终,矛盾即是运动,即是事物,即是过程,也即是思想。否认事物的矛盾就是否认了一切。这是共通的道理,古今中外,概莫能外。所以它是共性,是绝对性。然而这种共性,即包含于一切个性之中,无个性即无共性。假如除去一切个性,还有什么共性呢?因为矛盾的各各特殊,所以造成了个性。一切个性都是有条件地暂时地存在的,所以是相对的。

这一共性个性、绝对相对的道理,是关于事物矛盾的问题的精髓,不懂得它,就等于抛弃了辩证法。

四、主要的矛盾和主要的矛盾方面

在矛盾特殊性的问题中,还有两种情形必须特别地提出来加以分析,这就是主要的矛盾和主要的矛盾方面。

在复杂的事物的发展过程中,有许多的矛盾存在,其中必有一种是主要的矛盾,由于它的存在和发展规定或影响着其它矛盾的存在和发展。

例如在资本主义社会中,无产阶级和资产阶级这两个矛盾着的力量是主要的矛盾;其它的矛盾力量,例如,残存的封建阶级和资产阶级的矛盾,农民小资产者和资产阶级的矛盾,无产阶级和农民小资产者的矛盾,自由资产阶级和垄断资产阶级的矛盾,资产阶级的民主主义和资产阶级的法西斯主义的矛盾,资本主义国家相互间的矛盾,帝国主义和殖民地的矛盾,以及其它的矛盾,都为这个主要的矛盾力量所规定、所影响。

半殖民地的国家如中国,其主要矛盾和非主要矛盾的关系呈现着复杂的情况。

当着帝国主义向这种国家举行侵略战争的时候,这种国家的内部各阶级,除开一些叛国分子以外,能够暂时地团结起来举行民族战争去反对帝国主义。这时,帝国主义和这种国家之间的矛盾成为主要的矛盾,而这种国家内部各阶级的一切矛盾(包括封建制度和人民大众之间这个主要矛盾在内),便都暂时地降到次要和服从的地位。中国一八四○年的鸦片战争,一八九四年的中日战争\footnote{一八九四年(甲午年)发生的中日战争,也称甲午战争。这次战争是日本军国主义者蓄意挑起的。日本军队先向朝鲜发动侵略并对中国的陆海军进行挑衅,继即大举侵入中国的东北。在战争中,中国军队曾经英勇作战,但是由于清朝政府的腐败以及缺乏坚决反对侵略的准备,中国方面遭到了失败。一八九五年,清朝政府和日本订立了可耻的马关条约,这个条约的主要内容是:中国割让台湾全岛及所有附属各岛屿、澎湖列岛和辽东半岛(后来在俄、德、法三国干涉下,日本同意由清政府偿付白银三千万两“赎还”该半岛),赔偿军费银二万万两,允许日本人在中国通商口岸开设工厂,开辟沙市、重庆、苏州、杭州等地为商埠。},一九○○年的义和团战争和目前的中日战争,都有这种情形。

然而在另一种情形之下,则矛盾的地位起了变化。当着帝国主义不是用战争压迫而是用政治、经济、文化等比较温和的形式进行压迫的时候,半殖民地国家的统治阶级就会向帝国主义投降,二者结成同盟,共同压迫人民大众。这种时候,人民大众往往采取国内战争的形式,去反对帝国主义和封建阶级的同盟,而帝国主义则往往采取间接的方式去援助半殖民地国家的反动派压迫人民,而不采取直接行动,显出了内部矛盾的特别尖锐性。中国的辛亥革命战争,一九二四年至一九二七年的革命战争,一九二七年以后的十年土地革命战争,都有这种情形。还有半殖民地国家各个反动的统治集团之间的内战,例如在中国的军阀战争,也属于这一类。

当着国内革命战争发展到从根本上威胁帝国主义及其走狗国内反动派的存在的时候,帝国主义就往往采取上述方法以外的方法,企图维持其统治:或者分化革命阵线的内部,或者直接出兵援助国内反动派。这时,外国帝国主义和国内反动派完全公开地站在一个极端,人民大众则站在另一极端,成为一个主要矛盾,而规定或影响其它矛盾的发展状态。十月革命后各资本主义国家援助俄国反动派,是武装干涉的例子。一九二七年的蒋介石的叛变,是分化革命阵线的例子。

然而不管怎样,过程发展的各个阶段中,只有一种主要的矛盾起着领导的作用,是完全没有疑义的。

由此可知,任何过程如果有多数矛盾存在的话,其中必定有一种是主要的,起着领导的、决定的作用,其它则处于次要和服从的地位。因此,研究任何过程,如果是存在着两个以上矛盾的复杂过程的话,就要用全力找出它的主要矛盾。捉住了这个主要矛盾,一切问题就迎刃而解了。这是马克思研究资本主义社会告诉我们的方法。列宁和斯大林研究帝国主义和资本主义总危机的时候,列宁和斯大林研究苏联经济的时候,也告诉了这种方法。万千的学问家和实行家,不懂得这种方法,结果如堕烟海,找不到中心,也就找不到解决矛盾的方法。

不能把过程中所有的矛盾平均看待,必须把它们区别为主要的和次要的两类,着重于捉住主要的矛盾,已如上述。但是在各种矛盾之中,不论是主要的或次要的,矛盾着的两个方面,又是否可以平均看待呢?也是不可以的。无论什么矛盾,矛盾的诸方面,其发展是不平衡的。有时候似乎势均力敌,然而这只是暂时的和相对的情形,基本的形态则是不平衡。矛盾着的两方面中,必有一方面是主要的,他方面是次要的。其主要的方面,即所谓矛盾起主导作用的方面。事物的性质,主要地是由取得支配地位的矛盾的主要方面所规定的。

然而这种情形不是固定的,矛盾的主要和非主要的方面互相转化着,事物的性质也就随着起变化。在矛盾发展的一定过程或一定阶段上,主要方面属于甲方,非主要方面属于乙方;到了另一发展阶段或另一发展过程时,就互易其位置,这是依靠事物发展中矛盾双方斗争的力量的增减程度来决定的。

我们常常说“新陈代谢”这句话。新陈代谢是宇宙间普遍的永远不可抵抗的规律。依事物本身的性质和条件,经过不同的飞跃形式,一事物转化为他事物,就是新陈代谢的过程。任何事物的内部都有其新旧两个方面的矛盾,形成为一系列的曲折的斗争。斗争的结果,新的方面由小变大,上升为支配的东西;旧的方面则由大变小,变成逐步归于灭亡的东西。而一当新的方面对于旧的方面取得支配地位的时候,旧事物的性质就变化为新事物的性质。由此可见,事物的性质主要地是由取得支配地位的矛盾的主要方面所规定的。取得支配地位的矛盾的主要方面起了变化,事物的性质也就随着起变化。

在资本主义社会中,资本主义已从旧的封建主义社会时代的附庸地位,转化成了取得支配地位的力量,社会的性质也就由封建主义的变为资本主义的。在新的资本主义社会时代,封建势力则由原来处在支配地位的力量转化为附庸的力量,随着也就逐步地归于消灭了,例如英法诸国就是如此。随着生产力的发展,资产阶级由新的起进步作用的阶级,转化为旧的起反动作用的阶级,以至于最后被无产阶级所推翻,而转化为私有的生产资料被剥夺和失去权力的阶级,这个阶级也就要逐步归于消灭了。人数比资产阶级多得多、并和资产阶级同时生长、但被资产阶级统治着的无产阶级,是一个新的力量,它由初期的附属于资产阶级的地位,逐步地壮大起来,成为独立的和在历史上起主导作用的阶级,以至最后夺取政权成为统治阶级。这时,社会的性质,就由旧的资本主义的社会转化成了新的社会主义的社会。这就是苏联已经走过和一切其它国家必然要走的道路。

就中国的情形来说,帝国主义处在形成半殖民地这种矛盾的主要地位,压迫中国人民,中国则由独立国变为半殖民地。然而事情必然会变化,在双方斗争的局势中,中国人民在无产阶级领导之下所生长起来的力量必然会把中国由半殖民地变为独立国,而帝国主义则将被打倒,旧中国必然要变为新中国。

旧中国变为新中国,还包含着国内旧的封建势力和新的人民势力之间的情况的变化。旧的封建地主阶级将被打倒,由统治者变为被统治者,这个阶级也就会要逐步归于消灭。人民则将在无产阶级领导之下,由被统治者变为统治者。这时,中国社会的性质就会起变化,由旧的半殖民地和半封建的社会变为新的民主的社会。

这种互相转化的事情,过去已有经验。统治中国将近三百年的清朝帝国,曾在辛亥革命时期被打倒;而孙中山领导的革命同盟会,则曾经一度取得了胜利。在一九二四年至一九二七年的革命战争中,共产党和国民党联合的南方革命势力,曾经由弱小的力量变得强大起来,取得了北伐的胜利;而称雄一时的北洋军阀则被打倒了。一九二七年,共产党领导的人民力量,受了国民党反动势力的打击,变得很小了;但因肃清了自己内部的机会主义,就又逐步地壮大起来。在共产党领导的革命根据地内,农民由被统治者转化为统治者,地主则作了相反的转化。世界上总是这样以新的代替旧的,总是这样新陈代谢、除旧布新或推陈出新的。

革命斗争中的某些时候,困难条件超过顺利条件,在这种时候,困难是矛盾的主要方面,顺利是其次要方面。然而由于革命党人的努力,能够逐步地克服困难,开展顺利的新局面,困难的局面让位于顺利的局面。一九二七年中国革命失败后的情形,中国红军在长征中的情形,都是如此。现在的中日战争,中国又处在困难地位,但是我们能够改变这种情况,使中日双方的情况发生根本的变化。在相反的情形之下,顺利也能转化为困难,如果是革命党人犯了错误的话。一九二四年至一九二七年的革命的胜利,变为失败了。一九二七年以后在南方各省发展起来的革命根据地,至一九三四年都失败了。

研究学问的时候,由不知到知的矛盾也是如此。当着我们刚才开始研究马克思主义的时候,对于马克思主义的无知或知之不多的情况,和马克思主义的知识之间,互相矛盾着。然而由于努力学习,可以由无知转化为有知,由知之不多转化为知之甚多,由对于马克思主义的盲目性改变为能够自由运用马克思主义。

有人觉得有些矛盾并不是这样。例如,生产力和生产关系的矛盾,生产力是主要的;理论和实践的矛盾,实践是主要的;经济基础和上层建筑的矛盾,经济基础是主要的:它们的地位并不互相转化。这是机械唯物论的见解,不是辩证唯物论的见解。诚然,生产力、实践、经济基础,一般地表现为主要的决定的作用,谁不承认这一点,谁就不是唯物论者。然而,生产关系、理论、上层建筑这些方面,在一定条件之下,又转过来表现其为主要的决定的作用,这也是必须承认的。当着不变更生产关系,生产力就不能发展的时候,生产关系的变更就起了主要的决定的作用。当着如同列宁所说“没有革命的理论,就不会有革命的运动”\footnote{见列宁《俄国社会民主党人的任务》(《列宁全集》第2卷,人民出版社1984年版,第443页);并见列宁《怎么办?》第一章第四节(《列宁全集》第6卷,人民出版社1986年版,第23页)。}的时候,革命理论的创立和提倡就起了主要的决定的作用。当着某一件事情(任何事情都是一样)要做,但是还没有方针、方法、计划或政策的时候,确定方针、方法、计划或政策,也就是主要的决定的东西。当着政治文化等等上层建筑阻碍着经济基础的发展的时候,对于政治上和文化上的革新就成为主要的决定的东西了。我们这样说,是否违反了唯物论呢?没有。因为我们承认总的历史发展中是物质的东西决定精神的东西,是社会的存在决定社会的意识;但是同时又承认而且必须承认精神的东西的反作用,社会意识对于社会存在的反作用,上层建筑对于经济基础的反作用。这不是违反唯物论,正是避免了机械唯物论,坚持了辩证唯物论。

在研究矛盾特殊性的问题中,如果不研究过程中主要的矛盾和非主要的矛盾以及矛盾之主要的方面和非主要的方面这两种情形,也就是说不研究这两种矛盾情况的差别性,那就将陷入抽象的研究,不能具体地懂得矛盾的情况,因而也就不能找出解决矛盾的正确的方法。这两种矛盾情况的差别性或特殊性,都是矛盾力量的不平衡性。世界上没有绝对地平衡发展的东西,我们必须反对平衡论,或均衡论。同时,这种具体的矛盾状况,以及矛盾的主要方面和非主要方面在发展过程中的变化,正是表现出新事物代替旧事物的力量。对于矛盾的各种不平衡情况的研究,对于主要的矛盾和非主要的矛盾、主要的矛盾方面和非主要的矛盾方面的研究,成为革命政党正确地决定其政治上和军事上的战略战术方针的重要方法之一,是一切共产党人都应当注意的。

五、矛盾诸方面的同一性和斗争性

在懂得了矛盾的普遍性和特殊性的问题之后,我们必须进而研究矛盾诸方面的同一性和斗争性的问题。

同一性、统一性、一致性、互相渗透、互相贯通、互相依赖(或依存)、互相联结或互相合作,这些不同的名词都是一个意思,说的是如下两种情形:第一、事物发展过程中的每一种矛盾的两个方面,各以和它对立着的方面为自己存在的前提,双方共处于一个统一体中;第二、矛盾着的双方,依据一定的条件,各向着其相反的方面转化。这些就是所谓同一性。

列宁说:“辩证法是这样的一种学说:它研究对立怎样能够是同一的,又怎样成为同一的(怎样变成同一的),——在怎样的条件之下它们互相转化,成为同一的,——为什么人的头脑不应当把这些对立看作死的、凝固的东西,而应当看作生动的、有条件的、可变动的、互相转化的东西。”\footnote{见列宁《黑格尔〈逻辑学〉一书摘要》。新的译文是:“辩证法是一种学说,它研究对立面怎样才能够同一,是怎样(怎样成为)同一的——在什么条件下它们是相互转化而同一的,——为什么人的头脑不应该把这些对立面看作僵死的、凝固的东西,而应该看作活生生的、有条件的、活动的、彼此转化的东西。”(《列宁全集》第55卷,人民出版社1990年版,第90页)}

列宁这段话是什么意思呢?

一切过程中矛盾着的各方面,本来是互相排斥、互相斗争、互相对立的。世界上一切事物的过程里和人们的思想里,都包含着这样带矛盾性的方面,无一例外。单纯的过程只有一对矛盾,复杂的过程则有一对以上的矛盾。各对矛盾之间,又互相成为矛盾。这样地组成客观世界的一切事物和人们的思想,并推使它们发生运动。

如此说来,只是极不同一,极不统一,怎样又说是同一或统一呢?

原来矛盾着的各方面,不能孤立地存在。假如没有和它作对的矛盾的一方,它自己这一方就失去了存在的条件。试想一切矛盾着的事物或人们心中矛盾着的概念,任何一方面能够独立地存在吗?没有生,死就不见;没有死,生也不见。没有上,无所谓下;没有下,也无所谓上。没有祸,无所谓福;没有福,也无所谓祸。没有顺利,无所谓困难;没有困难,也无所谓顺利。没有地主,就没有佃农;没有佃农,也就没有地主。没有资产阶级,就没有无产阶级;没有无产阶级,也就没有资产阶级。没有帝国主义的民族压迫,就没有殖民地和半殖民地;没有殖民地和半殖民地,也就没有帝国主义的民族压迫。一切对立的成分都是这样,因一定的条件,一面互相对立,一面又互相联结、互相贯通、互相渗透、互相依赖,这种性质,叫做同一性。一切矛盾着的方面都因一定条件具备着不同一性,所以称为矛盾。然而又具备着同一性,所以互相联结。列宁所谓辩证法研究“对立怎样能够是同一的”,就是说的这种情形。怎样能够呢?因为互为存在的条件。这是同一性的第一种意义。

然而单说了矛盾双方互为存在的条件,双方之间有同一性,因而能够共处于一个统一体中,这样就够了吗?还不够。事情不是矛盾双方互相依存就完了,更重要的,还在于矛盾着的事物的互相转化。这就是说,事物内部矛盾着的两方面,因为一定的条件而各向着和自己相反的方面转化了去,向着它的对立方面所处的地位转化了去。这就是矛盾的同一性的第二种意义。

为什么这里也有同一性呢?你们看,被统治的无产阶级经过革命转化为统治者,原来是统治者的资产阶级却转化为被统治者,转化到对方原来所占的地位。苏联已经是这样做了,全世界也将要这样做。试问其间没有在一定条件之下的联系和同一性,如何能够发生这样的变化呢?

曾在中国近代历史的一定阶段上起过某种积极作用的国民党,因为它的固有的阶级性和帝国主义的引诱(这些就是条件),在一九二七年以后转化为反革命,又由于中日矛盾的尖锐化和共产党的统一战线政策(这些就是条件),而被迫着赞成抗日。矛盾着的东西这一个变到那一个,其间包含了一定的同一性。

我们实行过的土地革命,已经是并且还将是这样的过程,拥有土地的地主阶级转化为失掉土地的阶级,而曾经是失掉土地的农民却转化为取得土地的小私有者。有无、得失之间,因一定条件而互相联结,二者具有同一性。在社会主义条件之下,农民的私有制又将转化为社会主义农业的公有制,苏联已经这样做了,全世界将来也会这样做。私产和公产之间有一条由此达彼的桥梁,哲学上名之曰同一性,或互相转化、互相渗透。

巩固无产阶级的专政或人民的专政,正是准备着取消这种专政,走到消灭任何国家制度的更高阶段去的条件。建立和发展共产党,正是准备着消灭共产党和一切政党制度的条件。建立共产党领导的革命军,进行革命战争,正是准备着永远消灭战争的条件。这许多相反的东西,同时却是相成的东西。

大家知道,战争与和平是互相转化的。战争转化为和平,例如第一次世界大战转化为战后的和平,中国的内战现在也停止了,出现了国内的和平。和平转化为战争,例如一九二七年的国共合作转化为战争,现在的世界和平局面也可能转化为第二次世界大战。为什么是这样?因为在阶级社会中战争与和平这样矛盾着的事物,在一定条件下具备着同一性。

一切矛盾着的东西,互相联系着,不但在一定条件之下共处于一个统一体中,而且在一定条件之下互相转化,这就是矛盾的同一性的全部意义。列宁所谓“怎样成为同一的(怎样变成同一的),——在怎样的条件之下它们互相转化,成为同一的”,就是这个意思。

“为什么人的头脑不应当把这些对立看作死的、凝固的东西,而应当看作生动的、有条件的、可变动的、互相转化的东西”呢?因为客观事物本来是如此的。客观事物中矛盾着的诸方面的统一或同一性,本来不是死的、凝固的,而是生动的、有条件的、可变动的、暂时的、相对的东西,一切矛盾都依一定条件向它们的反面转化着。这种情况,反映在人们的思想里,就成了马克思主义的唯物辩证法的宇宙观。只有现在的和历史上的反动的统治阶级以及为他们服务的形而上学,不是把对立的事物当作生动的、有条件的、可变动的、互相转化的东西去看,而是当作死的、凝固的东西去看,并且把这种错误的看法到处宣传,迷惑人民群众,以达其继续统治的目的。共产党人的任务就在于揭露反动派和形而上学的错误思想,宣传事物的本来的辩证法,促成事物的转化,达到革命的目的。

所谓矛盾在一定条件下的同一性,就是说,我们所说的矛盾乃是现实的矛盾,具体的矛盾,而矛盾的互相转化也是现实的、具体的。神话中的许多变化,例如《山海经》中所说的“夸父追日”\footnote{《山海经》是一部中国古代地理著作,其中记载了不少远古的神话传说。夸父是《山海经•海外北经》上记载的一个神人。据说:“夸父与日逐走。入日,渴欲得饮,饮于河渭。河渭不足,北饮大泽。未至,道渴而死。弃其杖,化为邓林。”},《淮南子》中所说的“羿射九日”\footnote{羿是中国古代传说中的英雄,“射日”是关于他善射的著名故事。据西汉淮南王刘安(公元前二世纪人)及其门客所著《淮南子》一书说:“尧之时,十日并出,焦禾稼,杀草木,而民无所食。猰豸、凿齿、九婴、大风、封狶、修蛇,皆为民害。尧乃使羿……上射十日而下杀猰豸。……万民皆喜。”东汉著作家王逸(公元二世纪人)关于屈原诗篇《天问》的注释说:“淮南言,尧时十日并出,草木焦枯。尧命羿仰射十日,中其九日……留其一日。”},《西游记》中所说的孙悟空七十二变和《聊斋志异》中的许多鬼狐变人的故事等等,这种神话中所说的矛盾的互相变化,乃是无数复杂的现实矛盾的互相变化对于人们所引起的一种幼稚的、想象的、主观幻想的变化,并不是具体的矛盾所表现出来的具体的变化。马克思说:“任何神话都是用想象和借助想象以征服自然力,支配自然力,把自然力加以形象化;因而,随着这些自然力之实际上被支配,神话也就消失了。”\footnote{见马克思《〈政治经济学批判〉导言》(《马克思恩格斯选集》第2卷,人民出版社1972年版,第113页)。}这种神话中的(还有童话中的)千变万化的故事,虽然因为它们想象出人们征服自然力等等,而能够吸引人们的喜欢,并且最好的神话具有“永久的魅力”\footnote{见马克思《〈政治经济学批判〉导言》(《马克思恩格斯选集》第2卷,人民出版社1972年版,第114页)。}(马克思),但神话并不是根据具体的矛盾之一定的条件而构成的,所以它们并不是现实之科学的反映。这就是说,神话或童话中矛盾构成的诸方面,并不是具体的同一性,只是幻想的同一性。科学地反映现实变化的同一性的,就是马克思主义的辩证法。

为什么鸡蛋能够转化为鸡子,而石头不能够转化为鸡子呢?为什么战争与和平有同一性,而战争与石头却没有同一性呢?为什么人能生人不能生出其它的东西呢?没有别的,就是因为矛盾的同一性要在一定的必要的条件之下。缺乏一定的必要的条件,就没有任何的同一性。

为什么俄国在一九一七年二月的资产阶级民主革命和同年十月的无产阶级社会主义革命直接地联系着,而法国资产阶级革命没有直接地联系于社会主义的革命,一八七一年的巴黎公社终于失败了呢?为什么蒙古和中亚细亚的游牧制度又直接地和社会主义联系了呢?为什么中国的革命可以避免资本主义的前途,可以和社会主义直接联系起来,不要再走西方国家的历史老路,不要经过一个资产阶级专政的时期呢?没有别的,都是由于当时的具体条件。一定的必要的条件具备了,事物发展的过程就发生一定的矛盾,而且这种或这些矛盾互相依存,又互相转化,否则,一切都不可能。

同一性的问题如此。那末,什么是斗争性呢?同一性和斗争性的关系是怎样的呢?

列宁说:“对立的统一(一致、同一、合一),是有条件的、一时的、暂存的、相对的。互相排斥的对立的斗争则是绝对的,正如发展、运动是绝对的一样。”\footnote{见列宁《谈谈辩证法问题》。新的译文是:“对立面的统一(一致、同一、均势)是有条件的、暂时的、易逝的、相对的。相互排斥的对立面的斗争是绝对的,正如发展、运动是绝对的一样。”(《列宁全集》第55卷,人民出版社1990年版,第306页)}

列宁这段话是什么意思呢?

一切过程都有始有终,一切过程都转化为它们的对立物。一切过程的常住性是相对的,但是一种过程转化为他种过程的这种变动性则是绝对的。

无论什么事物的运动都采取两种状态,相对地静止的状态和显着地变动的状态。两种状态的运动都是由事物内部包含的两个矛盾着的因素互相斗争所引起的。当着事物的运动在第一种状态的时候,它只有数量的变化,没有性质的变化,所以显出好似静止的面貌。当着事物的运动在第二种状态的时候,它已由第一种状态中的数量的变化达到了某一个最高点,引起了统一物的分解,发生了性质的变化,所以显出显着地变化的面貌。我们在日常生活中所看见的统一、团结、联合、调和、均势、相持、僵局、静止、有常、平衡、凝聚、吸引等等,都是事物处在量变状态中所显现的面貌。而统一物的分解,团结、联合、调和、均势、相持、僵局、静止、有常、平衡、凝聚、吸引等等状态的破坏,变到相反的状态,便都是事物在质变状态中、在一种过程过渡到他种过程的变化中所显现的面貌。事物总是不断地由第一种状态转化为第二种状态,而矛盾的斗争则存在于两种状态中,并经过第二种状态而达到矛盾的解决。所以说,对立的统一是有条件的、暂时的、相对的,而对立的互相排除的斗争则是绝对的。

前面我们曾经说,两个相反的东西中间有同一性,所以二者能够共处于一个统一体中,又能够互相转化,这是说的条件性,即是说在一定条件之下,矛盾的东西能够统一起来,又能够互相转化;无此一定条件,就不能成为矛盾,不能共居,也不能转化。由于一定的条件才构成了矛盾的同一性,所以说同一性是有条件的、相对的。这里我们又说,矛盾的斗争贯串于过程的始终,并使一过程向着他过程转化,矛盾的斗争无所不在,所以说矛盾的斗争性是无条件的、绝对的。

有条件的相对的同一性和无条件的绝对的斗争性相结合,构成了一切事物的矛盾运动。

我们中国人常说:“相反相成。”\footnote{见东汉著名史学家班固(三二——九二)所著《汉书•艺文志》,原文是:“诸子十家,其可观者,九家而已。皆起于王道既微,诸侯力政,时君世主,好恶殊方。是以九家之术,蜂出并作,各引一端,崇其所善,以此驰说,取合诸侯。其言虽殊,辟犹水火,相灭亦相生也。仁之与义,敬之与和,相反而皆相成也。”}就是说相反的东西有同一性。这句话是辩证法的,是违反形而上学的。“相反”就是说两个矛盾方面的互相排斥,或互相斗争。“相成”就是说在一定条件之下两个矛盾方面互相联结起来,获得了同一性。而斗争性即寓于同一性之中,没有斗争性就没有同一性。

在同一性中存在着斗争性,在特殊性中存在着普遍性,在个性中存在着共性。拿列宁的话来说,叫做“在相对的东西里面有着绝对的东西”\footnote{见列宁《谈谈辩证法问题》。新的译文是:“相对中有绝对。”(《列宁全集》第55卷,人民出版社1990年版,第307页)}。

六、对抗在矛盾中的地位


在矛盾的斗争性的问题中,包含着对抗是什么的问题。我们回答道:对抗是矛盾斗争的一种形式,而不是矛盾斗争的一切形式。

在人类历史中,存在着阶级的对抗,这是矛盾斗争的一种特殊的表现。剥削阶级和被剥削阶级之间的矛盾,无论在奴隶社会也好,封建社会也好,资本主义社会也好,互相矛盾着的两阶级,长期地并存于一个社会中,它们互相斗争着,但要待两阶级的矛盾发展到了一定的阶段的时候,双方才取外部对抗的形式,发展为革命。阶级社会中,由和平向战争的转化,也是如此。

炸弹在未爆炸的时候,是矛盾物因一定条件共居于一个统一体中的时候。待至新的条件(发火)出现,才发生了爆炸。自然界中一切到了最后要采取外部冲突形式去解决旧矛盾产生新事物的现象,都有与此相仿佛的情形。

认识这种情形,极为重要。它使我们懂得,在阶级社会中,革命和革命战争是不可避免的,舍此不能完成社会发展的飞跃,不能推翻反动的统治阶级,而使人民获得政权。共产党人必须揭露反动派所谓社会革命是不必要的和不可能的等等欺骗的宣传,坚持马克思列宁主义的社会革命论,使人民懂得,这不但是完全必要的,而且是完全可能的,整个人类的历史和苏联的胜利,都证明了这个科学的真理。

但是我们必须具体地研究各种矛盾斗争的情况,不应当将上面所说的公式不适当地套在一切事物的身上。矛盾和斗争是普遍的、绝对的,但是解决矛盾的方法,即斗争的形式,则因矛盾的性质不同而不相同。有些矛盾具有公开的对抗性,有些矛盾则不是这样。根据事物的具体发展,有些矛盾是由原来还非对抗性的,而发展成为对抗性的;也有些矛盾则由原来是对抗性的,而发展成为非对抗性的。

共产党内正确思想和错误思想的矛盾,如前所说,在阶级存在的时候,这是阶级矛盾对于党内的反映。这种矛盾,在开始的时候,或在个别的问题上,并不一定马上表现为对抗性的。但随着阶级斗争的发展,这种矛盾也就可能发展为对抗性的。苏联共产党的历史告诉我们:列宁、斯大林的正确思想和托洛茨基、布哈林等人的错误思想的矛盾,在开始的时候还没有表现为对抗的形式,但随后就发展为对抗的了。中国共产党的历史也有过这样的情形。我们党内许多同志的正确思想和陈独秀、张国焘等人的错误思想的矛盾,在开始的时候也没有表现为对抗的形式,但随后就发展为对抗的了。目前我们党内的正确思想和错误思想的矛盾,没有表现为对抗的形式,如果犯错误的同志能够改正自己的错误,那就不会发展为对抗性的东西。因此,党一方面必须对于错误思想进行严肃的斗争,另方面又必须充分地给犯错误的同志留有自己觉悟的机会。在这样的情况下,过火的斗争,显然是不适当的。但如果犯错误的人坚持错误,并扩大下去,这种矛盾也就存在着发展为对抗性的东西的可能性。

经济上城市和乡村的矛盾,在资本主义社会里面(那里资产阶级统治的城市残酷地掠夺乡村),在中国的国民党统治区域里面(那里外国帝国主义和本国买办大资产阶级所统治的城市极野蛮地掠夺乡村),那是极其对抗的矛盾。但在社会主义国家里面,在我们的革命根据地里面,这种对抗的矛盾就变为非对抗的矛盾,而当到达共产主义社会的时候,这种矛盾就会消灭。

列宁说:“对抗和矛盾断然不同。在社会主义下,对抗消灭了,矛盾存在着。”\footnote{见列宁《在尼·布哈林〈过渡时期经济学〉一书上作的批注和评论》(《列宁全集》第60卷,人民出版社1990年版,第282页)。}这就是说,对抗只是矛盾斗争的一种形式,而不是它的一切形式,不能到处套用这个公式。

七、结论

说到这里,我们可以总起来说几句。事物矛盾的法则,即对立统一的法则,是自然和社会的根本法则,因而也是思维的根本法则。它是和形而上学的宇宙观相反的。它对于人类的认识史是一个大革命。按照辩证唯物论的观点看来,矛盾存在于一切客观事物和主观思维的过程中,矛盾贯串于一切过程的始终,这是矛盾的普遍性和绝对性。矛盾着的事物及其每一个侧面各有其特点,这是矛盾的特殊性和相对性。矛盾着的事物依一定的条件有同一性,因此能够共居于一个统一体中,又能够互相转化到相反的方面去,这又是矛盾的特殊性和相对性。然而矛盾的斗争则是不断的,不管在它们共居的时候,或者在它们互相转化的时候,都有斗争的存在,尤其是在它们互相转化的时候,斗争的表现更为显着,这又是矛盾的普遍性和绝对性。当着我们研究矛盾的特殊性和相对性的时候,要注意矛盾和矛盾方面的主要的和非主要的区别;当着我们研究矛盾的普遍性和斗争性的时候,要注意矛盾的各种不同的斗争形式的区别。否则就要犯错误。如果我们经过研究真正懂得了上述这些要点,我们就能够击破违反马克思列宁主义基本原则的不利于我们的革命事业的那些教条主义的思想;也能够使有经验的同志们整理自己的经验,使之带上原则性,而避免重复经验主义的错误。这些,就是我们研究矛盾法则的一些简单的结论。

\section{辩证法唯物论 讲授提纲 1937}

按:《辩证法唯物论(讲授提纲)》写于1937年,曾在《抗战大学》第6期至第8期(1938年4月至6月)连载。据以录入的中国人民解放军政治学院训练部翻印本,年代不详。第二章第十一节和第三章中的部分章节分别为《实践论》与《矛盾论》的最初版本,与后来的毛选版本有较多出入。文中所有注释均为录入者所加,文中不再说明。

\subsection{唯心论与唯物论}

(一)哲学中的两军对战

全部哲学史,都是唯心论和唯物论这两个互相对抗的哲学派别的斗争和发展的历史,一切哲学思潮和派别,都是这两个基本派别的变相。

各种哲学学说,都是隶属于一定社会阶级的人们所创造的。这些人们的意识,又是历史地被一定的社会生活所决定。所有的哲学学说,表现着一定社会阶级的需要,反映着社会生产力发展的水平和人类认识自然的历史阶段。哲学的命运,看哲学满足社会阶级的需要之程度如何而定。

唯心论和唯物论的社会根源,存在于阶级的矛盾的社会结构中。最初唯心论之发生是原始野蛮人类迷妄无知的产物。此后生产力发展,促使科学知识也随之发展,唯心论理应衰退,唯物论理应起而代之。然而从古至今,唯心论不但不曾衰退,反而发展起来,同唯物论竟长争高,互不相下,原因就在于社会有阶级的划分。一方面压迫阶级为着自己的利益,不得不发展与巩固其唯心论学说;一方面被压迫阶级同样为着自己的利益,不得不发展与巩固其唯物论学说。唯心论和唯物论学说都是作为阶级斗争的工具而存在,在阶级没有消灭以前,唯心论和唯物论的对战是不会消灭的。唯心论在自己的历史发展过程中,代表剥削阶级的意识形态,起着反动的作用。唯物论则是革命阶级的宇宙观,他在阶级社会内,从对反动哲学的唯心论之不断的战斗中生长与发展起来。由此,哲学中唯心论与唯物论的斗争,始终反映着反动阶级与革命阶级在利害上的斗争。哲学中的某一倾向,不管哲学者自身意识到与否,结局总是被他们所属阶级的政治方向所左右的。哲学上的任何倾向,总是直接间接助长着他们所属阶级的根本的政治利害。在这个意义下,哲学中的一定倾向的贯彻,便是他们所属阶级的政策之特殊形态。

马克思主义的哲学——辩证法唯物论的特征在于,在于要明确地理解一切社会意识(哲学也在内)的阶级性,公然声明它那无产阶级的性质,向有产阶级的唯心论哲学作坚决的斗争,并且把自己的特殊任务,从属于推翻资本主义组织、建立无产阶级专政,与建设社会主义社会的一般任务之下。在中国目前阶段上,哲学的任务,是从属于推翻帝国主义与半封建制度、彻底实现资产阶级的民主主义,并准备转变到社会主义与共产主义社会去的一般任务之下,哲学的理论与政治的实践是应该密切联系着的。

(二)唯心论与唯物论的区别

唯心论与唯物论的根本区别在那里呢?在对于哲学的根本问题,即精神与物质的关系问题(意识与存在的关系问题)之相反的回答。唯心论认精神(意识,观念,主体)为世界一切的根源,物质(自然界及社会客体)不过为其附属物。唯物论认物质离精神而独立存在,精神不过为其附属物。从这个根本问题的相反的回答出发,就生出一切问题上的分歧意见来。

在唯心论看来,世界或者是我们各种知觉的综合,或者是我们的或世界的理性所创造的精神过程。对外面的物质世界或者完全把它看成虚构的幻象,或者把它看成精神元素之物质的外壳。人类的认识,是主体的自动,是精神的自己产物。

唯物论相反,认宇宙的统一就在它的物质性。精神(意识)是物质的本性之一,是物质发展到一定阶段时才发生的。自然,物质,客观世界存在于精神之外,离精神而独立。人的认识,是客观外界的反映。

(三)唯心论发生与发展的根源

唯心论认物质为精神的产物,颠倒着实在世界的姿态,这种哲学的发生与发展的根源何在?

前面说过,最初唯心论之发生是原始野蛮人类迷妄无知的产物。但在生产发展之后促使唯心论形成哲学思潮之首先的条件,乃是体力劳动与精神劳动的分裂。社会生产力发展的结果,社会发生分工,分工再发展,分出了专门从事精神劳动的人们。但在生产力贫弱时期,两者的分裂还没有达到完全分离的程度。到了阶级出现、私产发生,剥削成为支配阶级存在的基础之时,就起了大变化了,精神劳动成为支配阶级的特权,体力劳动成为被压迫阶级的命运。支配阶级开始颠倒地去考察自己与被压迫阶级之间的相互关系,不是劳动者给他们以生活资料,反而是他们以生活资料给与劳动者,因此他们鄙视体力劳动,发生了唯心论的见解。消灭体力劳动与精神劳动的区别是消灭唯心论哲学的条件之一。

使唯心论哲学能够发展的社会根源,主要的还在于这种哲学意识它表现剥削阶级的利害。唯心论哲学在一切文化领域的优越,应该拿这个去说明。假如没有剥削阶级的存在,唯心论就会失掉它的社会根据。唯心论哲学之最后消灭,必须在阶级消灭与共产主义社会成立之后。

使唯心论能够发达、深化,并有能力同唯物论斗争,还须在人类的认识过程中找寻其本源。人类在使用概念来思考的时候,存在着溜到唯心论去的可能性。人类在思考时不能不使用概念,这就容易使我们的认识分裂为二方面:一方面,是个别的与特殊性质的事物,一方面是一般性质的概念(例如“延安是城”这个判断)。特殊和一般本来是互相联系不可分裂的,分裂就脱离了客观真理。客观真理是表现于一般与特殊之一致的。没有特殊,一般就不存在;没有一般,也不会有特殊。把一般同特殊脱离开来,即把一般当作客观实体看待,把特殊只当作一般之存在的形式,这就是一切唯心论所采用的方法。一切唯心论者都是拿意识、精神或观念来代替离开人的意识而独立存在的客观实体的。从这里出发,唯心论者便强调人类意识在社会实践中的能动性,他们不能指出意识受物质限制的这种唯物论的真理,却主张只有意识是能够动的,物质不过是不动的集合体。加上被阶级的本性所驱策,唯心论者便用一切方法把意识的能动性夸张起来,片面地发展了它,使这一方面在心智之中无限制地胀大,成为支配的东西,掩蔽着别一方面并使之服从,而把这一人工地胀大的东西确定为一般的宇宙观,以至化为物神或偶象。经济学上的唯心论,过分夸大交换中非本质的一方面,把供求法则提高到资本主义的根本法则。许多人看到科学在社会生活上发生了能动的作用,不知道这种作用受一定的社会生产关系所规定与限制,而作出科学是社会发动力的结论。唯心论历史家把英雄看成历史的创造者,唯心论政治家把政治看成万能的东西,唯心论军事家实行拼命主义的作战,唯心论革命家主张白朗基主义,×××\footnote{此处省略的为“蒋介石”。}说要复兴民族惟有恢复旧道德,都是过分夸张主观能动性的结果。我们的思维不能以一次反映来当作全体看的对象,而是构成一个具有接近于现实的,一切种类的无数色调的,生动的,认识之辩证法的过程。唯心论依据于思维的这种特性,夸大其个别方面,不能给过程以正确的反映,反把过程弄弯曲了。列宁说:“人类的认识不是直线的,而是曲线的。这一曲线之任何一段,都可以变为一段单独的完整的直线,这段直线就有引你陷入迷阵的可能。直线性和片面性是见树不见林和呆板固执性,主观主义和主观盲动性——这些就是唯心论的认识论的根源”\footnote{出自《谈谈辩证法问题》。参见《列宁全集》中文第2版,第55卷,第311页:“人的认识不是直线(也就是说,不是沿着直线进行的),而是无限地近似于一串圆圈、近似于螺旋的曲线。这一曲线的任何一个片断、碎片、小段都能被变成(被片面地变成)独立的完整的直线,而这条直线能把人们(如果只见树木不见森林的话)引到泥坑里去,引到僧侣主义那里去(在那里统治阶级的阶级利益就会把它巩固起来)。直线性和片面性,死板和僵化,主观主义和主观盲目性就是唯心主义的认识论根源。”}。哲学的唯心论是将认识的一个片段或一个方面,片面地夸张成为一种脱离物质脱离自然的神化的绝对体。唯心论就是宗教的教义,这是很对的。

马克思以前的唯物论(机械唯物论)没有强调思维在认识上的能动性,仅给以被动作用,把它当作反映自然的镜子看。机械唯物论对唯心论采取横暴的态度,不注意其认识论的根源,因此不能克服唯心论。只有辩证法唯物论,才正确地指出思维的能动性,同时又指出思维受物质的限制;指出思维从社会实践中发生,同时又能动地指导实践。只有这种辩证法的“知行合一”论,才能彻底地克服唯心论。

(四)唯物论发生与发展的根源

承认离意识而独立存在于外界的物质是唯物论的基础。这一基础是人类从实践中得到的。劳动生产的实践,阶级斗争的实践,科学实验的实践,使人类逐渐脱离迷信与妄想(唯心论),逐渐认识世界之本质,而到达于唯物论。

屈服于自然力之前,而只能使用简单工具的原始人类,不能说明周围的事变,因而求助于神灵,这就是宗教同唯心论的起源。

然而人类在长期生产过程中同周围的自然界接触,作用于自然界,变化着自然界,创造着衣食住用的东西,使之适合于人类的利益,使人类深信物质是客观地存在着。

人类在社会生活中,人同人之间互相发生关系与影响,在阶级社会中并且实行着阶级斗争。被压迫阶级考虑形势,估计力量,建立计划,在他们的斗争成功时,使他们确信自己的见解并不是幻想的产物,而是客观上存在着的物质世界的反映。被压迫阶级因为采取错误的计划而失败,又因为改正其计划而成功,使他们懂得只有主观的计划依靠于对客观世界的物质性与规律性的正确的认识,才能达到目的。

科学的历史给人类证明了世界的物质性及其规律性,使人类觉悟到宗教与唯心论的幻想之无用,而到达于唯物论的结论。

总之,人类的实践史——自然斗争史、阶级斗争史、科学史,在长久年月中,为了生活与斗争的必要,考虑物质的现实及其法则,证明了唯物论哲学的正确性,找到了自己斗争的思想工具——唯物论哲学。社会的生产发展越发进到高度,阶级斗争越发发展,科学认识越发暴露了自然的“秘密”,唯物论哲学就越发发展与巩固,人类便能逐渐从自然与社会的双重压迫下解放出来。

资产阶级在为了向封建阶级斗争的必要及无产阶级还没有威胁他们的时候,也曾经找到了并使用了唯物论作为自己斗争的工具,也曾经确信周围的事物是物质的产物,而不是精神的产物。直至他们自己变成了统治者,无产阶级的斗争又威胁着他们时,才放弃这个“无用”的工具,重新拿起另一个工具——哲学的唯心论。中国资产阶级的代言人戴季陶、吴稚晖,在1927年以前及其以后思想的变化——从唯物论到唯心论的变化,就是眼前的活证据。

资本主义的掘墓人——无产阶级,他们本质上是唯物论的。但由于无产阶级是历史上最进步的阶级,这就使得无产阶级的唯物论不同于资产阶级的唯物论,是更彻底更深刻的,只有辩证法的性质,没有机械论的性质。无产阶级吸收了人类全历史中的一切实践,经过他们的代言人与领导者——马克思、恩格斯之手,造成了辩证法唯物论,不但主张物质离人的意识而独立存在,而且主张物质是变化的,成为整个完整系统的崭新的世界观与方法论,这就是马克思主义的哲学。

\subsection{辩证法唯物论}

(一)辩证法唯物论是无产阶级革命的武器

这个问题在第一章中已经说过,这里再简单地说一点。

辩证法唯物论,是无产阶级的宇宙观。历史给予无产阶级以消灭阶级的任务,无产阶级就用辩证法唯物论作为他们斗争的精神上的武器,作为他们各种见解之哲学基础。辩证法唯物论这种宇宙观,只有当我们站在无产阶级的立场去认识世界的时候,才能够被我们正确地和完整地把握住;只有从这种立场出发,现实世界才能真正客观地被认识。这是因为一方面只有无产阶级才是最先进与最革命的阶级;又一方面,只有辩证法唯物论才是高度的和严密的科学性同彻底的和不妥协的革命性密切地结合着的一种最正确的和最革命的宇宙观和方法论。

中国无产阶级担负了经过资产阶级民主革命到达社会主义与共产主义的历史任务,必须采取辩证法唯物论作为自己精神的武器。如果辩证法唯物论——一种最正确最革命的宇宙观和方法论被中国共产党、及一切愿意站在无产阶级立场的广大革命份子所掌握,他们就能够正确地了解革命运动的发展变化,提出革命的任务,团结自己和同盟者的队伍,战胜反动的理论,采取正确的行动,避免工作的错误,达到解放中国与改造中国的目的。辩证法唯物论对于指导革命运动的干部人员尤属必修的科目,因为主观主义与机械观这两种错误的理论与工作方法,常常在干部人员中间存在着,因此常常引导干部人员违犯马克思主义,在革命运动中走入歧途。要避免与纠正这种缺点,只有自觉地研究与了解辩证法唯物论,把自己的头脑重新武装起来。

(二)旧的哲学遗产同辩证法唯物论的关系

现代的唯物论,不是过去各种哲学学说的简单的继承者,它是从反对过去统治哲学的斗争中,从科学解除其唯心论和神秘性的斗争中产生和成长起来的。马克思主义的哲学——辩证法唯物论,不但继承了唯心论的最高产物——黑格尔学说的成果,同时还克服了这一学说的唯心论,唯物地改造了他的辩证法。马克思主义又不但是一切过去唯物论发展的继续和完成,同时还是一切过去唯物论的狭隘性之反对者,即机械的直觉的唯物论(主要的是法国唯物论与费尔巴哈唯物论)之反对者。马克思主义的哲学——辩证法唯物论,继承了过去文化之科学的遗产,同时又给此种遗产以革命的改造,形成了一种历史上从来没有过的最正确最革命和最完备的哲理的科学。

中国在1919年五四运动以后,随着中国无产阶级自觉地走上政治舞台及科学水平之提高,发生了与发展着马克思主义的哲学运动。然而在它的第一时期,中国的唯物论思潮中唯物辩证法的了解还很微弱,受资产阶级影响的机械唯物论,和德波林派的主观主义风气占着主要的成分。1927年革命失败以后,马克思列宁主义的了解进了一步,唯物辩证法的思想逐渐发展起来。到了最近,由于民族危机与社会危机的严重性,也由于苏联哲学清算运动的影响,在中国思想界发展了一个广大的唯物辩证法运动。这个运动,目前虽还在青年的阶段上,然从其广大的姿态来看,它将随着中国与世界无产阶级同革命人民的革命斗争之发展,以横扫的阵势树立自己的权威,指导中国革命运动,勇往迈进,定下中国无产阶级领导中国革命进入胜利之途的基础。

由于中国社会进化的落后,中国今日发展着的辩证法唯物论哲学思潮,不是从继承与改造自己哲学的遗产而来的,而是从马克思列宁主义的学习而来的。然而要使辩证法唯物论思潮在中国深入与发展下去,并确定地指导中国革命向着彻底胜利之途,便必须同各种现存的反动哲学作斗争,在全国思想战线上树立批判的旗帜,并因而清算中国古代的哲学遗产,才能达到目的。

(三)辩证法唯物论中宇宙观和方法论的一致性

辩证法唯物论是无产阶级的宇宙观,同时又是无产阶级认识周围世界的方法和革命行动的方法,它是宇宙观和方法论的一致体。唯心论的马克思主义修正派认为辩证法唯物论的全部实质只在于它的“方法”。他们把方法从一般哲学的宇宙观割裂开来,把辩证法从唯物论割裂开来。他们不了解马克思主义的方法论——辩证法,不是如同黑格尔一样的唯心的辩证法,而是唯物的辩证法,马克思主义的方法论是丝毫也不能离开它的宇宙观的,另一方面机械唯物论者却又把马克思主义的哲学看作一般哲学的宇宙观,割去了它的辩证法,而且认为这种宇宙观就是机械的自然科学之各种结论。他们不了解马克思主义的唯物论不是简单的唯物论,而是辩证法的唯物论。对于马克思主义哲学之这两种割裂的看法都是错误的,辩证法唯物论是宇宙观和方法论的一致体。

(四)唯物辩证法的对象问题——唯物辩证法是研究什么的?

列宁把(作为马克思主义的哲理科学看的)唯物辩证法看做关于客观世界的发展法则及(在辩证法各范畴中反映这客观世界的)认识的发展法则的学问。他说:论理学不是关于思维的外在形式的学问,而是关于一切物质的,自然的,及精神的事物之发展法则的学问,即关于世界的一切具体内容及其认识之发展法则的学问。换言之,论理学是关于世界认识之历史的总计、总和、结论。列宁虽然把作为一般的科学方法论看的唯物辩证法的意义强调起来,然而这是因为辩证法系由世界认识的历史中得出来的结论。因此他说:“辩证法就是认识的历史”\footnote{参见《黑格尔〈逻辑学〉一书摘要》中《第2版序言》:“逻辑不是关于思维的外在形式的学说,而是关于“一切物质的、自然的和精神的事物”的发展规律的学说,即关于世界的全部具体内容的以及对它的认识的发展规律的学说,即对世界的认识的历史的总计、总和、结论。”(《列宁全集》中文第2版,第55卷,第77页)}。

上述列宁对于当作科学看的唯物辩证法及其对象所给与的定义,他的意思是说:第一、唯物辩证法与其他任何科学一样,有它的研究对象,这个对象便是自然、历史和人类思维之最一般的发展法则。并且研究的时候,唯物辩证法的任务,不是从头脑里想出存在于各现象间的关联,而是要在各现象本身中观察出它们之间的关系来。列宁的这种见解同少数派唯心论者把(事实上离开了具体科学及具体知识的)范畴的研究当做唯物辩证法的对象之间,存在着根本的区别,因为少数派唯心论者企图建立一个从认识历史社会科学和自然科学的现实发展中游离了的各范畴的哲学体系,这样他们就事实上放弃了唯物辩证法。第二、各个科学分科(数学、力学、化学、物理学、生物学、经济学及其他自然科学、社会科学),是研究物质世界及其认识之发展的各个方面。因此各个科学的法则是狭隘的,片面的,被各个具体研究领域所限制了的。唯物辩证法则不然,它是一切具体科学中的一切有价值的一般内容,及人类的其它一切科学认识之总计、结论、加工和普遍化。这样,唯物辩证法的概念、判断和法则,是极其广泛的(包含着一切科学的最一般的法则,因此也包含着物质世界的本质的)各种规律性和规定,这是一方面。在这方面,它是宇宙观。另一方面,唯物辩证法是从一切空想、僧侣、主义、和形而上学解放出来的真正科学认识上的论理学和认识论的基础,因此它同时又是研究具体科学的唯一确实的、有客观正确性的方法论。我们说唯物论辩证法或辩证法唯物论是宇宙观和方法论的一致体,在这里更加明白了。这样对于否认哲学存在权的马克思主义哲学的歪曲者和庸俗化者的错误也可以懂得了。

关于哲学对象问题,马克思、恩格斯和列宁,都反对使哲学脱离实在的现实,使哲学变为某种独立实质的东西。指出了那根据实在生活和实在关系的分析而生长出来的哲学之必然性,反对单单以论理观念和论理观念的自然做研究的对象,如同形式论理学及少数派唯心论的那种干法。所谓根据实在生活和实在关系的分析生长出来的哲学就是唯物辩证法这种论发展的学说。马克思,恩格斯和列宁,都解说唯物辩证法为论发展的学说。恩格斯称唯物辩证法为“论自然社会及思维之一般的发展法则”\footnote{出自《自然辩证法》中《[辩证法作为科学]•辩证法》。参见《马克思恩格斯选集》中文第3版,第3卷,第907页:“自然界、社会和思维的发展的一个一般规律”。}的学说。列宁把唯物辩证法看作“最多方面的,内容最丰富的,和最深刻的发展学说。”\footnote{出自《卡尔•马克思(传略和马克思主义概述)》中《辩证法》。参见《列宁全集》中文第2版,第26卷,第55页:“最全面、最富有内容、最深刻的发展学说。”}他们都认为在这种学说以外的其他一切哲学学说所述一切发展原则的公式,概属狭隘的无内容的“截去了自然和社会之实际发展过程的东西”\footnote{出自《卡尔•马克思(传略和马克思主义概述)》中《辩证法》。参见《列宁全集》中文第2版,第26卷,第55页:“把自然界和社会的实际发展过程(往往伴有飞跃、剧变、革命)弄得残缺不全。”}(列宁)。至于唯物辩证法之所以被称为最多方面的,内容最丰富的和最深刻的发展学说的原故,乃是因为唯物辩证法是最多方面地和最丰富地、最深刻地反映了自然和社会变化过程中的矛盾性和飞跃性,而不是因为别的东西。

在哲学对象问题中还要解决一个问题,就是辩证法、论理学及认识论的一致性的问题。

列宁着重指出辩证法、论理学及认识论的同一性,说这是“极其重要的问题”,说“三个名词是多余的,它们只是一个东西”\footnote{出自《黑格尔辩证法(逻辑学)的纲要》。参见《列宁全集》中文第2版,第55卷,第290页:“不必要三个词:它们是同一个东西。”},根本反对那些马克思主义修正派把三者当做完全各别独立的学说去处理的那种干法。

唯物辩证法是唯一科学的认识论,也是唯一科学的论理学。唯物辩证法研究吾人对外界认识的发生及发展,研究由不知到知,由不完全的知到更完全的知的转移,研究自然及社会的发展法则在人类头脑中日益深刻和日益增多的反映,这就是唯物辩证法与认识论的一致。唯物辩证法研究客观世界最一般的发展法则,研究客观世界最发展的姿态在思维中的反映形态。这就是唯物辩证法研究现实事物的各过程及各现象的发生、发展,消灭及相互转化的法则,同时又研究反映客观世界发展法则的人类思维的形态,这就是唯物辩证法与论理学的一致。

要彻底了解辩证法、论理学、认识论三者为什么是一个东西,我们看下面唯物辩证法怎样解决关于论理的东西与历史的东西之相互关系这个问题,就可以明白了。

恩格斯说:“对于一切哲学家的思维方法来说,黑格尔思维方法的长处就在于横亘在根底面的极其丰富的历史感,他的形式虽说是抽象的唯心论的,然而他的思想的发展却常常是与世界历史的发展平行着的。并且历史原来就是思想的检证。”\footnote{出自《卡尔•马克思〈政治经济学批判。第一分册〉》第2部分。参见《马克思恩格斯选集》中文第3版,第2卷,第12页:“黑格尔的思维方式不同于所有其他哲学家的地方,就是他的思维方式有巨大的历史感作基础。形式尽管是那么抽象和唯心,他的思想发展却总是与世界历史的发展平行着,而后者按他的本意只是前者的验证。”}“历史常常在飞跃地错杂地进行着。因为有这种情形,所以假若常常要依从历史的话,不但要注意许多不重要的材料,而且会不得不使思想行程中断。这时唯一适当的方法,就是论理的方法。然而这—论理的方法根本仍然是历史的方法,不过舍去了它那历史的形态与偶然性罢了”\footnote{出自《卡尔•马克思〈政治经济学批判。第一分册〉》第2部分。参见《马克思恩格斯选集》中文第3版,第2卷,第13页:“历史常常是跳跃式地和曲折地前进的,如果必须处处跟随着它,那就势必不仅会注意许多无关紧要的材料,而且也会常常打断思想进程;并且,写经济学史又不能撇开资产阶级社会的历史,这就会使工作漫无止境,因为一切准备工作都还没有做。因此,逻辑的方式是唯一适用的方式。但是,实际上这种方式无非是历史的方式,不过摆脱了历史的形式以及起扰乱作用的偶然性而已。”}。这种“论理发展与历史发展一致”的思想,是被马克思、恩格斯、列宁充分注意了的。“论理学的范畴,是外在的与活动之无数个别性的简约”\footnote{出自《黑格尔〈逻辑学〉一书摘要》中《第2版序言》。参见《列宁全集》中文第2版,第55卷,第75页:“逻辑的范畴是‘外部存在和活动的’‘无数’‘细节’的简化”。}。“范畴就是分离的阶段,帮助我们去认识这一个网和网的结节点的”\footnote{出自《黑格尔〈逻辑学〉一书摘要》中《第2版序言》。参见《列宁全集》中文第2版,第55卷,第78页:“范畴是区分过程中的梯级,是帮助我们认识和掌握自然现象之网的网上纽结。”}。“人的实践活动,把人类的意识几十亿次反复不息地应用到各种各样的论理学式子里面,这样,这些式子就得到了所谓公理的意义了”\footnote{出自《黑格尔〈逻辑学〉一书摘要》中《主观逻辑或概念论•第2篇 客观性》。参见《列宁全集》中文第2版,第55卷,第160页:“人的实践活动必须亿万次地使人的意识去重复不同的逻辑的式,以便这些式能够获得公理的意义。”}。“人类的实践,反复了几十亿次,才当做论理的式子固定在人类意识中。这些式子,都有着成见的永续性,因为是反复了几十亿次的结果,才有着公理的性质”\footnote{出自《黑格尔〈逻辑学〉一书摘要》中《主观逻辑或概念论•第3篇 观念》。参见《列宁全集》中文第2版,第55卷,第186页:“人的实践经过亿万次的重复,在人的意识中以逻辑的式固定下来。这些式正是(而且只是)由于亿万次的重复才有着先入之见的巩固性和公理的性质。”}。上述列宁的那些话,指明唯物辩证法的论理学的特点,不象形式论理学那样,把它的法则和范畴看成空虚的,脱离内容而独立的,对于内容无关心的形式,也不象黑格尔那样,把他看成脱离物质世界而独立发展的观念要素,而是把它当做反映到和移植到我们头脑里,并且经由头脑加工制造过的,物质运动的表现去处理。黑格尔立脚在存在和思维的同一性上,把辩证法、论理学和认识论的同一性当做唯心论的同一性去处理。反之,马克思主义的哲学里,辩证法、论理学和认识论的同一性,是建立在唯物论基础上的。只有用唯物论解决存在与思维的关系问题,只有站在反映论的立场上,才能使辩证法、论理学和认识论的问题得到彻底的解决。

用辩证法唯物论去解决论理的东西和历史的东西的相互关系的最好的模范,首先要算马克思的《资本论》。《资本论》中包含了资本主义社会的历史发展,同时又包含了这一社会的论理发展。《资本论》所分析的,是那把资本主义社会的发生、发展及消灭反映出来的各经济范畴的发展的辩证法。这问题之解决的唯物论性质,在于他以物质的客观历史做基础,在于把概念和范畴当做这一现实历史的反映。资本主义的理论\footnote{原文如此,恐为“论理”(即“逻辑”)误。}和历史的一致,资本主义的社会的论理学和认识论的一致,模范地表现在《资本论》里面,我们可以从它懂得一点辩证法、论理学和认识论一致的门径。

以上是辩证法唯物论的对象问题。

(五)物质论

马克思主义继续和发展哲学中的唯物论路线,正确地解决了思维与存在的关系问题,即彻底唯物地指出世界的物质性,物质的客观实在性,和物质对于意识的根源性(或意识对于存在的依赖关系)。

承认物质对于意识的根源性是以世界的物质性及其客观存在为前提的。隶属于唯物论营垒的第一个条件就承认物质世界离人的意识而独立存在——人类出现以前它就存在,人类出现以后也是离开人的意识而独立存在的。承认这一点是一切科学研究的根本前提。

拿什么来证明这一点呢?证据是多得很的。人类时刻同外界接触;还须用残酷的手段去对付外界(自然界同社会)的压迫和反抗;还不但应该而且能够克服这些压迫和反抗——所有这些在人类社会的历史发展中表现出来的人类社会实践的实在情形,就是最好的证据。经过了万里长征的红军,不怀疑经过地区连同长江大河雪山草地以及和它作战的敌军等等的客观存在,也不怀疑红军自己的客观存在,中国人不怀疑侵略中国的日本帝国主义同中国人自己的客观存在,抗日军政大学的学生也不怀疑这个大学和学生自己的客观存在。这些东西都是客观地离开我们意识而独立存在的物质的东西,这是一切唯物论的基本观点,也就是哲学的物质观。

哲学的物质观同自然科学的物质观是不相同的。如果说哲学的物质观在于指出物质的客观存在,所谓物质就是说的离开人的意识而独立存在的整个世界(这个世界作用于人的感官,引起人的感觉,并在感觉中得到反映)。那末这种说法是永远不起变化的,是绝对的。自然科学的物质观则在于研究物质的构造,例如从前的原子论,后来的电子论等等,这些说法是随着自然科学的进步而变化的,是相对的。

根据辩证法唯物论的见地去区别哲学的物质观与自然科学的物质观,是彻底贯彻哲学的唯物论方向之必须条件。在向唯心论和机械唯物论作斗争方面,有着重要的意义。

唯心论者根据电子论的发见轰传物质消灭的谬说,他们不知道关于物质构造之科学知识的进步,正是证明辩证法唯物论的物质论之正确性。因为表现在旧的物质概念中的某些物质属性(重量硬度,不可入性,惰性等等),经过现代自然科学的发现,即电子论的发现,证明这些属性仅存在于某几种物质形态中,而在其它物质形态中则不存在,这种事实,破除了旧唯物论对于物质观念的片面性与狭隘性,而对于承认世界的物质性及其客观存在之辩证法唯物论的物质观,却恰好证明其正确。原来辩证法唯物论的物质观,正是以多样性去看物质的世界的统一,就是物质多样性的统一。这种物质观,对于物质由一形态转化到另一形态之永久普遍的运动变化这一种事实,丝毫也没有矛盾。以太、电子、原子、分子、结晶体、细胞、社会现象、思维现象——这些都是物质发展的种种阶段,是物质发展史中的种种暂时形态。科学研究的深入,各种物质形态的发现(物质多样性的发现),只是丰富了辩证法唯物论的物质观的内容,那里还会有什么矛盾?区别哲学的物质观同自然科学的物质观是必要的,因为二者有广狭之别然而是不相矛盾的,因为广义的物质包括了狭义的物质。

辩证法唯物论的物质观,不承认世界有所谓非物质的东西(独立的精神的东西)。物质是永久与普遍存在的,不论在时间与空间上都是无阻的,如果说世界上有一种“从来如此”与“到处如此”的东西(就其统一性而言),那就是哲学上的所谓客观存在的物质。用彻底的唯物论见地(即唯物论辩证法见地)来看意识这种东西,那末所谓意识不是别的,它是物质运动的一种形态,是人类物质头脑的一种特殊性质,是使意识以外的物质过程反映到意识之中来的那种物质头脑的特殊性质。由此可知,我们区别物质同意识并把二者对立起来是有条件的。就是说:只在认识论的见地有意义。因为意识或思维只是物质(头脑)的属性,所以认识与存在的对立就是认识的物质同被认识的物质的对立,不会多一点。这种主体同客体的对立,离开认识论领域就毫无意义。假如在认识论以外还把意识同物质对立起来,就无异于背叛唯物论。世界上只有物质同它的各种表现,主体自身也是物质的,所谓世界的物质性(物质是永久与普遍的),物质的客观实在性与物质对于意识的根源性,就是这个意思。一句话,物质是世界的一切。“一统归于司马懿”,我们说“一统归于物质”。这就是世界的统一原理。

以上是辩证法唯物论的物质论。

(六)运动论(发展论)

辩证法唯物论的第一个基本原则在于它的物质论,即承认世界的物质性、物质客观实在性和物质对于意识的根源性,这种世界的统一原理,在前面物质论中已经解决了。

辩证法唯物论的第二个基本原则在于它的运动论(或发展论),即承认运动是物质存在的形式,是物质内在的属性,是物质多样性的表现,这就是世界的发展原理。世界的发展原理同上述世界的统一原理相结合,就成为辩证法唯物论的整个的宇宙观。世界不是别的,就是无限发展的物质世界(或物质世界是无限发展的)。

辩证法唯物论的运动观,对于(一)离开物质而思考运动,(二)离开运动而思考物质,(三)物质运动的简单化,都是不能容许的,辩证法唯物论的运动论,就是同这些唯心的、形而上学的、及机械的观点作明确而坚决的斗争建立起来的。

辩证法唯物论的运动论,首先是同哲学的唯心论及宗教的神道主义相对立的。一切哲学的唯心论及宗教的神道主义的本质,在于它们从否认世界的物质统一性出发,设想世界的运动及发展是没有物质的、或在最初是没有物质的、而是精神作用或上帝神力的结果。德国唯心论哲学家黑格尔认为现在的世界是从所谓“世界理念”发展而来的,中国的周易哲学及宋、明理学都作出唯心论的宇宙发展观。基督教说上帝创造世界,佛教及中国一切拜物教都把宇宙万物的运动发展归之于神力。所有这些离物质而思考运动的说法都和辩证法唯物论根本不相容。不但唯心论与宗教,就是马克思以前的一切唯物论及现在一切反马克思主义的机械唯物论,当他们说到自然现象时,是唯物论的运动论者,但一说到社会现象时,就无不离开物质的原因而归着于精神的原因了。

辩证法唯物论坚决驳斥所有这些错误的运动观,指出他们的历史限制性——阶级地位的限制与科学发展程度的限制,而把自己的运动观建设在以无产阶级立场及最发达的科学水准为基础的、彻底的唯物论上面。辩证法唯物论首先指出运动是物质存在的形式、是物质内在的属性(不是由外力推动的),设想没有物质的运动,同设想没有运动的物质是一样不可思议的事。把唯物的运动观同唯心的及唯神的运动观尖锐地对立着。

离开运动而思考物质,则有形而上学的宇宙不动论或绝对均衡论,他们认为物质是永远不变的,在物质中没有发展这回事,认为绝对的静止是物质的一般状态或原始状态。辩证法唯物论坚决反对这种意见,认为运动是物质存在的最普遍的形式,是物质内在的不可分离的属性。一切的静止与均衡仅有相对的意义,而运动则是绝对的。辩证法唯物论承认一切物质形态均有相对的静止或均衡之可能,并认为这是辨别物质,因而亦即辨别生命的最重要条件(恩格斯)。但认为静止或均衡只是运动的要素之一,是运动的一种特殊情况。离开运动而考察物质的错误,就在于把这种静止要素或均衡要素夸张起来,把它掩蔽了并代替了全体,把运动的特殊情况一般化、绝对化起来。中国古代形而上学思想家爱说的一句话:“天不变,道亦不变”,就是这样的宇宙不动论。他们也承认宇宙及社会现象的变动,但否认其本质的变动,在他们看来,宇宙及社会的本质是永远不变动的。他们之所以如此,主要的原因在于他们的阶级限制性,封建地主阶级如果也承认宇宙及社会的本质是运动与发展的,就无异在理论上宣布他们自己阶级的死刑。一切的反动势力,他们的哲学都是不动论。革命的阶级同民众,却眼睛看到了世界的发展原理,因而主张改造这个社会及世界,他们的哲学是辩证法唯物论。

此外辩证法唯物论也不承认简单化的运动观,就是说把一切的运动都归结到一种形式上去,即归结到机械的运动,这是旧唯物论宇宙观的特点。旧唯物论(十七八世纪的法国唯物论,十九世纪的德国费尔巴哈唯物论)也承认物质的永久存在和永久运动(承认运动的无限性),但仍然没有跳出形而上学的宇宙观,不去说他们在社会论上的见解依然是唯心论的发展观。就在自然论上,也把物质世界的统一,归结到某种片面的属性,即归结到运动的一个形态——机械的运动,这种运动的原因在外力,象机械一样,由外力推之而运动。他们不从本质上,也不从内部原因上去说明物质或运动、本质或关系的一切多样性,而从单纯的外面的发现形式上从外力原因上去说明它,这样在实际上就失掉了世界的多样性。他们把世界一切的运动,都解作场所的移动与数量的增减。物质某一瞬间在某一场所,另一瞬间则在另一场所,这样就叫做运动。如果有变化,也只是数量增减的变化,没有性质的变化,变化是循环的,是反复产生同一结果的。辩证法唯物论与此相反,不把运动看作单纯的场所移动及循环运动,而把它看作无限的质的多样性,看作由一形态向他一形态的转化,世界物质的统一和物质的运动,便是世界物质无限多样性的统一与运动。恩格斯说:“运动的一切高级形态必然同力学的(外的或分子的)运动形态结合着,例如:如果没有热和电气的变化,化学的作用就不可能,如果没有力学的(分子的)热量的、电气的、化学的变化等等,有机的生命也不可能,这当然是不能否认的。然而如果只有某些低级运动形态的存在,是决不能包括各种状态中主要形态的物体的”。这话是千真万确地合于事实。即使就单纯机械运动而论,也不能从形而上学的观点去解释它。须知一切运动形态都是辩证法的,虽然它们之间的辩证法内容的深度与多面性有着很大的差异。机械运动仍然是辩证法的运动,所谓物体某一瞬间“在”某处,其实是同时“在”某处,同时又不在某处,所谓“在”某处,所谓“不动”,仅是运动的一种特殊情况,它根本上依然是在运动,物体在被限制着的时间内和被限制着的空间内运动着。物体总是不绝地克服这种限制性跑出这种一定的有限的时间及空间的界限以外去成为不绝的运动之流。而且机械运动只是物质的运动形态之一,在实在的现实世界中,没有它的绝对独立的存在,它总是联系于别种运动形态的。热、化学的反应,光、电气,一直到有机现象与社会现象,都是质地上特殊的物质运动形态。十九世纪与二十世纪交界时期的自然科学的划时代的大功劳,就在于发现了运动转化法则,指出物质的运动总是由一形态转化成为另一形态,这样的转化的新形态是与旧形态本质上不同的。物质所以转化的原因不在外部而在内部,不是由于外部机械力的推动,而是由于内部存在着性质不同的互相矛盾的两种因素相争相斗推动着物质的运动与发展。由于这个运动转化法则的发现,辩证法唯物论就能够把世界的物质统一原理扩大到自然与社会的历史上去,不但把世界当作永远运动的物质去考察,而且把世界当做由低级形态到高级形态的无限前进运动的物质去考察,即把世界当作发展,当作过程去考察,做一句话来说:“统一的物质世界是一个发展的过程”。这样就把旧唯物论的循环论击破了。辩证法唯物论深刻地多方面地观察了自然及社会的运动形态,认为当作全体看的世界之发展过程是永久的(无始无终的)。但同时各个历史地进行的具体的运动形态又是暂时的(有始有终的),就是说它是在一定的条件下发生,并在一定的条件下消灭的。认为世界的发展过程由低级的运动形态生出高级的运动形态,表示了它的历史性与暂时性,但同时任何一个运动形态无不是处在永久的长流中(无始无终的长流中)。依据着对立斗争的法则(自己运动的原因),使每一运动形态总是较之先行形态进到了高一级的阶段,它是向前直进的,但同时就各个运动形态来说(就各个具体的发展来说),却也会发生转向运动或后退运动,前进运动同后退运动相结合,在全体上就成为复杂的螺旋运动,认为新的运动形态是作为旧的运动形态的对立物(反对物)而发生的,但同时新的运动形态又必然保存着旧的运动形态中的许多要素,新东西是从旧东西里面生长出来的。认为事物的新形态、新性质、新属性的出现,是由连续性的中断即经过冲突和破局而飞跃地产生的,但同时事物的连结和相互关系又决不会绝对破坏。最后辩证法唯物论认为世界无穷尽(无限),不但就其全体来看是这样的,同时就其局部来看也是这样的,电子不是同原子分子一样表现着一个复杂而无穷尽的世界么?

物质运动的根本形态,又规定根本的自然科学与社会科学各科目。辩证法唯物论把世界的发展当作无机界经过有机界而达到最高物质运动形态(社会)的一个前进运动去考察,这一运动形态的从属关系就成了和它相应的科学(无机界科学,有机界科学,社会科学的从属关系的基础)。恩格斯说:“各种分类的科学是把特定的运动形态或相互关联相互推移的一联的运动形态拿来分析,因此科学的分类就在于要依从着运动的固有顺序去把各个运动分类排列起来,仅在这一点来说,分类才有意义。”\footnote{出自《自然辩证法》中《物质的运动形式以及各门科学的联系》。参见《马克思恩格斯选集》中文第3版,第3卷,第943页:“每一门科学都是分析某一个别的运动形式或一系列互相关联和互相转化的运动形式的,因此,科学分类就是这些运动形式本身依其内在序列所进行的分类、排序,科学分类的重要性也正在于此。}

整个世界包括人类社会在内,是采取质地不相同的各种形式的物质的运动,因此也就不能忘记物质运动的各种具体形式这个问题。所谓“物质一般”与“运动一般”是没有的,世界上只有各种不同形式的具体的物质或运动。“物质和运动这些字眼只是一些简写的名词,在这些名词中,我们依照它们的共同特性是把各种不同的被感觉的事物一概包括在内的。”\footnote{出自《自然辩证法》中《[辩证法作为科学]•[认识]》。参见《马克思恩格斯选集》中文第3版,第3卷,第939页:“‘物质’和‘运动’这样的词无非是简称,我们就用这种简称把可感知的许多不同的事物依照其共同的属性概括起来。”}

以上就是辩证法唯物论的世界运动论或世界发展原理。这个学说是马克思主义哲学的精髓,是无产阶级的宇宙观与方法论,无产阶级及一切革命的人们如果拿着这个彻底科学的武器,他们就能够理解这个世界并改造这个世界。

(七)时空论

运动是物质存在的形式,空间和时间也是物质存在的形式,运动的物质存在于空间和时间中,并且物质的运动本身是以空间和时间这两种物质存在的形式为前提的。空间和时间不能与物质相分离。“物质存在于空间”这句话,是从物质本身具有伸张性,物质世界是内部存在着伸张性的世界,不是说物质被放在一种非物质的空虚的空间中。空间和时间都不是独立的非物质的东西,也不是我们感觉性的主观形式。它们是客观物质世界存在的形式。它们是客观的,不存在物质以外,物质也不存在于它们以外。

把空间和时间看作物质存在的形式的这种见解,是彻底的唯物论的见解。这种时空观,同下列几种唯心论的时空观是根本相反的:(一)康德主义的时空观,认时间和空间不是客观的实在,而是人类的直觉形式;(二)黑格尔主义的时空观,认发展着的时间和空间的概念,日益接近于绝对观念;(三)马赫主义的时空观,认时间和空间是“感觉的种类”,“使经验和谐化的工具”。所有这些唯心论观点,都不承认时间和空间的客观实在性,都不承认时间和空间的概念在自身发展中反映着物质存在的形式。这些错误理论,都被辩证法唯物论一个一个地驳翻了。

辩证法唯物论在时空问题上,不但要同上述那些唯心论观点作斗争,而且要同机械唯物论作斗争。特别显著的是牛顿的(机械论),他把空间看做同时间无关系的不动的空架子,物质被安置到这种空架子里面去。辩证法唯物论反对这种机械论,指出我们的时空观念是在发展的。“世界上除了运动的物质以外便没有别的东西,而运动的物质若不在空间和时间中便无运动之可能。人类关于空间和时间的概念是相对的,但是这些相对的概念积集起来就成为绝对的真理。这些相对的概念不断发展着,循着绝对真理的路线而前进,日益走近于绝对真理。人类关于空间时间概念的变动性,始终不能推翻二者的客观实在性,这正和关于物质的运动形式及其组织之科学知识的变动性,不能推翻外界的客观实在性,是一样的。”\footnote{出自《唯物主义和经验批判主义》第3章第5节。参见《列宁全集》中文第2版,第18卷,第180页:“世界上除了运动着的物质,什么也没有,而运动着的物质只能在空间和时间中运动。人类的时空观念是相对的,但绝对真理是由这些相对的观念构成的;这些相对的观念在发展中走向绝对真理,接近绝对真理。正如关于物质的构造和运动形式的科学知识的可变性并没有推翻外部世界的客观实在性一样,人类的时空观念的可变性也没有推翻空间和时间的客观实在性。”}

以上是辩证唯物的时空论。

(八)意识论

辩证唯物论认意识是物质的产物,是物质发展之一形式,是一定物质形态的特性。这种唯物主义同历史主义的意识论是和一切唯心论及机械唯物论对于这个问题的观点根本相反的。

依照马克思主义的见解,意识的来源,是由无意识的无机界发展到具有低级意识形态的动物界,再发展到具有高级意识形态的人类。高级意识形态不但同生理发展中的高级神经系统不可分离,而且同社会发展中的劳动生产不可分离。马克思、恩格斯曾经着重指出意识对物质生产发展的依赖关系,和意识同人类言语发展的关系。

所谓意识是一定物质形态的特性,这种物质形态就是组织复杂的神经系统,这样的神经系统只能发生于自然界进化的高级阶段上。整个无机界、植物界和低级的动物界,都没有认识在他们内面或外面发生着的那些过程的能力,它们是没有意识的。仅在有高级神经系统的动物体,才具有认识过程的能力,即具有自内反映或领悟这些过程的能力。吾人神经系统中的客观生理过程,是同它之内部取意识形式的主观表现相随而行的。凡就本身论是客观的东西,是某种物质过程,它对于具有头脑的实体却同时又是主观的心理的行为。

特殊思想实质的精神是没有的,有的只是思想的物质——脑子。这种思想的物质是有特别质地的物质,这种物质随着人类社会生活中言语的发展而达到高度的发展。这种物质具有思想这一种特殊性质,这是任何别的物质所不具备的。

然而庸俗唯物论者却认思想是脑子分泌出来的物质,这种见解歪曲了我们关于这个问题的观念。须知思想感情和意志的行为,不是具有重量和伸张性的东西,意识同重量伸张性等是同一物质之不同的性质。意识是运动的物质之内部状态,是反映着在运动的物质中所发生的生理过程的特殊性质。这种特殊性,同客观的神经作用过程不可分离,但又不与这过程相同,把这二者混同起来,推翻意识的特殊性,这就是庸俗唯物论的观点。

和这同样冒牌的马克思主义的机械论,附和心理学中某些资产阶级的左翼学派的见解,实质上也完全推翻了意识。他们把意识解作理化的生理的过程,认为高级实体的行为之研究,可以由客观生理学和生物学的研究去执行。他们不了解意识的本质之质的特殊性,看不到意识是人类社会实践的产物。他们把客体和主体之具体历史的一致,代之以主客的等同,代之以片面的机械的客观的世界。这种把意识混同于生理过程的观点,无异取消了思维与存在关系这个哲学中的根本的问题。

孟塞维克的唯心论企图用一种妥协理论去代替马克思主义的意识论,把唯物论同唯心论调和起来,他们拿客观主义同主观主义的原则,而这种原则既非机械的客观主义,也非唯心的主观主义,而是客观和主观之具体历史的一致。

可是还有怀疑论,这就是普列汉诺夫关于意识问题的物活论的见解。在他的“石子也是有意识的”一句名言中充分表现着。照他的意见,意识不是发生于物质发展过程中的,而是最初就存在于一切物质的。石子的、低级有机体的和人的意识之间,仅仅在于程度上的区别。这种反历史的见解,对于辩证唯物论认为意识是最后发生的具备着质的特殊性的见解,也是根本相反的。

只有辩证唯物论的意识论才是意识问题上的正确的理论。

(九)反映论

做一个彻底的唯物论者,单承认物质对于意识的根源性是不够的,还须承认意识对于物质的可认识性。

关于物质能否被认识的问题,是一个复杂的问题,是一切过去哲学都觉得无力对付的问题,只有辩证法唯物论能够给予正确的解决。在这个问题上,辩证法唯物论的立场既同不可知论相反,又同直率的实在论不同。

休谟同康德的不可知论,把认识的主体隔离开来,认为越出本体的界限是不可能的,“自在之物”和它的形象之间存在着不可跳过的深沟。

马赫主义的直率实在论,则把客体同感觉等同起来,认为真理在感觉中就已经成就了完成的形态。同时,他们不但不了解感觉是外界作用的结果,而且不了解主体在认识过程中的积极作用,即外界作用在主体的感觉机关和思想的脑子中所做的改造工夫(取印象和概念的形式表现出来)。

只有辩证法唯物论的反映论,肯定地答复了可认识性问题,成为马克思主义认识论的“灵魂”。根据这一理论,指明我们的印象和概念不但被客观事物所引起,而且还反映客观事物。指明印象和概念,既不象唯心论者所说的那样,是主体自动发展的产物,也不是不可知论者所说的那样,是客观事物的标符,而是客观事物的反映、照象和样本。

客观的真理是不依靠主体而独立存在的,它虽然反映在我们的感觉和概念中,但不是一下子就取完成的形态,而是一步一步完成的,认为客观真理在感觉中就已经取着完成形态,而被我们获得的那种直率实在论的见解是一种错误的见解。

客观真理在我们感觉和概念中虽不是一次就取完成的形态,然而不是不能认识的。辩证唯物论的反映论,反对不可知论的见解,认为意识是能够在认识过程中反映客观真理的。认识过程是一个复杂的过程,在这个过程中,当未被认识的“自在之物”,反映到我们的感觉印象、概念上来时,就变成“为我之物”了。感觉和思维,并不是如同康德所说的那样,把我们同外界隔离开来,而是把我们同外界联系起来的。感觉和思维就是客观外界的反映。思想的东西(印象和概念)并非别的,不过是“人类头脑中所转现出来和改造过来的物质的东西”(马克思)。在认识过程中,物质世界是愈走而愈接近地愈精确地愈多方面地和愈深刻地反映在我们的认识中。向着马赫主义和康德主义作两条战线的斗争,揭破直率实在论和不可知论的错误,是马克思主义认识论的任务。

唯物辩证法的反映论认为我们认识客观世界的能力是无限度的,这和不可知论者认为人的认识能力是有限度的那种见解根本相反。但我们之接近绝对真理,却每一次有其历史上的确定界限。列宁这样说:吾人知识之接近客观的绝对真理,是历史地有限度的。但是这一真理的存在是绝对的,我们不断地向真理接近也是绝对的。图画的外形是历史地有条件的,但这张图画描绘着客观上存在的模型则是绝对的,我们承认人的认识受历史条件的限制,真理是不能一次获得的。但我们不是不可知论者,我们又承认真理能够完成于人类认识的历史运动中。列宁还说:对于自然人类思想中的反映,不要死板板地或绝对地去了解他,认识不是无运动与无矛盾的,认识是处于永久的运动过程中,“即矛盾之发生和解决的永久的运动过程中”\footnote{出自《黑格尔〈逻辑学〉一书摘要》中《主观逻辑或概念论•第3篇 观念》。参见《列宁全集》中文第2版,第55卷,第165页:“矛盾的发生和解决的永恒过程中”。}。认识运动时一个复杂的充满着矛盾与斗争的运动,这就是辩证唯物论的认识论之见解。

一切哲学在认识论上的反历史的观点,都不把认识当作过程看待,因此都带着狭隘性。感觉主义的经验论之狭隘性,在感觉和概念之间挖开了深沟。理性主义学派的狭隘性,则使概念脱离了感觉。只有把认识当作过程看待的辩证唯物论的认识论(反映论)才彻底除去了这样狭隘性,把认识放在唯物的与辩证的地位。

反映论指出:反映过程不限于感觉和印象,也存在于思维中(抽象的概念中),认识是一个由感觉到思维的运动过程。列宁曾说:“反映自然的认识,不是简单的,直接的整体的反映,而是许多抽象的思考、概念、法则等等之形成过程”\footnote{出自《黑格尔〈逻辑学〉一书摘要》中《主观逻辑或概念论•第1篇 主观性》。参见《列宁全集》中文第2版,第55卷,第152页:“认识是人对自然界的反映。但是,这并不是简单的、直接的、完整的反映,而是一系列的抽象过程,即概念、规律等等的构成、形成过程”。}。

同时列宁还指出:由感觉到思维的认识过程,是飞跃式地进行的,在这一点上,列宁精确地阐明了:认识中的经验元素和理性元素相互关系之辩证唯物论的见解。许多哲学家都不了解认识的运动过程中,即从感觉到思维(从印象到概念)的运动过程中所发生的突变。因此理解这一由矛盾而产生的飞跃式的转变,即理解感觉和思维的一致为辩证的一致,便是理解了列宁反映论的本质之最重要的元素。

(十)真理论

真理是客观的,相对的,又是绝对的。这就是唯物辩证法的真理观。

真理首先是客观的。在承认了物质的客观实在性及物质对于意识的根源性之后,就等于承认了真理的客观性。所谓客观真理,就是说:客观存在的物质世界,是我们的知识或概念的内容之唯一来源,再也没有别的来源;只有唯心论者否认物质世界离人的意识而独立存在——这一唯物论的基本原则,才主张知识或概念是主观自主的,不要任何客观的内容,因而承认主观真理,否认客观真理。然而这是不合事实的。任何一种知识或一个概念,如果它不是反映客观世界的规律性,它就不是科学的知识,不是客观真理,而是主观地自欺欺人的迷信或妄想。人类以改变环境为目的之一切实际行动,不管是生产行动也罢,阶级斗争或民族斗争的行动也罢,其他任何一种行动也罢,都是受着思想(知识)的指挥的。这种思想如果不适合于客观的规律性,即客观规律性没有反映到行动的人的脑子里去,没有构成他的思想或知识的内容,那末这种行动是一定不能达到目的的。革命运动中所谓主观指导犯错误,就是指的这种情形。马克思主义所以成为革命的科学知识,就是因为它正确地反映了客观世界的实际规律,它是客观的真理。一切反马克思主义的思想所以都是错的东西,就是因为它们不根据于正确的客观规律,完全是主观的妄想。有人说,一般公认的就是客观真理(主观唯心论者波格达诺夫\footnote{亚历山大·亚历山德罗维奇·波格丹诺夫(Алекса́ Алекса́ндрович Богданов,1873年8月22日—1928年4月7日) 俄国和苏联内科医生、哲学家、社会学家、科幻小说作家,白俄罗斯族革命家。。}这样说)。照这种意见,那末,宗教和偏见也是客观真理了,因为宗教和偏见虽然实质上是谬见,可是却常常为多数人所公认;有时正确的科学思想反不及这些谬见的普及。唯物辩证法根本反对这种意见,认为只有正确地反映客观规律性的科学知识,才能被称为真理,一切真理必须是客观的。真理与谬说是绝对对立的,判断一切知识是否为真理,唯一的看他们是否反映客观的规律。如果不合乎客观规律,尽管是一般人都承认的,或革命运动中某些说得天花乱坠的理论,都只能把它当作谬说看待。

唯物辩证法真理论的第一个问题,是主观真理和客观真理的问题,它的答复是否认前者而承认后者。唯物辩证法真理论的第二个问题,是绝对真理和相对真理的问题,它的答复不是片面地承认或否认某一方面,而是同时承认它们,并指出它们正确的相互关系,即指出它们的辩证性。

唯物辩证法在承认客观真理时,就是承认了绝对真理的。因为当我们说知识的内容是客观世界的反映时,这就等于承认了我们知识的对象是那个永久的绝对的世界。“关于自然之一切真理的认识,就是永久的无穷的认识,因此它实质上是绝对的”\footnote{出自《自然辩证法》中《[辩证法作为科学]•[认识]》。参见《马克思恩格斯选集》中文第3版,第3卷,第938页:“对自然界的一切真实的认识,都是对永恒的东西、对无限的东西的认识,因而本质上是绝对的。”}。然而客观的绝对的真理不是一下子全部成为我们的知识,而是在我们认识之无穷的发展过程中,经过无数相对真理的介绍,而到达于绝对的真理。这无数相对真理之总和,就是绝对真理的表现。人类的思维,就它的本性说,能给我们以绝对真理,绝对真理乃由许多相对真理积集而成,科学发展的每一阶段,增加新的种子到这个绝对真理的总和中去。但是每一科学原理的真理界限却总是相对的。绝对真理仅能表现在无数相对真理之上,如果不经过相对真理的表现,绝对真理就无从认识。唯物辩证法不否认一切知识之相对性,但这只是指吾人知识接近于客观绝对真理的限度之历史条件性而言,而不是说知识本身只是相对的。一切科学上的发明,都是历史地有限度的和相对的,但是科学知识跟谬说不同,它显示着描画着客观的绝对的真理,这就是绝对真理与相对真理相互关系之辩证法的见解。

有两种见解:一种是形而上学的唯物论;另一种是唯心论的相对论。对于绝对真理与相对真理之相互关系问题都是不正确的。

形而上学的唯物论者,根据于他们的“物质世界无变化”的形而上学的基本原则,认为人类思维也是不变化的,即认为在人的意识中这一不变的客观世界,是一下子整个被摄取了。这就是说他们承认绝对真理,而这个绝对真理是一次被人获得的,他们把真理看成不动的,死的,不发展的东西。他们的错误不在于他们承认有绝对真理——承认这一点是正确的,而在于他们不了解真理的历史性,不把真理的获得看作一个认识的过程。不了解所谓绝对真理者,只能在人类认识的发展过程中一步一步地开发出来,而每一步向前的认识,都表现着绝对真理的内容,但对于全部真理说来,它具有相对的意义,并不能一下子获得绝对真理的全部。形而上学的唯物论关于真理的见解,表现了认识论一个极端。

认识论中关于真理问题的再一个极端,就是唯心论的相对论。他们否认知识之绝对真理,只承认它的相对意义。他们认为一切科学的发明,都不包含绝对真理,因而也不是客观真理,真理只是主观的与相对的。既然这样,那末一切谬说就都有存在的权利了,帝国主义侵略弱小民族,统治阶级剥削劳动群众,这些侵略主义与剥削制度也就是真理,因为真理横直只是主观的与相对的。否认客观真理与绝对真理的结果,必然到达这样的结论。并且唯心论的相对论,他们的目的本来就是要替统治阶级作辩护的,例如相对论的实用主义(或实验主义)之目的就在于此。

这样看来,不论是形而上学的唯物论,或是唯心论的相对论,都不能正确解决绝对真理和相对真理的相互关系的问题。只有唯物论辩证法,既给思维与存在相互关系问题以正确的解答,并且随之而来又确定了科学知识的客观性,再则,还同时给了绝对与相对真理以正确的理解。这就是唯物辩证法的真理论。

(十一)实践论(认识与实践的关系,理论与实际的关系,知与行的关系。)

马克思以前的唯物论,离开人的社会性,离开人的历史发展,去观察认识问题,因此不能了解认识对社会实践的依赖关系,即认识对生产与阶级斗争的依赖关系。

首先,马克思主义者认为人类的生产活动是最基本的实践活动,是决定其他一切活动的东西。人的认识,主要的依赖物质的生产活动,逐渐了解自然的现象、自然的性质(自然的规律性)、人与自然的关系;而且经过生产活动,同时也认识了人与人的相互关系。一切这些知识,离开生产活动是不能得到的。每个人以社会一员的资格,与其他社会成员协力从事生产活动,以解决人类物质生活问题,这是人的认识发展的基本来源。

人的社会实践,不限于生产活动一种形式,还有多种其他的形式,阶级斗争,政治生活,科学活动,总之,社会实际生活的一切领域都是社会的人所参加的。因此,人的认识,在物质生活以外,还从政治文化生活中(与物质生活密切联系)了解了人与人的各种复杂的关系。其中尤以各种形式的阶级斗争,给予人的认识发展以深刻的影响。在阶级社会中,各种思想无不打上阶级的烙印,就是这个原故。

因此,马克思主义者认为只有人们的社会实践,提给人们对于外界认识之真理性的标准。实际的情形是这样的,只有在社会实践过程中(物质生产过程中、阶级斗争过程中、科学实验过程中),人们达到了思想中所预想的结果时,人们的认识才会发生力量。农民如果得不到收获,工人如果做不成器物,罢工斗争,军队作战,民族革命,如果也都得不到胜利,那末这是为什么呢?这是因为人们的认识没有外界的过程的实况去反映这些过程的规律性。因而在他们的实践活动中不能达到预想的结果。人们要想得到胜利(即得到预想的结果),一定要自己的思想合于客观外界的规律性。如果不合,就会在实践中失败,人们经过失败之后,也就从失败取得教训,改正自己的思想使之适合于外界的规律性,人们就能变失败为胜利,所谓“失败者成功之母”,“吃一堑长一智”,就是这个道理。辩证唯物论的认识论把实践提到第一的地位,认为人的认识一点也不能离开实践,排斥一切否认实践重要性、使认识离开实践的错误理论。列宁这样说过:“实践高于(理论的)认识,因为它不但有一般性的价值,而且还有直接现实性的价值”\footnote{出自《黑格尔〈逻辑学〉一书摘要》中《主观逻辑或概念论•第3篇 观念》。参见《列宁全集》中文第2版,第55卷,第183页:“实践高于(理论的)认识,因为它不仅具有普遍性的品格,而且还具有直接现实性的品格。”}。马克思主义的哲学辩证唯物论的最显著的特点有两个:一个是它的阶级性,公然申明辩证唯物论是为无产阶级服务的;再一个是它的实践性,强调理论对于实践的依赖关系,理论来源于实践,又转过来为实践服务。判定认识或理论之是否真理,不是依主观上觉得如何而定,而是依客观上社会实践的结果如何而定。真理的标准只能是社会的实践。实践的观点是辩证唯物论的认识论之第一的与基本的观点。

然而人的认识究竟怎样从实践发生,而又服务于实践呢?这只要看一看认识的发展过程就会明了的。

原来人在实践过程中,开始只是看到过程中各个事物的现象方面,看到各个事物的片面,看到各个事物之间的外部联系。例如国民党考察团到延安的头一二天,看到了延安的地形、街道、屋宇,接触了许多的人,参加了宴会、晚会与群众大会,听到了各种说话,看到了各种文件,这些就是事物的现象,事物的各个片面以及这些事物的外部联系。这叫做认识的感性阶段,就是感觉与印象的阶段。也就是延安这些各别的事物作用于考察团先生们的感官,引起了他们的感觉,在他们的脑子中生起了许多的印象,以及这些印象间的大概的外部的联系,这是认识的第一个阶段。在这个阶段中人们还不能造成深刻的概念,作出理论的结论。

社会实践的继续,使人们在实践中引起感觉与印象的东西反复了多次,于是在人们的脑子里生起了一个认识过程中的突变,产生了概念。概念这种东西已经不是事物的现象,不是事物的各个片面,不是它们外部的联系,而是抓着了事物的本质,事物的全体,事物的内部联系了。概念同感觉,不但是数量上的差别,而且有了性质上的差别。循此继进,使用判断与推理的方法,就可生出理论的结论来。《三国演义》上所谓“眉头一皱计上心来”,我们普通说话所谓“让我想一想”,就是人在脑子中运用概念以作判断与推理的工夫。这是认识的第二个阶段,或叫论理阶段,是认识的第二个阶段。考察团先生们在他们集合了各种材料,加上他们“想了一想”之后,他们就能够作出“共产党抗日民族抗一战线与国共合作的政策是彻底的、诚恳的与真实的”这样一个判断了。在他们作出这个判断之后,如果他们对于团结救国也是真实的话,那末他们就能够进一步作出这样的结论:“国共合作是能够成功的”。这个概念、判断与推理的阶段,在人对于一个事物的整个认识过程中是最重要的一个阶段。认识之真正任务不在感性的认识,而在理性的认识。认识之真正任务在于经过感觉而达到于思维,到达于了解客观事物的内部矛盾,了解它的规律性,了解这一过程与那一过程间的内部联系,即到达于理论的认识。再重复地说,理性的认识所以和感性的认识不同,是因为感性的认识是属于事物之片面的、现象的、外部联系的东西,理性的认识则推进了一大步,到达了事物之全体的、本质的、内部联系的东西,到达了暴露周围世界之内的矛盾,因而能在周围世界之总体上,在周围世界一切方面之内部联系上,去把握周围世界的发展。

这种基于实践之由浅入深的唯物辩证法的认识发展过程的理论,在马克思主义以前,是没有一个人这样解决过的。马克思主义的辩证唯物论,第一次正确地解决了这个问题,唯物地而且辩证地指出了认识之深化的运动,指出了社会的人在他们的生产与阶级斗争之复杂的、经常反复的实践中,由感性认识到理性认识之推移的运动。列宁说过:“物质的抽象,自然的法则,价值的抽象及其他等等,即一切科学的(正确的、重要的、非瞎说的)抽象,都比较深刻、比较正确、比较完全地反映自然。”\footnote{出自《黑格尔〈逻辑学〉一书摘要》中《主观逻辑或概念论•概念总论》。参见《列宁全集》中文第2版,第55卷,第142页:“物质的抽象,自然规律的抽象,价值的抽象以及其它等等,一句话,一切科学的(正确的、郑重的、非瞎说的)抽象,都更深刻、更正确、更完全地反映着自然。”}列宁又曾这样指出:认识过程中两个阶段的特性,在低级阶段,认识表现为感性的,在高级阶段,认识表现为理性的,但任何阶段,都是统一的认识过程中的阶段。感性与理性二者的性质不同,但又不是互相分离的,它们在实践的基础上统一起来了。我们的实践证明:感觉到了的东西,我们不能立刻理解它,只有理解了的东西才更深刻地感觉它。感觉只解决现象问题,理解才解决本质问题。这些问题的解决,一点也不能离开实践。无论何人要认识什么事物,除了同那个事物接触,即生活于(实践于)那个事物的环境中,是没有法子解决的。不能在封建社会就预先认识资本主义社会的规律,因为资本主义还未出现,还无这种实践。马克思主义只能是资本主义社会的产物。不能在自由资本主义时代就预先认识帝国主义时代的某些特异的规律,因为帝国主义还未出现,还无这种实践,只有列宁和斯大林才能担当此项任务。马克思与列宁也不能在经济落后的殖民地产生,这是因为虽然同时但不同地。马克思、恩格斯、列宁之所以能够作出他们的理论,除了他们的天才条件之外,主要地是他们亲身参加了当时的阶级斗争与科学实验的实践,没有这后一个条件,任何天才也是不能成功的。“秀者不出门,全知天下事”,在技术不发达的古代只是一句空话,在技术发达的现代虽然可以实现这句话,然而真正亲知的是天下实践的人,那些人在他们实践中间取得了“知”,经过文字与技术的传达而到达于“秀才”之手,秀才乃能间接地“知天下事”。如果要直接地认识某种或某些事物,便只有亲身参加于变革现实、变革某种或某些事物的实践中,才能触到那种或那些事物的现象,也只有在亲身参加变革现实的实践中,才能暴露那种或那些事物的本质而理解它。这是任何人实际上走着的认识路程,不过有些人故意歪曲地说些反对的话罢了。世上最可笑的是那些“知识份子”,有了道听途说的一知半解,便自封为“天下第一”,多见其不自量而已。知识的问题是一个科学问题,来不得半点虚伪与骄傲,决定地需要的到是他的反面——诚实与谦逊的态度。你要有知识,你就得参加变革现实的实践。你要知道梨子的滋味,你就得变革梨子,亲口吃一吃。你要知道原子的组织同性质,你就得实行化学家的实验,变革原子的情况。你要知道革命的具体理论与方法,你就得参加革命。一切真知都是从直接经验发源来的。但人不能事事直接经验,事实上多数的知识都是间接经验的东西,这就是一切古代的与外域的知识。这些知识在古人在外人是直接经验的东西,如果在古人外人直接经验时是附合于列宁所说的条件:“科学的(正确的、重要的、非瞎说的)抽象”,那末它们是可靠的,否则便是不可靠。所以一个人的知识,不外直接经验与间接经验的两部分。而且在我为间接经验者,在人则仍属直接经验。因此,就知识的总体说来,无论何种知识都是不能离开直接经验的。任何知识的来源,在于人的肉体感官对客观外界的感觉,否认了这个感觉,否认了直接经验,否认了亲身参加变革现实的实践,他就不是唯物论者。“知识份子”之所以可笑,原因就在这个地方。中国商人有一句话:“要赚畜生钱,要跟畜生眠”。这句话对于商人赚钱是真理,对于认识论也是真理,离开实践的认识是不可能的。

为明了基于变革现实的实践而产生的唯物辩证法的认识运动——认识之逐渐深化的运动,下面再举出几个具体的例子。

无产阶级对于资本主义过程的认识,在其实践的初期——破坏机器与自发斗争时期,他们还只在感性认识的阶段,只认识资本主义个别现象的片面及其外部的联系。这时,他们还是一个所谓“自在的阶级”。但到了他们实践的后期——有意识有组织的阶级斗争与政治斗争的时期,由于实践,由于长期斗争的经验,教训了他们,他们就理解了资本主义社会的本质,理解了社会阶级的剥削关系,产生了马克思主义的理论,这时他们就造成了一个“自为的阶级”。

中国人民对于帝国主义的认识也是这样。第一阶段是表面的感性的认识,表现在太平天国运动与义和团运动等笼统的排外主义的斗争上。第二阶段才进到理性的认识,看出了帝国主义内部与外部的各种矛盾,并看出了帝国主义联合中国封建阶级以压榨中国人民大众的实质,这种认识是从五四运动前后才开始的。

我们再来看战争。战争的领导者,如果他们是一些没有战争经验的人,对于一个具体的战争(例如我们过去十年的苏稚埃战争)的深刻的指导规律,在开始阶段是不了解的。他们在开始阶段只是身历了许多作战的经验,而且败仗是很多的。然而由于这些经验(胜仗,特别是败仗的经验),使他们能够理解贯串整个战争的内部的东西,即那个具体的战争之规律性,懂得了战略与战术,因而能够有把握地去指导战争。此时,如果改换一个无经验的人去指导,又会要在吃了一些败仗之后(有了经验之后),才能理会战争的正确的规律。

常常听到一些同志在不能勇敢接受工作任务时说出来的一句话,就是说:他没有把握。为什么没有把握呢?因为他对这项工作的内容与环境没有规律性的了解,或者他从来就没有接触过这类工作,或者接触得不多,因而无从说到了解这类工作的规律性。及至把工作的情况同环境给以详细分析之后,他就觉得比较有了把握,愿意去做这项工作。如果这个人在这项工作中经过了一个时期(他有了这项工作的经验),而他又是一个肯虚心体察客观情况的人,不是一个主观地、片面地、表面地看问题的人,他就能够自己做出应该怎样进行工作的结论,他的工作勇气也就可以大大地提高。只有那些主观地、片面地与表面地看问题的人,跑到一个地方,不问环境的情况,不看事情的全体(事情的历史与全部现状),也不触到事情的本质(事情的性质及此一事情与其他事情的内部联系),就“自以为是”地发号施令起来,这样的人是没有不跌交子的。

由此看来,认识的过程,第一步是开始接触外界事情,属于感觉的阶段。第二步是综合感觉的材料加以改造和整顿,属于概念、判断、与推理的阶段。只有感觉的材料十分丰富(不是零碎不全)与合于实际(不是错觉),才能根据这样的材料造出正确的概念与理论来。

这里有两个要点须着重指明:第一个,在前面已经说过的,这里再重复说一说,就是理性认识依赖于感性认识的问题。如果以为理性认识可以不从感性认识得来,他就是一个唯心论者。哲学史上有所谓“唯理论”一派,就是只承认理性的实在性,不承认经验的实在性,以为只有理性靠得住,而感觉的经验是靠不住的。这一派的错误在于颠倒了事实。理性的东西所以靠得住,正由于它来源于感性,否则理性的东西就成了无源之水,无本之木,而只是主观自生的靠不住的东西了,从认识过程的秩序说来,感觉经验是第一的东西,我们强调社会实践在认识过程中的意义,就在于只有社会实践才能使人的认识开始发生,开始从客观外界得到感觉经验。一个闭目塞听、同客观外界根本绝缘的人,是无所谓认识的。认识发源于经验——这就是认识论的唯物论。

第二是认识有待于深化,有待于发展到理性阶段——这就是认识论的辩证法。如果以为认识可以停顿在低级的感性阶段,以为只有感性认识可靠,而理性认识是靠不住的,这便重复了历史上“经验论”的理论。这种理论的错误,在于不知道感觉材料固然是客观外界某些真实性的反映(不去说“经验只是内省体验的那种唯心的经验论”),但它们仅是片面的与表面的东西,这种反映是不完全的,是没有反映事物本质的。要完全地反映整个的事物,反映事物的本质,反映其内部联系规律性,就非经过思考作用,将丰富的感觉材料加以去粗取精、去伪存真、由此及彼、由表及里的改造制作工夫,造成概念及理论的系统不可,非从感性认识,改变到理性认识不可。这种改造过的认识,不是更空虚更不可靠了的认识,相反地,只要是在认识过程中根据于实践基础而科学地改造过的东西,正如列宁所说:它是更深刻、更正确、更完全地反映客观事物的东西。庸俗的事物主义家不是这样,他们尊重经验而看轻理论,因而不能通观客观过程的全体,缺乏明确的方针,没有远大的前途,沾沾自喜于一得之功与一孔之见。这种人如果指导革命,就会引导革命走上碰壁的地步。

理性认识依赖于感性认识,感性认识有待于发展到理性认识,这就是唯物辩证法的认识论。哲学上的认识论与经验论,都不懂得认识的历史性或辩证性,虽然各有片面的真理(对于唯物的唯理论与经验论而言,非指唯心的唯理论与经验论),但在认识论的全体上则都是错误的。由感性到理性之唯物辩证法的认识运动,对于一个小的认识过程(例如一个事物或一件工作)是如此,对于一个大的认识过程(例如一个社会或一个革命)也是如此。

然而认识运动至此还没有完结。唯物辩证法的认识运动,如果只到理性认识为止,那么还只说到问题的一半。而且对于马克思主义的哲学说来,还只说到非十分重要的那一半。马克思主义哲学认为十分重要的问题,不在于懂得了客观世界的规律性,因而能够解释宇宙,而在于拿了这种对于客观规律性的认识去改造宇宙。在马克思主义看来,理论是重要的,它的重要性充分地表现在列宁说过的一句话:“没有革命的理论,就没有革命的运动”\footnote{出自《俄国社会民主党人的任务》以及《怎么办?》第1章第4节。分别参见《列宁全集》中文第2版,第2卷,第443页;第6卷,第23页:“没有革命的理论,就不会有革命的运动。”}。人的一切行动(实践)都是受人的思想指导的,没有思想,当然就没有任何的行动。然而马克思主义看重理论,正是,也仅仅是,因为它能够指导行动。如果有了正确的理论,只在把它空谈一会,束之高阁,并不实行,那么这种理论再好也是没有用的。认识从实践始,经过实践得到了理论的认识,还须再回到实践去。认识的能动作用,不但表现于从感性的认识到理性的认识之能动的飞跃,更重要的还须表现于从理性的认识到革命的实践这一个飞跃。抓住了世界现实规律性的认识,必须把它再用到改造世界的实践中去,再用到生产的实践、革命的阶级斗争与民族斗争的实践以及科学实验的实践中去。这就是检验理论与发展理论的过程,是整个认识过程的继续。理论的东西或理性的认识之是否符合于客观真理性这个问题,在前面说的由感性到理性之认识运动中是没有完全解决的,也不能完全解决的。要完全地解决此问题,只有把理性的认识再回到社会实践中去,应用理论于实际,看它是否能够达到预想的目的。许多自然科学理论之所以被认为真理,不但在于发现此学说时,而且在于为尔后的科学实践所证实。马克思主义之所以被称为真理,也不但在于马克思等人科学地构成此学说时,而且在于为尔后革命的阶级斗争与民族斗争的实践所证实。辩证唯物论之是否为真理,在于经过无论什么人的实践都不能逃出它的范围。认识史的实践告诉我们,许多理论的真理性是不完全的,经过实践的检验而纠正了它们的不完全性。许多理论是错误的,经过实践的检验而纠正其错误。所谓“实践是真理的标准”,所谓“实践是认识论第一与基本的观点”,理由就在这个地方。斯大林说的好:“离开实践的理论,是空洞的理论,离开理论的实践,是盲目的实践”\footnote{出自《论列宁主义基础》第3部分《理论》。参见《斯大林选集》上卷,人民出版社,1979年,第199—200页:“离开革命实践的理论是空洞的理论,而不以革命理论为指南的实践是盲目的实践。”}。

说到这里,认识运动就完成了吗?我们的答复是完成了,又没有完成。社会的人投身于变革在某一一定发展阶段内之某一一定客观过程的实践中(不论是关于变革某一自然过程的实践,或变革某一社会过程的实践),由于客观过程的反映与主现能动性的作用,使得人的认识由感性的推移到了理性的,造成了大体上相应于该客观过程之法则性的理论、思想、计划、或方案,然后再应用这种理论、思想、计划或方案于该同一客观过程的实践,如果能够实现预想的目的,即将预定的理论、思想、计划、方案于该同一过程的实践中变为事实,或大体上变为事实,那末,对于这一具体过程的认识运动算是完成了。例如,在变革自然的过程中,某一工程计划的实现,某一科学假想的证实,某一器物的制成,某一农产的收获,在变革社会过程中,某一罢工的胜利,某一战争的胜利,某一教育计划的实现,某一救国团体的成立,都算实现了预想的目的。然而一般说来,不论在变革自然或变革社会的实践中,人们原定的理论、思想、计划、方案,毫无改变地实现出来之事,是很少的。这是因为从事变革现实的人们,常常受着许多的限制,不但常常受着科学条件与技术条件的限制,而且也受着客观过程表现程度的限制(客观过程的方面及本质尚未充分暴露)。在这种情形之下,由于实践中发现前所未料的情况,因而部分地改变理论、思想、计划、方案的事是常有的,全部地改变的事也是有的。即是说原定的理论、思想、计划、方案,部分或全部不合于实际,部分错了或全部错了的事,都是有的。许多时候须反复失败过多次,才能纠正错误的认识,才能到达于同客观过程的规律性相符合,因而才能够变主观的东西为客观的东西(即在实践中得与预想结果之正确的认识)。但不管怎样,到了这种时候,人们对于在某一一定发展阶段内之某一一定客观过程的认识运动,算是完成了。

然而对于过程之推移而言,人的认识运动是没有完成的。任何过程,不论是属于自然界的与属于社会的,由于内部的矛盾与斗争,都是向前推移向前发展的,人的认识运动也应跟着推移与发展。依社会运动来说,所贵于革命的指导者,不但在于当自己的理论、思想、计划、方案有错误时须得善于加以改正,如同上面已经说到的,而且在于当某一一定的客观过程已经从某一一定的发展阶段向另一一定的发展阶段推移转变的时候,须得善于使自己及参加革命的人员在主观认识上也跟着推移转变,即是要使新的革命任务与新的工作方案的提出,适合于新的情况的变化。革命时期情况的变化是很急速的,如果革命党人的认识不能随之而急速变化,就不能引导革命走向胜利。然而思想落后于实际的事是常有的,这是因为人的认识受了许多限制的原故。许多人受了阶级条件的限制(反动的剥削阶级,他们已无认识任何真理的能力,因而也没有改造宇宙的能力,相反地,他们变成了阻碍认识真理与改造世界的敌人),有些人受了劳动分工的限制(劳心、劳力的分工,各业之间的分工),有些人受了原来的错误思想的限制(唯心论与机械论等多属于剥削份子;但也有被剥削份子,由于剥削份子的教育而来),而一般的原因则在受限制于技术水平与科学水平的历史条件。无产阶级及其政党,应该利用自己天然优胜的阶级条件(这是任何别的阶级所没有的),利用新的技术与科学,利用马克思主义的世界观与方法论,紧密地依靠革命实践的基础,使自己的认识跟着客观情况的变化而变化,使理论的东西随历史的东西,平行并进,达到完满地改造世界的目的。

我们反对革命队伍中的顽固派,他们的思想不能随变化了的客观情况而前进,在历史上表现为右倾机会主义。中国1927年的陈独秀主义,苏联的布哈林主义,都属于这一类。这些人看不出矛盾的斗争已将客观过程推向前进了,而他们的认识仍然停止在旧阶段。一切顽固派的思想都有这样的特征。他们的思想离开了社会的实践,他们不能站在社会车轮的前头充任向导的工作,他们只知跟在车轮后面怨恨车轮走的太快了,企图把它向后拉,开倒车。

我们也反对“左”翼清谈主义。中国1930年的李立三主义,苏联在尚可作为一个共产主义派别看待时的托洛斯基主义(现在则已成最反动的派别),以及世界各国的超左思想,都属于这一类。他们的思想超过客观过程的一定发展阶段,有些把幻想看作真理,有些则把仅在将来有现实可能性的理想,强迫放在现时来做,离开了当前大多数人的实践,离开了当前的现实性,行动上表现为冒险主义。

唯心论与机械论,机会主义与冒险主义,都没有唯物辩证的认识论的根据,他们都是以主观同客观相分裂,以认识与实践相舍离为特征的。以科学的社会实践为特征的马克思主义的认识论,不能不坚决反对这些错误思想。马克思主义者承认,在绝对的总的宇宙发展过程中,各个具体过程的发展都是相对的,因而人的认识也在绝对的真理中对于在各个一定发展阶段上的具体过程之认识只有相对的真理。客观过程的发展是充满着矛盾与斗争的发展,人的认识运动也是充满着矛盾与斗争的发展。一切客观世界的辩证法的运动,都或先或后地能够反映到认识中来。实践中之发展与消灭的过程是无穷的,人的认识之发生、发展与消灭的过程也是无穷。根据于一定的理论、思想、计划、方案以从事于变革客观现实的实践,一次又一次地向前,人对客观现实的认识也就一次又一次地深化。客观现实世界的变化运动永远没有完结,人在实践中对真理的认识也永远没有完结。马克思主义没有结束真理,而是在实践中不断地开辟认识真理的道路。我们的结论是主观与客观、理论与实践、知与行的具体历史的统一,反对一切离开具体历史的“左”的或“右”的错误思想。

大宇宙中自然发展与社会发展到了今日的时代,正确地认识宇宙与改造宇宙的责任,已经历史地落在无产阶级及其政党的肩上。这种根据科学认识而定下来的改造世界的实践过程,在世界、在中国均已到达了一个历史的时节——自有历史以来未曾有过的重大时节,这就是整个儿地推翻世界与中国的黑暗面,把它转变过来成为前所未有的光明世界。无产阶级及革命人民改造世界的斗争,包括实现下述的任务:改造客观世界,也改造自己的主观世界——改造自己的认识能力,改造主观世界同客观世界的关系。地球上已经有一部分实行了这种改造,这就是苏联。他们还正在为自己为世界推进这种改造过程。中国人民与世界人民也都正开始或将要通过这样的改造过程。所谓被改造的客观世界,其中包括了一切反对改造的人们,他们的被改造,须通过强迫的阶段,然后才能进入自觉的阶段。世界到了全人类都自觉地改进自己与改造世界的时候,那就是世界的共产主义时代。

通过实践而发现真理,又通过实践而证实真理与发展真理。从感性认识而能动地发展到理性认识,又从理性认识而能动地指导革命实践,改造主观世界与客观世界。实践、认识、再实践、再认识的形式,循环发展以至无穷,而实践与认识之每一循环的内容,都比较地进到高一级的程度——这就是唯物辩证法的全部认识论,这就是唯物辩证法的知行统一观。

\subsection{唯物辩证法}

前面简述了“唯心论与唯物论”及“辩证法唯物论”两个问题。关于辩证法问题,仅有概略的提到,现在来系统地讲这个问题。

马克思主义的世界观(或叫宇宙观),是辩证法唯物论,不是形而上学的唯物论(或叫机械的唯物论)。这一点区别,是一个天翻地复的大问题。世界是一个什么样子的?从古至今有三种主要的答案:第一种是唯心论(不管是形而上学的唯心论,或辩证法的唯心论),说世界是心造的,引申起来又可说是神造的。第二种是机械唯物论,否认世界是心的世界,说世界是物质的世界,但物质是不发展的,不变化的。第三种是马克思主义的答案,推翻了前面两种,说世界不是心造的,也不是不发展的物质,而是发展的物质世界,这就是辩证法唯物论。马克思主义这样地看世界,把世界在从来人们眼睛中的样子翻转了过来,这不是天翻地复的大议论吗?世界是发展的物质世界,这种议论,在西洋古代的希腊就有人说过了,不过因为时代的限制,还只简单地笼统地说了一说,叫做朴素的唯物论。没有(也不可能有)科学的基础,然而议论是基本上正确的。黑格尔创造了辩证的唯心论,说世界是发展的,但是心造的,他是唯心发展论,其正确是发展论(即辩证论),其错误是唯心发展论。西洋十七、十八、十九三个世纪,法德等国的资产阶级唯物论,则是机械观的唯物论。他们说世界是物质世界,这是对的,说是象机械一样的运动,只有增减或位置的变化,没有性质上的变化,这是不对的。马克思继承了希腊朴素的辩证唯物论,改造了机械唯物论与辩证唯心论,造成了从古以来没有过的、放在科学基础之上的辩证唯物论,成为全世界无产阶级及一切被压迫人民的革命的武器。

唯物辩证法是马克思主义的科学方法论,是认识的方法,是论理的方法,然而它就是世界观。世界本来是发展的物质世界,这是世界观。拿了这样的世界观转过来去看世界,去研究世界上的问题,去想世界上的问题,去解决世界上的问题,去指导革命,去做工作,去从事生产,去指挥作战,去议论人家长短,这就是方法论。此外并没有别的什么单独的方法论。所以在马克思主义者手里,世界观同方法论是一个东西,辩证法、认识论、论理学,也是一个东西。

我们要系统地来讲唯物辩证法,就要讲到唯物辩证法的许多问题,这就是它的许多范畴、许多规律、许多法则(这几个名词是一个意思)。

唯物辩证法究竟有些什么法则呢?这些法则中那些是根本法则,那些是附从于根本法则而又为唯物辩证法学说中不可缺少不可不解决的方面、侧面或问题呢?所有这些法则,为什么不是主观自造的,而是客观世界本来的法则呢?对于这些法则的学习、了解,是为了什么呢?

这个完整的革命的唯物辩证法学说,创造于马克思与恩格斯,列宁发展了这个学说。到了现在苏联社会主义胜利与世界革命时期,这个学说又走上了新的发展阶段,更加丰富了它的内容。这个学说中包含的范畴首先是如下各项:

矛盾统一法则;

质量互变法则;

否定之否定法则。

以上是唯物辩证法的根本法则。除古代希腊的朴素唯物论曾经简单地无系统地指出了这些法则的某些意义,及黑格尔唯心地发展了这些法则外,都是被一切形而上学(所谓形而上学,就是反发展论的学说)所否定了的。直到马克思、恩格斯,才唯物地改造了黑格尔的这些法则,成为马克思主义世界观与方法论之最基本的部份。

唯物辩证法所包含的范畴,除了上述根本法则外,同这些根本法则联系着,还有如下各范畴:

1.本质与现象;

2.形式与内容;

3.原因与结果;

4.根据与条件;

5.可能与现实;

6.偶然与必然;

7.必然与自由;

8.链与环、等等。

这些范畴,有些是从来形而上学及唯心辩证法所着重研究过的,有些是从来哲学片面地研究过的,有些则是马克思主义新提出的。这些范畴,在马克思主义的革命理论家与实践家手里,揭去了从来哲学唯心的及形而上学的外衣,克服其片面性,发现了它们的真实形态,并且随着时代的进步,极大地丰富了它们的内容,成为革命的科学方法论中重要的成份。拿这些范畴同上述根本的范畴合在一起,就形成一个完整的深刻的唯物辩证法的系统。

所有这些法则或范畴,都不是人的思想自己造出来的,而是客观世界本来的法则。一切唯心论都说精神造出物质,那末,在他们看来,哲学的法则、原则、规律或范畴,自然更是心造的了。发挥了辩证法系统的黑格尔,就是这样的去看辩证法的。在他看来,辩证法不是从自然和社会的历史中抽取出来的法则,而是纯粹思想上的论理系统。人的思想造出了这一套系统之后,再把它们套到自然和社会上去。马克思、恩格斯揭去黑格尔的神秘的外衣,丢弃了它们的唯心论,把辩证法放在唯物论的地位。恩格斯说;“辩证法的法则,是从自然和人类历史抽取出来的,但他们并非别的,就是这两个历史发展领域的最普遍的发展法则,就实质论,可以归纳为质量互变,矛盾统一,否定之否定这三个根本法则”\footnote{出自《自然辩证法》中《[辩证法作为科学]•辩证法》。参见《马克思恩格斯选集》中文第3版,第3卷,第901页:“辩证法的规律是从自然界的历史和人类社会的历史中抽象出来的。辩证法的规律无非是历史发展的这两个方面和思维本身的最一般的规律。它们实质上可归结为下面三个规律:量转化为质和质转化为量的规律;对立的相互渗透的规律;否定的否定的规律。”}。辩证法法则是客观世界的法则,同时也是主观思想里头的法则,因为人的思想里头的法则不是别的,就是客观世界的法则通过实践在人类头脑中的反映。辩证法、认识论、论理学是一个东西,前面已经讲过了。

我们学习辩证法是为了什么呢?不为别的,单单为了要改造这个世界,要改造这个世界上面人与人、人与物的老关系。这个世界上面的人类,大多数过着苦难的日子,受着少数人所控制的各种政治经济制度的压迫。在我们中国这个地方生活着的人类,受着惨无人道的双重性制度的压迫——民族压迫与社会压迫,我们必须改变这些老关系,争取民族解放与社会解放。

要达到改造中国同世界的目的,为什么要学习辩证法呢?因为辩证法是自然同社会的最普遍的发展发展,我们明了它,就得到了一种科学的武器。在改造自然同社会的革命实践中,就有了同这种实践相适应的理论同方法。唯物辩证法本身是一种科学(一种哲理的科学),它是一切科学的出发点,又是方法论。我们的革命实践本身也是一种科学,叫做社会科学或政治科学。如果不懂得辩证法,则我们的事情是办不好的。革命中间的错误,无一不违反辩证法。但如懂得了它,那就能生出绝大的效果。一切做对了的事,考究起来,都是合乎辩证法的。因此,一切革命的同志们,首先是干部,都应用心地研究辩证法。

有人说:许多人懂得实际的辩证法,而且也是实际的唯物论者,他们虽没有读过辩证法书,可是做起事来是做得对的,实际上合乎唯物辩证法,他们就没有特别研究辩证法的必要了。这种话是不对的。唯物辩证法是一种完备的深刻的科学,实际上具有唯物的与辩证的头脑之革命者,他们虽从实践中学得了许多辩证法,但是没有系统化,没有如同已经成就的唯物辩证法那样的完备性与深刻性,因此还不能洞察运动的远大前途,不能分析复杂的发展进程,不能捉住重要的政治关节,不能处理各方面的革命工作,因此仍有学习辩证法的必要。

又有人说,辩证法是深奥难懂的,一般人没有学会的可能。这话也是不对的。辩证法是自然、社会与思想的法则,任何有了一些社会经验(生产与阶级斗争的经验)的人,他就本来了解了一些辩证法。社会经验更多的人,他本来了解的辩证法就更多些,不过还处在零乱的常识状态,没有完备的深刻的了解。拿着这种常识辩证法加以整理与深造,是并不困难的。辩证法之所以使人觉得困难,是因为没有善于讲解的辩证法书,中国许多辩证法书,不是错了,就是写的不好或不大好,使人望而生畏。所谓善于讲解的书,在于以通俗的言语,讲亲切的经验,这种书将来总是要弄出来的。我这个讲义也不是好的,因为我自己还在开始研究辩证法,还没有可能写出一本好书,也许将来有此可能,我也有这个志愿,但要依研究的情形才能决定。

以下分述辩证法的各个法则。

矛盾统一法则

这个法则,是辩证法最根本的法则。列宁说:“就根本意义上来讲,辩证法就是研究客体本质中的矛盾”\footnote{出自《黑格尔(哲学史讲演录)一书摘要》。参见《列宁全集》中文第2版,第55卷,第213页:“就本来的意义讲,辩证法是研究对象的本质自身中的矛盾。”}。所以列宁常称这个法则为辩证法的实质,又称之为辩证法的核心。因此,我们的辩证法,就从这个问题讲起,并且把这个问题讲得比其他问题详细一些。

这个问题中,包含着许多问题,这些问题就是:

1.两种发展观;

2.形式论理学的同一律,与辩证法的矛盾律;

3.矛盾的普遍性;

4.矛盾的特殊性;

5.主要的矛盾与主要的矛盾方面;

6.矛盾的同一性与斗争性;

7.对抗在矛盾中的地位。

下面逐一说明这些问题。

(一)两种发展观

人类思想史中,从来就有关于世界发展的两种见解,一种是形而上学的发展观;一种是辩证法的发展观。这两种发展观的区别何在呢?

形而上学的发展观

形而上学,亦称玄学,在历来的思想中,占着统治的地位,这种哲学的内容,是说明他们所谓处于经验以外的事物,即论绝对体、论实质等等的学说。在近代哲学中,所谓形而上学,是用静的观点去观察事物的一种思想方法,把世界一切事物的形态和种类看成是永远不变化的。这种思想,统治于十七十八世纪的欧洲。由于阶级斗争和科学发展的结果,到了现代,即十九、二十世纪,辩证法的思想就一日千里地走上了世界舞台,但形而上学却又以庸俗的进化论(庸俗的、即谓鄙陋的、简单的)的形态,顽固地对抗着辩证法。

所谓形而上学的与庸俗进化论的发展观,概括说来,是说发展就是数量的增减,外力的推动,场所的变化。一切事物及这些事物在人的思想上的反映,都是永远如此的。事物的特性,是事物原来就有的,不过开头取萌芽状态,后来进到显著的地步而已。说到社会的发展,他们就认为是某些永远不变其性质的特点之增长和反复。这些特点,例如资本主义的剥削、竞争、个人主义等等,就是在古代奴隶社会,甚至原始野蛮社会,都可以找得出来。说到社会发展的原因,就用社会外部的地理、气候条件去说明它。这种发展观,从事物外部去找发展的原因,反对事物因内部矛盾引起发展的学说,它就不能解释事物之质的多样性,不能解释一种质变化到他种质的现象。这种思想,在十七、十八世纪是自然绝对不变论(机械唯物论),在二十世纪是庸俗进化论(布哈林的均衡论)等。

辩证法的发展观

主张从事物自己里头,从一事物对他事物的关系里头,去研究事物的发展,即把事物的发展看做是事物内部必然的、独立的、自己的运动,即事物的自动。事物发展的根本原因,不在外面而在内面,在于事物内部的矛盾性,任何事物内部都有这种矛盾性,因此引起了事物的运动与发展。

这样看起来,辩证法的发展观,反对了形而上学的与庸俗进化论的外因论,或被动论。这是清楚的,单纯的外部原因只能引起事物之机械的运动,即范围之大或小,数量之增或减,不能说明世界上事物何以有性质上的千差万别。事实上,即便是外力推动的机械运动,也要通过事物内部的矛盾性。植物动物之单纯的增长,也不只是数量的增加,同时就发生性质的变化,单纯增长也是矛盾引起的发展。至于社会的发展,同样主要地不是外因而是内因。许多国家在差不多一样的地理气候条件下,各个国家发展的差异性和不平衡性,却非常之大。设同一个国家罢,在地理气候并没有变化的情形下,社会变化却是很大的。地球各国都有此种情形。旧俄帝国变为社会主义的苏联,单纯封建的闭关锁国的日本变为帝国主义的日本,封建的西班牙正在变化到人民民主的西班牙,这些国家的地理气候并没有变。几千年封建制度的中国是变化最少的,然而近来却起了大变动,正在变化到自由解放的新中国去,难道中国今天的地理气候同数十年前的有什么两样?很明显的,不是外因而是内因。自然界的变化,由于自然界事物内部矛盾的发展。社会变化,由于社会内部矛盾的发展。生产力与生产关系的矛盾,阶级之间的矛盾,推动了社会的前进。辩证法排除外因吗?并不排除的。外因是变化的条件,内因是变化的根据,外因通过内因而起作用。鸡蛋因得适当温度而变化为鸡子,但温度不能使石头变为鸡子,因为内的根据不同。帝国主义的压力加速了中国社会的变化,也是通过中国内部自己的规律性而起变化的。两军相争,一胜一败,所以胜败,皆决于内因,胜者或因其强,或因其指挥无误,败者或因其弱,或因其指挥失宜,外因通过内因而引起变化。1927年资产阶级战败了无产阶级,是通过了无产阶级内部的(共产党内部的)机会主义而起作用的。一个阶级或一个政党要引导革命归于胜利,依靠自己没有政治路线的错误,依靠自己政治上组织上的巩固。中国东北沦亡,华北危急,主要由于中国之弱(1927年革命失败,政权不在人民手里,造成了内战与独裁制度),日本帝国主义乃得乘机而入。驱逐日寇,主要依靠民族统一战线执行坚决的革命战争。“物必先腐也,而后虫生之,人必先疑也,而后谗入之”,这是苏东坡的名言。“内省不疚,夫何忧何惧”,这也是孔夫子的实话。一个人少年充实,他就不容易感受风寒;苏联至今没有受日本的侵袭,全是因为他的强固;雷公打豆腐,拣着软的欺了,全在自强,怨天尤人,都没有用,人定胜天,困难可以克服,外界的条件可以改变,这就是我们的哲学。

我们反对形而上学的发展观,主张辩证法的发展观。我们是变化论者,反对不变论,我们是内因论者,反对外因论。

(二)形式论理学的同一律与辩证法的矛盾律

上面说了形而上学的发展观与辩证法的发展观,这两种对于世界观上面的斗争,就形成了思想方法上面形式论理与辩证论理的斗争。

资产阶级的形式论理学上有三条根本规律,第一条叫做同一律,第二条叫做矛盾律,第三条叫做排中律。什么是同一律呢?同一律说:在思想过程中概念是始终不变化的,它永远等于自己。例如原素永远等于原素,中国永远等于中国,某人永远等于某人。它的公式是:甲等于甲。这一规律是形而上学的。恩格斯说它是旧宇宙观的根本规律。它的错误,在于不承认事物的矛盾与变化,因而从概念中除去了暂时性相对性,给与了永久性、绝对性。不知事物同反映事物的概念都是相对的变化的,某一原素并不永远等于某一原素,各种原素都在变化着。中国也不永远等于中国,中国在变化着,过去古老封建的中国同今后自由解放的中国是两个东西。某人也不永远等于某人,人的体格思想都在变化着。1925—1927年的蒋介石不等于1927年以后的蒋介石,现在以后的蒋介石又将不等于以前。思想中的概念是客观事物的反映,客观事物在变化,概念的内容也在变化。事实上永久等于自身的概念,世界上一个也没有。

什么是矛盾律呢?矛盾律说:概念自身不能同时包含二个或二个以上互相矛盾的意义,假如某一个概念中包含了二个矛盾的意义,就算是论理的错误。矛盾的概念,不能同时两边都对,或两边都不对,对的只能是其中的一边,它的公式是:甲不等于非甲。康德曾举出如下四种矛盾思想:世界在时间上是有始终的,在空间上是有限度的;世界在时间上没有始终,在空间上亦无限度。这是第一种。世上一切都是单纯的(不可再分的)物性组成的,世上没有单纯的东西,一切都是复杂的(可以再分的)。这是第二种。世上存在着自由的原因,世上没有任何的自由,一切都是必然的。这是第三种。世上存在着某种必然的实质,世上没有必然的东西,一切都是偶然的。这是第四种。康德把这些不可调和的,互相反对的原理,名之曰“二律矛盾”。但是他说这些只是人的思想上的矛盾,实际世界里是并不存在的。依照形式论理学的矛盾律,这些矛盾乃是一种错误,必须加以排除。但是实际上思想是事物的反映,事物无一不包含着矛盾,因之概念也无一不包含矛盾。这不是思想的错误,正是思想的正确。辩证论理的矛盾统一律,就在这个基础上面建立起来。只有形式论理排除矛盾的矛盾律,乃是真正的错误思想。矛盾律在形式论理学中只是同一律之消极的表现,作为同一律的一种补充,目的在于巩固所谓概念等于自身,甲等于甲的同一律。

排中律是什么呢?排中律说:在概念之两相反的意义中,正确的不是这个就是那个,决不会两个都不正确,而跑出第三个倒是正确的东西来。它的公式是:甲等于乙,或不等于乙,但不会等于丙。他们不知道事物同概念是发展着的,在事物同概念的发展过程中,不但表现其内部的矛盾因素,而且可以看见这些矛盾因素的移去、否定、解决,而转变成为非甲非乙的第三者,转变成为较高一级的新事物或新概念。正确的思想,不应排除第三者,不应排除否定之否定律。无产阶级同资产阶级矛盾着,照排中律说来,正确的不是前者,就是后者,不会是没有阶级的社会;然而恰好社会进化的过程不是停止于阶级斗争,而要走到无阶级的社会中去。中国同日本帝国主义矛盾着,但我们不但反对日本帝国主义的侵略,也不赞同中国独立后同日本处于永久敌对的地位,而主张经过民族革命及日本国内的革命,把两个民族进到自由联合的阶段去。资产阶级的民主主义同无产阶级的民主主义的对立也是一样,它们的更高一级是无国家无政府的时代,经过无产阶级民主去达到它。形式论理的排中律,也是它的同一律的补充,只承认概念的固定状态,反对它的发展,反对革命的飞跃,反对否定之否定的法则。

由此看来,整个形式论理学的规律,都是反对矛盾性,主张同一性,反对概念及事物的发展变化,主张概念及事物的凝固静止,是同辩证法正相反对的东西。

形式论理家为什么这样做?因为他们在事物的联系以外,在事物不间断的相互作用以外去看事物,即在静止中看事物,不在运动中看事物;在割断中看事物,不在联系中看事物。所以他们以为承认事物及概念中的矛盾性及否定之否定的关系,是不可能的,而主张了死板凝固的同一律。

辩证法则不然,在运动中联系中看事物,和形式论理学的同一律针锋相对,主张了革命的矛盾律。

辩证法认为思想上的矛盾不是别的,乃客观外界矛盾的反映。辩证法不拘泥于两条原则外表上似乎相冲突的情形(例如康德所举的四条矛盾原理及上面我所举的许多矛盾思想),而透视到事物内部的本质中。辩证法家的任务,在于做那些形式论理家所不做的工作,向着研究的对象,集中注意于找出它的矛盾的力量、矛盾的倾向、矛盾的方面、矛盾的定性之内部的联系来。客观世界与人的思想都是动的、辩证的,不是静的、形而上学的。革命的矛盾律(即矛盾统一法则)在辩证法中所以占据着最主要的位置,理由就在这个地方。

全部形式论理学只有一个中心,就是反动的同一律。全部辩证法也只有一个中心,就是革命的矛盾律。辩证法是否反对事物或概念的同一性呢?不反对的,辩证法承认事物或概念之相对的同一性。那末,辩证法为什么要反对形式论理学的同一律呢?因为形式论理学的同一律,是排除矛盾的绝对的同一律。辩证法承认事物或概念的同一性,说的是同时包含矛盾,同时又互相联结;这种同一性就是指矛盾之互相联结,它是相对的、暂时的。形式论理的同一律既然是排除矛盾的绝对的同一律,它就不得不提出反对一概念转变到它概念,一事物转变到它事物的排中律。而辩证法却把事物或概念的同一性看作暂时的、相对的、有条件的,而因矛盾的斗争引导事物或概念变化发展的这种规律,则是永久的、绝对的、无条件的。因为形式论理不反映事物的真相,因此辩证法不能容许其存在。科学的真理只有一个,这真理就是辩证法。

(三)矛盾的普遍性

这个问题,有两方面的意义:其一是说,矛盾存在于一切过程中;其二是说,每一过程中存在着自始至终的矛盾运动。这就叫做矛盾的普遍性或绝对性。

恩格斯说:“矛盾就是运动”\footnote{出自《反杜林论》第1编第12节《辩证法。量和质》。参见《马克思恩格斯选集》中文第3版,第3卷,第498页:“运动本身就是矛盾。”}。列宁对于矛盾统一法则所下的定义,说它就是“承认(发见)一切自然(社会和精神也在内)现象和过程中的相互排除的对立倾向。”\footnote{出自《谈谈辩证法问题》。参见《列宁全集》中文第2版,第55卷,第306页:“承认(发现)自然界的(也包括精神的和社会的一切现象和过程具有矛盾着的、相互排斥的、对立的倾向。”}这些意见是对的吗?是对的。一切事物中包含的矛盾方面之相互依赖和相互斗争,决定一切事物的生命,推动一切事物的发展。没有矛盾,就没有世界。因此,这一法则,是最普遍的法则,适用于客观世界的一切现象,也适用于思想现象。它在辩证法中,是一个最根本的、最具有决定意义的法则。

为什么说矛盾就是运动?恩格斯的说法,不是有人反驳过了的吗?这是因为马克思、恩格斯、列宁论矛盾的学说,变成无产阶级革命之最重要的理论基础,因此,引起了资产阶级理论家之拼命的攻击,总想推翻恩格斯这个“运动即矛盾”的定律,举起了他们的反驳,并且搬出了下述的理由。他们说:实在世界中运动的事物,是在各个不同的瞬间,经过各个不同的空间点,当事物处于某一点时,它就占据那一点,到另一点时,又占据另一点。这样,事物的运动是在空间和时间上分成许多段落的,这里没有任何的矛盾;如有矛盾就不能运动。

列宁指出这种说法的全部荒谬性。指出这种说法,事实上把不断的运动,看成在空间和时间上的许多段落,许多静止状态,结果是否定了运动。他们不知事物处于某一个新位置,是因为事物从空间的某一点走到另一点的结果,即运动的结果。所谓运动,就是处于一点,同时又不处于一点。没有这一个矛盾,没有这个连续和中断的统一,动和静,止和行的统一,运动就根本不可能。否定矛盾,就是否定运动。一切自然、社会和思想的运动,都是这样一种矛盾统一的运动。

矛盾,不只是简单的运动形式(例如上述的机械性的运动的基础),而且也是世界一切复杂的运动形式的基础。

生的过程,同它相反的死的过程,不可分的联系着,这不仅在各种有机体的生命中,或有机体内细胞们的生命中,都是如此。新与陈之代谢、生与死之更迭,这一矛盾统一的运动,是一切有机体的生活和发展的必要条件。如果没有这种矛盾,生命现象是不能想象的。

机械学中,任何一种“动作”,都带着内部的矛盾性,引起“反动作”,没有反动作,动作就无从说起。

数学中,任何一个数量都带有内部的矛盾性,都可能成为正数与负数,整数与零数。正数与负数,整数与零数,组成了数学的矛盾运动。

化学中分化化合的矛盾统一律,组成了化学变化的无量的运动,没有这一矛盾,化学现象就不能存在。

社会生活中,任何一种现象,都带有阶级的矛盾性,劳动的买卖如此,国家的组织如此,哲学的内容也是如此。阶级斗争,是阶级社会的根本规律。

战争中的攻守、进退、胜败,都是矛盾现象。失去一方,他方就不存在,双方斗争而又联结,组成了战争的总体,推动了战争的发展。

人的概念之每一差异,都应把它看作客观矛盾的反映。客观矛盾反映人主观的思想,组成了概念的矛盾运动,推动了思想的发展。

党内不同思想之对立与斗争是经常发生的,这是社会阶层的矛盾在党内的反映。党内没有矛盾和解决矛盾的思想斗争,党的生命也就停止了。

不论是简单的运动形式,或复杂的运动形式,不论是客观现象,或思想现象,矛盾是普遍地存在着,矛盾存在于一切过程中。

说到这里,有人要说:可以承认恩格斯同列宁的原则,矛盾即是运动,矛盾存在于一切过程中。但是所谓每一过程中自始至终的矛盾运动,那就未必然罢?不是德波林等人明明说过,在每一过程中并无所谓自始至终的矛盾运动吗?按照德波林的说法,矛盾是存在的,但只存在于过程发展之一定阶段上,不是一开始就在过程中发现。根据德波林,过程的发展循着如次的阶段:开始是简单的差异,随后发生对立,最后成为矛盾。这种公式究竟是对的,还是错的呢?

这是错的。所谓矛盾的普遍性,不但存在于一切过程中,而且存在于每一过程之一切发展阶段中,这才是马克思主义的革命的矛盾律。根据德波林一派,矛盾不是一开始就在过程中出现,须待过程发展到一定阶段才出现,那末,在那一瞬间以前,过程发展的原因不是由于内在矛盾即过程的分裂,而是由于外在原因了。这样,德波林回到形而上学的外因论、机械论去了。拿这种见解去分析具体问题,他们就看见在苏联的条件下工农之间只有差异,并无矛盾,完全同意于布哈林的意见。在分析法国革命时,他们就认为在革命前,工、农资产阶级合组的第三等级中,也只有差异,并无矛盾(郭列夫\footnote{即波里斯·伊萨科维奇·哥列夫(戈尔德曼)〔Борис Исаакович Горев, 1874—1937〕,原为孟什维克,苏联历史学家和哲学家。}的说法)。他们不知道世上的每一差异中就已经包含着矛盾,差异就是矛盾。劳资之间,从两阶级发生的瞬间起,就是互相矛盾的,仅仅没有激化而已。工农之间,即使在苏联条件下,他们的差异就是矛盾,仅仅不会激化成为对抗,不取阶级斗争的形态,不同于劳资间的矛盾,这是矛盾的差别性,而不是有无矛盾的问题。矛盾是普遍的、绝对的,存在于一切过程中,又贯串于一切过程的始终。新过程的发生是什么呢?乃是旧的统一和组成此统一的对立体,让位于新的统一和组成此统一的对立体,新过程就代替旧过程而发生,新过程包含着新矛盾,开始它自己的矛盾发展史。

过程之自始至终的矛盾运动,列宁指出马克思在资本论中模范地应用了这个原则。他指出,这是研究任何过程所必须应用的方法,列宁自己也正确地应用了它,贯彻于他的全部著作中。

“马克思在《资本论》中,首先,分析资产阶级社会(商品社会)之最单纯的、最普遍的、最根本的、最经常的、最日常的、数十亿万回被人亲眼看见的关系——商品交换。在这最单纯的现象之中(资产阶级社会的细胞之中),暴露了现代社会之一切矛盾(或一切矛盾的胚芽)。从那里开始的叙述,把这个矛盾的发展(成长及运动),这个社会的发展,在其个别部分的总和上,自始至终地指示于我们。”

列宁说了上面的话之后,接着说道:“这正是辩证法的一般的叙述方法或研究方法。”\footnote{出自《谈谈辩证法问题》。参见《列宁全集》中文第2版,第55卷,第307页:“这应该是一般辩证法的……叙述(以及研究)方法。”}

好,我们不用读桐城派的古文义法了,列宁告诉了我们更好的义法,这就是马克思主义的科学研究法。

(四)矛盾的特殊性

矛盾存在于一切过程中,矛盾贯串于每一过程之始终,这是矛盾的普遍性与绝对性,前面已经说过了。现在说的,是关于矛盾的特殊性与相对性。这个问题,应从几种情形中研究它。

首先是各种物质运动形式中的矛盾,都带特殊性。人的认识物质,就是认识物质的运动形式,因为除了运动的物质以外,世界上什么也没有。对于每一种运动形式,应当注意它和其它各种运动形式的共同点。但尤其重要的,成为我们认识事物的基础的东西,乃是注意它的特殊点,就是说,注意它同其它运动形式之质的区别。只有注意这点,才有可能区别事物。唯物辩证法指明:任何运动形式,其内部都包含着本身特殊的矛盾。这种特殊矛盾,就构成一事物区别于他事物之特殊的质。自然界存在着许多运动形式,机械运动、发声、发光、发热、电流、化分、化合等等都是。所有这些物质的运动形式,都是互相依存的,又是本质上互相区别的。每一运动形式所具有的特殊的本质,为它自己的特殊矛盾所规定。这种情形,不但自然界,社会现象和思想现象也是一样。每一社会形式和思想形式,都有它的特殊矛盾和特殊本质。

科学研究的区分,就是根据科学对象所具有的特殊矛盾性。因此,对于某一现象领域所特有的某种矛盾之研究,就构成某一门科学的对象。例如数学中的正数与负数,机械学中的作用与反作用,物理学中的阴电与阳电,化学中的化分与化合,社会科学中的生产力与生产关系,阶级斗争,军事学中的攻击与防御,哲学中的唯心与唯物、形而上学观与辩证观等等,都是因为具有特殊矛盾与特殊本质,才构成了不同的科学研究的对象。固然,如果不研究矛盾的普遍性,就无从发现事物运动发展的普遍原因;但如果不研究矛盾的特殊性,就无从确定一事物与他事物的特殊的本质,就无从发现事物运动发展的特殊原因.也就无从辨别事物,无从区分科学研究的领域。

不但要研究每一大系统的物质运动形式之特殊的矛盾性及其所规定的本质;而且要研究每一物质运动形式在其发展的长途中,每一过程的特殊矛盾及其本质。一切运动形式之每一发展过程内,都是不同质的,天下没有同型的矛盾,研究要着重一点。

不同质的矛盾,只有用不同质的方法才能解决。例如无产阶级与资产阶级的矛盾,用社会主义革命的方法去解决;人民大众与封建制度的矛盾.用民主革命的方法去解决;殖民地与帝国主义的矛盾,用民族战争去解决;无产阶级与农民的矛盾,用农业社会化去解决;共产党内的矛盾,用思想斗争去解决;社会与自然的矛盾,用发展生产力去解决,过程变化,旧过程与旧矛盾消灭,新过程与新矛盾发生,解决矛盾的方法也因之而不同。俄国二月革命与十月革命,所用以解决矛盾的方法是根本不同的。用不同的方法去对付不同的矛盾,这是原则。

为要暴露过程中的矛盾在其总体上、在其相互联结上的特殊性,就是说暴露过程的本质,必须暴露过程中矛盾各方面的特殊性,否则暴露过程本质为不可能,这是研究问题要十分注意的。

一个大过程中包含着许多矛盾。例如在中国资产阶级民主革命过程中,有整个中国社会对帝国主义的矛盾,在中国社会内部有封建制度同人民大众的矛盾,有无产阶级同资产阶级的矛盾,有农民小资产阶级同资产阶级的矛盾,有各个统治集团间的矛盾等等,情形是非常复杂的。这些矛盾不但各个有其特殊性,不能一律看待,而且每一矛盾的两方两,又各有其特点,也是不能一律看待的。我们从事中国革命的人,不但要对各个矛盾总体即矛盾之相互联结,了解其特殊性,而且只有从矛盾的各个方面着手研究,才能了解其总体。所谓了解矛盾之各个方面,就是了解它们每一方面各占何等特定的地位,各用何种具体形式同对方发生依存关系,在依存中及依存破裂后又各用何种具体方法同对方作斗争。研究这些问题,乃是十分重要的事情。列宁主义的主要特点,就是研究无产阶级同资产阶级作斗争之各种具体形式的科学。

研究问题,忌带主观性、片面性与表面性。所谓主观性,就是不知道客观地看问题,也就是不知道用唯物的观点去看问题。这一点,第二章中已经说过,本节末尾也还要说。现在来说片面性与表面性。所谓片面性,就是不知道全面地看问题。例如只了解中国一方、不了解日本一方,只了解共产党一方、不了解国民党一方,只了解无产阶级一方.不了解资产阶级一方,只了解农民一方、不了解地主一方,只了解顺利情形一方、不了解困难情形一方,只了解正人君子一方、不了解奸巧狡诈一方,只了解现在一方。不了解将来一方,只了解自己一方、不了解他人一方,只了解骄傲一方、不了解谦逊一方,只了解缺点一方、不了解成绩一方,只了解原告一方、不了解被告一方,只了解秘密工作一方、不了解公开工作一方,如此等等。一句话,不了解矛盾各方的特点。这就叫做片面地看问题,或叫做只看见局部,不看见全体,是不能找出解决矛盾的方法的(是不能完成革命任务的,是不能做好所任工作的,是不能正确发展党内思想斗争的)。孙子论军事说:“知己知彼,百战百胜。”他说的是矛盾的双方。唐太宗也说:“兼听则明,偏听则暗。”也懂得片面性不对。可是我们的同志看问题,往往带片面性,这样的人就往往碰钉子。乡下两家或两族相争,做和事老的,须熟识双方争论的原因、争点、现状、要求等等,才能思出和解的办法来。乡下有那种善于和事的人,遇有纠纷,总请他到,这种人实在懂得我们说的要了解矛盾各方面特点这一条辩证法。水浒传上宋公明三打祝家庄,两次都因情况不明,方法不对,打了败仗。后来改变方法,从调查情形入手,于是熟悉了盘陀路,拆散了李家庄、扈家庄与祝家庄的联盟,并且布置了藏在敌人营盘里的伏兵,第三次就打了胜仗。水浒传上有很多唯物辩证的范例,这个三打祝家庄,算是最好的一例,列宁屡次说到对问题要全面去看,坚决反对片面性,我们应该记得他的话。表面性,是说对矛盾总体与矛盾各方的特点,都不去看,否认深入事物里面精细研究矛盾特点的必要,仅仅远远地望一望,粗枝大叶地看到一点矛盾的形相,就想动手去解决矛盾(答复问题,解决纠纷,处理工作,指挥战争)。这样的干法,没有不出乱子的。不但全过程中矛盾运动在其相互联结上,在其各方面情况上,应该注意其特点;过程发展的各阶段,也有特点,也应该注意。过程的根本矛盾及为此根本矛盾所规定的过程之本质,非到过程完结之日是不会消灭的;但是过程的各个发展阶段,情形又往往互相区别。这是因为过程之根本矛盾的性质及过程的本质虽没有变化,但根本矛盾在各个发展阶段上采取逐渐激化的形式,并且,为根本矛盾所规定的许多大小矛盾中.有些是激化了,有些是暂时地局部地解决了,或缓和了,又有些是发生了,因此过程就显出阶段性来。

例如帝国主义之别于自由资本主义,无产阶级与资产阶级这个根本矛盾的性质及这个社会之资本主义的本质,并没有变;但是两阶级的矛盾激化了,独占资本与自由资本之间的矛盾发生了,各独占集团之间的矛盾发生了,资本输出与商品输出的矛盾发生了,宗主国与殖民地的矛盾激化了,各资本主义国家间的矛盾,即各国不平衡发展状态激化了,因此形成了帝国主义的特殊阶段。

拿从辛亥革命开始的中国民主革命过程的情形来看,也表现了若干特殊阶段。直到这一革命完成为止,也许还要经过若干阶段,虽然整个过程中根本矛盾的性质及过程之反帝反封建的民主革命的本质(其反面是半殖民地半封建的本质),并没有变;但中间经过辛亥失败,北洋军阀统治,第一次民族统一战线建立与大革命,统一战线的破裂与资产阶级的转入反革命,军阀战争,苏维埃战争,东四省丧失,苏维埃战争停止,国民党政策转变,第二次统一战线建立等等大事变,过去二十多年间已经通过了四五个发展阶段。这些阶段中,包含着有些矛盾激化(例如中日矛盾),有些矛盾部份地暂时地解决(例如北洋军阀的消灭,苏区没收地主土地),有些矛盾又重新发生(例如新军阀之间的斗争,苏区丧失后地主又重新收回土地)等等特殊的情形。

研究过程各阶段上矛盾的特性,不但在其联结上、在其总体上去看,也同样要从各个方面去看。

例如国共两党。国民党方面,在第一次统一战线时是革命的、有朝气的,它是各阶级的民主革命联盟。1927年以后,变到相反的方面,成为地主资产阶级的反动集团。西安事变后,又开始向新的方面转变。这就是国民党在三个阶段上的特点。形成这些特点,当然有种种的原因。共产党方面,第一次统一战线时期,它是幼年的党,对于革命的性质、任务、方法等等的认识,均表现了它的幼年性,因此发生了陈独秀主义;但是它领导了第一次大革命。1927年以后,领导了苏维埃战争,在同国际国内敌人斗争中锻炼了自己,创造了苏区与红军,但它也犯过一些政治上军事上的错误。1935年以后,它又领导了统一战线,提出了抗日民族战争与民主共和国的口号。这就是共产党在三个阶段上特点。形成这些特点,也有种种的原因。不研究这些特点,就不能了解两党在各个发展阶段上的特殊的相互关系(统一战线,统一战线破裂,再一个统一战线)。不但两党间,而且更根本的,还有这两个党向其他方面形成矛盾的对立。例如国民党同国外帝国主义的矛盾(有时取妥协形态),同国内人民大众的矛盾。共产党同国外帝国主义的矛盾,同国内剥削阶级的矛盾。由于这些矛盾,所以造成了两党的斗争,又造成了两党的合作。不了解这些矛盾方面的特点,不但不能了解这两个党各个同其他方面的关系,也不能了解两党之间的关系。国民党为什么有与共产党重新合作之可能?就是因为国民党受了日本压迫与人民不满而发生了自己内部变动的原故。

由此看来,不论研究何种矛盾的特性——各个物质运动形式的矛盾,各运动形式在各个发展过程的矛盾,各个发展过程的矛盾之各方面,各个发展过程在其各个发展阶段上的矛盾,以及各个发展阶段上矛盾之各方面,研究所有这些矛盾的特性,都不能带主观随意性,必须以对它们的具体分析为前提。离开具体分析,就决不能认识矛盾的特性。

这种具体分析,马克思、恩格斯给了我们以很好的模范。

当马克思.恩格斯把这一矛盾统一法则应用到社会历史过程的研究时,他们看出社会发展的根本原因,在于生产力和生产关系之间的矛盾,阶级斗争的矛盾,以及由这些矛盾所产生的经济基础同政治及思想的上层建筑之间的矛盾。

马克思把这—法则应用到资本主义社会经济结构的研究时,他看出这一社会的基本矛盾在于生产的社会性和占有的私人性之间的矛盾。这个矛盾,表现于在个别企业中生产的有组织性和在全社会中生产的无组织性之间的矛盾。这个矛盾的阶级表现则是资产阶级与无产阶级之间的矛盾。

马克思、恩格斯对于应用辩证法到客观现象的研究时,是这样不带任何主观随意性,而从客观现象的实际运动所包含的具体条件,去看出这些现象中的具体矛盾,矛盾各方面之具体的地位,矛盾之具体的相互关系等等。这种研究态度,是我们应当学习的,舍此便没有第二种研究法。

矛盾的普遍性与矛盾的特殊性之关系,就是矛盾的共性与个性之关系。其共性是矛盾存在于一切过程中,贯串于一切过程的始终,矛盾即是运动,即是事物,即是过程,即是世界,也即是思想。否认矛盾就是否认了一切。这是共通的道理,古今中外,概莫能例外。所以它是共性,是绝对性。然而这种共性,即包含于个性之中,共性表现于一切个性之中,无个性之存在,也就不能有共性之存在。假如除去了一切个性,还有什么共性呢?因矛盾之各各特殊,大宇长宙,无一同者,变化无穷,其存在也暂,所以是相对的。苏东坡说,“自其变者而观之,则天地曾不能以一瞬”。照现在的意思来说,可以说他说的是矛盾的特殊性,相对性。“自其不变者而观之,则物与我皆无尽。”说的是矛盾的普遍性,绝对性。这一共性个性、绝对相对的道理,是矛盾学说的精髓,懂得了它,就可以一通百通。古人所谓闻道,以今观之,就是闻这个矛盾之道。

(五)主要的矛盾与主要的矛盾方面

在矛盾特殊性问题中,有两种情形应该特别提出研究的,这就是主要的矛盾与主要的矛盾方面。

在复杂的过程中.有许多矛盾存在,其中一个是主要矛盾,由于它的存在与发展,规定或影响其他矛盾的存在与发展。

例如资本主义社会中,无产阶级与资产阶级的矛盾是主要的矛盾;其他如残存的封建势力与资产阶级的矛盾,农民小资产者与资产阶级的矛盾,无产阶级与农民小资产者的矛盾,自由资产阶级与金融资产阶级的矛盾,资产阶级民主主义与法西斯主义的矛盾,资本主义国家相互间的矛盾,帝国主义与殖民地的矛盾,以及其他矛盾等等,都为这个主要矛盾所规定、所影响。

半殖民地的社会如中国,其主要矛盾与非主要矛盾的关系呈现着复杂的情况。当半殖民地没有遭受帝国主义压迫时,其主要矛盾是封建或半封建制度与人民大众的矛盾,一切其他矛盾都受这个主要矛盾所规定。但当这种社会遭受帝国主义压迫时,内部的主要矛盾能够暂时地转化到非主要地位,而帝国主义与整个或差不多整个半殖民地社会之间的矛盾,能够占据主要的地位,规定一切其他矛盾的发展。这种时候,依帝国主义压迫及半殖民地人民革命的程度,而变化着或主要或非主要矛盾的地位。

例如当帝国主义向这种国家举行侵略战争,这种国家的内部各阶级,能够暂时地团结起来进行民族战争去反对帝国主义。这时,帝国主义与这种国家之间的矛盾成为主要矛盾,而这种国家内部各阶层的一切矛盾(包括封建制度与人民大众之间这个主要矛盾在内),便都暂时地降到次要与服从的地位。中国的鸦片战争,义和团战争,甲午中日战争,目前的中日战争,在外国,有美国的独立战争,南非洲同英国的战争,菲律宾同西班牙的战争等等,都是如此。

然而在另一种情形,则矛盾的地位起了变化。当着帝国主义不用战争压迫而用政治、经济、文化的形式进行比较温和的压迫,半殖民地国家的统治阶级就向帝国主义投降,二者之间结成同盟,由二者的对抗变成二者的统一,共同压迫人民大众。这时,人民大众往往采取用国内战争的形式,去反对帝国主义与封建阶级的联盟,而帝国主义则往往采取秘密援助国内的统治阶级压迫国内的革命战争,而不直接行动,显出了内部矛盾的特别尖锐性。例如中国的太平军战争,辛亥革命,1925—1927年的大革命,1927年以后的苏维埃战争。在外国,则有俄国的二月革命与十月革命(俄国也带了若干半殖民地性),中美洲南美洲若干带革命性的内战等等的情形,都是如此.还有半殖民地各统治集团之间的内战,也表现了内部矛盾尖锐的情况。在中国,在中南美,也是很多的,也属这一类。

当着国内战争发展到根本威胁帝国主义及其走狗国内统治者的存在时,帝国主义就往往采取上述方法以外的方法,企图维持其统治;或者分化革命阵线的内部,例如1927年中国资产阶级的叛变,或者直接出兵援助国内统治者,例如苏联内战的末期,今日的西班牙战争。这时,帝国主义与国内封建阶级乃至资产阶级完全站在一个极端,人民大众则站在另一极端,这时帝国主义与全殖民地之间这个外部的主要矛盾,封建势力与人民大众之间这个内部的主要矛盾,就几乎合并起来,成为一个主要矛盾,而规定其他矛盾的发展地位,情形是非常明显的。

然而不管怎样,过程发展之各个阶段中,只有一个主要矛盾起着领导的作用,是完全没有疑义的。

由此可知,任何过程如果有多数矛盾的话,其中必定有一个是主要的,起着领导的、决定的作用,其他则处于次要与服从的地位。因此,研究任何过程,首先要弄清它是单纯的过程,还是复杂的过程。如果是存在着二个以上矛盾的复杂过程的话,就要用全力找出它的主要矛盾。捉住了这个主要矛盾,一切问题就迎刃而解了。这是马克思研究资本主义社会告诉我们的方法。列宁研究帝国主义时,列宁和斯大林研究苏联过渡期经济时,也同样告诉了我们这种方法。万千的学问家、实行家,不懂得这种方法,结果如堕烟海.找不到中心,也就找不到解决矛盾的方法。

不能把过程中的矛盾平均看待,应把它们区别为主要的与次要的两类,着重于捉住主要矛盾,既如上述。但是矛盾之中,不论主要的或次要的,矛盾着的两个方面或侧面又是否可平均看待呢?也是不可以的。无论什么矛盾,也无论在什么时候。矛盾的方面或侧面,其发展是不平衡的。有时候似乎势均力敌,然而这是暂时的与相对的情形,基本形态则是不平衡,就是在似乎平衡之时,实际上也没有绝对的平衡。矛盾着的两方面中,必有一方是主要的,他方是次要的。其主要方面,即所谓矛盾起主导作用的方面。

然而这种情形不是固定的,矛盾的主要与非主要的方面互相转化着。在矛盾发展的一定过程或一定阶段上,主导方面属于甲方,非主导方面属于乙方;及到另一发展阶段或另一发展过程时,就互易其位置,这是依靠双方斗争的力量来决定的。

例如资本主义社会,在长时期中,资产阶级处于主要地位,起着主导的作用,无产阶级则服从之;但到革命前夜及革命之后,无产阶级就转化到主要地位,起着主导的作用,而资产阶级作了相反的转化。十月革命前后的苏联就是这种情形。

在资本主义社会中,资本主义已从过去封建社会时的附庸地位,转化成了主要力量,封建势力则由主要化为附庸。但何以解释日本及革命前的俄国呢?他们依然是封建势力占着优势,资本主义尚不起决定一切的作用。这是因为他们的矛盾方面尚未完成其决定的转化的原故。这种转化,因为时代的关系,已经不能走历史的老路,而为另一种情形的转化所代替,即是把地主阶级与资产阶级整个儿地转到被统治的地位,而由无产阶级与农民起来占据主导的方面。目前一切尚未完成资本主义转化的国家(中国也在内)都将走向这条新路,虽然并不跳过民主革命的阶段,可是这种革命是由无产阶级领导执行的。

帝国主义与整个中国社会的矛盾中,主导的方面属于前者,它在双方斗争中占着优势。然而事情也正在变化,在彼此对立的局面中,中国一方正由被压迫地位向自由独立的地位转化,而帝国主义则将转化到被打倒的地位。

中国国内封建势力同人民大众对抗的情况也正在变化,人民将依靠革命斗争把自己转化为主要与统治的力量。过去已有过例证,这就是南方革命势力由次要地位转化到主要地位,而北洋军阀则作了相反的转化。苏区中也有此种情形,农民由被统治者转化为统治者,地主则作了相反的转化。

以中国无产阶级与资产阶级的关系而言,资产阶级因握有生产手段与统治权,至今还居于主导地位,然在反帝反封建的革命领导上说来,由于无产阶级觉悟程度与革命的彻底性,却较之动摇的资产阶级反居于主导地位,这一点将影响到中国革命之前途。无产阶级要在政治上物质上都居于主导地位,只有联合农民与小资产阶级。果能如此,革命之决定的主导的作用就属于无产阶级了。

在农民与工人的矛盾中,目前工人的主导地位,曾经是由附庸地位转化而来,而农民作了相反的转化。在产业工人与手工工人的矛盾中,在熟练工人与非熟练工人的矛盾中,在城市与乡村的矛盾中,在劳心与劳力的矛盾中,在唯物论与唯心论的矛盾中,都作了同样的转化。

革命斗争中某些时候,困难条件超过顺利条件,这时,困难是矛盾的主要方面,顺利是其次要方面。然而由于革命党人的努力,利用已有的若干顺利条件作基础,能够逐渐克服困难,开展顺利的新局面,困难的主导地位转化到以顺利为主导。1927年革命失败后的情形,红军长征中的情形,都是如此。今日的中日战争,中国又处在十分困难的地位,但我们应该也能够努力于它的转变。在相反的情形,顺利也能转化为困难,如果是革命党人犯了错误的话。1925至27年的大革命的胜利转化为失败,中央苏区一、二、三、四次战争粉碎围剿的胜利转化为五次围剿的失败,等等皆是。

研究学问时,由不知到知的矛盾也是如此。没有研究马克思主义的人,不知或知之不多是矛盾的主要方面,精深博大的马克思主义则是矛盾的另一方面,然而由于学习的努力,可以由不知转化到知,由知之不多转化到知之甚多,我们的许多同志正是走的这条路。在相反的情形也一样,如果中途拒绝前进,或甚至想入非非走了邪路,已有的知可以化为不知,正确可以化为错误。考茨基、普列哈诺夫\footnote{即普列汉诺夫。}、陈独秀等人就是走了这条路。我们队伍中的若干自大主义者,如果他不改变,也有这种危险。

据我看来,一切矛盾方面之主导与非主导的地位,都是这样互相转化的。

有人觉得有些矛盾并不是这样。例如生产力与生产关系的矛盾,生产力是主导;理论与实践的矛盾,实践是主导;经济基础与上层建筑的矛盾,经济基础是主导。如此等等,它们并不互相转化。须知这是就一般情形而言,站在唯物论的基点上,它们确是不转化的绝对的东西。然而就历史上许多特殊情形而言,它们仍在转化着。生产力、实践、经济基础,一般表现主导的决定的作用,谁不承认这一点,谁就不是唯物论者。然而,生产关系、理论、上层建筑这些方面,有时亦表现其主导的决定的作用,这也是应该承认的。当着不变更生产关系,生产力就不能发展之时,生产关系的变更就起了主导的决定的作用。当着如同列宁所说的“没有革命理论,就没有革命运动”\footnote{出自《俄国社会民主党人的任务》以及《怎么办?》第1章第4节。分别参见《列宁全集》中文第2版,第2卷,第443页;第6卷,第23页:“没有革命的理论,就不会有革命的运动。”}之时,革命理论的提倡就起了主导的决定的作用。当着某一件事情(任何事情都是一样)要做,但还没有方针、方法、计划或政策之时,确定方针、方法、计划或政策,也就是主导的决定的东西。当着政治文化等等上层建筑阻碍着经济基础的发展时,对于政治文化上面的革新就成为主导的决定的东西了。这样来说,是否违反唯物论呢?不违反的。因为我们承认总的历史发展中是物质的东西决定精神的东西;但同时又承认而且应该承认,精神的东西之反作用。这不是违反唯物论,而正是避免机械唯物论,坚持了辩证唯物论。

在研究矛盾特殊性问题中,如果不研究过程中主要矛盾与非主要矛盾,及矛盾之主要方面与非主要方面这两种情形,也就是说,研究这两种的差别性,那就仍将陷入于抽象的研究,不能具体地懂得矛盾,因而也不能找出解决矛盾的正确方法来。这两种差别性或特殊性,都是矛盾的不平衡性。世界没有绝对地平衡发展的东西,所以成其为世界,我们应该反对平衡论(或均衡论)。矛盾之各种不平衡中,对于主要与非主要的矛盾、主要与非主要的矛盾方面之研究,成为革命政党正确决定其政治上战略战术的基本方法之一(军事上也是一样)。所以不能不充分注意这个问题。

(六)矛盾的同一性与斗争性

在解决了矛盾的普遍性与特殊性的问题之后,必须进而研究矛盾的同一性与斗争性的问题,矛盾统一律的研究才算全部地解决了。

同一性、统一性、一致性、互相渗透、互相贯通、互相依赖(或依存)、互相联结或互相合作,这些不同的名词都是一个意思,说的是如下两种情形:第一、过程每一矛盾的两方面,各以它方面为自己存在的前提,共处于一个不可分的统一体中;第二、矛盾的双方依据一定条件,各向着其相反的方面转化。这些就是所谓同一性。

列宁说:“辩证法是关于矛盾怎样能够是同一性,又怎样是同一性(怎样变成同一性),在怎样的条件之下矛盾变成同一性而互相转化。为什么人的思想不把这些矛盾当作死的、凝固了的东西去看,却当作生动的、附条件的、可变动的、互相转化的东西去看等等问题的学说。”\footnote{出自列宁《黑格尔〈逻辑学〉一书摘要》。参见《列宁全集》中文第2版,第55卷,第90页:“辩证法是一种学说,它研究对立面怎样才能够同一,是怎样(怎样成为)同一的——在什么条件下它们是相互转化而同一的,——为什么人的头脑不应该把这些对立面看作僵死的、凝固的东西,而应该看作活生生的、有条件的、活动的、彼此转化的东西。”}

列宁这句话,说的是什么意思呢?

一切过程中矛盾着的各方面,本是互相对立的,是彼此不融洽、不对头、不相好、不和气的,都是些充满怨气的冤家。世上一切过程、现象、事物、思想里面,都包含着选样带冤家性的方面,没有一个例外。单纯的过程只有一对冤家,复杂的过程却有二对以上的冤家。各对冤家之间,又互相成为冤家。这样组成过程、现象、事物,并推使发生运动。

如此说来,只是极不同一,极不统一,怎样又说是同一或统一呢?世事之怪就怪在这里,妙也就妙在这里。

原来矛盾的各方面,不能孤立存在。假如没有冤家一方,它自己这方就失掉了存在的条件。试想一切矛盾的事物,或人的心中矛盾的概念,矛盾的任何一方面能够独立存在吗?不能够的。没有生,死就不见;没有死,生也不见。没有上,就无所谓下;没有下,也无所谓上。没有祸,就无所谓福;没有福,就无所谓祸。没有顺利,就无所谓困难;没有困难,也无所谓顺利。没有资产阶级,就没有无产阶级;没有无产阶级,也没有资产阶级。没有殖民地,就不能有帝国主义的压迫;没有帝国主义的压迫,也就不能有殖民地。一切过程、现象、事物之内的对立,对立的双方都是这样,因一定的条件,一面互相对立,一面又互相联结、互相贯通、互相渗透、互相依赖、互相勾搭、又是冤家又聚头,这种性质,叫做同一性。一切矛盾都因一定条件具备着不同一性,所以称为矛盾。然而又具备着同一性,所以互相联结。列宁所谓辩证法研究怎样能够是同一性,就是说的这种情形。这是同一性的第一个意义。

然而单说了矛盾双方互为存在条件,双方之间有同一性,因而能够共处于一个统一体中,这样就够了吗?那是不够的。事情不是矛盾互相依存就完了,还没有完,重要的事情,还在矛盾之互相转化。事物内部矛盾的方面,因一定的条件而向着相反的方面转化了去,这就是矛盾的同一性之第二个意义。

为什么这里也有同一性呢?你看,生死关系中生向死转化,不论是有机体中或有机体内细胞的生命中,生总不能长久,而在一定条件下走向它的反对方面,变为死。死呢?也不是一死完事,又必在一定条件下产出新生命来,死变为生。试问如果没有联系、没有瓜葛、没有亲属关系,就是说没有同一性,为什么生死这样相反的东西之间能够互相转化呢?

被压迫被剥夺的无产阶级向着无产阶级专政,即不再被压迫、不再被剥夺的方面转化,而资产阶级却经过阶级崩溃转到受无产阶级国家统治方面。苏联已经这样做了,全世界也都将要这样做。试问其间没有在一定条件之下的联系与同一性,如何能够发生这样的变化?

帝国主义压迫殖民地与殖民地受帝国主义压迫的命运都不能长久,帝国主义者将要被殖民地人民与他本国人民的革命势力所推翻而站在人民的统治之下。殖民地和帝国主义内部的人民呢?却有解除压迫走到自由解放(被压迫的反面)之一日,二者之间由于一定条件有共同性、同一性。

1927年大革命的正规战争,转化为苏维埃的游击战争;开始时期的苏维埃游击战,又转化为后来的正规战争;今后又正在由苏维埃战争向着抗日的民族战争转化了去。这其间都因在一定条件下而发生同一性,相反的东西中间互相渗透、贯通、勾搭着。

国民党的带革命性的三民主义,因为它的阶级性及帝国主义的引诱(这就是条件),在1927年以后转化成为反动政策。又由于中日矛盾的尖锐化及共产党的统一战线政策(这也是条件),而被迫着转向抗日救亡的方面去。矛盾的东西这一个变到那一个,其间包含了这样的同一性。

苏区的土地革命,已经是并将要是这样的过程:拥有土地的地主阶级转化成为失掉土地的阶级,而曾经是失掉土地的农民却转化到取得士地的小私有者。有无、得失之间,因一定条件而互相联结,变为同一性。在社会主义条件之下,农民的私有制又将转化到社会主义农业的公有制,苏联已经这样做了,我们将来也会要这样做。私产与公产之间有一条由此达彼的桥梁,哲学上名之曰同一性,或互相渗透。

资产阶级民主主义与无产阶级民主主义是相反的,然而前者必会转化为后者。相反的东西中间,在一定条件下,就产生了相成的因素。

提高民族文化,正是准备转化到国际文化的条件。争取民主共和国,正是准备取消民主共和国转向新的国家制度的条件。巩固无产阶级专政,正是准备取消这种专政走到消灭任何国家制度的条件。建立与发展共产党,正是准备消灭共产党及一切党派的条件。建立革命军进行革命战争,正是准备了永远消灭战争的条件。这许多相反的东西,却同时又是相成的东西。

有些人说:共产党是国际主义者,决不会也不能同时又是爱国主义者。我们却宣称:我们是国际主义者,但同时因为我们是殖民地的党(条件),所以必须为着保卫祖国反对帝国主义的压迫而斗争,因为必须首先脱离帝国主义的压迫,才能参加世界的共产主义社会,这就使二者构成了同一性。爱国主义与国际主义,在一定条件下,相反而又相成。为什么帝国主义国家的共产党坚决反对爱国主义,因为那里的爱国主义只同资产阶级的利益有同一性,它同无产阶级的利益则是根本相反的。

有些人说:共产党不会也不能,同时又相信三民主义。我们却宣称:我们是坚持共产主义的党纲的,但是当前阶段的共产主义运动不是别的,正是坚决领导反帝反封建的民族民主运动(这就是条件),因此我们不但不反对,而且早已执行了真正的三民主义纲领(反帝的民族主义,工农苏维埃的民权主义,土地革命的民生主义);并且十年来真正的三民主义传统也仅仅在于共产党一方面。国民党除若干分子如宋庆龄、何香凝、李锡九等人而外,抛弃了这个传统。共产党的民主革命政纲不与真正的三民主义冲突,而且就是彻底的急进的三民主义,我们将经过民主阶段转变到共产主义。三民主义与共产主义不是一个东西,二者矛盾着,现在阶段与将来阶段不是一个东西,二者矛盾着,但是相反而又相成,因一定的条件造成了同一性。

还可以说一些最眼前的事情。战争与和平是矛盾的,但又是联结的。战争转化为和平(例如第一次大战转化为凡尔赛条约,中国的国内战争在西安事变后转化为国内和平),和平转化为战争(目前的世界和平是暂时的,即将转化为第二次大战;日本侵略东四省后几年的和平是暂时的,现已开始转化为大陆战争)。为什么?因为在一定条件下具备了同一性。中国无产阶级同资产阶级订立抗日的统一战线,这是矛盾的一方面;无产阶级须得提高政治的警觉性,密切注视资产阶级的政治动摇及其对于共产党的腐化作用与破坏作用,以保证党与阶级的独立性,这是矛盾的又一方面。各党的统一战线与各党的独立性,这样矛盾着的两方面,组成了当前的政治运动,两方面中去掉一方面,就没有党的政策,就没有统一战线了。我们给人民以自由,这是一方面,我们又给汉奸卖国贼破坏者以压制,这是又一方面。自由与不自由二者因一定条件而联系着,缺一就不行,这是矛盾的统一或同一性。共产党、苏维埃,以及我们主张的抗日政府之组织形式,是民主集中制的。它们是民主的,但又是集中的,二者矛盾而又统一着,因为在一定条件之下有同一性。苏联的无产阶级民主专政,我们过去十年的工农民主专政,它们是民主的,对于革命阶级而设;它们又是专制的(或叫独裁的),对于反革命阶级而设,极端相反的东西之间有同一性。

军队的休息、训练,同时就是作战胜利的条件。“养兵千日”,正是为了“用在一朝”。分开前进,同时就是到达协同攻击的条件(分进合击)。退却与防御,同时就是为着反攻与进攻(以退为进,以守为攻)。迂回不是别的,就是最有效地消灭敌人的方法(以迂为直)。向东方打一打,为的要在西方得手(声东击西)。分兵以争取群众,为了便于集中以消灭敌人;集中以消灭敌人,为了便于分兵以争取群众。要坚决执行命令,又容许在统一意图下有机动的自由。要严格执行纪律,又要发扬自觉自动性。允许陈述个人志趣,但最后还是要服从团体的决定。前方工作要紧,但后方工作不能抛弃不顾。身体不好需要调养,但紧张时候又要讲牺牲。谁不赞成生活优裕,但经济困难却要准备吃苦。军事操练是重要的,非此不能破敌;但政治工作又重要,非此也就要打败仗。老兵、老干部经验丰富,是值得宝贵的;但如果没有新兵、新干部,战争与工作就不能继续。勇猛要紧,也还要智谋;张飞虽不错,到底不如赵子龙。自己领导的局部工作是重要的;但他人领导的局部及全体工作也重要或更重要,小团体主义是不正确的。自己的意见与团体的或上级机关的意见相矛盾时,可以而且应该陈述自己的意见,可是绝不容许在自己意见未被团体或上级批准时,向任何其他人员自由发表;或甚至煽动下级人员反对上级的意见。这种少数服从多数,下级服从上级的纪律,是共产党与红军的起码的纪律。“良药苦口利于病,忠言逆耳利于行”;“祸兮福所倚,福兮祸所伏”;“爱而知其恶,恶而知其美”。顾前不顾后,叫做莽夫。知一不知二,未为贤者。

一切矛盾的东西,互相联系着,不但在一定条件之下共处于一个统一体中,而且在一定条件之下互相转化,这就是矛盾的同一性之全部意义。列宁所谓怎样是同一性,在怎样条件之下变成同一性而互相转化,就是这个意思。

“为什么人的思想不把这些矛盾当作死的、凝固了的东西去看,却当作生动的、附条件的、可变动的、互相转化的东西去看”呢?因为客观事物本来是如此的。客观事物中矛盾的统一或同一性,本来不是死的、凝固的,而是生动的、附条件的、可变动的、暂时的、相对的东西,一切矛盾都依一定条件向它们的反面转化着。

为什么鸡蛋转化为鸡子,而石头不能转化为鸡子呢?为什么战争与和平有同一性,而战争与石头却没有同一性呢?为什么人能生人却不能生狗呢?没有别的,就是因为矛盾的同一性要在一定条件之下,缺乏一定的必要的条件,就没有任何的同一性。

为什么俄国的民主革命与社会主义革命直接地联系着,而法国的民主革命没有直接联系社会主义革命,巴黎公社到底失败了呢?为什么外蒙古与中亚细亚的游牧制度又直接与社会主义联系了呢?为什么中国的革命可以避免资本主义前途,可以同社会主义直接联系起来,而避免再走英美法等的历史老路呢?为什么俄国1905年的革命同中国1911及1927年的革命都不与革命的胜利联系,却与失败联系了呢?为什么拿破仑一生的战争大都与胜利联系着,而滑铁炉\footnote{原文如此,即滑铁卢。}一战却军败身俘一蹶不振呢?为什么“可以修一条铁路往新疆,却不能修一条铁路往月球”呢?为什么德苏亲交变为敌视,而法苏敌视却又变为暂时的亲交呢?所有这些问题,没有别的,都是当前的具体条件的问题。一定的必要的条件具备,过程就发生矛盾,而且矛盾互相依存,又互相变化,否则一切都不可能。唐吉诃德的奋力同风车作战、孙悟空的十万八千里的筋斗云、阿丽斯的漫游奇境、鲁滨孙的漂流孤岛、阿Q的精神胜利、希特勒的世界统治、黑格尔的绝对精神、布哈林的均衡论、托洛茨基的不断革命、御用学者的思想统一、陈独秀的机会主义、亲日派的唯武器论以及中国古代传说中的杞人忧天、夸父追日等等,都不能成为矛盾的同一性,不能成为具体的矛盾,反在人间添些麻烦与笑话的资料,也就是这个道理。

同一性的问题如此。那末,什么是斗争性呢?同一性同斗争性的关系怎样呢?

列宁说:“矛盾的统一(合致,同一,均势),是有条件的、一时的、暂存的、相对的。互相排除的斗争则是绝对的,发展运动是绝对的”\footnote{出自列宁《谈谈辩证法问题》。参见《列宁全集》中文第2版,第55卷,第306页:“对立面的统一(一致、同一、均势)是有条件的、暂时的、易逝的、相对的。相互排斥的对立面的斗争是绝对的,正如发展、运动是绝对的一样。”}。这话怎讲?

一切过程都有始有终,一切过程都转化为它们的对立物。一切过程的常住性是相对的,但是一种过程转化为他种过程则是绝对的。矛盾的统一、同一、一致、常住性、联合性,被包含于矛盾的斗争之中,成为矛盾斗争之一因素.这就是列宁这句话的意思。

这就是说,单只承认矛盾引起运动是不够的,还须明白矛盾在那些状态引起运动。

矛盾在第一种统一(同一)状态引起运动,那是运动的特殊状态,日常生活中叫做静止、有常不变、不动、死、停顿、僵局、相持、和平、平衡、均势、调和、妥协、联合,等等,这些都是相对的、暂时的、有条件的。还须承认矛盾在第二种统一状态引起运动,即运动之一般状态。这就是统一物的分裂、斗争、生动、无常、活跃、变化、不和平、不平衡、不调和、不妥协、甚至冲突、对抗、或战争,这是绝对的。同一、统一、静、死等等相对的矛盾状态,包含于绝对的斗争的矛盾状态中,因为斗争贯彻于过程的始终,贯彻于一切过程之中,所以成其为绝对的东西。不懂这个道理,就是形而上学、机械论,实质上拒绝了辩证法。

国际间的和平条约是相对的,国际间的斗争是绝对的。阶级间的统一战线是相对的,阶级间的斗争是绝对的。党内思想上的一致是相对的,党内思想上的斗争是绝对的。自然现象中的平衡、凝聚、吸引、化合等等是相对的,而不平衡、不凝聚、排斥、分解等等是绝对的。当着过程在和平条约、统一战线、团结一致、平衡、凝聚、吸引、化合等等状态之时,矛盾与斗争也仍然存在着,不过没有取激化的形式,并不是没有了矛盾,停止了斗争。由于斗争,不绝地破坏一个相对状态而转到另一个相对状态去,破坏一种过程而转到另一种过程去,这种无所不在的斗争性,就是矛盾的绝对性。

前面我们说,两个相反的东西中间有同一性,所以二者能够共处于一个统一体中,又能够互相转化,这是说的条件性,即谓在一定条件之下,矛盾的东西能够统一起来,又能够互相转化;无此一定条件,就不能成为矛盾,不能共居,也不能转化。由于一定的条件才构成了矛盾的同一性,所以说同一性是有条件的、相对的。这里我们又说,斗争贯彻于过程的始终,并使一过程向着他过程转化,斗争无所不在,所以说斗争性是无条件的、绝对的。

有条件的相对的同一性与无条件的绝对的斗争性相结合,就构成了一切事物的矛盾运动。

为明了这一论点,下面再举出生死关系及劳资关系以为例。

有机体中旧细胞的死亡,是新细胞产生的前提,也是生活过程的前提。这里生死两矛盾方面互相统一于有机体中,而又互相转变着。生细胞变为死细胞,死细胞变为生细胞(生细胞从死细胞中脱胎出来)。但这种生死的统一,生死的共处于有机体中,是有条件的、暂时的、相对的。而生死的不并存,互相排斥、斗争、否定、转化、则始终如此,它是无条件的、永久的、绝对的。有机体中生的原素总是不绝地战胜死的原素,并且统治着死的原素,表示了斗争的绝对性。生,在一定条件之下转化为死;死,又在一定条件之下转化为生。这种条件使生死有同一性,能够互相转化。由于生死二矛盾物互相斗争,使得生必然转化为死,死必然转化为生。这种必然性,是无条件的、绝对的。由此看来,必须在一定的发展阶段上,必须有一定的温度环境等等条件,生死才能互相转变,互相有同一性。这是一个问题。所谓使生或死都带暂时性、相对性、就是条件不变也不能长生或长死,原因在于两者的斗争、否定、互相排除,这种情形是永久的、绝对的。这又是一个问题。

无产阶级替资产阶级创造剩余价值,资产阶级剥削无产阶级的劳动力,这是一个决定资本主义生存的统一的过程,劳资双方互为存在的条件。然而这种条件有一定的限度,就是要资本主义发展还在一定限度之内,过此限度,统一的过程发生破裂,出现了社会主义的革命。这种破裂是突然发生的,但又不是突然发生的,是从两阶级存在的一天起就开始准备着,双方的斗争是不断的,由此准备了突变。由此看来,两阶级的共存,由于一定的条件而保存着,这种一定条件下的共存,造成了两阶级的统一或同一性。两阶级又在一定的条件之下互相转化,使得剥削者被剥削、被剥削者变为剥削剥削者,资本主义社会变为社会主义社会,二矛盾物有一定条件下的同一性。这是一个问题。双方总是斗争,不但在统一体中是斗争的,尤其是革命的斗争,这种不可避免的状态是无条件的、绝对的、必然的。这又是一个问题。

在同一性中存在着斗争性。拿着列宁的话来说,叫做在“相对中存在着绝对”\footnote{出自《谈谈辩证法问题》。参见《列宁全集》中文第2版,第55卷,第307页:“相对中有绝对”。}。因此,矛盾的统一本身,也就是矛盾的斗争之一种表现或一种因素.这就是我们对于这个问题的结论。

根据这种结论,所谓阶级调和论与思想统一论是否还有立足之余地,也就不言而喻了。国际的阶级调和论,形成了各国工人运动中的机会主义派。他们没有别的作用,单单充当了资产阶级的走狗。中国也有所谓阶级调和论,却从资产阶级改良主义的口中唱出来,他们的目的也不为别的,在于专们欺骗无产阶级,使之永为资产阶级的奴隶。所谓思想统一的滥调,则由若干直接间接依靠官场吃饭、“文人学者”们吹打出来,目的无非在抹煞真理,阻碍革命的前进。真的科学的理论不是这些调儿,而是唯物辩证法的矛盾统一律。

(七)对抗在矛盾中的地位

在矛盾斗争性问题中,包含着对抗是什么的问题。我们回答道:一切过程是自始至终存在着矛盾的,矛盾的双方之间也是自始至终存在着斗争的。对抗是斗争的一种形式,不是一切矛盾都有,而是某些矛盾在其发展过程中到达了采取外部物体力量的形式而互相冲突时,矛盾的斗争便表现为对抗,对抗是矛盾斗争的特殊表现。

例如剥削阶级同被剥削阶级之间的矛盾,无论在奴隶社会也好,封建社会也好,资本主义社会也好,互相矛盾的两阶级长期并存于一个社会中,它们互相斗争着,但要待两阶级的矛盾发展到一定阶段时,双方才取外部对抗的形式,此时社会破裂,革命的战争就出现了。

炸弹的爆炸,小鸡的出卵,动物的脱胎,都是矛盾物共居于一个统一体中,待至一定时机,才取冲突、破局、决裂的形式。

各国之间的和平共居,乃至社会主义国家同资本主义国家也是一样,矛盾与斗争无日不存在,但战争却要在一定发展阶段上才能出现。

苏联的新经济政策容许资本主义成分的相当发展,列宁认为那时有在无产阶级专政下利用国家资本主义的可能。就是说,利用某些资产阶级成分发展生产力,同时使之受苏维埃法律的支配,并随时限制和排斥他们,这时社会主义与资本主义两矛盾体共处于社会主义社会之内,互相斗争,又互相联系。待到消灭富农及消灭资本主义残余的任务提出之后,两种成分的并存就成为不可能,而生死斗争的外部对抗形式就发生了。

国共两党第一次统一战线的情况也是如此。

然而许多过程、现象、事物中的矛盾是不发展成为对抗的。

例如共产党内正确思想与错误思想的矛盾,文化上先进与落后的矛盾,经济上城市与乡村的矛盾,生产力与生产关系的矛盾,生产与消费的矛盾,交换价值与使用价值的矛盾,各种技术分工的矛盾,阶级关系中工农的矛盾,自然界中的生与死、遗传与变异、寒与暑、昼与夜等类的矛盾,者没有对抗形态的存在。

布哈林把矛盾和对抗同一看待,因此,认为在完成了的社会主义社会中,对抗没有了,矛盾也没有了。列宁回答道:“这是极端不正确的,对抗和矛盾断然不同。在社会主义下,对抗消灭了,矛盾存在着。”\footnote{出自《在尼·布哈林〈过渡时期经济学〉一书上作的批注和评论》。参见《列宁全集》中文第2版,第60卷,第281—282页:“极不确切。对抗和矛盾完全不是一回事。在社会主义下,对抗将会消失,矛盾仍将存在。”。}布哈林是否认事物发展由于内部矛盾的推动之均衡论者,认为社会主义下没有矛盾,社会也可发展。

托洛茨基从另一极端出发,也把矛盾和对抗同一看待。因此,认为在社会主义下,工农之间不但存在着矛盾,而且将发展到对抗,如同劳资间的矛盾一样,只有用革命的方法才能解决。然而苏联却用农业社会化的方法解决了,并且是在一国社会主义的情况下解决了,无须如托派所谓要待至国际革命之时。

布哈林把矛盾降低到消灭,托派把矛盾提升到对抗,右倾与左倾的两极端,都不了解矛盾的问题。

解决一般矛盾的方法与解决对抗的方法是根本不同的,这是矛盾特殊性与解决矛盾的方法之特殊性应该有具体认识的问题。凡对抗都包含矛盾性,但凡矛盾不一定都取对抗的形态,总的区别就在这里。

×  ×  ×  ×

矛盾统一律是宇宙的根本法则,也是思想方法的根本法则。列宁称之为辩证法的核心。它是与形而上学的发展观相反的,它是与形式论理学的绝对的同一律相反的。矛盾存在于一切客观与主观事物的过程中,矛盾贯彻于一切过程的始终,这是矛盾的普遍性、绝对性。矛盾及矛盾的侧面各有其特点,人心之不同如其面,矛盾之不同如其形,这是矛盾之特殊性、相对性。矛盾着的东西依一定的条件有同一性,因此能够共居于一个统一体中,又能够互相转化到相反方面、这又是矛盾的特殊性、相对性。然而矛盾的斗争则是不绝的,不管在其共居时或其转化时,都有斗争的存在,尤其是表现在矛盾的转化时,这又是矛盾的普遍性、绝对性。研究矛盾的特殊性相对性时,要注意矛盾及矛盾方面之主要与非主要的区别;研究矛盾的斗争性时要注意矛盾的一般斗争形式与特殊斗争形式——即矛盾发展为对抗这种区别。这就是我们对于矛盾统一律的总结论。

\section{论持久战 1938/5}

* 这是毛泽东一九三八年五月二十六日至六月三日在延安抗日战争研究会的讲演。

问题的提起

(一)伟大抗日战争的一周年纪念,七月七日,快要到了。全民族的力量团结起来,坚持抗战,坚持统一战线,同敌人作英勇的战争,快一年了。这个战争,在东方历史上是空前的,在世界历史上也将是伟大的,全世界人民都关心这个战争。身受战争灾难、为着自己民族的生存而奋斗的每一个中国人,无日不在渴望战争的胜利。然而战争的过程究竟会要怎么样?能胜利还是不能胜利?能速胜还是不能速胜?很多人都说持久战,但是为什么是持久战?怎样进行持久战?很多人都说最后胜利,但是为什么会有最后胜利?怎样争取最后胜利?这些问题,不是每个人都解决了的,甚至是大多数人至今没有解决的。于是失败主义的亡国论者跑出来向人们说:中国会亡,最后胜利不是中国的。某些性急的朋友们也跑出来向人们说:中国很快就能战胜,无需乎费大气力。这些议论究竟对不对呢?我们一向都说:这些议论是不对的。可是我们说的,还没有为大多数人所了解。一半因为我们的宣传解释工作还不够,一半也因为客观事变的发展还没有完全暴露其固有的性质,还没有将其面貌鲜明地摆在人们之前,使人们无从看出其整个的趋势和前途,因而无从决定自己的整套的方针和做法。现在好了,抗战十个月的经验,尽够击破毫无根据的亡国论,也尽够说服急性朋友们的速胜论了。在这种情形下,很多人要求做个总结性的解释。尤其是对持久战,有亡国论和速胜论的反对意见,也有空洞无物的了解。“卢沟桥事变以来,四万万人一齐努力,最后胜利是中国的。”这样一种公式,在广大的人们中流行着。这个公式是对的,但有加以充实的必要。抗日战争和统一战线之所以能够坚持,是由于许多的因素:全国党派,从共产党到国民党;全国人民,从工人农民到资产阶级;全国军队,从主力军到游击队;国际方面,从社会主义国家到各国爱好正义的人民;敌国方面,从某些国内反战的人民到前线反战的兵士。总而言之,所有这些因素,在我们的抗战中都尽了他们各种程度的努力。每一个有良心的人,都应向他们表示敬意。我们共产党人,同其它抗战党派和全国人民一道,唯一的方向,是努力团结一切力量,战胜万恶的日寇。今年七月一日,是中国共产党建立的十七周年纪念日。为了使每个共产党员在抗日战争中能够尽其更好和更大的努力,也有着重地研究持久战的必要。因此,我的讲演就来研究持久战。和持久战这个题目有关的问题,我都准备说到;但是不能一切都说到,因为一切的东西,不是在一个讲演中完全说得了的。

(二)抗战十个月以来,一切经验都证明下述两种观点的不对:一种是中国必亡论,一种是中国速胜论。前者产生妥协倾向,后者产生轻敌倾向。他们看问题的方法都是主观的和片面的,一句话,非科学的。

(三)抗战以前,存在着许多亡国论的议论。例如说:“中国武器不如人,战必败。”“如果抗战,必会作阿比西尼亚。”抗战以后,公开的亡国论没有了,但暗地是有的,而且很多。例如妥协的空气时起时伏,主张妥协者的根据就是“再战必亡”\footnote{这种亡国论是国民党内部分领导人的意见。他们是不愿意抗日的,后来抗日是被迫的。卢沟桥事变以后,蒋介石一派参加抗日了,汪精卫一派就代表了亡国论,并准备投降日本,后来果然投降了。但是亡国论思想不但是在国民党内存在着,在某些中层社会中甚至在一部分落后的劳动人民中也曾经发生影响。这是因为国民党政府腐败无能,在抗日战争中节节失败,而日军则长驱直进,在战争的第一年中就侵占了华北和华中的大片土地,因而在一部分落后的人民中产生了严重的悲观情绪。}。有个学生从湖南写信来说:“在乡下一切都感到困难。单独一个人作宣传工作,只好随时随地找人谈话。对象都不是无知无识的愚民,他们多少也懂得一点,他们对我的谈话很有兴趣。可是碰了我那几位亲戚,他们总说:‘中国打不胜,会亡。’讨厌极了。好在他们还不去宣传,不然真糟。农民对他们的信仰当然要大些啊!”这类中国必亡论者,是妥协倾向的社会基础。这类人中国各地都有,因此,抗日阵线中随时可能发生的妥协问题,恐怕终战争之局也不会消灭的。当此徐州失守武汉紧张的时候,给这种亡国论痛驳一驳,我想不是无益的。

(四)抗战十个月以来,各种表现急性病的意见也发生了。例如在抗战初起时,许多人有一种毫无根据的乐观倾向,他们把日本估计过低,甚至以为日本不能打到山西。有些人轻视抗日战争中游击战争的战略地位,他们对于“在全体上,运动战是主要的,游击战是辅助的;在部分上,游击战是主要的,运动战是辅助的”这个提法,表示怀疑。他们不赞成八路军这样的战略方针:“基本的是游击战,但不放松有利条件下的运动战。”认为这是“机械的”观点\footnote{以上这些意见,都是共产党内的。在抗日战争的头半年内,党内存在着一种轻敌的倾向,认为日本不值一打。其根据并不是因为他们感觉自己的力量很大,他们知道共产党领导的军队和民众的有组织的力量在当时还是很小的;而是因为国民党抗日了,他们感觉国民党有很大的力量,可以有效地打击日本。他们只看见国民党暂时抗日的一面,忘记了国民党反动和腐败的一面,因而造成了错误的估计。}。上海战争时,有些人说:“只要打三个月,国际局势一定变化,苏联一定出兵,战争就可解决。”把抗战的前途主要地寄托在外国援助上面\footnote{这是蒋介石等人的意见。蒋介石国民党既已被迫抗战,他们就一心希望外国的迅速援助,不相信自己的力量,更不相信人民的力量。}。台儿庄胜利\footnote{一九三八年三月下旬至四月上旬,中国军队和日本侵略军在台儿庄(今属山东省枣庄市)一带进行过一次会战。在这次会战中,中国军队击败日军第五、第十两个精锐师团,取得了会战的胜利。}之后,有些人主张徐州战役\footnote{徐州战役是中国军队同日本侵略军在以徐州为中心的广大地区进行的一次战役。从一九三七年十二月起,华北、华中的日军分南北两线沿津浦铁路和台潍(台儿庄至潍县)公路进犯徐州外围地区。一九三八年四月上旬,中国军队在取得台儿庄会战的胜利后,继续向鲁南增兵,在徐州附近集结了约六十万的兵力;而日军在台儿庄遭到挫败以后,从四月上旬开始调集南北两线兵力二十多万人,对徐州进行迂回包围。中国军队在日军夹击和包围下,分路向豫皖边突围。五月十九日,徐州被日军占领。}应是“准决战”,说过去的持久战方针应该改变。说什么“这一战,就是敌人的最后挣扎”,“我们胜了,日阀就在精神上失了立场,只有静候末日审判”\footnote{这是当时《大公报》在一九三八年四月二十五日和二十六日社评中提出的意见。他们从一种侥幸心理出发,希望用几个台儿庄一类的胜仗就能打败日本,免得在持久战中动员人民力量,危及自己阶级的安全。当时国民党统治集团内普遍有这种侥幸心理。}。平型关一个胜仗,冲昏了一些人的头脑;台儿庄再一个胜仗,冲昏了更多的人的头脑。于是敌人是否进攻武汉,成为疑问了。许多人以为:“不一定”;许多人以为:“断不会”。这样的疑问可以牵涉到一切重大的问题。例如说:抗日力量是否够了呢?回答可以是肯定的,因为现在的力量已使敌人不能再进攻,还要增加力量干什么呢?例如说:巩固和扩大抗日民族统一战线的口号是否依然正确呢?回答可以是否定的,因为统一战线的现时状态已够打退敌人,还要什么巩固和扩大呢?例如说:国际外交和国际宣传工作是否还应该加紧呢?回答也可以是否定的。例如说:改革军队制度,改革政治制度,发展民众运动,厉行国防教育,镇压汉奸托派\footnote{抗日战争时期,托派在宣传上主张抗日,但是攻击中国共产党的抗日民族统一战线政策。把托派与汉奸相提并论,是由于当时在共产国际内流行着中国托派与日本帝国主义间谍组织有关的错误论断所造成的。},发展军事工业,改良人民生活,是否应该认真去做呢?例如说:保卫武汉、保卫广州、保卫西北和猛烈发展敌后游击战争的口号,是否依然正确呢?回答都可以是否定的。甚至某些人在战争形势稍为好转的时候,就准备在国共两党之间加紧磨擦一下,把对外的眼光转到对内。这种情况,差不多每一个较大的胜仗之后,或敌人进攻暂时停顿之时,都要发生。所有上述一切,我们叫它做政治上军事上的近视眼。这些话,讲起来好像有道理,实际上是毫无根据、似是而非的空谈。扫除这些空谈,对于进行胜利的抗日战争,应该是有好处的。

(五)于是问题是:中国会亡吗?答复:不会亡,最后胜利是中国的。中国能够速胜吗?答复:不能速胜,抗日战争是持久战。

(六)这些问题的主要论点,还在两年之前我们就一般地指出了。还在一九三六年七月十六日,即在西安事变前五个月,卢沟桥事变前十二个月,我同美国记者斯诺先生的谈话中,就已经一般地估计了中日战争的形势,并提出了争取胜利的各种方针。为备忘计,不妨抄录几段如下:

问:在什么条件下,中国能战胜并消灭日本帝国主义的实力呢?

答:要有三个条件:第一是中国抗日统一战线的完成;第二是国际抗日统一战线的完成;第三是日本国内人民和日本殖民地人民的革命运动的兴起。就中国人民的立场来说,三个条件中,中国人民的大联合是主要的。

问:你想,这个战争要延长多久呢?

答:要看中国抗日统一战线的实力和中日两国其它许多决定的因素如何而定。即是说,除了主要地看中国自己的力量之外,国际间所给中国的援助和日本国内革命的援助也很有关系。如果中国抗日统一战线有力地发展起来,横的方面和纵的方面都有效地组织起来,如果认清日本帝国主义威胁他们自己利益的各国政府和各国人民能给中国以必要的援助,如果日本的革命起来得快,则这次战争将迅速结束,中国将迅速胜利。如果这些条件不能很快实现,战争就要延长。但结果还是一样,日本必败,中国必胜。只是牺牲会大,要经过一个很痛苦的时期。

问:从政治上和军事上来看,你以为这个战争的前途会要如何发展?

答:日本的大陆政策已经确定了,那些以为同日本妥协,再牺牲一些中国的领土主权就能够停止日本进攻的人们,他们的想法只是一种幻想。我们确切地知道,就是扬子江下游和南方各港口,都已经包括在日本帝国主义的大陆政策之内。并且日本还想占领菲律宾、暹罗、越南、马来半岛和荷属东印度,把外国和中国切开,独占西南太平洋。这又是日本的海洋政策。在这样的时期,中国无疑地要处于极端困难的地位。可是大多数中国人相信,这种困难是能够克服的;只有各大商埠的富人是失败论者,因为他们害怕损失财产。有许多人想,一旦中国海岸被日本封锁,中国就不能继续作战。这是废话。为反驳他们,我们不妨举出红军的战争史。在抗日战争中,中国所占的优势,比内战时红军的地位强得多。中国是一个庞大的国家,就是日本能占领中国一万万至二万万人口的区域,我们离战败还很远呢。我们仍然有很大的力量同日本作战,而日本在整个战争中须得时时在其后方作防御战。中国经济的不统一、不平衡,对于抗日战争反为有利。例如将上海和中国其它地方割断,对于中国的损害,绝没有将纽约和美国其它地方割断对于美国的损害那样严重。日本就是把中国沿海封锁,中国的西北、西南和西部,它是无法封锁的。所以问题的中心点还是中国全体人民团结起来,树立举国一致的抗日阵线。这是我们早就提出了的。

问:假如战争拖得很长,日本没有完全战败,共产党能否同意讲和,并承认日本统治东北?

答:不能。中国共产党和全国人民一样,不容许日本保留中国的寸土。

问:照你的意见,这次解放战争,主要的战略方针是什么?

答:我们的战略方针,应该是使用我们的主力在很长的变动不定的战线上作战。中国军队要胜利,必须在广阔的战场上进行高度的运动战,迅速地前进和迅速地后退,迅速地集中和迅速地分散。这就是大规模的运动战,而不是深沟高垒、层层设防、专靠防御工事的阵地战。这并不是说要放弃一切重要的军事地点,对于这些地点,只要有利,就应配置阵地战。但是转换全局的战略方针,必然要是运动战。阵地战虽也必需,但是属于辅助性质的第二种的方针。在地理上,战场这样广大,我们作最有效的运动战,是可能的。日军遇到我军的猛烈活动,必得谨慎。他们的战争机构很笨重,行动很慢,效力有限。如果我们集中兵力在一个狭小的阵地上作消耗战的抵抗,将使我军失掉地理上和经济组织上的有利条件,犯阿比西尼亚的错误。战争的前期,我们要避免一切大的决战,要先用运动战逐渐地破坏敌人军队的精神和战斗力。

除了调动有训练的军队进行运动战之外,还要在农民中组织很多的游击队。须知东三省的抗日义勇军,仅仅是表示了全国农民所能动员抗战的潜伏力量的一小部分。中国农民有很大的潜伏力,只要组织和指挥得当,能使日本军队一天忙碌二十四小时,使之疲于奔命。必须记住这个战争是在中国打的,这就是说,日军要完全被敌对的中国人所包围;日军要被迫运来他们所需的军用品,而且要自己看守;他们要用重兵去保护交通线,时时谨防袭击;另外,还要有一大部力量驻扎满洲和日本内地。

在战争的过程中,中国能俘虏许多的日本兵,夺取许多的武器弹药来武装自己;同时,争取外国的援助,使中国军队的装备逐渐加强起来。因此,中国能够在战争的后期从事阵地战,对于日本的占领地进行阵地的攻击。这样,日本在中国抗战的长期消耗下,它的经济行将崩溃;在无数战争的消磨中,它的士气行将颓靡。中国方面,则抗战的潜伏力一天一天地奔腾高涨,大批的革命民众不断地倾注到前线去,为自由而战争。所有这些因素和其它的因素配合起来,就使我们能够对日本占领地的堡垒和根据地,作最后的致命的攻击,驱逐日本侵略军出中国。(斯诺:《西北印象记》)

抗战十个月的经验,证明上述论点的正确,以后也还将继续证明它。

(七)还在卢沟桥事变发生后一个多月,即一九三七年八月二十五日,中国共产党中央就在它的《关于目前形势与党的任务的决定》中,清楚地指出:

卢沟桥的挑战和平津的占领,不过是日寇大举进攻中国本部的开始。日寇已经开始了全国的战时动员。他们的所谓“不求扩大”的宣传,不过是掩护其进攻的烟幕弹。

七月七日卢沟桥的抗战,已经成了中国全国性抗战的起点。

中国的政治形势从此开始了一个新阶段,这就是实行抗战的阶段。抗战的准备阶段已经过去了。这一阶段的最中心的任务是:动员一切力量争取抗战的胜利。

争取抗战胜利的中心关键,在使已经发动的抗战发展为全面的全民族的抗战。只有这种全面的全民族的抗战,才能使抗战得到最后的胜利。

由于当前的抗战还存在着严重的弱点,所以在今后的抗战过程中,可能发生许多挫败、退却,内部的分化、叛变,暂时和局部的妥协等不利的情况。因此,应该看到这一抗战是艰苦的持久战。但我们相信,已经发动的抗战,必将因为我党和全国人民的努力,冲破一切障碍物而继续地前进和发展。

抗战十个月的经验,同样证明了上述论点的正确,以后也还将继续证明它。

(八)战争问题中的唯心论和机械论的倾向,是一切错误观点的认识论上的根源。他们看问题的方法是主观的和片面的。或者是毫无根据地纯主观地说一顿;或者是只根据问题的一侧面、一时候的表现,也同样主观地把它夸大起来,当作全体看。但是人们的错误观点可分为两类:一类是根本的错误,带一贯性,这是难于纠正的;另一类是偶然的错误,带暂时性,这是易于纠正的。但既同为错误,就都有纠正的必要。因此,反对战争问题中的唯心论和机械论的倾向,采用客观的观点和全面的观点去考察战争,才能使战争问题得出正确的结论。

问题的根据

(九)抗日战争为什么是持久战?最后胜利为什么是中国的呢?根据在什么地方呢?

中日战争不是任何别的战争,乃是半殖民地半封建的中国和帝国主义的日本之间在二十世纪三十年代进行的一个决死的战争。全部问题的根据就在这里。分别地说来,战争的双方有如下互相反对的许多特点。

(一○)日本方面:第一,它是一个强的帝国主义国家,它的军力、经济力和政治组织力在东方是一等的,在世界也是五六个著名帝国主义国家中的一个。这是日本侵略战争的基本条件,战争的不可避免和中国的不能速胜,就建立在这个日本国家的帝国主义制度及其强的军力、经济力和政治组织力上面。然而第二,由于日本社会经济的帝国主义性,就产生了日本战争的帝国主义性,它的战争是退步的和野蛮的。时至二十世纪三十年代的日本帝国主义,由于内外矛盾,不但使得它不得不举行空前大规模的冒险战争,而且使得它临到最后崩溃的前夜。从社会行程说来,日本已不是兴旺的国家,战争不能达到日本统治阶级所期求的兴旺,而将达到它所期求的反面——日本帝国主义的死亡。这就是所谓日本战争的退步性。跟着这个退步性,加上日本又是一个带军事封建性的帝国主义这一特点,就产生了它的战争的特殊的野蛮性。这样就要最大地激起它国内的阶级对立、日本民族和中国民族的对立、日本和世界大多数国家的对立。日本战争的退步性和野蛮性是日本战争必然失败的主要根据。还不止此,第三,日本战争虽是在其强的军力、经济力和政治组织力的基础之上进行的,但同时又是在其先天不足的基础之上进行的。日本的军力、经济力和政治组织力虽强,但这些力量之量的方面不足。日本国度比较地小,其人力、军力、财力、物力均感缺乏,经不起长期的战争。日本统治者想从战争中解决这个困难问题,但同样,将达到其所期求的反面,这就是说,它为解决这个困难问题而发动战争,结果将因战争而增加困难,战争将连它原有的东西也消耗掉。最后,第四,日本虽能得到国际法西斯国家的援助,但同时,却又不能不遇到一个超过其国际援助力量的国际反对力量。这后一种力量将逐渐地增长,终究不但将把前者的援助力量抵消,并将施其压力于日本自身。这是失道寡助的规律,是从日本战争的本性产生出来的。总起来说,日本的长处是其战争力量之强,而其短处则在其战争本质的退步性、野蛮性,在其人力、物力之不足,在其国际形势之寡助。这些就是日本方面的特点。

(一一)中国方面:第一,我们是一个半殖民地半封建的国家。从鸦片战争\footnote,太平天国\footnote,戊戌维新\footnote,辛亥革命,直至北伐战争,一切为解除半殖民地半封建地位的革命的或改良的运动,都遭到了严重的挫折,因此依然保留下这个半殖民地半封建的地位。我们依然是一个弱国,我们在军力、经济力和政治组织力各方面都显得不如敌人。战争之不可避免和中国之不能速胜,又在这个方面有其基础。然而第二,中国近百年的解放运动积累到了今日,已经不同于任何历史时期。各种内外反对力量虽给了解放运动以严重挫折,同时却锻炼了中国人民。今日中国的军事、经济、政治、文化虽不如日本之强,但在中国自己比较起来,却有了比任何一个历史时期更为进步的因素。中国共产党及其领导下的军队,就是这种进步因素的代表。中国今天的解放战争,就是在这种进步的基础上得到了持久战和最后胜利的可能性。中国是如日方升的国家,这同日本帝国主义的没落状态恰是相反的对照。中国的战争是进步的,从这种进步性,就产生了中国战争的正义性。因为这个战争是正义的,就能唤起全国的团结,激起敌国人民的同情,争取世界多数国家的援助。第三,中国又是一个很大的国家,地大、物博、人多、兵多,能够支持长期的战争,这同日本又是一个相反的对比。最后,第四,由于中国战争的进步性、正义性而产生出来的国际广大援助,同日本的失道寡助又恰恰相反。总起来说,中国的短处是战争力量之弱,而其长处则在其战争本质的进步性和正义性,在其是一个大国家,在其国际形势之多助。这些都是中国的特点。

(一二)这样看来,日本的军力、经济力和政治组织力是强的,但其战争是退步的、野蛮的,人力、物力又不充足,国际形势又处于不利。中国反是,军力、经济力和政治组织力是比较地弱的,然而正处于进步的时代,其战争是进步的和正义的,又有大国这个条件足以支持持久战,世界的多数国家是会要援助中国的。——这些,就是中日战争互相矛盾着的基本特点。这些特点,规定了和规定着双方一切政治上的政策和军事上的战略战术,规定了和规定着战争的持久性和最后胜利属于中国而不属于日本。战争就是这些特点的比赛。这些特点在战争过程中将各依其本性发生变化,一切东西就都从这里发生出来。这些特点是事实上存在的,不是虚造骗人的;是战争的全部基本要素,不是残缺不全的片段;是贯彻于双方一切大小问题和一切作战阶段之中的,不是可有可无的。观察中日战争如果忘记了这些特点,那就必然要弄错;即使某些意见一时有人相信,似乎不错,但战争的经过必将证明它们是错的。我们现在就根据这些特点来说明我们所要说的一切问题。

驳亡国论

(一三)亡国论者看到敌我强弱对比一个因素,从前就说“抗战必亡”,现在又说“再战必亡”。如果我们仅仅说,敌人虽强,但是小国,中国虽弱,但是大国,是不足以折服他们的。他们可以搬出元朝灭宋、清朝灭明的历史证据,证明小而强的国家能够灭亡大而弱的国家,而且是落后的灭亡进步的。如果我们说,这是古代,不足为据,他们又可以搬出英灭印度的事实,证明小而强的资本主义国家能够灭亡大而弱的落后国家。所以还须提出其它的根据,才能把一切亡国论者的口封住,使他们心服,而使一切从事宣传工作的人们得到充足的论据去说服还不明白和还不坚定的人们,巩固其抗战的信心。

(一四)这应该提出的根据是什么呢?就是时代的特点。这个特点的具体反映是日本的退步和寡助,中国的进步和多助。

(一五)我们的战争不是任何别的战争,乃是中日两国在二十世纪三十年代进行的战争。在我们的敌人方面,首先,它是快要死亡的帝国主义,它已处于退步时代,不但和英灭印度时期英国还处于资本主义的进步时代不相同,就是和二十年前第一次世界大战时的日本也不相同。此次战争发动于世界帝国主义首先是法西斯国家大崩溃的前夜,敌人也正是为了这一点才举行这个带最后挣扎性的冒险战争。所以,战争的结果,灭亡的不会是中国而是日本帝国主义的统治集团,这是无可逃避的必然性。再则,当日本举行战争的时候,正是世界各国或者已经遭遇战争或者快要遭遇战争的时候,大家都正在或准备着为反抗野蛮侵略而战,中国这个国家又是同世界多数国家和多数人民利害相关的,这就是日本已经引起并还要加深地引起世界多数国家和多数人民的反对的根源。

(一六)中国方面呢?它已经不能和别的任何历史时期相比较。半殖民地和半封建社会是它的特点,所以被称为弱国。但是在同时,它又处于历史上进步的时代,这就是足以战胜日本的主要根据。所谓抗日战争是进步的,不是说普通一般的进步,不是说阿比西尼亚抗意战争的那种进步,也不是说太平天国或辛亥革命的那种进步,而是说今天中国的进步。今天中国的进步在什么地方呢?在于它已经不是完全的封建国家,已经有了资本主义,有了资产阶级和无产阶级,有了已经觉悟或正在觉悟的广大人民,有了共产党,有了政治上进步的军队即共产党领导的中国红军,有了数十年革命的传统经验,特别是中国共产党成立以来的十七年的经验。这些经验,教育了中国的人民,教育了中国的政党,今天恰好作了团结抗日的基础。如果说,在俄国,没有一九○五年的经验就不会有一九一七年的胜利;那末,我们也可以说,如果没有十七年以来的经验,也将不会有抗日的胜利。这是国内的条件。

国际的条件,使得中国在战争中不是孤立的,这一点也是历史上空前的东西。历史上不论中国的战争也罢,印度的战争也罢,都是孤立的。惟独今天遇到世界上已经发生或正在发生的空前广大和空前深刻的人民运动及其对于中国的援助。俄国一九一七年的革命也遇到世界的援助,俄国的工人和农民因此胜利了,但那个援助的规模还没有今天广大,性质也没有今天深刻。今天的世界的人民运动,正在以空前的大规模和空前的深刻性发展着。苏联的存在,更是今天国际政治上十分重要的因素,它必然以极大的热忱援助中国,这一现象,是二十年前完全没有的。所有这些,造成了和造成着为中国最后胜利所不可缺少的重要的条件。大量的直接的援助,目前虽然还没有,尚有待于来日,但是中国有进步和大国的条件,能够延长战争的时间,促进并等候国际的援助。

(一七)加上日本是小国,地小、物少、人少、兵少,中国是大国,地大、物博、人多、兵多这一个条件,于是在强弱对比之外,就还有小国、退步、寡助和大国、进步、多助的对比,这就是中国决不会亡的根据。强弱对比虽然规定了日本能够在中国有一定时期和一定程度的横行,中国不可避免地要走一段艰难的路程,抗日战争是持久战而不是速决战;然而小国、退步、寡助和大国、进步、多助的对比,又规定了日本不能横行到底,必然要遭到最后的失败,中国决不会亡,必然要取得最后的胜利。

(一八)阿比西尼亚为什么灭亡了呢?第一,它不但是弱国,而且是小国。第二,它不如中国进步,它是一个古老的奴隶制到农奴制的国家,没有资本主义,没有资产阶级政党,更没有共产党,没有中国这样的军队,更没有如同八路军这样的军队。第三,它不能等候国际的援助,它的战争是孤立的。第四,这是主要的,抗意战争领导方面有错误。阿比西尼亚因此灭亡了。然而阿比西尼亚还有相当广大的游击战争存在,如能坚持下去,是可以在未来的世界变动中据以恢复其祖国的。

(一九)如果亡国论者搬出中国近代解放运动的失败史来证明“抗战必亡”和“再战必亡”的话,那我们的答复也是时代不同一句话。中国本身、日本内部、国际环境都和过去不相同。日本比过去更强了,中国的半殖民地和半封建地位依然未变,力量依然颇弱,这一点是严重的情形。日本暂时还能控制其国内的人民,也还能利用国际间的矛盾作为其侵华的工具,这些都是事实。然而在长期的战争过程中,必然要发生相反的变化。这一点现在还不是事实,但是将来必然要成为事实的。这一点,亡国论者就抛弃不顾了。中国呢?不但现在已有新的人、新的政党、新的军队和新的抗日政策,和十余年以前有很大的不同,而且这些都必然会向前发展。虽然历史上的解放运动屡次遭受挫折,使中国不能积蓄更大的力量用于今日的抗日战争——这是非常可痛惜的历史的教训,从今以后,再也不要自己摧残任何的革命力量了——然而就在既存的基础上,加上广大的努力,必能逐渐前进,加强抗战的力量。伟大的抗日民族统一战线,就是这种努力的总方向。国际援助一方面,眼前虽然还看不见大量的和直接的,但是国际局面根本已和过去两样,大量和直接的援助正在酝酿中。中国近代无数解放运动的失败都有其客观和主观的原因,都不能比拟今天的情况。在今天,虽然存在着许多困难条件,规定了抗日战争是艰难的战争,例如敌人之强,我们之弱,敌人的困难还刚在开始,我们的进步还很不够,如此等等,然而战胜敌人的有利条件是很多的,只须加上主观的努力,就能克服困难而争取胜利。这些有利条件,历史上没有一个时候可和今天比拟,这就是抗日战争必不会和历史上的解放运动同归失败的理由。

妥协还是抗战?腐败还是进步?

(二○)亡国论之没有根据,俱如上述。但是另有许多人,并非亡国论者,他们是爱国志士,却对时局怀抱甚深的忧虑。他们的问题有两个:一是惧怕对日妥协,一是怀疑政治不能进步。这两个可忧虑的问题在广大的人们中间议论着,找不到解决的基点。我们现在就来研究这两个问题。

(二一)前头说过,妥协的问题是有其社会根源的,这个社会根源存在,妥协问题就不会不发生。但妥协是不会成功的。要证明这一点,仍不外向日本、中国、国际三方面找根据。第一是日本方面。还在抗战初起时,我们就估计有一种酝酿妥协空气的时机会要到来,那就是在敌人占领华北和江浙之后,可能出以劝降手段。后来果然来了这一手;但是危机随即过去,原因之一是敌人采取了普遍的野蛮政策,实行公开的掠夺。中国降了,任何人都要做亡国奴。敌人的这一掠夺的即灭亡中国的政策,分为物质的和精神的两方面,都是普遍地施之于中国人的;不但是对下层民众,而且是对上层成分,——当然对后者稍为客气些,但也只有程度之别,并无原则之分。大体上,敌人是将东三省的老办法移植于内地。在物质上,掠夺普通人民的衣食,使广大人民啼饥号寒;掠夺生产工具,使中国民族工业归于毁灭和奴役化。在精神上,摧残中国人民的民族意识。在太阳旗下,每个中国人只能当顺民,做牛马,不许有一丝一毫的中国气。敌人的这一野蛮政策,还要施之于更深的内地。他的胃口很旺,不愿停止战争。一九三八年一月十六日日本内阁宣言的方针\footnote{一九三八年一月十六日,日本近卫内阁发表声明,宣布以武力灭亡中国的方针;同时宣称由于国民党政府仍在“策划抗战”,日本政府决定在中国扶植新的傀儡政权,“今后将不以国民政府为对手”。},至今坚决执行,也不能不执行,这就激怒了一切阶层的中国人。这是根据敌人战争的退步性野蛮性而来的,“在劫难逃”,于是形成了绝对的敌对。估计到某种时机,敌之劝降手段又将出现,某些亡国论者又将蠕蠕而动,而且难免勾结某些国际成分(英、美、法内部都有这种人,特别是英国的上层分子),狼狈为奸。但是大势所趋,是降不了的,日本战争的坚决性和特殊的野蛮性,规定了这个问题的一方面。

(二二)第二是中国方面。中国坚持抗战的因素有三个:其一,共产党,这是领导人民抗日的可靠力量。又其一,国民党,因其是依靠英美的,英美不叫它投降,它也就不会投降。又其一,别的党派,大多数是反对妥协、拥护抗战的。这三者互相团结,谁要妥协就是站在汉奸方面,人人得而诛之。一切不愿当汉奸的人,就不能不团结起来坚持抗战到底,妥协就实际上难于成功。

(二三)第三是国际方面。除日本的盟友和各资本主义国家的上层分子中的某些成分外,其余都不利于中国妥协而利于中国抗战。这一因素影响到中国的希望。今天全国人民有一种希望,认为国际力量必将逐渐增强地援助中国。这种希望不是空的;特别是苏联的存在,鼓舞了中国的抗战。空前强大的社会主义的苏联,它和中国是历来休戚相关的。苏联和一切资本主义国家的上层成分之唯利是图者根本相反,它是以援助一切弱小民族和革命战争为其职志的。中国战争之非孤立性,不但一般地建立在整个国际的援助上,而且特殊地建立在苏联的援助上。中苏两国是地理接近的,这一点加重了日本的危机,便利了中国的抗战。中日两国地理接近,加重了中国抗战的困难。然而中苏的地理接近,却是中国抗战的有利条件。

(二四)由此可作结论:妥协的危机是存在的,但是能够克服。因为敌人的政策即使可作某种程度的改变,但其根本改变是不可能的。中国内部有妥协的社会根源,但是反对妥协的占大多数。国际力量也有一部分赞助妥协,但是主要的力量赞助抗战。这三种因素结合起来,就能克服妥协危机,坚持抗战到底。

(二五)现在来答复第二个问题。国内政治的改进,是和抗战的坚持不能分离的。政治越改进,抗战越能坚持;抗战越坚持,政治就越能改进。但是基本上依赖于坚持抗战。国民党的各方面的不良现象是严重地存在着,这些不合理因素的历史积累,使得广大爱国志士发生很大的忧虑和烦闷。但是抗战的经验已经证明,十个月的中国人民的进步抵得上过去多少年的进步,并无使人悲观的根据。历史积累下来的腐败现象,虽然很严重地阻碍着人民抗战力量增长的速度,减少了战争的胜利,招致了战争的损失,但是中国、日本和世界的大局,不容许中国人民不进步。由于阻碍进步的因素即腐败现象之存在,这种进步是缓慢的。进步和进步的缓慢是目前时局的两个特点,后一个特点和战争的迫切要求很不相称,这就是使得爱国志士们大为发愁的地方。然而我们是在革命战争中,革命战争是一种抗毒素,它不但将排除敌人的毒焰,也将清洗自己的污浊。凡属正义的革命的战争,其力量是很大的,它能改造很多事物,或为改造事物开辟道路。中日战争将改造中日两国;只要中国坚持抗战和坚持统一战线,就一定能把旧日本化为新日本,把旧中国化为新中国,中日两国的人和物都将在这次战争中和战争后获得改造。我们把抗战和建国联系起来看,是正当的。说日本也能获得改造,是说日本统治者的侵略战争将走到失败,有引起日本人民革命之可能。日本人民革命胜利之日,就是日本改造之时。这和中国的抗战密切地联系着,这一个前途是应该看到的。

亡国论是不对的,速胜论也是不对的

(二六)我们已把强弱、大小、进步退步、多助寡助几个敌我之间矛盾着的基本特点,作了比较研究,批驳了亡国论,答复了为什么不易妥协和为什么政治可能进步的问题。亡国论者看重了强弱一个矛盾,把它夸大起来作为全部问题的论据,而忽略了其它的矛盾。他们只提强弱对比一点,是他们的片面性;他们将此片面的东西夸大起来看成全体,又是他们的主观性。所以在全体说来,他们是没有根据的,是错误的。那些并非亡国论者,也不是一贯的悲观主义者,仅为一时候和一局部的敌我强弱情况或国内腐败现象所迷惑,而一时地发生悲观心理的人们,我们也得向他们指出,他们的观点的来源也是片面性和主观性的倾向。但是他们的改正较容易,只要一提醒就会明白,因为他们是爱国志士,他们的错误是一时的。

(二七)然而速胜论者也是不对的。他们或则根本忘记了强弱这个矛盾,而单单记起了其它矛盾;或则对于中国的长处,夸大得离开了真实情况,变成另一种样子;或则拿一时一地的强弱现象代替了全体中的强弱现象,一叶障目,不见泰山,而自以为是。总之,他们没有勇气承认敌强我弱这件事实。他们常常抹杀这一点,因此抹杀了真理的一方面。他们又没有勇气承认自己长处之有限性,因而抹杀了真理的又一方面。由此犯出或大或小的错误来,这里也是主观性和片面性作怪。这些朋友们的心是好的,他们也是爱国志士。但是“先生之志则大矣”,先生的看法则不对,照了做去,一定碰壁。因为估计不符合真相,行动就无法达到目的;勉强行去,败军亡国,结果和失败主义者没有两样。所以也是要不得的。

(二八)我们是否否认亡国危险呢?不否认的。我们承认在中国面前摆着解放和亡国两个可能的前途,两者在猛烈地斗争中。我们的任务在于实现解放而避免亡国。实现解放的条件,基本的是中国的进步,同时,加上敌人的困难和世界的援助。我们和亡国论者不同,我们客观地而且全面地承认亡国和解放两个可能同时存在,着重指出解放的可能占优势及达到解放的条件,并为争取这些条件而努力。亡国论者则主观地和片面地只承认亡国一个可能性,否认解放的可能性,更不会指出解放的条件和为争取这些条件而努力。我们对于妥协倾向和腐败现象也是承认的,但是我们还看到其它倾向和其它现象,并指出二者之中后者对于前者将逐步地占优势,二者在猛烈地斗争着;并指出后者实现的条件,为克服妥协倾向和转变腐败现象而努力。因此,我们并不悲观,而悲观的人们则与此相反。

(二九)我们也不是不喜欢速胜,谁也赞成明天一个早上就把“鬼子”赶出去。但是我们指出,没有一定的条件,速胜只存在于头脑之中,客观上是不存在的,只是幻想和假道理。因此,我们客观地并全面地估计到一切敌我情况,指出只有战略的持久战才是争取最后胜利的唯一途径,而排斥毫无根据的速胜论。我们主张为着争取最后胜利所必要的一切条件而努力,条件多具备一分,早具备一日,胜利的把握就多一分,胜利的时间就早一日。我们认为只有这样才能缩短战争的过程,而排斥贪便宜尚空谈的速胜论。

为什么是持久战?

(三○)现在我们来把持久战问题研究一下。“为什么是持久战”这一个问题,只有依据全部敌我对比的基本因素,才能得出正确的回答。例如单说敌人是帝国主义的强国,我们是半殖民地半封建的弱国,就有陷入亡国论的危险。因为单纯地以弱敌强,无论在理论上,在实际上,都不能产生持久的结果。单是大小或单是进步退步、多助寡助,也是一样。大并小、小并大的事都是常有的。进步的国家或事物,如果力量不强,常有被大而退步的国家或事物所灭亡者。多助寡助是重要因素,但是附随因素,依敌我本身的基本因素如何而定其作用的大小。因此,我们说抗日战争是持久战,是从全部敌我因素的相互关系产生的结论。敌强我弱,我有灭亡的危险。但敌尚有其它缺点,我尚有其它优点。敌之优点可因我之努力而使之削弱,其缺点亦可因我之努力而使之扩大。我方反是,我之优点可因我之努力而加强,缺点则因我之努力而克服。所以我能最后胜利,避免灭亡,敌则将最后失败,而不能避免整个帝国主义制度的崩溃。

(三一)既然敌之优点只有一个,余皆缺点,我之缺点只有一个,余皆优点,为什么不能得出平衡结果,反而造成了现时敌之优势我之劣势呢?很明显的,不能这样形式地看问题。事情是现时敌我强弱的程度悬殊太大,敌之缺点一时还没有也不能发展到足以减杀其强的因素之必要的程度,我之优点一时也没有且不能发展到足以补充其弱的因素之必要的程度,所以平衡不能出现,而出现的是不平衡。

(三二)敌强我弱,敌是优势而我是劣势,这种情况,虽因我之坚持抗战和坚持统一战线的努力而有所变化,但是还没有产生基本的变化。所以,在战争的一定阶段上,敌能得到一定程度的胜利,我则将遭到一定程度的失败。然而敌我都只限于这一定阶段内一定程度上的胜或败,不能超过而至于全胜或全败,这是什么缘故呢?因为一则敌强我弱之原来状况就是相对的,不是绝对的;二则由于我之坚持抗战和坚持统一战线的努力,更加造成这种相对的形势。拿原来状况来说,敌虽强,但敌之强已为其它不利的因素所减杀,不过此时还没有减杀到足以破坏敌之优势的必要的程度;我虽弱,但我之弱已为其它有利的因素所补充,不过此时还没有补充到足以改变我之劣势的必要的程度。于是形成敌是相对的强,我是相对的弱;敌是相对的优势,我是相对的劣势。双方的强弱优劣原来都不是绝对的,加以战争过程中我之坚持抗战和坚持统一战线的努力,更加变化了敌我原来强弱优劣的形势,因而敌我只限于一定阶段内的一定程度上的胜或败,造成了持久战的局面。

(三三)然而情况是继续变化的。战争过程中,只要我能运用正确的军事的和政治的策略,不犯原则的错误,竭尽最善的努力,敌之不利因素和我之有利因素均将随战争之延长而发展,必能继续改变着敌我强弱的原来程度,继续变化着敌我的优劣形势。到了新的一定阶段时,就将发生强弱程度上和优劣形势上的大变化,而达到敌败我胜的结果。

(三四)目前敌尚能勉强利用其强的因素,我之抗战尚未给他以基本的削弱。其人力、物力不足的因素尚不足以阻止其进攻,反之,尚足以维持其进攻到一定的程度。其足以加剧本国阶级对立和中国民族反抗的因素,即战争之退步性和野蛮性一因素,亦尚未造成足以根本妨碍其进攻的情况。敌人的国际孤立的因素也方在变化发展之中,还没有达到完全的孤立。许多表示助我的国家的军火资本家和战争原料资本家,尚在唯利是图地供给日本以大量的战争物资\footnote{这里主要是指美国。自一九三七年到一九四○年,美国每年输入日本的物资占日本全部进口额的三分之一以上,其中战争物资占一半以上。},他们的政府\footnote{指英、美、法等帝国主义国家的政府。}亦尚不愿和苏联一道用实际方法制裁日本。这一切,规定了我之抗战不能速胜,而只能是持久战。中国方面,弱的因素表现在军事、经济、政治、文化各方面的,虽在十个月抗战中有了某种程度的进步,但距离足以阻止敌之进攻及准备我之反攻的必要的程度,还远得很。且在量的方面,又不得不有所减弱。其各种有利因素,虽然都在起积极作用,但达到足以停止敌之进攻及准备我之反攻的程度则尚有待于巨大的努力。在国内,克服腐败现象,增加进步速度;在国外,克服助日势力,增加反日势力,尚非目前的现实。这一切,又规定了战争不能速胜,而只能是持久战。

持久战的三个阶段

(三五)中日战争既然是持久战,最后胜利又将是属于中国的,那末,就可以合理地设想,这种持久战,将具体地表现于三个阶段之中。第一个阶段,是敌之战略进攻、我之战略防御的时期。第二个阶段,是敌之战略保守、我之准备反攻的时期。第三个阶段,是我之战略反攻、敌之战略退却的时期。三个阶段的具体情况不能预断,但依目前条件来看,战争趋势中的某些大端是可以指出的。客观现实的行程将是异常丰富和曲折变化的,谁也不能造出一本中日战争的“流年”来;然而给战争趋势描画一个轮廓,却为战略指导所必需。所以,尽管描画的东西不能尽合将来的事实,而将为事实所校正,但是为着坚定地有目的地进行持久战的战略指导起见,描画轮廓的事仍然是需要的。

(三六)第一阶段,现在还未完结。敌之企图是攻占广州、武汉、兰州三点,并把三点联系起来。敌欲达此目的,至少出五十个师团,约一百五十万兵员,时间一年半至两年,用费将在一百万万日元以上。敌人如此深入,其困难是非常之大的,其后果将不堪设想。至欲完全占领粤汉铁路和西兰公路,将经历非常危险的战争,未必尽能达其企图。但是我们的作战计划,应把敌人可能占领三点甚至三点以外之某些部分地区并可能互相联系起来作为一种基础,部署持久战,即令敌如此做,我也有应付之方。这一阶段我所采取的战争形式,主要的是运动战,而以游击战和阵地战辅助之。阵地战虽在此阶段之第一期,由于国民党军事当局的主观错误把它放在主要地位,但从全阶段看,仍然是辅助的。此阶段中,中国已经结成了广大的统一战线,实现了空前的团结。敌虽已经采用过并且还将采用卑鄙无耻的劝降手段,企图不费大力实现其速决计划,整个地征服中国,但是过去的已经失败,今后的也难成功。此阶段中,中国虽有颇大的损失,但是同时却有颇大的进步,这种进步就成为第二阶段继续抗战的主要基础。此阶段中,苏联对于我国已经有了大量的援助。敌人方面,士气已开始表现颓靡,敌人陆军进攻的锐气,此阶段的中期已不如初期,末期将更不如初期。敌之财政和经济已开始表现其竭蹶状态,人民和士兵的厌战情绪已开始发生,战争指导集团的内部已开始表现其“战争的烦闷”,生长着对于战争前途的悲观。

(三七)第二阶段,可以名之曰战略的相持阶段。第一阶段之末尾,由于敌之兵力不足和我之坚强抵抗,敌人将不得不决定在一定限度上的战略进攻终点,到达此终点以后,即停止其战略进攻,转入保守占领地的阶段。此阶段内,敌之企图是保守占领地,以组织伪政府的欺骗办法据之为己有,而从中国人民身上尽量搜括东西,但是在他的面前又遇着顽强的游击战争。游击战争在第一阶段中乘着敌后空虚将有一个普遍的发展,建立许多根据地,基本上威胁到敌人占领地的保守,因此第二阶段仍将有广大的战争。此阶段中我之作战形式主要的是游击战,而以运动战辅助之。此时中国尚能保有大量的正规军,不过一方面因敌在其占领的大城市和大道中取战略守势,一方面因中国技术条件一时未能完备,尚难迅即举行战略反攻。除正面防御部队外,我军将大量地转入敌后,比较地分散配置,依托一切敌人未占区域,配合民众武装,向敌人占领地作广泛的和猛烈的游击战争,并尽可能地调动敌人于运动战中消灭之,如同现在山西的榜样。此阶段的战争是残酷的,地方将遇到严重的破坏。但是游击战争能够胜利,做得好,可能使敌只能保守占领地三分之一左右的区域,三分之二左右仍然是我们的,这就是敌人的大失败,中国的大胜利。

那时,整个敌人占领地将分为三种地区:第一种是敌人的根据地,第二种是游击战争的根据地,第三种是双方争夺的游击区。这个阶段的时间的长短,依敌我力量增减变化的程度如何及国际形势变动如何而定,大体上我们要准备付给较长的时间,要熬得过这段艰难的路程。这将是中国很痛苦的时期,经济困难和汉奸捣乱将是两个很大的问题。敌人将大肆其破坏中国统一战线的活动,一切敌之占领地的汉奸组织将合流组成所谓“统一政府”。我们内部,因大城市的丧失和战争的困难,动摇分子将大倡其妥协论,悲观情绪将严重地增长。此时我们的任务,在于动员全国民众,齐心一致,绝不动摇地坚持战争,把统一战线扩大和巩固起来,排除一切悲观主义和妥协论,提倡艰苦斗争,实行新的战时政策,熬过这一段艰难的路程。此阶段内,必须号召全国坚决地维持一个统一政府,反对分裂,有计划地增强作战技术,改造军队,动员全民,准备反攻。此阶段中,国际形势将变到更于日本不利,虽可能有张伯伦\footnote{张伯伦(一八六九——一九四○),英国保守党领袖。一九三七年至一九四○年任英国首相。他主张迁就德、意、日法西斯对中国、埃塞俄比亚、西班牙、奥地利和捷克斯洛伐克等国家的侵略,实行妥协政策。}一类的迁就所谓“既成事实”的“现实主义”的调头出现,但主要的国际势力将变到进一步地援助中国。日本威胁南洋和威胁西伯利亚,将较之过去更加严重,甚至爆发新的战争。敌人方面,陷在中国泥潭中的几十个师团抽不出去。广大的游击战争和人民抗日运动将疲惫这一大批日本军,一方面大量地消耗之,又一方面进一步地增长其思乡厌战直至反战的心理,从精神上瓦解这个军队。日本在中国的掠夺虽然不能说它绝对不能有所成就,但是日本资本缺乏,又困于游击战争,急遽的大量的成就是不可能的。这个第二阶段是整个战争的过渡阶段,也将是最困难的时期,然而它是转变的枢纽。中国将变为独立国,还是沦为殖民地,不决定于第一阶段大城市之是否丧失,而决定于第二阶段全民族努力的程度。如能坚持抗战,坚持统一战线和坚持持久战,中国将在此阶段中获得转弱为强的力量。中国抗战的三幕戏,这是第二幕。由于全体演员的努力,最精彩的结幕便能很好地演出来。

(三八)第三阶段,是收复失地的反攻阶段。收复失地,主要地依靠中国自己在前阶段中准备着的和在本阶段中继续地生长着的力量。然而单只自己的力量还是不够的,还须依靠国际力量和敌国内部变化的援助,否则是不能胜利的,因此加重了中国的国际宣传和外交工作的任务。这个阶段,战争已不是战略防御,而将变为战略反攻了,在现象上,并将表现为战略进攻;已不是战略内线,而将逐渐地变为战略外线。直至打到鸭绿江边,才算结束了这个战争。第三阶段是持久战的最后阶段,所谓坚持战争到底,就是要走完这个阶段的全程。这个阶段我所采取的主要的战争形式仍将是运动战,但是阵地战将提到重要地位。如果说,第一阶段的阵地防御,由于当时的条件,不能看作重要的,那末,第三阶段的阵地攻击,由于条件的改变和任务的需要,将变成颇为重要的。此阶段内的游击战,仍将辅助运动战和阵地战而起其战略配合的作用,和第二阶段之变为主要形式者不相同。

(三九)这样看来,战争的长期性和随之而来的残酷性,是明显的。敌人不能整个地吞并中国,但是能够相当长期地占领中国的许多地方。中国也不能迅速地驱逐日本,但是大部分的土地将依然是中国的。最后是敌败我胜,但是必须经过一段艰难的路程。

(四○)中国人民在这样长期和残酷的战争中间,将受到很好的锻炼。参加战争的各政党也将受到锻炼和考验。统一战线必须坚持下去;只有坚持统一战线,才能坚持战争;只有坚持统一战线和坚持战争,才能有最后胜利。果然是这样,一切困难就能够克服。跨过战争的艰难路程之后,胜利的坦途就到来了,这是战争的自然逻辑。

(四一)三个阶段中,敌我力量的变化将循着下述的道路前进。第一阶段敌是优势,我是劣势。我之这种劣势,须估计抗战以前到这一阶段末尾,有两种不同的变化。第一种是向下的。中国原来的劣势,经过第一阶段的消耗将更为严重,这就是土地、人口、经济力量、军事力量和文化机关等的减缩。第一阶段的末尾,也许要减缩到相当大的程度,特别是经济方面。这一点,将被人利用作为亡国论和妥协论的根据。然而必须看到第二种变化,即向上的变化。这就是战争中的经验,军队的进步,政治的进步,人民的动员,文化的新方向的发展,游击战争的出现,国际援助的增长等等。在第一阶段,向下的东西是旧的量和质,主要地表现在量上。向上的东西是新的量和质,主要地表现在质上。这第二种变化,就给了我们以能够持久和最后胜利的根据。

(四二)第一阶段中,敌人方面也有两种变化。第一种是向下的,表现在:几十万人的伤亡,武器和弹药的消耗,士气的颓靡,国内人心的不满,贸易的缩减,一百万万日元以上的支出,国际舆论的责备等等方面。这个方面,又给予我们以能够持久和最后胜利的根据。然而也要估计到敌人的第二种变化,即向上的变化。那就是他扩大了领土、人口和资源。在这点上面,又产生我们的抗战是持久战而不能速胜的根据,同时也将被一些人利用作为亡国论和妥协论的根据。但是我们必须估计敌人这种向上变化的暂时性和局部性。敌人是行将崩溃的帝国主义者,他占领中国的土地是暂时的。中国游击战争的猛烈发展,将使他的占领区实际上限制在狭小的地带。而且,敌人对中国土地的占领又产生了和加深了日本同外国的矛盾。再则,根据东三省的经验,日本在相当长的时间内,一般地只能是支出资本时期,不能是收获时期。所有这些,又是我们击破亡国论和妥协论而建立持久论和最后胜利论的根据。

(四三)第二阶段,上述双方的变化将继续发展,具体的情形不能预断,但是大体上将是日本继续向下,中国继续向上\footnote{毛泽东在这里所预言的抗日战争相持阶段中中国方面可能的向上变化,在中国共产党领导下的抗日根据地是完全实现了。在国民党统治区,则因为以蒋介石为首的统治集团消极抗日、积极反共反人民,不但没有向上变化,反而向下变化了。因为这样,也激起了广大人民的反抗和觉悟。参见本书第三卷《论联合政府》第三部分关于这一切事实的分析。}。例如日本的军力、财力大量地消耗于中国的游击战争,国内人心更加不满,士气更加颓靡,国际更感孤立。中国则政治、军事、文化和人民动员将更加进步,游击战争更加发展,经济方面也将依凭内地的小工业和广大的农业而有某种程度的新发展,国际援助将逐渐地增进,将比现在的情况大为改观。这个第二阶段,也许将经过相当长的时间。在这个时间内,敌我力量对比将发生巨大的相反的变化,中国将逐渐上升,日本则逐渐下降。那时中国将脱出劣势,日本则脱出优势,先走到平衡的地位,再走到优劣相反的地位。然后中国大体上将完成战略反攻的准备而走到实行反攻、驱敌出国的阶段。应该重复地指出:所谓变劣势为优势和完成反攻准备,是包括中国自己力量的增长、日本困难的增长和国际援助的增长在内的,总合这些力量就能形成中国的优势,完成反攻的准备。

(四四)根据中国政治和经济不平衡的状态,第三阶段的战略反攻,在其前一时期将不是全国整齐划一的姿态,而是带地域性的和此起彼落的姿态。敌人采用各种分化手段破裂中国统一战线的企图,此阶段中并不会减弱,因此,中国内部团结的任务更加重要,务不令内部不调致战略反攻半途而废。此时期中,国际形势将变到大有利于中国。中国的任务,就在于利用这种国际形势取得自己的彻底解放,建立独立的民主国家,同时也就是帮助世界的反法西斯运动。

(四五)中国由劣势到平衡到优势,日本由优势到平衡到劣势,中国由防御到相持到反攻,日本由进攻到保守到退却——这就是中日战争的过程,中日战争的必然趋势。

(四六)于是问题和结论是:中国会亡吗?答复:不会亡,最后胜利是中国的。中国能够速胜吗?答复:不能速胜,必须是持久战。这个结论是正确的吗?我以为是正确的。

(四七)讲到这里,亡国论和妥协论者又将跑出来说:中国由劣势到平衡,需要有同日本相等的军力和经济力;由平衡到优势,需要有超过日本的军力和经济力;然而这是不可能的,因此上述结论是不正确的。

(四八)这就是所谓“唯武器论”,是战争问题中的机械论,是主观地和片面地看问题的意见。我们的意见与此相反,不但看到武器,而且看到人力。武器是战争的重要的因素,但不是决定的因素,决定的因素是人不是物。力量对比不但是军力和经济力的对比,而且是人力和人心的对比。军力和经济力是要人去掌握的。如果中国人的大多数、日本人的大多数、世界各国人的大多数是站在抗日战争方面的话,那末,日本少数人强制地掌握着的军力和经济力,还能算是优势吗?它不是优势,那末,掌握比较劣势的军力和经济力的中国,不就成了优势吗?没有疑义,中国只要坚持抗战和坚持统一战线,其军力和经济力是能够逐渐地加强的。而我们的敌人,经过长期战争和内外矛盾的削弱,其军力和经济力又必然要起相反的变化。在这种情况下,难道中国也不能变成优势吗?还不止此,目前我们不能把别国的军力和经济力大量地公开地算作自己方面的力量,难道将来也不能吗?如果日本的敌人不止中国一个,如果将来有一国或几国以其相当大量的军力和经济力公开地防御或攻击日本,公开地援助我们,那末,优势不更在我们一方面吗?日本是小国,其战争是退步的和野蛮的,其国际地位将益处于孤立;中国是大国,其战争是进步的和正义的,其国际地位将益处于多助。所有这些,经过长期发展,难道还不能使敌我优劣的形势确定地发生变化吗?

(四九)速胜论者则不知道战争是力量的竞赛,在战争双方的力量对比没有起一定的变化以前,就要举行战略的决战,就想提前到达解放之路,也是没有根据的。其意见实行起来,一定不免于碰壁。或者只是空谈快意,并不准备真正去做。最后则是事实先生跑将出来,给这些空谈家一瓢冷水,证明他们不过是一些贪便宜、想少费气力多得收成的空谈主义者。这种空谈主义过去和现在已经存在,但是还不算很多,战争发展到相持阶段和反攻阶段时,空谈主义可能多起来。但是在同时,如果第一阶段中国损失较大,第二阶段时间拖得很长,亡国论和妥协论更将大大地流行。所以我们的火力,应该主要地向着亡国论和妥协论方面,而以次要的火力,反对空谈主义的速胜论。

(五○)战争的长期性是确定了的,但是战争究将经过多少年月则谁也不能预断,这个完全要看敌我力量变化的程度才能决定。一切想要缩短战争时间的人们,惟有努力于增加自己力量减少敌人力量之一法。具体地说,惟有努力于作战多打胜仗,消耗敌人的军队,努力于发展游击战争,使敌之占领地限制于最小的范围,努力于巩固和扩大统一战线,团结全国力量,努力于建设新军和发展新的军事工业,努力于推动政治、经济和文化的进步,努力于工、农、商、学各界人民的动员,努力于瓦解敌军和争取敌军的士兵,努力于国际宣传争取国际的援助,努力于争取日本的人民及其它被压迫民族的援助,做了这一切,才能缩短战争的时间,此外不能有任何取巧图便的法门。

犬牙交错的战争

(五一)我们可以断言,持久战的抗日战争,将在人类战争史中表现为光荣的特殊的一页。犬牙交错的战争形态,就是颇为特殊的一点,这是由于日本的野蛮和兵力不足,中国的进步和土地广大这些矛盾因素产生出来的。犬牙交错的战争,在历史上也是有过的,俄国十月革命后的三年内战,就有过这种情形。但其在中国的特点,是其特殊的长期性和广大性,这将是突破历史纪录的东西。这种犬牙交错的形态,表现在下述的几种情况上。

(五二)内线和外线——抗日战争是整个处于内线作战的地位的;但是主力军和游击队的关系,则是主力军在内线,游击队在外线,形成夹攻敌人的奇观。各游击区的关系亦然。各个游击区都以自己为内线,而以其它各区为外线,又形成了很多夹攻敌人的火线。在战争的第一阶段,战略上内线作战的正规军是后退的,但是战略上外线作战的游击队则将广泛地向着敌人后方大踏步前进,第二阶段将更加猛烈地前进,形成了后退和前进的奇异形态。

(五三)有后方和无后方——利用国家的总后方,而把作战线伸至敌人占领地之最后限界的,是主力军。脱离总后方,而把作战线伸至敌后的,是游击队。但在每一游击区中,仍自有其小规模的后方,并依以建立非固定的作战线。和这个区别的,是每一游击区派遣出去向该区敌后临时活动的游击队,他们不但没有后方,也没有作战线。“无后方的作战”,是新时代中领土广大、人民进步、有先进政党和先进军队的情况之下的革命战争的特点,没有可怕而有大利,不应怀疑而应提倡。

(五四)包围和反包围——从整个战争看来,由于敌之战略进攻和外线作战,我处战略防御和内线作战地位,无疑我是在敌之战略包围中。这是敌对于我之第一种包围。由于我以数量上优势的兵力,对于从战略上的外线分数路向我前进之敌,采取战役和战斗上的外线作战方针,就可以把各路分进之敌的一路或几路放在我之包围中。这是我对于敌之第一种反包围。再从敌后游击战争的根据地看来,每一孤立的根据地都处于敌之四面或三面包围中,前者例如五台山,后者例如晋西北。这是敌对于我之第二种包围。但若将各个游击根据地联系起来看,并将各个游击根据地和正规军的阵地也联系起来看,我又把许多敌人都包围起来,例如在山西,我已三面包围了同蒲路(路之东西两侧及南端),四面包围了太原城;河北、山东等省也有许多这样的包围。这又是我对于敌之第二种反包围。这样,敌我各有加于对方的两种包围,大体上好似下围棋一样,敌对于我我对于敌之战役和战斗的作战,好似吃子,敌的据点(例如太原)和我之游击根据地(例如五台山),好似做眼。如果把世界性的围棋也算在内,那就还有第三种敌我包围,这就是侵略阵线与和平阵线的关系。敌以前者来包围中、苏、法、捷等国,我以后者反包围德、日、意。但是我之包围好似如来佛的手掌,它将化成一座横亘宇宙的五行山,把这几个新式孙悟空——法西斯侵略主义者,最后压倒在山底下,永世也不得翻身\footnote{这个比喻里所引用的神话故事,见明朝吴承恩所著的《西游记》第七回。这个神话故事说,孙悟空本是个猴子,他能够一个筋斗翻十万八千里,但是,他站在如来佛的手心上尽力翻筋斗,总是翻不出去。如来佛翻掌一扑,将五个手指化作五行山,把他压住。}。如果我能在外交上建立太平洋反日阵线,把中国作为一个战略单位,又把苏联及其它可能的国家也各作为一个战略单位,又把日本人民运动也作为一个战略单位,形成一个使法西斯孙悟空无处逃跑的天罗地网,那就是敌人死亡之时了。实际上,日本帝国主义完全打倒之日,必是这个天罗地网大体布成之时。这丝毫也不是笑话,而是战争的必然的趋势。

(五五)大块和小块——一种可能,是敌占地区将占中国本部之大半,而中国本部完整的区域只占一小半。这是一种情形。但是敌占大半中,除东三省等地外,实际只能占领大城市、大道和某些平地,依重要性说是一等的,依面积和人口来说可能只是敌占区中之小半,而普遍地发展的游击区,反居其大半。这又是一种情形。如果超越本部的范围,而把蒙古、新疆、青海、西藏算了进来,则在面积上中国未失地区仍然是大半,而敌占地区包括东三省在内,也只是小半。这又是一种情形。完整区域当然是重要的,应集大力去经营,不但政治、军事、经济等方面,文化方面也要紧。敌人已将我们过去的文化中心变为文化落后区域,而我们则要将过去的文化落后区域变为文化中心。同时,敌后广大游击区的经营也是非常之要紧的,也应把它们的各方面发展起来,也应发展其文化工作。总起来看,中国将是大块的乡村变为进步和光明的地区,而小块的敌占区,尤其是大城市,将暂时地变为落后和黑暗的地区。

(五六)这样看来,长期而又广大的抗日战争,是军事、政治、经济、文化各方面犬牙交错的战争,这是战争史上的奇观,中华民族的壮举,惊天动地的伟业。这个战争,不但将影响到中日两国,大大推动两国的进步,而且将影响到世界,推动各国首先是印度等被压迫民族的进步。全中国人都应自觉地投入这个犬牙交错的战争中去,这就是中华民族自求解放的战争形态,是半殖民地大国在二十世纪三十和四十年代举行的解放战争的特殊的形态。

为永久和平而战

(五七)中国抗日战争的持久性同争取中国和世界的永久和平,是不能分离的。没有任何一个历史时期像今天一样,战争是接近于永久和平的。由于阶级的出现,几千年来人类的生活中充满了战争,每一个民族都不知打了几多仗,或在民族集团之内打,或在民族集团之间打。打到资本主义社会的帝国主义时期,仗就打得特别广大和特别残酷。二十年前的第一次帝国主义大战,在过去历史上是空前的,但还不是绝后的战争。只有目前开始了的战争,接近于最后战争,就是说,接近于人类的永久和平。目前世界上已有三分之一的人口进入了战争,你们看,一个意大利,又一个日本,一个阿比西尼亚,又一个西班牙,再一个中国。参加战争的这些国家共有差不多六万万人口,几乎占了全世界总人口的三分之一。目前的战争的特点是无间断和接近永久和平的性质。为什么无间断?意大利同阿比西尼亚打了之后,接着意大利同西班牙打,德国也搭了股份,接着日本又同中国打。还要接着谁呢?无疑地要接着希特勒同各大国打。“法西斯主义就是战争”\footnote{一九三五年八月,季米特洛夫在共产国际第七次代表大会上所作的报告中说:“法西斯是肆无忌惮的沙文主义和侵略战争。”一九三七年七月,他又发表了题为《法西斯主义就是战争》的论文。},一点也不错。目前的战争发展到世界大战之间,是不会间断的,人类的战争灾难不可避免。为什么又说这次战争接近于永久和平?这次战争是在第一次世界大战所已开始的世界资本主义总危机发展的基础上发生的,由于这种总危机,逼使各资本主义国家走入新的战争,首先逼使各法西斯国家从事于新战争的冒险。我们可以预见这次战争的结果,将不是资本主义的获救,而是它的走向崩溃。这次战争,将比二十年前的战争更大,更残酷,一切民族将无可避免地卷入进去,战争时间将拖得很长,人类将遭受很大的痛苦。但是由于苏联的存在和世界人民觉悟程度的提高,这次战争中无疑将出现伟大的革命战争,用以反对一切反革命战争,而使这次战争带着为永久和平而战的性质。即使尔后尚有一个战争时期,但是已离世界的永久和平不远了。人类一经消灭了资本主义,便到达永久和平的时代,那时候便再也不要战争了。那时将不要军队,也不要兵船,不要军用飞机,也不要毒气。从此以后,人类将亿万斯年看不见战争。已经开始了的革命的战争,是这个为永久和平而战的战争的一部分。占着五万万以上人口的中日两国之间的战争,在这个战争中将占着重要的地位,中华民族的解放将从这个战争中得来。将来的被解放了的新中国,是和将来的被解放了的新世界不能分离的。因此,我们的抗日战争包含着为争取永久和平而战的性质。

(五八)历史上的战争分为两类,一类是正义的,一类是非正义的。一切进步的战争都是正义的,一切阻碍进步的战争都是非正义的。我们共产党人反对一切阻碍进步的非正义的战争,但是不反对进步的正义的战争。对于后一类战争,我们共产党人不但不反对,而且积极地参加。前一类战争,例如第一次世界大战,双方都是为着帝国主义利益而战,所以全世界的共产党人坚决地反对那一次战争。反对的方法,在战争未爆发前,极力阻止其爆发;既爆发后,只要有可能,就用战争反对战争,用正义战争反对非正义战争。日本的战争是阻碍进步的非正义的战争,全世界人民包括日本人民在内,都应该反对,也正在反对。我们中国,则从人民到政府,从共产党到国民党,一律举起了义旗,进行了反侵略的民族革命战争。我们的战争是神圣的、正义的,是进步的、求和平的。不但求一国的和平,而且求世界的和平,不但求一时的和平,而且求永久的和平。欲达此目的,便须决一死战,便须准备着一切牺牲,坚持到底,不达目的,决不停止。牺牲虽大,时间虽长,但是永久和平和永久光明的新世界,已经鲜明地摆在我们的前面。我们从事战争的信念,便建立在这个争取永久和平和永久光明的新中国和新世界的上面。法西斯主义和帝国主义要把战争延长到无尽期,我们则要把战争在一个不很久远的将来给以结束。为了这个目的,人类大多数应该拿出极大的努力。四亿五千万的中国人占了全人类的四分之一,如果能够一齐努力,打倒了日本帝国主义,创造了自由平等的新中国,对于争取全世界永久和平的贡献,无疑地是非常伟大的。这种希望不是空的,全世界社会经济的行程已经接近了这一点,只须加上多数人的努力,几十年工夫一定可以达到目的。

能动性在战争中

(五九)以上说的,都是说明为什么是持久战和为什么最后胜利是中国的,大体上都是说的“是什么”和“不是什么”。以下,将转到研究“怎样做”和“不怎样做”的问题上。怎样进行持久战和怎样争取最后胜利?这就是以下要答复的问题。为了这个,我们将依次说明下列的问题:能动性在战争中,战争和政治,抗战的政治动员,战争的目的,防御中的进攻,持久中的速决,内线中的外线,主动性,灵活性,计划性,运动战,游击战,阵地战,歼灭战,消耗战,乘敌之隙的可能性,抗日战争的决战问题,兵民是胜利之本。我们现在就从能动性问题说起吧。

(六○)我们反对主观地看问题,说的是一个人的思想,不根据和不符合于客观事实,是空想,是假道理,如果照了做去,就要失败,故须反对它。但是一切事情是要人做的,持久战和最后胜利没有人做就不会出现。做就必须先有人根据客观事实,引出思想、道理、意见,提出计划、方针、政策、战略、战术,方能做得好。思想等等是主观的东西,做或行动是主观见之于客观的东西,都是人类特殊的能动性。这种能动性,我们名之曰“自觉的能动性”,是人之所以区别于物的特点。一切根据和符合于客观事实的思想是正确的思想,一切根据于正确思想的做或行动是正确的行动。我们必须发扬这样的思想和行动,必须发扬这种自觉的能动性。抗日战争是要赶走帝国主义,变旧中国为新中国,必须动员全中国人民,统统发扬其抗日的自觉的能动性,才能达到目的。坐着不动,只有被灭亡,没有持久战,也没有最后胜利。

(六一)自觉的能动性是人类的特点。人类在战争中强烈地表现出这样的特点。战争的胜负,固然决定于双方军事、政治、经济、地理、战争性质、国际援助诸条件,然而不仅仅决定于这些;仅有这些,还只是有了胜负的可能性,它本身没有分胜负。要分胜负,还须加上主观的努力,这就是指导战争和实行战争,这就是战争中的自觉的能动性。

(六二)指导战争的人们不能超越客观条件许可的限度期求战争的胜利,然而可以而且必须在客观条件的限度之内,能动地争取战争的胜利。战争指挥员活动的舞台,必须建筑在客观条件的许可之上,然而他们凭借这个舞台,却可以导演出很多有声有色、威武雄壮的戏剧来。在既定的客观物质的基础之上,抗日战争的指挥员就要发挥他们的威力,提挈全军,去打倒那些民族的敌人,改变我们这个被侵略被压迫的社会国家的状态,造成自由平等的新中国,这里就用得着而且必须用我们的主观指导的能力。我们不赞成任何一个抗日战争的指挥员,离开客观条件,变为乱撞乱碰的鲁莽家,但是我们必须提倡每个抗日战争的指挥员变为勇敢而明智的将军。他们不但要有压倒敌人的勇气,而且要有驾驭整个战争变化发展的能力。指挥员在战争的大海中游泳,他们要不使自己沉没,而要使自己决定地有步骤地到达彼岸。作为战争指导规律的战略战术,就是战争大海中的游泳术。

战争和政治

(六三)“战争是政治的继续”,在这点上说,战争就是政治,战争本身就是政治性质的行动,从古以来没有不带政治性的战争。抗日战争是全民族的革命战争,它的胜利,离不开战争的政治目的——驱逐日本帝国主义、建立自由平等的新中国,离不开坚持抗战和坚持统一战线的总方针,离不开全国人民的动员,离不开官兵一致、军民一致和瓦解敌军等项政治原则,离不开统一战线政策的良好执行,离不开文化的动员,离不开争取国际力量和敌国人民援助的努力。一句话,战争一刻也离不了政治。抗日军人中,如有轻视政治的倾向,把战争孤立起来,变为战争绝对主义者,那是错误的,应加纠正。

(六四)但是战争有其特殊性,在这点上说,战争不即等于一般的政治。“战争是政治的特殊手段的继续”\footnote{参见列宁《第二国际的破产》和《社会主义与战争》(《列宁全集》第26卷,人民出版社1988年版,第235、327页)。}。政治发展到一定的阶段,再也不能照旧前进,于是爆发了战争,用以扫除政治道路上的障碍。例如中国的半独立地位,是日本帝国主义政治发展的障碍,日本要扫除它,所以发动了侵略战争。中国呢?帝国主义压迫,早就是中国资产阶级民主革命的障碍,所以有了很多次的解放战争,企图扫除这个障碍。日本现在用战争来压迫,要完全断绝中国革命的进路,所以不得不举行抗日战争,决心要扫除这个障碍。障碍既除,政治的目的达到,战争结束。障碍没有扫除得干净,战争仍须继续进行,以求贯彻。例如抗日的任务未完,有想求妥协的,必不成功;因为即使因某种缘故妥协了,但是战争仍要起来,广大人民必定不服,必要继续战争,贯彻战争的政治目的。因此可以说,政治是不流血的战争,战争是流血的政治。

(六五)基于战争的特殊性,就有战争的一套特殊组织,一套特殊方法,一种特殊过程。这组织,就是军队及其附随的一切东西。这方法,就是指导战争的战略战术。这过程,就是敌对的军队互相使用有利于己不利于敌的战略战术从事攻击或防御的一种特殊的社会活动形态。因此,战争的经验是特殊的。一切参加战争的人们,必须脱出寻常习惯,而习惯于战争,方能争取战争的胜利。

抗日的政治动员

(六六)如此伟大的民族革命战争,没有普遍和深入的政治动员,是不能胜利的。抗日以前,没有抗日的政治动员,这是中国的大缺陷,已经输了敌人一着。抗日以后,政治动员也非常之不普遍,更不说深入。人民的大多数,是从敌人的炮火和飞机炸弹那里听到消息的。这也是一种动员,但这是敌人替我们做的,不是我们自己做的。偏远地区听不到炮声的人们,至今还是静悄悄地在那里过活。这种情形必须改变,不然,拚死活的战争就得不到胜利。决不可以再输敌人一着,相反,要大大地发挥这一着去制胜敌人。这一着是关系绝大的;武器等等不如人尚在其次,这一着实在是头等重要。动员了全国的老百姓,就造成了陷敌于灭顶之灾的汪洋大海,造成了弥补武器等等缺陷的补救条件,造成了克服一切战争困难的前提。要胜利,就要坚持抗战,坚持统一战线,坚持持久战。然而一切这些,离不开动员老百姓。要胜利又忽视政治动员,叫做“南其辕而北其辙”,结果必然取消了胜利。

(六七)什么是政治动员呢?首先是把战争的政治目的告诉军队和人民。必须使每个士兵每个人民都明白为什么要打仗,打仗和他们有什么关系。抗日战争的政治目的是“驱逐日本帝国主义,建立自由平等的新中国”,必须把这个目的告诉一切军民人等,方能造成抗日的热潮,使几万万人齐心一致,贡献一切给战争。其次,单单说明目的还不够,还要说明达到此目的的步骤和政策,就是说,要有一个政治纲领。现在已经有了《抗日救国十大纲领》,又有了一个《抗战建国纲领》,应把它们普及于军队和人民,并动员所有的军队和人民实行起来。没有一个明确的具体的政治纲领,是不能动员全军全民抗日到底的。其次,怎样去动员?靠口说,靠传单布告,靠报纸书册,靠戏剧电影,靠学校,靠民众团体,靠干部人员。现在国民党统治地区有的一些,沧海一粟,而且方法不合民众口味,神气和民众隔膜,必须切实地改一改。其次,不是一次动员就够了,抗日战争的政治动员是经常的。不是将政治纲领背诵给老百姓听,这样的背诵是没有人听的;要联系战争发展的情况,联系士兵和老百姓的生活,把战争的政治动员,变成经常的运动。这是一件绝大的事,战争首先要靠它取得胜利。

战争的目的

(六八)这里不是说战争的政治目的,抗日战争的政治目的是“驱逐日本帝国主义,建立自由平等的新中国”,前面已经说过了。这里说的,是作为人类流血的政治的所谓战争,两军相杀的战争,它的根本目的是什么。战争的目的不是别的,就是“保存自己,消灭敌人”(消灭敌人,就是解除敌人的武装,也就是所谓“剥夺敌人的抵抗力”,不是要完全消灭其肉体)。古代战争,用矛用盾:矛是进攻的,为了消灭敌人;盾是防御的,为了保存自己。直到今天的武器,还是这二者的继续。轰炸机、机关枪、远射程炮、毒气,是矛的发展;防空掩蔽部、钢盔、水泥工事、防毒面具,是盾的发展。坦克,是矛盾二者结合为一的新式武器。进攻,是消灭敌人的主要手段,但防御也是不能废的。进攻,是直接为了消灭敌人的,同时也是为了保存自己,因为如不消灭敌人,则自己将被消灭。防御,是直接为了保存自己的,但同时也是辅助进攻或准备转入进攻的一种手段。退却,属于防御一类,是防御的继续;而追击,则是进攻的继续。应该指出:战争目的中,消灭敌人是主要的,保存自己是第二位的,因为只有大量地消灭敌人,才能有效地保存自己。因此,作为消灭敌人之主要手段的进攻是主要的,而作为消灭敌人之辅助手段和作为保存自己之一种手段的防御,是第二位的。战争实际中,虽有许多时候以防御为主,而在其余时候以进攻为主,然而通战争的全体来看,进攻仍然是主要的。

(六九)怎样解释战争中提倡勇敢牺牲呢?岂非与“保存自己”相矛盾?不相矛盾,是相反相成的。战争是流血的政治,是要付代价的,有时是极大的代价。部分的暂时的牺牲(不保存),为了全体的永久的保存。我们说,基本上为着消灭敌人的进攻手段中,同时也含了保存自己的作用,理由就在这里。防御必须同时有进攻,而不应是单纯的防御,也是这个道理。

(七○)保存自己消灭敌人这个战争的目的,就是战争的本质,就是一切战争行动的根据,从技术行动起,到战略行动止,都是贯彻这个本质的。战争目的,是战争的基本原则,一切技术的、战术的、战役的、战略的原理原则,一点也离不开它。射击原则的“荫蔽身体,发扬火力”是什么意思呢?前者为了保存自己,后者为了消灭敌人。因为前者,于是利用地形地物,采取跃进运动,疏开队形,种种方法都发生了。因为后者,于是扫清射界,组织火网,种种方法也发生了。战术上的突击队、钳制队、预备队,第一种为了消灭敌人,第二种为了保存自己,第三种准备依情况使用于两个目的——或者增援突击队,或者作为追击队,都是为了消灭敌人;或者增援钳制队,或者作为掩护队,都是为了保存自己。照这样,一切技术、战术、战役、战略原则,一切技术、战术、战役、战略行动,一点也离不开战争的目的,它普及于战争的全体,贯彻于战争的始终。

(七一)抗日战争的各级指导者,不能离开中日两国之间各种互相对立的基本因素去指导战争,也不能离开这个战争目的去指导战争。两国之间各种互相对立的基本因素展开于战争的行动中,就变成互相为了保存自己消灭敌人而斗争。我们的战争,在于力求每战争取不论大小的胜利,在于力求每战解除敌人一部分武装,损伤敌人一部分人马器物。把这些部分地消灭敌人的成绩积累起来,成为大的战略胜利,达到最后驱敌出国,保卫祖国,建设新中国的政治目的。

防御中的进攻,持久中的速决,内线中的外线

(七二)现在来研究抗日战争中的具体的战略方针。我们已说过了,抗日的战略方针是持久战,是的,这是完全对的。但这是一般的方针,还不是具体的方针。怎样具体地进行持久战呢?这就是我们现在要讨论的问题。我们的答复是:在第一和第二阶段即敌之进攻和保守阶段中,应该是战略防御中的战役和战斗的进攻战,战略持久中的战役和战斗的速决战,战略内线中的战役和战斗的外线作战。在第三阶段中,应该是战略的反攻战。

(七三)由于日本是帝国主义的强国,我们是半殖民地半封建的弱国,日本是采取战略进攻方针的,我们则居于战略防御地位。日本企图采取战略的速决战,我们应自觉地采取战略的持久战。日本用其战斗力颇强的几十个师团的陆军(目前已到了三十个师团)和一部分海军,从陆海两面包围和封锁中国,又用空军轰炸中国。目前日本的陆军已占领从包头到杭州的长阵线,海军则到了福建广东,形成了大范围的外线作战。我们则处于内线作战地位。所有这些,都是由敌强我弱这个特点造成的。这是一方面的情形。

(七四)然而在另一方面,则适得其反。日本虽强,但兵力不足。中国虽弱,但地大、人多、兵多。这里就产生了两个重要的结果。第一,敌以少兵临大国,就只能占领一部分大城市、大道和某些平地。由是,在其占领区域,则空出了广大地面无法占领,这就给了中国游击战争以广大活动的地盘。在全国,即使敌能占领广州、武汉、兰州之线及其附近的地区,但以外的地区是难于占领的,这就给了中国以进行持久战和争取最后胜利的总后方和中枢根据地。第二,敌以少兵临多兵,便处于多兵的包围中。敌分路向我进攻,敌处战略外线,我处战略内线,敌是战略进攻,我是战略防御,看起来我是很不利的。然而我可以利用地广和兵多两个长处,不作死守的阵地战,采用灵活的运动战,以几个师对他一个师,几万人对他一万人,几路对他一路,从战场的外线,突然包围其一路而攻击之。于是敌之战略作战上的外线和进攻,在战役和战斗的作战上,就不得不变成内线和防御。我之战略作战上的内线和防御,在战役和战斗的作战上就变成了外线和进攻。对其一路如此,对其它路也是如此。以上两点,都是从敌小我大这一特点发生的。又由于敌兵虽少,乃是强兵(武器和人员的教养程度),我兵虽多,乃是弱兵(也仅是武器和人员的教养程度,不是士气),因此,在战役和战斗的作战上,我不但应以多兵打少兵,从外线打内线,还须采取速决战的方针。为了实行速决,一般应不打驻止中之敌,而打运动中之敌。我预将大兵荫蔽集结于敌必经通路之侧,乘敌运动之际,突然前进,包围而攻击之,打他一个措手不及,迅速解决战斗。打得好,可能全部或大部或一部消灭他;打不好,也给他一个大的杀伤。一战如此,他战皆然。不说多了,每个月打得一个较大的胜仗,如像平型关台儿庄一类的,就能大大地沮丧敌人的精神,振起我军的士气,号召世界的声援。这样,我之战略的持久战,到战场作战就变成速决战了。敌之战略的速决战,经过许多战役和战斗的败仗,就不得不改为持久战。

(七五)上述这样的战役和战斗的作战方针,一句话说完,就是:“外线的速决的进攻战”。这对于我之战略方针“内线的持久的防御战”说来,是相反的;然而,又恰是实现这样的战略方针之必要的方针。如果战役和战斗方针也同样是“内线的持久的防御战”,例如抗战初起时期之所为,那就完全不适合敌小我大、敌强我弱这两种情况,那就决然达不到战略目的,达不到总的持久战,而将为敌人所击败。所以,我们历来主张全国组成若干个大的野战兵团,其兵力针对着敌人每个野战兵团之兵力而二倍之、三倍之或四倍之,采用上述方针,与敌周旋于广大战场之上。这种方针,不但是正规战争用得着,游击战争也用得着,而且必须要用它。不但适用于战争的某一阶段,而且适用于战争的全过程。战略反攻阶段,我之技术条件增强,以弱敌强这种情况即使完全没有了,我仍用多兵从外线采取速决的进攻战,就更能收大批俘获的成效。例如我用两个或三个或四个机械化的师对敌一个机械化的师,更能确定地消灭这个师。几个大汉打一个大汉之容易打胜,这是常识中包含的真理。

(七六)如果我们坚决地采取了战场作战的“外线的速决的进攻战”,就不但在战场上改变着敌我之间的强弱优劣形势,而且将逐渐地变化着总的形势。在战场上,因为我是进攻,敌是防御;我是多兵处外线,敌是少兵处内线;我是速决,敌虽企图持久待援,但不能由他作主;于是在敌人方面,强者就变成了弱者,优势就变成了劣势;我军方面反之,弱者变成了强者,劣势变成了优势。在打了许多这样的胜仗之后,总的敌我形势便将引起变化。这就是说,集合了许多战场作战的外线的速决的进攻战的胜利以后,就逐渐地增强了自己,削弱了敌人,于是总的强弱优劣形势,就不能不受其影响而发生变化。到那时,配合着我们自己的其它条件,再配合着敌人内部的变动和国际上的有利形势,就能使敌我总的形势走到平衡,再由平衡走到我优敌劣。那时,就是我们实行反攻驱敌出国的时机了。

(七七)战争是力量的竞赛,但力量在战争过程中变化其原来的形态。在这里,主观的努力,多打胜仗,少犯错误,是决定的因素。客观因素具备着这种变化的可能性,但实现这种可能性,就需要正确的方针和主观的努力。这时候,主观作用是决定的了。

主动性,灵活性,计划性

(七八)上面说过的战役和战斗的外线的速决的进攻战,中心点在于一个进攻;外线是说的进攻的范围,速决是说的进攻的时间,所以叫它做“外线的速决的进攻战”。这是实行持久战的最好的方针,也即是所谓运动战的方针。但是这个方针实行起来,离不了主动性、灵活性和计划性。我们现在就来研究这三个问题。

(七九)前面已说过了自觉的能动性,为什么又说主动性呢?自觉的能动性,说的是自觉的活动和努力,是人之所以区别于物的特点,这种人的特点,特别强烈地表现于战争中,这些是前面说过了的。这里说的主动性,说的是军队行动的自由权,是用以区别于被迫处于不自由状态的。行动自由是军队的命脉,失了这种自由,军队就接近于被打败或被消灭。一个士兵被缴械,是这个士兵失了行动自由被迫处于被动地位的结果。一个军队的战败,也是一样。为此缘故,战争的双方,都力争主动,力避被动。我们提出的外线的速决的进攻战,以及为了实现这种进攻战的灵活性、计划性,可以说都是为了争取主动权,以便逼敌处于被动地位,达到保存自己消灭敌人之目的。但主动或被动是和战争力量的优势或劣势分不开的。因而也是和主观指导的正确或错误分不开的。此外,也还有利用敌人的错觉和不意来争取自己主动和逼敌处于被动的情形。下面就来分析这几点。

(八○)主动是和战争力量的优势不能分离的,而被动则和战争力量的劣势分不开。战争力量的优势或劣势,是主动或被动的客观基础。战略的主动地位,自然以战略的进攻战为较能掌握和发挥,然而贯彻始终和普及各地的主动地位,即绝对的主动权,只有以绝对优势对绝对劣势才有可能。一个身体壮健者和一个重病患者角斗,前者便有绝对的主动权。如果日本没有许多不可克服的矛盾,例如它能一下出几百万至一千万大兵,财源比现在多过几倍,又没有民众和外国的敌对,又不实行野蛮政策招致中国人民拚死命反抗,那它便能保持一种绝对的优势,它便有一种贯彻始终和普及各地的绝对的主动权。但在历史上,这类绝对优势的事情,在战争和战役的结局是存在的,战争和战役的开头则少见。例如在第一次世界大战中,德国屈服的前夜,这时协约国变成了绝对优势,德国则变成了绝对劣势,结果德国失败,协约国获胜,这是战争结局存在着绝对的优势和劣势之例。又如台儿庄胜利的前夜,这时当地孤立的日军经过苦战之后,已处于绝对的劣势,我军则造成了绝对的优势,结果敌败我胜,这是战役结局存在着绝对的优势和劣势之例。战争或战役也有以相对的优劣或平衡状态而结局的,那时,在战争则出现妥协,在战役则出现对峙。但一般是以绝对的优劣而分胜负居多数。所有这些,都是战争或战役的结局,而非战争或战役的开头。

中日战争的最后结局,可以预断,日本将以绝对劣势而失败,中国将以绝对优势而获胜;但是在目前,则双方的优劣都不是绝对的而是相对的。日本因其具有强的军力、经济力和政治组织力这个有利因素,对于我们弱的军力、经济力和政治组织力,占了优势,因而造成了它的主动权的基础。但是因为它的军力等等数量不多,又有其它许多不利因素,它的优势便为它自己的矛盾所减杀。及到中国,又碰到了中国的地大、人多、兵多和坚强的民族抗战,它的优势再为之减杀。于是在总的方面,它的地位就变成一种相对的优势,因而其主动权的发挥和维持就受了限制,也成了相对的东西。中国方面,虽然在力量的强度上是劣势,因此造成了战略上的某种被动姿态,但是在地理、人口和兵员的数量上,并且又在人民和军队的敌忾心和士气上,却处于优势,这种优势再加上其它的有利因素,便减杀了自己军力、经济力等的劣势的程度,使之变为战略上的相对的劣势。因而也减少了被动的程度,仅处于战略上的相对的被动地位。然而被动总是不利的,必须力求脱离它。军事上的办法,就是坚决地实行外线的速决的进攻战和发动敌后的游击战争,在战役的运动战和游击战中取得许多局部的压倒敌人的优势和主动地位。通过这样许多战役的局部优势和局部主动地位,就能逐渐地造成战略的优势和战略的主动地位,战略的劣势和被动地位就能脱出了。这就是主动和被动之间、优势和劣势之间的相互关系。

(八一)由此也就可以明白主动或被动和主观指导之间的关系。如上所述,我之相对的战略劣势和战略被动地位,是能够脱出的,方法就是人工地造成我们许多的局部优势和局部主动地位,去剥夺敌人的许多局部优势和局部主动地位,把他抛入劣势和被动。把这些局部的东西集合起来,就成了我们的战略优势和战略主动,敌人的战略劣势和战略被动。这样的转变,依靠主观上的正确指导。为什么呢?我要优势和主动,敌人也要这个,从这点上看,战争就是两军指挥员以军力财力等项物质基础作地盘,互争优势和主动的主观能力的竞赛。竞赛结果,有胜有败,除了客观物质条件的比较外,胜者必由于主观指挥的正确,败者必由于主观指挥的错误。我们承认战争现象是较之任何别的社会现象更难捉摸,更少确实性,即更带所谓“盖然性”。但战争不是神物,仍是世间的一种必然运动,因此,孙子的规律,“知彼知己,百战不殆”,仍是科学的真理。错误由于对彼己的无知,战争的特性也使人们在许多的场合无法全知彼己,因此产生了战争情况和战争行动的不确实性,产生了错误和失败。然而不管怎样的战争情况和战争行动,知其大略,知其要点,是可能的。先之以各种侦察手段,继之以指挥员的聪明的推论和判断,减少错误,实现一般的正确指导,是做得到的。我们有了这个“一般地正确的指导”做武器,就能多打胜仗,就能变劣势为优势,变被动为主动。这是主动或被动和主观指导的正确与否之间的关系。

(八二)主观指导的正确与否,影响到优势劣势和主动被动的变化,观于强大之军打败仗、弱小之军打胜仗的历史事实而益信。中外历史上这类事情是多得很的。中国如晋楚城濮之战,楚汉成皋之战,韩信破赵之战,新汉昆阳之战,袁曹官渡之战,吴魏赤壁之战,吴蜀彝陵之战,秦晋淝水之战等等,外国如拿破仑的多数战役,十月革命后的苏联内战,都是以少击众,以劣势对优势而获胜。都是先以自己局部的优势和主动,向着敌人局部的劣势和被动,一战而胜,再及其余,各个击破,全局因而转成了优势,转成了主动。在原占优势和主动之敌则反是;由于其主观错误和内部矛盾,可以将其很好的或较好的优势和主动地位,完全丧失,化为败军之将,亡国之君。由此可知,战争力量的优劣本身,固然是决定主动或被动的客观基础,但还不是主动或被动的现实事物,必待经过斗争,经过主观能力的竞赛,方才出现事实上的主动或被动。在斗争中,由于主观指导的正确或错误,可以化劣势为优势,化被动为主动;也可以化优势为劣势,化主动为被动。一切统治王朝打不赢革命军,可见单是某种优势还没有确定主动地位,更没有确定最后胜利。主动和胜利,是可以根据真实的情况,经过主观能力的活跃,取得一定的条件,而由劣势和被动者从优势和主动者手里夺取过来的。

(八三)错觉和不意,可以丧失优势和主动。因而有计划地造成敌人的错觉,给以不意的攻击,是造成优势和夺取主动的方法,而且是重要的方法。错觉是什么呢?“八公山上,草木皆兵”,是错觉之一例。“声东击西”,是造成敌人错觉之一法。在优越的民众条件具备,足以封锁消息时,采用各种欺骗敌人的方法,常能有效地陷敌于判断错误和行动错误的苦境,因而丧失其优势和主动。“兵不厌诈”,就是指的这件事情。什么是不意?就是无准备。优势而无准备,不是真正的优势,也没有主动。懂得这一点,劣势而有准备之军,常可对敌举行不意的攻势,把优势者打败。我们说运动之敌好打,就是因为敌在不意即无准备中。这两件事——造成敌人的错觉和出以不意的攻击,即是以战争的不确实性给予敌人,而给自己以尽可能大的确实性,用以争取我之优势和主动,争取我之胜利。要做到这些,先决条件是优越的民众组织。因此,发动所有一切反对敌人的老百姓,一律武装起来,对敌进行广泛的袭击,同时即用以封锁消息,掩护我军,使敌无从知道我军将在什么地方什么时候去攻击他,造成他的错觉和不意的客观基础,是非常之重要的。过去土地革命战争时代的中国红军,以弱小的军力而常打胜仗,得力于组织起来和武装起来了的民众是非常之大的。民族战争照规矩应比土地革命战争更能获得广大民众的援助;可是因为历史的错误\footnote{蒋介石、汪精卫等在一九二七年背叛革命以后,进行十年的反人民战争,同时又在国民党统治区实行法西斯统治。这就使得中国人民没有可能广泛地组织起来。这个历史错误是应该由蒋介石为首的国民党反动派负责的。},民众是散的,不但仓卒难为我用,且时为敌人所利用。只有坚决地广泛地发动全体的民众,方能在战争的一切需要上给以无穷无尽的供给。在这个给敌以错觉和给敌以不意以便战而胜之的战争方法上,也就一定能起大的作用。我们不是宋襄公,不要那种蠢猪式的仁义道德。我们要把敌人的眼睛和耳朵尽可能地封住,使他们变成瞎子和聋子,要把他们的指挥员的心尽可能地弄得混乱些,使他们变成疯子,用以争取自己的胜利。所有这些,也都是主动或被动和主观指导之间的相互关系。战胜日本是少不了这种主观指导的。

(八四)大抵日本在其进攻阶段中,因其军力之强和利用我之主观上的历史错误和现时错误,它是一般地处于主动地位的。但是这种主动,已随其本身带着许多不利因素及其在战争中也犯了些主观错误(详论见后),与乎我方具备着许多有利因素,而开始了部分的减弱。敌之在台儿庄失败和山西困处,就是显证。我在敌后游击战争的广大发展,则使其占领地的守军完全处于被动地位。虽则敌人此时还在其主动的战略进攻中,但他的主动将随其战略进攻的停止而结束。敌之兵力不足,没有可能作无限制的进攻,这是他不能继续保持主动地位的第一个根源。我之战役的进攻战,在敌后的游击战争及其它条件,这是他不能不停止进攻于一定限度和不能继续保持主动地位的第二个根源。苏联的存在及其它国际变化,是第三个根源。由此可见,敌人的主动地位是有限制的,也是能够破坏的。中国如能在作战方法上坚持主力军的战役和战斗的进攻战,猛烈地发展敌后的游击战争,并从政治上大大地发动民众,我之战略主动地位便能逐渐树立起来。

(八五)现在来说灵活性。灵活性是什么呢?就是具体地实现主动性于作战中的东西,就是灵活地使用兵力。灵活地使用兵力这件事,是战争指挥的中心任务,也是最不容易做好的。战争的事业,除了组织和教育军队,组织和教育人民等项之外,就是使用军队于战斗,而一切都是为了战斗的胜利。组织军队等等固然困难,但使用军队则更加困难,特别是在以弱敌强的情况之中。做这件事需要极大的主观能力,需要克服战争特性中的纷乱、黑暗和不确实性,而从其中找出条理、光明和确实性来,方能实现指挥上的灵活性。

(八六)抗日战争战场作战的基本方针,是外线的速决的进攻战。执行这个方针,有兵力的分散和集中、分进和合击、攻击和防御、突击和钳制、包围和迂回、前进和后退种种的战术或方法。懂得这些战术是容易的,灵活地使用和变换这些战术,就不容易了。这里有时机、地点、部队三个关节。不得其时,不得其地,不得于部队之情况,都将不能取胜。例如进攻某一运动中之敌,打早了,暴露了自己,给了敌人以预防条件;打迟了,敌已集中驻止,变为啃硬骨头。这就是时机问题。突击点选在左翼,恰当敌之弱点,容易取胜;选在右翼,碰在敌人的钉子上,不能奏效。这就是地点问题。以我之某一部队执行某种任务,容易取胜;以另一部队执行同样任务,难于收效。这就是部队情况问题。不但使用战术,还须变换战术。攻击变为防御,防御变为攻击,前进变为后退,后退变为前进,钳制队变为突击队,突击队变为钳制队,以及包围迂回等等之互相变换,依据敌我部队、敌我地形的情况,及时地恰当地给以变换,是灵活性的指挥之重要任务。战斗指挥如此,战役和战略指挥也是如此。

(八七)古人所谓“运用之妙,存乎一心”,这个“妙”,我们叫做灵活性,这是聪明的指挥员的出产品。灵活不是妄动,妄动是应该拒绝的。灵活,是聪明的指挥员,基于客观情况,“审时度势”(这个势,包括敌势、我势、地势等项)而采取及时的和恰当的处置方法的一种才能,即是所谓“运用之妙”。基于这种运用之妙,外线的速决的进攻战就能较多地取得胜利,就能转变敌我优劣形势,就能实现我对于敌的主动权,就能压倒敌人而击破之,而最后胜利就属于我们了。

(八八)现在来说计划性。由于战争所特有的不确实性,实现计划性于战争,较之实现计划性于别的事业,是要困难得多的。然而,“凡事预则立,不预则废”,没有事先的计划和准备,就不能获得战争的胜利。战争没有绝对的确实性,但不是没有某种程度的相对的确实性。我之一方是比较地确实的。敌之一方很不确实,但也有朕兆可寻,有端倪可察,有前后现象可供思索。这就构成了所谓某种程度的相对的确实性,战争的计划性就有了客观基础。近代技术(有线电、无线电、飞机、汽车、铁道、轮船等)的发达,又使战争的计划性增大了可能。但由于战争只有程度颇低和时间颇暂的确实性,战争的计划性就很难完全和固定,它随战争的运动(或流动,或推移)而运动,且依战争范围的大小而有程度的不同。战术计划,例如小兵团和小部队的攻击或防御计划,常须一日数变。战役计划,即大兵团的行动计划,大体能终战役之局,但在该战役内,部分的改变是常有的,全部的改变也间或有之。战略计划,是基于战争双方总的情况而来的,有更大的固定的程度,但也只在一定的战略阶段内适用,战争向着新的阶段推移,战略计划便须改变。战术、战役和战略计划之各依其范围和情况而确定而改变,是战争指挥的重要关节,也即是战争灵活性的具体的实施,也即是实际的运用之妙。抗日战争的各级指挥员,对此应当加以注意。

(八九)有些人,基于战争的流动性,就从根本上否认战争计划或战争方针之相对的固定性,说这样的计划或方针是“机械的”东西。这种意见是错误的。如上条所述,我们完全承认:由于战争情况之只有相对的确实性和战争是迅速地向前流动的(或运动的,推移的),战争的计划或方针,也只应给以相对的固定性,必须根据情况的变化和战争的流动而适时地加以更换或修改,不这样做,我们就变成机械主义者。然而决不能否认一定时间内的相对地固定的战争计划或方针;否认了这点,就否认了一切,连战争本身,连说话的人,都否认了。由于战争的情况和行动都有其相对的固定性,因而应之而生的战争计划或方针,也就必须拿相对的固定性赋予它。例如,由于华北战争的情况和八路军分散作战的行动有其在一定阶段内的固定性,因而在这一定阶段内赋予相对的固定性于八路军的“基本的是游击战,但不放松有利条件下的运动战”这种战略的作战方针,是完全必要的。战役方针,较之上述战略方针适用的时间要短促些,战术方针更加短促,然而都有其一定时间的固定性。否认了这点,战争就无从着手,成为毫无定见,这也不是、那也不是,或者这也是、那也是的战争相对主义了。没有人否认,就是在某一一定时间内适用的方针,它也是在流动的,没有这种流动,就不会有这一方针的废止和另一方针的采用。然而这种流动是有限制的,即流动于执行这一方针的各种不同的战争行动的范围中,而不是这一方针的根本性质的流动,即是说,是数的流动,不是质的流动。这种根本性质,在一定时间内是决不流动的,我们所谓一定时间内的相对的固定性,就是指的这一点。在绝对流动的整个战争长河中有其各个特定阶段上的相对的固定性——这就是我们对于战争计划或战争方针的根本性质的意见。

(九○)在说过了战略上的内线的持久的防御战和战役战斗上的外线的速决的进攻战,又说过了主动性、灵活性和计划性之后,我们可以总起来说几句。抗日战争应该是有计划的。战争计划即战略战术的具体运用,要带灵活性,使之能适应战争的情况。要处处照顾化劣势为优势,化被动为主动,以便改变敌我之间的形势。而一切这些,都表现于战役和战斗上的外线的速决的进攻战,同时也就表现于战略上的内线的持久的防御战之中。

运动战,游击战,阵地战

(九一)作为战争内容的战略内线、战略持久、战略防御中的战役和战斗上的外线的速决的进攻战,在战争形式上就表现为运动战。运动战,就是正规兵团在长的战线和大的战区上面,从事于战役和战斗上的外线的速决的进攻战的形式。同时,也把为了便利于执行这种进攻战而在某些必要时机执行着的所谓“运动性的防御”包括在内,并且也把起辅助作用的阵地攻击和阵地防御包括在内。它的特点是:正规兵团,战役和战斗的优势兵力,进攻性和流动性。

(九二)中国版图广大,兵员众多,但军队的技术和教养不足;敌人则兵力不足,但技术和教养比较优良。在此种情形下,无疑地应以进攻的运动战为主要的作战形式,而以其它形式辅助之,组成整个的运动战。在这里,要反对所谓“有退无进”的逃跑主义,同时也要反对所谓“有进无退”的拚命主义。

(九三)运动战的特点之一,是其流动性,不但许可而且要求野战军的大踏步的前进和后退。然而,这和韩复榘式的逃跑主义\footnote{一九三七年日本侵略军在占领北平、天津以后,不久即分兵沿津浦铁路南下,进攻山东省。多年统治山东的国民党军阀韩复榘不战而逃。在一九三七年十二月下旬至一九三八年一月上旬的十多天里,他就放弃了山东中部和西南部的大片国土,从济南一直逃到山东、河南的边境。}是没有相同之点的。战争的基本要求是:消灭敌人;其另一要求是:保存自己。保存自己的目的,在于消灭敌人;而消灭敌人,又是保存自己的最有效的手段。因此,运动战决不能被韩复榘一类人所借口,决不是只有向后的运动,没有向前的运动;这样的“运动”,否定了运动战的基本的进攻性,实行的结果,中国虽大,也是要被“运动”掉的。

(九四)然而另一种思想也是不对的,即所谓有进无退的拚命主义。我们主张以战役和战斗上的外线的速决的进攻战为内容的运动战,其中包括了辅助作用的阵地战,又包括了“运动性的防御”和退却,没有这些,运动战便不能充分地执行。拚命主义是军事上的近视眼,其根源常是惧怕丧失土地。拚命主义者不知道运动战的特点之一是其流动性,不但许可而且要求野战军的大踏步的进退。积极方面,为了陷敌于不利而利于我之作战,常常要求敌人在运动中,并要求有利于我之许多条件,例如有利的地形、好打的敌情、能封锁消息的居民、敌人的疲劳和不意等。这就要求敌人的前进,虽暂时地丧失部分土地而不惜。因为暂时地部分地丧失土地,是全部地永久地保存土地和恢复土地的代价。消极方面,凡被迫处于不利地位,根本上危及军力的保存时,应该勇敢地退却,以便保存军力,在新的时机中再行打击敌人。拚命主义者不知此理,明明已处于确定了的不利情况,还要争一城一地的得失,结果不但城和地俱失,军力也不能保存。我们历来主张“诱敌深入”,就是因为这是战略防御中弱军对强军作战的最有效的军事政策。

(九五)抗日战争的作战形式中,主要的是运动战,其次就要算游击战了。我们说,整个战争中,运动战是主要的,游击战是辅助的,说的是解决战争的命运,主要是依靠正规战,尤其是其中的运动战,游击战不能担负这种解决战争命运的主要的责任。但这不是说:游击战在抗日战争中的战略地位不重要。游击战在整个抗日战争中的战略地位,仅仅次于运动战,因为没有游击战的辅助,也就不能战胜敌人。这样说,是包括了游击战向运动战发展这一个战略任务在内的。长期的残酷的战争中间,游击战不停止于原来地位,它将把自己提高到运动战。这样,游击战的战略作用就有两方面:一是辅助正规战,一是把自己也变为正规战。至于就游击战在中国抗日战争中的空前广大和空前持久的意义说来,它的战略地位是更加不能轻视的了。因此,在中国,游击战的本身,不只有战术问题,还有它的特殊的战略问题。这个问题,我在《抗日游击战争的战略问题》一文里面已经说到了。

前面说过,抗日战争三个战略阶段的作战形式,第一阶段,运动战是主要的,游击战和阵地战是辅助的。第二阶段,则游击战将升到主要地位,而以运动战和阵地战辅助之。第三阶段,运动战再升为主要形式,而辅之以阵地战和游击战。但这个第三阶段的运动战,已不全是由原来的正规军负担,而将由原来的游击军从游击战提高到运动战去担负其一部分,也许是相当重要的一部分。从三个阶段来看,中国抗日战争中的游击战,决不是可有可无的。它将在人类战争史上演出空前伟大的一幕。为此缘故,在全国的数百万正规军中间,至少指定数十万人,分散于所有一切敌占地区,发动和配合民众武装,从事游击战争,是完全必要的。被指定的军队,要自觉地负担这种神圣任务,不要以为少打大仗,一时显得不像民族英雄,降低了资格,这种想法是错误的。游击战争没有正规战争那样迅速的成效和显赫的名声,但是“路遥知马力,事久见人心”,在长期和残酷的战争中,游击战争将表现其很大的威力,实在是非同小可的事业。并且正规军分散作游击战,集合起来又可作运动战,八路军就是这样做的。八路军的方针是:“基本的是游击战,但不放松有利条件下的运动战。”这个方针是完全正确的,反对这个方针的人们的观点是不正确的。

(九六)防御的和攻击的阵地战,在中国今天的技术条件下,一般都不能执行,这也就是我们表现弱的地方。再则敌人又利用中国土地广大一点,回避我们的阵地设施。因此阵地战就不能用为重要手段,更不待说用为主要手段。然而在战争的第一第二两阶段中,包括于运动战范围,而在战役作战上起其辅助作用的局部的阵地战,是可能的和必要的。为着节节抵抗以求消耗敌人和争取余裕时间之目的,而采取半阵地性的所谓“运动性的防御”,更是属于运动战的必要部分。中国须努力增加新式武器,以便在战略反攻阶段中能够充分地执行阵地攻击的任务。战略反攻阶段,无疑地将提高阵地战的地位,因为那时敌人将坚守阵地,没有我之有力的阵地攻击以配合运动战,将不能达到收复失地之目的。虽然如此,第三阶段中,我们仍须力争以运动战为战争的主要形式。因为战争的领导艺术和人的活跃性,临到像第一次世界大战的中期以后西欧地区那样的阵地战,就死了一大半。然而在广大版图的中国境内作战,在相当长的时间内中国方面又还保存着技术贫弱这种情况,“把战争从壕沟里解放”的事,就自然出现。就在第三阶段,中国技术条件虽已增进,但仍不见得能够超过敌人,这样也就被逼着非努力讲求高度的运动战,不能达到最后胜利之目的。这样,整个抗日战争中,中国将不会以阵地战为主要形式,主要和重要的形式是运动战和游击战。在这些战争形式中,战争的领导艺术和人的活跃性能够得到充分地发挥的机会,这又是我们不幸中的幸事啊!

消耗战,歼灭战

(九七)前头说过,战争本质即战争目的,是保存自己,消灭敌人。然而达此目的的战争形式,有运动战、阵地战、游击战三种,实现时的效果就有程度的不同,因而一般地有所谓消耗战和歼灭战之别。

(九八)我们首先可以说,抗日战争是消耗战,同时又是歼灭战。为什么?敌之强的因素尚在发挥,战略上的优势和主动依然存在,没有战役和战斗的歼灭战,就不能有效地迅速地减杀其强的因素,破坏其优势和主动。我之弱的因素也依然存在,战略上的劣势和被动还未脱离,为了争取时间,加强国内国际条件,改变自己的不利状态,没有战役和战斗的歼灭战,也不能成功。因此,战役的歼灭战是达到战略的消耗战之目的的手段。在这点上说,歼灭战就是消耗战。中国之能够进行持久战,用歼灭达到消耗是主要的手段。(九九)但达到战略消耗目的的,还有战役的消耗战。大抵运动战是执行歼灭任务的,阵地战是执行消耗任务的,游击战是执行消耗任务同时又执行歼灭任务的,三者互有区别。在这点上说,歼灭战不同于消耗战。战役的消耗战,是辅助的,但也是持久作战所需要的。

(一○○)从理论上和需要上说来,中国在防御阶段中,应该利用运动战之主要的歼灭性,游击战之部分的歼灭性,加上辅助性质的阵地战之主要的消耗性和游击战之部分的消耗性,用以达到大量消耗敌人的战略目的。在相持阶段中,继续利用游击战和运动战的歼灭性和消耗性,再行大量地消耗敌人。所有这些,都是为了使战局持久,逐渐地转变敌我形势,准备反攻的条件。战略反攻时,继续用歼灭达到消耗,以便最后地驱逐敌人。

(一○一)但是在事实上,十个月的经验是,许多甚至多数的运动战战役,打成了消耗战;游击战之应有的歼灭作用,在某些地区,也还未提到应有的程度。这种情况的好处是,无论如何我们总算消耗了敌人,对于持久作战和最后胜利有其意义,我们的血不是白流的。然而缺点是:一则消耗敌人的不足;二则我们自己不免消耗的较多,缴获的较少。虽然应该承认这种情况的客观原因,即敌我技术和兵员教养程度的不同,然而在理论上和实际上,无论如何也应该提倡主力军在一切有利场合努力地执行歼灭战。游击队虽然为了执行许多具体任务,例如破坏和扰乱等,不能不进行单纯的消耗战,然而仍须提倡并努力实行在战役和战斗之一切有利场合的歼灭性的作战,以达既能大量消耗敌人又能大量补充自己之目的。

(一○二)外线的速决的进攻战之所谓外线,所谓速决,所谓进攻,与乎运动战之所谓运动,在战斗形式上,主要地就是采用包围和迂回战术,因而便须集中优势兵力。所以,集中兵力,采用包围迂回战术,是实施运动战即外线的速决的进攻战之必要条件。然而一切这些,都是为着歼灭敌人之目的。

(一○三)日本军队的长处,不但在其武器,还在其官兵的教养——其组织性,其因过去没有打过败仗而形成的自信心,其对天皇和对鬼神的迷信,其骄慢自尊,其对中国人的轻视等等特点;这是日本军阀多年的武断教育和日本的民族习惯造成的。我军对之杀伤甚多、俘虏甚少的现象,主要原因在此。这一点,过去许多人是估计不足的。这种东西的破坏,需要一个长的过程。首先需要我们重视这一特点,然后耐心地有计划地从政治上、国际宣传上、日本人民运动上多方面地向着这一点进行工作;而军事上的歼灭战,也是方法之一。在这里,悲观主义者可以据之引向亡国论,消极的军事家又可以据之反对歼灭战。我们则相反,我们认为日本军队的这种长处是可以破坏的,并且已在开始破坏中。破坏的方法,主要的是政治上的争取。对于日本士兵,不是侮辱其自尊心,而是了解和顺导他们的这种自尊心,从宽待俘虏的方法,引导他们了解日本统治者之反人民的侵略主义。另一方面,则是在他们面前表示中国军队和中国人民不可屈服的精神和英勇顽强的战斗力,这就是给以歼灭战的打击。在作战上讲,十个月的经验证明歼灭是可能的,平型关、台儿庄等战役就是明证。日本军心已在开始动摇,士兵不了解战争目的,陷于中国军队和中国人民的包围中,冲锋的勇气远弱于中国兵等等,都是有利于我之进行歼灭战的客观的条件,这些条件并将随着战争之持久而日益发展起来。在以歼灭战破坏敌军的气焰这一点上讲,歼灭又是缩短战争过程提早解放日本士兵和日本人民的条件之一。世界上只有猫和猫做朋友的事,而没有猫和老鼠做朋友的事。

(一○四)另一方面,应该承认在技术和兵员教养的程度上,现时我们不及敌人。因而最高限度的歼灭,例如全部或大部俘获的事,在许多场合特别是在平原地带的战斗中,是困难的。速胜论者在这点上面的过分要求,也属不对。抗日战争的正确要求应该是:尽可能的歼灭战。在一切有利的场合,每战集中优势兵力,采用包围迂回战术——不能包围其全部也包围其一部,不能俘获所包围之全部也俘获所包围之一部,不能俘获所包围之一部也大量杀伤所包围之一部。而在一切不利于执行歼灭战的场合,则执行消耗战。对于前者,用集中兵力的原则;对于后者,用分散兵力的原则。在战役的指挥关系上,对于前者,用集中指挥的原则;对于后者,用分散指挥的原则。这些,就是抗日战争战场作战的基本方针。

乘敌之隙的可能性

(一○五)关于敌之可胜,就是在敌人的指挥方面也有其基础。自古无不犯错误的将军,敌人之有岔子可寻,正如我们自己也难免出岔子,乘敌之隙的可能性是存在的。从战略和战役上说来,敌人在十个月侵略战争中,已经犯了许多错误。计其大者有五。

一是逐渐增加兵力。这是由于敌人对中国估计不足而来的,也有他自己兵力不足的原因。敌人一向看不起我们,东四省得了便宜之后,加之以冀东、察北的占领,这些都算作敌人的战略侦察。他们得来的结论是:一盘散沙。据此以为中国不值一打,而定出所谓“速决”的计划,少少出点兵力,企图吓溃我们。十个月来,中国这样大的团结和这样大的抵抗力,他们是没有料到的,他们把中国已处于进步时代,中国已存在着先进的党派、先进的军队和先进的人民这一点忘掉了。及至不行,就逐渐增兵,由十几个师团一次又一次地增至三十个。再要前进,非再增不可。但由于同苏联对立,又由于人财先天不足,所以日本的最大的出兵数和最后的进攻点都不得不受一定的限制。

二是没有主攻方向。台儿庄战役以前,敌在华中、华北大体上是平分兵力的,两方内部又各自平分。例如华北,在津浦、平汉、同蒲三路平分兵力,每路伤亡了一些,占领地驻守了一些,再前进就没有兵了。台儿庄败仗后,总结了教训,把主力集中徐州方向,这个错误算是暂时地改了一下。

三是没有战略协同。敌之华中、华北两集团中,每一集团内部是大体协同的,但两集团间则很不协同。津浦南段打小蚌埠时,北段不动;北段打台儿庄时,南段不动。两处都触了霉头之后,于是陆军大臣来巡视了,参谋总长来指挥了,算是暂时地协调了一下。日本地主资产阶级和军阀内部存在着颇为严重的矛盾,这种矛盾正在向前发展着,战争的不协同是其具体表现之一。

四是失去战略时机。这点显着地表现在南京、太原两地占领后的停顿,主要的是因为兵力不足,没有战略追击队。五是包围多歼灭少。台儿庄战役以前,上海、南京、沧州、保定、南口、忻口、临汾诸役,击破者多,俘获者少,表现其指挥的笨拙。这五个——逐渐增加兵力,没有主攻方向,没有战略协同,失去时机,包围多歼灭少,是台儿庄战役以前日本指挥的不行之点。台儿庄战役以后,虽已改了一些,然根据其兵力不足和内部矛盾诸因素,求不重犯错误是不可能的。且得之于此者,又失之于彼。例如,将华北兵力集中于徐州,华北占领地就出了大空隙,给予游击战争以放手发展的机会。

以上是敌人自己弄错,不是我们使之错的。我们方面,尚可有意地制造敌之错误,即用自己聪明而有效的动作,在有组织的民众掩护之下,造成敌人错觉,调动敌人就我范围,例如声东击西之类,这件事的可能性前面已经说过了。所有这些,都说明:我之战争胜利又可在敌之指挥上面找到某种根源。虽然我们不应把这点作为我之战略计划的重要基础,相反,我之计划宁可放在敌人少犯错误的假定上,才是可靠的做法。而且我乘敌隙,敌也可以乘我之隙,少授敌以可寻之隙,又是我们指挥方面的任务。然而敌之指挥错误,是事实上已经存在过,并且还要发生的,又可因我之努力制造出来的,都足供我之利用,抗日将军们应该极力地捉住它。敌人的战略战役指挥许多不行,但其战斗指挥,即部队战术和小兵团战术,却颇有高明之处,这一点我们应该向他学习。

抗日战争中的决战问题

(一○六)抗日战争中的决战问题应分为三类:一切有把握的战役和战斗应坚决地进行决战,一切无把握的战役和战斗应避免决战,赌国家命运的战略决战应根本避免。抗日战争不同于其它许多战争的特点,又表现在这个决战问题上。在第一第二阶段,敌强我弱,敌之要求在于我集中主力与之决战。我之要求则相反,在选择有利条件,集中优势兵力,与之作有把握的战役和战斗上的决战,例如平型关、台儿庄以及许多的其它战斗;而避免在不利条件下的无把握的决战,例如彰德等地战役所采的方针。拚国家命运的战略的决战则根本不干,例如最近之徐州撤退。这样就破坏了敌之“速决”计划,不得不跟了我们干持久战。这种方针,在领土狭小的国家是做不到的,在政治太落后了的国家也难做到。我们是大国,又处进步时代,这点是可以做到的。如果避免了战略的决战,“留得青山在,不愁没柴烧”,虽然丧失若干土地,还有广大的回旋余地,可以促进并等候国内的进步、国际的增援和敌人的内溃,这是抗日战争的上策。急性病的速胜论者熬不过持久战的艰难路程,企图速胜,一到形势稍为好转,就吹起了战略决战的声浪,如果照了干去,整个的抗战要吃大亏,持久战为之葬送,恰恰中了敌人的毒计,实在是下策。不决战就须放弃土地,这是没有疑问的,在无可避免的情况下(也仅仅是在这种情况下),只好勇敢地放弃。情况到了这种时候,丝毫也不应留恋,这是以土地换时间的正确的政策。历史上,俄国以避免决战,执行了勇敢的退却,战胜了威震一时的拿破仑\footnote{一八一二年,拿破仑以五十万大军进攻俄国。当时俄军只有二十万人左右。为了避免不利于自己的决战,俄军实行战略退却,一直到放弃和焚毁了莫斯科。拿破仑的军队在深入俄国国土以后,遭到了俄国广大军民的坚决反抗,陷于饥寒困苦、后路被切断、四面被包围的绝境,最后不得不从莫斯科撤退。这时,俄军乘机大举反攻,拿破仑军仅剩二万余人逃离俄国国境。}。中国现在也应这样干。

(一○七)不怕人家骂“不抵抗”吗?不怕的。根本不战,与敌妥协,这是不抵抗主义,不但应该骂,而且完全不许可的。坚决抗战,但为避开敌人毒计,不使我军主力丧于敌人一击之下,影响到抗战的继续,一句话,避免亡国,是完全必需的。在这上面发生怀疑,是战争问题上的近视眼,结果一定和亡国论者走到一伙去。我们曾经批评了所谓“有进无退”的拚命主义,就是因为这种拚命主义如果成为一般的风气,其结果就有使抗战不能继续,最后引向亡国的危险。

(一○八)我们主张一切有利条件下的决战,不论是战斗的和大小战役的,在这上面不容许任何的消极。给敌以歼灭和给敌以消耗,只有这种决战才能达到目的,每个抗日军人均须坚决地去做。为此目的,部分的相当大量的牺牲是必要的,避免任何牺牲的观点是懦夫和恐日病患者的观点,必须给以坚决的反对。李服膺、韩复榘等逃跑主义者的被杀,是杀得对的。在战争中提倡勇敢牺牲英勇向前的精神和动作,是在正确的作战计划下绝对必要的东西,是同持久战和最后胜利不能分离的。我们曾经严厉地指斥了所谓“有退无进”的逃跑主义,拥护严格纪律的执行,就是因为只有这种在正确计划下的英勇决战,才能战胜强敌;而逃跑主义,则是亡国论的直接支持者。

(一○九)英勇战斗于前,又放弃土地于后,不是自相矛盾吗?这些英勇战斗者的血,不是白流了吗?这是非常不妥当的发问。吃饭于前,又拉屎于后,不是白吃了吗?睡觉于前,又起床于后,不是白睡了吗?可不可以这样提出问题呢?我想是不可以的。吃饭就一直吃下去,睡觉就一直睡下去,英勇战斗就一直打到鸭绿江,这是主观主义和形式主义的幻想,在实际生活里是不存在的。谁人不知,为争取时间和准备反攻而流血战斗,某些土地虽仍不免于放弃,时间却争取了,给敌以歼灭和给敌以消耗的目的却达到了,自己的战斗经验却取得了,没有起来的人民却起来了,国际地位却增长了。这种血是白流的吗?一点也不是白流的。放弃土地是为了保存军力,也正是为了保存土地;因为如不在不利条件下放弃部分的土地,盲目地举行绝无把握的决战,结果丧失军力之后,必随之以丧失全部的土地,更说不到什么恢复失地了。资本家做生意要有本钱,全部破产之后,就不算什么资本家。赌汉也要赌本,孤注一掷,不幸不中,就无从再赌。事物是往返曲折的,不是径情直遂的,战争也是一样,只有形式主义者想不通这个道理。

(一一○)我想,即在战略反攻阶段的决战亦然。那时虽然敌处劣势,我处优势,然而仍适用“执行有利决战,避免不利决战”的原则,直至打到鸭绿江边,都是如此。这样,我可始终立于主动,一切敌人的“挑战书”,旁人的“激将法”,都应束之高阁,置之不理,丝毫也不为其所动。抗日将军们要有这样的坚定性,才算是勇敢而明智的将军。那些“一触即跳”的人们,是不足以语此的。第一阶段我处于某种程度的战略被动,然在一切战役上也应是主动的,尔后任何阶段都应是主动。我们是持久论和最后胜利论者,不是赌汉们那样的孤注一掷论者。

兵民是胜利之本

(一一一)日本帝国主义处在革命的中国面前,是决不放松其进攻和镇压的,它的帝国主义本质规定了这一点。中国不抵抗,日本就不费一弹安然占领中国,东四省的丧失,就是前例。中国若抵抗,日本就向着这种抵抗力压迫,直至它的压力无法超过中国的抵抗力才停止,这是必然的规律。日本地主资产阶级的野心是很大的,为了南攻南洋群岛,北攻西伯利亚起见,采取中间突破的方针,先打中国。那些认为日本将在占领华北、江浙一带以后适可而止的人,完全没有看到发展到了新阶段迫近了死亡界线的日本帝国主义,已经和历史上的日本不相同了。我们说,日本的出兵数和进攻点有一定的限制,是说:在日本一方面,在其力量基础上,为了还要举行别方面的进攻并防御另一方面的敌人,只能拿出一定程度的力量打中国打到它力所能及的限度为止;在中国一方面,又表现了自己的进步和顽强的抵抗力,不能设想只有日本猛攻,中国没有必要的抵抗力。日本不能占领全中国,然而在它一切力所能及的地区,它将不遗余力地镇压中国的反抗,直至日本的内外条件使日本帝国主义发生了进入坟墓的直接危机之前,它是不会停止这种镇压的。日本国内的政治只有两个出路:或者整个当权阶级迅速崩溃,政权交给人民,战争因而结束,但暂时无此可能;或者地主资产阶级日益法西斯化,把战争支持到自己崩溃的一天,日本走的正是这条路。除此没有第三条路。那些希望日本资产阶级中和派出来停止战争的,仅仅是一种幻想而已。日本的资产阶级中和派,已经作了地主和金融寡头的俘虏,这是多年来日本政治的实际。日本打了中国之后,如果中国的抗战还没有给日本以致命的打击,日本还有足够力量的话,它一定还要打南洋或西伯利亚,甚或两处都打。欧洲战争一起来,它就会干这一手;日本统治者的如意算盘是打得非常之大的。当然存在这种可能:由于苏联的强大,由于日本在中国战争中的大大削弱,它不得不停止进攻西伯利亚的原来计划,而对之采取根本的守势。然而在出现了这种情形之时,不是日本进攻中国的放松,反而是它进攻中国的加紧,因为那时它只剩下了向弱者吞剥的一条路。那时中国的坚持抗战、坚持统一战线和坚持持久战的任务,就更加显得严重,更加不能丝毫懈气。

(一一二)在这种情况下,中国制胜日本的主要条件,是全国的团结和各方面较之过去有十百倍的进步。中国已处于进步的时代,并已有了伟大的团结,但是目前的程度还非常之不够。日本占地如此之广,一方面由于日本之强,一方面则由于中国之弱;而这种弱,完全是百年来尤其是近十年来各种历史错误积累下来的结果,使得中国的进步因素限制在今天的状态。现在要战胜这样一个强敌,非有长期的广大的努力是不可能的。应该努力的事情很多,我这里只说最根本的两方面:军队和人民的进步。

(一一三)革新军制离不了现代化,把技术条件增强起来,没有这一点,是不能把敌人赶过鸭绿江的。军队的使用需要进步的灵活的战略战术,没有这一点,也是不能胜利的。然而军队的基础在士兵,没有进步的政治精神贯注于军队之中,没有进步的政治工作去执行这种贯注,就不能达到真正的官长和士兵的一致,就不能激发官兵最大限度的抗战热忱,一切技术和战术就不能得着最好的基础去发挥它们应有的效力。我们说日本技术条件虽优,但它终必失败,除了我们给以歼灭和消耗的打击外,就是它的军心终必随着我们的打击而动摇,武器和兵员结合不稳。我们相反,抗日战争的政治目的是官兵一致的。在这上面,就有了一切抗日军队的政治工作的基础。军队应实行一定限度的民主化,主要地是废除封建主义的打骂制度和官兵生活同甘苦。这样一来,官兵一致的目的就达到了,军队就增加了绝大的战斗力,长期的残酷的战争就不患不能支持。

(一一四)战争的伟力之最深厚的根源,存在于民众之中。日本敢于欺负我们,主要的原因在于中国民众的无组织状态。克服了这一缺点,就把日本侵略者置于我们数万万站起来了的人民之前,使它像一匹野牛冲入火阵,我们一声唤也要把它吓一大跳,这匹野牛就非烧死不可。我们方面,军队须有源源不绝的补充,现在下面胡干的“捉兵法”、“买兵法”\footnote{国民党政府扩军的一种办法,是派军警四处捉拿人民去当兵,捉来的兵用绳捆索绑,形同囚犯。略为有钱的人,就向国民党政府的官吏行贿,出钱买人代替。},亟须禁止,改为广泛的热烈的政治动员,这样,要几百万人当兵都是容易的。抗日的财源十分困难,动员了民众,则财政也不成问题,岂有如此广土众民的国家而患财穷之理?军队须和民众打成一片,使军队在民众眼睛中看成是自己的军队,这个军队便无敌于天下,个把日本帝国主义是不够打的。

(一一五)很多人对于官兵关系、军民关系弄不好,以为是方法不对,我总告诉他们是根本态度(或根本宗旨)问题,这态度就是尊重士兵和尊重人民。从这态度出发,于是有各种的政策、方法、方式。离了这态度,政策、方法、方式也一定是错的,官兵之间、军民之间的关系便决然弄不好。军队政治工作的三大原则:第一是官兵一致,第二是军民一致,第三是瓦解敌军。这些原则要实行有效,都须从尊重士兵、尊重人民和尊重已经放下武器的敌军俘虏的人格这种根本态度出发。那些认为不是根本态度问题而是技术问题的人,实在是想错了,应该加以改正才对。

(一一六)当此保卫武汉等地成为紧急任务之时,发动全军全民的全部积极性来支持战争,是十分严重的任务。保卫武汉等地的任务,毫无疑义必须认真地提出和执行。然而究竟能否确定地保卫不失,不决定于主观的愿望,而决定于具体的条件。政治上动员全军全民起来奋斗,是最重要的具体的条件之一。不努力于争取一切必要的条件,甚至必要条件有一不备,势必重蹈南京等地失陷之覆辙。中国的马德里在什么地方,看什么地方具备马德里的条件。过去是没有过一个马德里的,今后应该争取几个,然而全看条件如何。条件中的最基本条件,是全军全民的广大的政治动员。

(一一七)在一切工作中,应该坚持抗日民族统一战线的总方针。因为只有这种方针才能坚持抗战,坚持持久战,才能普遍地深入地改善官兵关系、军民关系,才能发动全军全民的全部积极性,为保卫一切未失地区、恢复一切已失地区而战,才能争取最后胜利。

(一一八)这个政治上动员军民的问题,实在太重要了。我们之所以不惜反反复复地说到这一点,实在是没有这一点就没有胜利。没有许多别的必要的东西固然也没有胜利,然而这是胜利的最基本的条件。抗日民族统一战线是全军全民的统一战线,决不仅仅是几个党派的党部和党员们的统一战线;动员全军全民参加统一战线,才是发起抗日民族统一战线的根本目的。

结  论

(一一九)结论是什么呢?结论就是:“在什么条件下,中国能战胜并消灭日本帝国主义的实力呢?要有三个条件:第一是中国抗日统一战线的完成;第二是国际抗日统一战线的完成;第三是日本国内人民和日本殖民地人民的革命运动的兴起。就中国人民的立场来说,三个条件中,中国人民的大联合是主要的。”“这个战争要延长多久呢?要看中国抗日统一战线的实力和中日两国其它许多决定的因素如何而定。”“如果这些条件不能很快实现,战争就要延长。但结果还是一样,日本必败,中国必胜。只是牺牲会大,要经过一个很痛苦的时期。”“我们的战略方针,应该是使用我们的主力在很长的变动不定的战线上作战。中国军队要胜利,必须在广阔的战场上进行高度的运动战。”“除了调动有训练的军队进行运动战之外,还要在农民中组织很多的游击队。”“在战争的过程中……使中国军队的装备逐渐加强起来。因此,中国能够在战争的后期从事阵地战,对于日本的占领地进行阵地的攻击。这样,日本在中国抗战的长期消耗下,它的经济行将崩溃;在无数战争的消磨中,它的士气行将颓靡。中国方面,则抗战的潜伏力一天一天地奔腾高涨,大批的革命民众不断地倾注到前线去,为自由而战争。所有这些因素和其它的因素配合起来,就使我们能够对日本占领地的堡垒和根据地,作最后的致命的攻击,驱逐日本侵略军出中国。”(一九三六年七月与斯诺谈话)“中国的政治形势从此开始了一个新阶段,……这一阶段的最中心的任务是:动员一切力量争取抗战的胜利。”“争取抗战胜利的中心关键,在使已经发动的抗战发展为全面的全民族的抗战。只有这种全面的全民族的抗战,才能使抗战得到最后的胜利。”“由于当前的抗战还存在着严重的弱点,所以在今后的抗战过程中,可能发生许多挫败、退却,内部的分化、叛变,暂时和局部的妥协等不利的情况。因此,应该看到这一抗战是艰苦的持久战。但我们相信,已经发动的抗战,必将因为我党和全国人民的努力,冲破一切障碍物而继续地前进和发展。”(一九三七年八月《中共中央关于目前形势与党的任务的决定》)这些就是结论。亡国论者看敌人如神物,看自己如草芥,速胜论者看敌人如草芥,看自己如神物,这些都是错误的。我们的意见相反:抗日战争是持久战,最后胜利是中国的——这就是我们的结论。

(一二○)我的讲演至此为止。伟大的抗日战争正在开展,很多人希望总结经验,以便争取全部的胜利。我所说的,只是十个月经验中的一般的东西,也算一个总结吧。这个问题值得引起广大的注意和讨论,我所说的只是一个概论,希望诸位研究讨论,给以指正和补充。

\section{抗日民族战争与抗日民族统一战线发展的新阶段 1938/10}

按:一九三八年十月十二日至十四日在中共扩大的六中全会的报告。

同志们,我代表中央政治局向扩大的六中全会作报告,我准备说些什么呢?我要说的分为下述几部分:(一)五中全会到六中全会;(二)抗战十五个月的总结;(三)抗日民族战争与抗日民族统一战线发展的新阶段;(四)全民族的当前紧急任务;(五)长期战争与长期合作;(六)中国反侵略战争与世界反法西斯运动;(七)中国共产党在民族战争中的地位;(八)党的七次全国代表大会。我要说的就是这些问题。

同志们,在全国炮火连天全世界战争危机紧迫的环境中来开我们的六中全会扩大会,我们要做些什么工作呢?我们的目的何在呢?我们一定要同全中国一切爱国党派一切爱国同胞永远的团结起来,克服新的困难,动员新的力量,在目前,停止敌之进攻,在将来,实行我之反攻,达到驱逐日本帝国主义建立三民主义共和国之目的,我们一定要自由,我们一定要胜利——这就是我们的目的,也就是我的报告的总方向。

一、五中全会到六中全会

(1)扩大的六中全会之召集

我们党的中央全体会议,自从一九三四年一月在江西开过第五次中央全会以来,差不多五个年头了。因为各中央委员分散工作于国内外各种不同的环境,使我们不能聚集在一块。此次则除了几个同志之外最大多数的中央委员都到了,而且到了全国各地许多领导工作的同志,使我们的这次中央全会成为第六次全国代表大会以来人数到得最多的一次。本来第七次全国代表大会准备在本年召集的,因为战争紧张的原故,不得不把七大推迟到明年,而当前时局向我们提出了许多的问题,必须作明确的解决,以便争取抗战的胜利,所以召集了这次扩大的中央全会。

(2)五中全会到六中全会

五年以来,我们党经历了许多重大的事变。最大与最主要的是:由国内党各派各阶级互相对立的局面转到了抗日民族统一战线,由国内战争转到了抗日战争。

过去国内战争形成的原因,在于一九二七年不幸破裂了国共两党的统一战线,这是由于当时的历史环境造成的。

抗日民族统一战线的政策,又是怎样形成的呢?乃是由于新的历史环境。大家都已非常明白,自从九一八事变以来,中华民族的敌人日本帝国主义,经过侵略东四省的第一步,进到准备并实行向全中国侵略的第二步骤。这种空前的历史事变,使得国内国际情况都发生了变化。首先变化了与变化着国内各个阶层,各个党派,各个集团之间的相互关系,同时也变化了与变化着国际间的相互关系。因而我们的党,根据这种空前的历史事变,根据新的国内国际关系,沿着远在一九三三年就已开始采取了的新的政治立场(在三个条件下与国民党内任何愿意同我们合作的成分订立抗日作战协议)的道路,把它提高到抗日民族统一战线的新政策,因而发表了一九三五年八月的宣言,十二月的决议,一九三六年八月的致国民党书,九月的民主共和国决议,并且根据了这些,使得我们能够在当年十二月间发生的西安事变,坚持了和平解决的方针,并于一九三七年二月送致国民党三中全会一个团结抗日的具体建议。去年五月,召集了一次临时性的代表大会(名曰苏区代表大会,有当时苏区非苏区及红军代表参加),通过了“抗日民族统一战线在目前阶段的任务”,批准了红军实行改编为国民革命军,苏维埃实行改为民主制。这样,就在事实上由国内战争的状态转到了开始建立抗日民族统一战线的新时期。当时,中国国民党也逐渐改变了它的政策,逐渐转到了团结抗日的立场。假如没有国民党政策的转变,要建立抗日民族统一战线是不可能的。那时,救国团体在许多地方有了组织,其它党派亦有了抗日要求。由于国共两党双方政策的转变,由于蒋介石先生的领导,由于全国军民的拥护,由于其它集团与其它党派的协力,就使得日本帝国主义灭亡全中国的侵略步骤,遭遇了前所未有的全民族的反抗。去年七月七日芦沟桥事变发生之后,全中国就在民族领袖与最高统帅蒋委员长的统一领导之下,发出了神圣的正义的炮声,全中国形成了一个空前的抗日大团结,形成了伟大的抗日民族统一战线。芦沟桥事变后的第二月,即去年八月,我们党发布了一个抗日救国的十大纲领。同时,八路军改编完成,开赴华北作战。九月二十二日,我们党发表了公开宣布以三民主义为基础与国民党精诚团结共赴国难的宣言。第二日,国民党、国民政府与国民革命军的最高领袖蒋介石先生发表了承认共产党合法存在并与之团结救国的谈话。从此以后,以国共两党为基础的抗日民族统一战线,就完全建立起来了。十二月,为着巩固与发展抗日民族统一战线,我们党又发表了愿与国民党不但合作抗日而且合作建国的宣言。此时未久,南方的红军游击队改编为国民革命军新编第四军,开赴江南作战。从此以后,抗日团结便日益进步了。

同志们,这种由两党十年战争转到两党重新合作,并在极端困难的条件之下执行了这个转变,奠定了两党长期合作的始基,是经过了许多的艰难曲折才完成的。然而由于中央与全党的努力,总算是完成了。共产国际完全同意我们党的这个新的政治路线(共产国际决议,九月八日“新华日报”)。并为了中华民族的胜利,号召全世界各国共产党与无产阶级援助中国的抗日战争。

同志们,假如没有国共两党为基础的抗日民族统一战线的发起,建立与坚持,如此伟大的抗日民族革命战争之发动,持久与争取胜利,是不可能的。现在全中国全世界的人都明白:中华民族是站起来了!一百年来受人欺凌,侮辱,侵略,压迫,特别是九一八事变以来那种难堪的奴隶地位,是改变过来了。全中国人手执武器走上了民族自卫战争的战场,全中国的最后胜利,即中华民族自由解放的曙光,己经发现了。

我们知道,我们今天的这一伟大的民族战争,和中国过去一切历史时期的战争都不相同。因为这个战争是为了把中华民族从半殖民地状态中从亡国灭种危险中解放出来的战争,而且这个战争是在中华民族历史上最进步的时期进行的﹔同时,又是在我们的敌人日本帝国主义自寻死路走向崩溃的时期进行的﹔同时,又是在全世界先进人类正准备着空前广大与空前深刻的斗争力量以便抵抗与战胜德日意法西斯魔王争取世界光明前途的时期进行的。这样三方面因素的结合——以中国进步并且继续进步为主要基础的三方面的结合,就保证了我们的抗日战争一定能够最后取得胜利,而自由解放的新中国一定要出现于东亚,并成为未来光明世界中一个极重要的组成部分。这样的一个中国,不但将造福于四万万五千万中国人,而且将造福于全人类。

(3)六中全会的任务

这次扩大的六中全会,是处于抗日战争将要进入一个新的发展阶段的重要关头开会的,扩大的六中全会担负了重大的历史任务。

完全不错。抗日战争英勇奋斗了一年多,全国有了伟大的团结与伟大的进步,给了日本帝国主义以严重的打击,虽然失地很多,但同时就有了很多胜利,这是无可否认的事实。战争发展下去,主要的由于中国继续进步,同时配合着日本增加困难,国际助我增强,最后胜利定属于我,不属于敌,这也是可以预断的。谁要是看不见过去的伟大成绩与未来的胜利前途,谁就要陷入悲观主义的深坑而不能自拔。然而单看到这一方面,是不够的,抗日战争还有另一方面,还有它的消极方面,这就是我们面前摆着的许多困难。目前的情况告诉我们,一年多以来中国所有的奋斗,团结,进步,胜利,还未能阻止敌人的前进,还没有反攻敌人的力量。武汉现正处于敌人的威胁中,敌人还要向广州、长沙及西北等地进攻。因此全国人民都盼望共产党发表意见,新的环境提出了许多问题。同志们,我们必须发表意见,必须解决问题。对的,我们党早已发表了意见,许多根本问题也早已解决了。但新的环境要求我们发布新的意见,解决新的问题。

什么是新的问题呢?

如何在现有基础上增加新的力量,渡过战争难关,停止敌之进攻,准备我之反攻,达到驱逐敌人之目的,这就是当前问题的关键。这个问题在全国无数人们中议论著,焦思着。我们应不应该回答这个问题呢?无疑是应该回答的。

这个问题展开于各方面,发生了许多的问题。

例如,十五个月抗战的经验,究竟证明了什么呢?十五个月经验证明抗战是长期的,还是短期的呢?战略方针是持久胜敌,还是速战胜敌呢?最后胜利是中国的,还是敌人的?抗战有出路,还是妥协有出路呢?如果战争是长期的,又用什么方法去支持长期战争与取得最后胜利呢?所有这些是否能在十五个月经验中找到根据给以明确的回答?并是否可以依据这些过去基础而在抗战的新阶段中起其积极作用,藉以克服新的困难争取新的胜利呢?这些都是重要的问题,这是一类的问题。

又如,整个抗日战争将怎样发展变化呢?所谓新阶段,究将是一种什么性质的阶段呢?假定武汉不守,战争趋势将怎么样呢?今后全国努力的方向,即全中华民族的当前紧急任务,应是什么呢?有些什么好办法足以渡过战争的难关呢?这些更是重要的问题,这又是一类问题。

又如,国共合作的前途与远景将会怎样呢?共产党有何根据来说长期合作呢?共产党有何办法来改善两党之间的关系呢?所谓不但合作抗战而且合作建国,究将建立一个什么国呢?三民主义与共产主义的关系怎么样呢?这也是很重要的问题,这又是一类问题。

又如,世界风云如此紧急,其趋势将怎样呢?中国抗日战争与世界反法西斯运动有何利害关系呢?这也是重要的问题,这又是一类问题。

还有,中国共产党在民族战争中的地位如何呢?共产党员为着其党的政治方针而奋斗时,其工作态度应该怎样呢?共产党有什么更好的方法同他党合作,同人民联系,足使艰难的时局走向顺利呢?共产党的内部关系怎样?有些什么好方法团结全党使之在抗日战争中尤其在当前艰难的时局中起其大的作用呢?共产党的七次代表大会究将怎样呢?这也是重要的问题,这又是一类问题。

所有这些问题,都是党内党外迫切要求解决的。近几个月来,我们经常遇到要求回答这些问题的人们。

同志们,我们的国家是一个广大而复杂的国家,而这个国家现正处于同一个强的帝国主义作决死的斗争,这个斗争现在已接近了一个新的发展阶段,目前正处于向新阶段发展的过渡期间。我们的扩大的六中全会是在这个时候开会的,扩大的六中全会的责任非常重大,我们要解决许多的问题。

二、抗战十五个月的总结

(1)十五个月经验证明了什么?

让我们从十五个月的经验说起罢。

十五个月抗战的经验给了我们以什么呢?我以为主要的有三方面。第一,证明了抗日战争是长期的不是短期的,因而抗战的战略方针是持久战而不是速决战。第二,证明了中国的抗战能够取得最后胜利,悲观论者之没有根据。第三,证明了支持长期战争与取得最后胜利之唯一正确的道路,在于统一团结全民族,力求进步与依靠民众,藉以克服困难,争取胜利。而不是其它。

(2)抗日战争是长期的不是短期的,战略方针是持久战不是速决战

抗战初起之时,许多人不从敌我力量基本上的对比出发,而从若干一时的与表面的现象出发,设想战争不久就可解决,速胜思想笼罩一时。然而蒋委员长在去年双十节即明白指出:“此次抗战,非一年半载可了,必经非常之困苦与艰难始可获得最后之胜利。”我们则在很早的时候就指出了抗日战争的长期性,决不是一个短时间可以解决的。“战争的结果,日本必败,中国必胜,只是牺牲要大,要经过一个很痛苦的时期。”(一九三六年七月十六日毛泽东与斯诺谈话)“应该看到这一抗战是艰苦的持久战。”(一九三七年八月十五日中共中央关于目前形势与党的任务的决定)所有这些都指出:抗日战争是长期的不是短期的,战略方针是持久战不是速决战,抗战十五个月的经验完全证明其正确。

理由何在?在于敌强我弱,敌是优势,我是劣势,敌是帝国主义国家,我是半殖民地国家。

我们很早就指出过,战胜日本帝国主义要有三个条件:第一是中国的进步,这是基本的,主要的;第二是日本的困难;第三是国际的援助。让我们来看一看这些条件在抗战十五个月中已经怎样了?一句话回答:已经有了一个基础,但距必要的程度还很远。

拿第一个条件(战胜敌人基本的主要的条件),中国的进步来说,十五个月来确已有了一个基础,似惟有继续进步,才能最后胜敌。所谓中国的进步,包括国内政治、军事、党务、民运、文化教育等一切方面。这些方面的进步,十五个月来是非常显著的。然而单拿这些已有的东西,还不能停止敌之进攻,实行我之反攻。反攻必须有一个准备时期,必须经过全民族的努力,以便我们民族中一切生动力量有了一个广大的与深刻的发动,才有反攻胜敌之可能。因此速胜论是没有根据的,他忘记了敌强我弱这一特点,忘记了敌是优势,我是劣势,敌是帝国主义国家,我是半殖民地国家。中国具有很大的潜势力,发动起来足以使自已转败为胜,转弱为强,根本变化敌我形势,然而还待今后的努力,不是现成的事实。

拿第二条件,日本困难来说,也是这样。十五个月中,敌人出兵百万,伤亡数十万,用费数十万万,军队锐气日减,财政经济日形竭蹶。国际舆论纷起谴责,这些都是日本的野蛮侵略与中国的英勇抗战造成的结果。然而敌人这些已存的困难,还不足以停止他的进攻及利于我们的反攻,还须待到敌有更大的困难我有更大的进步之时,才是反攻胜敌之机会。因此,速胜论在敌情方面也没有根据,十五个月经验已经证明了。

拿第三条件,国际助我一点来说,现在也还未至最大有利之时。十五个月来,我们有了国际间广大的舆论声援,苏联和其它民主国家根据国联决议己经给了我们许多帮助,证明了我们不是孤立的。然而我们必须看到国际和平阵线各国有其各不相同的情况。资本主义国家,人民助我,政府则取某种程度的中立态度,其资产阶级则利用战争做生意,还在大量输送军火与军火原料给日本。社会主义国家,根本上不同于资本主义国家,在援华问题上已经具体的表现出来;然而国际形势目前还不容许它作超过现时程度的援助。因此,我们对国际援助暂时决不应作过大希望。抛开自力更生的方针,而主要地寄其希望于外援,无疑是十分错误的。十五个月经验证明:只有主要依靠自力更生,同时不放松外援之争取,才是正确的道路。在这点上,过去经验也否定了速胜论。

总起来说,不论中国方面,敌人方面,国际方面,十五个月经验,都证明速胜论主张之毫无根据,相反,显露了战争的长期性与残酷性。因此,我们的战略方针,决不能是速决战,而应该是持久战。持久胜敌——这就是抗日战争的唯一正确方针。过去不相信这种方针的,现在事实给了明白的教训,应该再没有疑问了!

这就是十五个月抗战的第一个总结。

(3)最后胜利是中国的,悲观论者毫无根据

抗战以前,唯武器论大张旗鼓,认为中国武器不如人,战必亡,中国必会作亚比西尼亚。抗战以后,这种议论表面没有了,但暗中流行着,抗战每至一紧张关头,这种议论必兴风作浪一次。认为中国应该停战议和,不堪再战,再战必亡。我们则相反,我们认为中国武器诚不如人,但武器是可以用人的努力增强的,战争胜负主要决定于人而不决定于物。持久抗战的结果,依据于全民族的努力,中国必能逐渐克服自己的弱点,增加自已的力量,化被动为主动,化劣势为优势;同时敌人方面的困难必逐渐增加,国际方面对我之援助必逐渐增大。综合这些因素,最后必能战胜日本帝国主义。蒋委员长早已明白宣示:“战事既起,惟有拼全民族之生命,牺牲到底,再无中途停顿妥协之理。”(去年七月庐山谈话)“此次抗战为国民革命过程中所必经,为被侵略民族对侵略者争取独立生存之战争,与通常交战国势均力敌者大异其趣。故凭借不在武器与军备而在坚强不屈之革命精神与坚强不拔之民族意识。”(去年十二月告国民书)中国共产党亦早已指出:“日本在中国抗战的长期消耗下,它的经济行将崩溃,它的士气行将颓靡,中国方面则抗战的潜伏力一天一天的奔腾高涨。所有这些因素和其它因素配合起来,就使我们能够对日本占领地作最后的攻击,驱逐日本的侵略军出中国。”(毛泽东与斯诺谈话)“我们相信,已经发动的抗战必将因为全国人民的努力冲破一切障碍物而继续的前进与发展。只要真的组织千百万群众进入抗日民族统一战线,抗日战争的胜利是无疑的。”(中共去年八月决定)所有这些,都被十五个月的经验证明了。悲观论或亡国论者认为敌人强不可抗,中国不堪一战,妥协才是出路等等荒谬说法,已经证明是完全错的了。

理由何在?在于敌强我弱仅是一方面的事实,敌人尚有弱点存在,中国尚有优点存在。

什么是敌人的弱点呢?第一,他是比较小的国家,他的兵力、财力不足,经不起长期的消耗。由于他的兵力不足,在中国的坚强抵抗面前又不得不分散与消耗,使他无法占领全中国。即在其占领地区,亦实际只能占领大城市,大道与某些平原地带,其它仍然是中国的。第二,敌人战争的性质是帝国主义的,是退步的。他的内部矛盾迫使他举行侵咯战争,并迫使他采用异常野蛮的掠夺政策。这样,就使他的战争,一方面变为同整个中华民族绝对对立的战争,迫着中国无论什么阶层,无论什么党派,都不能不团结起来坚决抗战。另一方面,变为同他本国人民大众绝对对立的战争,日本帝国主义悉率人财以应战争的结果,已经在他国内人民与前线士兵中间逐渐酝酿了许多的不满,战争发展下去,无疑有迫使他的人民与士兵大众走上用坚决的方法反对战争本身的趋势。这些都在十五个月中已经开始证明了的。这一点,就是存在于敌人自己方面而使敌人必归失败的最主要的根据。第三,正是由于敌人战争的性质是帝国主义的,换句话说,损人利己的,就不得不把他自己同一切和他利害相反的国家处于对立地位。除了两三个法西斯国家之外,一切国家,尤其是这些国家的人民大众,都不赞成日本的侵略战争。这样,就使得日本不得不日益缩小其国际活动的范围,日益处于孤立地位。这也是十五个月中已经开始证明了的。

这样,日本国度比较的小,影响到他兵力、财力的不足﹔日本战争的退步性﹔日本国际地位的孤立﹔这三者同时结合在一块,成为日本战争中存在着的先天性的弱点与困难,而这些弱点与困难又正在日益发展之中。对于这些,亡国论者与悲观主义者是瞎子,他们全没有看见,而仅仅看见了敌强我弱这一点。所以亡国论与悲观主义在敌情方面并没有根据,因而他们的妥协政策只能是亡国政策。我们是最后胜利论者,我们的观点则在敌情方面有充足的根据,十五个月经验已经开始证明了。

什么是我们的优点呢?第一,我们是大国,地大,物博,人多,兵多。不管敌人占去了我们主要的大城市与交通线,然而我们还有大块土地作为我们长期抗战与争取最后胜利的根据地。即在敌占地区我们也还有许多游击战争根据地。这个特点,是和小国如捷克、比利时等根本不同的。这是我们的第一个优点。第二,我们今天的抗日战争不同于中国一切历史时期的战争,我们的战争是民族革命战争,是进步的战争。不但战争本身的性质是进步的,而且这个战争是在中国前所未有的进步基础之上进行的。二十世纪四十年代的中国,不同于一切历史时期的中国,我们有了比之任何历史时期都不相同的进步的人民,进步的政党与进步的军队。在这个基础之上进行的抗日民族革命战争,它本身包含着可能继续发展进步的伟大力量。这一点,就是存在于我们方面而使我们足以支持长期战争取得最后胜利的主要的根据,十五个月的经验证明,在原有进步基础上进行着的伟大神圣的民族革命战争,已经推动了全中国的进步,旧的民族腐败传统是在破坏着,新的民族进步力量是在生长着,一个全民族统一团结进步发展的伟大过程是在向前完成着。抗战以前的中国不同于抗战以后的中国,这是有眼睛的人都能看到的了。而抗战第一阶段(这个阶段目前尚未完结)的中国又将不同于抗战尔后阶段的中国,也已经可以预断。还有,第三,我们的抗日战争不是孤立的。不管资本主义国家现时还保存其许多矛盾政策,也不管国际局势可能暂时地影响到各国助我的程度,但中国抗日战争与世界反侵略反法西斯的斗争,是不可分离地结合着。反对日本侵略战争的不仅是中国人,还有欧洲人,美洲人,非洲人,澳洲人,以及其它亚洲人。十五个月来世界各国的同情与援助,给了我们以这种确信。主要地依靠自力更生的中国,能够同时配合着世界的援助,因为今天的世界已不是从前的世界,整个世界先进人类已成为休戚相关的一体,敌人要使我们陷于孤立的企图,只会是徒然的。

这样,我们是一个很大的国家,我们的战争是进步的战争,我们又有国际的援助,这三者同时结合在一块。这些都是我们的有利条件,不但已经存在,并在日益发展之中。在这里,亡国论者与悲观主义者同样是瞎子,他们一点也看不见,而只看见我们是弱国,是劣势,是半殖民地这一点,喃喃发出其“抗战必亡”、“再战必亡”的胡说,其中许多坏蛋就根据这种胡说暗地进行其投降妥协的阴谋。我们相反,我们要根据十五个月经验中已经证明了的东西,向全党全国明确地指出我们国家与我们战争的长处与短处,有利条件与不利条件,并指出长处与有利条件在全战争中占居着优势,号召全国努力奋斗,发挥自已的长处,增强自己的有利条件,克服自己的短处与不利条件,为争取最后胜利而斗争。最后胜利将是谁的呢?我们确定地答复:中国的。在这个基础上决定我们的政策:坚决抗战还是动摇妥协呢?我们确定地答复:决不能有任何的动摇妥协,只有坚决抗战才是出路。东四省之沦亡,奥国之灭亡,捷国之瓜分,都并非因为抗战,这是有目共见的。现在还是一样,在中国的许多优良条件下,抗战必兴,但如走妥协道路,则灭亡无可避免。因此坚决反对妥协论,反对悲观主义,唤起全民奋战到底,乃是唯一无二的方针。

总起来说,敌强我弱这个矛盾着的对比,决定了战争的长期性,决定了持久战的战略方针,我们是持久胜敌论者,不是速胜论者。敌小,我大;敌人战争是退步的,我们战争是进步的:敌之国际地位比较孤立,我则比较能得外援:这几个矛盾着的对比,又决定了战争的最后胜利定属于我,不属于敌。这就是十五个月抗战经验的第二个总结。

(4)支持长期战争与争取最后胜利的唯一道路,在于统一团结全民族,力求进步,依靠民众

抗日战争是长期的,最后胜利是中国的,这两个基本问题已从十五个月抗战经验中证明了。但支持长期战争与争取最后胜利的具体方案如何?则过去国人的意见是不一致的。许多人认为照老样下去就可以了,他们不注意团结全国,不注意军事、政治、文化、党务、民运等各方面的改进,甚至加重磨擦,阻碍进步。我们则从来不赞成这种意见,认为唯有全民族的统一团结,力求进步,依靠民众,才能支持长期战争与取得最后胜利,否则是不可能的。中国国民党在其抗战建国纲领中明白指出:“欲求抗战必胜建国必成,固有赖于本党同志之努力,尤须全国人民戮力同心共同负担。”中国共产党亦早已指出了:“抗战时期最中心的任务,是动员一切力量,争取抗战胜利。而争取抗战胜利的中心关键,在使已经发动的抗战发展为全面的全民族的抗战,只有这种全面的全民族的抗战才能使抗战得到最后的胜利。”(中共去年八月决定)这些都是完全正确的,十五个月经验己经证明了。

抗战以来,把国内各个互相对立的阶级、党派、集团都团结起来了,各个不同的区域,不同的军队,都统一于国民政府与军事委员会指挥之下了,抗战十五个月的坚持,没有这个统一团结是不可能的。也只有抗战,才能统一团结各方面。这种统一团结,说是抗日民族统一战线。但十五个月经验又向我们证明:敌人破坏阴谋之严重与内部团结巩固之不足。抗战为什么遭受很多挫折,为什么至今还不能停止敌之进攻,实行我之反攻,除了客观原因之外,统一战线力量之不足,统一战线还没有必要的扩大与巩固,是其最主要的原因。由此可知,只有更加统一团结全民族,巩固与扩大抗日民族统一战线,才能支持长期战争与争取最后胜利,这是第一。第二,十五个月抗战,不但推动了全民族的团结,同时又暴露了这种团结之不足;而且推动了军事、政治、文化、党务、民运各方面的进步,同时又暴露了这种进步之不足。支持长期战争与争取最后胜利,必须发动全民族各阶层中一切生动力量,而欲达此目的,非从军事、政治、文化、党务、民运等各方面力求进步不可。没有各方面的更大的进步,就不能发动全民族一切生动力量,也就不能更进一步的统一团结全民族。第三,十五个月抗战又证明了民众援助抗战力量之伟大﹔同时也证明了民众力量之仅在开始发动,因而使抗战得不到民众的广大援助而过受了许多挫折。从此得到教训,国人必须进一步的认识抗战依靠民众这个基本问题。依靠民众则一切困难能够克服,任何强敌能够战胜,离开民众则将一事无成。中国今后的进步,还必须充分表现在发动民众力量这一方面。

总之,支持长期战争与争取最后胜利的唯一正确道路,在于巩固与扩大全民族的统一团结,在于力求进步以发动全民族的生动力量,在于依靠民众以克服困难,这就是我们的第三个总结。

同志们,坚持抗战,坚持持久战,力求团结与进步——这就是十五个月抗战的基本教训,也就是今后抗战的总方针。我们是能够战胜敌人的,只要我们与全国坚持这个总方针,并作了长期的广大的努力。抗日战争正在向清一个新阶段发展,新阶段中有许多新的任务,但这个总方针是不变的,十五个月经验作了我们观察新的形势提出新的任务的基础。

三、抗日民族战争与抗日民族统一战线发展的新阶段

(1)研究战争与统一战线的规律性是决定政策的基础

同志们,在我们总结了过去经验之后,重要的问题,在于看一看当前形势发展的趋向。抗日战争与抗日民族统一战线将会怎样从过去基础之上向前变化发展的?这是我们现在要答复的问题,这一点对于我们解决当前的问题有重要的意义。因为如果对于整个抗日战争变化发展的行程没有一个大概的估计,我们就只能跟着战争打圈子,让战争把自己束缚起来,而不能将其放在自己的控制之下,加之以调节整理,造出为战争所必需的条件,引导战争向我们所要求的方向走去,争取战争的胜利。因此必须懂得抗日战争的规律性,才能实现对于它的战略指导,才能决定为战争服役的一切战略,战术,政策,计划与方案。对于抗日民族统一战线也是一样,只有我们研究了与认识了它的规律性,我们才能有效地推动统一战线使之进入巩固发展之途,而为战争的胜利起其支柱的作用。

我们现在先来说战争问题。

(2)特定的历史条件与主观能力的优劣决定战争的发展过程

历史上的战争有一个阶段就完结的,例如一九○五年的日俄战争,只有日军进攻,俄军败退,就结束了。又如意亚战争,也只有意大利进攻,亚比西尼亚失败,就告结束。中国一九二六年开始的反对北洋军阀的战争,也是一样。这是一种情形,这是由于一方面双方强弱不同,又一方面双方指导能力优劣不敌而造成的,这是第一类战争。第二类战争,以两个阶段宣告完结。例如法俄战争,拿破仑从进攻到退却,俄国从退却到反攻,双方都有两个阶段。中国古代有名的吴魏赤壁之役,秦晋肥水之役,也是这样。虽则两军强弱不同,但弱者善于利用其它优良条件,给以正确指导,故于退却之后,接着反攻,战胜敌人。但是还有第三类战争,例如外国的七年战争,八年战争,三十年战争,百年战争,乃至二十年前四年的欧洲大战(特别表现于西战场),都有三个阶段。甲方进攻,乙方退却,为第一阶段。双方相持不决,为时甚长,为第二阶段。乙方反攻,甲方退却,为第三阶段。中国历史上也有许多这类的战争。这类战争的特点,在于有一个较长的或很长的相持阶段,这也是由于特定的历史条件与战争指导集团的特性而造成的。

中日战争属于那一类战争呢?我以为是属于第三类战争的。这是由于双方不同的历史条件与不同的战争指导集团之特殊情形而造成的。

(3)中日战争的长期性表现于战争的三个阶段

中日战争的长期性将表现于在敌则进攻,相持,退却,在我则防御,相持,反攻,这样三个阶段之中。由于敌强我弱(敌是优势,我是劣势,敌是帝国主义国家,我是半殖民地国家),故出现了敌方进攻,我方防御的第一阶段。不说退却而说防御,是说以战略的运动防御即节节抵抗的姿态而表现其退却,不是一下子干脆退却。但又由于在敌则小国,退步,寡助,在我则大国,进步,多助这些特殊的条件,我之英勇抗战又使敌在进攻中受到分散的困难与消耗的损失,而不得不于一定时机结束其战略上的进攻,转入军事上保守其占领地而从政治上与经济封锁上向我进攻的阶段。此时敌虽消耗,但一时尚未消耗到使之转入失败的程度;我虽坚决抗战与各方面向前进步,但一时也难进步到足以转入反攻驱敌出国的程度。依上诸因,一个双方相持的第二阶段,或中间阶段,就形成了。由于第二阶段中敌之困难与我之进步俱日增,又配合着国际有利于我不利于敌之形势,就能使敌强我弱敌优我劣的原来状态逐渐发生变化,逼到在全局看来日益于敌不利而有利于我之局面,再到敌我平衡,再到我优敌劣,彼时,就可转入我之反攻,敌之退却的第三阶段了。

上述三个阶段的看法,是依据敌我既存的与将来可能发生的双方相反对比之具体条件而作出的一种对于整个战争过程的估计,现在并不是事实,而是一种可能的趋势。要依我之主观努力,创造出为这种可能趋势所必要的条件,才能使可能趋势变为事实。然而依据既存条件,加上正确指导与全民族广大而坚持的努力,是能够使这种可能趋势变为事实的。

(4)速胜论者与亡国论者都反对这种估计

速胜论者反对三阶段论,认为我能迅速反攻,无需乎要一个中间阶段。这是不对的。抗日战争面前存在若许多困难,克服这些困难需要一定的时间,迅速反攻是不可能的。速胜论者的反对三阶段,是因为他们一方面过低估计了敌人力量,一方面又过高估计了自己力量的原故。亡国论者也反对三阶段,认为相持与反攻都是不可能的,中国只是一个亚比西尼亚。这是不对的。他们与速胜论者相反,过高估计了敌人力量,而过低估计了自己力量。在他们面前只有黑暗,承认敌人能够灭亡全中国,我之抵抗与努力只是徒劳,办到敌我相持亦不可能,更不说什么反攻胜敌了。因此,必须一方面反对速胜论,又一方面反对亡国论,才能坚持我们的三阶段论。而在当前情况下,反对亡国论比之反对速胜论更加重要。另有一些人,口头上赞成持久战,但不赞成三阶段论。这也是不对的。所谓持久战,所谓长期战争,表现在什么地方呢?表现在战争的三个阶段之中。如果承认持久战或长期战争,又不赞成三个阶段,那么,所谓持久与长期就是完全抽象的东西,没有任何的内容与现实,因而就不能实现任何实际的战略指导与任何实际的抗战政策了。实际上,这种意见仍属于速胜论,不过穿上“持久战”的外衣罢了。

(5)三阶段论与国际形势的关系

当张高峰事件发生之时,国内一部分舆论兴高彩烈,以为日苏战争如果爆发,中国就可以转入反攻,无需乎要持久战了。在这种观点下,三阶段论当然不能成立,我们的估计是错误的了。这是主要依靠外援的思想,是速胜思想之一种。然而国际形势不是照着这些朋友们的主观志愿发展的,而是依照它自己的规律。世界的主要重心在欧洲,东方是环绕着它的重要部分。世界的主要和平阵线国家与主要法西斯国家,正在为着欧洲战争危机问题,在西方纠缠不清,无论是各大国间的战争前夜或战争爆发,西方的各大小国家都将以解决欧洲问题放在议程的第一位,东方问题则不得不暂时放在第二位。拿这种情况来看中日战争,迅速反攻的两阶段论也是没有理由的。我们必须以自力更生为主,我们不但不怕三阶段,而且正要造成三阶段。三阶段是中日战争的规律,不但在敌我力量对比上有其根据,而且也在国际形势上有其根据。

(6)相持阶段是战争的枢纽

三个阶段的主要特点,在于包含一个过渡的中间阶段。这就是说,第一,我之抗战必须用尽一切努力去停止敌之进攻,假如敌之进攻不能在一定时间与一定地区停止下来,就无所谓性质不同的三个阶段。第二,相持阶段出现了时,必须用尽一切努力去准备我之反攻所必需的一切条件,设若不然,就不能过渡到反攻阶段里去,而只是永远的相持,也无所谓三阶段。在这里,对于速胜论者,我们肯定地说:必须经过一个准备时期,才能团结全国,克服困难,生长新的力量,同时配合着敌人的困难,国际的援助,然后实行反攻,驱敌出国,否则是不可能的。拿主要依靠自力胜敌的观点来看问题,不可避免的要作出这个结论。对于亡国论者与悲观主义者,我们肯定地说、只有这个过渡阶段,才是全战争的枢纽。中国化为殖民地还是获得解放,不决定于第一阶段中主要的大城市与交通线之丧失,而决定于第二阶段中全民族努力的程度。大城市与交通线的丧失是可惜的,增加了敌人的力量,减少了自己的力量。然而很多没有丧失的东西尚可作为制胜敌人的资本,唉声叹气于宝物的丧失是无益的。第一阶段中保存着的领土与各种有生力量,特别是已经获取的军事、政治、文化、党务、民运等各方面的进步,是最可宝贵的,这是第二阶段中继续进步与准备反攻的基础。然而这仅仅是继续进步与准备反攻的基础,还不能决定反攻,决定反攻的东西是第二阶段中增加上来的力量,没有伟大的新生力量之增加,反攻只是空唤的。

(7)三个阶段的特点,第一阶段

抗日战争三个阶段的特点,已经出现了的,尚未出现而可以预计得到的,有概略指明之必要,对于指导战争与决定政策有重要的关系。

第一阶段有些什么特点或重要标志呢?

有如下三方面的东西。

第一,中国方面:民族统一战线的形成,全国军队的参战,抗战的坚决性,国民党抗战建国纲领的发布,国民参政会的开会,共产党及其它党派的取得合法地位,游击战争的创造,全国军队的进步,民众运动的发展等等。这些都是中国方面表现进步的大事件。但同时,却又有许多不利事件与不良现象,例如:主要大城市、交通线与主要工商业的丧失,土地与人口的丧失,全国进步的不平衡(有些地方进步得非常之慢),政治制度之一般还仅在开始走向民主化,顽固分子与腐败现象的存在,妥协倾向的酝酿等等。

第二,敌人方面:军力财力的消耗,世界舆论的责备,军纪的败坏,军队战斗力的相对地减弱,国内人心与前线军心不满的酝酿,张高峰战争的失败,汉奸军队的难于组成及已经组成者的无能等等。这些都是表现其困难的大事件。但同时却又有表现其能力的东西,那就是:进攻的坚决性,军力的顽强,占领地的扩大,政治组织力的强韧,阴谋机关的有力等等。

第三,国际方面:援华运动的增长,苏联力量的壮大及其对于中国的援助,这些都是有利于中国的东西。但是还有别的东西:欧洲大战的酝酿,英日间某种程度的妥协倾向,各国军火原料的助敌,这些都于中国不利。

以上中国,日本,国际的许多东西,都是抗战第一阶段中十五个月来表现的特点。这些特点,将分别生其影响于新的阶段之中。

(8) 第二阶段

在假定武汉不守的情况之下,战争形势又将出现许多新的东西。虽然敌占武汉并不即等于旧阶段的完结,新阶段的开始,由现在敌人尚能继续进攻到他被迫停止进攻之时的这段时间,还是一个由旧阶段转向新阶段去的过渡期间。虽然如此,但武汉不守成为事实之后,就将发生许多新的情况。

武汉不守之后,以及新阶段的大部分时间,可以预计的基本情况,将是一方面更加困难,又一方面则更加进步。这是新阶段中的基本特点。

更加困难将表现于下述各方面:(一)由于主要的大城市与交通线之丧失,国家政权与作战阵地就在地域上被敌分割了,由此将发生许多新的问题;(二)财政经济之异常困难;(三)英日某种程度的妥协倾向(或相反,在日本坚持独占与威胁南洋的条件下,英日有进一步冲突的可能);(四)如果敌攻广州,中国主要的海道交通有被割断之虞,国际援助将暂时的部分的减弱;(五)全国性伪政权有形成的可能及其对于抗日阵线的影响;(六)抗日阵线中部分叛变的可能,妥协空气的增长;(七)悲观情绪的生长,意见纷歧现象的增加等等。这些都是可能发生而将加诸抗日战争身上的新困难事项。估计到这些困难,才便于有准备有计划地克服之。

更加进步将表现在下述各方面:(一)蒋委员长与国民党的坚持抗战方针及其在政治上的更加进步;(二)国共关系的改善,抗日民族统一战线的巩固与扩大;(三)军队改造工作的进步;(四)游击战争的广大发展;(五)国家民主化的进步;(六)民众运动的更大发展;(七)新的战时财政经济政策的实施;(八)抗战文化教育的提高;(九)苏联援助的继续与可能增加及中苏关系的更加亲密等等。

整个第二阶段即相持阶段,是中国准备反攻的阶段。其时间长短,依敌我力量变化的程度及国际环境如何而定。但我们应该准备长期战争,熬过这一段艰难路程,胜利的坦途就到来了。

第二阶段中虽然敌我在战略上是相持的,但仍有广泛的战争,主要表现于主力军在正面防御,而广大游击战争则发展于敌人的后方。那时,游击战争在许多重要战略地区将变为非常艰苦的战争,现在就应该准备对付这种艰苦。

(9) 第三阶段

具体情况不能预计。但彼时必是我之反攻条件业已准备完毕,同时敌之困难程度大大增加起来,国际形势又大大于我有利。彼时战争形势,不是战略防御或战略相持,而是战略反攻了;不是战略内线,而是战略外线了。彼时国内政治上必须有大的进步,军事上必须有新式技术,否则反攻是不可能的。

(10)保卫武汉是争取时间问题不是死守问题

保卫武汉斗争的目的,一方面在于消耗敌人,又一方面在于争取时间便于我全国工作之进步,而不是死守据点。到了战况确实证明不利于我而放弃则反为有利之时,应以放弃地方保存军力为原则,因此必须避免大的不利决战。战略决战,在一二两阶段中都是不应有的,那足以妨碍抗战的坚持与反攻的准备,因此必须避免。避免战略决战而力争有利条件下的战役与战斗的决战,应是持久战的方针之一。于必要时机与一定条件下放弃某些无可再守的城市,不但是被迫的不得已的,而且是诱敌深入,分散、消耗与疲惫敌人的积极的政策。在坚持抗战而非妥协投降的大前提下,必要时机放弃某些据点,是持久战方针内所许可的,并无为之震惊的必要。

(11)由目前过渡到相持阶段

只有停止敌之进攻,才有利于我之准备反攻。而要达此目的,还须给一个大的努力。故由目前过渡到敌人被迫停止其战略进攻,转入保守其占领地,出现整个敌我相持的阶段之时,还是一个斗争的过程,须克服许多困难才能达到。因为敌在占领武汉之后,还不会立即结束其进攻,他必定还想向西安、宜昌、长沙、衡州、梧州、北海、南昌、汕头、福州等地及其附近地区进攻。我要停止敌之进攻,还须针对着敌人这种企图继续执行战略的运动防御战,用极大努力进行坚持的战斗,再行大量地消耗敌人而又不为敌人所算,使敌之进攻不得不停止,把战局过渡到敌我相持的有利局面。

(12)但相持局面快要到来了

敌人占领武汉之后,他的兵力不足与兵力分散的弱点将更形暴露了。如果他再要进攻西安、宜昌、长沙、南昌、梧州、福州等地并作占领之企图,他的兵力不足与兵力分散之弱点所给予他的极大困难,必将发展到他的进攻阶段之最高度,这就是我之正面主力军的顽抗与我之敌后庞大领土内游击战争的威胁,所加给敌人兵力不足(他不能足)与兵力分散(他不能不分散)现象上的极大困难。这一形势在敌则兵力不足与兵力分散,在我则正面防御与敌后威胁,这是敌之极大劣势,我之极大优势。当然,在整个敌我力量对比上说来,敌强我弱敌优我劣的基本形势并未变化,这只有在长期相持阶段内我用全民族的极大努力,并配合国外条件,才能使之变化。然而敌在进攻武汉的战斗中,他之强的力量已经进一步发挥了。这种强的力量之进一步发挥,一方面固然给了我们以损失,然而同时就给了他自己以困难。因为敌之强的力量(同时即是其不足的与分散的力量)在其作了进一步的发挥之后,气力势将衰退下去,就不得不使其总的战略进攻接近了一个顶点。我们承认敌之进攻还有一点余威,并最好与最恰当的是估计到他的这点余威还相当的大,因此还有充分可能他要攻略西安、宜昌、长沙、南昌、梧州、福州等处及其附近地区,甚至要准备他向着整个粤汉路与西兰公路之进攻。然而这在总的敌人力量上将只是一点余威。在日本的整个国力上说来,他要北防苏联,东防美国,南对英法,内镇人民,他只有那么多的力量,可能使用于中国方面的用的差不多了。并且在其正面与占领地内必须对付的广泛战争还依然存在,日苏,日美,日英,日法之间的矛盾在增长着,国内政府与人民的矛盾,前线官长与士兵的矛盾,大量支出与财政竭蹶的矛盾在加深着,这些都是使得敌人大大皱眉的地方。我们及全国人民必须看到这些地方,不为主要大城市与交通线之丧失所震惊,赞助政府调整全国之作战,有计划地部署粤汉路、陇海路、西兰公路及其它战略地区之作战,部署庞大敌后地区之游击战争,捉住敌人兵力不足与兵力分散的弱点,给以更多的消耗,促使更大的分散,使战争胜利地与确定地转入敌我相持的新阶段,这是全国当前的紧急任务。

(13)敌力在逐渐减少我力在逐渐增加中

敌人是否增加了力量呢?就其原有的力量来说,没有什么增加,相反,他的力量大大地减少了。敌人原有的军力与经济力,是大大消耗了。十五个月战争中,他的军力伤亡了数十万人,消耗了大量的武器,弹药,与军用资材,毁灭了数百架飞机与百余艘军舰,支出了数十万万元经费,这个消耗在日本历史上是空前的。直到他被迫停止其战略进攻之时为止,他还要消耗一大批力量。在这点上,他的盟友希特勒早已大大地发起愁来了。然则敌人毫无力量的增加吗?有的,这就是对于中国主要的大城市与交通线及部分乡村之占领,从各国手里及中国民族资本手里夺取了市场,从中国手里夺取了资源与生产工具,夺取了许多人力,这些是日本战争之唯一目的,他是暂时地增加了这些东西。然而问题是:这些东西能抵偿已有的战争消耗吗?不能,消耗了的全部战争“投资”是已经消耗了,他要取偿还需付以新的生产投资。问题又是:抛开日本生产投资之无能不说,假定他能的话,他能取偿其战争消耗吗?也不能。因为往后依然存在着广大战争,依然每天要消耗。只要有广大的敌后游击战争存在,例如现在华北的游击战争那样,他的取偿是很困难的。由于不断的战争,他将不但不能取偿旧的,而且还须支付新的,只要我们的抗战坚持下去,日本的这条可怜命运是大体确定了的。我们说日本在第二战略阶段即相持阶段中将逐渐化强为弱,化优为劣,这种继续消耗是决定的一方面。现在说到中国方面。中国力量究竟是减少了呢?还是增加了?我们的回答是减少了。又增加了。减少的是原有力量的质与量,这表现在军队人员武器的消耗,人口、工业、土地与资源的损失等上面,这是使得我们感到困难的重要的一方面。然而不是没有增加的,增加的是新的质与量,这表现在全国的团结,军队的进步,政治的进步,文化的进步,人民觉悟程度与组织程度的提高。主力军虽后退了,游击战争却前进了。一部分地方虽损失了,另一部分地方却进步了。问题是在:增加的程度今天还不够,今天还不够停止敌之进攻:今后更不够实行我之反攻,因此发生了必须用广大持久的努力去增加新的力量的问题。而这种增加,即全民族各个阶层中生动力量之更大发动与党政军民各方面之更大进步,基于今后之广大持久的努力是完全可能的。在主要的依靠自己生长的力量,再配合之以敌人困难之加重,国际助我之增强,就能使整个敌我形势发生变化,由敌优我劣之现时形势,先走到敌我平衡,再走到我优敌劣,这就是长期相持阶段中必须解决也可能解决的根本问题。

(14)敌据城市我据乡村,所以战争是长期的,但乡村能够最后战胜城市

于是问题在:敌人占领中国主要的大城市与交通线之后,敌据城市以对我,我据乡村以对敌,乡村能够战胜城市吗?答复:有困难,但是能够的。抗日战争的长期性,不但由于敌是帝国主义国家,我是半殖民地国家,而且由于这个帝国主义又复占据我之城市,我则退至乡村以抗敌,因而造成了长期性,速胜论在此是毫无根据的。然而今天中国的城市乡村问题,与资本主义外国的城市乡村问题有性质上的区别。在资本主义国家,城市在实质上形式上都统制了乡村,城市之头一断,乡村之四肢就不能生存。不能设想,在英、美、法、德、日、意等国,能够支持长期反城市的乡村农民战争。半殖民地小国也不可能。半殖民地大国如中国,在数十年前也很困难。半殖民地大国如中国,在今天,却产生了这种可能。这里明显的是三个三位一体的条件。第一是半殖民地条件。在半殖民地,城市虽带着领导性质,但不能完全统制乡村,因为城市太小,乡村太大,广大的人力物力在乡村不在城市。第二是大国的条件。失去一部,还有一部。敌以少兵临大国。加以我之坚强抵抗,就迫使敌人发生了兵力不足与兵力分散的困难,这样就不但给了我一个总的抗日根据地,即大后方,例如云、贵、川等地,使敌无法占领:而且在敌后也给了我以广大游击活动的地盘,例如华北、华中、华南等地,使敌无法全占。第三是今日的条件。如果在数十年前中国被一个强大帝国主义国家武装侵占,例如英占印度那样,那是难免亡国的。今天则不同,今天主要的是中国进步了,有了新的政党,军队与人民,这是胜敌的基本力量。其次是敌人退步了,日本帝国主义的社会经济发展过程已临到衰老的境界,日本资本主义的发展造成了与造成着把他自己送进坟墓的条件。又其次是国际形势变化了,旧的世界接近死灭,新的世界已见曙光。这些道理,我在“论持久战”中已详说过了。总之,在今天的半殖民地大国如中国,存在着许多优良条件,利于我们组织坚持的长期的广大的战争,去反对占领城市的敌人,用犬牙交错的战争,将城市包围起来,孤立城市,从长期战争中逐渐生长自己力量,变化敌我形势,再配合之以世界的变动,就能把敌人驱逐出去而恢复城市。毫无疑义,乡村反对城市就在今天的中国也是困难的,因为城市总是集中的,乡村总是分散的,敌人占领我主要的大城市与交通线之后,我之行政区域与作战阵地就在地域上被分割,给了我们以很多困难,这就规定了抗日战争的长期性与残酷性。然而我们必须说,乡村能够战胜城市,因为有上述三位一体的条件。在内战条件下,极小部分的乡村支持了长期反对城市的战争,还当帝国主义各国一致反共的时期。谁能说在民族战争条件下,又当帝国主义阵营分裂之时,中国以极大部分的乡村,不能支持长期战争去反对城市敌人呢?毫无疑义是能够的。并且现在的所谓乡村,与内战时期的乡村有很大不同,不但地域广大,而且在云、贵、川等省大后方中,尚有许多城市与许多工业,尚可与外国联络,尚可建设。依据于大后方的保持与敌后游击战争根据地的建立,从长时期中生息我之力量,削弱敌之力量,加上将来国际有利条件之配合,就能举行反攻,收回城市。蒋委员长在去年十二月告国民书中指出﹕中国持久抗战,其最后胜利之中心不但不在南京,抑且不在各大城市,而实寄于全国之乡村与广大强固之民心。”这是完全正确的,战争虽困难,胜利前途是存在的。

(15)妥协危机严重存在,但是能够克服的

我们早就说过,一部分患着恐日病的人们时刻企图动摇政府的抗战决心,主张所谓和平妥协,过去曾见之于南京失守之后,现在又在蠢蠢欲动了,这是敌人阴谋在抗日阵线内部的反映。这种危机是严重存在的,国人必须充分注意,不让亲日派得售其奸。亲日派的企图和敌人的企图是一致的,必然集中于反蒋反共,假令得售其奸,抗战的前途就成大问题。因此,全国上下憬然觉悟于敌人阴谋与内部反间之可畏,自动自觉地努力反对这种阴谋,一刻也不容放松。这种危机是否能够克服呢?那是能够的。在国共两党及一切爱国志士团结一致并作了必要的努力之后,克服妥协危机,驱除助敌张目的妖魔鬼怪,而把抗战坚持下去,不但是绝对必要的,而且是完全可能的。因为亲日派究竟没有多大的势力,抗日派的势力大于亲日派。

(16)相持阶段中游击战争的新形势

新阶段中,正面防御的是主力军,敌后游击战争将暂时变为主要的形式。但敌后游击战争在敌我相持的新阶段中,将采取一种新形势发展着。什么是游击战争的新形势呢?即第一,在广大地区中仍能广泛的发展。这是因为在我则土地广大,在敌则兵力不足与兵力分散,只要我能坚持发展游击战争的方针并正确地指导之,敌要根本限制我之发展是不可能的。但第二,在某些重要战咯地区,例如华北与长江下流一带,势将遇到敌人残酷的进攻,平原地带将难于保存大的兵团,山地将成为主要的根据地,某些地区的游击部队可能暂时的缩小其数量,现在就应准备这一形势的到来。在现在,为了策应正面主力军的战斗,为了准备转入新阶段,应把敌后游击战争大体分为两种地区。一种是游击战争充分发展了的地区如华北,主要方针是巩固已经建立了的基础,以准备新阶段中能够战性敌之残酷进攻,坚持根据地。又一种是游击战争尚未充分发展,或正开始发展的地区,如华中一带,主要方针是迅速的发展游击战争,以免敌人回师时游击战争发展的困难。在将来,为了配合正面防御使主力军得到休息整理机会,为了生长力量准备战略反攻,必须用尽一切努力坚持保卫根据地的游击战争,在长期坚持中,把游击部队锻炼成为一枝生力军,拖住敌人,协助正面。一般说来,新阶段中敌后游击战争是比较前一阶段要困难得多的,我们必须预先看到这种困难,承认这种困难,不可因为前一阶段的发展容易而冲昏了头脑,因为敌人一定要转过去进攻游击战争。然而是能够坚持的,一切敌后工作的领导人们必须要有这种自信心。因为民族战争中的游击战争,不论敌人如何的强,总比内战时的条件优良的多。在这里,争取与瓦解伪军以孤立日寇,是非常重要的任务。

(17)抗日战争发展的新阶段同时即是抗日民族统一战线发展的新阶段

以上说的都是抗日战争的形势问题,以下要说到抗日民族统一战线的形势。

抗日战争发展到了新的阶段之时,同时即是抗日民族统一战线发展到一个新的阶段之时。由于新阶段中将遇到比前更多的困难,抗日民族统一战线也就应该适应这种情况而表现其向困难斗争并将任何困难战而胜之之伟力。为了在目前过渡期间以及到了将来的新阶段,中国抗日民族统一战线不是表示其对于困难之无能,而是表示其具有克服困难之伟力,就必须认真的巩固统一战线与扩大统一战线。长期的战争必须有长期的统一战线才能支持,战争的长期性与统一战线的长期性,是不能分离的。

(18)国民党有光明的前途

抗日民族统一战线是以国共两党为基础的,而两党中以国民党为第一大党,抗战的发动与坚持,离开国民党是不能设想的。国民党有它光荣的历史,主要的是推翻满清,建立民国,反对袁世凯,建立过联俄、联共、工农政策,举行了民国十五六年的大革命,今天又在领导着伟大的抗日战争。它行三民主义的历史传统,有孙中山先生蒋介石先生前后两个伟大的领袖,有广大忠忱爱国的党员。所有这些,都是国人不可忽视的,这些都是中国历史发展的结果。

抗日战争的进行与抗日民族统一战线的组成中,国民党居于领导与基干的地位。十五个月来,全国各个抗日党派都有进步,国民党的进步也是显著的。它召集了临时代表大会,发布了抗战建国纲领,召集了国民参政会,开始组织了三民主义青年团,承认了各党各派合法存在与共同抗日建国,实行了某种程度的民主权利,军事上与政治机构上的某些改革,外交政策的适合抗日要求等等,都是具有历史意义的大事件。只要在坚持抗战与坚持统一战线的大前提之下,可以预断,国民党的前途是光明的。

然而至今仍有不少的人对于国民党存在着一种不正确的观察,他们对于国民党的前途是怀疑的。他们对于国民党怀疑的问题,就是能否继续抗战,能否继续进步,与能否成为抗日建国的民族联盟的问题,而这三个问题是互相结合不可分离的。我们的意见怎样呢?我们认为国民党有光明的前途,根据各种主客观条件,它是能够继续抗战,继续进步,与成为抗日建国的民族联盟的。

由于敌人进攻的坚决性,敌人对于中国各个阶层的严重的打击与掠夺,全国军队对于敌人的愤怒与抗战的英勇,全国人民抗日运动的高涨,国际有利形势之存在等事实,基本上决定了全中国与国民党的政治方向。第一,任何党派,包括国民党共产党及其它抗日的政党与团体在内,是非继续抗战下去不可的,谁不继续抗战,谁就只有一条当汉奸的出路,此外没有任何出路。第二,任何党派,只要它是继续抗战的,就非继续进步不可。诚然,国内政治的进步不迅速不普遍,因此招致了战争的损失。但也正因为损失,使得今后不能不在政治、军事、文化、党务、民运各方面求进步,以便能够抗拒敌人,恢复失地。这不论是当权的国民党也罢,其它党派也罢,都非继续进步不可。第三,国内进步的重要一环,是国民党组织形式的民主化,使其本身变为抗日建国的民族联盟,变为抗日民族统一战线的最好的组织形式。这种可能有没有呢?我以为也是有的。抗日战争的大势所趋,国民党如果不向广大民众开门,容纳全国爱国党派与爱国志士于一个伟大组织之中,那要担负起继续抗战与战胜敌人的艰难任务是不可能的。在国民党五十多年的历史中,每遇大的革命斗争时,总是把它自己变为革命民族联盟的,最显著而最有历史意义的有两次。第一次,从同盟会组成到辛亥革命,孙中山先生为了反对满清建立民国之目的,联合了一切反满的革命党派(从光复会到哥老会),在这个期间,它的党员充满了英勇斗争的事迹,再接再厉,富于朝气,因而取得了辛亥革命的成功。第二次,民国十三年至十六年,为了反帝反军阀之目的,对内联合了工农与共产党,对外联合了社会主义的苏联,建立了有名的“三大政策”,因而创设了黄埔,建立了党军,取得了北伐战争的胜利。所有这些、不但表现了国民党统一战线政策的发展,也表现了三民主义的发展。今天是国民党历史上第三次变为革命民族联盟的时机,为了反对日本帝国主义与建立三民主义共和国,必须也可能把它自己变为抗日建国的民族联盟。这一伟大的运动已在开始,承认共产党与其它党派的合法存在,承认八路军加人国民革命军系统,抗战建国纲领中明白宣布“欲求抗战必胜建国必成,固有赖于本党同志之努力,尤须全国人民戮力同心共同负担”,以及国民参政会的召集与三民主义青年团的组织,都表示了把它自身变为抗日建国的民族联盟之开始。现在问题是:共产党对于国民党的这一扩大组织的运动将取何种态度?赞成还是反对?我们说,我们任何时候都是赞成国比党把它自己扩大发展成为革命民族联盟的。民十三国民党改组之时,我们就取了赞助政策。今天更当民族危机万分严重之际,我们将尽一切可能赞助之。理由是抗日的友军越大越好,单单一个共产党的发展进步,是不够打退日本帝国主义的。处于第二党地位的中国共产党,虽然发起了与坚持了各党各派各军的统一战线,并在自己的组织上向着广大革命志士开门,用以力争抗日的胜利。但若处于第一党地位的国民党依然保存过去那样的老状态,那就对于抗战,对于统一战线,都非常不利的。抗战将不能获胜,全民族陷于危险,共产党与无产阶级也逃不脱这种危险。所以共产党不但不反对,而且十分希望与坚决赞助国民党扩大与巩固其组织,实行党内民主化,并使其本身变为革命的民族联盟,以利继续抗战与争取最后胜利。

(19)但在国民党的前途上尚有障碍物,须努力克服才能发展

国民党的光明前途是存在的,其进步与发展是可能的,蒋介石先生及国民党的大多数是在领导和推动国民党前进。然而谁都明白,国民党中还存在着一些守旧分子,障碍着国民党进步的速度与程度。由于这些分子的存在,并与社会上许多守旧分子相结合,就在民族革命战争的洪流中造成了一股逆流,顽固地抵抗进步之舟,相当有力地阻挠着国家民主化,阻挠着一切为抗战必需的进步事业之推行,阻挠着蒋介石先生在历次宣言、谈话、演说、命令中所说很多很好的方针方法之推行,阻挠着国民党抗战建国纲领之实施,阻挠着国民政府救国法令之实施,阻挠着民众运动之发展,这些都是事实,都是国民党进步所以不迅速不普遍与抗战所以受到许多不利的重大原因。他们是反对国民党进步,反对国民党发展,甚至主张妥协的,如果这些分子占居优势,那中国的民族解放事业就要受到极大挫折,所以值得严重注意。然而我们坚决相信,这种守旧势力是不能永久存在的,是没有占优势也难于占优势的,他们是逆流,但并非主流。在蒋委员长的领导,国民党大多数人的努力与全国人民的赞助之下,这种守旧倾向是能够克服的。共产党坚决赞助国民党的进步,而对于阻碍进步的守旧分子则希望他们弃旧图新,一同进步。我们希望这些人变一变,“君子之过如日月蚀”,改变过来就是好的。事实上我们已经看到了不少守旧的人在抗战过程中进步了,今后必仍有许多人会从抗战教训中发生觉悟而和大众一齐进步起来。这就是抗日战争中守旧分子的可变性。但也可能有少数人变得更坏,甘心被抗战巨涛席卷以去,这也是可变性的一面,对于这种人,就没有什么可惜的了。

(20)其它党派同样有光明前途

一切加入抗日战争与抗日民族统一战线的党派,在坚持抗战与坚持统一战线的大前提下,都有发展的前途,我们都愿意与之建立长期合作,并给以尽可能的赞助。这不论对于第三党,国家青年党,国家社会党,救国会派,或其它任何集团,任何党派,都是一样。很明显的,所谓一切党派在坚持抗战坚持统一战线的大前提下都有光明前途,是包括了克服各党内部守旧倾向这种努力的。如果存在着不利抗战与统一战线的守旧倾向而任其发展下去,那就有断送其光明前途的危险。这不论是国民党也好,共产党也好,其它党派也好,都是一样,都应充分注意的。

(21)中国抗日民族统一战线的特点

由于中国的历史原因,使得今天中国的抗日民族统一战线,不同于任何外国的统一战线,如人民阵线等,也不同于中国历史上的统一战线,如第一次国共合作等,有它今天的中国的特点。认识这些特点,对于巩固与扩大抗日民族统一战线,是有非常严重的意义的。

这些特点是什么呢?归结起来,共有八个,即是:全民族抗日的,长期性的,不平衡的,有军队的,有十五年经验的,大多数民众尚无组织的,三民主义的,处于新的国际环境中的。

首先是全民族抗日的。这个特点规定了我们统一战线的根本性质。一方面,我们统一战线的目的,是为了反对侵入国土的异族日本帝国主义而建立起来与发展起来的。又一方面,我们统一战线的组成,是包括全民族所有不同党派,不同阶级,不同军队,不同国内民族之一个最广大团体。由于是反对异族侵略的,所以组织成分能够如此之广大。由于组织成分之异常广大,所以这个统一战线具有伟大的力量:但同时,统一战线内部又难免许多相互间的磨擦,而须恰当地调整之,才能达到团结对外之目的。我们统一战线的这种最基本的特点——政治目的之反对异族侵略与组织成分之异常广大,不同于法国与西班牙的人民阵线,也不同于第一次大革命时期的民族阵线(当时的国共合作),使得今天的统一战线产生了许多的特殊内容与特殊结果,这是今天统一战线的第一个特点与优点,虽然在其组织复杂一方面不免同时包含着缺点。

第二是长期性的。这个特点是从第一个特点产生的。由于这个统一战线是用民族战争反对日本帝国主义的,而日本帝国主义是一个强的帝国主义,就产生了抗日战争的长期性,因而又产生了统一战线的长期性。这一点,我在报告的第五部分还要说到的,这是一切政策的出发点。这一点也和第一次国共合作不相同。

第三是不平衡的。由于历史原因,造成了各党派各阶层政治力量的不平衡,同时在地域的分布上也表现这种不平衡。国民党是第一个具有实力的大党,共产党是第二党,其它又在其次。这一情况,产生了许多特殊的东西。

第四是有军队的。国共两党都有军队——这个特殊历史条件造成的结果,不是缺点而是优点。由于有两党的军队,使得抗日战争中两党克尽分工合作的最善责任,互相观摩激励的好处也更多了。这一点和西班牙相同;但和法国不同,和第一次国共合作也不同,这也是使得两党能够长期合作的因素之一。

第五是有十五年经验的。一九二四至二七年第一次国共合作的四年,一九二七至三六年国共分裂的九年,现在国共重新合作又有了两年,这个十五年中合作——分裂——又合作的经验,最深刻地教育了国共两党、其它党派与全国人民,结论是:只应合作,不应分裂。这也是长期合作的基础之一。这种宝贵经验,世界各国都没有,第一次国共合作时也没有的。

第六是大多数民众尚无组织的。这是中国的特点,西洋各国与此不同,所以是一个缺点,使得统一战线缺乏现成有组织的民众基础。但同时,各党之间可以分工地去组织民众,不须挤在一块老是磨擦,因为有的是尚无组织的民众,正待组织起来以应抗战之急需。

第七是三民主义的。抗日民族统一战线以三民主义为政治基础,不但是合作抗日的基础,而且是合作建国的基础。三民主义中的民族主义将引导这个合作到争取全民族解放,其民权主义将引导这个合作到彻底的建立民主国家,其民生主义则更可能引导这个合作到很长的时期,三民主义的政治纲领与政治思想保证着统一战线的长期性。

第八是处在新的国际环境中的。今天的世界政治经济条件,比之第一次合作与两党内战两时期都不同。今天只有一部分帝国主义国家如日德意等反对国共合作与抗日民族统一战线。另一部分帝国主义国家,由于他们与日本的矛盾,现时也并不反对我们的统一战线,反而采取赞助的立场。所有国家的先进人民都是赞助我们的,苏联更是诚挚的赞助。这种新的国际环境,对于我们的长期合作有重大影响。

深刻地研究与认识上述这些特点,才能采取恰当的政治上的政策与工作上的态度。不是头痛医头脚痛医脚地应付政治问题与工作问题,而是站在科学的基础上正确地解决问题,抗日战争的胜利与抗日民族统一战线的巩固与扩大,是需要这种科学基础的。

(22)统一战线的新形势

抗日战争的新阶段中,抗日民族统一战线必须以一种新的姿态出现,才能应付战争的新局面。这种新姿态,就是统一战线的广大的发展与高度的巩固。十五个月团结抗战的教训,将促使各党认识这种发展与巩固之必要。发展方面,是扩大各党的组织与扩大民众的组织。巩固方面,是各党采取新的政策与新的工作,减少相互间的磨擦,办到真正的精诚团结,共赴国难。抗战新阶段中存在着许多的困难,唯有统一战线各党广大发展其组织与高度巩固各党的内部关系与各党之间的相互关系,才能有力地执行新的政治任务,战胜新的困难,达到停止敌之进攻与准备我之反攻之目的。这就是抗日战争新形势中统一战线的新形势,由于各党的共同努力与全国人民的热烈拥护,这种新形势的到来是完全可能的。

四、全民族的当前紧急任务

根据过去抗战的总结与当前抗战及统一战线发展新阶段的估计,全民族的当前紧急任务应该是什么呢?应该和过去有些什么不同呢?

总的任务应该是:坚持抗战,坚持持久战,巩固与扩大统一战线,以便克服困难,停止敌之进攻,准备力量,实行我之反攻,达到最后驱逐敌人之目的。

分别说来,有如下各方面的具体任务。一切抗日民族统一战线的组织成分,应该赞助政府,并在政府领导之下,动员全民族实行起来,共产党员应成为执行这些任务的模范。

(1)高度的发扬民族自尊心与自信心,坚持抗战到底,克服悲观情绪,反对妥协企图

估计到新的抗战形势下,必有一部分人,因为主要大城市与交通线的丧失,财政经济的困难,国际援助的不及时,因而发生着与增长着对于抗战前途悲观失望的情绪。而日寇,汉奸,亲日派,必将利用这种情绪,大放其和平妥协空气,企图动摇我抗战的决心。因此,全民族的第一任务,在于高度发扬民族自尊小与自信心,克服一部分人的悲观情绪,坚决拥护政府继续抗战的方针,反对任何投降妥协的企图,坚持抗战到底。这一任务,比过去任何时期为重要。

为此目的,必须动员报纸,刊物,学校,宣传团体,文化艺术团体,军队政治机关,民众团体,及其它一切可能力量,向前线官兵,后方守备部队,沦陷区人民,全国民众,作广大之宣传鼓动,坚定地有计划地执行这一方针,主张抗战到底,反对投降妥协,清洗悲观情绪,反复地指明最后胜利的可能胜与必然胜,指明妥协就是灭亡,抗战才有出路,号召全民族团结起来,不怕困难,不怕牺牲,我们一定要自由,我们一定要胜利,用以达到全国一致继续抗战之目的。

为此目的,一切宣传鼓动应顾到下述各方面。一方面,利用已经产生并正在继续产生的民族革命典型(英勇抗战,为国捐躯,乎型关,台儿庄,八百壮士,游击战争沟前进,慷慨捐输,华侨爱国等等)向前线后方国内国外,广为传播。又一方面,揭发,清洗,淘汰民族阵线中存在着与增长着的消极性(妥协倾向,悲观情绪,腐败现象等等)。再一方面,将敌人一切残暴兽行的具体实例,向全国公布,向全世界控诉,用以达到提高民族觉悟,发扬民族自尊心与自信心之目的。须知这种觉悟与自信心之不足,是大大妨碍着克服困难与准备反攻的基本任务的。

(2)拥护蒋委员长,拥护国民政府,拥护国共合作,反对分歧与分裂,反对任何的汉奸政府

新环境中,敌人的方针,必然集中于反蒋反共,建立全国性的汉奸政府,企图推翻蒋委员长,推棉国民政府,破环国共合作与全国团结。针对着敌人的这种方针,全民族的第二个任务,在于号召全国,全体一致诚心诚意的拥护蒋委员长,拥护国民政府,拥护国共合作,拥护全国团结,反对敌人听施任何卦利于蒋委员长,国民政府,国共合作与全国团结的行为,反对任何的汉奸政府统治中国。

为此目的,必须调节国共两党之关系,调节中央与地方之关系,调节抗战各军之关系,调节政府与人氏之关系。在这些关系中,提倡公平合理互助互爱之精神,减少磨擦,减少意见纷歧现象,反对利用困难与政府为难之行为。号召全国严重注意敌人,汉奸,亲日派在我们内部的挑拔离间,制造不满,制造纷歧,鼓励磨擦之阴谋鬼计。务使蒋委员长与国民政府的威信不受任何影响,务使国共合作与全国团结日益亲密起来,树立在困难环境中继续抗战的坚固重心,用以对抗敌人与汉奸政府,克服困难,准备反攻。

(3)提高主力军的战斗力,保卫华中华南与西北,停止敌之进攻

针对着敌人现时进攻武汉并继续进攻华南与西北之企图,全民族的第三个任务,在于提高主力军的战斗力,整理现有军队,增编新的军队,为保卫华中华南与西北而战,停止敌人进攻。为此目的,中国主力军方面,第一,必须发展高度的运动战,同时辅助之以必要的与可能的阵地防御,节节抗拒敌人,消耗敌之实力。第二,必须在大后方建立可能的军事工厂,并提高制造能力,接济前线的枪械与弹药。第三,必须在军队中认真实施民族革命的政治工作,实施政治文化娱乐等教育,提高全军英勇奋斗持久苦战的精神。第四,就现有物质基础改善士兵生活,在连队中组织经济委员会,由士兵管理伙食。第五,提倡自觉纪律,废止打骂制度,提倡官兵之间的亲爱团结,以改善官兵关系。第六,实行公买公卖,待人和气,不强迫征粮,不强迫拉夫,不强迫当兵,改取政治动员方式解决食粮、夫役与新兵问题,以改善军民关系。第七,在前线各军之间,前线与后方各军之间,提高友爱互助精神,作战则互相策应,工作则互相观摩,消除互相观望互相嫉忌等不良现象,以改善各军之间的关系。第八,整理现有军队,补充缺额,同时增编新的军队,加紧教育训练,以利持久作战。用这一切办法,提高主力军的战斗力,为保卫华中华南与西北而战,为停止敌之进攻与准备我之反攻而奋斗。

(4)广大地发展敌后游击战争,创立和巩固我之根据地,缩小敌之占领地,配合主力军作战

针对着敌之目的在于还要继续向我进攻,又将于一定时机抽兵进攻游击战争,企图巩固其占领地,使中国反攻困难,全氏族的第四个任务,就在于广大地发展敌人后方的游击战争,创立许多根据地,巩固已有的根据地,用以缩小敌之占领地,在目前,配合主力军为停止敌之进攻而战,在将来,配合主力军为实行反攻而战。半殖民地中国抗日民族战争的重要特点之一,在于游击战争的广大胜与长期性,没有这种游击战争,便不能牵制大量敌军,有力地配合正面主力军之作战,而停止敌之进攻;便不能使敌人占领地限制于一定地带,使之无法全部占领;便不能在敌人后方建立多数的抗日堡垒,坚持游击战争,拜准备将来配合主力军之战略反攻。因此,第一,必须广大地发展一切敌人后方地带的游击战争,并创立多数的游击战争根据地,巩固己经建立起来的根据地。第二,必须依照华北榜样,留置或派遣足够数量之正规军队于敌后各个战略区域,作为长期坚持游击战争的骨干。这些军队应该逐渐学会游击战术,加强政治工作,发展民众运动,创立根据地,开帮助敌后民众游击队与游击战争逐渐提高到正规军与正规战争的道路上去。第三,一切战区与敌人后方,必须发动所有男女人民卫国保乡的热忱,除动员他们大批加入脱离生产的游击队与补充留置敌后的正规军外,把他们组织到半军事性质的抗日人民自卫队中去。抗日人民自卫队的组织,应成为一切战区与敌人后方的普遍与经常的民兵制度,他们是不脱离生产的。第四,必须协助人民组织广泛的游击队。这是脱离生产的,各县各区都应该有,成为袭击敌人保卫地方的普遍的小队伍。第五,必须建立游击部队中的政治工作,加强其军事政治文化娱乐的教育,用以提高其战斗力。第六,必须建立游击部队中正确的军事政冶制度,实现官兵的乎等待遇,经济公开。第七,改造土匪部队,便他们走上抗日;肃清我军后方的及被敌利用的土匪。第八,游击战争的军火接济是一个极重要问题,一方面,大后方尽可能的接济他们:又一方面,每个游击战争根据地都必须尽量设法建立小的兵工厂,办到自制弹药、步枪、手榴弹等的程度;使游击战争无军火缺乏之虞。第九,依照敌情与我之战略需要,重新划分散后各地作战区域与行政区域,使之适合新的战争情况。第十,必须依照战略需要,统一敌后各部队与行政区之领导,以便集中抗敌力量,消除内部矛盾;但应反对互相吞并的军阀行为。

(5)提高军事技术,创立机械化兵团,准备反攻实力

敌以不及我数之兵力而能节节深入者,除了我之政治原因外,我之技术落后是主要原因。针对着敌之长处与我之短处,全民族的第五个任务,在于提高军事技术,增加飞机大炮战车等之数且与使用人材,为着实行反攻而准备实力。为此目的,一方面须就现有的及可能继续增加的制造能力从部分制造与修理开始,认真从事这个工作。另一方面,多方设法从外国输入新式武器,用以逐渐改各军队的装备,创立真正现代化的机械兵团。毫无疑义,我们应该从实际看问题,在现在,实际上战斗着的是大数量的低级武器,因此,我们应该号召全国军队与人民武装,相信低级武器也能胜敌,提高政治精神,改善作战方法,发展游击战争,以补新式技术之不足。不在这方面着重致力,我们就忽视了当前的实际问题,无以克服目前的困难。在将来。为着准备战略反攻,非提高新式技术建设新式军队不可,须知没有现代新式技术装备的足够数量的军队,要实行反攻。收复失地是不可能的。不在这方面提高注意力,拜就可能范围内认真开始去做,我们就只看见现在,忘记了将来,无以克服前途的困难。在人力物力丰富的中国,只要政治条件改善,动员方法进步,加之以外国的协助,逐渐改善技术装备,决不是不可能的。

(6)实行集中领导下的民主政治,密切政府与人民的联系,发挥抗日政权的最大能力

敌人乘我弱点之处,不但在军事.而且在政治,在我政治制度之不民主化,不能与广大人民发生密切的联系。为补救此弱点,全民族的第六个任务,在于实行集中领导下的民主制度,没有这一方面的改进,要最后战胜日寇也是不可能的。民主政治是发动全民族一切生动力量的推进机,有了这种制度,全国人民的抗日积极性将会不可计量地发动起来,成为取之不尽用之不竭的深厚渊源。我全民族彻底地统一团结的伟大过程之完成,也只有依靠民主制度之建立。关于这一点,须从各方面实际有所设施。第一,国民参政会的开会已开始了国家民主化的第一步.尔后应使该会工作公开的顺利的进行,该会议决事项碰全部付之实施,并依据该会已经决定的万案认真的建立各省各级地方参政会,推进民主政治。第二,保证抗战建国纲领所规定的人民言论、出版、集会、结社、信仰等自由权在全国范围之充分实施。这种自由是在抗城建国范围之内的,只有充分地保证了这种自由,才利于普遍发挥抗日建国的力量。这里问题是保证中夫法令在各地方之实施而不受地方之随意限制。应该限制的只是不利于抗日建国的那种自由,即汉奸、亲日派的自由,其它都不应在限制之列。第三,从战区与敌人后方开始实行多量的民主制。例如:民选各级地方政府再由上级加以委任。战区各级政府采用民主集权的委员制,拜设立各级人民代表机关。战区政府增设某些必要的工作部门;改变公文程序;清除贪污腐化无能分子,吸收抗日积极分子;减低薪俸,提倡艰苦生活;用以适合战区艰苦、复杂与流动的环境。战区地方政府在中央政府统一领导下,有颁布地方单行法令之汉。战区男女公民,除汉奸外,均有选举被选举汉,均有言论、出版、集会、结社与武装抗日之自由。战区一切抗日党派均有公开合法地位,等等。在战区尤其在敌人后方,没有这些敌治改革,要支持长期艰苦的抗日战争是不可能的。以上这些办法,都是为着密切政府巧人氏的联系,增加政行的实力,使之能在抗日战争中起其最大的作用。没有问题,全国任何地方政府,应集中于中央政府领导之下,不应因行政区域在地域上之被散分割而有任何不尊重中央领导的表现。全国必须是依照中央法令而推行民主制的。但全国必须是统一于中央的。

(7)扩大统一的民众运动,全力援助战争

全民族的第七个任务,在于扩大各种民众运动,并使之统一起来,全力援助战争。长期艰苦的抗日战争,一切须取给于民众,没有普遍发展的并全国统一的民众运动,要长期支持战争是不可能的。尤其在战区与敌人后方,极须这样做。抗日战争正在遇到新的困难,唯有动员民众,才能有效地克服这些困难。在全国,尤其在战区与敌人后方,极应做到下列各项:第一,保障一切抗日民众团体与抗日民众运动的自由,确立民众团体在法律上的地位。第二,物质上帮助民众团体,尊重民众团体的独立性。第三,认真建立有广大群众参加的工人、农民、青年、妇女、商人、自由职业者、文化人与儿童的各种救国会,并使之依照地域与职业两种原则建立联各的组织。第四,发动民众积极参加各方面的抗战工作,积极援助政府与军队,尤其在战区不可或缓。

(8)改良民众生活,激发民众的抗战热忱与生产热忱

改良民众生活问题,过去实行的人微弱了,因此不能激发广大劳动人民的抗战热忱与生产热忱,对于坚持长期战争是非常不利的。因此,今后全民族的第八个任务,在于实行下列各项改良民众生活的政策。第一,优待抗日军人家属与残废的抗日军人。第二,救济战区灾民难民及失业工人。第三,在战区及敌后开始适当的减租减息。第四,调剂粮食及重要的日常必需品。第五,适当的增加工资,改善工人职员的待遇。第六,承认工人农民对雇主地主的团体契约极。第七,禁止雇主、地主、师父、工关等对工人学徒的虐待打骂。实行这些初步的生活改良办法之后。必能提高工农贫民群众拥护政府,参加战争与参加生产的积极性,不但战争需要的一切动员帮助将大大改观,而且工业农业生产的数量质量与商业的流通也会大大增加与提高起来,国家财政也就在新的农工商业基础之上而得到满意的解决。

(9)实行新的战时财政经济政策,渡过战争难关

主要的大城市与交通线丧失之后,国家财政经济必大现困难,没有新的有效的办法,便无以渡过战争的难关。然而只要实行新的政策,动员人民力量,便任何困难也能够克服。因此,全民族的第九个任务,在于实行一种新的战时财政经济政策。主要事项如下:第一,新政策以保障抗日武装部队一切必要供给,满足人民必需品的要求,拜和敌人的经济封锁与经济破环作斗争为目的。第二,有计划的在内地重新建立国防工业,从小规模的急需的部门开始,逐渐发展改进;吸收政府、民间与外国三方面的资力;并从政治上动员工人,保障其最低限度的物质待遇,改良工厂管理制度,以提高生产率。这些,不但是必需的,而且是可能的。第三,用政治动员与政府法令相配合,发展全国农业与手工业生产,组织春耕秋收运动,便全国农业手工业在新的姿态下发展起来。在战区注意保护农具牲畜及手工作坊。保证被隔断区域的经济自给。第四,保护私人工商业的自由营业,同时,注意发展合作事业。第五,在有钱出钱原则下,改订各种旧税为统一的累进税,取消苛杂和摊派制度,以舒民力而利税收。第六,用政治动员与政府法令相配合,征募救国公债、救国公粮,拜发动人民自动捐助经费及粮食,供给作战军队,以充实财政收入。第七,有计划的与敌人发行伪币及破坏法币的政策作斗争,允许被隔断区域设立地方银行,发行地方纸币。第八,厉行廉洁运动,改订薪饷办法,按照最低生活标准规定大体上平等的薪馅制度。第九,由国家银行办理低利藉贷,协助生产事业的发展及商品的流通。第十,恢复与发展战区的邮电交通。以上所指,不过大端,必须有认真改革旧制实行新制的决心,并持之以效力,才能消除新的困难,支持长期战争,其重小在于组织广大人民的生产积极性,使之为着战争供给而效力。中国的抗战是在一种特殊情况之下进行的,主要的大城市与交通线被敌占领,抗战的主要依靠是乡村与农民。农民是有伟大力量支持战争的,但须实行必要的政治方面与经济方面的改革。这里所说各项新政策,就是根据这种特殊情况而提出的。

(10)实行抗战教育政策,便教育为长期战争服务

在一切为着战争的原则下,一切文化教育事业均应使之适合战争的需要,因此全民族的第十个任务,在于实行如下各项的文化教育政策。第一,改订学制,废除不急需与不必要的课程.改变管理制度,以教授战争所必需之课程及发扬学生的学日积极性为原则。第二,创设并扩大增强各种干部学校,培养大批的抗日干部。第三.广泛发展民众教育,组织各种补习学校,识字运动,戏剧运动,歌咏运动,体育运动,创办敌前敌后各种地方通俗报纸,提高人民的民族文化与民族觉悟。第四,办理义务的小学教育,以民族精神教育新后代。一切这些,也必须拿政治上动员民力与政府的法令相配合,主要的在于发动人民自己教育自己,而政府给以恰当的指导与调整,给以可能的物质帮助,单靠政府用有限财力办的几个学校、报纸等等,是不足完成提高民族文化与民族觉悟之伟大任务的。抗战以来,教育制度已在变化中,尤其战区有了显著的改进。但至今还没有整个制度适应抗战需要的变化,这种情形是不好的。伟大的抗战必须有伟大的抗战教育运动与之相配合,二者间的不配合现象亟应免除。

(11)力争国际援助,集中反对日本帝国主义

从长期战争与集中反对日本帝国主义的原则出发,组织一切可能的外援,是不可忽视的。因此,当前的第十一个任务,在于第一,坚决反对一部分人所谓走德意路线的主张,因为这实际上是一种准备对敌投降的步骤。第二,力争各民主国家与苏联对我物质援助之增加,同时尽力促成各国实行国联制裁日本之决议。第三,设立一定机关,系统的收集一切敌军暴行制成具体的文书、报告,宣扬国外,唤起全世界注意,起来惩罚日本法西斯。第四,从各党派各人民团体推出代表,组织国际宣传团体,周游列国,唤起各国人民与政府的对我同情,与我国政府的外交活动相配合。第五,保护一切同情国家在中国的侨民及其和平通商传教等事业。第六,注意保护华侨利益,并经过华侨的努力推进各国反日援华运动。一切这些,不管各国助我之程度如何暂时的没有增加或甚至可能部分的减少,国联决议可能依然是一句好听的话,我们都应努力的做。根据抗战的长期性,外交方针也应着眼于长期,不重在眼前的利益,而重在将来的增援,这一点远见是必要的。

(12)建立中国与日本兵民及朝鲜、台湾等被压迫民族的反侵略统一战线,共同反对日本帝国主义

日本帝国主义的侵略战争,不但是危害中华民族的,同时也是危害日本全体兵民与朝鲜、台湾等被压迫民族的,要使日本的侵略战争失败下去,必须中日两大民族的军民大众及朝鲜、台湾等被压迫民族作广大而坚持的共同努力,建立共同的反侵略统一战线。为此目的,全民族的第十二个任务,在于:第一,向两国人民士兵大众及朝鲜、台湾民族提出这个反侵略统一战线的方针,号召他们为此而斗争。第二,由政府下令所有抗日军队抗日游击队全体官兵一律学习必要数量与恰当内容的日本话,由高级政治部准备与派出教日本话的教员到各军队中实行施教,从学几句话起到能够同日军官兵讲一篇道理为止,教育全体抗日官兵向全体敌军士兵与下级军官作口头宣传,同时补助之以文字图画宣传,逐渐感化他们,要求他们同意建立共同的反侵略统一战线,使百余万日本侵略军变成我们的友军,退出中国,推翻日本法西斯。第三,尊重与优待敌军俘虏,给以教育,经过他们去影响其余,为建立反侵略统一战线而努力。第四,设法从日本内地组织反侵略的文化人员到中国来参加这一斗争。第五,保护在中国的诚实的日本侨民。第六,教育我国军民大众,一方面提高民族自尊心,又一方面则须纠正军队与人民中的一些错误思想,区别日本帝国主义与日本人民,区别敌军军官与士兵,并区别上级军官与下级军官。实行了上述的方针与办法,付以广大而坚持的努力,这个反侵略统一战线是能够建立起来的。中国的胜利,主要依靠自己力量的增加;但同时,敌人的困难与国际的援助,必须争取其配合。在敌人困难方面,除了因我之坚持长期战争给以逐渐的消耗,努力外交活动使敌日陷于孤立而外,和日本兵民大众及朝鲜、台湾等民族建立其公同反侵略战线的政策,是不可缺少的部分。日本侵略战争愈延长,这一个统一战线便愈有建立的基础。

(13)团结中华各族,一致对日

我们的抗日民族统一战线,不但是国内各个党派各个阶级的,而且是国内各个民族的。针对着敌人已经进行并还将加紧进行分裂我国内各少数民族的诡计,当前的第十三个任务,就在于团结各民族为一体,共同对付日寇。为此目的,必须注意下述各点:第一,允许蒙、回、藏、苗、猺、夷、番各民族与汉族有平等权利,在共同对日原则之下,有自已管理自已事务之权,同时与汉族联合建立统一的国家。第二,各少数民族与汉族杂居的地方,当地政府须设置由当地少数民族的人员组成的委员会,作为省县政府的-部门,管理和他们有关事务,调节各族间的关系,在省县政府委员中应有他们的位置。第三,尊重各少数民族的文化、宗教、习惯、不但不应强迫他们学汉文汉语,而且应赞助他们发展用各族自己言语文字的文化教育。第四,纠正存在着的大汉族主义,提倡汉人用平等态度和各族接触,使日益亲善密切起来,同时禁止任何对他们带侮辱性与轻视性的言语,文字,与行动。上述政策,一方面,各少数民族应自己团结起来争取实现,一方面应由政府自动实施,才能彻底改善国内各族的相互关系,真正达到团结对外之目的,怀柔羁縻的老办法是行不通的了。

(14)厉行锄奸运动,巩固前线与后方

新的形势下,汉奸,敌探,托派,亲日派必然较前更加猖獗,大肆其造谣,污蔑,分裂,破坏的阴谋,因此当前的第十四个任务,在于实现下列办法,厉行锄奸运动。第一,唤起前线与后方一切军民人等的警觉性,严密注视汉奸,敌探,托派,亲日派之活动,依照政府法令,毫不容情的镇压之。第二,注意保护国家机密,以叛国罪惩办泄漏机密之叛徒。第三,学校教科书中加进锄奸一课,实施提高警觉性的教育。第四,军队中设置各级管理锄奸工作之部门,民众团体中人民自卫队中设置锄奸小组,国家警察加重锄奸教育,使奸徒在众目集视下无法藏身。抗战以来,吃这些奸徒们的亏真是不可计量的了。前线的将士,惊叹汉奸之多与损害作战利益之大,早已异口同声。即在后方,单是泄漏国家机密与引导敌机惨炸二事办已天人共愤。长期抗战中如不肃清奸徒,将不能设想战争的胜利,发动广大民众之民族革命的警觉性,厉行上述锄奸办法,并使之成为广泛的运动,是争取胜利不可缺少的严重的任务。应该指出:锄奸运动应注意区别首要与胁从,自觉的与被骗的,坚决分子与动摇分子,分别处理,前者从重,后者从轻,并注意争取后者使之回心向善,决不可一例看待。还须注意确实证据,勿用刑讯,严防诬陷。锄奸目的在肃清真正奸徒:只有用正确政策与正确方法,才能达到目的。

(15)发展国共两党及各抗日党派,强固统一战线,支持长期战争

所有前述各项紧急任务,有待于抗日民族统一战线各党派推动全民族,在蒋委员长统一领导之下,坚决的实行起来,而欲达到此目的,非发展统一战线中各个党派的组织力量不可。现有力量,无论何党都太小,都需发展,而大大发展国共两党尤为当前的紧急任务。在这个发展的任务中,各党均应互相赞助他党的发展,而不可互相嫉忌与互相妨碍。须知只要是抗日党派,任何一党的发展,都于抗日有利。没有问题,统一战线以国共两党为基础,而两党中又以国民党为主干,我们承认这个事实。因此,我们是坚决拥护蒋委员长及其领导下之国民政府与国民党的,并号召全国一致拥护。承认与拥护这个主干而又同时发展各党,是互相联系并不互相冲突的。

在数量上,我以为国民党应发展至五百万以上,共产党及其它党派应发展至一百万以上,在一个四万万五千万人口的大民族中,当着伟大抗战时代,吸引数百万优秀分子加人各抗日党派,不但是必需的,而且是完全可能的。诚能如此,则抗日民族统一战线就扩大了,随之也将更巩固了,执行战胜敌人的一切任务就有了充分的保证,支持长期战争与长期合作,驱除日寇与建设三民主义新中国的根本目的就不患不能达到。

五、长期战争与长期合作

现在,我们专就抗日民族统一战线的长期性问题来讨论一番,向着异常关心国共两党关系的人们所已经发生了的许多疑问,作一个全盘的答复,这一点,对于巩固与扩大抗日民族统一战线,巩固与扩大国共合作,顺利地执行当前紧急任务,渡过战争的难关,是有重要意义的。

问题有如下各点:战争的长期性决定合作的长期性,战争中的合作决定战争后的合作,长期合作的内容与主要条件,三民主义与共产主义,长期合作的组织形式,长期合作中的互助互让政策,民主共和国问题。这些,都是很多人所关心的,我们都得明确的给以答复。

(1)战争的长期性决定合作的长期性

由于抗日战争是长期的,整个抗日民族统一战线也能够且必须是长期的,其中主要的两个党——国民党与共产党的合作,也能够且必须是长期的,这是一切政策的出发点。因此,我们的政策,无论如何要一个长期的民族统一战线,要一个长期合作;无论如何要共同维持一个统一政府,反对纷歧与分裂,方才有利于渡过战争难关,对抗敌人破坏,打退日本帝国主义,并于战后完成建立新中国的任务。这是和一九二四至一九二七年的国共合作根本不同的,那次是短期的,这次是长期的。

(2)战争中的合作决定战争后的合作

所谓长期合作,不但是在战争中的,而且是在战争后的。抗日战争是长期的,战争中的合作已经算得是长期的了。但是还不够,我们希望继续合作下去,也一定要继续合作下去。这有什么保证呢?保证就在:战争中的合作决定着战争后的合作。抗日民族统一战线中主要的国共两党,必须同患难,共生死,力求进步,并经过长期的努力,才能打退日本帝国主义,否则不能。战争之后,这样长期同过患难的有了进步的两个党,就造成了继续合作的基础。那时的国内国际条件将更有利于合作,也是现在想得到的。没有疑义,战争中的合作必有其各个合作阶段的内容,战争后的合作将更有新的内容。然而战争中的合作,将决定着战争后也能够合作,这不是没有根据的预断。

(3)长期合作的内容与主要条件

所谓长期合作就是长期的民族统一战线,所有阶级,从资本家到工人,所有政党,从国民党到共产党,所有民族,从汉族到苗猺弱小民族,所有军队,从中央军到八路军,所有政府,从国民政府到陕甘宁边区政府,只有民族叛徒除外,一切都在内,而且是长期在内的。民族统一战线内,有些人在长期战争中,当着熬不过艰苦斗争,个人利益超过民族利益时,会要变为民族叛徒的,因此民族统一战线是要不断地把这些民族叛徒们除外的。但这些除外,依然是民族统一战线。其理由,即长期合作的主要条件,首先是敌人战争的野蛮性与长期性。由于敌人战争的野蛮性,严重地危害着全民族各个阶层的生存,这样就迫使上层阶级也不得不与其它阶级一道抗日。上层阶级中一部分是难免退出抗日战线的,但其它部分和其它阶级大体一样,是受压迫的,不反抗便无出路。又由于这种野蛮性的战争是长期的,就决定了合作是长期的。这些是决定长期合作的一方面。但是还有第二方面,要合作中的各党,首先是国共两党,采取正确的政策,进行必要的工作。什么样的政策与工作呢?应该是从长期战争与长期合作的基点出发而规定出来与实行起来的政策与工作。应该是照顾现在又照顾将来,照顾这一阶级又照顾那一阶级,照顾这一党派又照顾那一党派,照顾这一军队又照顾那一军队,照顾这一民族又照顾那一民族的政策与工作。否则政策不对,工作不行,自乱步骤,将使合作难于持久。这样,一方面,敌人战争的野蛮性与长期性,又一方面,统一战线中的正确政策与必要工作,就使中国的民族统一战线不但应该是长期的,而且能够是长期的。是民族战线,不是人民阵线。是包括战争中与战争后的国共合作,不是企图在战争后又分裂又内战的国共合作。

(4)三民主义与共产主义

三民主义是抗日民族统一战线与国共合作的政治基础,但是三民主义与共产主义的关系如何呢?共产党员对三民主义应取何种态度呢?直至现在还有一些人不清楚,因此有再一次解释的必要。还在一九三六年五月间开的我们党的临时性的代表大会上,就通过了如下的关于“坚决实行三民主义”的提纲:

“共产党是否同意三民主义?我们的答复是同意的。三民主义有它的历史变化。孙中山先生的革命的三民主义,曾经因为同共产党合作与坚决执行而取得人民的信仰,发动了一九二五──二七年的胜利的大革命。又曾经因为排斥共产党(清党运动),实行相反的政策,而失去人民的信仰,招致革命的失败。现在则因民族危机与社会危机极端严重,全国人民与国民党中爱国分子,因而有两党合作的迫切要求。因此重新整顿三民主义的精神,在对外独立解放的民族主义,对内民主自由的民权主义,与增进人民幸福的民生主义之下,两党重新合作,并领导人民坚决的实行起来,是完全适合于中国革命的历史要求,而应为每个共产党员所明白认识的。共产党决不抛弃其社会主义与共产主义理想,他们将经过资产阶级民主革命阶段达到社会主义与共产主义的阶段。共产党有自己的党纲与政纲,其党纲是社会主义与共产主义,这是与三民主义有区别的。其民主革命政纲,亦比国内任何党派为彻底,但对于国民党第一次及第二次代表大会所宣布的三民主义纲领,则是基不上没有冲突的。因此我们不但不拒绝三民主义,而且愿意坚决实行三民主义,而且要求国民党同我们一道实行三民主义,而且号召全国人民实行三民主义,使国民党,共产党,全国人民,共同一致为民族独立,民权自由,民生幸福这三大目标而奋斗。”(“中国抗日民族统一战线在目前阶段的任务”,第十一页)

去年九月二十二日,我们党的中央为公布国共合作成立的宣言中,又着重地说到:“孙中山先生的三民主义为中国今日之必需,本党愿为其彻底实现而奋斗。”

一个共产主义的政党为什么采取这种态度呢?很明显的,民族独立,民权自由,与民生幸福,正是共产党在民族民主革命阶段所要求实现的总目标,也是全国人民要求实现的总目标,并非某一党派单独要求的东西。只要看一看从共产党诞生以来的文献,它的政治纲领,就会明白。因此,在过去,不但在一九二四至二七年国共两党第一次合作时期,我们共产党员曾经坚决实行了三民主义。就在一九二七年两党合作不幸破裂之后,我们的一切做法,也没有违背三民主义。那时,我们坚决地反对帝国主义,这是符合于民族主义的:我们实行了人民代表会议的政治制度,这是符合于民权主义的:我们又实行了耕者有其田的土地制度,这是符合于民生主义的。那时,我们的一切做法,并未超过资产阶级民主革命基本范畴的私有财产制。在现在抗战的阶段与战后彻底完成民主共和国的阶段,都是三民主义的阶段,都是资产阶级民主革命性质的阶段。为了彻底完成这个民主阶段的任务,一切共产党员,毫无疑义,应该依照自己的一贯的革命总方针,自己的决议与宣言,同中国国民党与全国其它党派,全国广大人民一道,诚心诚意的实行三民主义。谁要是不忠实于三民主义的信奉与实行,谁就是口是心非,表里不一,谁就不是一个忠实的马克思主义者。在中国,任何忠实的马克思主义者,他是同时具有现时实际任务与将来远大理想两种责任的。并且应该懂得:只有现时的实际任务获得尽可能彻底的完成,才能有根据有基础地发展到将来的远大理想那个阶段去。所谓将来的远大理想,就是共产主义,这是人类最美满的社会制度,孙中山先生也曾经认为必要实行它,才能解决将来的社会问题。所谓现在的实际任务,就是三民主义,这是“求国际地位平等,求政治地位平等,求经济地位平等”的现阶段的基本任务,是国共两党与全国人民的共同要求。因此,共产党员应该如象他们研究共产主义一样,好好研究三民主义,用马克思主义的眼光,研究三民主义的理论,研究如何使三民主义具体地见之实施,研究如何用正确的三民主义思想教育人民大众,使之由了解而变为积极行动,为打退日本帝国主义,建设三民主义新中国而斗争。

(5)长期合作的组织形式

为了保证长期合作,还要解决合作的组织形式问题,我们曾经批驳了一党主义,不论是对于过去历史上说,对于当前任务上说,对于中国社会性质上说,所谓一党主义都是没有根据的,都是做不到的,行不通的,违背一致团结抗日建国的大目标,有百害而无一利的。那么,各党共存,而互相结合为一个抗日民族统一战线,要不要一种统一的共同的组织呢?要的,必要的,没有这种统一的共同的组织,不利于团结抗日,更不利于长期合作。因此,各党应该认真研究,找到一种最适合于长期合作的统一的共同的组织形式。现在我们就来研究一下。

由于中国政治经济及各党派的历史特点,今天看来,抗日民族统一战线可能有下述三种组织形式。

第一种,国民党本身变为民族联盟,各党派加入国民党而又保存其独立性,但与第一次国共合作不同。如果国民党同意共产党员加入,我们将取何种态度呢?首先,我们是赞成这种办法的,因为这是抗日民族统一战线最好的一种统一组织形式,有利于抗日建国。不但共产党,任何其它抗日党派都可加人国民党,只要国民党同意,我们是决不反对的。如果这样做,那我们可以实行同十三年合作不相同的办法,即第一,所有加入国民党的共产党员都是公开的,将加入党员之名单提交国民党的领导机关。第二,不招收任何国民党员加入共产党,有要求加入的,劝他们顾全大局,不要加入。第三,如果我们的青年党员得到国民党同意,加入三民主义青年团的话,也是一样,不组秘密党团,不收非共产党员入党。用这种办法,可以大家相安,有利无害。这是第一种统一战线的组织形式。

第二种统一战线的组织形式,就是各党共同组织民族联盟,拥戴蒋介石先生作这个联盟的最高领袖,各党以平等形式互派代表组织中央以至地方的各级共同委员会,为着执行共同纲领处理共同事务而努力。这也是一种很好的形式,我们也是赞成的。这种形式,我们很早就提议了,可惜还没有实行。

第三种统一战线的组织形式,就是现在的办法,没有成文,不要固定,遇事协商,解决两党有关之问题。但这种形式太不密切,许多问题不能恰当的及时的得到解决。例如许多大政方针之推行,下级磨擦问题之调整,都因没有一种固定组织,让它延缓下去,所以这种办法对于长期合作是不利的。然而如果第一二种办法不行,这种办法暂时也只得仍之。

总之,长期战争中的长期合作,组织形式问题也是一个重要问题。我们极力赞成有一种统一的形式,使之利于长期合作。

(6)长期合作中的互助互让政策

长期战争需要长期的统一战线,前已说过,这是一切政策的出发点。因此,共产党员在其工作中,在其同友党发生关系中,随时随地都要顾到这个长期性。凡于长期合作有利的,应该坚决的勇敢的做,不利的,则决不应做。

这里就发生各党之间互助互让的问题。说互助,例如各党都要发展,都要巩固,任何一党除了发展与巩固自己之外,还应对友党的发展与巩固取赞助态度。国民党的发展与巩固,共产党员应取何种态度呢?一句话,赞助之。其理由是国民党的发展巩固利于抗日战争,利于全民族,因而也利于劳动人民与共产党,我在前面已经说过了。现在国民党组织三民主义青年团,共产党员应取什么态度呢?没有问题,取赞助态度。我们希望三民主义青年团有广大的发展,依照蒋介石先生关于三民主义青年团的宣言做去,该团的发展是有光明前途的。也正是为着赞助,我们对于该团现行办法中之某些事项,希望有所修正,不然,好的动机,将难得好的结果。三民主义青年团应该成为全国广大青年群众团结救国的统一组织,吸收各党各派各界的青年个人与青年团体加入进去,成为使整个青年一代集体地受到民族革命的教育训练之一个大集团。因此,组织上应该民主化,政治上应该发扬团员的自动自觉精神,发扬青年群众的积极性。这是我们对于三民主义青年团的态度与意见。

互助就不是互害,损人利己,在个人道德是不对的,在民族道德更加不对。因此,无理的磨擦,甚至捉人杀人等事,无论如何是要不得的:共产党员决不应该以此对待友党。而如若友党以此对待我们时,我们也决不各置之不理。凡无理的事必须以严正态度对待之,才是待己待人的正道。互相规过,是朋友间的美德,也是政党间应该提倡的作风。

统一战线中有什么互让呢?有的。我们曾经在政治上作过一些让步,那就是停止没收土地,改编红军,改变苏区制度,这是一种政治上的让步,这是为了建立统一战线团结全民共同对敌的必要步骤。我们的友党也作了让步,那就是承认共产党的合法地位等等。这种为了团结抗日为了长期合作的互让政策,是很好的,很对的。只有政治上胡涂或别有用心的人:才会说:共产党投降了国民党,或国民党投降了共产党。

现在我们又主张所有各统一战线中的党派,互不在对方内部招收党员,组织支部,进行秘密活动。我们认为这种政策是必要的。现在当然和过去不同,在过去内战时期,国共两党间除了公开的战争之外,还互相使用秘密手段,进行破坏对方的活动。合作以后,当然不应有互相破坏的动机与行为了,但是互相在对方内部秘密招收党员组织支部的办法,也应该停止,使彼此安心,才能适合于长期战争中长期合作之目的。我们现在正式向国民党同志申明:我们停止在你们内部作招收党员组织支部的活动,不管统一战线采取何种的共同组织形式,我们都是这样做。但同时,也希望你们这样做。双方约定之后:下级党员如有违背,由违背之一方的上级负责处理。

此外,双方同志接触,应采谦和,尊敬,商量态度,不采傲慢,轻视,独断态度,以改善双方之关系,这也是必要的。

一切我们所说的,共产党员应该首先实行,不管对方某些人员或尚未用同样的政策,方法,态度对待我们,但我们仍然这样做,做的久了,对方某些一时尚未明白的人员也会明白了。

共产党员对于一切为国为民的事业,应该坚持自己的立场,始终不变地向着战胜日寇建立新中国的方向走去,谁要违背了这种立场,这个方向,谁就丧失了共产党员的资格。但共产党员又必须有互助互让的精神,必须有尊重友党及和友党同志用谦和商量态度解决问题的精神,一切有友党同志的地方,都应和他们商量解决有关事项,不应独断。没有这种精神,就不能巩固统一团结,所谓为国为民事业,战胜日寇建立新中国之目的,也就达不到。因此,决不能把必要的互让政策解释为消极行为。不但互助是积极的,互让也是积极的,因为必要的让步,是巩固两党合作求得更好的团结与更大的进步之不可缺少的条件。

(7)民主共和国问题

虽然我们的党还在一九三六年的九月间,就公布了关于建立民主共和国的决议案,虽然中央同志曾经多次的说明过这个问题。但外间对于我们的主张仍有许多不明白的。这是一个关于抗战前途的问题。抗战的结果将怎么样呢?所谓抗战建国,照共产党的意思,究将建立一个什么国呢?这是存在着的问题。再一次解释这个问题,对于巩固各党各派长期合作的信心,是有利益的。

建立一个什么国呢?一句话答复:建立一个三民主义共和国。

我们所谓民主共和国就是三民主义共和国,它的性质是三民主义的。按照孙中山先生的说法,就是一个“求国际地位平等,求政治地位平等,求经济地位平等”的国家。第一,这个国家是一个民族主义的国家。它是一个独立国,它不受任何外国干涉,同时也不去干涉任何外国。即是说,改变中国原来的半殖民地地位,它独立起来了;但同时,无论它强盛到什么程度,决不把自己变为帝国主义,而是以平等精神同一切尊重中国独立的友邦和平往来,共存互惠。对国内各民族,给予平等权利,而在自愿原则下互相团结,建立统一的政府。第二,这个国家是一个民权主义的国家。国内人民,政治地位一律平等;各级官吏是民选的:政治制度是民主集中制;设立人民代表会议的国会与地方议会;凡十八岁以上的公民,除犯罪者外,不分阶级、男女、民族、信仰与文化程度,都有选举与被选举权。国家给予人民以言论,出版,集会,结社,信仰,居住,迁徒之自由,并在政治上物质上保护之。第三,这个国家是一个民生主义的国家。它不否认私有财产制。但须使工人有工作,并改良劳动条件。农民有土地,并废除苛捐杂税重租重利。学生有书读,并保证贫苦者入学。其它各界都有事做,能够充分发挥其天才。一句话,使人人有衣穿,有饭吃,有书读,有事做。我们所谓民主共和国,就是这样一种国家,就是真正三民主义的中华民国。不是苏维埃,也不是社会主义。

中国要变为这样一个国家,要同谁作斗争呢?要同日本帝国主义作斗争。日本帝国主义剥夺我们的独立,我们就要向他要独立。日本帝国主义把我们当奴隶,我们就要向他要自由。日本帝国主义使我们陷入饥寒交迫,我们就要向他要饭吃。怎样要法?用枪口向他要。一句话,赶走日本帝国主义,就有一个独立自由幸福的三民主义新中华民国。

六、中国的反侵略战争与世界的反法西斯运动

(1)中国与世界不可分

中国已紧密地与世界联成一体,中日战争是世界战争的一部分,中国抗日战争的胜利不能离开世界而孤立起来。新的抗战形势中可能暂时地减少一部分外国的援助,加重了中国自力更生的意义,中国无论何时也应以自力更生为基本立脚点。但中国不是孤立也不能孤立,中国与世界紧密联系的事实,也是我们的立脚点,而且必须成为我们的立脚点。我们不是也不能是闭关主义者,中国早已不能闭关,现在更是一个世界性的帝国主义用战争闯进全中国来,全中国人都关心世界与中国的关系,尤其关心目前欧洲时局的变动。所以我们来分析一下当前的国际形势,是有意义的。

(2)重新分割世界的第二次世界大战已经开始

资本帝国主义的本性,不但是和本国人民大众矛盾的,是和殖民地半殖民地矛盾的,是和社会主义国家矛盾的,而且是帝国主义诸国之间自相矛盾的。这最后一种矛盾在历史上的最尖锐表现,就是二十年前的世界大战。那次两组帝国主义互战的结果,产生了新的国际形势。战后世界政治经济新的发展的结果,使得世界又临到新的大战面前。在东方日寇侵略东四省西方希特勒登台之后,新的重分世界的战争业已开始了。“法西斯主义就是战争”,一点也不错,在此情势下,一方面日德意组成了侵略阵线,实行大规模的侵略。另方面各民主国家却为保守已得利益而在和平的名义之下准备战争;但至今不愿用实力制裁侵略者,尤其是英国的妥协政策实际上帮助了侵略者。在这种情况下,中国东四省首先被牺牲,接着亚比西尼亚亡于意大利,西班牙则助长了叛军的气焰,中国又受到日寇新的大规模的侵略,到最近,奥国与捷克又先后牺牲于希特勒。全世界已有六万万人口进入了战争,范围普及到亚、非、欧三洲,这就是新的世界战争的现时状况。

(3)现时世界战争的特点

由于一方面日德意诸法西斯国家的坚决的侵略意志,又一方面各民主国家不愿实力制裁尤其是英国的妥协政策,使得新的世界战争的现时状态表现了和第一次世界大战的不同特点,这就是首先侵略中间国家与采取各种不同的战争形式。中国、亚比西尼亚、西班牙、奥大利、捷克等国,都是半独立国家或小国,日德意诸国就拣了这些肥肉先行吞蚀。在侵略这些中间国家中,侵略者采取了三种特殊的战争形式。第一种是日本对中国,意大利对亚比西尼亚的战争,这是公开的直接的大规模的战争,但是在不宣而战的形式下进行的,开了战争史上的新纪元。采取这种不宣而战政策的目的,在于侵略者利用各民主国家的无意制裁尤其是英国的妥协政策,暂时避免和它们的直接冲突,便利其先夺取中间国家的行动。第二种是意德两国侵略西班牙的方式,采取了援助叛军的办法,这是历史上老办法的重演,历史上这类办法是有过的。第三种是希特勒侵略奥捷两国的方式,这里没有战争的表面(没有打响),但有战争的实际,出动了强大兵力占领奥国全部与捷克一部,并使捷克余部归属其统制,这是不战而亡人国的最巧妙的办法。这三种战争形式的采用,都是由于一方面,侵略国本身力量还不充足,暂时未便和各大国直接作战,因而采取了巧妙的战争方法,企图使自己先行壮大起来,同时即是使各大国削弱起来,再与各大国作战。又一方面,则是各民主国家不愿制裁侵略者,尤其是英国的怯儒妥协政策的结果,这种政策实际上援助了侵略者,便利其侵略各中间国家。

(4)英国妥协政策将引导法西斯各国实行更大规模的战争

以张伯伦为首的英国保守党内阁,正在逐步进行其所谓四强合作的政策,慕尼黑协议之后,欧洲政局有暂时逆转的可能。英国大部分保守党的政策,历来是以排斥苏联妥协德意为原则的,由于他们畏惧苏联的强盛,畏惧自己过早卷入战争,畏惧本国人民运动与殖民地独立运动,早已决心牺牲西班牙、奥国、捷克等国,成就其排斥苏联妥协德意的企图。过去因为保守党内部的不统一,法国人民阵线的积极政策,国内国际舆论的责备,而没有成功。现在则利用了英国及全欧人民不愿战争的心理,利用了法国佛兰亭党的右倾,在希特勒威迫之下,订立了慕尼黑协议。这个协议是英国妥协政策的结果,假如英国不改变它的政策,势将引导法西斯各国进行更大规模的冒险战争。各大国间的战争虽暂时还可能不爆发,暂时限制于侵略中间国家的过程虽还在继续者,但最后势必引导各大国卷入空前残酷的战争里去,这是没有疑义的前途。“搬起石头打自己的脚”,这就是张伯伦政策的必然结果。

(5)全世界多数人类在逐渐动员中

在资本主义各国方面,由于经济的总危机,资本主义已走到毫无出路的地步,六万万人口的战争牵动了全世界,新的更大的战争在威胁全人类。在社会主义国家方面,则一切都是光明的,进步的,强盛的。在这两种相反的对比之下,全世界大多数人类逐渐地找到了如何保卫自己与解放自己的方向,正在用空前的广大性与空前的深刻性逐步地团结自己并准备斗争。第一次世界大战,二十年来社会主义国家的强盛,资本主义国家的衰落,六七年来法西斯国家的侵略战争,中国的伟大抗日战争,西班牙的人民战争,乃至张伯伦的妥协政策等等,逐渐地教育了英法等国与全世界的人民,使他们懂得惟有组织与斗争才是出路,惟有团结世界一切自求解放的人类为一体,惟有世界人民与被压迫民族的统一战线,才有出路。这个全世界人民觉悟,组织,斗争,与统一战线的伟大过程,是在向前发展着,但须经过广大而艰苦的努力才能完成。法西斯的战争威胁与张伯伦的妥协政策,最后将遇到伟大的反抗,这也是没有疑义的前途,也是法西斯战争与张伯伦政策的必然的结果。

(6)中国反侵略战争与世界反法西斯运动的配合

过去的,大家都明白,各民主国家在某种程度上都是援助中国的,主要是其人民的同情中国,苏联的援助则更加积极。现在,由于日寇进攻的深入,又加深了英美法苏对日本的矛盾。虽然英国在西方的妥协政策可能搬用到东方,为了企图多少保存在日本占领区的商业,为了幻想减轻日本对南洋的威胁,英国有可能同日本进行某种程度的妥协,但根本妥协是困难的,至少暂时有困难,这是日本独占政策的结果,东方问题与西方问题在当前具体情况上有某种程度上的区别。日本的深入进攻,进一步加深了日美间的矛盾,苏联与中国的友谊是增长的,中美苏三国有进一步亲近的可能。但是我们第一不可忘记资本主义国家与社会主义国家的区别,第二不可忘记资本主义国家之政府与资本主义国家之人民的区别,第三,更加不可忘记现时与将来的区别,我们对前者不应寄以过高的希望。应该努力争取前者一切可能的援助,在一定程度上不但是可能的,而且是事实,但过高希望则不适宜。中华民族解放运动与外援的配合主要的是和先进国家与全世界广大人民反法西斯运动之将来的配合,以自力更生为主同时不放松争取外援的方针,应该放在这种基点之上。

七、中国共产党在民族战争中的地位

(1) 问题的性质

同志们!我们有一个光明的前途,中国必须战胜日本帝国主义,也能够战胜他。但由现在到达那个光明前途的中间,存在着一段艰难的路程。为着一个光明的中国而斗争的我们与全民族,必须有步骤地同日寇这个黑暗势力作战,而要战胜他,只有经过长期战争。在这个战争中,共产党员处于何种地位呢?他们要怎么样做才算克尽其最善的努力呢?抗战以来的经验,我们也总结了;当前的形势,我们也估计了;全民族的紧急任务,我们也提出了;用长期合作支持长期战争的理论与方法,我们也说明了;国际形势,我们也分析了。那么,还有什么呢?同志们!还有一点,这就是中国共产党在民族战争中处于何种地位的问题,这就是共产党员应该怎样认识自己,加强自己,团结自己,才算在民族战争中尽了自己最大责任的问题。

(2) 爱国主义与国际主义

国际主义者的共产党员,是否可以同时又是一个爱国主义者呢?可以的,应该的,看什么历史条件来决定。有日本侵略者与希特勒的爱国主义,有我们的爱国主义。对于日本侵略者与希特勒,共产党员是坚决反对所谓爱国主义的,日本共产党与德国共产党他们都是战争失败主义者,用一切方法使日本侵略者与希特勒的战争失败,越失败得彻底越好。日本共产党与德国共产党都应该这样干,也正在这样干。理由是:日本侵略者与希特勒的战争,是侵害世界人民也侵害其本国人民的。对于我们,爱国主义与国际主义密切结合着,我们的口号是为保卫祖国反对侵略者而战。对于我们,失败主义是罪恶,全力援助蒋委员长与国民政府是天职,是责无旁贷,在这里,不能有一点消极性。理由是:只有为着保卫祖国而战才能出全民族于水火,只有全民族的解放才能有无产阶级与劳动人民的解放,爱国主义就是国际主义在民族革命战争中的实施。为此理由,每一个共产党员必须发挥其全部的积极性,英勇坚决地走上民族革命战争的战场,拿每一枪口瞄准日本侵略者,不容有任何的消极。必须用全力援助友党友军,不容有任何坐观成败的心理。为此理由,我们的党从九一八事变开始,就提出了用民族自卫战争反抗日本侵略者的口号;后来又提出了与坚持了抗日民族统一战线,命令红军改编为抗日的国民革命军开赴前线作战,命令自己的党员站在抗日战争的设前线,为保卫祖国流最后一滴血。这些做法,这些爱国主义,一切都是正当的,应该的,必须的,正是国际主义在中国的发挥,一点也没有违背国际主义。只有政治上胡涂或别有用心的人,才会闭着眼晴瞎说我们的做法不对,瞎说我们抛弃了国际主义。

(3) 共产党员在民族战争中的模范作用

根据上述理由,共产党员应在民族战争中表现其高度的积极性,而这种积极性,应使之具体表现于各方面,即应在各方面起其先锋的与模范的作用。我们这个战争,是在困难环境之中进行的。这种困难环境的形成,在于我们民族的广大生动力量至今还只在开始发动之中,大多数民众的民族觉悟、民族自尊心与自信心之不足,大多数民众的无组织,军力的不坚强,经济的落后,政治的不民主化,腐败现象与悲观情绪的存在,统一战线内部团结巩固之不足,这些都是形成困难环境的主要原因。为此原故,共产党员不能不自觉地担负起团结全民提高落后的重大责任。在这里,共产党员的先锋作用与模范作用是十分重要的。在八路军与新四军,应该成为英勇作战的模范,执行命令的模范,纪律的模范,政治工作的模范与内部团结统一的模范。共产党员在与友党友军发生关系中,应该坚持统一团结的立场,坚持统一战线的纲领,成为实行抗战任务的模范。应该言必信,行必果,不要傲慢态度,诚心诚意地同友党友军商量问题,协同工作,成为统一战线中各党相互关系的模范。共产党员在政府工作中,应该是十分廉洁,不用私人,多做工作,少取报酬的模范。共产党员在民众运动中,应该是民众的朋友,而不是民众的上司,是晦人不倦的教师,而不是官僚主义的政客。共产党员无论何时何地都不应以个人利益放在第一位,个人利益服从于民族的与群众的利益。因此,自私自利,消极怠工,贪污腐化,风头主义等等,是最可鄙的。而大公无私,积极努力,克己奉公,埋头苦干等等精神,才是值得尊敬的模范。共产党员应与党外一切先进分子协同一致,为着团结全民提高落后而努力。必须懂得,共产党员不过是全民族中一小部分,党外存在着广大先进分子与积极分子,我们必须和他们协同工作。那种“只有自己好,别人都不行”的想法,是完全不对的。共产党员对于落后人们的态度,不是轻视他们,看不起他们,而是尊重他们,亲近他们,团结他们,说服他们:鼓励他们前进。共产党员对于在工作中犯过错误的人们,除了不可救药者外,不是采取排斥态度,而是采取规劝态度,使之翻然改进,弃旧图新。共产党员应是实事求是的模范,又是具有远见卓识的模范。因为只有实事求是,才能完成确定的任务;只有远见卓识,才能不失前进的方向。因此,共产党员又应成为学习的模范,他们每天都是民众的教师,但又每天都是民众的学生,只有向民众学习,向环境学习,向友党友军学习,了解了它们,才能对于工作实事求是,对于前途有远见卓识。长期战争与艰难环境中,只有共产党员协同友党友军与人民大众中之一切先进分子,高度地发挥其先锋的与模范的作用,才能动员全民族一切生动力量,提高落后,克服困难,为战胜敌人、创造新中国而奋斗。

(4) 团结全民族与反对民族阵线中的奸细

克服困难战胜敌人的中心任务,是团结全民族,巩固扩大统一战线,发动全民族各个阶层中的一切生动力量,这是唯一无二的方针。但同时,民族统一战线中已经存在着并且还要混进来起其破坏作用的奸细,这就是暗藏的、外表上以抗日面貌出现的那些汉奸、托派、亲日派分子。共产党员应该随时注意这些奸细,并以真凭实据为基础,按照具体情况,揭发这些奸细们的罪恶,同时劝告友党友军与人民大众不要上他们的当。提高对于民族奸细们的政治警觉性,共产党员负了重要的责任。揭发与清除奸细,是与扩大巩固民族统一战线不能分离的。

(5) 扩大共产党与防止奸细混入

为了克服困难战胜敌人,共产党必须扩大其组织,向着真诚革命,而又信仰党的主义,拥护党的政策,并愿意服从纪律,努力工作的广大工人农民与青年积极分子开门,使党变为伟大的带群众性的党。在这里,关门主义倾向是不能容许的。但同时,对于奸细混入的警觉性也决不可少。日本帝国主义的特务机关,时刻企图破坏我们的党,时刻企图利用暗藏的汉奸、托派、亲日派、腐化分子、投机分子,装扮积极面目,混入我们党里。对于这些分子的警惕与严防,是一刻也不应该放松的。不可因为怕奸细而把自己的党关起门来,大胆的发展党是我们确定了的方针。但同时,又不可因为大胆发展而疏忽对于奸细与投机分子乘机侵入的警戒。“大胆发展而又不让一个坏分子侵入”,这就是我们发展党的总方针。

(6) 坚持统一战线与坚持党的独立性

如果中国只有一个阶级,一个党派,那就再不要什么统一战线。所谓统一战线,就是拿两个以上的阶级与党派之存在作前提的。坚持抗日民族统一战线才能胜敌,并须是长期的坚持,这是确定了的方针。但同时,必须保持加入统一战线中的任何党派在思想上政治上与组织上的独立性,不论国民党也好,共产党也好,其它党派也好,都是一样。三民主义中的民权主义是什么呢?在党派问题上说来,就是容许联合统一,同时又容许其独立共存。否认独立性,只谈统一性,这是背弃民权主义的思想,不但我们共产党不能同意,任何党派也是不能同意的。没有问题,统一战线中,独立性不能超过统一性,而是服从统一性,统一战线中的独立性,只是也只能是相对的东西。不这样做,就不算坚持统一战线,就要破坏团结对敌的总方针。但同时,决不能抹杀这种相对的独立性,无论思想上也好,政治上也好,组织上也好,各党必须有相对的自由权。如果被人抹杀或自己抛弃这种相对的独立性或自由权,也同样将破坏团结对敌,破坏统一战线。这是每个共产党员,同时也是每个友党党员,应该明白的。

阶级斗争与民族斗争的关系也是一样。抗日战争中,一切服从抗日利益是总原则,阶级斗争必须服从于民族斗争的利益与要求,而决不应是相违背。但同时,在阶级社会存在的条件下,阶级斗争不能消灭,也无法消灭,企图根本否认阶级斗争存在的理论是歪曲的理论。我们不是否认它,而是调节它,我们提倡的互助互让政策,不但适用于党派关系,基本的也适用于阶级关系。为了团结抗日,应实行一种调节各阶级相互关系的恰当的政策,不应使劳苦大众毫无政治上与生活上的保证,同时也应顾到富有者的利益,这样去适合团结对敌的总要求。

(7) 照顾全局,照顾多数,及和同盟者一道来干

共产党员在领导群众,参加统一战线,并和敌人奋斗时,照顾全局,照顾多数,及和同盟者一道干的精神,是不可忽视的。共产党员应该值得以局部需要服从全局需要之必要。在局部情形看来认为可行,而在全局看来认为不可行时,应以局部服从全局。反之也是一样,在局部情形看来认为不可行,而在全局看来认为可行时,也应以局部服从全局。这就是照顾全局的观点。共产党员决不可脱离群众的多数。置多数人的情况于不顾,而率领少数先进队伍单独冒进,这是不能成功的。随时注意组织先进分子与广大群众之间的密切联系,这就是照顾多数的观点。在一切有同盟者存在的地方,遇事应和同盟者协同去干,独断专行,把同盟者置之不理的态度,是不对的。这些,都是共产党员领导艺术与工作精神方面不可忽视的地方。一个好的共产党员,应该是善于照顾全局,善于照顾多数,并善于与同盟者协同工作的,违背了这些,就不是一个好党员。

(8) 干部政策

中国共产党是在一个几万万人的大民族中领导伟大革命斗争的党,没有多数才德兼备的领导干部,是不能完成其历史任务的。十七年来,我们党已经培养了不少的领导人才,军事、政治、文化、党务、民运各方面,都有了我们的骨干,这是党的光荣,也是全民族的光荣。但同时,现有的骨干还不足以支撑斗争的大厦,还须广大地培养人才。伟大的民族革命斗争中,已经涌出并正在继续涌出无数的天才家、领导者。我们的责任,就在于组织他们,培养他们,爱护他们,并善于使用他们。“政治路线确定之后,干部就是决定的因素”,我们不要忘记这个真理。在这里,依靠原有干部的基础但不自满于这个基础,是必须的。因此,坚持而有计划地培养大批的新干部,应是我们的战斗任务。

不但关心党的干部,还要关心非党干部。党外存在着很多的人才,共产党不能把他们置之度外。去掉孤傲习气,善于同非党干部共事,真心诚意地团结他们,同时善意地给以帮助,对待他们以热烈的同志的态度,把他们的积极性与天才组织到抗战建国的伟大事业中去,是每一个共产党员的责任。画己自封、目无余子的态度,是不对的。

必须善于识别干部。对于干部长短优劣的识别,不但看他的表现,而且看他的本质,不但看他的一时一事,而且看他的全历史与全工作,这是识别干部的正确方法。在这里,粗心大意,任情逞性,是不能解决问题的。

必须善于使用干部。领导者的责任与工作,归结起来,只有两件事:出主意,用干部。一切计划、决议、命令、指示、文告、著述、讲演等等,都属于“出主意”一类。使这一切“主意”见之实行,必须团结干部,推动他们去做,都属于“用干部”一类。这两件事,在中国习惯上,就是所谓“用人行政”。在这个使用干部的问题上,我们民族历史中历来有两个表现邪正两派互相对立的路线,一个是“任人唯贤”,一个是“任人唯亲”。前者是明君贤臣用人的方针,后者是昏君奸臣用人的方针。我们今天来说使用干部问题,是站在革命立场上的,根本与古代有区别,但也离不开“任人唯贤”这个标准。以喜怒为爱憎,阿谀逢迎者奖,骨鲠正直者罚,在古时要不得,在我们也要不得。列宁、斯大林的干部政策,在于以坚决执行党的路线,服从党的纪律,与群众有密切联系,有独立工作能力,积极肯干,不为私利等等为标准,而不是其它。在这里,过去张国焘的干部政策正是相反。在张国焘,正是阿谀者奖,正直者罚,拉拢私党,别有企图,他的小组织派别活动,是有了深长历史的。然而也正是他这种以个人为中心而不以党的政治原则为中心的干部政策,走到了他的目的之反面,一切干部都脱离了他,结果剩下了张国焘寡人一名,叛党而去,这是一个大教训。半殖民地半封建社会中政治经济的落后性,反映到党内,就是自由主义、个人风头主义与派别活动等恶劣倾向的根源。估计到这种根源的存在,坚持列宁斯大林的组织路线与干部政策,反对不正派不公道的恶劣倾向,巩固党在正确路线上的统一团结,这是中央以至全党同志的责任。

必须善于爱护干部。在党的培养与艰苦斗争中创造出来的干部,是民族的珍宝,全党的荣誉,应为全党同志所尊重所爱护,各级领导机关则负有用实际办法达到爱护目的之责任。有些什么办法呢?第一,指导他们。这就是让他们放手工作,使他们敢于负责,不怕犯错误;但同时,又适时地与恰当地给以关于工作环境、工作方针与工作方法的指示,使他们能在党的政治路线下发挥其创造性。第二,提高他们。这就是给以学习理论与方法的机会,教育他们,使之在思想上在领导能力上较之过去提高一步。第三,检查他们的工作。不是每天检查,而是适时检查,帮助他们总结经验,纠正缺点,扩张成果。这是必要的,有委托而无检查,及至犯了严重错误,方才加以注意;不是爱护干部的办法。第四,改造他们。这就是对于有缺点的、犯错误的、有不正确思想的干部,用主要的说服方法,不得已时则用斗争方法,使他们改变过来。在这里,耐心是必要的。在并非大的原则错误又非说而不服的情况下,不适当地轻易地给人戴上“机会主义”、“小资产阶级意识”等等大帽子的方法,不适当地轻易地采用“开展斗争”的方法,都是不对的。第五,照顾他们的困难。干部的疾病问题、生活问题、家庭问题等事,党的领导机关应给以热忱的亲切的同志的关心,漠然置之冷淡不理的态度是不对的。疾病必须医治调养,生活求其切合工作需要,家庭问题在可能范围内也须助其解决。一切这些,在物质与环境许可的限度内给以照顾,对于激励干部的工作精神,团结全党为一体的目的上,是有重要意义的。

(9) 党的纪律

十七年来,尤其是五中全会以来的党的斗争经验,证明了有在党内,八路军与新四军内,继续坚持铁的纪律的必要。纪律是执行路线的保证,没有纪律,党就无法率领群众与军队进行胜利的斗争。在过去,由于克服了张国焘一类破坏纪律的倾向,保证了抗日民族统一战线与抗日战争的顺利执行。在今后,又必须坚持这种纪律,才能团结全党,克服新的困难,争取新的胜利。在这里,几个基本原则是不容忽视的:(一)个人服从组织:(二)少数服从多数;(三)下级服从上级;(四)全党服从中央。这些就是党的民主集中制的具体实施,谁破坏了它们,谁就破坏了党的民主集中制,谁就给了党的统一团结与党的革命斗争以极大损害。为此原故,党的各级领导机关,应该根据上述那些基本原则,给全党尤其是新党员以必要的纪律教育。过去经验证明:有些破坏纪律的人,由于他们不懂得什么是党的纪律。有些明知故犯的人,例如张国焘一类,则利用一部分党员的无知以售其奸。所以纪律教育,不但在养成一般党员服从纪律的良好作风上,是必要的:而且在监督党的领袖使之服从纪律,也有其必要。党的纪律是带着强制性的:但同时,它又必须是建立在党员与干部的自觉性上面,决不是片面的命令主义。为此原故,从中央以至地方的领导机关,应制定一种党规,把它当作党的法纪之一部分。一经制定之后,就应不折不扣地实行起来,以统一各级领导机关的行动,并使之成为全党的模范。

(10)党的民主

处在伟大斗争面前的中国共产党,要求整个党的领导机关,全党的党员与干部,高度地发挥其积极性,才能引导斗争向胜利。所谓发挥积极性,不能只是一句空话,必须集中表现在领导机关、干部与党员的创造能力,负责精神,工作的活跃,敢于与善于提出问题,发表意见,批评缺点,以及对于领导机关与领导干部从爱护观点出发的监督作用,等等上面。没有这些,所谓积极性就是空的。而这些积极性的发挥,有赖于党内生活制度的民主化,没有或缺乏民主生活,是不能达到发挥积极性之目的的。大批能干人材的创造,也只有在民主生活中才有可能。

由于我们国家至今还没有民主生活,反映到党内,就产生了民主生活不足的现象,这种现象,实在妨碍着全党积极性的充分发挥。同时,也就影响到统一战线中,民众运动中,民主化之不足。为此原故,必须在党内施行民主教育,使党员懂得什么叫做民主生活,民主制与集中制的联系,并如何实行民主集中制。这样才能做到:一方面,确实扩大了党内民主生活:又一方面,不至于走到极端民主化,走到自由放任主义。

在军队中的党,也须增加必要的民主生活,以更提高党员的积极性,增强军队的战斗力。但同时,军队党的民主应少于地方党的民主,应是为着巩固军队纪律与增强战斗力的。而不是削弱纪律与战斗力。在地方党,也应是有利于巩固党的纪律与增强党的战斗力,而不是相反的。

扩大党内民主,是巩固党与发展党的必要步骤,是使党在伟大斗争中生动活跃,胜任愉快,生长新的力量,突破战争难关的有用的与重要的武器。

(11)我们党已经从两条战线斗争中巩固与壮大起来

十七年来,我们的党,一般地,已经学会了使用这个马克思主义的武器──思想上政治上及工作上的两条战线斗争的方法,一方面反对右倾机会主义,又一方面反对左倾机会主义。

五中全会以前,我们党反对了陈独秀的右倾机会主义与李立三的左倾机会主义。由于这两次党内斗争的胜利,使党获得了伟大的进步。五中全会以后,又有过两次有历史意义的党内斗争,这就是遵义会议与开除张国焘。

由于遵义会议纠正了在反五次围剿斗争中所犯的左倾机会主义性质的严重的原则错误,团结了党与红军,使得中央与红军主力胜利地完成了长征,转到了抗日的前进阵地,执行了抗日民族统一战线的新政策。由于巴西会议与延安会议(反张国焘路线错误是从巴西开始而在延安完成的)反对了张国焘的右倾机会主义,使得全部红军会合一处,全党更加团结起来,进行了英勇的抗日斗争。这两种机会主义错误都是在国内战争中产生的,它们的特点是战争中的错误。

这两次党内斗争所得的教训在什么地方呢?在于:(一)由于不认识中国革命战争中的特点而产生的,表现于反五次围剿斗争中的严重的原则错误,包含着不顾主客观条件的“左”的急性病倾向,这种倾向极端不利于革命战争,同时也不利于任何革命运动。要指出:当时的这种错误并非党的总路线的错误,而是执行当时总路线所犯的战争策略与战争方式上的严重原则错误。(二)张国焘的机会主义,则是革命战争中的右倾机会主义,其内容是他的退却路线、军阀主义与反党行为的综合。只有克服了它,才能使得本质很好而且作了长期英勇斗争的红军第四方面军尤其是它的广大的干部与党员,从张国焘的机会主义统制之下解放出来,转到中央的正确路线之下。(三)中央苏区时期的伟大的组织工作,不论军事建设也好,政府工作也好,民众工作也好,党的建设也好,是有大的成就的,没有这种组织工作与前线的英勇战斗相配合,要支持当时残酷的斗争是不可能的。然而在当时党的干部政策与组织原则方面,是犯了严重原则错误的,这表现在宗派倾向,惩办主义,与思想斗争中的过火政策。这是过去李立三路线的残余未能肃清的结果,也是当时政治上原则错误的结果。这些错误,也因遵义会议得到了纠正,使党转到了全般正确的干部政策与组织原则之下来了。在张国焘的组织路线方面,则是完全离开了党的一切原则,破坏了党的纪律,从小组织活动一直发展到反党反中央反国际的行为。中央对于张国焘的罪恶的路线错误与反党行为,曾经尽了一切可能的努力去克服它,并图挽救张国焘本人。但到了张国焘不但坚持不变,采取了两面派的行为,而且最后实行叛党,就不得不坚决开除他的党籍。这一开除,不但获得了全党的拥护,而且获得了一切忠实于民族解放事业的人们的拥护。共产国际已经批准了这一开除,并指出:张国焘是一个逃兵与叛徒。

以上这些教训与成功,给了我们在今后团结全党,巩固思想上、政治上与组织上的一致,胜利地执行抗日战争与抗日民族统一战线之必要的前提。我们的党已经从两条战线斗争中巩固与壮大起来了。

(12)当前的两条战线斗争

在今后新的抗战形势中,政治上反对右的悲观主义,将是头等重要的。但同时,反对“左”的急性病,也仍然要注意。在统一战线问题上,党的组织与民众组织问题上,则须继续反对“左”的关门主义倾向,以便实现长期合作,发展党,与发展民众运动。但同时,无条件的合作,无条件的发展,这种右倾机会主义倾向也要注意,否则也就要妨碍合作,妨碍发展,而变为投降主义的合作与无原则的发展了。

两条战线斗争必须切合于具体对象的实际情况,决不能抽象地看问题,一般的指出与具体的应用,是有区别的。所谓“乱戴帽子”的坏习惯,也就是说的那种抽象地应用这个方法之不对。在反倾向斗争中,反对两面派的行为,是值得严重注意的。因为两面派行为的最大危险性,在于它可能发展到小组织行动,张国焘的历史就是证据。阳奉阴违,口是心非,当面说得好听,背后又在捣鬼,这就是两面派行为的实质。提高干部与党员对于两面派行为的注意力,是巩固党的纪律之重要的要求。

(13)学习

一般地说,一切有相当研究能力的共产党员,都要研究马克思、恩格斯、列宁、斯大林的理论,都要研究我们民族的历史,都要研究当前运动的情况与趋势;并经过他们,去教育那些文化水准较低的党员。特殊地说,干部应该着重地研究这些东西,中央委员会与高级干部尤其应该加紧研究。指导一个伟大的革命运并使之向着胜利,没有革命理论,没有历史知识,没有实际运动的了解,就不能有胜利。

马克思、恩格斯、列宁、斯大林的理论,是“放之四海而皆准”的理论。不是把他们的理论当作教条看,而是当作行动的指南。不是学习马克思列宁主义的字母,而是学习他们视察问题与解决问题的立场与方法。只有这个行动指南,只有这个立场与方法,才是革命的科学,才是引导我们认识革命对象与指导革命运动的唯一正确的方针。中国党的马克思主义的修养,现已较前大有进步,但还说不到普遍与深入。在这方面,我们较之若干外国的兄弟党,未免逊色。而我们的任务,是在领导一个四万万五千万人口的大民族,进行着空前的历史斗争。所以普遍地深入地研究理论的任务,对于我们,是一个亟待解决并须着重致力才能解决的大问题。我们努力罢,从我们这次扩大的六中全会之后,来一个全党的学习竞赛,看谁真正学到了一点东西,看谁学的更多一点,更好一点。我们的工作做得还不错,但如果不加深一步地学习理论,就无法使我们的工作做得更好一些,而只有使我们的工作做得更好一些,才有我们的胜利。因此,学习理论是胜利的条件。在主要领导责任的观点上说,如果中国有一百个至二百个系统地而不是零碎地,实际地而不是空洞地,学会了马克思主义的同志,那将是等于打倒一个日本帝国主义。同志们,我们一定要学习马克思主义。

学习我们的历史遗产,用马克思主义的方法给以批判的总结,是我们学习的另一任务。我们这个大民族数千年的历史,有它的发展法则,有它的民族特点,有它的许多珍贵品。对于这个,我们还是小学生。今天的中国是历史的中国之一发展,我们是马克思主义的历史主义者,我们不应该割断历史。从孔夫子到孙中山,我们应该给以总结,我们要承继这一份珍贵的遗产。承继遗产,转过来就变为方法,对于指导当前的伟大运动,是有着重要的帮助的。共产党员是国际主义的马克思主义者,但马克思主义必须通过民族形式才能实现:没有抽象的马克思主义,只有具体的马克思主义。所谓具体的马克思主义,就是通过民族形式的马克思主义,就是把马克思主义应用到中国具体环境的具体斗争中去,而不是抽象地应用它。成为伟大中华民族之一部分而与这个民族血肉相联的共产党员,离开中国特点来谈马克思主义,只是抽象的空洞的马克思主义。因此,马克思主义的中国化,使之在其每一表现中带着中国的特性,即是说,按照中国的特点去应用它,成为全党亟待了解并亟须解决的问题。洋八股必须废止,空洞抽象的调头必须少唱,教条主义必须休息,而代替之以新鲜活泼的、为中国老百姓所喜闻乐见的中国作风与中国气派。把国际主义的内容与民族形式分离起来,是一点也不懂国际主义的人们的干法,我们则要把二者紧密地结合起来。在这个问题上,我们队伍中存在着的一些严重的缺点,是应该认真除掉的。

当前运动的特点是什么? 它有什么规律性,如何指导这个运动?这些都是最实际不过的问题。直到今天,我们还没有懂得日本帝国主义的全部,也还没有懂得中国的全部。运动在发展中,又有新的东西在后头,新东西是层出不穷的。研究这个运动的全面及其发展,是我们要时刻光起眼晴注意的大课题。如果有人拒绝对于这些作认真的过细的研究,那他就不过是一个西班牙的唐·吉诃德,再加一个中国的阿Q,而不是一个马克思主义者。如何研究? 用马克思主义的工具——唯物辩证法。向谁研究? 我们的先生多得很——工人、农民、小资产阶级、资本家、地主、日本帝国主义,还有全世界,他们都是我们的研究对象,同时又都是我们的先生,我们应该向他们或多或少的学到一点东西。

学习的敌人是自己的满足,要认真学习一点东西,须从不自满始。对自己,“学而不厌”,对人家,“诲人不倦”,我们应取这种态度。

(14)团结全党到团结全民族

伟大的斗争需要伟大的力量,团结全民族,发动全民族一切生动力量进入这个斗争中去,是我们确定了的方针,而要达此目的,中国共产党内部的团结,是有重大作用的,是最基本的条件。遵义会议与克服张国焘错误之后,我们的党是第六次全国代表大会以来最团结最统一的时期了。现在我们党内,无论在政治路线上,战略方针上,时局估计与任务提出上,中央委员会与全党,意见都是一致的。这种政治原则的一致,是团结的基本条件,党员与党员,干部与干部,领导者与领导者之间的相互关系,习惯上所谓人事关系,我们也学会了许多正确的恰当的方法,造成了在正确政治原则下的和衷共济的空气,有了更好的相互关系。由于地区的广大,情况的复杂,工作部门的不同,不同的意见是难免的,应该的,党内民主的实际,就是容许任何不同意见的提出与讨论。也正是由于民主方法保证着交换意见,并使之概括起来作出结论,形成全党一致的方针。在这里,客观地与全面地看问题的态度,不杂主观成见与意气,不要片面的看问题,这种马克思主义的方法,我们也逐渐的学会了,这又保证着党的团结。我们是科学的马克思主义者,自以为是的成见与意气用事的作风,是无用的长物。经过了十七年锻炼的中国共产党及它的领导人员,已经有了老练的作风了。所有这些,就能保证中央以至全党的团结一致,就能在全民族中形成一个团结一致的核心与重心,推动抗战进到胜利。同志们,全党团结起来,全民族团结起来,胜利一定是我们的!

八、召集党的七次代表大会

现在我来说最后一个问题,召集七次大会的问题。

同志们,我们党的全国代表大会,自从一九二八年开过第六次代表大会以来,由于环境的原因,已有十年没有开大会了。去年十二月政治局会议决定准备召集七次代表大会,但准备工作尚未完成,因此今年尚难召集。此次全会扩大会应该讨论加紧这个准备工作的问题,并决定在不久时间实行召集大会。这次大会的政治意义是重大的,它将总结过去的经验,主要的是全国抗战与抗日民族统一战线的经验。讨论国内国际的政治形势。讨论如何进一步的团结全民族,团结国共两党及其它党派,进一步的巩固与扩大抗日民族统一战线。讨论如何在长期战争与长期合作中争取抗战最后胜利的方针方法与计划。讨论如何动员全国工人阶级及劳动人民更积极的参加抗战。并应讨论党在斯的情况下如何进一步的团结自己,加强自已,巩固自己与国民党、其它党派及全国人民的联系,以便顺利地执行抗日民族统一战线的总方针。除了这些政治的与组织的问题之外,七次大会应该选举新的中央委员会,将全党中最有威信的许多领导同志选进中央委员会来,加强对于全党工作的领导。同志们,这次大会的意义如此重大,因此,扩大的六中全会闭幕之后,诸位同志回到各地工作,便应在努力发展党与巩固党的基础之上,依照民主的方法,适时地进行选举,使那些最优秀的最为党员群众所信托的干部与党员有机会当选为大会的代表,使七次大会能够集全党优秀代表于一堂,保证大会的成功。我们相信,这次全国代表大会一定能够成功,一定能够给日本帝国主义的侵略战争以最庄严的最有力量的回答,让日本帝国主义在我们的全国代表大会面前发起抖来,滚到东洋大海里去,中华民族是一定要胜利的。

我的报告就此完结。

●解放五七期 一九三八.一一.二五

※新华日报(重庆) 一九三八.一二

※论新阶段 新群众丛书二十二新华日报社 一九三八.一二

※论新阶段中国革命战争指导理论之四新民主出版社(香港) 一九四八.四

※文献卷之三.四 风雨书屋 一九三八.一二 一九三九.一

◎中国共产党的六中全会文献 重庆新华日报馆 一九三九

\section{中国革命与中国共产党 1939/12/15}

【编者按:从这篇原文可以看到,毛选中这篇文章谈到中国封建社会的资本主义萌芽的那些段落,都是原文所没有的,是事后补充的。有一小段落承认中小地主可以在反对帝国主义和大地主的时候,保持中立或暂时参加斗争,在收入毛选的时候则删掉。原文谈到四个阶级联盟的时候,并没有强调无产阶级领导,但在收入毛选的时候就补加了无产阶级领导的话。此外各种含有政治意义的修改也不少。】

第一章 中国社会

第一节 中华民族

我们中国是世界上最大国家之一,他的领土超过了整个欧洲的面积。在这个广大领土之上,有广大的肥田沃地,给我们以衣食之源;有纵横全国的大小山脉、大小高原、平原,给我们生长了广大的森林,贮藏了丰富的矿产;有很多的江河湖泽,给我们以舟楫与灌溉之利; 有很长的海岸线,给我们以交通海外各民族的方便。从很早的古代起,我们中华民族的祖先就劳动、生息、繁殖在这块广大土地之上。

现在中国的国境:在东北、西北和西境的一部与社会主义苏维埃共和国联盟接壤。西方的一部和西南方与印度、不丹、尼泊尔接壤。南方与暹逻、缅甸和安南接壤,并和台湾邻近。东方与日本邻近和朝鲜接壤。这个地理上的国际环境,给予中国革命造成了外部的有利条件和困难条件。有利的是:与苏联接壤,与欧美一切主要帝国主义国家隔离较远,在其周围的许多国家中大部都是殖民地半殖民地国家。困难的是:日本帝国主义利用其海、陆、空与中国接近的关系,时刻都在迫害着中国的生存和中国的革命。

我们中国现在拥有四万万五千万人口,差不多占了全世界人口的四份之一。在这四万万五千万人口中,十分之九为汉人。此外,还有回人、蒙人、藏人、满人、苗人、夷人、黎人等等许多少数民族,虽开化的程度不同,但他们都有了长久的历史。中国是一个由多数民族结合而成的拥有广大人口的国家。

中华民族的发展(主要是汉族的发展),和世界上别的大民族同样,曾经经过了若干万年平等而无阶级的原始共产主义社会的生活。而从原始共产主义社会崩溃、社会生活转入阶级生活那个时代开始,经过奴隶社会、封建社会,直到现在,已有了五千年之久。在中华民族主要是汉族的开化史上,有素称发达的农业和手工业,有许多伟大的思想家、科学家、发明家、政治家与军事家,有丰富的文化典籍,还在三千年前,中国就有了指南针的发明。还在一千七百年前,已经发明了造纸法。在一千二百年前,已经发明了刻版印刷。在八百年前,更发明了活字印刷。火药的应用,也远在欧人之前。所以中国是世界文明发达最早的国家之一,中国已有了五千年的文明史。

中华民族不但是以刻苦耐劳著称于世,同时又是酷爱自由富于革命传统的民族。以汉族的历史为例,证明中国人民是不能忍受黑暗势力的统治的,他们每次都用革命的手段达到推翻与改造这种统治的目的。在汉族的数千年的历史上,有过几百次的农民暴动,反抗地主贵族的黑暗统治; 而每次朝代的更换,都是由于农民暴动的力量才能得到成功的。中华民族的各族人民对于外来民族的压迫都是不愿意的,都是要用反抗的手段解除这种压迫的。他们只赞成平等的联合,而不赞成互相压迫。在中华民族的几千年的历史中,产生了很多的民族英雄与革命领袖,产生了很多的革命军事家、政治家、文学家与思想家。所以中华民族又是一个有光荣革命传统和优秀历史遗产的民族。

第二节 古代的封建社会

中国虽是一个伟大的民族国家,虽是一个地广人众、历史悠久而又富于革命传统与优秀遗产的国家;可是中国由从脱离奴隶制度进到封建制度以后,就长期的停顿起来。这个封建制度,自周秦以来一直延续了三千多年。由于封建制度的延续,就使得中国的经济、政治、文化,都长期的陷在发展迟缓甚至停滞的状态中。三千年来的中国社会是一个封建的社会。

中国封建时代的经济制度和政治制度,是由以下的各个主要特点造成的:

一、自足自给的自然经济占主要地位。农民不但生产自己需要的农产品,而且生产自己需要的大部分手工业品。农业交付地主贵族的地租,也主要是地主们自己享用,不是为了交换。那时虽有交换的发展,但在整个经济中不起决定的作用。

二、封建的统治阶级——地主、贵族以至皇帝,他们拥有最大部分的土地,而在农民则很少土地,或完全没有土地。农民用自己的工具去耕种地主、贵族和皇室的土地,并将收获的四成、五成、六成甚至七成,奉献给地主、贵族、皇室们享乐,这种农民实际上还是农奴。

三、不但地主、贵族和皇室依靠剥削农民的地租过活,而且地主阶级的国家又强迫农民缴纳贡税并强迫农民从事无偿的劳役,去养活一大群的国家官吏及为了镇压农民之用的军队。

四、保护这种封建剥削制度的,便是地主阶级的封建国家。如果说周是诸侯割据称雄的封建国家,那么自秦始皇统一中国以后,就建立了专制主义的中央集权的封建国家,同时,在某种程度上仍旧保留着封建割据的状态。在封建国家中皇帝有至高无上的绝对的权力,在各地方设官职以掌兵、刑、钱、谷等事,并依靠地主绅士作为全部封建统治的基础。

中国历代的农民,就在这种封建的经济剥削和封建的政治压迫之下,过着贫穷困苦的奴隶式的生活。农民被束缚于封建制度之下,没有人身的自由,地主对农民有随意打骂甚至处死之权,农民是没有任何政治权利的。由于地主阶级这样残酷的剥削和压迫所造成的农民的极端穷苦和落后,就是中国社会几千年在经济上和社会生活上停滞不前的基本原因。

封建社会的主要矛盾,是农民阶级与地主阶级的矛盾。

而在这样的社会中,只有农民与手工业工人是创造财富与创造文化的基本的阶级。

地主阶级对于农民的残酷的经济剥削和政治压迫,曾经不能不在历史上掀起无数的农民暴动以反抗地主阶级的统治。从秦朝的陈胜、吴广、项羽、刘邦,汉朝的新市、平林、赤眉、黄巾、铜马,隋朝的李密、窦建德,唐朝的黄巢,宋朝的宋江、方腊,元期的朱元璋,明朝的李自成,直至清朝的太平天国,总共不下数百次,都是农民的反抗运动,都是农民的革命战争。中国历史上农民暴动与农民战争的规模之大,是世界历史上所没有的。只有这种农民暴动与农民战争,才是中国历史进化的真正动力。因为每次农民暴动与农民战争的结果,都打击了当时的封建统治,因而也就多少变动了社会的生产关系与多少推动了社会生产力的发展。只是由于当时还没有新的生产力与新的生产方式,没有新的阶级力量,没有先进的政党,因为这种农民战争与农民暴动得不到先进阶级与先进政党的领导如同现在的无产阶级与共产党能够正确领导农民暴动与农民战争。这样,就使当时的农民革命总是陷于失败,总是在革命中与革命后被地主贵族利用了去,当作他们改朝换代的工具。这样,就在每次农民革命斗争停息以后,虽然多少有些进步,但是封建的经济关系和封建的政治制度,基本上依然继续下去。

这种情况,直至近百年来,才发生新的变化。

第三节 现代的殖民地、半殖民地、半封建社会

中国过去三千多年的社会是封建社会,前面已经说明了。那么,中国现在的社会是否还是完全的封建社会呢? 不是,中国已经变化了。自从一八四○年的鸦片战争以来,中国已经一步一步的变成了一个半殖民地半封建的社会。自从一九三一年“九一八”事变日本帝国主义武装侵略中国以来,中国又变成了一个殖民地、半殖民地、半封建的社会。现在我们就来说明这种变化的过程。

如第二节所述,中国的封建社会继续了三千多年。直到十九世纪的中叶,由于外国资本主义的侵入,才使这个社会的内部发生了重大的根本的变化。

外国资本主义的侵入,曾经对中国的社会经济起了分解的作用。因为外国资本主义的侵入,一方面破坏了中国自给自足的自然经济,破坏了城市的手工业及农民的家庭手工业;又一方面则促进了中国城乡商品经济的发展。

这些情形,不仅对中国封建经济的基础起了解体的作用,同时又给中国资本主义生产的发生造成了某些客观的条件与可能。因为自然经济的破坏,给资本主义造成了商品的销售市场,而大量农民和手工业者的破产,又给资本主义造成了劳动力的购买市场。

事实上,由于外国资本主义的刺激与封建经济结构的某些破坏,还在十九世纪下半期,还在六十年前,就开始有一部分商人、地主和官僚投资于新式工业。到了同世纪末年和二十世纪初,到了四十年前,中国民族资本主义便开始了初步的发展。到了二十年前,即第一次帝国主义世界大战的时期,由于欧美帝国主义国家忙于战争,暂时放松了对于中国的压迫,中国的民族工业,主要是纺织业、面粉业和丝织业,曾经得到了进一步的发展。在这一时期中,中国的纺织业,由一九一五年的二十二个厂,增加到一九二二年的四十四个厂;面粉业由一九一六年的六十七个厂,增加到一百零七个厂;丝织业增加了六十三个厂。在这一时期中,中国的银行,也增加了一百零八个。

中国民族资本主义发生和发展的过程,就是中国资产阶级与无产阶级发生和发展的过程,如果一部分的商人、地主和官僚是中国资产阶级的前身,那么,一部分的农民和手工业工人就是中国无产阶级的前身了。中国的资产阶级与无产阶级,作为两个特殊的社会阶级来看,他们是新产生的,他们是中国历史上没有过的阶级,他们从封建社会脱胎而来,构成了新的社会阶级。他们是两个互相关联又互相对立的阶级,他们是中国旧社会(即封建社会)产出的双生子。但是中国无产阶级的发生与发展,不但是伴随中国民族资产阶级的发生与发展而来,而且是伴随帝国主义在中国直接经营企业而来,所以中国无产阶级较之中国资产阶级的年龄和资格更老些,所以他的社会力量和社会基础也更广大些。

可是,上面所述的这一资本主义发展的新变化,还只是帝国主义侵入中国以来所发生的变化的一个方面,还有与这个变化同时存在而阻碍这个变化的另一个方面,这就是帝国主义勾结中国封建残余压迫中国资本主义的发展。

帝国主义列强侵入中国的目的,决不是要把封建的中国变成资本主义的中国。帝国主义列强的目的与此相反,他们是要把中国变成他们的半殖民地与殖民地。

帝国主义列强为了这个目的,曾经对中国采用了并且还继续采用着如同下面所说的那些军事的、政治的、经济的以及文化的一切压迫手段,使中国一步一步的变成了半殖民地与殖民地:

一、用战争打败了中国之后,帝国主义国家便抢去了中国的许多属国与一部分领土。日本占领了朝鲜、台湾、琉球、澎湖群岛与旅顺,英国占领了缅甸、不丹、尼泊尔与香港,法国占领了安南,而蕞尔小国如葡萄牙也占领了我们的澳门。割地之外,又索去了巨大的赔款。这样就大大打击了中国这个庞大的封建帝国。

二、帝国主义根据条约有在中国驻扎海军与陆军之权,有领事裁判权,并把全中国划分为几个帝国主义的势力范围。

三、帝国主义根据条约控制了中国一切重要的通商口岸,并把许多通商口岸划出一部分土地作为他们直接管理的租界,他们控制了中国的海关与对外贸易,控制了中国的交通事业(海上的、陆上的、空中的与内河的)。因此他们便能够使中国的农业生产服从于帝国主义的需要。

四、帝国主义还在中国经营了许多轻工业和一部分重工业的企业,以便直接利用中国的原料与廉价的劳动力,并以此与中国的民族工业进行直接的竞争。

五、帝国主义经过借款给中国政府,并在中国开设银行,垄断了中国的金融财政。因此他们就不但在商品竞争上压倒了中国的民族资本主义,而且在金融上、财政上扼住了中国的咽喉。

六、帝国主义从中国的通商都市直至穷乡僻壤,造成一个买办的和商业高利贷的剥削网,造成了为帝国主义服务的买办阶级和商业高利贷阶级,以便利其剥削广大的中国农民。

七、于买办阶级之外,帝国主义还需要一个更大的社会力量,作为他们统治中国的支柱,这种社会力量就是中国的封建残余。他们“首先和以前社会构造的统治阶级——封建地主、商业与高利贷资产阶级结了联盟,以进攻占大多数的民众。帝国主义到处企图保持资本主义前期的榨取形式(尤其是在乡村)用作反动联盟生存的基础”。(共产国际六次大会殖民地与半殖民地运动大纲)“帝国主义及其所有财政和军事力量之在中国,就是拥护且推动那些封建残余及其全部军阀官僚的上层建筑物,使他欧化又使他成为守旧的力量”。(一九二七年斯大林在共产国际执委会的演说)

八、为了造成中国军阀混战与镇压中国人民的必要。帝国主义曾经供给中国政府以大量的军火与大批的军事顾问。

九、帝国主义在所有上述这些办法之外,对于麻醉中国人民的精神一方面,也不放松,这就是他们的文化政策。传教、办学校、办报纸与吸引留学生等,就是这个政策的实施,其目的在于造就服从他们的知识干部与愚弄广大的中国人民。

十、帝国主义用所有上述各种办法一步一步的把中国变成了半殖民地。这种局面,都是帝国主义在多次残酷战争之后所造成的。例如一八四○年英国的鸦片战争,一八五七年英法联军的进攻北京,一八八四年的中法战争,一八九四年的中日战争,一九○○年的八国联军进攻北京。在这些战争之后,中国就沦为各主要帝国主义国家共同宰割和互相争夺的半殖民地,出现了上述半殖民地局面。而且一九三一年“九一八”以来,日本帝国主义的大举进攻,更使中国一大块土地沦为日本的殖民地。

上述这些情形,就是帝国主义侵入中国以后新的变化的又一方面,就是把一个封建的中国变为一个半封建半殖民地与殖民地的中国的血迹斑斑的图画。

由此可以明白,帝国主义侵略中国,有其促使中国封建社会解体的一方面,使中国发生了资本主义因素,起一个封建社会变成了半封建的社会; 但同时,残酷地统治了中国,把一个独立的中国变成了殖民地与半殖民地的社会。

将这两个方面的情形综合起来说,我们这个殖民地、半殖民地、半封建的社会,有如下的几个特点:

一、封建时代的自足自给的自然经济是被破坏了; 但是,无论在抗战的中国政府统治区域,无论在日本占领区域,封建剥削的根基——地主阶级对农民的封建剥削,不但依旧保持着,而且与买办资本和高利贷资本的剥削结合在一起,在中国的社会经济生活中,占着显然的优势。

二、民族资本主义有了某些发展,并在中国政治的、文化的生活中起了一定的作用; 但是,他没有成为中国社会经济的主要形式,他的力量是很软弱的,他是对于外国帝国主义和国内封建残余都有联系的。尤其是“九一八”以来,民族工业的绝大部分被日本帝国主义所摧毁,所掠夺,更大大改变了中国的局面。

三、皇帝和贵族的专制主义是被推翻了; 但代之而起的不是军阀官僚的统治,就是地主与大资产阶级联盟的专政。在沦陷区则是日本帝国主义及其傀儡的统治。

四、帝国主义不但操纵了中国的财政和经济的命脉,并且操纵了中国的政治和军事的力量,在沦陷区,则一切被日本帝国主义所独占。

五、由于中国是在许多帝国主义统治或半统治之下,由于中国实际上处于长期的不统一状态之中,又由于中国的土地广大,就使得中国的经济、政治与文化的发展,表现出极端的不平衡。

六、由于帝国主义和封建残余的双重压迫,特别是由于日本帝国主义的大举进攻,中国的广大人民尤其是农民,日益贫困以至破产,他们过着饥寒交迫与毫无政治权利的生活。中国人民的特殊的贫困与不自由,是世界各民族中所少有的。

殖民地、半殖民地、半封建的中国社会的特点就是这样。

决定这种情况的,主要是日本帝国主义与国际帝国主义的势力,是外国帝国主义与国内封建残余相结合的结果。

帝国主义与中华民族的矛盾,封建残余与人民大众的矛盾,这就是现时中国社会的主要矛盾。(当然还有别的矛盾,例如资产阶级与无产阶级的矛盾,统治阶级内部的矛盾等)而帝国主义与中华民族的矛盾乃是各种矛盾中的最主要的矛盾。这些矛盾的斗争及其尖锐化,就不能不造成日益发展的革命运动。伟大的近代与现代的中国革命,是在这些基本矛盾的基础之上发生与发展起来的。

第二章 中国革命

第一节 百年来的革命运动

帝国主义与中国封建残余相结合,把中国变为半殖民地与殖民地的过程,也就是中国人民反抗帝国主义及其走狗的过程。从鸦片战争、太平天国运动、中法战争、中日战争、戊戌政变、义和团运动、辛亥革命、五四运动、五三十运动、北伐战争、土地革命,直至现在的抗日战争,都表现了中国人民不甘屈服于帝国主义及其走狗的不断的反抗精神。

中国人民,百年以来,不屈不挠再接再厉的英勇斗争,使得帝国主义至今不能灭亡全中国,也永远不能灭亡全中国。

现在日本帝国主义虽然竭其全力大举进攻中国,虽然有许多地主与大资产阶级分子,例如公开的汪精卫与暗藏的汪精卫之流,已经投降敌人或准备投降敌人,但英勇的中国人民不但奋战了三年之久,而且必然还要奋战下去,不到驱逐日本帝国主义出中国,使中国得到了完全的解放,是决不会停止的。

中国人民的民族革命斗争,从一八四○年的鸦片战争算起,已经有了整整一百年的历史了,从一九一一年的辛亥革命算起,也有了三十年的历史了。这个革命的过程,现在还未完结,革命的任务还没有显著的成就,还要求全国人民,首先是中国共产党,担负起坚决奋斗的责任。

那么,这个革命的对象究竟是谁? 这个革命的任务究竟是甚么呢? 这个革命的动力是什么? 这个革命的性质是什么? 这个革命的前途又是什么呢?这些就是我们下面要来说明的。

第二节 中国革命的对象

依照第一章第三节的分析,我们已经知道了: 中国现时的社会,是一个殖民地、半殖民地,半封建性质的社会。只有认清中国社会的性质,才能认清中国革命的对象、中国革命的任务、中国革命的动力、中国革命的性质、中国革命的前途转变。所以,认清中国社会的性质,就是说,认清中国的“国情”,乃是认清一切革命问题的基本根据。

中国现时社会的性质,既然是殖民地、半殖民地、半封建的性质,那么, 中国现阶段革命的主要对象或主要敌人, 究竟是谁呢?

不是别的,就是帝国主义与半封建势力,就是外国的资产阶级与本国的地主阶级。因为在现阶段的中国社会中,压迫和阻止中国社会向前发展的主要的东西不是别的,正是他们二者,二者互相勾结以压迫中国人民,而以帝国主义的民族压迫为最大的压迫,因而帝国主义是中国人民的第一个和最凶恶的敌人。

在日本武力侵入中国以来,中国革命的主要敌人是日本帝国主义与勾结日本公开投降或准备投降的一切汉奸。

中国资产阶级本来也是受着帝国主义压迫的,他们也曾经领导过光荣的革命斗争,也曾经在革命中起过主要的或部分的领导作用,例如:辛亥革命、北伐战争与当前的抗日战争。但是,他们曾经在一九二七至一九三六年这一个长时期内勾结帝国主义,并与地主阶级结成反动同盟,背叛了曾经援助过他们的朋友——共产党、无产阶级、农民阶级与其他小资产阶级,背叛了中国革命,变成了人民的公敌,造成了革命的失败。所以当时革命的人民与革命的政党(共产党),曾经不得不把资产阶级也当作革命对象之一。在抗日战争中,大地主大资产阶级的一部分,以汪精卫为代表,已经叛变,已经变成汉奸,所以抗日的人民也已经不得不把这些背叛民族利益的大资产阶级分子当作革命对象之一。

由此也可以明白,中国革命的敌人是异常强大的。中国革命的敌人不但有强大的帝国主义,而且有强大的半封建势力,而且在一定时期内还有勾结帝国主义与半封建势力以与人民为敌的资产阶级,因此,那种轻视中国革命敌人力量的观点是不正确的。

在这样的敌人面前,中国革命的长期性与残酷性就发生了。因为我们的敌人是异常强大的,革命力量就非在长期间内不能聚积与锻炼成为一个足以最后战胜敌人的力量。因为敌人对中国革命的镇压是异常残酷的,革命力量就非磨炼与发挥自己的顽强性,不能坚持自已的阵地与夺取敌人的阵地。因此,那种以为中国革命力量瞬间就可以组成,中国革命斗争顷刻就可以胜利的观点是不正确的。

在这样的敌人面前,中国革命的方法,中国革命的主要形式,不能是和平的,而必须是武装的,也就决定了。因为我们的敌人不给中国人民以和平活动的可能,中国人民没有任何的政治自由。斯大林说:“中国革命的特点是武装的人民反对武装的反革命”,这是异常正确的规定。因此,那种轻视武装斗争,轻视革命战争,轻视游击战争,轻视军队工作的观点,是不正确的。

在这样的敌人面前,革命的特殊根据地问题也就发生了。因为强大的帝国主义及其在中国的反动同盟军,总是长期地占据着中国的中心城市,如果革命队伍不愿意和帝国主义及其走狗妥协,而要坚持奋斗下去,如果革命队伍要准备蓄积和锻炼自己的力量,并避免与强大敌人在力量不够时作决定胜负的战斗,那就必须把落后的农村造成先进的巩固的根据地,造成军事上、政治上、经济上、文化上的伟大革命阵地,借以反对利用城市进攻农村区域的凶恶敌人,借以在长期战斗中逐渐争取革命的全部胜利。在这种情形下面,由于中国经济发展的不平衡(农村经济不完全依赖城市),由于中国土地的广大(革命势力有回旋的余地),由于中国反革命营垒内部不统一和充满着各种矛盾,由于中国革命主力军的农民的斗争是在共产党的领导之下,这样,就使在一方面,中国革命有在农村区域首先胜利的可能;而在另一方面,则又造成了革命的不平衡状态,而使争取革命全部胜利的事业带来了长期性和艰苦性。由此也就可以明白,在这种特殊根据地上进行的长期革命斗争,主要的是在中国共产党领导之下的农民游击战争。因此,忽视以农村区城作革命根据地的观点,忽视对农民进行艰苦工作的观点,忽视游击战争的观点,都是不正确的。

但是着重武装斗争,不是说可以放弃其他形式的斗争;相反,没有武装斗争以外的各种形式的斗争相配合,武装斗争就不能胜利。着重农村根据地上的工作,不是说可以放弃城市工作及尚未成为根据地的其他广大农村中的工作;相反,没有城市工作及其他农村工作,革命根据地就处于孤立,革命就会失败。而且革命的最后目的,是夺取作为敌人主要根据地的城市,没有必要的足够的城市工作,就不能达此目的。

由此也就可以明白,为要使革命在农村与城市都胜利,不破坏敌人斗争的主要工具即敌人的军队,也是不可能的。因此,瓦解敌军的工作就成为极端重要的工作。

由此也就可以明白,在敌人长期占领的反动的黑暗的城市与反动的黑暗的农村中进行共产党的宣传工作与组织工作,不能采取急性病的冒险主义的方针,必须采取长期埋伏积蓄力量以待时机的方针。其领导人民对敌斗争的策略,必须利用一切可以利用的公开合法的法律命令及社会习惯所许可的范围,从有理、有利、有节的观点出发,一步一步与稳打稳扎的去进行,决不是大唤大叫与横冲直撞的办法所能成功的。

第三节 中国革命的任务

既然现阶段上中国革命的敌人主要是帝国主义与封建残余。那么,现阶段上中国革命的任务是什么呢?

毫无疑义,主要的就是打击这两个敌人,就是对外推翻帝国主义压迫的民族革命,对内推翻封建残余压迫的民主革命,而首先的任务便是打击帝国主义的民族革命。

中国革命的两大任务,是互相关联的。如果不推翻帝国主义的统治,就不能消灭封建残余,因为帝国主义是封建残余的主要支持者。反之,如果不肃清封建残余,也不能推翻帝国主义的统治,因为封建残余是帝国主义统治中国的主要社会基础。所以,民族革命与民主革命这样两个基本任务,是互相分别,又是互相统一的。

中国今天的民族革命任务,主要是反对侵入国土的日本帝国主义,而民主革命任务,则是在抗日战争中为了争取胜利的必要而去完成的,两个革命任务已经联系在一起了,那种把民族革命与民主革命分为截然对立的两个革命阶段的观点,已经是不合时宜的了。

第四节 中国革命的动力

根据现阶段中国社会性质、中国革命对象、中国革命任务的分析与规定,中国革命的动力是什么呢?

既然中国社会是一个殖民地、半殖民地、半封建的社会,既然中国革命所反对的对象主要的是外国帝国主义在中国的统治和内部的半封建势力,既然中国革命的任务是在推翻这两个压迫者的民族革命与民主革命,那么,在中国社会的各个阶级各个阶层中,有些什么阶级有些什么阶层可以充当反对帝国主义与反对封建势力的力量呢? 这就是现阶段上中国革命的动力问题,认清这个革命动力问题,才能正确的解决中国革命的基本策略问题。

现阶段的中国社会里,有些什么阶级呢? 有地主阶级、有资产阶级,这两个阶级都是上层统治阶级。又有无产阶级、有农民阶级、有各种类型的小资产阶级,这后面三个阶级,在今天的最广大领土上,还是被统治阶级。

所有这些阶级,他们对中国革命的态度和立场如何,全依他们在社会经济中所占的地位来决定。所以,社会经济的性质,不仅规定了革命的对象与任务,又规定了革命的动力。

我们现在就来分析一下中国社会的各阶级。

一、地主阶级

地主阶级是封建残余的代表,是帝国主义统治中国的主要社会基础,是剥削农民和压迫农民的阶级,是在政治上、经济上、文化上阻碍中国社会前进而没有丝毫利益的阶级。

因此,作为阶级来说,地主阶级是革命的对象,而不是革命的动力。

但是地主阶级中,最反动的是大地主阶层。至于中小地主,特别是破产与半破产的小地主,则有比较不同的情形。当革命还是反对帝国主义与大地主时,他们往往是能够保守中立或暂时的参加斗争的。尤其是从这个阶层出身而受过科学教育的知识分子,很多都能够这样做。

在抗日战争中,一部分大地主跟着一部分大资产阶级(投降派)已经投降日寇变为汉奸了,另一部分大地主跟着另一部分大资产阶级(顽固派),虽然还留在抗战营垒内,亦已非常的动摇。但是许多中小地主出身的开明绅士即带有若干资本主义色彩的地主们,尚有抗日的积极性,尚值得团结他们一道抗日。

二、资产阶级

资产阶级有带买办性的大资产阶级和民族资产阶级的区别。

带买办性的大资产阶级是直接为帝国主义的外国资本家服务并为他们所豢养的阶级,他们和农村中的半封建势力有着千丝万缕的联系。因此,在中国革命史上,大资产阶级历来不是中国革命的动力,而是中国革命的对象。

但因为中国带买办性的大资产阶级是分属于各个帝国主义的,在各个帝国主义间的矛盾尖锐地对立着的时候,在革命主要是反对某一个帝国主义的时候,属于别的帝国主义系统之下的买办阶级也有可能在一定程度上与一定时间内参加当前的反帝国主义战线。但一到他们的主子起来反对中国革命时,他们也就立即反对革命了。

在抗日战争中,亲日派大资产阶级(投降派),已经投降或准备投降了。欧美派大资产阶级(顽固派)虽然尚留在抗日营垒内,亦是非常动摇着,他们就是一面抗日与一面反共的两面派人物。 我们对于大资产阶级投降派的政策是把他们当作敌人看待,坚决的打倒他们。而对于大资产阶级的顽固派,则是革命的两面政策,即一方面是联合他们,因为他们还在抗日,还应该利用他们一点残余的抗日积极性;又一方面是同他们作坚决斗争,因为他们执行破坏团结的反共反人民的高压政策,没有斗争就会危害抗战与团结。

民族资产阶级是两重性的阶级。

一方面,民族资产阶级受帝国主义的压迫,及受封建残余的束缚,所以,他们同帝国主义与封建残余有矛盾。从这一方面说来,他们是革命的动力之一,在中国革命史上,他们也曾经表现过这种反帝国主义与反官僚军阀政府的积极性。

但是又一方面,由于他们在经济上、政治上的软弱性,由于他们同帝国主义与封建残余并未完全断绝经济上的联系,所以,他们又没有彻底反帝反封建的勇气。这种情形,特别在民众革命力量强大起来的时候,表现得最明显。

民族资产阶级的这种两重性,决定了他们在一定时期中和一定程度上能够参加反帝国主义与反官僚军阀政府的革命,他们可以成为革命的一种动力。而在另一时期,就有跟在大资产阶级后面,作为反革命的助手的危险。

但是在中国的民族资产阶级主要都是中等资产阶级,他们虽然在一九二七年以后一九三一年(九一八事变)以前跟随着大地主大资产阶级反对过革命,但是这个阶层基本上还没有掌握过政权,而受当政的大地主大资产阶级的反动政策所限制。在抗日时期内,这个阶层不但与大资产阶级投降派有区别,而且与大资产阶级顽固派也有区别,至今仍是我们的较好的同盟者,因此对于这个阶层采取慎重的政策是完全必要的。

三、各种类型的小资产阶级

其中有知识分子、有城市贫民、有职员、有手工业者与自由职业者、有小商人。

所有这些小资产阶级,同农民阶级,都受帝国主义、封建残余与大资产阶级的压迫,日益走向破产没落的境地。

因此,小资产阶级是革命的动力之一,是无产阶级的可靠的同盟者。小资产阶级也只有在无产阶级领导之下,才能得到解放。

我们现在就来分析一下各种类型的没有把农民包括在内的小资产阶级。

第一是知识分子和青年学生。

数十年来,中国已造成了一个很大的知识分子群与青年学生群。在这一群人中间,除去一部分接近帝国主义与大资产阶级并为他们服务而反对民众的知识分子外,一般是受帝国主义封建残余与大资产阶级的压迫,使他们遭受着失业、失学的威胁的。因此,他们有很大的革命性。他们或多或少的有了现代的科学知识,富于政治感觉,他们在现阶段的中国革命中能够起着先锋的与桥梁的作用。辛亥革命前的留学生运动、一九一九年的五四运动、一九二五年的五三十运动、一九三五年的一二九运动,就是显明的例证。尤其是广大的比较贫苦的知识分子与半知识分子,能够和工农一道,参加和拥护革命。马克思列宁主义思想在中国的广大传播与接收,首先也是在知识分子与青年学生中。 革命力量的组织与革命事业的建设,离开革命的知识分子的参加,是不能成功的。

但是知识分子在其未与民众的革命斗争打成一片,在其未下决心为民众利益服务并使其生活群众化之时,他们的思想往往是空虚的,他们的行动往往是动摇的。因此,中国的广大革命知识分子虽有先锋的与桥梁的作用,但不是所有这些知识分子都能参加革命到最后胜利的,其中一部分,到了革命的紧急关头时,就往往脱离革命队伍采取消极态度,其中少数人竟会变成革命的敌人。陈独秀、张国焘就是少数知识分子的代表。知识分子往往有一种主观的个人主义自大性,这种缺点,只有在长期群众斗争中才能洗刷干净。

第二是城市贫民。

这个阶层中,包括破产了的手工业者、小贩、离乡别井到城市寻找职业而不得的农民,以及大群依靠不定劳动维持生活的苦力。他们是一个很大的群众,他们的地位大体上和贫农的地位相当,是一种半无产阶级。他们的地位推动他们起来拥护革命,并使他们容易接受无产阶级的领导,所以他们是很好的革命力量,和贫农一样,是无产阶级的天然的同盟者。

第三是职员。

工商业机关中的职员,国家机关与文化机关中的广大月薪生活者,都属于这一类。他们是依靠出卖精神劳动或技艺而生活的人,是不剥削他人的。他们受失业威胁又非常之大,因此,也是重要的革命力量。这一类人是一个相当广大的群众。经济建设、国家建设与文化建设,是不能离开他们的。

第四是手工业者与自由职业者。

手工业者是独立生产者,是一个很大的群众,是现时中国经济建设的一个担负者。他们不但遭受外国商品竞争的打击,而且无力摆脱商业高利贷资本的罗网,所以他们能够站在革命的方面,他们也是重要的革命力量之一。他们当中的一部分是雇用少数工人的,另一部分则是不雇工人的。这后一部分人,是更加可靠的同盟者。

自由职业者例如医生等人虽是他们在思想意识上常常受资产阶级的影响,但他们是与手工业者属于同一范畴的,也是社会生活不可缺少的部门,也是受帝国主义与封建势力压迫的,所以也可以成为革命的力量。

第五是小商人。

他们一般是受帝国主义与大资产阶级的压迫,且是一个很大的群众。这一阶层的下层分子,是不剥削别人劳动,而遭受高利贷剥削的,所以他们在革命中是一支有用的力量。只有那些剥削他人劳动而又同帝国主义买办阶级或封建残余有联系的上层分子,才是对革命表现动摇态度的人们。

四、农民阶级

农民在全国总人口中占百分之八十,是现时中国国民经济的主要担负者。

农民一般都是小资产阶级,但他们的内部是在激烈分化的过程中。

第一是富农。约占农村人口百分之五左右(连地主一起共占农村人口百分之十左右),被称为农村的资产阶级。中国的富农大多带有半封建性,并与城市资产阶级联系着。但革命政府不应把富农看成与地主无分别的阶层,不应过早采取打击富农经济的政策,因为富农的生产在一定时期中是不可缺少的。

第二是中农。在中国农村人口中约占百分之二十左右。中农一般不剥削别人,在经济上能够自给自足(但在年成丰收时能有些许盈余,有时也利用一点雇佣劳动或放一点小债),而受帝国主义、地主阶级与大资产阶级的剥削,除一部分富裕中农外,多是土地不足并没有政治权利的。中农不但能够坚决参加反帝革命与土地革命,并且是能够参加社会主义革命的,因此全部中农都可以成为无产阶级的可靠的同盟者,中农是很好的革命动力之一。中农态度的向背是决定革命胜负的因素,尤其在土地革命之后,中农成了农村中的大多数的时候是如此。

第三是贫农。中国的贫农连同雇农在内,约占农村人口百分之七十。贫农是没有土地或土地不足的广大农村群众,是农村中的半无产阶级,是中国革命的最广大的动力,是无产阶级的天然的和最可靠的同盟者,是中国革命队伍的主力军。中农和贫农都只有在无产阶级的领导之下,才能得到解放;而无产阶级也只有向中农、贫农结成坚固的联盟,才能领导革命到胜利,否则是不可能的。农民这个名称所包括的内容,主要的也正是指的中农和贫农。

五、无产阶级

中国无产阶级中,现代产业工人约占二百五十万至三百万。城市手工业雇佣劳动者约占千二百万,此外还有广大的农村无产阶级。

中国无产阶级有他的许多特出的优点,使他在中国革命中能够成为领导的力量。

中国无产阶级有那些特出的优点呢?

第一、中国无产阶级身受三重压迫(帝国主义、资产阶级、封建势力),而这些压迫的严重性与残酷性,是世界各民族中少见的。因此。他们在革命斗争中,比任何别的阶级来得特别坚决和特别彻底。在殖民地半殖民地的中国又没有西欧那样的社会改良主义的经济基础(但须注意,中国民族改良主义有时容易在一部分工人中发生影响),所以除极少数的工贼之外,整个阶级都是最革命的。

第二、中国无产阶级,开始走上革命的舞台,就在本阶级的革命政党——中国共产党领导之下,成为中国社会里最有觉悟性的阶级。

第三、中国无产阶级同广大农民有一种天然的联系(由于刚从农业破产出身的成分占大多数),便利于他们同农民结成亲密的革命联盟。

因此,虽然中国无产阶级有其不可避免的弱点,例如人数较少(同农民比较)年龄较轻(同资本主义国家的无产阶级比较)、文化水准较低(同资产阶级比较);然而他们终究成为中国革命的最基本的动力, 中国革命如果没有无产阶级的参加与领导,就必然不能胜利。远之如辛亥革命,因为当时还没有无产阶级的自觉的参加,因为那时还没有共产党,所以流产了。近之如一九二五——二七年的大革命,因为这时有了无产阶级的自觉的参加,因为这时有了共产党,所以在一个时期内取得了很大的胜利。但又因为资产阶级后来背叛了他们同无产阶级的联盟,背叛了共同的革命纲领,同时也由于那时中国无产阶级及其政党还没有丰富的革命经验,结果又遭受了失败。抗战以来,因为无产阶级和共产党参加了抗日民族统一战线的领导。所以团结了全民族,发动了与坚持了伟大的抗战。

中国无产阶级,在共产党领导之下,完全懂得:他们自己虽然是一个最有觉悟性和最有组织性的阶级,但如果单凭自己一个阶级的力量,是不能胜利的,而要胜利,就必须在各种不同情形下团结一切可能的革命阶级与阶层,组织革命的统一战线。在中国社会的各阶级中,农民是工人阶级的坚固的同盟军,城市小资产阶级也是可靠的同盟军,民族资产阶级则是在一定时期中与一定程度上的同盟军,这是现代中国革命的历史所已经证明了的根本规律之一。

中国的殖民地与半殖民地地位,造成了中国农村中与城市中广大的失业人群。在这个人群中,有许多人被迫到没有任何谋生的正当途径,不得不找寻所谓不名誉的或不正当的职业过活,这就是乞丐、盗贼、流氓、娼妓与许多迷信职业家的来源。这个阶层是动摇的阶层,其中一部分容易被反动势力所收买;另一部份则颇有革命性。但是他们缺乏建设性,破坏有余而建设不足,就又成为流寇主义与无政府思想的来源。因此,应该善于引导他们,注意组织他们的革命性,而防止他们那种不正当的破坏性。

以上这些,就是我们对于中国革命动力的分析。

第五节 中国革命的性质

我们已经明白了中国社会的性质,亦即中国的特殊国情,这是解决中国一切革命问题的最基本的根据。我们又明白了中国革命的对象、中国革命的任务、中国革命的动力,这些都是由于中国社会的特殊性质,由于中国的特殊国情而发生的关于现阶段中国革命的基本问题。在明白了所有这些之后,那么,我们就可以明白现阶段中国革命的另一个基本问题,即中国革命的性质是什么了。

现阶段的中国革命究竟是一种什么性质的革命呢? 资产阶级民主主义的革命,还是无产阶级社会主义的革命呢? 显然的,不是后者,而是前者。

既然中国社会还是一个殖民地、半殖民地、半封建的社会,既然中国革命的敌人主要的还是帝国主义与半封建势力,既然中国革命的任务是在推翻这两个主要敌人的民族革命与民主革命; 而推翻这两个敌人的革命动力,有时还有民族资产阶级及一部分大资产阶级的参加,即使大资产阶级背叛革命而成了革命的敌人,革命的锋芒也不是向着一切资本主义与资本主义的私有财产,而是向着帝国主义与封建独占。即然如此,所以现阶段中国革命的性质,不是无产阶级社会主义的,而是资产阶级民主主义的。

但是现时中国的资产阶级民主主义革命,已不是旧式的一般的资产阶级民主主义革命,这种革命已经过时了,而是新式的特殊的资产阶级民主主义革命。这种革命正在中国与一切殖民地半殖民地国家发展起来,我们称这种革命为新民主主义的革命。这种新民主主义的革命是世界无产阶级社会主义革命的一部分,他是坚决反对帝国主义即国际资本主义的。他在政治上是几个革命阶级联合起来对于帝国主义者及汉奸反动派的革命民主专政,反对把中国社会造成资产阶级专政的社会。他在经济上是把帝国主义者及汉奸反动派的大资本大企业收归国家经营,把大土地分配给农民所有,同时扶助私人的中小企业,并不废除富农经济。因此,这种新式的民主革命,虽然一方面是替资本主义扫清道路,但在另一方面又是替社会主义创造前提。中国现时的革命阶段,是为了终结殖民地、半殖民地、半封建社会与建立社会主义社会之间的一个过渡阶段,是一个新民主主义的新的革命过程。这个过程是从第一次世界大战与俄国十月革命之后才发生的,在中国则是从一九一九年五四运动开始的。所谓新民主主义的革命,就是在无产阶级领导之下的人民大众反帝反封建的革命,就是各革命阶级统一战线的革命。中国必须经过这个革命,才能进一步发展到社会主义革命,否则是不可能的。

这种新民主主义的革命,与欧美各国历史上的民主革命大不相同,他不造成资产阶级专政,而是造成各革命阶级统一战线的专政。在抗日战争中,应该建立的抗日民主政权,乃是抗日民族统一战线的政权,他既不是资产阶级的“一党专政”,也不是无产阶级的“一党专政”,而是抗日民族统一战线的“几党专攻”,只要是赞成抗日又赞成民主的人们,不问属于何党何派,都有参加政权的资格。

这种新民主主义的革命也与社会主义革命不相同,他只推翻帝国主义与汉奸反动派,而不推翻任何尚能参加反帝反封建的一切资本主义成分。

这种新民主主义革命,同孙中山在一九二四年所宣布的三民主义革命(国民党第一次全国代表大会宣言)在基本上是一致的。因为孙中山在这个宣言上说:“近世各国所谓民权制度,往往为资产阶级所专有,适成为压迫平民之工具,盖国民党之民权主义,则为一般平民所共有,非少数人所得而私也。”又说:“凡本国人及外国人之企业,或有独占的性质,或规模过大为私人之力所不能办者,如银行、铁路、航路之属,由国家经营管理之,使私有资本制度不能操纵国民之生计,此则节制资本之要旨也。”孙中山又在其遗嘱上提出“必须唤起民众与以平等待我之民族共同奋斗”的内政外交的根本原则。所有这些,乃是区别于适应于旧的国际国内环境之旧民主主义的三民主义,而改造成了适应于新的国际国内环境之新民主主义的三民主义。中国共产党在一九三七年九月二十二日发表宣言,声明“三民主义为中国今日之必需,本党愿为其彻底实现而奋斗”,就是指的这种三民主义,而不是任何别的三民主义。这种三民主义即是孙中山三大政策,即联俄、联共与农工政策的三民主义,在新的国际国内条件下,离开三大政策的三民主义,就不是革命的三民主义(关于共产主义与三民主义只是在民主革命政纲上基本相同,而其它一切方面则均不相同,这一问题这里不来说他)。

这样,就使中国的资产阶级民主革命,无论就其斗争阵线(统一战线)来说,就其国家组成来说,均不能忽视无产阶级、农民阶级、知识分子与其他小资产阶级的地位,谁要是想撇开中国的无产阶级、农民阶级、知识分子与其他小资产阶级,就一定不能解决中华民族的命运,一定不能解决中国的任何问题。中国现阶段革命所要造成的民主共和国,一定要是一个工人、农民与知识分子在其内面占一定地位起一定作用的民主共和国,换言之,即是一个工人、农民、知识分子、小资产阶级与其他一切反帝反封建分子之革命联盟的民主共和国。这种共和国的彻底完成,只有在无产阶级的政策领导之下才有可能。

第六节 中国革命的前途

在将现阶段中国社会的性质,中国革命的对象、任务、动力与性质这些基本问题弄清楚了之后,那么,对于中国革命的前途问题,就是说,中国资产阶级民主革命与无产阶级社会革命的关系问题,中国革命的现在阶段与将来阶段的关系问题,也就容易明白了。

因为既然现阶段中国资产阶级民主主义的革命,不是一般的旧式的资产阶级民主主义革命,而是特殊的新式的民主主义革命,而是新民主主义的革命,而中国革命现时又是处在二十世纪四十与五十年代的新的国际环境中,即处在社会主义向上高涨,资本主义向下低落的国际环境中,处在第二次帝国主义大战中与第二次世界革命的前夜,那么,中国革命的前途,不是资本主义的,而是社会主义的,也就没有疑义了。

没有问题,现阶段的中国革命既然是为了变更现在的殖民地、半殖民地、半封建社会的地位,即为了完成一个新民主主义的革命而奋斗,那么,在革命胜利之后,因为革命肃清了资本主义发展道路上的障碍物,资本主义经济在中国社会中会有一个相当程度的发展,是可以想象到的,也是不足为怪的。资本主义有个相当程度的发展,这是经济落后的中国在民主革命胜利之后不可避免的结果,当然,不容否认这只是中国革命的一方面结果,不会是它的全部结果。中国革命的全部结果是:一方面有资本主义因素的发展,又一方面有社会主义因素的发展。这种社会主义因素是什么呢?就是无产阶级与共产党在全国政治势力中的比重的增长,农民、知识分子与小资产阶级或者已经或者可能承认无产阶级与共产党的领导权。所有这一切,便都是社会主义的因素,加以国际环境的有利,便使中国资产阶级民主革命的最后结果,避免资本主义前途,实现社会主义前途,不能不具有极大可能性了。

第七节 中国革命的两重任务与中国共产党

总结本章各节所述,我们可以明白,整个中国革命是包含着两重任务的,这就是说,中国革命是包括资产阶级民主主义性质的革命(新民主主义的革命)与无产阶级社会主义性质的革命,现在阶段的革命与将来阶段的革命这样两重任务的。而这两重革命任务的领导,都是担负在中国无产阶级的政党——共产党的双肩之上,离开了中国共产党的领导,任何革命都不能成功。

完成中国资产阶级民主主义革命(新民主主义革命),并准备在一切必要条件具备之时把他转变到社会主义的革命阶段上去,这就是中国共产党光荣的伟大的全部革命任务,每个共产党员都应为此而奋斗,绝对不能半途而废。有些幼稚的共产党员,以为我们只有现在阶段的民主主义革命的任务,没有将来阶段的社会主义革命的任务,或者说,现在的革命或土地革命即是社会主义的革命,应该着重指出,这些观点都是错误的。每个共产党员须知,整个中国的共产主义运动,是包括民主革命与社会革命两个阶段在内的全部革命运动,这是两个性质不同的革命过程,只有完成前一个革命过程才可能去完成后一个革命过程。民主革命是社会革命的必要准备,社会革命是民主革命的必然趋势。而一切共产主义者的最后目的,则是在于力争社会主义与共产主义社会的最后的完成。只有认清民主革命与社会革命的区别,同时又认清二者的联系,才能正确的领导中国革命。

领导中国民主主义革命与中国社会主义革命这样两个伟大的革命到彻底的完成,除了中国共产党之外,是没有任何一个别的政党(不论资产阶级或小资产阶级政党)能够担负的。而中国共产党则从自己建党的一天起,就把这样的两重任务放在自己的双肩之上了,并且已为此而艰苦奋斗了整整十八年。

这样的任务是非常光荣的,但同时也就是非常艰苦的,没有一个全国范围的、广大群众性的,思想上、政治上、组织上完全巩固的,布尔塞维克的中国共产党, 是不能完成的。因此,如何建设这样一个共产党,乃是每一个共产党员的责任。

以下,我们就来逐步讨论中国共产党的建设问题。

○ 四四年版毛泽东选集卷一

※ 四七年版毛泽东选集卷二

※ 中国革命与中国共产党 解放社

※ 中国革命与中国共产党 正报社 一九四八

\section{新民主主义论 1940/1}

* 这是毛泽东一九四○年一月九日在陕甘宁边区文化协会第一次代表大会上的讲演,原题为《新民主主义的政治与新民主主义的文化》,载于一九四○年二月十五日延安出版的《中国文化》创刊号。同年二月二十日在延安出版的《解放》第九十八、九十九期合刊登载时,题目改为《新民主主义论》。

一、中国向何处去

抗战以来,全国人民有一种欣欣向荣的气象,大家以为有了出路,愁眉锁眼的姿态为之一扫。但是近来的妥协空气,反共声浪,忽又甚嚣尘上,又把全国人民打入闷葫芦里了。特别是文化人和青年学生,感觉锐敏,首当其冲。于是怎么办,中国向何处去,又成为问题了。因此,趁着《中国文化》\footnote{《中国文化》是一九四○年二月在延安创刊的杂志,一九四一年八月终刊。}的出版,说明一下中国政治和中国文化的动向问题,或者也是有益的。对于文化问题,我是门外汉,想研究一下,也方在开始。好在延安许多同志已有详尽的文章,我的粗枝大叶的东西,就当作一番开台锣鼓好了。对于全国先进的文化工作者,我们的东西,只当作引玉之砖,千虑之一得,希望共同讨论,得出正确结论,来适应我们民族的需要。科学的态度是“实事求是”,“自以为是”和“好为人师”那样狂妄的态度是决不能解决问题的。我们民族的灾难深重极了,惟有科学的态度和负责的精神,能够引导我们民族到解放之路。真理只有一个,而究竟谁发现了真理,不依靠主观的夸张,而依靠客观的实践。只有千百万人民的革命实践,才是检验真理的尺度。我想,这可以算作《中国文化》出版的态度。

二、我们要建立一个新中国

我们共产党人,多年以来,不但为中国的政治革命和经济革命而奋斗,而且为中国的文化革命而奋斗;一切这些的目的,在于建设一个中华民族的新社会和新国家。在这个新社会和新国家中,不但有新政治、新经济,而且有新文化。这就是说,我们不但要把一个政治上受压迫、经济上受剥削的中国,变为一个政治上自由和经济上繁荣的中国,而且要把一个被旧文化统治因而愚昧落后的中国,变为一个被新文化统治因而文明先进的中国。一句话,我们要建立一个新中国。建立中华民族的新文化,这就是我们在文化领域中的目的。

三、中国的历史特点

我们要建立中华民族的新文化,但是这种新文化究竟是一种什么样子的文化呢?

一定的文化(当作观念形态的文化)是一定社会的政治和经济的反映,又给予伟大影响和作用于一定社会的政治和经济;而经济是基础,政治则是经济的集中的表现\footnote{“政治是经济的最集中的表现”一语,见列宁《论工会、目前局势及托洛茨基同志的错误》(《列宁全集》第40卷,人民出版社1986年版,第212页)。}。这是我们对于文化和政治、经济的关系及政治和经济的关系的基本观点。那末,一定形态的政治和经济是首先决定那一定形态的文化的;然后,那一定形态的文化又才给予影响和作用于一定形态的政治和经济。马克思说:“不是人们的意识决定人们的存在,而是人们的社会存在决定人们的意识。”\footnote{见马克思《〈政治经济学批判〉序言》(《马克思恩格斯选集》第2卷,人民出版社1972年版,第82页)。}他又说:“从来的哲学家只是各式各样地说明世界,但是重要的乃在于改造世界。”\footnote{见马克思《关于费尔巴哈的提纲》。新的译文是:“哲学家们只是用不同的方式解释世界,而问题在于改变世界。”(《马克思恩格斯选集》第1卷,人民出版社1972年版,第19页)}这是自有人类历史以来第一次正确地解决意识和存在关系问题的科学的规定,而为后来列宁所深刻地发挥了的能动的革命的反映论之基本的观点。我们讨论中国文化问题,不能忘记这个基本观点。

这样说来,问题是很清楚的,我们要革除的那种中华民族旧文化中的反动成分,它是不能离开中华民族的旧政治和旧经济的;而我们要建立的这种中华民族的新文化,它也不能离开中华民族的新政治和新经济。中华民族的旧政治和旧经济,乃是中华民族的旧文化的根据;而中华民族的新政治和新经济,乃是中华民族的新文化的根据。

所谓中华民族的旧政治和旧经济是什么?而所谓中华民族的旧文化又是什么?

自周秦以来,中国是一个封建社会,其政治是封建的政治,其经济是封建的经济。而为这种政治和经济之反映的占统治地位的文化,则是封建的文化。

自外国资本主义侵略中国,中国社会又逐渐地生长了资本主义因素以来,中国已逐渐地变成了一个殖民地、半殖民地、半封建的社会。现在的中国,在日本占领区,是殖民地社会;在国民党统治区,基本上也还是一个半殖民地社会;而不论在日本占领区和国民党统治区,都是封建半封建制度占优势的社会。这就是现时中国社会的性质,这就是现时中国的国情。作为统治的东西来说,这种社会的政治是殖民地、半殖民地、半封建的政治,其经济是殖民地、半殖民地、半封建的经济,而为这种政治和经济之反映的占统治地位的文化,则是殖民地、半殖民地、半封建的文化。

这些统治的政治、经济和文化形态,就是我们革命的对象。我们要革除的,就是这种殖民地、半殖民地、半封建的旧政治、旧经济和那为这种旧政治、旧经济服务的旧文化。而我们要建立起来的,则是与此相反的东西,乃是中华民族的新政治、新经济和新文化。

那末,什么是中华民族的新政治、新经济,又什么是中华民族的新文化呢?

中国革命的历史进程,必须分为两步,其第一步是民主主义的革命,其第二步是社会主义的革命,这是性质不同的两个革命过程。而所谓民主主义,现在已不是旧范畴的民主主义,已不是旧民主主义,而是新范畴的民主主义,而是新民主主义。

由此可以断言,所谓中华民族的新政治,就是新民主主义的政治;所谓中华民族的新经济,就是新民主主义的经济;所谓中华民族的新文化,就是新民主主义的文化。

这就是现时中国革命的历史特点。在中国从事革命的一切党派,一切人们,谁不懂得这个历史特点,谁就不能指导这个革命和进行这个革命到胜利,谁就会被人民抛弃,变为向隅而泣的可怜虫。

四、中国革命是世界革命的一部分

中国革命的历史特点是分为民主主义和社会主义两个步骤,而其第一步现在已不是一般的民主主义,而是中国式的、特殊的、新式的民主主义,而是新民主主义。那末,这个历史特点是怎样形成的呢?它是一百年来就有了的,还是后来才发生的呢?

只要研究一下中国的和世界的历史发展,就知道这个历史特点,并不是从鸦片战争以来就有了的,而是在后来,在第一次帝国主义世界大战和俄国十月革命之后,才形成的。我们现在就来研究这个形成过程。

很清楚的,中国现时社会的性质,既然是殖民地、半殖民地、半封建的性质,它就决定了中国革命必须分为两个步骤。第一步,改变这个殖民地、半殖民地、半封建的社会形态,使之变成一个独立的民主主义的社会。第二步,使革命向前发展,建立一个社会主义的社会。中国现时的革命,是在走第一步。

这个第一步的准备阶段,还是自从一八四○年鸦片战争以来,即中国社会开始由封建社会改变为半殖民地半封建社会以来,就开始了的。中经太平天国运动、中法战争、中日战争、戊戌变法、辛亥革命、五四运动、北伐战争、土地革命战争、直到今天的抗日战争,这样许多个别的阶段,费去了整整一百年工夫,从某一点上说来,都是实行这第一步,都是中国人民在不同的时间中和不同的程度上实行这第一步,实行反对帝国主义和封建势力,为了建立一个独立的民主主义的社会而斗争,为了完成第一个革命而斗争。而辛亥革命,则是在比较更完全的意义上开始了这个革命。这个革命,按其社会性质说来,是资产阶级民主主义的革命,不是无产阶级社会主义的革命。这个革命,现在还未完成,还须付与很大的气力,这是因为这个革命的敌人,直到现在,还是非常强大的缘故。孙中山先生说的“革命尚未成功,同志仍须努力”,就是指的这种资产阶级民主主义的革命。

然而中国资产阶级民主主义革命,自从一九一四年爆发第一次帝国主义世界大战和一九一七年俄国十月革命在地球六分之一的土地上建立了社会主义国家以来,起了一个变化。

在这以前,中国资产阶级民主主义革命,是属于旧的世界资产阶级民主主义革命的范畴之内的,是属于旧的世界资产阶级民主主义革命的一部分。

在这以后,中国资产阶级民主主义革命,却改变为属于新的资产阶级民主主义革命的范畴,而在革命的阵线上说来,则属于世界无产阶级社会主义革命的一部分了。

为什么呢?因为第一次帝国主义世界大战和第一次胜利的社会主义十月革命,改变了整个世界历史的方向,划分了整个世界历史的时代。

在世界资本主义战线已在地球的一角(这一角占全世界六分之一的土地)崩溃,而在其余的角上又已经充分显露其腐朽性的时代,在这些尚存的资本主义部分非更加依赖殖民地半殖民地便不能过活的时代,在社会主义国家已经建立并宣布它愿意为了扶助一切殖民地半殖民地的解放运动而斗争的时代,在各个资本主义国家的无产阶级一天一天从社会帝国主义的社会民主党的影响下面解放出来并宣布他们赞助殖民地半殖民地解放运动的时代,在这种时代,任何殖民地半殖民地国家,如果发生了反对帝国主义,即反对国际资产阶级、反对国际资本主义的革命,它就不再是属于旧的世界资产阶级民主主义革命的范畴,而属于新的范畴了;它就不再是旧的资产阶级和资本主义的世界革命的一部分,而是新的世界革命的一部分,即无产阶级社会主义世界革命的一部分了。这种革命的殖民地半殖民地,已经不能当作世界资本主义反革命战线的同盟军,而改变为世界社会主义革命战线的同盟军了。

这种殖民地半殖民地革命的第一阶段,第一步,虽然按其社会性质,基本上依然还是资产阶级民主主义的,它的客观要求,是为资本主义的发展扫清道路;然而这种革命,已经不是旧的、被资产阶级领导的、以建立资本主义的社会和资产阶级专政的国家为目的的革命,而是新的、被无产阶级领导的、以在第一阶段上建立新民主主义的社会和建立各个革命阶级联合专政的国家为目的的革命。因此,这种革命又恰是为社会主义的发展扫清更广大的道路。这种革命,在其进行中,因为敌情和同盟军的变化,又分为若干的阶段,然而其基本性质是没有变化的。

这种革命,是彻底打击帝国主义的,因此它不为帝国主义所容许,而为帝国主义所反对。但是它却为社会主义所容许,而为社会主义的国家和社会主义的国际无产阶级所援助。

因此,这种革命,就不能不变成无产阶级社会主义世界革命的一部分。

“中国革命是世界革命的一部分”,这一正确的命题,还是在一九二四年至一九二七年的中国第一次大革命时期,就提出了的。这是中国共产党人提出,而为当时一切参加反帝反封建斗争的人们所赞成的。不过那时这一理论的意义还没有发挥,以致人们还只是模糊地认识这个问题。

这种“世界革命”,已不是旧的世界革命,旧的资产阶级世界革命早已完结了;而是新的世界革命,而是社会主义的世界革命。同样,这种“一部分”,已经不是旧的资产阶级革命的一部分,而是新的社会主义革命的一部分。这是一个绝大的变化,这是自有世界历史和中国历史以来无可比拟的大变化。

中国共产党人提出的这一正确的命题,是根据斯大林的理论的。

斯大林还在一九一八年所作十月革命一周年纪念的论文时,就说道:

“十月革命的伟大的世界意义,主要的是:第一,它扩大了民族问题的范围,把它从欧洲反对民族压迫的斗争的局部问题,变为各被压迫民族、各殖民地及半殖民地从帝国主义之下解放出来的总问题;第二,它给这一解放开辟了广大的可能性和现实的道路,这就大大地促进了西方和东方的被压迫民族的解放事业,把他们吸引到胜利的反帝国主义斗争的巨流中去;第三,它从而在社会主义的西方和被奴役的东方之间架起了一道桥梁,建立了一条从西方无产者经过俄国革命到东方被压迫民族的新的反对世界帝国主义的革命战线。”\footnote{见斯大林《十月革命和民族问题》(《斯大林选集》上卷,人民出版社1979年版,第126页)。}

从这篇文章以后,斯大林曾经多次地发挥了关于论述殖民地半殖民地的革命脱离了旧范畴,改变成了无产阶级社会主义革命一部分的理论。解释得最清楚明确的,是斯大林在一九二五年六月三十日发表的同当时南斯拉夫的民族主义者争论的文章。这篇文章载在张仲实译的《斯大林论民族问题》一书上面,题目叫做《再论民族问题》。其中有这么一段:

“舍米契引证了斯大林在一九一二年年底所著《马克思主义与民族问题》那本小册子中的一个地方。那里曾说:‘在上升的资本主义的条件之下,民族的斗争是资产阶级相互之间的斗争。’显然,他企图以此来暗示他给当前历史条件下的民族运动的社会意义所下的定义是正确的。然而,斯大林那本小册子是在帝国主义战争以前写的,那时候民族问题在马克思主义者看来还不是一个具有全世界意义的问题,那时候马克思主义者关于民族自决权的基本要求不是当作无产阶级革命的一部分,而是当作资产阶级民主革命的一部分。自那时候起,国际形势已经根本地改变了,战争和俄国十月革命已把民族问题从资产阶级民主革命的一部分变成了无产阶级社会主义革命的一部分了,——要是看不清这一点,那就未免太可笑了。列宁还在一九一六年十月间,就在他的《民族自决权讨论的总结》一文中说过,民族问题中关于民族自决权的基本点,已不再是一般民主运动的一部分,它已经变成一般无产阶级的、社会主义革命的一个构成部分了。列宁以及俄国共产主义的其它代表者关于民族问题的以后的一些著作,我就不用讲了。现在,当我们由于新的历史环境而进入于一个新的时代——无产阶级革命的时代,舍米契在这一切以后却引证斯大林在俄国资产阶级民主革命时期所写的那本小册子中的一个地方,这能有什么意义呢?它只能有这样一个意义,就是舍米契是离开时间和空间,不顾到活的历史环境来引证的,因而违反了辩证法的最基本的要求,没有考虑到在某一个历史环境下是正确的东西在另一个历史环境下就可以成为不正确的。”

由此可见,有两种世界革命,第一种是属于资产阶级和资本主义范畴的世界革命。这种世界革命的时期早已过去了,还在一九一四年第一次帝国主义世界大战爆发之时,尤其是在一九一七年俄国十月革命之时,就告终结了。从此以后,开始了第二种世界革命,即无产阶级的社会主义的世界革命。这种革命,以资本主义国家的无产阶级为主力军,以殖民地半殖民地的被压迫民族为同盟军。不管被压迫民族中间参加革命的阶级、党派或个人,是何种的阶级、党派或个人,又不管他们意识着这一点与否,他们主观上了解了这一点与否,只要他们反对帝国主义,他们的革命,就成了无产阶级社会主义世界革命的一部分,他们就成了无产阶级社会主义世界革命的同盟军。

中国革命到了今天,它的意义更加增大了。在今天,是在由于资本主义的经济危机和政治危机已经一天一天把世界拖进第二次世界大战的时候;是在苏联已经到了由社会主义到共产主义的过渡期,有能力领导和援助全世界无产阶级和被压迫民族,反抗帝国主义战争,打击资本主义反动的时候;是在各资本主义国家的无产阶级正在准备打倒资本主义、实现社会主义的时候;是在中国无产阶级、农民阶级、知识分子和其它小资产阶级在中国共产党的领导之下,已经形成了一个伟大的独立的政治力量的时候。在今天,我们是处在这种时候,那末,应该不应该估计中国革命的世界意义是更加增大了呢?我想是应该的。中国革命是世界革命的伟大的一部分。

这个中国革命的第一阶段(其中又分为许多小阶段),其社会性质是新式的资产阶级民主主义的革命,还不是无产阶级社会主义的革命,但早已成了无产阶级社会主义的世界革命的一部分,现在则更成了这种世界革命的伟大的一部分,成了这种世界革命的伟大的同盟军。这个革命的第一步、第一阶段,决不是也不能建立中国资产阶级专政的资本主义的社会,而是要建立以中国无产阶级为首领的中国各个革命阶级联合专政的新民主主义的社会,以完结其第一阶段。然后,再使之发展到第二阶段,以建立中国社会主义的社会。

这就是现时中国革命的最基本的特点,这就是二十年来(从一九一九年五四运动算起)的新的革命过程,这就是现时中国革命的生动的具体的内容。

五、新民主主义的政治

中国革命分为两个历史阶段,而其第一阶段是新民主主义的革命,这是中国革命的新的历史特点。这个新的特点具体地表现在中国内部的政治关系和经济关系上又是怎样的呢?下面我们就来说明这种情形。

在一九一九年五四运动以前(五四运动发生于一九一四年第一次帝国主义大战和一九一七年俄国十月革命之后),中国资产阶级民主革命的政治指导者是中国的小资产阶级和资产阶级(他们的知识分子)。这时,中国无产阶级还没有当作一个觉悟了的独立的阶级力量登上政治的舞台,还是当作小资产阶级和资产阶级的追随者参加了革命。例如辛亥革命时的无产阶级,就是这样的阶级。

在五四运动以后,虽然中国民族资产阶级继续参加了革命,但是中国资产阶级民主革命的政治指导者,已经不是属于中国资产阶级,而是属于中国无产阶级了。这时,中国无产阶级,由于自己的长成和俄国革命的影响,已经迅速地变成了一个觉悟了的独立的政治力量了。打倒帝国主义的口号和整个中国资产阶级民主革命的彻底的纲领,是中国共产党提出的;而土地革命的实行,则是中国共产党单独进行的。

由于中国民族资产阶级是殖民地半殖民地国家的资产阶级,是受帝国主义压迫的,所以,虽然处在帝国主义时代,他们也还是在一定时期中和一定程度上,保存着反对外国帝国主义和反对本国官僚军阀政府(这后者,例如在辛亥革命时期和北伐战争时期)的革命性,可以同无产阶级、小资产阶级联合起来,反对它们所愿意反对的敌人。这是中国资产阶级和旧俄帝国的资产阶级的不同之点。在旧俄帝国,因为它已经是一个军事封建的帝国主义,是侵略别人的,所以俄国的资产阶级没有什么革命性。在那里,无产阶级的任务,是反对资产阶级,而不是联合它。在中国,因为它是殖民地半殖民地,是被人侵略的,所以中国民族资产阶级还有在一定时期中和一定程度上的革命性。在这里,无产阶级的任务,在于不忽视民族资产阶级的这种革命性,而和他们建立反帝国主义和反官僚军阀政府的统一战线。

但同时,也即是由于他们是殖民地半殖民地的资产阶级,他们在经济上和政治上是异常软弱的,他们又保存了另一种性质,即对于革命敌人的妥协性。中国的民族资产阶级,即使在革命时,也不愿意同帝国主义完全分裂,并且他们同农村中的地租剥削有密切联系,因此,他们就不愿和不能彻底推翻帝国主义,更加不愿和更加不能彻底推翻封建势力。这样,中国资产阶级民主革命的两个基本问题,两大基本任务,中国民族资产阶级都不能解决。至于中国的大资产阶级,以国民党为代表,在一九二七年至一九三七年这一个长的时期内,一直是投入帝国主义的怀抱,并和封建势力结成同盟,反对革命人民的。中国的民族资产阶级也曾在一九二七年及其以后的一个时期内一度附和过反革命。在抗日战争中,大资产阶级的一部分,以汪精卫为代表,又已投降敌人,表示了大资产阶级的新的叛变。这又是中国资产阶级同历史上欧美各国的资产阶级特别是法国的资产阶级的不同之点。在欧美各国,特别在法国,当它们还在革命时代,那里的资产阶级革命是比较彻底的;在中国,资产阶级则连这点彻底性都没有。

一方面——参加革命的可能性,又一方面——对革命敌人的妥协性,这就是中国资产阶级“一身而二任焉”的两面性。这种两面性,就是欧美历史上的资产阶级,也是同具的。大敌当前,他们要联合工农反对敌人;工农觉悟,他们又联合敌人反对工农。这是世界各国资产阶级的一般规律,不过中国资产阶级的这个特点更加突出罢了。

在中国,事情非常明白,谁能领导人民推翻帝国主义和封建势力,谁就能取得人民的信仰,因为人民的死敌是帝国主义和封建势力、而特别是帝国主义的缘故。在今日,谁能领导人民驱逐日本帝国主义,并实施民主政治,谁就是人民的救星。历史已经证明:中国资产阶级是不能尽此责任的,这个责任就不得不落在无产阶级的肩上了。

所以,无论如何,中国无产阶级、农民、知识分子和其它小资产阶级,乃是决定国家命运的基本势力。这些阶级,或者已经觉悟,或者正在觉悟起来,他们必然要成为中华民主共和国的国家构成和政权构成的基本部分,而无产阶级则是领导的力量。现在所要建立的中华民主共和国,只能是在无产阶级领导下的一切反帝反封建的人们联合专政的民主共和国,这就是新民主主义的共和国,也就是真正革命的三大政策的新三民主义共和国。

这种新民主主义共和国,一方面和旧形式的、欧美式的、资产阶级专政的、资本主义的共和国相区别,那是旧民主主义的共和国,那种共和国已经过时了;另一方面,也和苏联式的、无产阶级专政的、社会主义的共和国相区别,那种社会主义的共和国已经在苏联兴盛起来,并且还要在各资本主义国家建立起来,无疑将成为一切工业先进国家的国家构成和政权构成的统治形式;但是那种共和国,在一定的历史时期中,还不适用于殖民地半殖民地国家的革命。因此,一切殖民地半殖民地国家的革命,在一定历史时期中所采取的国家形式,只能是第三种形式,这就是所谓新民主主义共和国。这是一定历史时期的形式,因而是过渡的形式,但是不可移易的必要的形式。

因此,全世界多种多样的国家体制中,按其政权的阶级性质来划分,基本地不外乎这三种:(甲)资产阶级专政的共和国;(乙)无产阶级专政的共和国;(丙)几个革命阶级联合专政的共和国。

第一种,是旧民主主义的国家。在今天,在第二次帝国主义战争爆发之后,许多资本主义国家已经没有民主气息,已经转变或即将转变为资产阶级的血腥的军事专政了。某些地主和资产阶级联合专政的国家,可以附在这一类。

第二种,除苏联外,正在各资本主义国家中酝酿着。将来要成为一定时期中的世界统治形式。

第三种,殖民地半殖民地国家的革命所采取的过渡的国家形式。各个殖民地半殖民地国家的革命必然会有某些不同特点,但这是大同中的小异。只要是殖民地或半殖民地的革命,其国家构成和政权构成,基本上必然相同,即几个反对帝国主义的阶级联合起来共同专政的新民主主义的国家。在今天的中国,这种新民主主义的国家形式,就是抗日统一战线的形式。它是抗日的,反对帝国主义的;又是几个革命阶级联合的,统一战线的。但可惜,抗战许久了,除了共产党领导下的抗日民主根据地外,大部分地区关于国家民主化的工作基本上还未着手,日本帝国主义就利用这个最根本的弱点,大踏步地打了进来;再不变计,民族的命运是非常危险的。

这里所谈的是“国体”问题。这个国体问题,从前清末年起,闹了几十年还没有闹清楚。其实,它只是指的一个问题,就是社会各阶级在国家中的地位。资产阶级总是隐瞒这种阶级地位,而用“国民”的名词达到其一阶级专政的实际。这种隐瞒,对于革命的人民,毫无利益,应该为之清楚地指明。“国民”这个名词是可用的,但是国民不包括反革命分子,不包括汉奸。一切革命的阶级对于反革命汉奸们的专政,这就是我们现在所要的国家。

“近世各国所谓民权制度,往往为资产阶级所专有,适成为压迫平民之工具。若国民党之民权主义,则为一般平民所共有,非少数人所得而私也。”这是一九二四年在国共合作的国民党的第一次全国代表大会宣言中的庄严的声明。十六年来,国民党自己违背了这个声明,以致造成今天这样国难深重的局面。这是国民党一个绝大的错误,我们希望它在抗日的洗礼中改正这个错误。

至于还有所谓“政体”问题,那是指的政权构成的形式问题,指的一定的社会阶级取何种形式去组织那反对敌人保护自己的政权机关。没有适当形式的政权机关,就不能代表国家。中国现在可以采取全国人民代表大会、省人民代表大会、县人民代表大会、区人民代表大会直到乡人民代表大会的系统,并由各级代表大会选举政府。但必须实行无男女、信仰、财产、教育等差别的真正普遍平等的选举制,才能适合于各革命阶级在国家中的地位,适合于表现民意和指挥革命斗争,适合于新民主主义的精神。这种制度即是民主集中制。只有民主集中制的政府,才能充分地发挥一切革命人民的意志,也才能最有力量地去反对革命的敌人。“非少数人所得而私”的精神,必须表现在政府和军队的组成中,如果没有真正的民主制度,就不能达到这个目的,就叫做政体和国体不相适应。

国体——各革命阶级联合专政。政体——民主集中制。这就是新民主主义的政治,这就是新民主主义的共和国,这就是抗日统一战线的共和国,这就是三大政策的新三民主义的共和国,这就是名副其实的中华民国。我们现在虽有中华民国之名,尚无中华民国之实,循名责实,这就是今天的工作。

这就是革命的中国、抗日的中国所应该建立和决不可不建立的内部政治关系,这就是今天“建国”工作的唯一正确的方向。

六、新民主主义的经济

在中国建立这样的共和国,它在政治上必须是新民主主义的,在经济上也必须是新民主主义的。

大银行、大工业、大商业,归这个共和国的国家所有。“凡本国人及外国人之企业,或有独占的性质,或规模过大为私人之力所不能办者,如银行、铁道、航路之属,由国家经营管理之,使私有资本制度不能操纵国民之生计,此则节制资本之要旨也。”这也是国共合作的国民党的第一次全国代表大会宣言中的庄严的声明,这就是新民主主义共和国的经济构成的正确的方针。在无产阶级领导下的新民主主义共和国的国营经济是社会主义的性质,是整个国民经济的领导力量,但这个共和国并不没收其它资本主义的私有财产,并不禁止“不能操纵国民生计”的资本主义生产的发展,这是因为中国经济还十分落后的缘故。

这个共和国将采取某种必要的方法,没收地主的土地,分配给无地和少地的农民,实行中山先生“耕者有其田”的口号,扫除农村中的封建关系,把土地变为农民的私产。农村的富农经济,也是容许其存在的。这就是“平均地权”的方针。这个方针的正确的口号,就是“耕者有其田”。在这个阶段上,一般地还不是建立社会主义的农业,但在“耕者有其田”的基础上所发展起来的各种合作经济,也具有社会主义的因素。

中国的经济,一定要走“节制资本”和“平均地权”的路,决不能是“少数人所得而私”,决不能让少数资本家少数地主“操纵国民生计”,决不能建立欧美式的资本主义社会,也决不能还是旧的半封建社会。谁要是敢于违反这个方向,他就一定达不到目的,他就自己要碰破头的。

这就是革命的中国、抗日的中国应该建立和必然要建立的内部经济关系。

这样的经济,就是新民主主义的经济。

而新民主主义的政治,就是这种新民主主义经济的集中的表现。

七、驳资产阶级专政

这种新民主主义政治和新民主主义经济的共和国,是全国百分之九十以上的人民都赞成的,舍此没有第二条路走。

走建立资产阶级专政的资本主义社会之路吗?诚然,这是欧美资产阶级走过的老路,但无如国际国内的环境,都不容许中国这样做。

依国际环境说,这条路是走不通的。现在的国际环境,从基本上说来,是资本主义和社会主义斗争的环境,是资本主义向下没落,社会主义向上生长的环境。要在中国建立资产阶级专政的资本主义社会,首先是国际资本主义即帝国主义不容许。帝国主义侵略中国,反对中国独立,反对中国发展资本主义的历史,就是中国的近代史。历来中国革命的失败,都是被帝国主义绞杀的,无数革命的先烈,为此而抱终天之恨。现在是一个强大的日本帝国主义打了进来,它是要把中国变成殖民地的;现在是日本在中国发展它的资本主义,却不是什么中国发展资本主义;现在是日本资产阶级在中国专政,却不是什么中国资产阶级专政。不错,现在是帝国主义最后挣扎的时期,它快要死了,“帝国主义是垂死的资本主义”\footnote{参见列宁《帝国主义是资本主义的最高阶段》(《列宁全集》第27卷,人民出版社1990年版,第437页)。}。但是正因为它快要死了,它就更加依赖殖民地半殖民地过活,决不容许任何殖民地半殖民地建立什么资产阶级专政的资本主义社会。正因为日本帝国主义陷在严重的经济危机和政治危机的深坑之中,就是说,它快要死了,它就一定要打中国,一定要把中国变为殖民地,它就断绝了中国建立资产阶级专政和发展民族资本主义的路。

其次,是社会主义不容许。这个世界上,所有帝国主义都是我们的敌人,中国要独立,决不能离开社会主义国家和国际无产阶级的援助。这就是说,不能离开苏联的援助,不能离开日本和英、美、法、德、意各国无产阶级在其本国进行反资本主义斗争的援助。虽然不能说,中国革命的胜利一定要在日本和英、美、法、德、意各国或其中一二国的革命胜利之后,但须加上它们的力量才能胜利,这是没有疑义的。尤其是苏联的援助,是抗战最后胜利决不可少的条件。拒绝苏联的援助,革命就要失败,一九二七年以后反苏运动\footnote{指蒋介石叛变革命以后国民党政府所进行的一系列的反苏运动:一九二七年十二月十三日国民党反动派枪杀广州苏联副领事;同月十四日南京国民党政府下“绝俄令”,不承认各省苏联领事,勒令各省苏联商业机构停止营业。一九二九年七月蒋介石又受帝国主义的唆使,在东北向苏联挑衅,不久引起军事冲突。}的教训,不是异常明显的吗?现在的世界,是处在革命和战争的新时代,是资本主义决然死灭和社会主义决然兴盛的时代。在这种情形下,要在中国反帝反封建胜利之后,再建立资产阶级专政的资本主义社会,岂非是完全的梦呓?

如果说,由于特殊条件(资产阶级战胜了希腊的侵略,无产阶级的力量太薄弱),在第一次帝国主义大战和十月革命之后,还有过一个基马尔式的小小的资产阶级专政的土耳其\footnote{基马尔,又译凯末尔(一八八一——一九三八),第一次世界大战后土耳其民族商业资产阶级的代表。在第一次世界大战后,英帝国主义指使希腊对土耳其进行武装侵略,土耳其人民得到苏俄的援助,于一九二二年战胜了希腊军队。一九二三年土耳其建立了资产阶级专政的共和国,基马尔被选为总统。},那末,在第二次世界大战和苏联已经完成社会主义建设之后,就决不会再有一个土耳其,尤其决不容许有一个四亿五千万人口的土耳其。由于中国的特殊条件(资产阶级的软弱和妥协性,无产阶级的强大和革命彻底性),中国从来也没有过土耳其的那种便宜事情。一九二七年中国第一次大革命失败之后,中国的资产阶级分子不是曾经高唱过什么基马尔主义吗?然而中国的基马尔在何处?中国的资产阶级专政和资本主义社会又在何处呢?何况所谓基马尔的土耳其,最后也不能不投入英法帝国主义的怀抱,一天一天变成了半殖民地,变成了帝国主义反动世界的一部分。处在今天的国际环境中,殖民地半殖民地的任何英雄好汉们,要就是站在帝国主义战线方面,变为世界反革命力量的一部分;要就是站在反帝国主义战线方面,变为世界革命力量的一部分。二者必居其一,其它的道路是没有的\footnote{一九五八年九月二日,毛泽东在同巴西记者马罗金和杜特列夫人谈话时对这个观点作了修正。他指出:在《新民主主义论》中讲到,第二次世界大战爆发以后,殖民地和半殖民地的资产阶级,要就是站在帝国主义战线方面,要就是站在反帝国主义战线方面,没有其它的道路。事实上,这种观点只适合于一部分国家。对于印度、印度尼西亚、阿拉伯联合共和国(按:阿拉伯联合共和国,一九五八年由埃及同叙利亚合并组成。一九六一年叙利亚脱离阿联,成立阿拉伯叙利亚共和国。一九七一年阿联改名为阿拉伯埃及共和国)等国家却不适用,它们是民族主义国家。拉丁美洲也有许多这样的国家。这些国家既不站在帝国主义的一边,也不站在社会主义的一边,而站在中立的立场,不参加双方的集团,这是适合于它们现时的情况的。}。

依国内环境说,中国资产阶级应该获得了必要的教训。中国资产阶级,以大资产阶级为首,在一九二七年的革命刚刚由于无产阶级、农民和其它小资产阶级的力量而得到胜利之际,他们就一脚踢开了这些人民大众,独占革命的果实,而和帝国主义及封建势力结成了反革命联盟,并且费了九牛二虎之力,举行了十年的“剿共”战争。然而结果又怎么样呢?现在是当一个强大敌人深入国土、抗日战争已打了两年之后,难道还想抄袭欧美资产阶级已经过时了的老章程吗?过去的“剿共十年”并没有“剿”出什么资产阶级专政的资本主义社会,难道还想再来试一次吗?不错,“剿共十年”“剿”出了一个“一党专政”,但这乃是半殖民地半封建的专政。而在“剿共”四年(一九二七年至一九三一年的“九一八”)之后,就已经“剿”出了一个“满洲国”;再加六年,至一九三七年,就把一个日本帝国主义“剿”进中国本部来了。如果有人还想从今日起,再“剿”十年,那就已经是新的“剿共”典型,同旧的多少有点区别。但是这种新的“剿共”事业,不是已经有人捷足先登、奋勇担负起来了吗?这个人就是汪精卫,他已经是大名鼎鼎的新式反共人物了。谁要加进他那一伙去,那是行的,但是什么资产阶级专政呀,资本主义社会呀,基马尔主义呀,现代国家呀,一党专政呀,一个主义呀,等等花腔,岂非更加不好意思唱了吗?如果不入汪精卫一伙,要入抗日一伙,又想于抗日胜利之后,一脚踢开抗日人民,自己独占抗日成果,来一个“一党专政万岁”,又岂非近于做梦吗?抗日,抗日,是谁之力?离了工人、农民和其它小资产阶级,你就不能走动一步。谁还敢于去踢他们,谁就要变为粉碎,这又岂非成了常识范围里的东西了吗?但是中国资产阶级顽固派(我说的是顽固派),二十年来,似乎并没有得到什么教训。不见他们还在那里高叫什么“限共”、“溶共”、“反共”吗?不见他们一个《限制异党活动办法》之后,再来一个《异党问题处理办法》,再来一个《处理异党问题实施方案》吗?好家伙,这样地“限制”和“处理”下去,不知他们准备置民族命运于何地,也不知他们准备置其自身于何地?我们诚心诚意地奉劝这些先生们,你们也应该睁开眼睛看一看中国和世界,看一看国内和国外,看一看现在是什么样子,不要再重复你们的错误了。再错下去,民族命运固然遭殃,我看你们自己的事情也不大好办。这是断然的,一定的,确实的,中国资产阶级顽固派如不觉悟,他们的事情是并不美妙的,他们将得到一个自寻死路的前途。所以我们希望中国的抗日统一战线坚持下去,不是一家独霸而是大家合作,把抗日的事业弄个胜利,才是上策,否则一概是下策。这是我们共产党人的衷心劝告,“勿谓言之不预也”。

中国有一句老话:“有饭大家吃。”这是很有道理的。既然有敌大家打,就应该有饭大家吃,有事大家做,有书大家读。那种“一人独吞”、“人莫予毒”的派头,不过是封建主的老戏法,拿到二十世纪四十年代来,到底是行不通的。

我们共产党人对于一切革命的人们,是决不排斥的,我们将和所有愿意抗日到底的阶级、阶层、政党、政团以及个人,坚持统一战线,实行长期合作。但人家要排斥共产党,那是不行的;人家要分裂统一战线,那是不行的。中国必须抗战下去,团结下去,进步下去;谁要投降,要分裂,要倒退,我们是不能容忍的。

八、驳“左”倾空谈主义

不走资产阶级专政的资本主义的路,是否就可以走无产阶级专政的社会主义的路呢?

也不可能。

没有问题,现在的革命是第一步,将来要发展到第二步,发展到社会主义。中国也只有进到社会主义时代才是真正幸福的时代。但是现在还不是实行社会主义的时候。中国现在的革命任务是反帝反封建的任务,这个任务没有完成以前,社会主义是谈不到的。中国革命不能不做两步走,第一步是新民主主义,第二步才是社会主义。而且第一步的时间是相当地长,决不是一朝一夕所能成就的。我们不是空想家,我们不能离开当前的实际条件。

有些恶意的宣传家,故意混淆这两个不同的革命阶段,提倡所谓“一次革命论”,用以证明什么革命都包举在三民主义里面了,共产主义就失了存在的理由;用这种“理论”,起劲地反对共产主义和共产党,反对八路军新四军和陕甘宁边区。其目的,是想根本消灭任何革命,反对资产阶级民主革命的彻底性,反对抗日的彻底性,而为投降日寇准备舆论。这种情形,是日本帝国主义有计划地造成的。因为日本帝国主义在占领武汉后,知道单用武力不能屈服中国,乃着手于政治进攻和经济引诱。所谓政治进攻,就是在抗日阵线中诱惑动摇分子,分裂统一战线,破坏国共合作。所谓经济引诱,就是所谓“合办实业”。在华中华南,日寇允许中国资本家投资百分之五十一,日资占百分之四十九;在华北,日寇允许中国资本家投资百分之四十九,日资占百分之五十一。日寇并允许将各中国资本家原有产业,发还他们,折合计算,充作资本。这样一来,一些丧尽天良的资本家,就见利忘义,跃跃欲试。一部分资本家,以汪精卫为代表,已经投降了。再一部分资本家,躲在抗日阵线内的,也想跑去。但是他们做贼心虚,怕共产党阻挡他们的去路,更怕老百姓骂汉奸。于是打伙儿地开了个会,决议:事先要在文化界舆论界准备一下。计策已定,事不宜迟,于是雇上几个玄学鬼\footnote{毛泽东在这里是指张君劢及其一伙。张君劢在五四运动后宣扬一种自称为“新玄学”的唯心主义的哲学思想,提倡自孔孟以至宋明理学的所谓“精神文明”,同时又鼓吹“自由意志”,一九二三年引起了一场“科学与玄学”的争论,当时张君劢被称为“玄学鬼”。一九三八年十二月,他经蒋介石授意,发表《致毛泽东先生一封公开信》,主张取消八路军、新四军及陕甘宁边区,要求“将马克思主义暂搁一边”,为蒋介石张目。},再加几名托洛茨基,摇动笔杆枪,就乱唤乱叫、乱打乱刺了一顿。于是什么“一次革命论”呀,共产主义不适合中国国情呀,共产党在中国没有存在之必要呀,八路军新四军破坏抗日、游而不击呀,陕甘宁边区是封建割据呀,共产党不听话、不统一、有阴谋、要捣乱呀,来这么一套,骗那些不知世事的人,以便时机一到,资本家们就很有理由地去拿百分之四十九或五十一,而把全民族的利益一概卖给敌人。这个叫做偷梁换柱,实行投降之前的思想准备或舆论准备。这班先生们,像煞有介事地提倡“一次革命论”,反对共产主义和共产党,却原来不为别的,专为百分之四十九或五十一,其用心亦良苦矣。“一次革命论”者,不要革命论也,这就是问题的本质。

但是还有另外一些人,他们似乎并无恶意,也迷惑于所谓“一次革命论”,迷惑于所谓“举政治革命与社会革命毕其功于一役”的纯主观的想头;而不知革命有阶段之分,只能由一个革命到另一个革命,无所谓“毕其功于一役”。这种观点,混淆革命的步骤,降低对于当前任务的努力,也是很有害的。如果说,两个革命阶段中,第一个为第二个准备条件,而两个阶段必须衔接,不容横插一个资产阶级专政的阶段,这是正确的,这是马克思主义的革命发展论。如果说,民主革命没有自己的一定任务,没有自己的一定时间,而可以把只能在另一个时间去完成的另一任务,例如社会主义的任务,合并在民主主义任务上面去完成,这个叫做“毕其功于一役”,那就是空想,而为真正的革命者所不取的。

九、驳顽固派

于是资产阶级顽固派就跑出来说:好,你们共产党既然把社会主义社会制度推到后一个阶段去了,你们既然又宣称“三民主义为中国今日之必需,本党愿为其彻底实现而奋斗”\footnote{见一九三七年九月二十二日发表的《中共中央为公布国共合作宣言》。},那末,就把共产主义暂时收起好了。这种议论,在所谓“一个主义”的标题之下,已经变成了狂妄的叫嚣。这种叫嚣,其本质就是顽固分子们的资产阶级专制主义。但为了客气一点,叫它作毫无常识,也是可以的。

共产主义是无产阶级的整个思想体系,同时又是一种新的社会制度。这种思想体系和社会制度,是区别于任何别的思想体系和任何别的社会制度的,是自有人类历史以来,最完全最进步最革命最合理的。封建主义的思想体系和社会制度,是进了历史博物馆的东西了。资本主义的思想体系和社会制度,已有一部分进了博物馆(在苏联);其余部分,也已“日薄西山,气息奄奄,人命危浅,朝不虑夕”,快进博物馆了。惟独共产主义的思想体系和社会制度,正以排山倒海之势,雷霆万钧之力,磅礴于全世界,而葆其美妙之青春。中国自有科学的共产主义以来,人们的眼界是提高了,中国革命也改变了面目。中国的民主革命,没有共产主义去指导是决不能成功的,更不必说革命的后一阶段了。这也就是资产阶级顽固派为什么要那样叫嚣和要求“收起”它的原因。其实,这是“收起”不得的,一收起,中国就会亡国。现在的世界,依靠共产主义做救星;现在的中国,也正是这样。

谁人不知,关于社会制度的主张,共产党是有现在的纲领和将来的纲领,或最低纲领和最高纲领两部分的。在现在,新民主主义,在将来,社会主义,这是有机构成的两部分,而为整个共产主义思想体系所指导的。因为共产党的最低纲领和三民主义的政治原则基本上相同,就狂叫“收起”共产主义,岂非荒谬绝伦之至?在共产党人,正因三民主义的政治原则有和自己的最低纲领基本上相同之点,所以才有可能承认“三民主义为抗日统一战线的政治基础”,才有可能承认“三民主义为中国今日之必需,本党愿为其彻底实现而奋斗”,否则就没有这种可能了。这是共产主义和三民主义在民主革命阶段上的统一战线,孙中山所谓“共产主义是三民主义的好朋友”\footnote{见一九二四年孙中山《三民主义·民生主义》第二讲(《孙中山全集》第9卷,中华书局1986年版,第386页)。},也正是指的这种统一战线。否认共产主义,实际上就是否认统一战线。顽固派也正是要奉行其一党主义,否认统一战线,才造出那些否认共产主义的荒谬说法来。

“一个主义”也不通。在阶级存在的条件之下,有多少阶级就有多少主义,甚至一个阶级的各集团中还各有各的主义。现在封建阶级有封建主义,资产阶级有资本主义,佛教徒有佛教主义,基督徒有基督主义,农民有多神主义,近年还有人提倡什么基马尔主义,法西斯主义,唯生主义\footnote{一九三三年,国民党中央组织部部长陈立夫发表《唯生论》一书,宣扬宇宙的实质是“生命之流”,万物的根本问题在于“求生”,用来反对阶级斗争的学说;并认为宇宙万物各有一个重心,以人类社会现象来说,就是只能有一个领袖,否则就无法维持其均衡和生存。这种唯生主义的理论是为国民党反动派实行法西斯专政服务的。},“按劳分配主义”\footnote{山西军阀阎锡山曾标榜过“按劳分配”的口号。其主要内容是:用军事方法强迫劳动人民在村公所控制的固定份地上,或官办的工厂、商店里,从事农奴式的劳动,只将很小一部分劳动果实,按劳动情况分配给劳动者。},为什么无产阶级不可以有一个共产主义呢?既然是数不清的主义,为什么见了共产主义就高叫“收起”呢?讲实在话,“收起”是不行的,还是比赛吧。谁把共产主义比输了,我们共产党人自认晦气。如若不然,那所谓“一个主义”的反民权主义的作风,还是早些“收起”吧!

为了免除误会,并使顽固派开开眼界起见,关于三民主义和共产主义的异同,有清楚指明之必要。

三民主义和共产主义两个主义比较起来,有相同的部分,也有不同的部分。

第一,相同部分。这就是两个主义在中国资产阶级民主革命阶段上的基本政纲。一九二四年孙中山重新解释的三民主义中的革命的民族主义、民权主义和民生主义这三个政治原则,同共产主义在中国民主革命阶段的政纲,基本上是相同的。由于这些相同,并由于三民主义见之实行,就有两个主义两个党的统一战线。忽视这一方面,是错误的。

第二,不同部分。则有:(一)民主革命阶段上一部分纲领的不相同。共产主义的全部民主革命政纲中有彻底实现人民权力、八小时工作制和彻底的土地革命纲领,三民主义则没有这些部分。如果它不补足这些,并且准备实行起来,那对于民主政纲就只是基本上相同,不能说完全相同。(二)有无社会主义革命阶段的不同。共产主义于民主革命阶段之外,还有一个社会主义革命阶段,因此,于最低纲领之外,还有一个最高纲领,即实现社会主义和共产主义社会制度的纲领。三民主义则只有民主革命阶段,没有社会主义革命阶段,因此它就只有最低纲领,没有最高纲领,即没有建立社会主义和共产主义社会制度的纲领。(三)宇宙观的不同。共产主义的宇宙观是辩证唯物论和历史唯物论,三民主义的宇宙观则是所谓民生史观,实质上是二元论或唯心论,二者是相反的。(四)革命彻底性的不同。共产主义者是理论和实践一致的,即有革命彻底性。三民主义者除了那些最忠实于革命和真理的人们之外,是理论和实践不一致的,讲的和做的互相矛盾,即没有革命彻底性。上述这些,都是两者的不同部分。由于这些不同,共产主义者和三民主义者之间就有了差别。忽视这种差别,只看见统一方面,不看见矛盾方面,无疑是非常错误的。

明白了这些之后,就可以明白,资产阶级顽固派要求“收起”共产主义,这是什么意思呢?不是资产阶级的专制主义,就是毫无常识了。

一○、旧三民主义和新三民主义

资产阶级顽固派完全不知道历史的变化,其知识的贫乏几等于零。他们既不知道共产主义和三民主义的区别,也不知道新三民主义和旧三民主义的区别。

我们共产党人承认“三民主义为抗日民族统一战线的政治基础”,承认“三民主义为中国今日之必需,本党愿为其彻底实现而奋斗”,承认共产主义的最低纲领和三民主义的政治原则基本上相同。但是这种三民主义是什么三民主义呢?这种三民主义不是任何别的三民主义,乃是孙中山先生在《中国国民党第一次全国代表大会宣言》中所重新解释的三民主义。我愿顽固派先生们,于其“限共”、“溶共”、“反共”等工作洋洋得意之余,也去翻阅一下这个宣言。原来孙中山先生在这个宣言中说道:“国民党之三民主义,其真释具如此。”就可知只有这种三民主义,才是真三民主义,其它都是伪三民主义。只有《中国国民党第一次全国代表大会宣言》里对于三民主义的解释才是“真释”,其它一切都是伪释。这大概不是共产党“造谣”吧,这篇宣言的通过,我和很多的国民党员都是亲眼看见的。

这篇宣言,区分了三民主义的两个历史时代。在这以前,三民主义是旧范畴的三民主义,是旧的半殖民地资产阶级民主革命的三民主义,是旧民主主义的三民主义,是旧三民主义。

在这以后,三民主义是新范畴的三民主义,是新的半殖民地资产阶级民主革命的三民主义,是新民主主义的三民主义,是新三民主义。只有这种三民主义,才是新时期的革命的三民主义。

这种新时期的革命的三民主义,新三民主义或真三民主义,是联俄、联共、扶助农工三大政策的三民主义。没有三大政策,或三大政策缺一,在新时期中,就都是伪三民主义,或半三民主义。

第一,革命的三民主义,新三民主义,或真三民主义,必须是联俄的三民主义。现在的事情非常明白,如果没有联俄政策,不同社会主义国家联合,那就必然是联帝政策,必然同帝国主义联合。不见一九二七年之后,就已经有过这种情形吗?社会主义的苏联和帝国主义之间的斗争一经进一步尖锐化,中国不站在这方面,就要站在那方面,这是必然的趋势。难道不可以不偏不倚吗?这是梦想。全地球都要卷进这两个战线中去,在今后的世界中,“中立”只是骗人的名词。何况中国是在同一个深入国土的帝国主义奋斗,没有苏联帮助,就休想最后胜利。如果舍联俄而联帝,那就必须将“革命”二字取消,变成反动的三民主义。归根结底,没有“中立”的三民主义,只有革命的或反革命的三民主义。如果照汪精卫从前的话,来一个“夹攻中的奋斗”\footnote{汪精卫在一九二七年叛变革命之后不久写过一篇东西,题为《夹攻之奋斗》(载1927年7月25日《汉口民国日报》)。},来一个“夹攻中奋斗”的三民主义,岂不勇矣哉?但可惜连发明人汪精卫也放弃(或“收起”)了这种三民主义,他现在改取了联帝的三民主义。如果说帝亦有东帝西帝之分,他联的是东帝,我和他相反,联一批西帝,东向而击,又岂不革命矣哉?但无如西帝们要反苏反共,你联它们,它们就要请你北向而击,你革命也革不成。所有这些情形,就规定了革命的三民主义,新三民主义,或真三民主义,必须是联俄的三民主义,决不能是同帝国主义联合反俄的三民主义。

第二,革命的三民主义,新三民主义,或真三民主义,必须是联共的三民主义。如不联共,就要反共。反共是日本帝国主义和汪精卫的政策,你也要反共,那很好,他们就请你加入他们的反共公司。但这岂非有点当汉奸的嫌疑吗?我不跟日本走,单跟别国走。那也滑稽。不管你跟谁走,只要反共,你就是汉奸,因为你不能再抗日。我独立反共。那是梦话。岂有殖民地半殖民地的好汉们,能够不靠帝国主义之力,干得出如此反革命大事吗?昔日差不多动员了全世界帝国主义的气力反了十年之久还没有反了的共,今日忽能“独立”反之吗?听说外边某些人有这么一句话:“反共好,反不了。”如果传言非虚,那末,这句话只有一半是错的,“反共”有什么“好”呢?却有一半是对的,“反共”真是“反不了”。其原因,基本上不在于“共”而在于老百姓,因为老百姓欢喜“共”,却不欢喜“反”。老百姓是决不容情的,在一个民族敌人深入国土之时,你要反共,他们就要了你的命。这是一定的,谁要反共谁就要准备变成齑粉。如果没有决心准备变自己为齑粉的话,那就确实以不反为妙。这是我们向一切反共英雄们的诚恳的劝告。因之清楚而又清楚,今日的三民主义,必须是联共的三民主义,否则,三民主义就要灭亡。这是三民主义的存亡问题。联共则三民主义存,反共则三民主义亡,谁能证明其不然呢?

第三,革命的三民主义,新三民主义,或真三民主义,必须是农工政策的三民主义。不要农工政策,不真心实意地扶助农工,不实行《总理遗嘱》上的“唤起民众”,那就是准备革命失败,也就是准备自己失败。斯大林说:“所谓民族问题,实质上就是农民问题。”\footnote{一九二五年三月三十日,斯大林在共产国际执行委员会南斯拉夫委员会会议上的演讲《论南斯拉夫的民族问题》中说:“……农民是民族运动的主力军,没有农民这支军队,就没有而且也不可能有声势浩大的民族运动。所谓民族问题实质上是农民问题,正是指这一点说的。”(《斯大林全集》第7卷,人民出版社1958年版,第61页)}这就是说,中国的革命实质上是农民革命,现在的抗日,实质上是农民的抗日。新民主主义的政治,实质上就是授权给农民。新三民主义,真三民主义,实质上就是农民革命主义。大众文化,实质上就是提高农民文化。抗日战争,实质上就是农民战争。现在是“上山主义”\footnote{在中国共产党内,曾经有些教条主义者讥笑毛泽东注重农村革命根据地为“上山主义”。毛泽东在这里是用教条主义者的这句讽刺话,说明农村革命根据地的伟大作用。}的时候,大家开会、办事、上课、出报、着书、演剧,都在山头上,实质上都是为的农民。抗日的一切,生活的一切,实质上都是农民所给。说“实质上”,就是说基本上,并非忽视其它部分,这是斯大林自己解释过了的。中国有百分之八十的人口是农民,这是小学生的常识。因此农民问题,就成了中国革命的基本问题,农民的力量,是中国革命的主要力量。农民之外,中国人口中第二个部分就是工人。中国有产业工人数百万,有手工业工人和农业工人数千万。没有各种工业工人,中国就不能生活,因为他们是工业经济的生产者。没有近代工业工人阶级,革命就不能胜利,因为他们是中国革命的领导者,他们最富于革命性。在这种情形下,革命的三民主义,新三民主义或真三民主义,必然是农工政策的三民主义。如果有什么一种三民主义,它是没有农工政策的,它是并不真心实意扶助农工,并不实行“唤起民众”的,那就一定会灭亡。

由此可知,离开联俄、联共、扶助农工三大政策的三民主义,是没有前途的。一切有良心的三民主义者,必须认真地考虑到这点。

这种三大政策的三民主义,革命的三民主义,新三民主义,真三民主义,是新民主主义的三民主义,是旧三民主义的发展,是孙中山先生的大功劳,是在中国革命作为社会主义世界革命一部分的时代产生的。只有这种三民主义,中国共产党才称之为“中国今日之必需”,才宣布“愿为其彻底实现而奋斗”。只有这种三民主义,才和中国共产党在民主革命阶段中的政纲,即其最低纲领,基本上相同。

至于旧三民主义,那是中国革命旧时期的产物。那时的俄国是帝国主义的俄国,当然不能有联俄政策;那时国内也没有共产党,当然不能有联共政策;那时工农运动也没有充分显露自己在政治上的重要性,尚不为人们所注意,当然就没有联合工农的政策。因此,一九二四年国民党改组以前的三民主义,乃是旧范畴的三民主义,乃是过时了的三民主义。如不把它发展到新三民主义,国民党就不能前进。聪明的孙中山看到了这一点,得了苏联和中国共产党的助力,把三民主义重新作了解释,遂获得了新的历史特点,建立了三民主义同共产主义的统一战线,建立了第一次国共合作,取得了全国人民的同情,举行了一九二四年至一九二七年的革命。

旧三民主义在旧时期内是革命的,它反映了旧时期的历史特点。但如果在新时期内,在新三民主义已经建立之后,还要翻那老套;在有了社会主义国家以后,要反对联俄;在有了共产党之后,要反对联共;在工农已经觉悟并显示了自己的政治威力之后,要反对农工政策;那末,它就是不识时务的反动的东西了。一九二七年以后的反动,就是这种不识时务的结果。语曰:“识时务者为俊杰。”我愿今日的三民主义者记取此语。

如果是旧范畴的三民主义,那就同共产主义的最低纲领没有什么基本上相同之点,因为它是旧时期的,是过时了的。如果有什么一种三民主义,它要反俄、反共、反农工,那就是反动的三民主义,它不但和共产主义的最低纲领没有丝毫相同之点,而且是共产主义的敌人,一切都谈不上。这也是三民主义者应该慎重地考虑一番的。

但是无论如何,在反帝反封建的任务没有基本上完成以前,新三民主义是不会被一切有良心的人们放弃的。放弃它的只是那些汪精卫、李精卫之流。汪精卫、李精卫们尽管起劲地干什么反俄、反共、反农工的伪三民主义,自会有一班有良心的有正义感的人们继续拥护孙中山的真三民主义。如果说,一九二七年反动之后,还有许多真三民主义者继续为中国革命而奋斗,那末,在一个民族敌人深入国土的今天,这种人无疑将是成千成万的。我们共产党人将始终和一切真诚的三民主义者实行长期合作,除了汉奸和那班至死不变的反共分子外,我们是决不抛弃任何友人的。

一一、新民主主义的文化

上面,我们说明了中国政治在新时期中的历史特点,说明了新民主主义共和国问题。下面,我们就可以进到文化问题了。

一定的文化是一定社会的政治和经济在观念形态上的反映。在中国,有帝国主义文化,这是反映帝国主义在政治上经济上统治或半统治中国的东西。这一部分文化,除了帝国主义在中国直接办理的文化机关之外,还有一些无耻的中国人也在提倡。一切包含奴化思想的文化,都属于这一类。在中国,又有半封建文化,这是反映半封建政治和半封建经济的东西,凡属主张尊孔读经、提倡旧礼教旧思想、反对新文化新思想的人们,都是这类文化的代表。帝国主义文化和半封建文化是非常亲热的两兄弟,它们结成文化上的反动同盟,反对中国的新文化。这类反动文化是替帝国主义和封建阶级服务的,是应该被打倒的东西。不把这种东西打倒,什么新文化都是建立不起来的。不破不立,不塞不流,不止不行,它们之间的斗争是生死斗争。

至于新文化,则是在观念形态上反映新政治和新经济的东西,是替新政治新经济服务的。

如我们在第三节中已经提过的话,中国自从发生了资本主义经济以来,中国社会就逐渐改变了性质,它不是完全的封建社会了,变成了半封建社会,虽然封建经济还是占优势。这种资本主义经济,对于封建经济说来,它是新经济。同这种资本主义新经济同时发生和发展着的新政治力量,就是资产阶级、小资产阶级和无产阶级的政治力量。而在观念形态上作为这种新的经济力量和新的政治力量之反映并为它们服务的东西,就是新文化。没有资本主义经济,没有资产阶级、小资产阶级和无产阶级,没有这些阶级的政治力量,所谓新的观念形态,所谓新文化,是无从发生的。

新的政治力量,新的经济力量,新的文化力量,都是中国的革命力量,它们是反对旧政治旧经济旧文化的。这些旧东西是由两部分合成的,一部分是中国自己的半封建的政治经济文化,另一部分是帝国主义的政治经济文化,而以后者为盟主。所有这些,都是坏东西,都是应该彻底破坏的。中国社会的新旧斗争,就是人民大众(各革命阶级)的新势力和帝国主义及封建阶级的旧势力之间的斗争。这种新旧斗争,即是革命和反革命的斗争。这种斗争的时间,从鸦片战争算起,已经整整一百年了;从辛亥革命算起,也有了差不多三十年了。

但是如前所说,革命亦有新旧之分,在某一历史时期是新的东西,在另一历史时期就变为旧的了。在中国资产阶级民主革命的一百年中,分为前八十年和后二十年两个大段落。这两大段落中,各有一个基本的带历史性质的特点,即在前八十年,中国资产阶级民主革命是属于旧范畴的;而在后二十年,由于国际国内政治形势的变化,便属于新范畴了。旧民主主义——前八十年的特点。新民主主义——后二十年的特点。这种区别,在政治上如此,在文化上也是如此。

在文化上如何表现这种区别呢?这就是我们要在下面说明的问题。

一二、中国文化革命的历史特点

在中国文化战线或思想战线上,“五四”以前和“五四”以后,构成了两个不同的历史时期。

在“五四”以前,中国文化战线上的斗争,是资产阶级的新文化和封建阶级的旧文化的斗争。在“五四”以前,学校与科举之争\footnote{“学校”指当时效法欧美资本主义国家的教育制度。“科举”指中国原有的封建考试制度。十九世纪末,中国提倡“维新”的知识分子主张废除科举,兴办学校;封建顽固派竭力反对这种主张。},新学与旧学之争,西学与中学之争,都带着这种性质。那时的所谓学校、新学、西学,基本上都是资产阶级代表们所需要的自然科学和资产阶级的社会政治学说(说基本上,是说那中间还夹杂了许多中国的封建余毒在内)。在当时,这种所谓新学的思想,有同中国封建思想作斗争的革命作用,是替旧时期的中国资产阶级民主革命服务的。可是,因为中国资产阶级的无力和世界已经进到帝国主义时代,这种资产阶级思想只能上阵打几个回合,就被外国帝国主义的奴化思想和中国封建主义的复古思想的反动同盟所打退了,被这个思想上的反动同盟军稍稍一反攻,所谓新学,就偃旗息鼓,宣告退却,失了灵魂,而只剩下它的躯壳了。旧的资产阶级民主主义文化,在帝国主义时代,已经腐化,已经无力了,它的失败是必然的。

“五四”以后则不然。在“五四”以后,中国产生了完全崭新的文化生力军,这就是中国共产党人所领导的共产主义的文化思想,即共产主义的宇宙观和社会革命论。五四运动是在一九一九年,中国共产党的成立和劳动运动的真正开始是在一九二一年,均在第一次世界大战和十月革命之后,即在民族问题和殖民地革命运动在世界上改变了过去面貌之时,在这里中国革命和世界革命的联系,是非常之显然的。由于中国政治生力军即中国无产阶级和中国共产党登上了中国的政治舞台,这个文化生力军,就以新的装束和新的武器,联合一切可能的同盟军,摆开了自己的阵势,向着帝国主义文化和封建文化展开了英勇的进攻。这支生力军在社会科学领域和文学艺术领域中,不论在哲学方面,在经济学方面,在政治学方面,在军事学方面,在历史学方面,在文学方面,在艺术方面(又不论是戏剧,是电影,是音乐,是雕刻,是绘画),都有了极大的发展。二十年来,这个文化新军的锋芒所向,从思想到形式(文字等),无不起了极大的革命。其声势之浩大,威力之猛烈,简直是所向无敌的。其动员之广大,超过中国任何历史时代。而鲁迅,就是这个文化新军的最伟大和最英勇的旗手。鲁迅是中国文化革命的主将,他不但是伟大的文学家,而且是伟大的思想家和伟大的革命家。鲁迅的骨头是最硬的,他没有丝毫的奴颜和媚骨,这是殖民地半殖民地人民最可宝贵的性格。鲁迅是在文化战线上,代表全民族的大多数,向着敌人冲锋陷阵的最正确、最勇敢、最坚决、最忠实、最热忱的空前的民族英雄。鲁迅的方向,就是中华民族新文化的方向。

在“五四”以前,中国的新文化,是旧民主主义性质的文化,属于世界资产阶级的资本主义的文化革命的一部分。在“五四”以后,中国的新文化,却是新民主主义性质的文化,属于世界无产阶级的社会主义的文化革命的一部分。

在“五四”以前,中国的新文化运动,中国的文化革命,是资产阶级领导的,他们还有领导作用。在“五四”以后,这个阶级的文化思想却比较它的政治上的东西还要落后,就绝无领导作用,至多在革命时期在一定程度上充当一个盟员,至于盟长资格,就不得不落在无产阶级文化思想的肩上。这是铁一般的事实,谁也否认不了的。

所谓新民主主义的文化,就是人民大众反帝反封建的文化;在今日,就是抗日统一战线的文化。这种文化,只能由无产阶级的文化思想即共产主义思想去领导,任何别的阶级的文化思想都是不能领导了的。所谓新民主主义的文化,一句话,就是无产阶级领导的人民大众的反帝反封建的文化。

一三、四个时期

文化革命是在观念形态上反映政治革命和经济革命,并为它们服务的。在中国,文化革命,和政治革命同样,有一个统一战线。

这种文化革命的统一战线,二十年来,分为四个时期。第一个时期是一九一九年到一九二一年的两年,第二个时期是一九二一年到一九二七年的六年,第三个时期是一九二七年到一九三七年的十年,第四个时期是一九三七年到现在的三年。

第一个时期是一九一九年五四运动到一九二一年中国共产党成立。这一时期中以五四运动为主要的标志。

五四运动是反帝国主义的运动,又是反封建的运动。五四运动的杰出的历史意义,在于它带着为辛亥革命还不曾有的姿态,这就是彻底地不妥协地反帝国主义和彻底地不妥协地反封建主义。五四运动所以具有这种性质,是在当时中国的资本主义经济已有进一步的发展,当时中国的革命知识分子眼见得俄、德、奥三大帝国主义国家已经瓦解,英、法两大帝国主义国家已经受伤,而俄国无产阶级已经建立了社会主义国家,德、奥(匈牙利)、意三国无产阶级在革命中,因而发生了中国民族解放的新希望。五四运动是在当时世界革命号召之下,是在俄国革命号召之下,是在列宁号召之下发生的。五四运动是当时无产阶级世界革命的一部分。五四运动时期虽然还没有中国共产党,但是已经有了大批的赞成俄国革命的具有初步共产主义思想的知识分子。五四运动,在其开始,是共产主义的知识分子、革命的小资产阶级知识分子和资产阶级知识分子(他们是当时运动中的右翼)三部分人的统一战线的革命运动。它的弱点,就在只限于知识分子,没有工人农民参加。但发展到六三运动\footnote{一九一九年的五四爱国运动,至六月初转入一个新的阶段,以六月三日北京学生反抗军警镇压,集会讲演开始,由学生的罢课,发展到上海、南京、天津、杭州、武汉、九江及山东、安徽各地的工人罢工,商人罢市。五四运动至此遂成为有无产阶级、城市小资产阶级和民族资产阶级参加的广大群众运动。}时,就不但是知识分子,而且有广大的无产阶级、小资产阶级和资产阶级参加,成了全国范围的革命运动了。五四运动所进行的文化革命则是彻底地反对封建文化的运动,自有中国历史以来,还没有过这样伟大而彻底的文化革命。当时以反对旧道德提倡新道德、反对旧文学提倡新文学为文化革命的两大旗帜,立下了伟大的功劳。这个文化运动,当时还没有可能普及到工农群众中去。它提出了“平民文学”口号,但是当时的所谓“平民”,实际上还只能限于城市小资产阶级和资产阶级的知识分子,即所谓市民阶级的知识分子。五四运动是在思想上和干部上准备了一九二一年中国共产党的成立,又准备了五卅运动和北伐战争。当时的资产阶级知识分子,是五四运动的右翼,到了第二个时期,他们中间的大部分就和敌人妥协,站在反动方面了。

第二个时期,以中国共产党的成立和五卅运动、北伐战争为标志,继续了并发展了五四运动时三个阶级的统一战线,吸引了农民阶级加入,并且在政治上形成了这个各阶级的统一战线,这就是第一次国共两党的合作。孙中山先生之所以伟大,不但因为他领导了伟大的辛亥革命(虽然是旧时期的民主革命),而且因为他能够“适乎世界之潮流,合乎人群之需要”,提出了联俄、联共、扶助农工三大革命政策,对三民主义作了新的解释,树立了三大政策的新三民主义。在这以前,三民主义是和教育界、学术界、青年界没有多大联系的,因为它没有提出反帝国主义的口号,也没有提出反封建社会制度和反封建文化思想的口号。在这以前,它是旧三民主义,这种三民主义是被人们看成为一部分人为了夺取政府权力,即是说为了做官,而临时应用的旗帜,看成为纯粹政治活动的旗帜。在这以后,出现了三大政策的新三民主义。由于国共两党的合作,由于两党革命党员的努力,这种新三民主义便被推广到了全中国,推广到了一部分教育界、学术界和广大青年学生之中。这完全是因为原来的三民主义发展成了反帝反封建的三大政策的新民主主义的三民主义之故;没有这一发展,三民主义思想的传播是不可能的。

在这一时期中,这种革命的三民主义,成了国共两党和各个革命阶级的统一战线的政治基础,“共产主义是三民主义的好朋友”,两个主义结成了统一战线。以阶级论,则是无产阶级、农民阶级、城市小资产阶级、资产阶级的统一战线。那时,以共产党的《向导周报》\footnote{《向导》周报是中共中央的机关报,一九二二年九月十三日在上海创刊,一九二七年七月十八日在武汉终刊。},国民党的上海《民国日报》\footnote{上海《民国日报》于一九一六年一月创刊,国民党一大后正式成为国民党的机关报。在中国共产党的影响和国民党左派的努力下,曾经宣传过反对帝国主义和反对封建主义的主张。一九二五年十一月以后,曾被西山会议派把持,成为国民党右派的报纸。一九四七年停刊。}及各地报纸为阵地,曾经共同宣传了反帝国主义的主张,共同反对了尊孔读经的封建教育,共同反对了封建古装的旧文学和文言文,提倡了以反帝反封建为内容的新文学和白话文。在广东战争和北伐战争中,曾经在中国军队中灌输了反帝反封建的思想,改造了中国的军队。在千百万农民群众中,提出了打倒贪官污吏打倒土豪劣绅的口号,掀起了伟大的农民革命斗争。由于这些,再由于苏联的援助,就取得了北伐的胜利。但是大资产阶级一经爬上了政权,就立即结束了这次革命,转入了新的政治局面。

第三个时期是一九二七年至一九三七年的新的革命时期。因为在前一时期的末期,革命营垒中发生了变化,中国大资产阶级转到了帝国主义和封建势力的反革命营垒,民族资产阶级也附和了大资产阶级,革命营垒中原有的四个阶级,这时剩下了三个,剩下了无产阶级、农民阶级和其它小资产阶级(包括革命知识分子),所以这时候,中国革命就不得不进入一个新的时期,而由中国共产党单独地领导群众进行这个革命。这一时期,是一方面反革命的“围剿”,又一方面革命深入的时期。这时有两种反革命的“围剿”:军事“围剿”和文化“围剿”。也有两种革命深入:农村革命深入和文化革命深入。这两种“围剿”,在帝国主义策动之下,曾经动员了全中国和全世界的反革命力量,其时间延长至十年之久,其残酷是举世未有的,杀戮了几十万共产党员和青年学生,摧残了几百万工人农民。从当事者看来,似乎以为共产主义和共产党是一定可以“剿尽杀绝”的了。但结果却相反,两种“围剿”都惨败了。作为军事“围剿”的结果的东西,是红军的北上抗日;作为文化“围剿”的结果的东西,是一九三五年“一二九”青年革命运动的爆发。而作为这两种“围剿”之共同结果的东西,则是全国人民的觉悟。这三者都是积极的结果。其中最奇怪的,是共产党在国民党统治区域内的一切文化机关中处于毫无抵抗力的地位,为什么文化“围剿”也一败涂地了?这还不可以深长思之吗?而共产主义者的鲁迅,却正在这一“围剿”中成了中国文化革命的伟人。

反革命“围剿”的消极的结果,则是日本帝国主义打进来了。这就是为什么全国人民至今还是非常痛恨那十年反共的最大原因。

这一时期的斗争,在革命方面,是坚持了人民大众反帝反封建的新民主主义和新三民主义;在反革命方面,则是在帝国主义指挥下的地主阶级和大资产阶级联盟的专制主义。这种专制主义,在政治上,在文化上,腰斩了孙中山的三大政策,腰斩了他的新三民主义,造成了中华民族的深重的灾难。

第四个时期就是现在的抗日战争时期。在中国革命的曲线运动中,又来了一次四个阶级的统一战线,但是范围更放大了,上层阶级包括了很多统治者,中层阶级包括了民族资产阶级和小资产阶级,下层阶级包括了一切无产者,全国各阶层都成了盟员,坚决地反抗了日本帝国主义。这个时期的第一阶段,是在武汉失陷以前。这时全国各方面是欣欣向荣的,政治上有民主化的趋势,文化上有较普遍的动员。武汉失陷以后,为第二阶段,政治情况发生了许多变化,大资产阶级的一部分,投降了敌人,其另一部分也想早日结束抗战。在文化方面,反映这种情况,就出现了叶青、张君劢等人的反动和言论出版的不自由。

为了克服这种危机,必须同一切反抗战、反团结、反进步的思想进行坚决的斗争,不击破这些反动思想,抗战的胜利是无望的。这一斗争的前途如何?这是全国人民心目中的大问题。依据国内国际条件,不论抗战路程上有多少困难,中国人民总是要胜利的。全部中国史中,五四运动以后二十年的进步,不但赛过了以前的八十年,简直赛过了以前的几千年。假如再有二十年的工夫,中国的进步将到何地,不是可以想得到的吗?一切内外黑暗势力的猖獗,造成了民族的灾难;但是这种猖獗,不但表示了这些黑暗势力的还有力量,而且表示了它们的最后挣扎,表示了人民大众逐渐接近了胜利。这在中国是如此,在整个东方也是如此,在世界也是如此。

一四、文化性质问题上的偏向

一切新的东西都是从艰苦斗争中锻炼出来的。新文化也是这样,二十年中有三个曲折,走了一个“之”字,一切好的坏的东西都考验出来了。

资产阶级顽固派,在文化问题上,和他们在政权问题上一样,是完全错误的。他们不知道中国新时期的历史特点,他们不承认人民大众的新民主主义的文化。他们的出发点是资产阶级专制主义,在文化上就是资产阶级的文化专制主义。一部分所谓欧美派的文化人\footnote{毛泽东在这里所说的一部分欧美派文化人是指以胡适等为代表的一些人物。}(我说的是一部分),他们曾经实际赞助过国民党政府的文化“剿共”,现在似乎又在赞助什么“限共”、“溶共”政策。他们不愿工农在政治上抬头,也不愿工农在文化上抬头。资产阶级顽固派的这条文化专制主义的路是走不通的,它同政权问题一样,没有国内国际的条件。因此,这种文化专制主义,也还是“收起”为妙。

当作国民文化的方针来说,居于指导地位的是共产主义的思想,并且我们应当努力在工人阶级中宣传社会主义和共产主义,并适当地有步骤地用社会主义教育农民及其它群众。但整个的国民文化,现在也还不是社会主义的。

新民主主义的政治、经济、文化,由于其都是无产阶级领导的缘故,就都具有社会主义的因素,并且不是普通的因素,而是起决定作用的因素。但是就整个政治情况、整个经济情况和整个文化情况说来,却还不是社会主义的,而是新民主主义的。因为在现阶段革命的基本任务主要地是反对外国的帝国主义和本国的封建主义,是资产阶级民主主义的革命,还不是以推翻资本主义为目标的社会主义的革命。就国民文化领域来说,如果以为现在的整个国民文化就是或应该是社会主义的国民文化,这是不对的。这是把共产主义思想体系的宣传,当作了当前行动纲领的实践;把用共产主义的立场和方法去观察问题、研究学问、处理工作、训练干部,当作了中国民主革命阶段上整个的国民教育和国民文化的方针。以社会主义为内容的国民文化必须是反映社会主义的政治和经济的。我们在政治上经济上有社会主义的因素,反映到我们的国民文化也有社会主义的因素;但就整个社会来说,我们现在还没有形成这种整个的社会主义的政治和经济,所以还不能有这种整个的社会主义的国民文化。由于现时的中国革命是世界无产阶级社会主义革命的一部分,因而现时的中国新文化也是世界无产阶级社会主义新文化的一部分,是它的一个伟大的同盟军;这种一部分,虽则包含社会主义文化的重大因素,但是就整个国民文化来说,还不是完全以社会主义文化的资格去参加,而是以人民大众反帝反封建的新民主主义文化的资格去参加的。由于现时中国革命不能离开中国无产阶级的领导,因而现时的中国新文化也不能离开中国无产阶级文化思想的领导,即不能离开共产主义思想的领导。但是这种领导,在现阶段是领导人民大众去作反帝反封建的政治革命和文化革命,所以现在整个新的国民文化的内容还是新民主主义的,不是社会主义的。

在现时,毫无疑义,应该扩大共产主义思想的宣传,加紧马克思列宁主义的学习,没有这种宣传和学习,不但不能引导中国革命到将来的社会主义阶段上去,而且也不能指导现时的民主革命达到胜利。但是我们既应把对于共产主义的思想体系和社会制度的宣传,同对于新民主主义的行动纲领的实践区别开来;又应把作为观察问题、研究学问、处理工作、训练干部的共产主义的理论和方法,同作为整个国民文化的新民主主义的方针区别开来。把二者混为一谈,无疑是很不适当的。

由此可知,现阶段上中国新的国民文化的内容,既不是资产阶级的文化专制主义,又不是单纯的无产阶级的社会主义,而是以无产阶级社会主义文化思想为领导的人民大众反帝反封建的新民主主义。

一五、民族的科学的大众的文化

这种新民主主义的文化是民族的。它是反对帝国主义压迫,主张中华民族的尊严和独立的。它是我们这个民族的,带有我们民族的特性。它同一切别的民族的社会主义文化和新民主主义文化相联合,建立互相吸收和互相发展的关系,共同形成世界的新文化;但是决不能和任何别的民族的帝国主义反动文化相联合,因为我们的文化是革命的民族文化。中国应该大量吸收外国的进步文化,作为自己文化食粮的原料,这种工作过去还做得很不够。这不但是当前的社会主义文化和新民主主义文化,还有外国的古代文化,例如各资本主义国家启蒙时代的文化,凡属我们今天用得着的东西,都应该吸收。但是一切外国的东西,如同我们对于食物一样,必须经过自己的口腔咀嚼和胃肠运动,送进唾液胃液肠液,把它分解为精华和糟粕两部分,然后排泄其糟粕,吸收其精华,才能对我们的身体有益,决不能生吞活剥地毫无批判地吸收。所谓“全盘西化”\footnote{所谓“全盘西化”,是一部分资产阶级学者的主张。他们主张中国一切东西都要完全模仿欧美资本主义国家。}的主张,乃是一种错误的观点。形式主义地吸收外国的东西,在中国过去是吃过大亏的。中国共产主义者对于马克思主义在中国的应用也是这样,必须将马克思主义的普遍真理和中国革命的具体实践完全地恰当地统一起来,就是说,和民族的特点相结合,经过一定的民族形式,才有用处,决不能主观地公式地应用它。公式的马克思主义者,只是对于马克思主义和中国革命开玩笑,在中国革命队伍中是没有他们的位置的。中国文化应有自己的形式,这就是民族形式。民族的形式,新民主主义的内容——这就是我们今天的新文化。

这种新民主主义的文化是科学的。它是反对一切封建思想和迷信思想,主张实事求是,主张客观真理,主张理论和实践一致的。在这点上,中国无产阶级的科学思想能够和中国还有进步性的资产阶级的唯物论者和自然科学家,建立反帝反封建反迷信的统一战线;但是决不能和任何反动的唯心论建立统一战线。共产党员可以和某些唯心论者甚至宗教徒建立在政治行动上的反帝反封建的统一战线,但是决不能赞同他们的唯心论或宗教教义。中国的长期封建社会中,创造了灿烂的古代文化。清理古代文化的发展过程,剔除其封建性的糟粕,吸收其民主性的精华,是发展民族新文化提高民族自信心的必要条件;但是决不能无批判地兼收并蓄。必须将古代封建统治阶级的一切腐朽的东西和古代优秀的人民文化即多少带有民主性和革命性的东西区别开来。中国现时的新政治新经济是从古代的旧政治旧经济发展而来的,中国现时的新文化也是从古代的旧文化发展而来,因此,我们必须尊重自己的历史,决不能割断历史。但是这种尊重,是给历史以一定的科学的地位,是尊重历史的辩证法的发展,而不是颂古非今,不是赞扬任何封建的毒素。对于人民群众和青年学生,主要地不是要引导他们向后看,而是要引导他们向前看。

这种新民主主义的文化是大众的,因而即是民主的。它应为全民族中百分之九十以上的工农劳苦民众服务,并逐渐成为他们的文化。要把教育革命干部的知识和教育革命大众的知识在程度上互相区别又互相联结起来,把提高和普及互相区别又互相联结起来。革命文化,对于人民大众,是革命的有力武器。革命文化,在革命前,是革命的思想准备;在革命中,是革命总战线中的一条必要和重要的战线。而革命的文化工作者,就是这个文化战线上的各级指挥员。“没有革命的理论,就不会有革命的运动”\footnote{见列宁《俄国社会民主党人的任务》(《列宁全集》第2卷,人民出版社1984年版,第443页);并见列宁《怎么办?》第一章第四节(《列宁全集》第6卷,人民出版社1986年版,第23页)。},可见革命的文化运动对于革命的实践运动具有何等的重要性。而这种文化运动和实践运动,都是群众的。因此,一切进步的文化工作者,在抗日战争中,应有自己的文化军队,这个军队就是人民大众。革命的文化人而不接近民众,就是“无兵司令”,他的火力就打不倒敌人。为达此目的,文字必须在一定条件下加以改革,言语必须接近民众,须知民众就是革命文化的无限丰富的源泉。

民族的科学的大众的文化,就是人民大众反帝反封建的文化,就是新民主主义的文化,就是中华民族的新文化。

新民主主义的政治、新民主主义的经济和新民主主义的文化相结合,这就是新民主主义共和国,这就是名副其实的中华民国,这就是我们要造成的新中国。

新中国站在每个人民的面前,我们应该迎接它。

新中国航船的桅顶已经冒出地平线了,我们应该拍掌欢迎它。

举起你的双手吧,新中国是我们的。

\section{给秦邦宪的信 1944/8/31}

博古同志:

此文\footnote{指原题为《把新民主主义社会的基础建立在家庭里》的《解放日报》社论草稿。}改了一些,原拟发表,后觉改的与原文各部分颇不调和,故决定不发表,而请报馆\footnote{指解放日报社。}另写一篇不涉及许多根本问题的文章发表了。关于原文不妥之处,我的意见如下:原文着重改造家庭,关于联系群众运动方面说的很少(即在已发表的那篇社论\footnote{指一九四四年八月二十五日《解放日报》就山西襄垣李来成建立新式家庭一事发表的社论,题为《发扬根据地新式家庭》。}上亦是如此),而问题的重点,恰是使家庭改造与群众运动联系起来。这种群众运动,有当地的不脱离家庭的群众运动——变工队及合作社,自卫军及民兵,乡议会,小学、识字组及秧歌队,以及各种群众的临时集会;有脱离家庭、远离农村的群众运动——进军队(才有革命军),进工厂(才有劳动力市场),进学校(才有知识分子)以及其他出外做事等。民主革命的中心目的就是从侵略者、地主、买办手下解放农民,建立近代工业社会。“巩固家庭”的口号,只有和上述种种革命运动联系起来,才是革命的口号。农民的家庭是必然要破坏的,进军队、进工厂就是一个大破坏,就是纷纷“走出家庭”。实际上,我们是提倡“走出家庭”与“巩固家庭”的两重政策。扩军、归队、招工人、招学生(这后二项将来必多)、移民、出外做革命工作、找其他职业等等,都是提倡走出家庭,这个数目,在现在敌后战场是很大的,在战后也将是很大的。剩下的男女老幼,才是提倡巩固其家庭。在内战时的兴国县,有些家庭,剩下来待我们巩固的,竟至占人口的少数。只要有一个大的时局变动,例如打下北平之类,我们居住的这个现在很少变动的边区农村家庭人口,也将有许多人走出家庭。实际上,不断地走出,不断地巩固,这就是我们的需要。所以,根本否定“五四”口号,根本反对走出家庭,是不应该也不可能的。

没有社会活动(战争、工厂、减租、变工队等),家庭是不可能改造的。襄垣李来成家的改造,正是在社会群众运动的大浪潮中才获得。农村家庭从封建到民主的改造,不能由孤立的家庭成员从什么书上或报上看了好意见而获得,只能经过群众运动。

此外,新民主主义社会的基础是工厂(社会生产,公营的与私营的)与合作社(变工队在内),不是分散的个体经济。分散的个体经济——家庭农业与家庭手工业是封建社会的基础,不是民主社会(旧日民主、新民主、社会主义,一概在内)的基础,这是马克思主义区别于民粹主义的地方。简单言之,新民主主义社会的基础是机器,不是手工。我们现在还没有获得机器,所以我们还没有胜利。如果我们永远不能获得机器,我们就永远不能胜利,我们就要灭亡。现在的农村是暂时的根据地,不是也不能是整个中国民主社会的主要基础。由农业基础到工业基础,正是我们革命的任务。

以上,请给艾、陆、余\footnote{艾,指艾思奇,当时任解放日报社副刊部部长。陆,指陆定一(一九○六——一九九六),江苏无锡人,当时任《解放日报》总编辑。余,指余光生(一九○六——一九七八),浙江镇海人,当时任《解放日报》副总编辑。}三同志一阅。如有意见,请告知。

敬礼!

毛泽东

八月三十一日

又,我在改文中加上了解放个性,这也是民主对封建革命必然包括的。有人说我们忽视或压制个性,这是不对的。被束缚的个性如不得解放,就没有民主主义,也没有社会主义。

\section{关于国家资本主义的批语 1953/7/9}

按:一九五三年七月九日,在夏季全国财经工作会议的一个文件上的批语。

中国现在的资本主义经济其绝大部分是在人民政府管理之下的,用各种形式和国营社会主义经济联系着的,并受工人监督的资本主义经济。这种资本主义经济已经不是普通的资本主义经济,而是一种特殊的资本主义经济,即新式的国家资本主义经济。它主要地不是为了资本家的利润而存在,而是为了供应人民和国家的需要而存在。不错,工人们还要为资本家生产一部分利润,但这只占全部利润中的一小部分,大约只占四分之一左右,其余的四分之三是为工人(福利费)为国家(所得税)及为扩大生产设备(其中包含一小部分是为资本家生产利润的)而生产的。因此,这种新式国家资本主义经济是带着很大的社会主义性质的,是对工人和国家有利的。

\section{论十大关系 1956/4/25}

最近几个月,中央政治局听了中央工业、农业、运输业、商业、财政等三十四个部门的工作汇报,从中看到一些有关社会主义建设和社会主义改造的问题。综合起来,一共有十个问题,也就是十大关系。

提出这十个问题,都是围绕着一个基本方针,就是要把国内外一切积极因素调动起来,为社会主义事业服务。过去为了结束帝国主义、封建主义和官僚资本主义的统治,为了人民民主革命的胜利,我们就实行了调动一切积极因素的方针。现在为了进行社会主义革命,建设社会主义国家,同样也实行这个方针。但是,我们工作中间还有些问题需要谈一谈。特别值得注意的是,最近苏联方面暴露了他们在建设社会主义过程中的一些缺点和错误,他们走过的弯路,你还想走?过去我们就是鉴于他们的经验教训,少走了一些弯路,现在当然更要引以为戒。

什么是国内外的积极因素?在国内,工人和农民是基本力量。中间势力是可以争取的力量。反动势力虽是一种消极因素,但是我们仍然要作好工作,尽量争取化消极因素为积极因素。在国际上,一切可以团结的力量都要团结,不中立的可以争取为中立,反动的也可以分化和利用。总之,我们要调动一切直接的和间接的力量,为把我国建设成为一个强大的社会主义国家而奋斗。

下面我讲十个问题。

一、重工业和轻工业、农业的关系

重工业是我国建设的重点。必须优先发展生产资料的生产,这是已经定了的。但是决不可以因此忽视生活资料尤其是粮食的生产。如果没有足够的粮食和其它生活必需品,首先就不能养活工人,还谈什么发展重工业?所以,重工业和轻工业、农业的关系,必须处理好。

在处理重工业和轻工业、农业的关系上,我们没有犯原则性的错误。我们比苏联和一些东欧国家作得好些。像苏联的粮食产量长期达不到革命前最高水平的问题,像一些东欧国家由于轻重工业发展太不平衡而产生的严重问题,我们这里是不存在的。他们片面地注重重工业,忽视农业和轻工业,因而市场上的货物不够,货币不稳定。我们对于农业、轻工业是比较注重的。我们一直抓了农业,发展了农业,相当地保证了发展工业所需要的粮食和原料。我们的民生日用商品比较丰富,物价和货币是稳定的。

我们现在的问题,就是还要适当地调整重工业和农业、轻工业的投资比例,更多地发展农业、轻工业。这样,重工业是不是不为主了?它还是为主,还是投资的重点。但是,农业、轻工业投资的比例要加重一点。

加重的结果怎么样?加重的结果,一可以更好地供给人民生活的需要,二可以更快地增加资金的积累,因而可以更多更好地发展重工业。重工业也可以积累,但是,在我们现有的经济条件下,轻工业、农业积累得更多更快些。

这里就发生一个问题,你对发展重工业究竟是真想还是假想,想得厉害一点,还是差一点?你如果是假想,或者想得差一点,那就打击农业、轻工业,对它们少投点资。你如果是真想,或者想得厉害,那你就要注重农业、轻工业,使粮食和轻工业原料更多些,积累更多些,投到重工业方面的资金将来也会更多些。

我们现在发展重工业可以有两种办法,一种是少发展一些农业、轻工业,一种是多发展一些农业、轻工业。从长远观点来看,前一种办法会使重工业发展得少些和慢些,至少基础不那么稳固,几十年后算总账是划不来的。后一种办法会使重工业发展得多些和快些,而且由于保障了人民生活的需要,会使它发展的基础更加稳固。

二、沿海工业和内地工业的关系

我国的工业过去集中在沿海。所谓沿海,是指辽宁、河北、北京、天津、河南东部、山东、安徽、江苏、上海、浙江、福建、广东、广西。我国全部轻工业和重工业,都有约百分之七十在沿海,只有百分之三十在内地。这是历史上形成的一种不合理的状况。沿海的工业基地必须充分利用,但是,为了平衡工业发展的布局,内地工业必须大力发展。在这两者的关系问题上,我们也没有犯大的错误,只是最近几年,对于沿海工业有些估计不足,对它的发展不那么十分注重了。这要改变一下。

过去朝鲜还在打仗,国际形势还很紧张,不能不影响我们对沿海工业的看法。现在,新的侵华战争和新的世界大战,估计短时期内打不起来,可能有十年或者更长一点的和平时期。这样,如果还不充分利用沿海工业的设备能力和技术力量,那就不对了。不说十年,就算五年,我们也应当在沿海好好地办四年的工业,等第五年打起来再搬家。从现有材料看来,轻工业工厂的建设和积累一般都很快,全部投产以后,四年之内,除了收回本厂的投资以外,还可以赚回三个厂,两个厂,一个厂,至少半个厂。这样好的事情为什么不做?认为原子弹已经在我们头上,几秒钟就要掉下来,这种形势估计是不合乎事实的,由此而对沿海工业采取消极态度是不对的。

这不是说新的工厂都建在沿海。新的工业大部分应当摆在内地,使工业布局逐步平衡,并且利于备战,这是毫无疑义的。但是沿海也可以建立一些新的厂矿,有些也可以是大型的。至于沿海原有的轻重工业的扩建和改建,过去已经作了一些,以后还要大大发展。

好好地利用和发展沿海的工业老底子,可以使我们更有力量来发展和支持内地工业。如果采取消极态度,就会妨碍内地工业的迅速发展。所以这也是一个对于发展内地工业是真想还是假想的问题。如果是真想,不是假想,就必须更多地利用和发展沿海工业,特别是轻工业。

三、经济建设和国防建设的关系

国防不可不有。现在,我们有了一定的国防力量。经过抗美援朝和几年的整训,我们的军队加强了,比第二次世界大战前的苏联红军要更强些,装备也有所改进。我们的国防工业正在建立。自从盘古开天辟地以来,我们不晓得造飞机,造汽车,现在开始能造了。

我们现在还没有原子弹。但是,过去我们也没有飞机和大炮,我们是用小米加步枪打败了日本帝国主义和蒋介石的。我们现在已经比过去强,以后还要比现在强,不但要有更多的飞机和大炮,而且还要有原子弹。在今天的世界上,我们要不受人家欺负,就不能没有这个东西。怎么办呢?可靠的办法就是把军政费用降到一个适当的比例,增加经济建设费用。只有经济建设发展得更快了,国防建设才能够有更大的进步。

一九五○年,我们在党的七届三中全会上,已经提出精简国家机构、减少军政费用的问题,认为这是争取我国财政经济情况根本好转的三个条件之一。第一个五年计划期间,军政费用占国家预算全部支出的百分之三十。这个比重太大了。第二个五年计划期间,要使它降到百分之二十左右,以便抽出更多的资金,多开些工厂,多造些机器。经过一段时间,我们就不但会有很多的飞机和大炮,而且还可能有自己的原子弹。

这里也发生这么一个问题,你对原子弹是真正想要、十分想要,还是只有几分想,没有十分想呢?你是真正想要、十分想要,你就降低军政费用的比重,多搞经济建设。你不是真正想要、十分想要,你就还是按老章程办事。这是战略方针的问题,希望军委讨论一下。

现在我们把兵统统裁掉好不好?那不好。因为还有敌人,我们还受敌人欺负和包围嘛!我们一定要加强国防,因此,一定要首先加强经济建设。

四、国家、生产单位和生产者个人的关系

国家和工厂、合作社的关系,工厂、合作社和生产者个人的关系,这两种关系都要处理好。为此,就不能只顾一头,必须兼顾国家、集体和个人三个方面,也就是我们过去常说的“军民兼顾”、“公私兼顾”。鉴于苏联和我们自己的经验,今后务必更好地解决这个问题。

拿工人讲,工人的劳动生产率提高了,他们的劳动条件和集体福利就需要逐步有所改进。我们历来提倡艰苦奋斗,反对把个人物质利益看得高于一切,同时我们也历来提倡关心群众生活,反对不关心群众痛痒的官僚主义。随着整个国民经济的发展,工资也需要适当调整。关于工资,最近决定增加一些,主要加在下面,加在工人方面,以便缩小上下两方面的距离。我们的工资一般还不高,但是因为就业的人多了,因为物价低和稳,加上其它种种条件,工人的生活比过去还是有了很大改善。在无产阶级政权下面,工人的政治觉悟和劳动积极性一直很高。去年年底中央号召反右倾保守,工人群众热烈拥护,奋战三个月,破例地超额完成了今年第一季度的计划。我们需要大力发扬他们这种艰苦奋斗的精神,也需要更多地注意解决他们在劳动和生活中的迫切问题。

这里还要谈一下工厂在统一领导下的独立性问题。把什么东西统统都集中在中央或省市,不给工厂一点权力,一点机动的余地,一点利益,恐怕不妥。中央、省市和工厂的权益究竟应当各有多大才适当,我们经验不多,还要研究。从原则上说,统一性和独立性是对立的统一,要有统一性,也要有独立性。比如我们现在开会是统一性,散会以后有人散步,有人读书,有人吃饭,就是独立性。如果我们不给每个人散会后的独立性,一直把会无休止地开下去,不是所有的人都要死光吗?个人是这样,工厂和其它生产单位也是这样。各个生产单位都要有一个与统一性相联系的独立性,才会发展得更加活泼。

再讲农民。我们同农民的关系历来都是好的,但是在粮食问题上曾经犯过一个错误。一九五四年我国部分地区因水灾减产,我们却多购了七十亿斤粮食。这样一减一多,闹得去年春季许多地方几乎人人谈粮食,户户谈统销。农民有意见,党内外也有许多意见。尽管不少人是故意夸大,乘机攻击,但是不能说我们没有缺点。调查不够,摸不清底,多购了七十亿斤,这就是缺点。我们发现了缺点,一九五五年就少购了七十亿斤,又搞了一个“三定”,就是定产定购定销,加上丰收,一少一增,使农民手里多了二百多亿斤粮食。这样,过去有意见的农民也说“共产党真是好”了。这个教训,全党必须记住。

苏联的办法把农民挖得很苦。他们采取所谓义务交售制等项办法,把农民生产的东西拿走太多,给的代价又极低。他们这样来积累资金,使农民的生产积极性受到极大的损害。你要母鸡多生蛋,又不给它米吃,又要马儿跑得好,又要马儿不吃草。世界上哪有这样的道理!~义务交售制,是苏联一九三三年至一九五七年实行的国家收购农产品的一项主要办法。集体农庄和个体农户每年必须按照国家规定的义务交售的数量和价格向国家提供农产品。~

我们对农民的政策不是苏联的那种政策,而是兼顾国家和农民的利益。我们的农业税历来比较轻。工农业品的交换,我们是采取缩小剪刀差,等价交换或者近乎等价交换的政策。我们统购农产品是按照正常的价格,农民并不吃亏,而且收购的价格还逐步有所增长。我们在向农民供应工业品方面,采取薄利多销、稳定物价或适当降价的政策,在向缺粮区农民供应粮食方面,一般略有补贴。但是就是这样,如果粗心大意,也还是会犯这种或那种错误。鉴于苏联在这个问题上犯了严重错误,我们必须更多地注意处理好国家同农民的关系。

合作社同农民的关系也要处理好。在合作社的收入中,国家拿多少,合作社拿多少,农民拿多少,以及怎样拿法,都要规定得适当。合作社所拿的部分,都是直接为农民服务的。生产费不必说,管理费也是必要的,公积金是为了扩大再生产,公益金是为了农民的福利。但是,这几项各占多少,应当同农民研究出一个合理的比例。生产费管理费都要力求节约。公积金公益金也要有个控制,不能希望一年把好事都做完。

除了遇到特大自然灾害以外,我们必须在增加农业生产的基础上,争取百分之九十的社员每年的收入比前一年有所增加,百分之十的社员的收入能够不增不减,如有减少,也要及早想办法加以解决。

总之,国家和工厂,国家和工人,工厂和工人,国家和合作社,国家和农民,合作社和农民,都必须兼顾,不能只顾一头。无论只顾哪一头,都是不利于社会主义,不利于无产阶级专政的。这是一个关系到六亿人民的大问题,必须在全党和全国人民中间反复进行教育。

五、中央和地方的关系

中央和地方的关系也是一个矛盾。解决这个矛盾,目前要注意的是,应当在巩固中央统一领导的前提下,扩大一点地方的权力,给地方更多的独立性,让地方办更多的事情。这对我们建设强大的社会主义国家比较有利。我们的国家这样大,人口这样多,情况这样复杂,有中央和地方两个积极性,比只有一个积极性好得多。我们不能像苏联那样,把什么都集中到中央,把地方卡得死死的,一点机动权也没有。

中央要发展工业,地方也要发展工业。就是中央直属的工业,也还是要靠地方协助。至于农业和商业,更需要依靠地方。总之,要发展社会主义建设,就必须发挥地方的积极性。中央要巩固,就要注意地方的利益。

现在几十只手插到地方,使地方的事情不好办。立了一个部就要革命,要革命就要下命令。各部不好向省委、省人民委员会下命令,就同省、市的厅局联成一线,天天给厅局下命令。这些命令虽然党中央不知道,国务院不知道,但都说是中央来的,给地方压力很大。表报之多,闹得泛滥成灾。这种情况,必须纠正。

我们要提倡同地方商量办事的作风。党中央办事,总是同地方商量,不同地方商量从来不冒下命令。在这方面,希望中央各部好好注意,凡是同地方有关的事情,都要先同地方商量,商量好了再下命令。

中央的部门可以分成两类。有一类,它们的领导可以一直管到企业,它们设在地方的管理机构和企业由地方进行监督;有一类,它们的任务是提出指导方针,制定工作规划,事情要靠地方办,要由地方去处理。

处理好中央和地方的关系,这对于我们这样的大国大党是一个十分重要的问题。这个问题,有些资本主义国家也是很注意的。它们的制度和我们的制度根本不同,但是它们发展的经验,还是值得我们研究。拿我们自己的经验说,我们建国初期实行的那种大区制度,当时有必要,但是也有缺点,后来的高饶反党联盟,就多少利用了这个缺点。以后决定取消大区,各省直属中央,这是正确的。但是由此走到取消地方的必要的独立性,结果也不那么好。我们的宪法规定,立法权集中在中央。但是在不违背中央方针的条件下,按照情况和工作需要,地方可以搞章程、条例、办法,宪法并没有约束。我们要统一,也要特殊。为了建设一个强大的社会主义国家,必须有中央的强有力的统一领导,必须有全国的统一计划和统一纪律,破坏这种必要的统一,是不允许的。同时,又必须充分发挥地方的积极性,各地都要有适合当地情况的特殊。这种特殊不是高岗的那种特殊,而是为了整体利益,为了加强全国统一所必要的特殊。

还有一个地方和地方的关系问题,这里说的主要是地方的上下级关系问题。省市对中央部门有意见,地、县、区、乡对省市就没有意见吗?中央要注意发挥省市的积极性,省市也要注意发挥地、县、区、乡的积极性,都不能够框得太死。当然,也要告诉下面的同志哪些事必须统一,不能乱来。总之,可以和应当统一的,必须统一,不可以和不应当统一的,不能强求统一。正当的独立性,正当的权利,省、市、地、县、区、乡都应当有,都应当争。这种从全国整体利益出发的争权,不是从本位利益出发的争权,不能叫做地方主义,不能叫做闹独立性。

省市和省市之间的关系,也是一种地方和地方的关系,也要处理得好。我们历来的原则,就是提倡顾全大局,互助互让。

在解决中央和地方、地方和地方的关系问题上,我们的经验还不多,还不成熟,希望你们好好研究讨论,并且每过一个时期就要总结经验,发扬成绩,克服缺点。

六、汉族和少数民族的关系

对于汉族和少数民族的关系,我们的政策是比较稳当的,是比较得到少数民族赞成的。我们着重反对大汉族主义。地方民族主义也要反对,但是那一般地不是重点。

我国少数民族人数少,占的地方大。论人口,汉族占百分之九十四,是压倒优势。如果汉人搞大汉族主义,歧视少数民族,那就很不好。而土地谁多呢?土地是少数民族多,占百分之五十到六十。我们说中国地大物博,人口众多,实际上是汉族“人口众多”,少数民族“地大物博”,至少地下资源很可能是少数民族“物博”。

各个少数民族对中国的历史都作过贡献。汉族人口多,也是长时期内许多民族混血形成的。历史上的反动统治者,主要是汉族的反动统治者,曾经在我们各民族中间制造种种隔阂,欺负少数民族。这种情况所造成的影响,就在劳动人民中间也不容易很快消除。所以我们无论对干部和人民群众,都要广泛地持久地进行无产阶级的民族政策教育,并且要对汉族和少数民族的关系经常注意检查。早两年已经作过一次检查,现在应当再来一次。如果关系不正常,就必须认真处理,不要只口里讲。

在少数民族地区,经济管理体制和财政体制,究竟怎样才适合,要好好研究一下。

我们要诚心诚意地积极帮助少数民族发展经济建设和文化建设。在苏联,俄罗斯民族同少数民族的关系很不正常,我们应当接受这个教训。天上的空气,地上的森林地下的宝藏,都是建设社会主义所需要的重要因素,而一切物质因素只有通过人的因素,才能加以开发利用。我们必须搞好汉族和少数民族的关系,巩固各民族的团结,来共同努力于建设伟大的社会主义祖国。

七、党和非党的关系

究竟是一个党好,还是几个党好?现在看来,恐怕是几个党好。不但过去如此,而且将来也可以如此,就是长期共存,互相监督。

在我们国内,在抗日反蒋斗争中形成的以民族资产阶级及其知识分子为主的许多民主党派,现在还继续存在。在这一点上,我们和苏联不同。我们有意识地留下民主党派,让他们有发表意见的机会,对他们采取又团结又斗争的方针。一切善意地向我们提意见的民主人士,我们都要团结。像卫立煌、翁文灏这样的有爱国心的国民党军政人员,我们应当继续调动他们的积极性。就是那些骂我们的,像龙云、梁漱溟、彭一湖之类,我们也要养起来,让他们骂,骂得无理,我们反驳,骂得有理,我们接受。这对党,对人民,对社会主义比较有利。

中国现在既然还有阶级和阶级斗争,就不会没有各种形式的反对派。所有民主党派和无党派民主人士虽然都表示接受中国共产党的领导,但是他们中的许多人,实际上就是程度不同的反对派。在“把革命进行到底”、抗美援朝、土地改革等等问题上,他们都是又反对又不反对。对于镇压反革命,他们一直到现在还有意见。他们说《共同纲领》好得不得了,不想搞社会主义类型的宪法,但是宪法起草出来了,他们又全都举手赞成。事物常常走到自己的反面,民主党派对许多问题的态度也是这样。他们是反对派,又不是反对派常常由反对走到不反对。

共产党和民主党派都是历史上发生的。凡是历史上发生的东西,都要在历史上消灭。因此,共产党总有一天要消灭,民主党派也总有一天要消灭。消灭就是那么不舒服?我看很舒服。共产党,无产阶级专政,哪一天不要了,我看实在好。我们的任务就是要促使它们消灭得早一点。这个道理,过去我们已经说过多次了。

但是,无产阶级政党和无产阶级专政,现在非有不可,而且非继续加强不可。否则,不能镇压反革命,不能抵抗帝国主义,不能建设社会主义,建设起来也不能巩固。列宁关于无产阶级政党和无产阶级专政的理论,决没有像有些人说的那样“已经过时”。无产阶级专政不能没有很大的强制性。但是,必须反对官僚主义,反对机构庞大。在一不死人二不废事的条件下,我建议党政机构进行大精简,砍掉它三分之二。

话说回来,党政机构要精简,不是说不要民主党派。希望你们抓一下统一战线工作,使他们和我们的关系得到改善,尽可能把他们的积极性调动起来为社会主义服务。

八、革命和反革命的关系

反革命是什么因素?是消极因素,破坏因素,是积极因素的反对力量。反革命可不可以转变?当然,有些死心塌地的反革命不会转变。但是,在我国的条件下,他们中间的大多数将来会有不同程度的转变。由于我们采取了正确的政策,现在就有不少反革命被改造成不反革命了,有些人还做了一些有益的事。

有几点应当肯定:

第一点,应当肯定,一九五一年和一九五二年那一次镇压反革命是必须的。有这么一种意见,认为那一次镇压反革命也可以不搞。这种意见是错误的。

对待反革命分子的办法是:杀、关、管、放。杀,大家都知道是什么一回事。关,就是关起来劳动改造。管,就是放在社会上由群众监督改造。放,就是可捉可不捉的一般不捉,或者捉起来以后表现好的,把他放掉。按照不同情况,给反革命分子不同的处理,是必要的。

现在只说杀。那一次镇压反革命杀了一批人,那是些什么人呢?是老百姓非常仇恨的、血债累累的反革命分子。六亿人民的大革命,不杀掉那些“东霸天”、“西霸天”,人民是不能起来的。如果没有那次镇压,今天我们采取宽大政策,老百姓就不可能赞成。现在有人听到说斯大林杀错了一些人,就说我们杀的那批反革命也杀错了,这是不对的。肯定过去根本上杀得对,在目前有实际意义。

第二点,应当肯定,还有反革命,但是已经大为减少。在胡风问题出来以后,清查反革命是必要的。有些没有清查出来的,还要继续清查。要肯定现在还有少数反革命分子,他们还在进行各种反革命破坏活动,比如把牛弄死,把粮食烧掉,破坏工厂,盗窃情报,贴反动标语,等等。所以,说反革命已经肃清了,可以高枕无忧了,是不对的。只要中国和世界上还有阶级斗争,就永远不可以放松警惕。但是,说现在还有很多反革命,也是不对的。

第三点,今后社会上的镇反,要少捉少杀。社会上的反革命因为是老百姓的直接冤头,老百姓恨透了,所以少数人还是要杀。他们中的多数,要交给农业合作社去管制生产,劳动改造。但是,我们还不能宣布一个不杀,不能废除死刑。

第四点,机关、学校、部队里面清查反革命,要坚持在延安开始的一条,就是一个不杀,大部不捉。真凭实据的反革命,由机关清查,但是公安局不捉,检察机关不起诉,法院也不审判。一百个反革命里面,九十几个这样处理。这就是所谓大部不捉。至于杀呢,就是一个不杀。

什么样的人不杀呢?胡风、潘汉年、饶漱石这样的人不杀,连被俘的战犯宣统皇帝、康泽这样的人也不杀。不杀他们,不是没有可杀之罪,而是杀了不利。这样的人杀了一个,第二个第三个就要来比,许多人头就要落地。这是第一条。第二条,可以杀错人。一颗脑袋落地,历史证明是接不起来的,也不像韭菜那样,割了一次还可以长起来,割错了,想改正错误也没有办法。第三条,消灭证据。镇压反革命要有证据。这个反革命常常就是那个反革命的活证据,有官司可以请教他。你把他消灭了,可能就再找不到证据了。这就只有利于反革命,而不利于革命。第四条,杀了他们,一不能增加生产,二不能提高科学水平,三不能帮助除四害,四不能强大国防,五不能收复台湾。杀了他们,你得一个杀俘虏的名声,杀俘虏历来是名声不好的。还有一条,机关里的反革命跟社会上的反革命不同。社会上的反革命爬在人民的头上,而机关里的反革命跟人民隔得远些,他们有普遍的冤头,但是直接的冤头不多。这些人一个不杀有什么害处呢?能劳动改造的去劳动改造,不能劳动改造的就养一批。反革命是废物,是害虫,可是抓到手以后,却可以让他们给人民办点事情。

但是,要不要立条法律,讲机关里的反革命一个不杀呢?这是我们的内部政策,不用宣布,实际上尽量做到就是了。假使有人丢个炸弹,把这个屋子里的人都炸死了,或者一半,或者三分之一,你说杀不杀?那就一定要杀。

机关肃反实行一个不杀的方针,不妨碍我们对反革命分子采取严肃态度。但是,可以保证不犯无法挽回的错误,犯了错误也有改正的机会,可以稳定很多人,可以避免党内同志之间互不信任。不杀头,就要给饭吃。对一切反革命分子,都应当给以生活出路,使他们有自新的机会。这样做,对人民事业,对国际影响,都有好处。

镇压反革命还要作艰苦的工作,大家不能松懈。今后,除社会上的反革命还要继续镇压以外,必须把混在机关、学校、部队中的一切反革命分子继续清查出来。一定要分清敌我。如果让敌人混进我们的队伍,甚至混进我们的领导机关,那会对社会主义事业和无产阶级专政造成多么严重的危险,这是大家都清楚的。

九、是非关系

党内党外都要分清是非。如何对待犯了错误的人,这是一个重要的问题。正确的态度应当是,对于犯错误的同志,采取“惩前毖后,治病救人”的方针,帮助他们改正错误,允许他们继续革命。过去,在以王明为首的教条主义者当权的时候,我们党在这个问题上犯了错误,学了斯大林作风中不好的一面。他们在社会上不要中间势力,在党内不允许人家改正错误,不准革命。

《阿Q正传》是一篇好小说,我劝看过的同志再看一遍,没看过的同志好好地看看。 鲁迅在这篇小说里面, 主要是写一个落后的不觉悟的农民。他专门写了“不准革命”一章,说假洋鬼子不准阿Q革命。其实,阿Q当时的所谓革命,不过是想跟别人一样拿点东西而已。可是,这样的革命假洋鬼子也还是不准。我看在这点上,有些人很有点像假洋鬼子。他们不准犯错误的人革命,不分犯错误和反革命的界限,甚至把一些犯错误的人杀掉了。我们要记住这个教训。无论在社会上不准人家革命,还是在党内不准犯错误的同志改正错误,都是不好的。

对于犯了错误的同志,有人说要看他们改不改。我说单是看还不行,还要帮助他们改。这就是说,一要看,二要帮。人是要帮助的,没有犯错误的人要帮助,犯了错误的人更要帮助。人大概是没有不犯错误的,多多少少要犯错误,犯了错误就要帮助。只看,是消极的,要设立各种条件帮助他改。是非一定要搞清楚,因为党内的原则争论,是社会上阶级斗争在党内的反映,是不允许含糊的。按照情况,对于犯错误的同志采取恰如其分的合乎实际的批评,甚至必要的斗争,这是正常的,是为了帮助他们改正错误。对犯错误的同志不给帮助,反而幸灾乐祸,这就是宗派主义。

对于革命来说,总是多一点人好。犯错误的人,除了极少数坚持错误、屡教不改的以外,大多数是可以改正的。正如得过伤寒病的可以免疫一样,犯过错误的人,只要善于从错误中取得教训,也可以少犯错误。倒是没有犯过错误的人容易犯错误,因为他容易把尾巴翘得高。我们要注意,对犯错误的人整得过分,常常整到自己身上。高岗本来是想搬石头打人的,结果却打倒了自己。好意对待犯错误的人,可以得人心,可以团结人。对待犯错误的同志,究竟是采取帮助态度还是采取敌视态度,这是区别一个人是好心还是坏心的一个标准。

“惩前毖后,治病救人”的方针,是团结全党的方针,我们必须坚持这个方针。

十、中国和外国的关系

我们提出向外国学习的口号,我想是提得对的。现在有些国家的领导人就不愿意提,甚至不敢提这个口号。这是要有一点勇气的,就是要把戏台上的那个架子放下来。

应当承认,每个民族都有它的长处,不然它为什么能存在?为什么能发展?同时,每个民族也都有它的短处。有人以为社会主义就了不起,一点缺点也没有了。哪有这个事?应当承认,总是有优点和缺点这两点。我们党的支部书记,部队的连排长,都晓得在小本本上写着,今天总结经验有两点,一是优点,一是缺点。他们都晓得有两点,为什么我们只提一点?一万年都有两点。将来有将来的两点,现在有现在的两点,各人有各人的两点。总之,是两点而不是一点。说只有一点,叫知其一不知其二。

我们的方针是,一切民族、一切国家的长处都要学,政治、经济、科学、技术、文学、艺术的一切真正好的东西都要学。但是,必须有分析有批判地学,不能盲目地学,不能一切照抄,机械搬用。他们的短处、缺点,当然不要学。

对于苏联和其它社会主义国家的经验,也应当采取这样的态度。过去我们一些人不清楚,人家的短处也去学。当着学到以为了不起的时候,人家那里已经不要了,结果栽了个斤斗,像孙悟空一样,翻过来了。比如,过去有人因为苏联是设电影部、文化局,我们是设文化部、电影局,就说我们犯了原则错误。他们没有料到,苏联不久也改设文化部,和我们一样。有些人对任何事物都不加分析,完全以“风”为准。今天刮北风,他是北风派,明天刮西风,他是西风派,后来又刮北风,他又是北风派。自己毫无主见,往往由一个极端走到另一个极端。

苏联过去把斯大林捧得一万丈高的人,现在一下子把他贬到地下九千丈。我们国内也有人跟着转。中央认为斯大林是三分错误,七分成绩,总起来还是一个伟大的马克思主义者,按照这个分寸,写了《关于无产阶级专政的历史经验》。三七开的评价比较合适。斯大林对中国作了一些错事。第二次国内革命战争后期的王明“左”倾冒险主义,抗日战争初期的王明右倾机会主义,都是从斯大林那里来的。解放战争时期,先是不准革命,说是如果打内战,中华民族有毁灭的危险。仗打起来,对我们半信半疑。仗打胜了,又怀疑我们是铁托式的胜利,一九四九、一九五○两年对我们的压力很大。可是,我们还认为他是三分错误,七分成绩。这是公正的。

社会科学,马克思列宁主义,斯大林讲得对的那些方面,我们一定要继续努力学习。我们要学的是属于普遍真理的东西,并且学习一定要与中国实际相结合。如果每句话,包括马克思的话,都要照搬,那就不得了。我们的理论,是马克思列宁主义的普遍真理同中国革命的具体实践相结合。党内一些人有一个时期搞过教条主义,那时我们批评了这个东西。但是现在也还是有。学术界也好,经济界也好,都还有教条主义。

自然科学方面,我们比较落后,特别要努力向外国学习。但是也要有批判地学,不可盲目地学。在技术方面,我看大部分先要照办,因为那些我们现在还没有,还不懂,学了比较有利。但是,已经清楚的那一部分,就不要事事照办了。

外国资产阶级的一切腐败制度和思想作风,我们要坚决抵制和批判。但是,这并不妨碍我们去学习资本主义国家的先进的科学技术和企业管理方法中合乎科学的方面。工业发达国家的企业,用人少,效率高,会做生意,这些都应当有原则地好好学过来,以利于改进我们的工作。现在,学英文的也不研究英文了,学术论文也不译成英文、法文、德文、日文同人家交换了。这也是一种迷信。对外国的科学、技术和文化,不加分析地一概排斥,和前面所说的对外国东西不加分析地一概照搬,都不是马克思主义的态度,都对我们的事业不利。

我认为,中国有两条缺点,同时又是两条优点。

第一,我国过去是殖民地、半殖民地,不是帝国主义,历来受人欺负。工农业不发达,科学技术水平低,除了地大物博,人口众多,历史悠久,以及在文学上有部《红楼梦》等等以外,很多地方不如人家,骄傲不起来。但是,有些人做奴隶做久了, 感觉事事不如人,在外国人面前伸不直腰,像《法门寺》里的贾桂\footnote{贾桂是京剧《法门寺》里明朝宦官刘瑾的亲信奴才。}一样,人家让他坐,他说站惯了,不想坐。在这方面要鼓点劲,要把民族自信心提高起来,把抗美援朝中提倡的“藐视美帝国主义”的精神发展起来。

第二,我们的革命是后进的。虽然辛亥革命打倒皇帝比俄国早,但是那时没有共产党,那次革命也失败了。人民革命的胜利是在一九四九年,比苏联的十月革命晚了三十几年。在这点上,也轮不到我们来骄傲。苏联和我们不同,一、沙皇俄国是帝国主义,二、后来又有了一个十月革命。所以许多苏联人很骄傲,尾巴翘得很高。

我们这两条缺点,也是优点。我曾经说过,我们一为“穷”,二为“白”。“穷”,就是没有多少工业,农业也不发达。“白”,就是一张白纸,文化水平、科学水平都不高。从发展的观点看,这并不坏。穷就要革命,富的革命就困难。科学技术水平高的国家,就骄傲得很。我们是一张白纸,正好写字。

因此,这两条对我们都有好处。将来我们国家富强了,我们一定还要坚持革命立场,还要谦虚谨慎,还要向人家学习,不要把尾巴翘起来。不但在第一个五年计划期间要向人家学习,就是在几十个五年计划之后,还应当向人家学习。一万年都要学习嘛!这有什么不好呢?

一共讲了十点。这十种关系,都是矛盾。世界是由矛盾组成的。没有矛盾就没有世界。我们的任务,是要正确处理这些矛盾。这些矛盾在实践中是否能完全处理好,也要准备两种可能性,而且在处理这些矛盾的过程中,一定还会遇到新的矛盾,新的问题。但是,像我们常说的那样,道路总是曲折的,前途总是光明的。我们一定要努力把党内党外、国内国外的一切积极的因素,直接的、间接的积极因素,全部调动起来,把我国建设成为一个强大的社会主义国家。

\section{关于正确处理人民内部矛盾的问题 1957/2/27}

按:这是毛泽东同志在最高国务会议第十一次(扩大)会议上的讲话。后来毛泽东根据原始记录加以整理,作了若干补充,一九五七年六月十九日在《人民日报》发表。

关于正确处理人民内部矛盾的问题,这是一个总题目。为了叙述的方便,分为十二个小题目。在这里,也要说到敌我矛盾的问题,但是重点是讨论人民内部的矛盾问题。

一、两类不同性质的矛盾

我们的国家现在是空前统一的。资产阶级民主革命和社会主义革命的胜利,以及社会主义建设的成就,迅速地改变了旧中国的面貌。祖国的更加美好的将来,正摆在我们的面前。人民所厌恶的国家分裂和混乱的局面,已经一去不复返了。我国的六亿人民正在工人阶级和共产党的领导下,团结一致地进行着伟大的社会主义建设。国家的统一,人民的团结,国内各民族的团结,这是我们的事业必定要胜利的基本保证。但是,这并不是说在我们的社会里已经没有任何的矛盾了。没有矛盾的想法是不符合客观实际的天真的想法。在我们的面前有两类社会矛盾,这就是敌我之间的矛盾和人民内部的矛盾。这是性质完全不同的两类矛盾。

为了正确地认识敌我之间和人民内部这两类不同的矛盾应该首先弄清楚什么是人民,什么是敌人。人民这个概念在不同的国家和各个国家的不同的历史时期,有着不同的内容。拿我国的情况来说,在抗日战争时期,一切抗日的阶级、阶层和社会集团都属于人民的范围,日本帝国主义、汉奸、亲日派都是人民的敌人。在解放战争时期,美帝国主义和它的走狗即官僚资产阶级、地主阶级以及代表这些阶级的国民党反动派,都是人民的敌人;一切反对这些敌人的阶级、阶层和社会集团,都属于人民的范围。在现阶段,在建设社会主义的时期,一切赞成、拥护和参加社会主义建设事业的阶级、阶层和社会集团,都属于人民的范围;一切反抗社会主义革命和敌视、破坏社会主义建设的社会势力和社会集团,都是人民的敌人。

敌我之间的矛盾是对抗性的矛盾。人民内部的矛盾,在劳动人民之间说来,是非对抗性的;在被剥削阶级和剥削阶级之间说来,除了对抗性的一面以外,还有非对抗性的一面。人民内部的矛盾不是现在才有的,但是在各个革命时期和社会主义建设时期有着不同的内容。在我国现在的条件下,所谓人民内部的矛盾,包括工人阶级内部的矛盾,农民阶级内部的矛盾,知识分子内部的矛盾,工农两个阶级之间的矛盾,工人、农民同知识分子之间的矛盾,工人阶级和其它劳动人民同民族资产阶级之间的矛盾,民族资产阶级内部的矛盾,等等。我们的人民政府是真正代表人民利益的政府,是为人民服务的政府,但是它同人民群众之间也有一定的矛盾。这种矛盾包括国家利益、集体利益同个人利益之间的矛盾,民主同集中的矛盾,领导同被领导之间的矛盾,国家机关某些工作人员的官僚主义作风同群众之间的矛盾。这种矛盾也是人民内部的一个矛盾。一般说来,人民内部的矛盾,是在人民利益根本一致的基础上的矛盾。

在我们国家里,工人阶级同民族资产阶级的矛盾属于人民内部的矛盾。工人阶级和民族资产阶级的阶级斗争一般地属于人民内部的阶级斗争,这是因为我国的民族资产阶级有两面性。在资产阶级民主革命时期,它有革命性的一面,又有妥协性的一面。在社会主义革命时期,它有剥削工人阶级取得利润的一面,又有拥护宪法、愿意接受社会主义改造的一面。民族资产阶级和帝国主义、地主阶级、官僚资产阶级不同。工人阶级和民族资产阶级之间存在着剥削和被剥削的矛盾,这本来是对抗性的矛盾。但是在我国的具体条件下,这两个阶级的对抗性的矛盾如果处理得当,可以转变为非对抗性的矛盾,可以用和平的方法解决这个矛盾。如果我们处理不当,不是对民族资产阶级采取团结、批评、教育的政策,或者民族资产阶级不接受我们的这个政策,那末工人阶级同民族资产阶级之间的矛盾就会变成敌我之间的矛盾。

敌我之间和人民内部这两类矛盾的性质不同,解决的方法也不同。简单地说起来,前者是分清敌我的问题,后者是分清是非的问题。当然,敌我问题也是一种是非问题。比如我们同帝国主义、封建主义、官僚资本主义这些内外反动派,究竟谁是谁非,也是是非问题,但是这是和人民内部问题性质不同的另一类是非问题。

我们的国家是工人阶级领导的以工农联盟为基础的人民民主专政的国家。这个专政是干什么的呢?专政的第一个作用,就是压迫国家内部的反动阶级、反动派和反抗社会主义革命的剥削者,压迫那些对于社会主义建设的破坏者,就是为了解决国内敌我之间的矛盾。例如逮捕某些反革命分子并且将他们判罪,在一个时期内不给地主阶级分子和官僚资产阶级分子以选举权,不给他们发表言论的自由权利,都是属于专政的范围。为了维护社会秩序和广大人民的利益,对于那些盗窃犯、诈骗犯、杀人放火犯、流氓集团和各种严重破坏社会秩序的坏分子,也必须实行专政。专政还有第二个作用,就是防御国家外部敌人的颠覆活动和可能的侵略。在这种情况出现的时候,专政就担负着对外解决敌我之间的矛盾的任务。专政的目的是为了保卫全体人民进行和平劳动,将我国建设成为一个具有现代工业、现代农业和现代科学文化的社会主义国家。谁来行使专政呢?当然是工人阶级和在它领导下的人民。专政的制度不适用于人民内部。人民自己不能向自己专政,不能由一部分人民去压迫另一部分人民。人民中间的犯法分子也要受到法律的制裁,但是,这和压迫人民的敌人的专政是有原则区别的。在人民内部是实行民主集中制。我们的宪法规定:中华人民共和国公民有言论、出版、集会、结社、游行、示威、宗教信仰等等自由。我们的宪法又规定:国家机关实行民主集中制,国家机关必须依靠人民群众,国家机关工作人员必须为人民服务。我们的这个社会主义的民主是任何资产阶级国家所不可能有的最广大的民主。我们的专政,叫做工人阶级领导的以工农联盟为基础的人民民主专政。这就表明,在人民内部实行民主制度,而由工人阶级团结全体有公民权的人民,首先是农民,向着反动阶级、反动派和反抗社会主义改造和社会主义建设的分子实行专政。所谓有公民权,在政治方面,就是说有自由和民主的权利。

但是这个自由是有领导的自由,这个民主是集中指导下的民主,不是无政府状态。无政府状态不符合人民的利益和愿望。

匈牙利事件发生以后,我国有些人感到高兴。他们希望在中国也出现一个那样的事件,有成千上万的人上街,去反对人民政府。他们的这种希望是同人民群众的利益相违反的,是不可能得到人民群众支持的。匈牙利的一部分群众受了国内外反革命力量的欺骗,错误地用暴力行为来对付人民政府,结果使得国家和人民都吃了亏。几个星期的骚乱,给予经济方面的损失,需要长时间才能恢复。我国另有一些人在匈牙利问题上表现动摇,是因为他们不懂得世界上的具体情况。他们以为在我们的人民民主制度下自由太少了,不如西方的议会民主制度自由多。他们要求实行西方的两党制,这一党在台上,那一党在台下。但是这种所谓两党制不过是维护资产阶级专政的一种方法,它绝不能保障劳动人民的自由权利。实际上,世界上只有具体的自由,具体的民主,没有抽象的自由,抽象的民主。在阶级斗争的社会里,有了剥削阶级剥削劳动人民的自由,就没有劳动人民不受剥削的自由。有了资产阶级的民主,就没有无产阶级和劳动人民的民主。有些资本主义国家也容许共产党合法存在,但是以不危害资产阶级的根本利益为限度,超过这个限度就不容许了。要求抽象的自由、抽象的民主的人们认为民主是目的,而不承认民主是手段。民主这个东西,有时看来似乎是目的,实际上,只是一种手段。马克思主义告诉我们,民主属于上层建筑,属于政治这个范畴。这就是说,归根结蒂,它是为经济基础服务的。自由也是这样。民主自由都是相对的,不是绝对的,都是在历史上发生和发展的。在人民内部,民主是对集中而言,自由是对纪律而言。这些都是一个统一体的两个矛盾着的侧面,它们是矛盾的,又是统一的,我们不应当片面地强调某一个侧面而否定另一个侧面。在人民内部,不可以没有自由,也不可以没有纪律;不可以没有民主,也不可以没有集中。这种民主和集中的统一,自由和纪律的统一,就是我们的民主集中制。在这个制度下,人民享受着广泛的民主和自由;同时又必须用社会主义的纪律约束自己。这些道理,广大人民群众是懂得的。

我们主张有领导的自由,主张集中指导下的民主,这在任何意义上都不是说,人民内部的思想问题、是非的辨别问题,可以用强制的方法去解决。企图用行政命令的方法,用强制的方法解决思想问题,是非问题,不但没有效力,而且是有害的。我们不能用行政命令去消灭宗教,不能强制人们不信教。不能强制人们放弃唯心主义,也不能强制人们相信马克思主义。凡属于思想性质的问题,凡属于人民内部的争论问题,只能用民主的方法去解决,只能用讨论的方法、批评的方法、说服教育的方法去解决,而不能用强制的、压服的方法去解决。人民为了有效地进行生产、进行学习和有秩序地过生活,要求自己的政府、生产的领导者、文化教育机关的领导者发布各种适当的带强制性的行政命令。没有这种行政命令,社会秩序就无法维持,这是人们的常识所了解的。这同用说服教育的方法去解决人民内部的矛盾,是相辅相成的两个方面。为着维持社会秩序的目的而发布的行政命令,也要伴之以说服教育,单靠行政命令,在许多情况下就行不通。

在一九四二年,我们曾经把解决人民内部矛盾的这种民主的方法,具体化为一个公式,叫做“团结——批评——团结”。讲详细一点,就是从团结的愿望出发,经过批评或者斗争使矛盾得到解决,从而在新的基础上达到新的团结。按照我们的经验,这是解决人民内部矛盾的一个正确的方法。一九四二年,我们采用了这个方法解决共产党内部的矛盾,就是教条主义者和广大党员群众之间的矛盾,教条主义思想和马克思主义思想之间的矛盾。“左”倾教条主义者从前采用的党内斗争方法叫做“残酷斗争,无情打击”。这是一个错误的方法。我们在批评“左”倾教条主义的时候,就没有采取这个老方法,而采取了一个新方法,就是从团结的愿望出发,经过批评或者斗争,分清是非,在新的基础上达到新的团结。这个方法是在一九四二年整风的时候采用的。经过几年之后,到一九四五年中国共产党召开第七次全国代表大会的时候,果然达到了全党团结的目的,因此就取得了人民革命的伟大胜利。在这里,首先需要从团结的愿望出发。因为如果在主观上没有团结的愿望,一斗势必把事情斗乱,不可收拾,那还不是“残酷斗争,无情打击”?那还有什么党的团结?从这个经验里,我们找到了一个公式:团结——批评——团结。或者说,惩前毖后,治病救人。我们把这个方法推广到了党外。在各抗日根据地里,我们处理领导和群众的关系,处理军民关系、官兵关系、几部分军队之间的关系、几部分干部之间的关系,都采用了这个方法,并且得到了伟大的成功。这个问题,在我们党的历史上,还可以追溯到更远。自从一九二七年我们在南方建立革命军队和革命根据地开始,关于处理党群关系、军民关系、官兵关系以及其它人民内部关系,就是采用这个方法的。不过到了抗日时期,我们就把这个方法建立在更加自觉的基础之上了。全国解放以后,我们对民主党派和工商界也采取了“团结——批评——团结”这个方法。我们现在的任务,就是要在整个人民内部继续推广和更好地运用这个方法,要求所有的工厂、合作社、商店、学校、机关、团体,总之,六亿人口,都采用这个方法去解决他们内部的矛盾。

在一般情况下,人民内部的矛盾不是对抗性的。但是如果处理得不适当,或者失去警觉,麻痹大意,也可能发生对抗。这种情况,在社会主义国家通常只是局部的暂时的现象。这是因为社会主义国家消灭了人剥削人的制度,人民的利益在根本上是一致的。匈牙利事件所表现的那种范围相当宽广的对抗行动,是因为有内外反革命因素在起作用的缘故。这是一种特殊的也是暂时的现象。社会主义国家内部的反动派同帝国主义者互相勾结,利用人民内部的矛盾,挑拨离间,兴风作浪,企图实现他们的阴谋。匈牙利事件的这种教训,值得大家注意。

许多人觉得,提出采用民主方法解决人民内部矛盾的问题是一个新的问题。事实并不是这样。马克思主义者从来就认为无产阶级的事业只能依靠人民群众,共产党人在劳动人民中间进行工作的时候必须采取民主的说服教育的方法,决不允许采取命令主义态度和强制手段。中国共产党忠实地遵守马克思列宁主义的这个原则。我们历来就主张,在人民民主专政下面,解决敌我之间的和人民内部的这两类不同性质的矛盾,采用专政和民主这样两种不同的方法。这个意思,在我们党的过去的许多文件里和党的许多负责人的言论里,曾经说得很多。我在一九四九年所写的《论人民民主专政》里曾经说过:“对人民内部的民主方面和对反动派的专政方面,互相结合起来,就是人民民主专政”,解决人民内部的问题,“使用的方法,是民主的即说服的方法,而不是强迫的方法”。我在一九五○年六月第二次政治协商会议上的讲话里,又说过:“人民民主专政有两个方法。对敌人说来是用专政的方法,就是说在必要的时期内,不让他们参与政治活动,强迫他们服从人民政府的法律,强迫他们从事劳动并在劳动中改造他们成为新人。对人民说来则与此相反,不是用强迫的方法,而是用民主的方法,就是说必须让他们参与政治活动,不是强迫他们做这样做那样,而是用民主的方法向他们进行教育和说服的工作。这种教育工作是人民内部的自我教育工作,批评和自我批评的方法就是自我教育的基本方法。”过去我们已经多次讲过用民主方法解决人民内部矛盾这个问题,并且在工作中基本上就是这样做的,很多干部和人民都在实际上懂得这个问题。为什么现在又有人觉得这是一个新问题呢?这是因为过去国内外的敌我斗争很尖锐,人民内部矛盾还不像现在这样被人们注意的缘故。

许多人对于敌我之间的和人民内部的这两类性质不同的矛盾分辨不清,容易混淆在一起。应该承认,这两类矛盾有时是容易混淆的。我们在过去工作中也曾经混淆过。在肃清反革命分子的工作中,错误地把好人当坏人,这种情形,从前有过,现在也还有。我们的错误没有扩大化,是由于我们在政策中规定了必须分清敌我,错了就要平反。

马克思主义的哲学认为,对立统一规律是宇宙的根本规律。这个规律,不论在自然界、人类社会和人们的思想中,都是普遍存在的。矛盾着的对立面又统一,又斗争,由此推动事物的运动和变化。矛盾是普遍存在的,不过按事物的性质不同,矛盾的性质也就不同。对于任何一个具体的事物说来,对立的统一是有条件的、暂时的、过渡的,因而是相对的,对立的斗争则是绝对的。这个规律,列宁讲得很清楚。这个规律,在我国,懂得的人逐渐多起来了。但是,对于许多人说来,承认这个规律是一回事,应用这个规律去观察问题和处理问题又是一回事。许多人不敢公开承认我国人民内部还存在着矛盾,正是这些矛盾推动着我们的社会向前发展。许多人不承认社会主义社会还有矛盾,因而使得他们在社会矛盾面前缩手缩脚,处于被动地位;不懂得在不断地正确处理和解决矛盾的过程中,将会使社会主义社会内部的统一和团结日益巩固。这样,就有必要在我国人民中,首先是在干部中,进行解释,引导人们认识社会主义社会中的矛盾,并且懂得采取正确的方法处理这种矛盾。

社会主义社会的矛盾同旧社会的矛盾,例如同资本主义社会的矛盾,是根本不相同的。资本主义社会的矛盾表现为剧烈的对抗和冲突,表现为剧烈的阶级斗争,那种矛盾不可能由资本主义制度本身来解决,而只有社会主义革命才能够加以解决。社会主义社会的矛盾是另一回事,恰恰相反,它不是对性的矛盾,它可以经过社会主义制度本身,不断地得到解决。

在社会主义社会中,基本的矛盾仍然是生产关系和生产力之间的矛盾,上层建筑和经济基础之间的矛盾。不过社会主义社会的这些矛盾,同旧社会的生产关系和生产力的矛盾、上层建筑和经济基础的矛盾,具有根本不同的性质和情况罢了。我国现在的社会制度比较旧时代的社会制度要优胜得多。如果不优胜,旧制度就不会被推翻,新制度就不可能建立。所谓社会主义生产关系比较旧时代生产关系更能够适合生产力发展的性质,就是指能够容许生产力以旧社会所没有的速度迅速发展,因而生产不断扩大,因而使人民不断增长的需要能够逐步得到满足的这样一种情况。旧中国在帝国主义、封建主义和官僚资本主义的统治下,生产力的发展一直是非常缓慢的。解放前五十多年间,全国除东北外,钢的生产一直只有几万吨;加上东北,全国的最高年产量也不过是九十多万吨。在一九四九年,全国钢产量只有十几万吨。但是全国解放不过七年,钢的生产便已达到四百几十万吨。旧中国几乎没有机器制造业,更没有汽车制造业和飞机制造业,而这些现在都建立起来了。当人民推翻了帝国主义、封建主义和官僚资本主义的统治之后,中国要向哪里去?向资本主义,还是向社会主义?有许多人在这个问题上的思想是不清楚的。事实已经回答了这个问题:只有社会主义能够救中国。社会主义制度促进了我国生产力的突飞猛进的发展,这一点,甚至连国外的敌人也不能不承认了。

但是,我国的社会主义制度还刚刚建立,还没有完全建成,还不完全巩固。在工商业的公私合营企业中,资本家还拿取定息,也就是还有剥削;就所有制这点上说,这类企业还不是完全的社会主义性质的。农业生产合作社和手工业生产合作社有一部分也还是半社会主义性质的;完全社会主义化的合作社在所有制的某些个别问题上,还需要继续解决。在各经济部门中的生产和交换的相互关系,还在按照社会主义的原则逐步建立,逐步找寻比较适当的形式。在全民所有制经济和集体所有制经济里面,在这两种社会主义经济形式之间,积累和消费的分配问题是一个复杂的问题,也不容易一下子解决得完全合理。总之,社会主义生产关系已经建立起来,它是和生产力的发展相适应的;但是,它又还很不完善,这些不完善的方面和生产力的发展又是相矛盾的。除了生产关系和生产力发展的这种又相适应又相矛盾的情况以外,还有上层建筑和经济基础的又相适应又相矛盾的情况。人民民主专政的国家制度和法律,以马克思列宁主义为指导的社会主义意识形态,这些上层建筑对于我国社会主义改造的胜利和社会主义劳动组织的建立起了积极的推动作用,它是和社会主义的经济基础即社会主义的生产关系相适应的;但是,资产阶级意识形态的存在,国家机构中某些官僚主义作风的存在,国家制度中某些环节上缺陷的存在,又是和社会主义的经济基础相矛盾的。我们今后必须按照具体的情况,继续解决上述的各种矛盾。当然,在解决这些矛盾以后,又会出现新的问题,新的矛盾,又需要人们去解决。例如,在客观上将会长期存在的社会生产和社会需要之间的矛盾,就需要人们时常经过国家计划去调节。我国每年作一次经济计划,安排积累和消费的适当比例,求得生产和需要之间的平衡。所谓平衡,就是矛盾的暂时的相对的统一。过了一年,就整个说来,这种平衡就被矛盾的斗争所打破了,这种统一就变化了,平衡成为不平衡,统一成为不统一,又需要作第二年的平衡和统一。这就是我们计划经济的优越性。事实上,每月每季都在局部地打破这种平衡和统一,需要作出局部的调整。有时因为主观安排不符合客观情况,发生矛盾,破坏平衡,这就叫做犯错误。矛盾不断出现,又不断解决,就是事物发展的辩证规律。

现在的情况是:革命时期的大规模的急风暴雨式的群众阶级斗争基本结束,但是阶级斗争还没有完全结束;广大群众一面欢迎新制度,一面又还感到还不大习惯;政府工作人员经验也还不够丰富,对一些具体政策的问题,应当继续考察和探索。这就是说,我们的社会主义制度还需要有一个继续建立和巩固的过程,人民群众对于这个新制度还需要有一个习惯的过程,国家工作人员也需要一个学习和取得经验的过程。在这个时候,我们提出划分敌我和人民内部两类矛盾的界限,提出正确处理人民内部矛盾的问题,以便团结全国各族人民进行一场新的战争——向自然界开战,发展我们的经济,发展我们的文化,使全体人民比较顺利地走过目前的过渡时期,巩固我们的新制度,建设我们的新国家,就是十分必要的了。

二、肃反问题

肃清反革命分子的问题是敌我矛盾的斗争问题。在人民内部,有些人对于肃反问题的看法,也有一些不同。有两种人的意见,和我们的意见不相同。有右倾思想的人不分敌我,认敌为我。广大群众认为是敌人的人,他们却认为是朋友。有“左”倾思想的人则把敌我矛盾扩大化,以至把某些人民内部的矛盾也看作敌我矛盾,把某些本来不是反革命的人也看作反革命。这两种看法都是错误的,都不能正确地处理肃反问题,也不能正确地估计我们的肃反工作。

为了正确地估计我国的肃反工作,我们不妨看一看匈牙利事件对于我们国家的影响。匈牙利事件发生以后,在我国一部分知识分子中有些动荡,但是没有引起什么风浪。这是什么原因呢?应该说,原因之一,就是我们相当彻底地肃清了反革命。

当然,我们国家的巩固,首先不是由于肃反。我们国家的巩固,首先是由于我们有经过几十年革命斗争锻炼的共产党和解放军,有经过几十年革命斗争锻炼的劳动人民。我们的党和军队是在群众中生了根的,是在长期革命火焰中锻炼出来的是有战斗力的。我们的人民共和国是经过革命根据地逐步发展起来的,不是突然建立起来的。有些民主人士也受过不同程度的锻炼,同我们共过患难。有些知识分子经历过反对帝国主义和反动势力的斗争的锻炼,许多人经历过解放以后的以分清敌我界限为目标的思想改造。此外,我们国家的巩固,还由于我们的经济措施根本上是正确的;人民生活是稳定的,并且逐步有所改善;我们对于民族资产阶级和其它阶级的政策,也是正确的,等等。但是,我们在肃清反革命方面的成功,无疑是我们国家巩固的重要原因之一。由于这一切,我们的大学生虽然还有许多人是非劳动人民家庭出身的子女,但是除了少数例外,都是爱国的,都是拥护社会主义的,他们在匈牙利事件时期没有发生波动。民族资产阶级也是这样。更不要说工农基本群众了。

解放以后,我们肃清了一批反革命分子。一些有严重罪行的反革命分子被处了死刑。这是完全必要的,这是广大群众的要求,这是为了解放长期被反革命分子和各种恶霸分子压迫的广大群众,也就是为了解放生产力。我们如果不这样做,人民群众就会抬不起头来。从一九五六年以来,情况就根本改变了。就全国说来,反革命分子的主要力量已经肃清。我们的根本任务已经由解放生产力变为在新的生产关系下面保护和发展生产力。有些人不了解我们今天的政策适合于今天的情况,过去的政策适合于过去的情况,想利用今天的政策去翻过去的案,想否定过去肃反工作的巨大成绩,这是完全错误的,这是人民群众所不允许的。

我们的肃反工作,成绩是主要的,但是也有错误。过火的,漏掉的,都有。我们的方针是:“有反必肃,有错必纠”。我们在肃反工作中的路线是群众肃反的路线。采取了群众路线,工作中当然也会发生毛病,但是毛病会比较少一些,错误会比较容易纠正些。群众在斗争中得到了经验。做得正确,得了做得正确的经验。犯了错误,也得了犯错误的经验。

在肃反工作中,凡是已经发现了的错误,我们都已经采取了或者正在采取纠正的步骤。没有发现的,一经发现,我们就准备纠正。原来在什么范围内弄错的,也应该在什么范围内宣布平反。我提议今年或者明年对于肃反工作全面检查一次,总结经验,发扬正气,打击歪风。中央由人大常委会和政协常委会主持,地方由省市人民委员会和政协委员会主持。在检查工作的时候,我们对广大干部和积极分子不要泼冷水,而要帮助他们。向广大干部和积极分子泼冷水是不对的。但是发现了错误,一定要改正。无论公安部门、检察部门、司法部门、监狱、劳动改造的管理机关,都应该采取这个态度。我们希望人大常务委员、政协委员、人民代表,凡是有可能的,都参加这样的检查。这对于健全我们的法制,对于正确处理反革命分子和其它犯罪分子,会有帮助的。

目前关于反革命分子的情况,可以用这样两句话来说明:还有反革命,但是不多了。首先是还有反革命。有人说,已经没有了,天下太平了,可以把枕头塞得高高地睡觉了。这是不合事实的。事实是还有(当然不是说每一个地方每一个单位都有),还必须继续和他们作斗争。必须懂得,没有肃清的暗藏的反革命分子是不会死心的,他们必定要乘机捣乱。美帝国主义者和蒋介石集团经常还在派遣特务到我们这里来进行破坏活动。原有的反革命分子肃清了,还可能出现一些新的反革命分子。如果我们丧失警惕性,那就会上大当,吃大亏。不管什么地方出现反革命分子捣乱,就应当坚决消灭他。但是就全国来说,反革命分子确实不多了。如果说现在全国还有很多反革命分子,这个意见也是错误的。如果接受这种估计,结果也会搞乱。

三、农业合作化问题

我国有五亿多农业人口,农民的情况如何,对于我国经济的发展和政权的巩固,关系极大。我认为,情况根本上是好的。合作化完成了,这就解决了我国社会主义工业化同个体农业经济之间的大矛盾。合作化迅速完成,有些人担心会不会出毛病。幸好,毛病有一些,不大,基本上是健全的。农民生产很起劲,虽然去年的水旱风灾比过去几年中哪一年都大,但是全国的粮食仍然增产。现在有一些人却在说合作化不行,合作化没有优越性,吹来了一股小台风。合作化究竟有没有优越性呢?今天会场上发的文件里面,有一个关于河北省遵化县王国藩合作社的材料,大家可以看一看。这个合作社所在的地方是一个山地,历来很穷,年年靠人民政府运粮去救济。一九五三年开始办社的时候,人们把它叫做“穷棒子社”。经过了四年艰苦奋斗,一年一年好起来,绝大多数的社员成了余粮户。王国藩合作社能做到的,别的合作社,在正常情况下也应该能做到,或者时间长一点也应该能做到。由此可见,那些说合作化不好了的议论是没有根据的。

由此也可看出,合作社一定要在艰苦奋斗中建立起来。任何新生事物的成长都是要经过艰难曲折的。在社会主义事业中,要想不经过艰难曲折,不付出极大努力,总是一帆风顺,容易得到成功,这种想法,只是幻想。

积极拥护合作社的是些什么人呢?是绝大多数贫农和下中农,他们占农村人口百分之七十以上。其余的人,大多数也对合作社寄予希望。真正不满意的只占极少数。许多人没有分析这种情况,没有对合作社的成绩和缺点以及缺点产生的根源作全面的考察,把局部和片面当成了全体,这就在一些人中间刮起了一阵所谓合作社没有优越性的小台风。

要多少时间合作社才能巩固,认为合作社没有优越性的议论才会收场呢?根据许多合作社发展的经验来看,大概需要五年,或者还要多一点时间。现在,全国大多数的合作社还只有一年多的历史,我们就要求它们那么好,这是不合理的。依我看,第一个五年计划期内建成合作社,第二个五年计划期内合作社能得到巩固,那就很好了。

合作社正在经历一个逐步巩固的过程。它还存在着一些需要解决的矛盾。例如,在国家同合作社之间,在合作社内部,在合作社同合作社相互之间,都有一些矛盾需要解决。

我们必须经常注意从生产问题和分配问题上处理上述矛盾。在生产问题上,一方面,合作社经济要服从国家统一经济计划的领导,同时在不违背国家的统一计划和政策法令下保持自己一定的灵活性和独立性;另一方面,参加合作社的各个家庭,除了自留地和其它一部分个体经营的经济可以由自己作出适当的计划以外,都要服从合作社或者生产队的总计划。在分配问题上,我们必须兼顾国家利益、集体利益和个人利益。对于国家的税收、合作社的积累、农民的个人收入这三方面的关系,必须处理适当,经常注意调节其中的矛盾。国家要积累,合作社也要积累,但是都不能过多。我们要尽可能使农民能够在正常年景下,从增加生产中逐年增加个人收入。

许多人说农民苦,这种意见对不对呢?就一方面说来是对的。这就是说,由于我国被帝国主义者和他们的代理人压迫剥削了一百多年,变成一个很穷的国家,不但农民的生活水平低,工人和知识分子的生活水平也都还低。要有几十年时间,经过艰苦的努力,才能将全体人民的生活水平逐步提高起来。这样说“苦”就恰当了。就另一方面说来是不对的。这就是说,解放七年以来,农民生活没有改善,单单改善了工人的生活。其实,工人农民的生活,除极少数人以外,都已经有了一些改善。解放以来,农民免除了地主的剥削,生产逐年发展。以粮食为例,一九四九年全国产粮只有二千一百几十亿斤,到一九五六年产粮达到三千六百几十亿斤,增加了将近一千五百亿斤。国家征收的农业税并不算重,每年只有三百多亿斤。每年以正常价格从农民那里购粮也只有五百多亿斤。两项共八百几十亿斤。这些粮食销售在农村和农村附近的集镇的,占了一半以上。由此看来,不能说农民生活没有改善。我们准备在几年内,把征粮和购粮的数量大体上稳定在八百几十亿斤的水平上,使农业得到发展,使合作社得到巩固,使现在还存在的农村中一小部分缺粮户不再缺粮,除了专门经营经济作物的某些农户以外,统统变为余粮户或者自给户,使农村中没有了贫农,使全体农民达到中农和中农以上的生活水平。至于简单地拿农民每人每年平均所得和工人每人每年平均所得相比较,说一个低了,一个高了,这是不适当的。工人的劳动生产率比农民高得多,而农民的生活费用比城市工人又省得多,所以不能说工人特别得到国家的优待。有少部分工人的工资以及有些国家机关工作人员的工资是高了一些,农民看了不满意是有理由的,斟酌情况作一些适当的调整,是必要的。

四、工商业者问题

我国社会制度的改革,除了农业合作化和手工业合作化以外,私营工商业改变为公私合营企业,也在一九五六年完成了。这件事所以做得这样迅速和顺利,是跟我们把工人阶级同民族资产阶级之间的矛盾当做人民内部矛盾来处理,密切相关的。这个阶级矛盾是否完全解决了呢?还没有。还要经过相当的时间才能够完全解决。但是现在有些人说:资本家已经改造得和工人差不多了,用不着再改造了。甚至有人说,资本家比工人还要高明一点。也有人说,如果要改造,为什么工人阶级不改造?这些议论对不对呢?当然不对。

在建设社会主义社会的过程中,人人需要改造,剥削者要改造,劳动者也要改造,谁说工人阶级不要改造?当然,剥削者的改造和劳动者的改造是两种不同性质的改造,不能混为一谈。工人阶级要在阶级斗争中和向自然界的斗争中改造整个社会,同时也就改造自己。工人阶级必须在工作中不断学习,逐步克服自己的缺点,永远也不能停止。拿我们这些人来说,很多人每年都有一些进步,也就是说,每年都在改造。我这个人从前就有过各种非马克思主义的思想,马克思主义是后来才接受的。我在书本上学了一点马克思主义,初步地改造了自己的思想,但是主要的还是在长期阶级斗争中改造过来的。而且今后还要继续学习,才能再有一些进步,否则就要落后了。难道资本家就那么高明,反而再不需要改造了吗?

有人说,中国资产阶级现在已经没有两面性了,只有一面性。这是不是事实呢?不是事实。一方面,资产阶级分子已经成为公私合营企业中的管理人员,正处在由剥削者变为自食其力的劳动者的转变过程中;另一方面,他们现在还在公私合营的企业中拿定息,这就是说,他们的剥削根子还没有脱离。他们同工人阶级的思想感情、生活习惯还有一个不小的距离。怎么能说已经没有了两面性呢?就是不拿定息,摘掉了资产阶级的帽子,也还需要一个相当的时间继续进行思想改造。如果认为资产阶级已经没有了两面性,那末资本家的改造和学习的任务也就没有了。

应该说,这种意见不仅不符合工商业者的实际情况,也不符合工商业者大多数人的愿望。在过去几年中,大多数工商业者都是愿意学习的,并且有了显着的进步。工商业者的彻底改造必须是在工作中间,他们应当在企业内同职工一起劳动,把企业作为自我改造的基地。但是经过学习改变自己的某些旧观点,也是重要的。工商业者的学习,应当以自愿为基础。许多工商业者在讲习班里学习了几十天,回到工厂,同工人群众和公方代表有了更多的共同的语言,改善了共同工作的条件。他们从亲身的经验懂得,继续学习,继续改造自已,对于他们是有益的。刚才所说的那种认为不需要学习,不需要改造的意见,并不能代表工商业者中大多数人的意见,只是少数人的意见。

五、知识分子问题

我国人民内部的矛盾,在知识分子中间也表现出来了。过去为旧社会服务的几百万知识分子,现在转到为新社会服务,这里就存在着他们如何适应新社会需要和我们如何帮助他们适应新社会需要的问题。这也是人民内部的一个矛盾。

我国知识分子的大多数,在过去七年中已经有了显着的进步。他们表示赞成社会主义制度。他们中间有许多人正在用功学习马克思主义,有一部分人已经成为共产主义者。这部分人目前虽然还是少数,但是正在逐渐增多。当然,知识分子中间有一些人现在仍然怀疑或者不同意社会主义,这部分人只占少数。

我国的艰巨的社会主义建设事业,需要尽可能多的知识分子为它服务。凡是真正愿意为社会主义事业服务的知识分子,我们都应当给予信任,从根本上改善同他们的关系,帮助他们解决各种必须解决的问题,使他们得以积极地发挥他们的才能。我们有许多同志不善于团结知识分子,用生硬的态度对待他们,不尊重他们的劳动,在科学文化工作中不适当地干预那些不应当干预的事务。所有这些缺点必须加以克服。

广大的知识分子虽然已经有了进步,但是不应当因此自满。为了充分适应新社会的需要,为了同工人农民团结一致,知识分子必须继续改造自己,逐步地抛弃资产阶级的世界观而树立无产阶级的、共产主义的世界观。世界观的转变是一个根本的转变,现在多数知识分子还不能说已经完成了这个转变。我们希望我国的知识分子继续前进,在自己的工作和学习的过程中,逐步地树立共产主义的世界观,逐步地学好马克思列宁主义,逐步地同工人农民打成一片,而不要中途停顿,更不要向后倒退,倒退是没有出路的。由于我国的社会制度已经起了变化,资产阶级思想的经济基础已经基本上消灭了,这就使大量知识分子的世界观不但有了改变的必要,而且有了改变的可能。但是世界观的彻底改变需要一个很长的时间,我们应当耐心地做工作,不能急躁。事实上必定会有一些人在思想上始终不愿意接受马克思列宁主义,不愿意接受共产主义,对于这一部分人不要苛求;只要他们服从国家的要求,从事正常的劳动,我们就应当给他们以适当工作的机会。

在知识分子和青年学生中间,最近一个时期,思想政治工作减弱了,出现了一些偏向。在一些人的眼中,好像什么政治,什么祖国的前途、人类的理想,都没有关心的必要。好像马克思主义行时了一阵,现在就不那么行时了。针对着这种情况,现在需要加强思想政治工作。不论是知识分子,还是青年学生,都应该努力学习。除了学习专业之外,在思想上要有所进步,政治上也要有所进步,这就需要学习马克思主义,学习时事政治。没有正确的政治观点,就等于没有灵魂。过去的思想改造是必要的,收到了积极的效果。但是在做法上有些粗糙,伤了一些人,这是不好的。这个缺点,今后必须避免。思想政治工作,各个部门都要负责任。共产党应该管,青年团应该管,政府主管部门应该管,学校的校长教师更应该管。我们的教育方针,应该使受教育者在德育、智育、体育几方面都得到发展,成为有社会主义觉悟的有文化的劳动者。要提倡勤俭建国。要使全体青年们懂得,我们的国家现在还是一个很穷的国家,并且不可能在短时间内根本改变这种状态,全靠青年和全体人民在几十年时间内,团结奋斗,用自己的双手创造出一个富强的国家。社会主义制度的建立给我们开辟了一条到达理想境界的道路,而理想境界的实现还要靠我们的辛勤劳动。有些青年人以为到了社会主义社会就应当什么都好了,就可以不费气力享受现成的幸福生活了,这是一种不实际的想法。

六、少数民族问题

我国少数民族有三千多万人,虽然只占全国总人口的百分之六,但是居住地区广大,约占全国总面积的百分之五十至六十。所以汉族和少数民族的关系一定要搞好。这个问题的关键是克服大汉族主义。在存在有地方民族主义的少数民族中间,则应当同时克服地方民族主义。无论是大汉族主义或者地方民族主义,都不利于各族人民的团结,这是应当克服的一种人民内部的矛盾。在这一方面,我们已经做了一些工作,在大多数少数民族地区民族关系比较从前大有改进,但是仍然存在着一些尚待解决的问题。在一部分地区,大汉族主义和地方民族主义都还严重地存在,必须给以足够的注意。经过各族人民几年来的努力,我国少数民族地区绝大部分都已经基本上完成了民主改革和社会主义改造。西藏由于条件还不成熟,还没有进行民主改革。按照中央和西藏地方政府的十七条协议,社会制度的改革必须实行,但是何时实行,要待西藏大多数人民群众和领袖人物认为可行的时候,才能作出决定,不能性急。现在已决定在第二个五年计划期间不进行改革。在第三个五年计划期内是否进行改革,要到那时看情况才能决定。

七、统筹兼顾、适当安排

这里所说的统筹兼顾,是指对于六亿人口的统筹兼顾。我们作计划、办事、想问题,都要从我国有六亿人口这一点出发,千万不要忘记这一点。为什么要提出这样一个问题,难道还有人不知道我国有六亿人口吗?知道是知道的,不过办起事来有些人就忘记了,似乎人越少越好,圈子紧缩得越小越好。抱有这种小圈子主义的人们,对于这样一种思想是抵触的:调动一切积极因素,团结一切可能团结的人,并且尽可能地将消极因素转变为积极因素,为建设社会主义社会这个伟大的事业服务。我希望这些人扩大眼界,真正承认我国有六亿人口,承认这是一个客观存在,这是我们的本钱。我国人多,是好事,当然也有困难。我们各方面的建设事业都在蓬勃地发展着,成绩很大,但是,在目前社会大变动的过渡时期,困难问题还是很多的。又发展又困难,这就是矛盾。任何矛盾不但应当解决,也是完全可以解决的。我们的方针是统筹兼顾、适当安排。无论粮食问题,灾荒问题,就业问题,教育问题,知识分子问题,各种爱国力量的统一战线问题,少数民族问题,以及其它各项问题,都要从对全体人民的统筹兼顾这个观点出发,就当时当地的实际可能条件,同各方面的人协商,作出各种适当的安排。决不可以嫌人多,嫌人落后,嫌事情麻烦难办,推出门外了事。我这样说,是不是要把一切人一切事都由政府包下来呢?当然不是。许多人,许多事,可以由社会团体想办法,可以由群众直接想办法,他们是能够想出很多好的办法来的。而这也就包括在统筹兼顾、适当安排的方针之内,我们应当指导社会团体和各地群众这样做。

八、关于百花齐放、百家争鸣、长期共存、互相监督

百花齐放,百家争鸣,长期共存,互相监督,这几个口号是怎样提出来的呢?它是根据中国的具体情况提出来的,是在承认社会主义社会仍然存在着各种矛盾的基础上提出来的,是在国家需要迅速发展经济和文化的迫切要求上提出来的。百花齐放、百家争鸣的方针,是促进艺术发展和科学进步的方针,是促进我国的社会主义文化繁荣的方针。艺术上不同的形式和风格可以自由发展,科学上不同的学派可以自由争论。利用行政力量,强制推行一种风格,一种学派,禁止另一种风格,另一种学派,我们认为会有害于艺术和科学的发展。艺术和科学中的是非问题,应当通过艺术界科学界的自由讨论去解决,通过艺术和科学的实践去解决,而不应当采取简单的方法去解决。为了判断正确的东西和错误的东西,常常需要有考验的时间。历史上新的正确的东西,在开始的时候常常得不到多数人承认,只能在斗争中曲折地发展。正确的东西,好的东西,人们一开始常常不承认它们是香花,反而把它们看作毒草。哥白尼关于太阳系的学说,达尔文的进化论,都曾经被看作是错误的东西,都曾经经历艰苦的斗争。我国历史上也有许多这样的事例。同旧社会比较起来,在社会主义社会中,新生事物的成长条件,和过去根本不同了,好得多了。但是压抑新生力量,压抑合理的意见,仍然是常有的事。不是由于有意压抑,只是由于鉴别不清,也会妨碍新生事物的成长。因此,对于科学上、艺术上的是非,应当保持慎重的态度,提倡自由讨论,不要轻率地作结论。我们认为,采取这种态度可以帮助科学和艺术得到比较顺利的发展。

马克思主义也是在斗争中发展起来的。马克思主义在开始的时候受过种种打击,被认为是毒草。现在它在世界上的许多地方还在继续受打击,还被认为是毒草。在社会主义国家里,马克思主义的地位不同了。但是就是在社会主义国家,还是有非马克思主义的思想存在,也有反马克思主义的思想存在。在我国,虽然社会主义改造,在所有制方面说来,已经基本完成,革命时期的大规模的急风暴雨式的群众阶级斗争已经基本结束,但是,被推翻的地主买办阶级的残余还是存在,资产阶级还是存在,小资产阶级刚刚在改造。阶级斗争并没有结束。无产阶级和资产阶级之间的阶级斗争,各派政治力量之间的阶级斗争,无产阶级和资产阶级之间在意识形态方面的阶级斗争,还是长时期的,曲折的,有时甚至是很激烈的。无产阶级要按照自己的世界观改造世界,资产阶级也要按照自己的世界观改造世界。在这一方面,社会主义和资本主义之间谁胜谁负的问题还没有真正解决。无论在全人口中间,或者在知识分子中间,马克思主义者仍然是少数。因此,马克思主义仍然必须在斗争中发展。马克思主义必须在斗争中才能发展,不但过去是这样,现在是这样,将来也必然还是这样。正确的东西总是在同错误的东西作斗争的过程中发展起来的。真的、善的、美的东西总是在同假的、恶的、丑的东西相比较而存在,相斗争而发展的。当着某一种错误的东西被人类普遍地抛弃,某一种真理被人类普遍地接受的时候,更加新的真理又在同新的错误意见作斗争。这种斗争永远不会完结。这是真理发展的规律,当然也是马克思主义发展的规律。

我国社会主义和资本主义之间在意识形态方面的谁胜谁负的斗争,还需要一个相当长的时间才能解决。这是因为资产阶级和从旧社会来的知识分子的影响还要在我国长期存在,作为阶级的意识形态,还要在我国长期存在。如果对于这种形势认识不足,或者根本不认识,那就要犯绝大的错误,就会忽视必要的思想斗争。思想斗争同其它的斗争不同,它不能采取粗暴的强制的方法,只能用细致的讲理的方法。现在社会主义在意识形态的斗争中,具有优胜的条件。政权的基本力量是在无产阶级领导下的劳动人民手里。共产党有强大的力量和很高的威信。在我们的工作中尽管有缺点,有错误,但是每一个公正的人都可以看到,我们对人民是忠诚的,我们有决心有能力同人民在一起把祖国建设好,我们已经得到巨大的成就,并且将继续得到更巨大的成就。资产阶级分子和从旧社会来的知识分子的绝大多数都是爱国的,他们愿意为蒸蒸日上的社会主义祖国服务,并且懂得如果离开社会主义事业,离开共产党所领导的劳动人民,他们就会无所依靠,而不可能有任何光明的前途。

人们问:在我们国家里,马克思主义已经被大多数人承认为指导思想,那末,能不能对它加以批评呢?当然可以批评。马克思主义是一种科学真理,它是不怕批评的。如果马克思主义害怕批评,如果可以批评倒,那末马克思主义就没有用了。事实上,唯心主义者不是每天都在用各种形式批评马克思主义吗?抱着资产阶级思想、小资产阶级思想而不愿意改变的人们,不是也在用各种形式批评马克思主义吗?马克思主义者不应该害怕任何人批评。相反,马克思主义者就是要在人们的批评中间,就是要在斗争的风雨中间,锻炼自己,发展自己,扩大自己的阵地。同错误思想作斗争,好比种牛痘,经过了牛痘疫苗的作用,人身上就增强免疫力。在温室里培养出来的东西,不会有强大的生命力。实行百花齐放、百家争鸣的方针,并不会削弱马克思主义在思想界的领导地位,相反地正是会加强它的这种地位。

对于非马克思主义的思想,应该采取什么方针呢?对于明显的反革命分子,破坏社会主义事业的分子,事情好办,剥夺他们的言论自由就行了。对于人民内部的错误思想,情形就不相同。禁止这些思想,不允许这些思想有任何发表的机会,行不行呢?当然不行。对待人民内部的思想问题,对待精神世界的问题,用简单的方法去处理,不但不会收效,而且非常有害。不让发表错误意见,结果错误意见还是存在着。而正确的意见如果是在温室里培养出来的,如果没有见过风雨,没有取得免疫力,遇到错误意见就不能打胜仗。因此,只有采取讨论的方法,批评的方法,说理的方法,才能真正发展正确的意见,克服错误的意见,才能真正解决问题。

资产阶级、小资产阶级,他们的思想意识是一定要反映出来的。一定要在政治问题和思想问题上,用各种办法顽强地表现他们自己。要他们不反映不表现,是不可能的。我们不应当用压制的办法不让他们表现,而应当让他们表现,同时在他们表现的时候,和他们辩论,进行适当的批评。毫无疑问,我们应当批评各种各样的错误思想。不加批评,看着错误思想到处泛滥,任凭它们去占领市场,当然不行。有错误就得批判,有毒草就得进行斗争。但是这种批评不应当是教条主义的,不应当用形而上学方法,应当力求用辩证方法。要有科学的分析,要有充分的说服力。教条主义的批评不能解决问题。我们是反对一切毒草的,但是我们必须谨慎地辨别什么是真的毒草,什么是真的香花。我们要同群众一起来学会谨慎地辨别香花和毒草,并且一起来用正确的方法同毒草作斗争。

我们在批判教条主义的时候,必须同时注意对修正主义的批判。修正主义,或者右倾机会主义,是一种资产阶级思潮,它比教条主义有更大的危险性。修正主义者,右倾机会主义者,口头上也挂着马克思主义,他们也在那里攻击“教条主义”。但是他们所攻击的正是马克思主义的最根本的东西。他们反对或者歪曲唯物论和辩证法,反对或者企图削弱人民民主专政和共产党的领导,反对或者企图削弱社会主义改造和社会主义建设。在我国社会主义革命取得基本胜利以后,社会上还有一部分人梦想恢复资本主义制度,他们要从各个方面向工人阶级进行斗争,包括思想方面的斗争。而在这个斗争中,修正主义者就是他们最好的助手。

百花齐放、百家争鸣这两个口号,就字面看,是没有阶级性的,无产阶级可以利用它们,资产阶级也可以利用它们,其它的人们也可以利用它们。所谓香花和毒草,各个阶级、阶层和社会集团也有各自的看法。那末,从广大人民群众的观点看来,究竟什么是我们今天辨别香花和毒草的标准呢?在我国人民的政治生活中,应当怎样来判断我们的言论和行动的是非呢?我们以为,根据我国的宪法的原则,根据我国最大多数人民的意志和我国各党派历次宣布的共同的政治主张,这种标准可以大致规定如下:(一)有利于团结全国各族人民,而不是分裂人民;(二)有利于社会主义改造和社会主义建设,而不是不利于社会主义改造和社会主义建设;(三)有利于巩固人民民主专政,而不是破坏或者削弱这个专政;(四)有利于巩固民主集中制,而不是破坏或者削弱这个制度;(五)有利于巩固共产党的领导,而不是摆脱或者削弱这种领导;(六)有利于社会主义的国际团结和全世界爱好和平人民的国际团结,而不是有损于这些团结。这六条标准中,最重要的是社会主义道路和党的领导两条。提出这些标准,是为了帮助人民发展对于各种问题的自由讨论,而不是为了妨碍这种讨论。不赞成这些标准的人们仍然可以提出自己的意见来辩论。但是大多数人有了明确的标准,就可以使批评和自我批评沿着正确的轨道前进,就可以用这些标准去鉴别人们的言论行动是否正确,究竟是香花还是毒草。这是一些政治标准。为了鉴别科学论点的正确或者错误,艺术作品的艺术水准如何,当然还需要一些各自的标准。但是这六条政治标准对于任何科学艺术的活动也都是适用的。在我国这样的社会主义国家里,难道有什么有益的科学艺术活动会违反这几条政治标准的吗?

以上所说的观点,都是从我国的具体的历史条件出发的。各个社会主义国家和各国共产党的情况各不相同。因此,我们并不认为,它们必须或者应当采取中国的做法。

“长期共存、互相监督”这个口号,也是我国具体的历史条件的产物。这个口号并不是突然提出来的,它已经经过了好几年的酝酿。长期共存的思想已经存在很久了。到去年,社会主义制度已基本建立,这些口号就明确地提出来了。为什么要让资产阶级和小资产阶级的民主党派同工人阶级政党长期共存呢?这是因为凡属一切确实致力于团结人民从事社会主义事业的、得到人民信任的党派,我们没有理由不对它们采取长期共存的方针。我在一九五○年六月第二次政治协商会议上,就已经这样说过:“只要谁肯真正为人民效力,在人民还有困难的时期内确实帮了忙,做了好事,并且是一贯地做下去,并不半途而废,那末,人民和人民的政府是没有理由不要他的,是没有理由不给他以生活的机会和效力的机会的。”这里所说的,也就是各党派可以长期共存的政治基础。共产党同各民主党派长期共存,这是我们的愿望,也是我们的方针。至于各民主党派是否能够长期存在下去,不是单由共产党一方面的愿望作决定,还要看各民主党派自己的表现,要看它们是否取得人民的信任。各党派互相监督的事实,也早已存在,就是各党派互相提意见,作批评。所谓互相监督,当然不是单方面的,共产党可以监督民主党派,民主党派也可以监督共产党。为什么要让民主党派监督共产党呢?这是因为一个党同一个人一样,耳边很需要听到不同的声音。大家知道,主要监督共产党的是劳动人民和党员群众。但是有了民主党派,对我们更为有益。当然,各民主党派和共产党相互之间所提的意见,所作的批评,也只有在合乎我们在前面所说的六条政治标准的情况下,才能够发挥互相监督的积极作用。因此,我们希望各民主党派都能注意思想改造,争取和共产党一道长期共存,互相监督,以适应新社会的需要。

九、关于少数人闹事问题

一九五六年,在个别地方发生了少数工人学生罢工罢课的事件。这些人闹事的直接的原因,是有一些物质上的要求没有得到满足;而这些要求,有些是应当和可能解决的,有些是不适当的和要求过高、一时还不能解决的。但是发生闹事的更重要的因素,还是领导上的官僚主义。这种官僚主义的错误,有一些是要由上级机关负责,不能全怪下面。闹事的另一个原因是对于工人、学生缺乏思想政治教育。一九五六年,还有少数合作社社员闹社的事件,主要原因也是领导上的官僚主义和对于群众缺乏教育。

应该承认:有些群众往往容易注意当前的、局部的、个人的利益,而不了解或者不很了解长远的、全国性的、集体的利益。不少青年人由于缺少政治经验和社会生活经验,不善于把旧中国和新中国加以比较,不容易深切了解我国人民曾经怎样经历千辛万苦的斗争才摆脱了帝国主义和国民党反动派的压迫,而建立一个美好的社会主义社会要经过怎样的长时间的艰苦劳动。因此,需要在群众中间经常进行生动的、切实的政治教育,并且应当经常把发生的困难向他们作真实的说明,和他们一起研究如何解决困难的办法。

我们是不赞成闹事的,因为人民内部的矛盾可以用“团结——批评——‘团结”的方法去解决,而闹事总会要造成一些损失,不利于社会主义事业的发展。我们相信,我国广大的人民群众是拥护社会主义的,他们很守纪律,很讲道理,决不无故闹事。但是这并不是说,在我国已经没有了发生群众闹事的可能性。在这个问题上,我们应当注意的是:(一)为了从根本上消灭发生闹事的原因,必须坚决地克服官僚主义,很好地加强思想政治教育,恰当地处理各种矛盾。只要做到这一条,一般地就不会发生闹事的问题。(二)如果由于我们的工作做得不好,闹了事,那就应当把闹事的群众引向正确的道路,利用闹事来作为改善工作、教育干部和群众的一种特殊手段,解决平日所没有解决的问题。应当在处理闹事的过程中,进行细致的工作,不要用简单的方法去处理,不要“草率收兵”。对于闹事的带头人物,除了那些违犯刑法的分子和现行反革命分子应当法办以外,不应当轻易开除。在我们这样大的国家里,有少数人闹事,并不值得大惊小怪,倒是足以帮助我们克服官僚主义。

在我们社会里,也有少数不顾公共利益、蛮不讲理、行凶犯法的人。他们可能利用和歪曲我们的方针,故意提出无理的要求来煽动群众,或者故意造谣生事,破坏社会的正常秩序。对于这种人,我们并不赞成放纵他们。相反,必须给予必要的法律的制裁。惩治这种人是社会广大群众的要求,不予惩治则是违反群众意愿的。

十、坏事能否变成好事?

如像我在上面讲过的,在我们的社会中,群众闹事是坏事,是我们所不赞成的。但是这种事件发生以后,又可以促使我们接受教训,克服官僚主义,教育干部和群众。从这一点上说来,坏事也可以转变成为好事。乱子有二重性。我们可以用这个观点去看待一切乱子。

匈牙利事件不是好事,这是大家清楚的。但是它也有二重性。由于匈牙利的同志们在事件的发展过程中间处理得正确,结果匈牙利事件由坏事转变成了一件好事。匈牙利现在比过去巩固了,社会主义阵营各国也都得了教训。

同样,一九五六年下半年发生的反共反人民的世界性的风潮,当然是坏事。但是它教育了和锻炼了各国共产党和工人阶级,这就变成好事。在许多国家里,有一批人在这个风潮里退出了党。一部分党员退党,党的人数减少了,当然是坏事。但是也有一方面的好处。那些动摇分子不愿意继续干下去了,退走了,大多数坚定的党员更好团结奋斗,为什么不好呢?

总之,我们必须学会全面地看问题,不但要看到事物的正面,也要看到它的反面。在一定的条件下,坏的东西可以引出好的结果,好的东西也可以引出坏的结果。老子在二千多年以前就说过:“祸兮福所倚,福兮祸所伏。”日本打到中国,日本人叫胜利。中国大片土地被侵占,中国人叫失败。但是在中国的失败里面包含着胜利,在日本的胜利里面包含着失败。历史难道不是这样证明了吗?

现在世界各国的人们都在谈论着会不会打第三次世界大战。对于这个问题,我们也要有精神准备,也要有分析。我们是坚持和平反对战争的。但是,如果帝国主义一定要发动战争,我们也不要害怕。我们对待这个问题的态度,同对待一切“乱子”的态度一样,第一条,反对;第二条,不怕。第一次世界大战以后,出了一个苏联,两亿人口。第二次世界大战以后,出了一个社会主义阵营,一共九亿人口。如果帝国主义者一定要发动第三次世界大战,可以断定,其结果必定又要有多少亿人口转到社会主义方面,帝国主义剩下的地盘就不多了,也有可能整个帝国主义制度全部崩溃。

矛盾着的对立的双方互相斗争的结果,无不在一定条件下互相转化。在这里,条件是重要的。没有一定的条件,斗争着的双方都不会转化。世界上最愿意改变自己地位的是无产阶级,其次是半无产阶级,因为一则全无所有,一则有也不多。现在美国操纵联合国的多数票和控制世界很多地方的局面只是暂时的,这个局面总有一天要起变化。中国的穷国地位和在国际上无权的地位也会起变化,穷国将变为富国,无权将变为有权——向相反的方向转化。在这里,决定的条件就是社会主义制度和人民团结一致的奋斗。

十一、关于节约

我想在这里谈一下节约的问题。我们要进行大规模的建设,但是我国还是一个很穷的国家,这是一个矛盾。全面地持久地厉行节约,就是解决这个矛盾的一个方法。

在一九五二年“三反”运动中,我们反对过贪污、浪费和官僚主义,而着重在反对贪污。一九五五年提倡过节约,重点是在非生产性的基本建设中反对了过高的标准,在工业生产中节约原料,成绩很大。那时,节约的方针还没有在国民经济各部门中认真地推行,也没有在一般机关、部队、学校、人民团体中认真地推行。今年要求在全国各方面提倡节约,反对浪费。我们对建设工作还缺乏经验。在过去几年有很大的成绩,同时也有浪费。我们必须逐步地建设一批规模大的现代化的企业以为骨干,没有这个骨干就不能使我国在几十年内变为现代化的工业强国。但是多数企业不应当这样做,应当更多地建立中小型企业,并且应当充分利用旧社会遗留下来的工业基础,力求节省,用较少的钱办较多的事。在去年十一月中共二中全会更着重地提出了厉行节约反对浪费的方针以后,几个月来已经开始发生效果。这一次节约运动必须彻底地持久地进行。反对浪费,同批判其它缺点错误一样,好比洗脸。人不是每天都要洗脸吗?中国共产党、民主党派、无党派民主人士、知识分子、工商业者、工人、农民、手工业者,总之,我们六亿人口都要实行增产节约,反对铺张浪费。这不但在经济上有重大意义,在政治上也有重大意义。在我们的许多工作人员中间,现在滋长着一种不愿意和群众同甘苦,喜欢计较个人名利的危险倾向,这是很不好的。我们在增产节约运动中要求精简机关,下放干部,使相当大的一批干部回到生产中去,就是克服这种危险倾向的一个方法。要使全体干部和全体人民经常想到我国是一个社会主义的大国,但又是一个经济落后的穷国,这是一个很大的矛盾。要使我国富强起来,需要几十年艰苦奋斗的时间,其中包括执行厉行节约、反对浪费这样一个勤俭建国的方针。

十二、中国工业化的道路

这里所讲的工业化道路的问题,主要是指重工业、轻工业和农业的发展关系问题。我国的经济建设是以重工业为中心,这一点必须肯定。但是同时必须充分注意发展农业和轻工业。

我国是一个大农业国,农村人口占全国人口的百分之八十以上,发展工业必须和发展农业同时并举,工业才有原料和市场,才有可能为建立强大的重工业积累较多的资金。大家知道,轻工业和农业有极密切的关系。没有农业,就没有轻工业。重工业要以农业为重要市场这一点,目前还没有使人们看得很清楚。但是随着农业的技术改革逐步发展,农业的日益现代化,为农业服务的机械、肥料、水利建设、电力建设、运输建设、民用燃料、民用建筑材料等等将日益增多,重工业以农业为重要市场的情况,将会易于为人们所理解。在第二个五年计划和第三个五年计划期间,如果我们的农业能够有更大的发展,使轻工业相应地有更多的发展,这对于整个国民经济会有好处。农业和轻工业发展了,重工业有了市场,有了资金,它就会更快地发展。这样,看起来工业化的速度似乎慢一些,但是实际上不会慢,或者反而可能快一些。经过三个五年计划,或者再多一些时间,我国的钢产量仍然可能由解放前最高年产量,即一九四三年的九十多万吨,发展到二千万吨,或者更多一点。这样,城乡人民都会感到高兴。

关于经济问题今天不准备多讲。经济建设我们还缺乏经验,因为才进行七年,还需要积累经验。对于革命我们开始也没有经验,翻过斤斗,取得了经验,然后才有全国的胜利。我们要求在取得经济建设方面的经验,比较取得革命经验的时间要缩短一些,同时不要花费那么高的代价。代价总是需要的,就是希望不要有革命时期所付的代价那么高。必须懂得,在这个问题上是存在着矛盾的,即社会主义社会经济发展的客观规律和我们主观认识之间的矛盾,这需要在实践中去解决。这个矛盾,也将表现为人同人之间的矛盾,即比较正确地反映客观规律的一些人同比较不正确地反映客观规律的一些人之间的矛盾,因此也是人民内部的矛盾。一切矛盾都是客观存在的,我们的任务在于尽可能正确地反映它和解决它。

为了使我国变为工业国,我们必须认真学习苏联的先进经验。苏联建设社会主义已经有四十年了,它的经验对于我们是十分宝贵的。大家看吧,谁给我们设计和装备了这么多的重要工厂呢?美国给我们没有?英国给我们没有?他们都不给。只有苏联肯这样做,因为它是社会主义国家,是我们的同盟国家。除了苏联以外,东欧一些兄弟国家也给了我们一些帮助。完全不错,一切国家的好经验我们都要学,不管是社会主义国家的,还是资本主义国家的,这一点是肯定的。但是主要的还是要学苏联。学习有两种态度。一种是教条主义的态度,不管我国情况,适用的和不适用的,一起搬来。这种态度不好。另一种态度,学习的时候用脑筋想一下,学那些和我国情况相适合的东西,即吸取对我们有益的经验,我们需要的是这样一种态度。

巩固同苏联的团结,巩固同一切社会主义国家的团结,这是我们的基本方针,基本利益所在。再就是亚非国家以及一切爱好和平的国家和人民,我们应当巩固和发展同他们的团结。有了这两种力量的团结,我们就不孤立了。至于帝国主义国家,我们也要团结那里的人民,并且争取同那些国家和平共处,做些生意,制止可能发生的战争,但是决不可以对他们怀抱一些不切实际的想法。

《毛泽东选集》第五卷,人民出版社 1977 年 4 月第 1 版,第 363-402 页

\newpage
\chapter{建国以来我国国防战略的四次重大调整 叶晖南 1999}

出处:当代中国史研究 出版日期:1999年3期

\newpage

国防战略方针是国家武装力量建设和使用的根本依据,是涉及军队和国防所有工作的总原则。我国建国以来的国防战略方针继承了毛泽东军事思想中积极防御的原则,并在不同的历史发展阶段赋予不同的内涵。从50年代起到现在,我国的国防战略方针曾有过四次大的调整,即50年代中期的“积极防御,防敌突袭”﹔60年代中期到80年代中期(实际执行到70年代末)的“立足于早打、大打、打核战争”﹔80年代中期的“应付和打赢局部战争”和1993年以来的应付和打赢“现代技术特别是高技术条件下的局部战争”。

\section{“积极防御,防敌突袭”(1956~1964年)}

1949年全国大陆解放后,人民解放军的任务也随之发生变化,保卫祖国成了国家武装力量的主要任务。朝鲜战争结束以后,新中国的周边态势出现了自鸦片战争以来少有的稳定。中国选择了“一边倒”的外交政策,成为当时世界上社会主义阵营的重要力量。同时,新中国首倡和平共处五项原则,积极开展和平外交,同南亚各国建立良好的外交关系,使我国南部边境保持大体的安宁。只有东南沿海,受美国支持的国民党集团还对我国大陆的安全构成一定的威胁。即便如此,安全还是有相对的保障,任何帝国主义国家开几艘军舰打上几炮就能让中国屈服的时代已经一去不复返了。

外敌明火执仗的大规模入侵的可能性虽然不大,但采取突然袭击的手段发动战争的威胁却是新中国领导人不得不考虑的问题。毛泽东在1955年召开的党的全国代表会议上指出:“今后帝国主义如果发动战争,很可能像第二次世界大战时期那样,进行突然的袭击。因此,我们在精神上和物质上都要有所准备,当着突然事变发生的时候,才不至于措手不及。”根据这个思想,中央军委在1956年3月召开扩大会议,对国家的军事战略进行专门的研讨。会上国防部长彭德怀作了《关于保卫祖国的战略方针和国防建设问题》的报告。这个报告首次明确了我国的国防战略方针,指出:为了有效地防止帝国主义的突然袭击,保卫人民革命和国家建设的成果,保卫国家的主权飞领土完整和安全,在未来的反侵略战争中,应该采取积极防御的战略方针。这次军委扩大会议的召开和彭德怀讲话的发表,标志着建国后的第一个国防战略的确立。

“积极防御,防敌突袭”的战略方针主要是针对当时以美国为首的帝国主义集团可能发动的突袭,为此,中国进行了必要的准备。这期间,党召开了第八次全国代表大会,制定了把党的工作中心转移到社会主义建设上来的正确路线,为此国防建设也要服从和配合党的中心工作。在如何处理国防建设和经济建设的矛盾问题上,毛泽东在《论十大关系》一文中认为:“可靠的办法就是把军政费用降到一个适当的比例,增加经济建设费用。只有经济建设发展得更快了,国防建设才能够有更大的发展。”对于投资巨大的高精尖武器的研发,他说:“这里也发生这么一个问题,你对原子弹是真正想要飞十分想要,还是只有几分想要,没有十分想呢?你是真正想要,十分想要,你就降低军政费用的比重,多搞经济建设。你不是真正想要,十分想要,你就还是按老章程办事。”

在这个思想指导下,国防建设在三个方面进行了不懈的努力:大规模地裁减军队员额和经费﹔大力加强海空军及技术兵种的实力﹔自力更生,有重点地开展军生产和科研。为此,军队在1955年和1958年两次大规模地裁减员额。与此同时,海空军及其他技术兵种的实力得到了加强,整个军队的战斗力有了明显的提高。解放初期,我军改善战备基本上依靠从苏联购买,第一个五年计划实施后,我国在苏联的帮助下新建和扩建了79个具有一定规模的兵工厂,开始仿制苏式装备。到50年代末,从各种轻武器到常规火炮、装甲战车、作战飞机、小型水面舰艇,我国都能自行生产了。原子弹和导弹的研制开发也取得了长足的进展。完整的国防科研和生产体系初步建立,我军武器装备的水平与世界先进水平的距离大跨度地缩小。

\section{“准备早打、大打、打核战争”(1964~1985年)}

我从50年代后期到60年代初,国的国家安全形势变得严峻起来,中苏两党两国悲剧性的分裂、中印边境的武装冲突、台湾海峡的紧张局势及美国在越南战争中把战火烧到了北方等都使中国周边的平静被打破。在这种环境下,中国的国防战略进行了第二次调整,从和平时期转入临战状态。

这个战略方针的转变始于1964年10月22日,毛泽东在一项指示中批道:必须立足于战争,从准备大打、早打出发,积极备战,立足于早打、大打、打原子战争。我们不仅要在战略部署、后方设施,作战准备和国防工业建设等方面充分注意这个问题。同时也要在国民经济建设等方面充分注意这个问题。毛泽东的这个批示意味着1956年军委扩大会议制定的国防战略将被新的国防战略取而代之。1969年4月召开的党的第九次全国代表大会的政治报告对这个方针正式加以确认:“决不可忽视美帝、苏修发动大规模战争的危险性。我们要做好充分准备,准备他们大打,准备他们早打,准备他们打常规战争,也准备他们打核大战。”

在准备打仗的指导思想下,从60年代起,国防建设采取了许多重大举措。备战整军,大规模进行三线建设,加速研制核武器、运载工具及成立“第二炮兵”部队、组建民兵师和生产建设兵团等等。

由于国家倾全力进行备战,投入了大量的人力物力,这一时期国防建设在某些方面取得了辉煌的成就。这主要是在战略武器方面的进展,如原子弹、氢弹的爆炸,人造卫星的发射及回收,核潜艇的下水等都标志着我国的国防建设取得了巨大的进步。但是,这个战略方针是建立在对形势判断失误基础上的,因而其负面影响亦显而易见。其实并非所有的人都认为战争已经到了一触即发的边缘,1969年,“九大”刚刚召开过后不久,毛泽东即交给陈毅、徐向前、聂荣臻、叶剑英等四位元帅一项任务,要他们一面到工厂“蹲点”,一面看看有关国际材料,每月讨论2-3次,对战争与和平问题作出判断。同年7月11日,四老帅在给中央的报告《对战争形势的初步估计》中,作出了与“九大”政治报告里对形势估计不同的结论,报告中直言不讳地说,“我们认为,在可以预见的时期内,美帝、苏修单独或联合发动大规模侵华战争的可能性都不大。”但这个正确的判断在当时未能引起决策者的高度重视,其后,国家改变了外交战略,推行“一条线”战略,联美抗苏,这样一来被两面夹击的被动局势虽然缓解了,但对苏关系依然紧张,“准备打仗”的弦一直难以松下来。而且这个战略方针实施的时间贯穿60年代中期到80年代中期整整20年,其中60年代中期到70年代中期最甚,军费开支的加大,军队员额的激增,加上“一切以战备的观点来衡量”的思维定式严重制约了国民经济的发展,造成了中国经济建设的落后局面。经济建设的落后反过采又影响到国防建设,虽然战略武器搞上去了,但常规武器又重新同世界先进水平拉大了距离。同时,国防建设本身也在“左”的思想指导下受到很大损失。

\section{“应付和打赢局部战争”(1985-1993年)}

“文化大革命”结束后,主要是党的十一届三中全会以后,党的指导思想发生了根本性的转变,以阶级斗争的观点看待一切的指导思想终被摈弃,党的工作中心转移到现代化建设的正确轨道上来。这就要求其他一切工作包括外交和国防都要围绕和配合中心工作,而其中首先就要在思想观念上有一番转变。例如怎样看待当前的国际形势,世界的大趋势是“革命与战争”的走向还是“和平与发展”的问题等。在过去相当长的一段时间内,我们的基本看法沿袭了一条公式:不是战争引起革命,就是革命制止战争。同时还认为新的世界战争的爆发是不可避免的。如果坚持这样一种认识,要贯彻党的十一届三中全会制定的中心工作转移就困难了,因此,60年代中期起影响我们的对国际形势判断的基本理论、看法和70年代形成的“一条线”的外交战略和准备打仗的国防战略就与十一届三中全会后党的指导思想相冲突了,再度调整不可避免。尽管这个调整相对经济的改革开放滞后了一段时间,但总还是一步一步在起着变化。

从80年代初起,我们不再坚持世界性的大战近在咫尺,转而认为战争虽然不可避免,但可以推迟。1980年,邓小平在《目前的形势和任务》一文表达了那一时期党在战争与和平问题上的认识。他指出:“八十年代无论对于国际国内,都是十分重要的年代。国际上很难预料会发生什么问题,但是可以说是非常动荡、充满危机的年代。当然,我们有信心,如果反霸权主义斗争搞得好,可以推迟战争的爆发,争取更长一点时间的和平。这是可能的,我们也正是这样努力的。”

1982-1985年间是我国外交战略实现根本转变的时期。1982年党的“十二”大召开,确立了独立自主的和平外交方针。此后中国在外交上逐渐远离冷战思维,用更加客观的目光来审视世界事务,制定对外方针。同年,邓小平在会见联合国秘书长德奎利亚尔时,明确地表达了中国人民强烈的和平愿望,他说:“中国希望至少二十年不打仗。我们面临发展和摆脱落后的任务。我们摆在第一位的任务是在本世纪末实现现代化的一个初步目标,……所以我们希望有一个和平的国际环境。一打仗,这个计划就吹了,只好拖延。从现在到本世纪末是一个阶段,再加上三十至五十年,就是说我们希望至少有五十到七十年的和平时间。”

到了1983年,我们在战争与和平问题的认识上又进了一步,同年3月,邓小平在视察了江苏等地回京后,对部分中央领导人讲了这样一段话:“现在的问题是要注意争取时间,该上的要上。大战打不起来,不要怕,不存在什么冒险的问题。以前总是担心打仗,每年总要说一次。现在看,担心得过分了。我看至少十年打不起来。”这番话说明在中国领导人的心目中战争威胁的阴影已经被驱散,争取时间,抓紧建设成为他们最关心的头等大事。

随着认识的深入,我们对于国际形势的判断日臻成熟,1985年6月,邓小平在军委扩大会议上的讲话,对这一问题作了准确的概括,他说:“粉碎'四人帮'以后,特别是党的十一届三中全会以后,我们对国际形势的判断有变化,对外关系也有变化,这是两个重要的转变。

第一个转变,是对战争与和平问题的认识。过去我们的观点一直是战争不可避免,而且迫在眉睫,……这几年,我们仔细地观察了形势,认为就打世界大战来说,只有两个超级大国有资格,一个苏联,一个美国,而这两个国家都还不敢打。……还要看到,世界新科技革命蓬勃发展,经济、科技在世界竞争中的地位日益突出,这种形势,无论美国、苏联、其他发达国家和发展中国家都不能不认真对待。由此得出结论,在较长时间内不发生大规模的世界战争是有可能的,维护世界和平是有希望的。根据对世界大势的这些分析,以及对我们周围环境的分析,我们改变了原来认为战争的危险很迫近的看法。

第二个转变,是我们的对外政策。过去有一段时间,针对苏联霸权主义的威胁,我们搞了'一条线'的战略,就是从日本到欧洲一直到美国这样的'一条线'。现在我们改变了这个战略,这是一个重大的转变。……现在树立我们是一个和平力量、制约战争力量的形象十分重要,我们实际上也要担当这个角色。”

邓小平的这“两个转变”是对党的十一届三中全会之后中国对国际形势的看法的正确总结。围绕着以经济建设为中心的党的基本路线,这次军委扩大会议确认了新时期的军事战略,即从长期以来立足于早打、大打、打核战争的临战状态,转变到和平时期以应付和打赢局部战争为主的方针。会议要求充分利用和平时期,在服从国家经济建设的前提下,抓紧时间,有计划,有步骤地进行现代化为中心的国防建设。以增强军队在现代化条件下的作战能力。

为了适应新的战略方针,80年代以后军队采取了一系列的大动作:军委主席邓小平亲自领导了百万大裁军的行动。。1982年9月根据中央军委的决定,军委炮兵、装甲兵、工程兵等技术兵种部裁减合并为总参谋部的兵种部,铁道兵部队集体转业到铁道部。在大量裁撤指挥机关的同时增加和恢复了总参四部及所属的电子对抗部队。1987年又组建了陆军航空兵部队,以适应现代化战争的需要。1982年,成立国防科工委,以加强国防现代化的科学保障体系。1985年后,撤消合并了大军区,由原来11个军区合并成7个。军区编制人员减少50%。1985年还撤消了多年沿用的野战军编制,组建机动性、突击力、防护力飞快速反应能力更强的集团军。在这场变革中全军共撤掉了31个军级单位,4054个师团单位。但在集团军编制内各技术兵种的分量则大大加强,使技术兵种在陆军的总数在历史上第一次超过步兵的总数。新时期的战略方针顺应了国际国内形势的变化与发展,最大限度地配合了我国改革开放后的经济建设大局和保障国家安全的需要。自觉服从和服务于大局,突出质量建军是这一战略方针的显著特点。

\section{“打赢现代技术特别是高技术条件下的局部战争”(1993年至今)}

90年代初,世界军情发生了两个显著的变化,一个是冷战结束,两大军事集团之一的华沙条约组织瓦解。另一个是以微电子和信息科学为代表的高科技在军事领域里的运用有了革命性的发展。第一个变化使得大规模的战争更加不可能,局部战争的发生几率更高了,第二个变化则导致一场正在发生的军事革命。以倾泻钢铁为特征的大工业时代的战争模式将成为过去,而以高科技及信息之间的角逐的军备竞争正方兴未艾。在这种趋势下,世界各国纷纷加紧军事变革,力求建立一支质高量少,合理够用的精干武装力量。

面对新的情况,以江泽民为核心的新一代中央军委及时对我国的国防战略方针进行了又一次的重大调整01993年初召开的军委扩大会议决定将我军的战略方针基点放在打赢现代技术特别是高技术条件下的局部战争,加速人民解放军的质量建设,提高应急作战能力。为贯彻新的战略方针,中央军委还提出了以“科技强军”为中心思想的“两个根本性转变”,即由应付一般条件下的局部战争向打赢现代技术特别是高技术条件下局部战争转变﹔在军队建设上,由数量规模型向质量效能型、人力密集型向科技密集型转变。

新的军事战略方针,强调了科学技术在军事领域运用的极端重要性,为赶上世界军事发展的潮头,同时也找到一条适合中国国情的国防发展道路指明了方向。沿着这个方向,我军建设又有了新的发展。

继80年代人民解放军百万大裁军之后,党的十五大又作出再裁军50万的重大决定。经过两次大规模的裁军,我军的总兵力由原来的400万降至250万左右。与此同时,我军的体制编制、国防科研及工业的管理体制、军队的教育训练制度等都在有条不紊地进行改革。成立中国人民解放军总装备部,是我军统帅机关体制的一个重要变革,此举证明我军已经把改善武器装备以适用未来高科技条件下的局部战争摆到了至关重要的位置。

高科技要求的是高投入,而国家建设又要求“军队要忍耐”,“过几年紧日子”。

针对这个矛盾,我国国防科研走了一条集中有限财力、物力在主要方向上突进的路子,以确保一旦有事,对付任何强敌都有“还手之力”。

新时期的国防战略方针是对毛泽东积极防御思想在新的历史条件下的发展,这一方针既着眼于当代世界军事发展的前沿,及时准确地把握住了时代的脉搏,又同我国现代化经济建设的整体利益珠联璧合,为国家安全提供了强有力的可靠保障。毫无疑义,这个国防战略方针是唯一正确的选择。

\newpage

\chapter{重温毛泽东的战略思想 张文木 2013}

出处:本文刊发于《政治经济学评论》2013年第4期。2013年10月第一次修订;2014年12月第二次修订。转自观察者网。

\newpage

每当面临新的历史性难题,大多数中国人都会想到毛泽东。20世纪40年代,周恩来同志曾向全党发出“学习毛泽东”的号召,他说:“毛主席坚持把马克思列宁主义的普遍真理具体化在中国土壤上,生长出来成为群众的力量,所以中国革命得到如此伟大的胜利。到今天,不仅中国共产党尊敬他,凡是得到革命胜利果实的人民,一定都会逐渐心悦诚服地信服他。” 事实上,即使在半个多世纪后的今天,学习毛泽东,重温和掌握毛泽东的战略思想,提高国家战略能力以应对新世纪的挑战,仍是摆在21世纪中国共产党人和全体中国人民面前的重要的任务。

一

毛泽东思想并不产生于书斋中的空想,而是产生于中国共产党解决中华民族面临的现实问题的实践中。

20世纪40年代,中国已被帝国主义国家分裂成五六块,这是中国最危险的时期,即使到了抗战胜利的1945年,中国仍面临前门驱虎,后门进狼,再次被分裂和瓜分的危险。

1944年10月9日,在欧洲战事接近结束时,丘吉尔来到莫斯科,与斯大林秘密达成了瓜分东欧的“百分比协定” 。11月19日即斯大林与丘吉尔达成那份秘密瓜分欧洲的“百分比协定”后的一个多月,斯大林便接见法国共产党中央总书记多列士,要求法国共产党放下武器,参加“盟国所承认的政府”,斯大林说:“保留武装力量的共产党的地位是软弱的,将来也会是软弱的。要维护这种地位是困难的。因此,必须把武装力量改组为另一种组织,一种政治组织,而把武器收藏起来。”

1945年年2月10日,也就是在太平洋战争即将结束的前夕,罗斯福和斯大林拉上极不情愿的丘吉尔在雅尔塔会议上背着中国达成了瓜分中国的“雅尔塔秘密协定”,并以出兵东北对日作战和战后承认国民党政权为条件迫使蒋介石于8月14日承认了这一协定。此后斯大林用对待法国共产党参加“盟国所承认的政府”的方式又要求中国共产党到重庆与国民政府谈判。重庆谈判后,中国共产党党内有些同志确实为斯大林的“威望”所吓倒,产生走法共道路的“和平民主”思潮。1946年2月1日,中央下发《中央关于目前形势与任务的指示》,认为“从此中国即走上了和平民主建设的新阶段”。关于今后军队与党的关系,文件指出:“我党即将参加政府,各党派亦将到解放区进行各种社会活动,以至参加解放区政权,我们的军队即将整编为正式国军及地方保安队、自卫队等。在整编后的军队中,政治委员、党的支部、党务委员会等即将取消,党将停止对于军队的直接指导(在几个月之后开始实行),不再向军队发出直接的指令,我党与军队的关系,将依照国民党与其军队的关系。”这份文件最后表示:“必须指出党内目前主要危险倾向,是一部分同志中的狭隘的关门主义。由于国民党的反动政策及18年的国共尖锐斗争,党内党外均有许多人不相信内战真能停止,和平真能实现,不相信蒋介石国民党在各方面逼迫下,也能实行民主改革,并能继续与我党合作建国,不相信和平民主新阶段已经到来,因而采取怀疑态度,对于许多工作不愿实行认真的转变,不愿用心学习非武装的群众的与议会的斗争形式。因此各地党委应详细解释目前的新形势与新任务,很好地克服这些偏向。这些党外人士比党员还要左,我们应当好好说服他们。由于整个政治形势的发展,中央相信这种偏向是不难克服的,但在以后一个时期,国内和平民主新阶段更加确定,并为广大群众看清之后,在国民党实行若干重大改革之后,右倾情绪即可能生长起来,并可能成为主要危险倾向,那时我们就要注意克服右倾情绪。但在今天则应注意克服一部分群众观点中的左倾关门主义。 当时,就连苏联大使都认为“没有可怕的危险”了,相信中共“应学习法国的经验,今后主要任务是争取群众。”

斯大林曾支持的南斯拉夫共产党领袖铁托于1948年与斯大林反目后,斯大林对中国共产党产生了深深的怀疑,在战后他更加将苏联利益作为绝对原则,对二战结束后出现的国际共产主义运动高潮则日益淡漠,如果再考虑到中共党内的“和平民主”思潮,这些都对正处于中国命运大决战关键时刻的毛泽东形成巨大压力。但毛泽东以中华民族的利益为最高利益,他坚决顶住了这种压力并说服党内纠正了“和平民主”的思想倾向,带领全党对国民党反动派展开了积极的思想和武装斗争。

在中共取得“三大战役”决定性的胜利后,美国用李宗仁换下有那么点“半独立性” 的蒋介石,李宗仁于1949年1月22日就任“代总统”,随后提出“划江而治”的方案,而在此三四个月前,即1948年8、9月间朝鲜半岛出现“和平民主”的东方“样板”:半岛南北正式和平裂分为两个国家。大概是受到朝鲜半岛形势的鼓舞,在中共取得决定性胜利并决定过江统一全中国的前夕,斯大林开始频繁暗示共产党军队不要南下过江。1948年12月斯大林转给毛泽东一封国民党政府请求苏联居中调停国共之争的请求信。其意不言自明,就是要求毛泽东接受苏联出面接续马歇尔的“调停”,此为毛泽东断然拒绝。1949年伊始,就在毛泽东准备一鼓作气打过长江的当口,1月10日、11日、14日、15日,斯大林反复发电报给毛泽东,建议与国民党继续和谈,建立和平,称“如果中国共产党直接拒绝与南京和谈,则向世人宣布它主张继续进行内战” 。至于斯大林的真实目的,1945年7月斯大林在与蒋经国的谈话中表得很明白,他告诉蒋经国:“非要把外蒙古拿过来不可!我不把你当作一个外交人员来谈话,我可以告诉你:条约是靠不住的。再则,你还有一个错误,你说,中国没有力量侵略俄国,今天可以讲这话,但是只要你们中国能够统一,比任何国家的进步都要快。”

对此,毛泽东洞若观火。1948年12月雷洁琼先生曾随团受邀来到西柏坡,她问毛泽东怎样看待“划江而治”? 她回忆说:

毛主席笑了,笑声很大,很鼓励人。毛主席笑着说,美国和苏联立场虽然不同,但在这个问题上都是站在他们各自的利益上给我们增加压力,用军事实力政治实力形成了一种国际国内舆论,一种暂时性表面化的社会基础。这就是从表面上看、暂时性看问题,不顾一切代价追求“和平”,而不管这种和平能不能长久。决定国家大事,应该从国家和人民的长远利益根本利益考虑问题。为了一个统一的新中国,我们中国共产党必须透过现象看本质,放弃暂时抓长远,将革命进行到底。如果不是这样,搞什么划江而治,将后患无穷。在中国历史上每一次分裂,再次统一都要很长时间,人民会付出好多倍的代价!事关举国长远大计,我们共产党一定要站在人民的立场,看得远一点,不受其他国家的影响。

此时的毛泽东,以中国人民的利益为最高利益,没有听从斯大林的“劝阻”,决然过江,并于1948年12月30日发表《将革命进行到底》予以回应。1949年4月21日即国民党政府拒绝中共代表团提交的《国内和平协定最后修正案》后的第二天,毛泽东与朱德共同发出《向全国进军的命令》,号令全军坚决、彻底、干净、全部地歼灭中国境内的一切敢于抵抗的国民党反动派,解放全中国。中国人民解放军百万大军在东起江苏江阴、西至江西湖口的1000余里的战线上渡过长江。4月20日,解放军首先遇到英国舰只“紫石英号”的挑衅。人民解放军予以坚决打击,“紫石英号”被重创后逃出长江口 。4月26日,丘吉尔在英国下院以老牌海权大国的傲慢,要求英国政府派两艘航空母舰“实行武力报复”。艾德礼也在当天表示:英国有权开动军舰进入中国的长江 。4月30日,毛泽东为中国人民解放军总部发言人起草关于“英国军舰暴行”的声明说:“长江是中国的内河,你们英国人有什么权利将军舰开进来?没有这种权利。中国的领土主权,中国人民必须保卫,绝对不允许外国政府来侵犯。”

反观同一时期的印度,在朝鲜半岛分裂的前一年,印度在“和平民主”中已分裂为两个国家。1947年8月14、15日,印度次大陆正式分裂为巴基斯坦和印度两国。受着西式“民主”熏陶的尼赫鲁从英国人手中接收的是只能以英人的意志行事的“总理”虚位,他就任总理后既无力进行所有制变革,也无力进行社会革命,这是因为尼赫鲁组阁时手中——与毛泽东领导的中国革命不同——没有一支听命于印度国大党的武装力量。面对英国人分裂印度的“建议”,尼赫鲁更是一筹莫展,只能说些无奈的空话,他说:“去接受一种分裂的原则,或者不如说是去接受一种不带强迫统一印度的原则,可能会使人们对于它的后果加以冷静而沉着的考虑;而这一来,就会认识到统一是对各方面都有益的。”

比较同期的中国,在中国有一支听命于中国共产党的强大的人民军队,而尼赫鲁领导的印度国大党却信奉什么“非暴力不合作”,如此不要说军队,就是必要的财力也不足 。尼赫鲁手头既无钱也无枪,有的尽是会搞议会政治的干部。眼看着印度的分裂,他们无可奈何。反观同期的中国共产党,由于有了一支听命于党的军队,党才得以与国际国内分裂势力进行有力的斗争并于两年后实现中国统一。

看了这段历史,我们就会对毛泽东说的“没有一个人民的军队,便没有人民的一切”的论断及以毛泽东在古田会议确立的“党指挥枪”的原则的深远意义,有更深切的体会。毛泽东根据中国革命的经验说:“对于这个问题,切不可只发空论。”

大国军事的关键不在一个“大”字,而在于谁领导,用什么思想武装。与一般的军队不同,用毛泽东思想武装的中国人民解放军,不仅仅是一支能打硬战的武装力量,更重要的它还是党和国家实现其政治目标的武装力量。

有什么样的领袖,国家便会有什么样的命运:今天在朝鲜半岛仍是南北炮口相向,在印度原来的版土上有着两颗互为目标的原子弹,而在中国则是统一使用管理的核武器、盘上世界屋脊的青藏铁路,以及飞入太空的“神舟”和潜入深海的“蛟龙”。看到新中国建设的巨大成就,多年后李宗仁也对自己当年坚持与共产党划江而治的行为作了深深的悔罪,他说

“如果美国人全力支持我,使我得以沿长江和毛泽东划分中国,中国就会陷入象今天的朝鲜、德国、老挝和越南同样悲惨的局面了。南部政府靠美国生存,而北部政府也只能仰苏联鼻息,除各树一帜,互相残杀外,二者都无法求得真正之独立。又因中国是六亿人的大国,这样一来,她就会陷于比前面提到过的三个小国家更为深重的痛苦之中,而民族所受的创伤则恐怕几代人也无法治好了。如果这种事情真的发生了,在我们敬爱的祖国的未来历史上,我会成为什么样的罪人呢?”

“殷鉴不远,在夏后之世。” 奥地利在19世纪曾是雄视大半个欧洲的奥地利帝国(1804~1867)和奥匈帝国(1867~1918)的国都,其首相梅特涅在拿破仑失败后的维也纳会议上还是决定欧洲政治的关键人物。可它到20世纪却成了在地图上需要用放大镜才能找到的袖珍小国。公元800年,查理大帝将群雄纷争的欧洲归于一统,其历史贡献类似中国的秦始皇,可在不到半个世纪的时间里,统一的欧洲就为查理大帝的三个“崽卖爷田不心疼”的孙子于843年竟用一纸“凡尔登条约” 一分为三,这不仅奠定了后来意、法、德三国的雏形,而且在欧洲大陆地缘政治中深埋了极难修复的破碎性的根基,这反过来为欧洲绵延千年之久的混战及为地处欧洲大陆边缘的不列颠岛国最终成长为世界大国提供了天然的地缘政治条件。面对同样的事件,中国人就幸运得多。公元前403年,周天子威烈王正式分封韩、赵、魏为诸候,由此导致国家分裂,战国纷争。好在此种乱局于公元前221年为秦王嬴政定为一统,中国由此有了在亚洲迄今不能撼动的主体性大国地位。 20世纪日本人很重视英国利用欧洲地缘政治的破碎性操纵欧洲大陆的经验,并于40年代将中国分割成类似欧洲那样的四分五裂的局面。有幸的是,毛泽东领导的中国共产党带领中国人民在1949年再次实现了国家统一。

今天回首当时那段的历史,我们真得要感谢毛泽东同志及他领导的中国共产党。正是有了毛泽东同志的担纲,我们今天才有统一的中国和在东亚洲居有主体地位的中国大版图。这对世界,尤其是对亚洲政治稳定所产生的意义是巨大的。它使亚洲版图的碎化程度开始出现从边缘向中心地区(即中国)逐级大幅递减趋势,这样的版图分布特征符合原子结构 及其稳定的力学原理。这样的地区版块结构有利于以中国为中心和主体的东亚合力的形成。为此毛泽东曾说:“中国是亚洲的重心” 。与此相反,欧洲大陆国家分布普遍处于对称型破碎状态,其对称破碎化程度则由边缘向中心逐渐上升,这使欧洲大陆的地缘政治从中心地带便出现了过于细碎因而极难修复的根基。这样的地缘政治形势对欧洲的历史产生了负面影响是如此之大,以至两次世界大战的导火索都出现在欧洲。如果我们知道这些,那就会知道毛泽东统一中国对于中国及亚洲的意义,就不能不为毛泽东那一代国家领导人的远大眼光所折服,并对维护中国的国家统一抱有铁血决心。

二

毛泽东反手打天下,也反手治天下。 1949年10月建国,次年初毛泽东就西东开弓,出兵西藏,年末进军朝鲜。由此为新中国打下了至今不可动摇的国基。

印度1947年独立,美国、印度的一些人开始策动西藏脱离中国,同年中国还在内战,无力西顾。尼赫鲁是西洋秀才,花架子,压根就不知道毛泽东的厉害,想乘中国的乱局,造成西藏与中国分离的形势。谁知毛泽东于1950年初便挥师西进,一步到位,等尼赫鲁反应过来,西藏已在中国手里了。毛泽东此举意义重大,西藏使中国有了广阔的战略纵深,这也使我们在西部地区减少了很多边防驻军。如果1948年尼赫鲁先下手将西藏控制在印度支持的“藏独”分子手中,并将“西藏问题”国际化,看看今天的台湾,我们就不难理解今天的国家的西部安全将会遇到什么样的麻烦,至少航天、核试基地就在人家的就近监控之下,杨利伟的太空之行也就困难多了。其次,和平解放西藏,雅鲁藏布江——印度的布拉马普拉河的上游地带——就在我们主权之内,这样在水资源分配上才有我们今天相当主动的地位。这些,假若当时按照现在某些人的“布哈林式”的眼光,认为1949年底共产党刚执政,就应该集中精力发展经济,追求GDP,那我们中国今天就不会有这样好的地缘政治位势,更不会有今天这样的经济成就。

一波未平,又波又起。1950年下半年,美国军事介入东北亚朝鲜半岛,逼着毛泽东表态。按今天一些人的想法,中国根本就不应当出兵抗美援朝,应先以发展经济为中心。但当时中国就那么一点重工业,东北是振兴中国的基地,相当于今天的上海。面对美国的咄咄逼势,毛泽东主动打出去,虽然把美国给得罪了,却换来了苏联支持。这样就打出了一个相对有利的地缘政治格局,同时也保护了东北的安全。这些在建国第二年就决定并完成的惊天伟业,并非“布哈林式”的账房先生,更不是那些天真的书生们所能想得到和做得到的。这只能是我们的毛泽东及其战友们所能想得到、做得到的。

朝鲜战场上的胜利,使中国在国际上的地位大为上升。苏联加大了对中国的援助,并东南亚国家获得了极大的敬意。1955年万隆会议上,周恩来在会上受到英雄般的欢迎,欢迎并不为周恩来个人,而是为我们新中国打赢了美国。我们当时一穷二白,硬是将战争打赢了,这使中国在第三世界中威信很高,欧洲人也佩服新中国的领导人。东南亚国家曾受西方人和日本人的殖民压迫,也希望中国强大。中国强了就有号召力。中国人在万隆会议上受到欢迎,美国人不高兴,但亚洲人高兴。今天我们的外交要学习毛泽东以斗争求和平的外交艺术。

有人说,中国入朝参与是由于“莫斯科施加的压力”,被斯大林拖下水的结果 ,而毛泽东决定抗美援朝的动机“更多的成分是出于意识形态,而不是站在现实基础上的考虑” 。事实并不是这样。

与李自成初取天下时面临的形势相似,1949年10月1日新中国建立,1950年6月东北亚便燃起了战火——6月27日美国出兵朝鲜。遥望东北亚正在升起的战云,毛泽东不可能不想到李自成大顺政权因痛失关外而甫立即亡的教训。在李自成的“种种的错误” 中,造成“后来失败的大漏洞” 因而最具颠覆性的错误恐怕就是从战略上忽视“关外问题”对于新政权生死存亡的迫切关联性。郭沫若批评说,李自成入主北京城后因小事丢失山海关,是“实在是太不通政略” 。毛泽东注意到这一历史教训。1949年初,从西柏坡准备进京的毛泽东风趣地将此行比喻为“进京赶考:毛泽东在回答周恩来“我们应当都能考试及格,不要退回来”的话时说:“退回去就失败了。我们决不当李自成,我们都希望考个好成绩。”

事实上,早在朝鲜战争爆发甚至在1949年新中国建立之前,毛泽东就注意到东北亚和平问题。1949年5月14日,金日成特使金一曾拜会毛泽东和朱德,双方谈到并评估了朝鲜北方对南方的军事行动。当时毛泽东对金日成说“朝鲜随时可能发生军事行动”,建议金日成“应估计到这种情况,并做好周密准备”。毛泽东还帮助金日成分析其对南方采取军事行动的几种可能结果,说:“在朝鲜的战争可能是速决的,也可能是持久的。持久战对你们不利,因为这样日本就可能卷进来,并帮助南朝鲜‘政府’”;毛泽东话锋一转说:“你们不用担心,因为有苏联在旁边,有我们在东北。必要时,我们可以给你们悄悄地派去中国士兵。都是黑头发,谁也分不清。” 显然,当时朝鲜形势的稳定与否,是新政权能否避免李自成甲申年哺立即亡的悲剧的关键因素,也是毛泽东“我们决不当李自成”这句话所包含的重要内容。

尽管历史进入20世纪,但“关外问题”——这时已转变为东北亚问题——仍对中国政治稳定具有生死的意义。1937年3月,毛泽东在《祭黄帝陵文》中说:“琉台不守,三韩 为墟。” 这是说琉球、台湾和朝鲜半岛的齿唇依存的关系;但1895年日本在“甲午海战”后窃取中国台湾、1910年全面占领朝鲜、1937年发动全面侵华战争的诸事件所展示的连贯逻辑同样表明:“三韩”不保,中原为墟。朝鲜半岛是中国——当然也是俄国——东方安全的重要屏障:此门洞开,且不论由此可能造成的中国东北动乱及其对中国工农业经济的影响,仅从地缘政治上及近现代历史经验看,更会直接威胁中国京畿重地,并对中国的整体稳定造成重大冲击。事实上,郭沫若提出的“关外问题”并非始于明代,自隋朝始,它就日益成为中国政治稳定的“软肋”:隋之后中国历史上多次出现全国性的长期战乱,其爆发源头多出自关外,在这一地区任何动荡都会很快传递到北京政治中枢,如果中枢对此反应无力或失败,接踵而至的就是中央政权退至中国长江一线,其结果要么是国家分裂,要么是政权更迭。二者必居其一。隋炀帝和唐太宗都曾意识到但无力消除酝酿于东北关外的乱源,至明时东北亚已成为中国政治风暴持续发作的风口。

如果熟悉自隋之后的中国历史,就会明白毛泽东回答周恩来“退回去就失败了。我们决不当李自成”这句话所包含的历史经验。恰恰是基于这样的“现实基础上的考虑”而并非如某些人认为的“更多的成分是出于意识形态”的原因,在新中国成立的第二年毛泽东即作出“抗美援朝”的决定。此举彻底杜绝了新中国重蹈李自成因忽视或失控于“关外问题”而功败垂成的任何可能。1950年6月美国出兵东北亚;在此,值得我们留意学习的是:毛泽东一改隋、唐远征为援助朝鲜的方式,借苏联的支持并在金日成的邀请下出兵抗美援朝,一举将“关外问题”远远推到“三八线”之外。中国东北从而华北由此稳定至今。

毛泽东同志对中国历史的洞察及在此基础上作出的“抗美援朝”的正确决策使新中国避免了李自成政权悲剧,但这并不意味着今后的中国可以永远远离这一悲剧。2013年7月11日,习近平同志来到西柏坡说:“60多年过去了,我们取得了巨大进步,中国人民站起来了,富起来了,但我们面临的挑战和问题依然严峻复杂,应该说,党面临的‘赶考’远未结束。” 这些“挑战和问题”——比如曾被隋炀帝、唐太宗均意识到却无力解决,明末乃至民国甲申年均遭遇到,最终由毛泽东圆满解决的东北亚问题——将会不断出现并需要我们回应,在这方面,我们还要学习毛泽东同志,从毛泽东战略思想中汲取营养并向人民交出正确的答卷。

有人说,中国出兵朝鲜却丢了台湾 。事实并不是这样。

中国人民志愿军1950年10月25日才入朝作战,而在6月27日美国就宣布封锁台湾海峡。当时斯大林告诉中国,朝鲜战场上打不败美国,中国“甚至连台湾也得不到” 。事实也是这样,如果中国当时示弱,那今天的台海形势就会更糟。看看同时期蒋介石讨好美国的结果,就知道美国人历来都是“柿子捡软的捏”。

也有人说,中国参战是被苏联人利用了 ,并认为“这种‘胜利’在很大程度上只是心理上的” ,事实并不是这样。

我们知道,在1945年“雅尔塔秘密协定”中,苏联拥有在大连的“优先权益”和在旅顺驻军权和中长铁路的使用权。这些都是苏联在远东的战略利益,这些利益又为蒋介石国民党政府所承认。1950年中苏友好同盟条约中,苏联也是极不情愿地表示要废止这项秘密条约,但主张保留形式。中国革命的胜利发展从根本上改变了远东政治的格局,也迫使苏联重新调整对华政策。随之而来的朝鲜战争及中国军人打败美国人的结果,让最具现实主义政治眼光的斯大林也看明白:中国人在朝鲜战场上驱逐了美国人的同时,也驱逐了在中国东北的苏联人。结果苏联于1953年继而1955年初将中东铁路及旅顺港交还中国。至此,苏联在雅尔塔条约中已吞到嘴里并为蒋介石政府承认了的战略利益又悉数吐出,这对在二战中所向披靡且大获红利的斯大林来说不能不说是一种说不出的痛楚:一场战争下来,中国军队由弱变强,中国在苏联的支援下组建了强大的中国空军并将美国赶到三八线以南,而苏联却告别了自彼得大帝起就追求、1905年俄国人在此与日本人流血争夺而斯大林又在第二次世界大战后刚刚获得的大连和旅顺这两个进入太平洋的最便捷的不冻港。

毛泽东的军事艺术在于张合有度和恰到好处。中国及时在三八线停火,而没有接受斯大林打过三八线的要求,中国军队因此没有透支。现在回头看,如果中国接受斯大林的建议,拒绝停战打过三八线,那中国必然要透支国力,结果反而对苏联,尤其是对驻扎在旅顺的苏联海军会形成严重依赖,这反易受制于苏联。果真如此,后来的大连旅顺移交可能就不会那样不折不扣。1953年7月中美停战,当年1月苏联移交中东铁路。3月5日,斯大林病逝。斯人已逝,在中国问题上却是甘苦自知。

不仅如此,这样的后果使独立不久的外蒙古通往海参崴和辽东半岛出海口的关键陆上通道从苏联控制区转入已为中国完全控制的中国东北境内。这不仅将“雅尔塔秘密协定”对中国安全造成的负面影响压至最小,大大缓解了外蒙古独立对新中国安全造成的压力,而且还使外蒙古由此与中国产生了天然的依存关系。这对蒙古的发展和中蒙两国的未来关系定位有着如何估计都不会过高的意义。

高超的战略思想与毛泽东博大的胸怀和长远的历史眼光是紧密相联的。尽管毛泽东对斯大林在中国革命进程中的一些作法也有怨气,斯大林也有对不住中国革命的错误,但毛泽东仍能准确地把握斯大林的历史地位,认为他为中国革命做出的贡献仍是第一位的。斯大林曾在战争期间说过:“我知道,在我死后会有人向我的坟墓抛垃圾的。但历史之风会残酷无情地将它吹掉!” 斯大林去世后,就在其坟墓上堆其越来越多的“垃圾”的同时,中国却涌动起“历史之风”:就在苏联政府摘下斯大林画像并在全苏联“声讨”斯大林的时候,斯大林的画像还是始终与马克思、恩格斯、列宁并列在中国北京天安门广场。“斯大林如果地下有知,在这种情况下肯定会对毛泽东感激万分的”。 中国学者袁南生认为:“实实在在地说,死后的斯大林最大的、真正的知己是毛泽东。” 笔者深以为然。

三

毛泽东说“我们决不当李自成”,这并不意味着毛泽东拒绝李自成的成功经验。比较李自成流动作战和毛泽东红军长征的路线,我们会发现二者取得天下的共同点是他们都注意对中国地缘政治规律的研究:在低潮时均没有选择入川实行诸葛亮式的偏安,而是主动折师北上进入贴近中原的陕南商洛山和陕北高原蛰伏并由此再度崛起,随后便长驱直入北京。

三百年间与此相反的人物是张献忠和张国焘。张献忠与李自成分离后转战中原,于1640年和1644年两度步诸葛亮后尘由江淮西向入川并立都成都作偏安选择,张献忠本人连同他的政权于1646年被清军剿灭 。由此后推三个世纪,毛泽东与张国焘于1935年6月也发生过南下还是北上的争论:当时毛泽东指出张国焘的南下方案“事实上会使一、四两方面军被逼退到西康地区……如果我们被敌人封锁在这个地区,将成为瓮中之鳖” ;而张国焘则反唇相讥说“我看蒋与川敌间矛盾极多,南打又为真正进攻,决不会做瓮中之鳖” 。事后张国焘归队及中国革命从陕北成功的经验已使毛张这场争论的是非有了明确的结论;同样的理由,当年李自成入川后迅速北上折入商洛山的决策要远比张献忠入川作诸葛亮式的盘踞高明。今天再读并比较这两段历史,可以进一步补充的结论则是,如果当年张国焘真与中央分裂并入川实行割据,其结局决不会比三百年前同期的张献忠更好。

今天回头看,长征路上毛泽东与张国焘关于北上还是南下的争论并以红军北上为结果,这实在是符合“天道”即符合中国地缘政治和中国革命规律的伟大决择——用司马迁的话概括就是“非必险固便形势利也,盖若天所助也。” 当年中国共产党的胜利,有“天命”即顺应历史规律的成份,也有“人谋”的因素。在前者中,司马迁揭示说“夫作事者必于东南,收功实者常于西北。” 在后者中,首功当推毛泽东。鉴于对中国西南地缘政治有更为深刻历史洞察,毛泽东在长征路上弃南而就北,将中国革命带入高潮。

说到大西南,让人想起1962年对印自卫反击战。1962年的中国形势真是“高天滚滚寒流急”:中国国内刚刚经历了严重的自然灾害,蒋介石在东南准备反攻大陆,苏联在北方制造新疆居民“外逃”事件,印度借机在中印边境燃起战火。印度事关西南稳定,但毛泽东总体上还是认为与美国不同,印度是朋友,不能真打。1962年10月中国对印自卫反击战高调开启,一月后却又轻轻落下:一仗下来,只是将印度的屁股打响了些,但没有让它伤筋动骨。在战争规模上,毛泽东巧妙利用了古巴导弹危机,将它控制在有限范围内,不让第三国卷入。10月20日,美国封锁古巴海域,苏美剑拔弩张。当天中国全线反击。11月20日,肯尼迪宣布最后结束封锁,11月21日,苏联也对军队解除了动员令。当天中国也对印全面停火。一月后,中国又将印度俘虏养得黑胖黑胖,把印军的枪擦得干干净净交还给他们。战争结果与朝鲜战争不同,朝鲜战争中中国军队硬是将美国硬逼到三八线以南,而1962年的对印自卫反击战则没有将印度硬逼到“麦线”以南。在前者,毛泽东意在打出国格和平等,在后者,毛泽东意在西南方向打出持久和平,至于中印两国间的领土纠纷,毛泽东则留将来从长计议。研究一下毛泽东1962年西南一役,其目的不是打赢,而是为了中国大西南赢得一个长期的和平环境。

今天总结这场战争,我们看到毛泽东的军事艺术大张大合,但张合有度,其战略战术兼融三国时曹操进退汉中和诸葛亮七擒孟获的古典智慧:退和进战于瞬间,毕诸葛“七擒”之功于一役,让人体会出了毛泽东军事艺术所呈现出的那种“进而不可御者,冲其虚也;退而不可追者,速而不可及也。故我欲战,敌虽高垒深沟,不得不与我战者,攻其所必救也;我不欲战,虽画地而守之,敌不能与我战者,乘其所之也。” 的极高境界。每读史及此,令人不由击节赞叹并为之神往。反观毛泽东出手过的东北和西南,至今均无战事。

说到这些,我们真要感谢毛泽东及其战友们,感谢毛泽东同志及在他领导下的中国人民解放军。

四

20世纪60年代初,中国外交进入“雪压冬云白絮飞”的困难时期,但同时也是中国历史将要发生重要转折的前期。在美国打压中国的同时,中国北方盟友的表现更让人忧虑。1959年初,苏共“二十一大”召开,赫鲁晓夫宣称把世界战争排除在社会生活之外的现实可能性业已产生。6月,苏联政府单方面撕毁了中苏于1957年签订的国防新技术协定。9月,艾森豪威尔与赫鲁晓夫举行会谈,以牺牲中国利益为代价,形成所谓“戴维营精神”,推销苏美合作共同主宰世界的方针——这很像今天一些人醉心并亲切地称之为“G2”(还有人更亲切地称之为“chinamerica”,如果音意合译就是“亲美国”)的琼阁梦幻。9月15日 美苏举行“戴维营会议”,随后赫鲁晓夫就来到北京劝说中国“不要用武力去试探资本主义制度的稳定性” ,10月,毛泽东作诗讽刺正沉醉于“G2”共治的赫鲁晓夫说:“西海如今出圣人,涂脂抹粉上豪门”;“列宁火焰成灰烬,人类从此入大同”。 1963年8月5日,苏、美、英三国在莫斯科签订了《禁止在大气层、外层空间和水下进行核武器试验条约》,这是“G2共治”的第一个重大结果,明眼人一看就知道这是针对当时已经拥有成熟核技术的中国和刚成功进行了核试爆的法国的。针对这个条约,毛泽东讽刺并愤怒地说:“不见前年秋月朗,订了三家条约。还有吃的,土豆烧熟了,再加牛肉。不须放屁,试看天地翻覆。” 有人说毛泽东这首词用字不雅,这反说明,当时毛泽东对赫鲁晓夫“涂脂抹粉上豪门”的外交政策的愤怒已臻极点。

赫鲁晓夫的对华政策对正处经济困难中的中国来说更是雪上加霜。1959年3月19日,与台湾国民党准备“反攻大陆计划”东西呼应,中国西藏发生武装叛乱,达赖喇嘛随后逃往印度。4月27日,印度总理尼赫鲁在人民院就西藏局势发表讲话,鼓吹召开新德里、北京、拉萨三方的所谓“圆桌会议”。9月9日苏联塔斯社发表一篇关于中国和印度边界武装冲突的声明,公开偏袒印度一方,并随后给印度15亿卢布的贷款。9月30日至10月2日,赫鲁晓夫访问北京,指责中国共产党,干涉中国内部事务。赫鲁晓夫希望中国配合他设想的“G2”共治的大局,中国不从,两党两国关系由此恶化。1960年7月始,苏联不断在中苏边界寻衅。1961年,正值中国经济最困难的时期,苏联要求中国本息一起偿还抗美援朝时苏联援华军事物资的贷款。1962年4、5月间,苏联当局通过其驻中国新疆的机构和人员,在伊犁、塔城地区引诱和胁迫数万名中国公民流入苏联境内。10月,印度军队又从西南方面对中国领土发动大规模全线进攻,中国被迫进行自卫反击。此后中印关系全面恶化。1963年起,苏联大量增兵中苏边境,对中国北疆形成新的军事压力。如果再考虑到东南方面蒋介石也利用中国内政外交的困难积极准备其“反攻大陆的计划”,中国东南、西南、北方三面安全骤然形成共振性恶化形势。 美国学者费正清在书中说:“在北京看来,在1962年夏天融汇成了一种互相配合的威胁。” 如果再考虑到1959年后中国国内还出现三年自然灾害,以及1964年8月“北部湾事件”后,美国大规模轰炸越南北方,战火向中国边境蔓延的形势,当时中国真是遇到了“雪压冬云白絮飞、万花纷谢一时稀” 的艰难处境。而当时毛泽东的心情却是愈挫愈奋,他在诗中说“独有英雄驱虎豹,更无豪杰怕熊罴。梅花欢喜漫天雪,冻死苍蝇未足奇。”

苏美对中国压力的层层加码终于有了毛泽东的明确回应:1964年10月16日,中国第一颗原子弹试爆成功。毛泽东告诉大家:“在今天的世界上,我们要不受人家欺负,就不能没有这个东西。” 1964年10月11日,赫鲁晓夫下台。但苏联对中国施加压力却是有增无减,勃列日涅夫在中苏边境和中蒙边境驻军激增近百万,这对中国北方安全形成重大压力。对此,毛泽东给予更坚决的回应:1969年9月23、29日,中国成功进行了地下核试验和氢弹予以回应。

毛泽东明白,国际关系中的“朋友”的含义,就是打不败的对手。毛泽东面对国际霸权主义的坚决斗争终于迎来了不利于苏联却有利于中国的国际大变局。1972年2月21日,尼克松对中国进行了为期一周的访问,与毛泽东在瞬间握手言和。2月28日,中美双方在上海发表《联合公报》。1973年2月,美国国务卿基辛格再次访华,5月,中美双方分别在对方首都设立联络处。与此相配合的是美国费城爱乐乐团首次在北京演出。

1972年尼克松访问中国,此前他最担心的是毛主席不接见,他在工作日记中写道:“我们应该很快同毛会见,并且我们不能陷入这样的境地,即当我会见他时他高高在上,好比我走上阶梯而他却站在阶梯的顶端。” 当听到周总理要接见时,尼克松仅带了基辛格和温斯顿?洛德来到毛主席的书房,学着周总理称毛泽东为“主席”。据基辛格回忆:当尼克松列举了一系列需要共同关注的国家时,毛泽东说:这些问题可同周总理谈,我们谈“哲学问题” 。哲学问题当然就是方向问题,这些会谈为未来中美关系发展确定了方向。谈话结束时,尼克松握着毛泽东的手说:“我们在一起可以改变世界。”毛泽东则举重若轻地回答:“我就不送你了。”

毛泽东就是这样,在国家外交处于最困难的时期,以斗争求和平,敢于斗争,也善于斗争,在对手的敬畏中主导着历史的方向,同时也为十年后的中国改革开放布下了伏笔。

五

在中国,土地问题解决的好坏是赢得人民支持的关键。我们知道,后人常将秦国能够“奋六世之余烈,振长策而御宇内,吞二周而亡诸侯” 解释为秦之“严刑峻法”,这其实说不通。因为在刑罚的残酷性上其他六国并不比秦国逊色,从某种意义上说,刑罚的残酷性往往与国家获得人民的支持程度为反比存在。与其他六国政策比较,商鞅建立的军功与土地奖励相联系的产权制度是秦王朝获得人民(在当时主要是农民)支持的关键因素。这个以土地奖励耕战的制度使支持秦王朝的社会基础扩大到最底层的广大农民,这些人在其他六国,只能从贵族手中而非国家手中获得或租得土地。这样,与将其支持力量建立在贵族基础上而“严刑峻法”并不逊于秦的其他六国比较,秦国由此获得的社会基础就显得广大得多。如果没有这样广大的社会基础,那么,仅靠“严刑峻法”,只能更快地加速秦朝的灭亡。事实正是这样。商鞅为秦王朝建立了小农所有制,这使秦朝获得比其他六国更广大的社会基础从而统一了中国;同样,小农经济所具有的先天分散且易瓦解的脆弱性又使建立其上的王朝——如果没有新的土地资源的扩充的话——就难以长期维持。这不仅是造成秦王朝,甚至是从古代直到现代中国的蒋家王朝灭亡的主要规律性原因。

毛泽东同志注意到并成功运用了这个规律。20世纪初中国农业及小土地农民大面积解体,毛泽东同志制定了正确的土地政策,将土地问题与中国革命前途联系起来,中国共产党由此获得巨大的革命资源并建立了新中国,随后又迅速将小农经济归并到社会主义集体经济,避免了历史上必然出现的小农在获得土地后即迅速两极分化的恶果并由此建立了以“工农联盟”为基础的新中国。

在此,对我们最有学习意义的,是建国后毛泽东同志在全国范围开始的生产资料改造工程并依此团结和组织全国人民所选择的时机。

1950年6月25日,朝鲜内战爆发,6月26日美国总统下令美国远东地区的部队支援韩国军队作战。6月27日杜鲁门宣布他已命令第七舰队进驻台湾海峡。9月15日,美军在朝鲜西海岸仁川登陆,9月28日,美军占领汉城。与此同期,美国联合整个西方国家对新中国战略物资“禁运” 也全面升级。

面对这样的压力,新中国需要更为坚固的社会支持力量。毛泽东同志首先从所有制而不象蒋介石那样四处“发饷”(接近今天的“发红包”)着手在国内组织共产党政权的支持力量。就在美国占领汉城的当天(1950年9月28日),中国政府宣布《中华人民共和国土地改革法》 ,在全国范围内开展土改运动。新中国政府依靠贫农、雇农,团结中农,中立富农,有步骤、有分别地消灭封建剥削制度,并在改造旧制度,用人民的力量肃清国内敌对分子的同时,也团结了全国农民和发展了农业生产力。到1952年9月,也就是在朝鲜战争接近尾声,中朝两国人民取得了决定性胜利的时候,中国全国90%以上农业人口获得约7亿亩土地,使农民免除了3000万吨粮食的地租,在新中国最困难的时候,中国共产党获得了中国主体人口即农民的政治支持。1951年5月,毛泽东对周世钊说:

我们志愿军武器远不如美帝,但常常把美帝打得狼狈逃窜。这是为什么呢?没有别的理由,这是因为我们的志愿军都是翻身的农民和工人,他们认识这个战争是为保家卫国而战。可以说,我们这回抗美援朝的战争是打品质仗,是什么武器也不易抵挡的。

一个美国的记者说,美国的军队再花20年也打不到鸭绿江。我看再打200年,他们也没有希望打到鸭绿江。

就在中国抗美援朝战争即将取得胜利的前夕,毛泽东预见到西方不会甘心在朝鲜战场上的失败,必将对新中国经济实施更大的封锁和压力,而应对这样的困难需要更为广泛的人民基础。1953年6月15日,毛泽东又在中央政治局扩大会议上进一步提出“党在社会主义时期的总路线和总任务”,及时对新民主主义时期的所有制进行社会主义的改造。到1956年中国在所有制方面基本完成了对农业、手工业和资本主义工商业的社会主义改造。这为1956年之后的社会主义建设提供了基本没有产权交易支出的经济制度:生产资料完全掌握在国家所有制和集体所有制手中,国民收入不经私有产权之间的交易而纯粹进入再生产领域,这既解放了中国国内的生产力,又避免了同期印度出现的因保留私有制而造成的产权支出过大、生产资金严重短缺,及由此产生的国家建设对国际金融的绝对依赖。 毛泽东同志领导中国共产党用社会主义制度团结和组织全国人民,将西方在同时期实施的对华经济禁运的外部压力转化为强大的社会主义建设的动力。

反者道之动。今天回头看,如果当时美国不对中国进行全面的经济封锁,并让中国提前“融入全球化”、与西方“接轨”,那中国的所有制改造工程就可能不那么彻底,用于国家建设的资金就会被庞杂的产权交易大量截流和耗掉,从而使中国在建国之初就面临同期印度同样面临的GDP增长与两极分化及由此产生的对海外金融绝对依赖同步扩大的困局 。果真如此,那对正处起步阶段的中国的国民经济体系基本建设而言,无疑是比战争更为严重的国家灾难:新中国的支持力量就容易溃散,有组织的人民就会由此转化为无组织的流民,共产党建立的新中国及其社会支持基础就会由此解体,如此一来,中国就将再次陷入历史上那周而复始的社会甫稳即乱的周期律之中。

这样说决不是事后危言,曾在印度已有相当殖民统治经验的英国人最早看到这一点。1949年8月,他们就为美国提出在外部遏制的同时,从内部以“商业关系”和平瓦解新中国红色政权的策略,当月19日英国外交部的《备忘录》指出:

外国商业团体构成了西方在华影响的主要部分之一。我们认为,在亚洲的铁幕后面尽可能长久地保持最大限度的西方的触角和影响,是极为重要的事情。我们尤其要记住,事实将会证明,中共政权最力所不及的任务之一可能就是严密地管辖和控制在单个的中国人中深深扎根的经商爱好,只要利用中国人的经商天分,损害共产主义事业的希望依然存在,彻底放弃我们的在华地位便至少可以说为时过早。

要保证中国在国际斗争中立于不败之地,就需要占国内百分之八九十的人民而不是少数“精英”的支持。但人民,一定是有组织的人群,能够组织起人民并形成支持国家的力量的并不主要来自GDP等物质条件,也不主要来自简单的“惠民”施舍,而是实实在在的公共占有的生产资料所有制。反之,劳动人民一旦失去生产资料公有制和由此产生的人民生产主导的市场条件,其身份则立即转变为流民。流民人数的增长是历史上社会动荡乃至国家衰落的根本原因。例如沙俄时期经济高速发展并未能使其赢得第一次世界大战,而苏联的经济进步却使其赢得第二次世界大战的胜利并由此成为联合国的主要创始国。西方反共老手丘吉尔于1959年12月21日在斯大林诞辰80周年时也无不感慨地说斯大林“他接过俄国时,俄国只有木犁,他撒手人寰时,俄国已经拥有核武器。” 曾长期留学苏联的蒋经国对此看得明白,1945年他对斯大林说:“苏联在对德战争中取得胜利的主要原因,是没有私有制。”

现在回头看来,毛泽东那一代共产党人选择社会主义所有制——而不是以其他“言不及义” 的“普惠”政策——为突破口团结和组织人民,将外部压力转化为人民支持国家动力的治国经验,对于今天中国政治家而言,需要认真领会和学习。

六

毛泽东对新中国社会主义建设的重要贡献是将社会主义所有制改造的成果及建立其上的中国发展与工农联盟而非以往的资本联盟相联系。

社会主义所有制的本质是保证而不是剥夺,更不是否定由人民掌握着的生产资料的所有制。目前在中国是集体所有制和国家所有制,这是中国工农联盟的基础。只有发展、壮大和不断巩固这种所有制形式,才能使城乡间的人口形成双向即自由来回的而不是今天这样为资本驱使的主要流向城市的单向流动。只有农民工的生活在城乡之间来回都有可靠即制度性的保障时,社会才能稳定,城市商品住房经营才能成为兼顾资本与消费者利益平衡的即社会主义的商品经营。而能保证进城务工人员自愿回流农村的因素,在现阶段不仅仅是家庭土地承包权,而是保证农民土地权利的农村集体所有制和保障城市工人权利的国家所有制。没有社会主义公有制,就不会有稳定的工农联盟,从而就没有社会主义国家的政治基础和政治稳定。早在1959年底毛泽东就注意到这一点并指出解决问题的方法,他在阅读《苏联政治经济学教科书》时批注道:

在社会主义工业化过程中,随着农业机械化的发展,农业人口会减少。如果让减少下来的农业人口,都拥到城市里来,使城市人口过分膨胀,那就不好。从现在起,我们就要注意这个问题。要防止这一点,就要使农村的生活水平和城市的生活水平大致一样,或者还好一些。有了公社,这个问题就可能得到解决。每个公社将来都要有经济中心,要按照统一计划,大办工业,使农民就地成为工人。公社要有高等学校,培养自己所需要的高级知识分子。做到了这一些,农村的人口就不会再向城市盲目流动。

值得注意的是,今天中国农村经济已有使普通劳动者与土地生产资料永久分离的危险 ,有些地方官员连同一些学者,为了一点“房地产”或某些资本集团的利益无视劳动者的长远利益,用所谓“城市户口”“城市房产权”,在没有充分就业保障的前提下,使进城农民与生产资料所有权从而与集体所有制相分离 。更有报纸发表推波助澜的“调查文章”,说“郊区农民不想种地盼拆迁致富” 。这样舆论引导的后果显然是危险的,因为这些流入城市的人口如不能在城市获得稳定的就业保障而又在农村“无立锥之地”的话,当年他们养不起耕地的困境就会迅速转化为养不起用地权换来的房权的困境。在这样的情况下,他们今天卖掉房子的速度比当年卖掉土地(经营权)的速度要快得多。

与失去土地相比,没有或失去住房的人群对社会稳定会形成更直接的破坏力,而目前中国城市中的天价商品房反过来又会使已涌入城市却又不能再回到农村的“市民”退为城乡地带的流民。而流民历来就是社会大动乱的温床。历史往往有惊人的相似之处。一百多年前,恩格斯同样面临并研究过这个问题。他在1887年1月10日为《论住宅问题》一书第二版写的序言中说:

当一个古老的文明国家这样从工场手工业和小生产向大工业过渡,并且这个过渡还由于情况极其顺利而加速的时期,多半也就是“住宅短缺”的时期。一方面,大批农村工人突然被吸引到发展为工业中心的大城市里来;另一方面,这些老城市的布局已经不适合新的大工业的条件和与此相应的交通;街道在加宽,新的街道在开辟,铁路铺到市里。正当工人成群涌入城市的时候,工人住宅却在大批拆除。于是就突然出现了工人以及以工人为主顾的小商人和小手工业者的住宅缺乏现象。在一开始就作为工业中心而产生的城市中,这种住宅缺乏现象几乎不存在。

恩格斯接着指出解决这个问题的出路在于建立起无产阶级国家政权后消灭城乡差别。现在我们已经建立了社会主义国家,但我们仍处于社会主义初级阶段,城乡差别还将长期存在。由此产生于资本主义条件下的一些负面因素,如果控制不好也同样会产生对社会主义国家不利的后果。那么,上述住宅问题的不利后果是什么呢?恩格斯以德国为例说:

农村家庭工业和工场手工业被机器和工厂生产所消灭,在德国就意味着千百万农村生产者的生计被断绝,几乎一半德国小农被剥夺,不只是家庭工业转化为工厂生产,而且农民经济转化为资本主义的大农业,小地产转化为领主的大农场——也就是意味着一场牺牲农民而有利于资本和大地产的工业和农业革命。如果德国注定连这个变革也要在旧的社会条件下完成,那末这样的变革毫无疑问会成为一个转折点。如果那时其他任何一国的工人阶级都还没有首先发动,那么德国一定会开始攻击,而形成“光荣战斗军”的农民子弟一定会给予英勇援助。

这样,资产阶级的和小资产阶级的空想——给每个工人一幢归他所有的小屋子,从而以半封建的方式把他束缚在他的资本家那里——现在就变成完全另一个样子了。实现这种空想,就是把一切农村房主变成工业的家庭工人,结束那些被卷入“社会旋涡”的小农的旧日的闭塞状态以及由此产生的政治上极其低下的状况;就是使工业革命推广到农业地区,从而把居民中最不活动最保守的阶级变成革命的苗圃,这一切的结果,就是从事家庭工业的农民被机器剥夺,被机器强制地推上起义的道路。

若再结合中国目前的日益严峻的“住宅短缺”问题来看,我们便会认识到,恩格斯所指出的现象是现代国家——不管其性质如何——在社会转型中很难避免的。现在需要我们考虑的是如何避免越来越多的被住房短缺抛弃的流民转化为“光荣战斗军”,“从事家庭工业的农民被机器剥夺,被机器强制地推上起义的道路”的历史恶果发生在当代中国。

当时恩格斯提出根本解决这一问题的方案是“消灭城乡对立” 。目前看来,资本主义国家和社会主义国家在相当的时间内都做不到这一点。但资本主义国家却用转移危机的方式将本国内部的“城乡对立”转变为外部世界的“南北对立”。以南北世界日益深刻的对立缓和了本国城乡对立及由此引发的日益严重的阶级对立。但这条道路对后发国家,尤其是后来的社会主义国家来说已不可重复。对于当代中国而言,我们只有依靠社会主义的制度优势——而不是什么“社区花园”、慈善式的“社区服务”和言不及义的“改革”——来解决我们面临的“住宅短缺”及由此可能引发的政治稳定问题。

改革在任何国家都是一种有阶级属性的行为,言不及义的“改革”是要不得的。毛泽东曾批评这样的政策是“言不及义,好行小惠,难矣哉” 。中国的改革要有适合中国国情的标准,这就是社会主义制度的标准,人民的标准。我们知道改革能解放生产力,但革命也能解放生产力。如果改革将中国生产力改到需要革命来进一步解放,那中国的改革就失去了历史进步作用。中国改革的底线就是不能把共产党改到人民的对立面;国企改革,不能改出“二七大罢工”。

中国共产党执政的基础不能基于资本财团——这是蒋介石走过且失败的道路,而应该基于工农联盟——这是以毛泽东同志为核心的党的领导集体实践已经证明并获得巨大成就的道路。工农联盟的基础是国家和集体所有制,这是社会主义所有制的基础部分。没有它,人民就不能保住手中的生产资料,而失去了生产资料,我们的人民就会转化为如毛泽东在《湖南农民运动考察报告》一文中形容的“上无片瓦,下无插针之地” 的贫民和流民,这样党的执政基础也就名存实亡。而建立在流民基础上的国家政权,就会象目前中东一些国家,一遇外来压力即刻崩溃。

历史反复表明,国家政权的政治生命周期的长短及其相应的抗压能力的强弱,与其所依靠的社会基础的大小为正比,而社会基础的大小又与其所依赖的所有制形式所容纳和解放的劳动力的广泛程度为正比。1927年中国共产党与中国国民党之间的战略能力的差距,是这一观点的有力证明。此前,国共合作开展北伐,实现中国统一,大得人心,这时蒋介石手头几乎有无限的人才和人力资源可供调配,北伐战场上也是捷报频传。1927年始,他向工农开刀,转靠买办封建势力,这便失去工农支持,以至在1948年国共两党进行大决战的关键时刻,国民党靠“发红包”和“抓壮丁”补充军事编制,但重赏之下已无勇夫;蒋介石方面已是“巧妇难为无米之炊”,而共产党方面则有源源不断且自觉参加的人力资源可随时投入战场。这为毛泽东的战略方针的顺利实施和共产党在全国战场取得胜利提供了充分的物质条件。同样,也是由于我们用社会主义制度而不是别的什么制度团结了全国人民,才使新中国冲破国际国内的重重恶浪,取得一个又一个胜利。

七

战略是刀尖上的哲学。哲学关乎方向,没有方向的战略只能沦为玄学;刀剑是实现战略的利器,只有正确方向而没有实施战略的利器,这样的战略绝无实现的可能。国家战略不能只是请客吃饭和绘画绣花。做秀只能在极次要的问题上产生效果,但在核心利益上若再玩这些“花活”则必败无疑。战略,尤其是国家战略的实施是一定是要带刀子的,刀子是用于解决敌我矛盾的工具。当前的中国战略研究一定要认真学习毛泽东同志写的《矛盾论》 。只有弄清了谁是我们的敌人,才能知道谁是我们的朋友;而没有敌人即满眼都是“战略伙伴”的战略则一定会遇到缺少盟友的尴尬。共产党早期的战略只有共产主义纲领,而没有明确和具体的敌我判断,结果在蒋介石“四一二”屠杀中近乎全军覆没。此后共产党人认识到战略真得不能绘画绣花,战略最终是要刺刀见红的。关于此,马克思说得好:

人的思维是否具有客观的真理性,这并不是一个理论的问题,而是一个实践的问题。人应该在实践中证明自己思维的真理性,及自己思维的现实性和力量,亦即自己思维的此岸性。关于离开实践的思维是否具有现实性的争论,是一个纯粹经院哲学的问题。

“政策和策略是党的生命。”而“只有党的政策和策略全部走上正轨,中国革命才有胜利的可能。” 在这方面要认真学习毛泽东同志写的《实践论》 。从相当的意义上说,战略是确定方向的学问,而策略是寻找战略边界的学问。事物的性质是由其对立的方面规定的,因而只有找到合理边界的战略才是有意义和可实施的。

但是,这些的道理对处于巅峰时期的国家来说,则容易被忽视。就在美国即将成为世界霸主的1943年,美国战略思想家沃尔特?李普曼(Walter Lippmann) 在《美国外交政策》一书中对美国人的“世界主义”情绪提出了警告。他写道:“美国必须在它的目的和力量之间保持平衡,使它的宗旨在它的手段可以到达的范围之内,也使它的手段可以达成它的宗旨;使它的负担和它的力量相称,也使它的力量足够来完成它的‘责任’:要是不确立起这个原则,那根本就谈不到什么外交政策。” 1947年,李普曼针对乔治?凯南(George F. Kennan ) 的“遏制”政策和以此为基础的“杜鲁门主义”的危险出版了《冷战》一书指出了美国安全的脆弱性,他提醒美国政府不要忘记在“目的和力量之间保持平衡”。

李普曼的旨在为美国卸除“杜鲁门主义所加于我们的负担” 的看法对20世纪50年代处于巅峰期的美国政治家们来说已难以接受,他们宁愿听信丘吉尔意在骄纵美国的“铁幕演说”,偏爱英国历史学家阿诺德?J.汤因比(Arnold J.Toynbee)让美国担当世界民主领袖的“倡议”,采纳英国人喜欢的约瑟夫?麦卡锡(Joseph R. McCarthy) 和凯南等不冷静的激进建议,挥师冲向世界,这使美国在20世纪70年代陷入全面危机,国力大幅衰落。

与美国相反,李普曼的文章却在中国受到毛泽东的长期关注。据统计,从1949年到1972年间,新华社电讯稿中提到李普曼的约有350篇,其中全文转载李普曼观点的稿件就有百篇之多,引用比较集中的时间段是1956年至1958年。 1958年11月12日,李普曼在《纽约先驱论坛报》上发表《苏联的挑战》一文,认为西方的军事集团和基地包围政策不能遏止共产主义的发展。新华社《参考资料》第2512期刊载了这篇文章,毛泽东读了李普曼的这篇文章后,写下批语:“此件印发。值得一看。”

乔治?凯南(George F. Kennan )的“遏制”战略拖垮美国的原因,是他为美国设计战略边界过于庞大,结果让美国老虎吃天,耗尽了力气。20世纪80年代,已入暮年的凯南对自己曾经提出的“遏制”战略更是后悔不迭,他在一次讲演中说:

这种军事化不仅对我们的外交政策,而且对我们的整个社会都有严重影响。它造成国民经济的畸型发展,这一点我和许多人都越看越清楚了。每年我们都把国民收入的很大一部分用于生产并出口武器装备,保持庞大的武装力量和设施。这么搞的结果对我国的经济生产实力不会有好处,只不过使我们每年都不能把成百亿美元用作生产投资。这些年来,我们已被迫使自己习惯于这种情况。这个习惯已经达到我曾大胆称之为真正民族乖癖的程度。我们现在已经不可能在不出现严重后遗症的情况下把它甩掉。除了数以百万计的穿军服的人以外,还有成百万的人们已经习惯于从庞大的军事工业体系中谋得生计。数以千计的企业靠军工维持,更不用说那些工会和社区了。军工已经成了使我国经济极其不稳定的那些预算赤字的根源。在军需品的生产者和销售者与华盛顿购买者之间已经建立起复杂而极其有害的联系。换一句话说,由于我们在和平时期维持庞大的军事机构并向其他国家出售大批军火,成千上万的既得利益者业已形成,也就是说,我们在冷战中造成一个庞大的既得利益集团。我们已经使自己依赖于这种可憎的行径。而且如今我们对它的依赖程度已经很深,以致可以毫无偏见地说:假如没有俄国人和他们那莫须有的邪恶作为我们黩武有理的根据,我们还会想出另一些敌手来代替他们。

尼克松是纠正凯南战略失误的政治家。1972年他在前往中国的飞机上说,我要去跟毛泽东谈哲学。他说的“哲学”就是两个国家的国力边界及其合作的边界。毛泽东与尼克松这两个有哲学的政治家一握手,这个世界就变了。事实上,毛泽东注意到乔治?凯南(George F. Kennan ) 为美国设计的战略边界过于庞大的“遏制”战略是拖垮美国的原因。1972年年初,在尼克松访华后不久,毛泽东在一个批示中告诫全党:“深挖洞,广积粮,不称霸。” 毛泽东意在警示未来中国不要重犯美国扩张目标与国家资源不匹配导致国家衰落的错误。

学习毛泽东,不仅要学习毛泽东同志坚定正确的政治立场,更要学习毛泽东成熟的政治素养,它不仅包括制定战略的能力,而且还包括为实现战略而具备的制定政策和策略能力。

1972年有两件值得研究的小事。第一件是当年毛泽东对到访的尼克松说“我喜欢右派” ,第二件是毛泽东要求到中央工作不久的王洪文阅读《后汉书》中的《刘盆子传》 。毛泽东意在警示王洪文,政治问题并不是靠你登高一呼就可以解决的。你没有沙场历练,如果再不向老同志学习、多长进,结果会像刘盆子那样即使身居高位,成为历史上的匆匆过客。如果将毛泽东两次谈话内容联系起来,可知毛泽东当时的忧心所在。随尼克松来访的基辛格巧妙道出了毛泽东的担忧,他说:“美国的左派只能夸夸其谈的事,右派却能做到。” 1973年3月10日,中共中央作出《关于恢复邓小平同志的党组织生活和国务院副总理的职务的决定》。

基辛格说的问题,在当时既存在于美国和苏联,也存在于中国。王明的“左倾”空谈和赫鲁晓夫的右倾机会主义给中国和苏联的社会主义事业带来了巨大灾难 ,这让毛泽东在晚年对当时那些只知空喊的“左”派——这些人用斯大林的比喻就是“属于那些没有经验的人,或者像飞蛾投火的共青团员” ——的治国能力深感担忧,他担忧未来的中国会陷入北宋那种靠“诵文书,习程课”就能入仕、或苏联那种靠赫鲁晓夫式的机会主义就能晋升领导高层的干部制度。值得体会的是,就在赫鲁晓夫下台的1964年,毛泽东将培养无产阶级事业接班人问题提上议事日程。他对党的事业接班人的条件除了立场可靠外,更加强调政治经验的成熟。5月15日,毛泽东在北京举行的中央工作会议上说:“无产阶级的革命接班人总是要在大风大浪中成长的。” 此前两个多月,作为干部制度改革的配套措施,毛泽东总结宋明以来的亡国教训说“烦琐哲学总是要灭亡的”;他要求“教育革命”,他希望在新的教育制度中学生不会脱离实际,不会“成为书呆子,成为教条主义者,修正主义者” ,其目的就是要防止远如大宋近如苏联的悲剧在中国重演。1905年对日战争失败后,俄国各地流传着一则笑话,说当时俄国人在远东对付日本人用的是圣像,而日本人回敬俄国人的却是子弹。 毛泽东当时最担心的是中国高层出现那种手中只有“圣像”而没有“子弹”、更无使用“子弹”的沙场经验,或出现像赫鲁晓夫和戈尔巴乔夫那种只知“卫星上天”而不知“红旗落地”的人物。

八

1980年,中国改革开放的伟大历史进程即将启航,在这个决定中国发展方向的关键时刻,邓小平同志特别告诫我们:“毛泽东思想这个旗帜丢不得。”“我们能够取得现在这样的成就,都是同中国共产党的领导、同毛泽东同志的领导分不开的。恰恰在这个问题上,我们的许多青年缺乏了解。”

事实上,作为我们的对手,尤其是有份量的对手,一刻也没有忘记毛泽东并在认真地研究毛泽东的战略思想。1972年基辛格随尼克松见到毛泽东时说:“我在哈佛大学教书时,指定我的学生要读主席的选集。” 时任美国总统福特在当天的唁电中说:“在任何时代成为历史伟人的人是很少的。毛主席是其中的一位。他的领导是几十年来改造中国的决定性因素,他的著作给人类文化留下了深刻的印记。他的确是我们时代的一位杰出人物。” 美国人说,他们不怕中国军事现代化,就怕中国军人毛泽东思想化。

历史表明,中国人民解放军在建立新中国的历史进程中战果辉煌,在中华民族自立于世界民族之林的不懈奋斗中立下了丰功伟绩,这一切应当归功于毛泽东的战略思想。没有共产党就没有新中国,但没有毛泽东思想,就不会有中国人民解放军。毛泽东思想是中国人民解放军的真正灵魂和战无不胜的力量源泉。

这些从另一个侧面说明,毛泽东思想是当代中国自立于民族之林的具有基础意义的思想资源,是中国人民解放军的真正灵魂和战无不胜的力量源泉。毛泽东的战略思想贯通传统与现代并实现了二者之间的完美结合,在应对当前复杂的国际斗争形势中,是我们需要结合新的实践深入学习和运用的思想精华。

\newpage

\chapter{孙向晨}

出处: https://www.rujiazg.com/author/250

\newpage

\section{民族国家、文明国家与天下意识 2014}

来源:《探索与争鸣》2014年第9期

时间:2014年9月24日

一 现代“民族国家”观念的确立及其局限性

17世纪以来,西方逐渐形成了现代的国家理论,其中最为突出的特征是以个体权利为基础的国家理论。霍布斯将传统的自然法与自然权利相区别,法的力量在于约束人,而权利的概念在于伸张人的自由。政治权力莫不来自于每一个人的“自然权利”。基于权利的转让,形成了“主权”概念,以及由此而来的国家理论。

这种现代主权国家的理论较之传统理论的区别在于,政治权力不再来源于人类之上的神灵,或是来自自然的秩序,而是非常明确地界定为来源于每一个人的自然权利。在这方面,霍布斯、洛克、卢梭都给出了相当完备的论述。但在这种论述中出现了另外的问题,那就是国家认同问题。在霍布斯的论述中,个体对于国家的认同和贡献都是非常薄弱的,甚至只能通过一种交换,即国家保护个体,个体奉献国家来加以解说,这成了霍布斯的难题。卢梭第一个较为系统地论述了这个问题,也就是说,卢梭要解答:当我们把国家的基础定位在每一个个体身上时,个体与整体的关系究竟该如何确立?在霍布斯所论述的“契约”关系之外,卢梭认识到了国家的整体意志问题,也就是“公意”问题,以及个体如何依从于整体的意志。在卢梭对“人民”的具体论述中,已经展现了“民族”的意味,一种具有强烈自我认同意识的政治群体。

从历史上看,传统的政治世界常常以帝国方式存在于世。比如古代世界的雅典帝国、波斯帝国、罗马帝国,中世纪是神圣罗马帝国、阿拉伯帝国,再晚一点的如奥匈帝国。神圣罗马帝国之后分裂为德意志、法兰西和意大利,这是现代民族国家的雏形,1648年《威斯特伐利亚和约》更是确立了现代民族国家体系的雏形。阿拉伯帝国分裂了,奥斯曼土耳其帝国分裂了,奥匈帝国分裂了,传统帝国在近代大都以民族形式分裂为各个“民族国家”。在各民族国家分立的态势下,最终在力量上达成均势,由此建立了现代的、民族国家的世界体系。

我们看到,主权国家的理论与民族问题并不是一开始就结合在一起的,这些概念有不同的来源,以后逐渐结合起来成为了现代国家的支柱性概念。“民族国家”成为应对现代国家建立之后如何进行自我认同的主要措施,个体本位和民族认同共同铸就了现代国家,表现为现代的民主政治和民族主义。现代民主政治很好地顺应了个体本位的政治需要,而民族主义则顺应了均质化的个体之间所需要的凝聚力。在西方封建时代,是以等级制为基础,贵族彼此的认同要远甚于国家的认同。现代“民族”概念成功地提供了一种基于平等地位,消弭内部差异的认同文化,这与现代性强调的个体本位文化非常匹配,于是“民族”的概念把前现代基于等级观念的人在新的个体文化环境中重新凝聚起来,民族主义对于霍布斯的难题是一种解毒剂,通过强调个体对民族国家的忠诚和奉献来予以解决,从而成为现代国家的核心凝聚力。

基于个体本位,现代世界提供了一种均质化文明的前景,但在现实中,以西方为例,人民依然是以族群的方式生存的,因此民族主义对外在均质化的个体中建立了族群在文化上的差别认同,对内则建立起均质化的文化,这为现代民主政治的展开创造了条件。人类整体的均质化文化或者说“大同文化”一直是乌托邦思想家的梦想,但在现实中真正行之有效的却是差异化的民族主义意识形态。它为一个国家提供了某种均质化的文化环境,一旦文化上形成巨大差异,民主政治反而会助长民族的分裂主义倾向,即便在西方发达国家,西班牙的加泰罗尼亚、英国的苏格兰、加拿大的魁北克等地区都有强烈的分裂主义倾向。现代社会的理论由于是建立在个体本位的基础上,社会分殊大,离心力大,因此一个社会如果没有足够强有力的文化凝聚力,社会就会分崩离析,国家就会分裂。源自西方的世界体系基本上就是靠民族主义来建国的,尽管它现在号称已经走向了后民族国家。

这套基于民族主义的国家体系事实上并不是中国文化传统所熟悉的。“民族”的概念在汉语中具有高度的歧义性。其中既有“Nation”的意思,又有“ethnic”的意思,甚至隐隐地还有“race”的意思。Nation就其本意而言有出生地的涵义,与地域有很大的关系,但现代“nation”概念有很强的人为性,具有双重涵意:文化的和政治的。“文化的”意味着一种历史上积累下来的语言、宗教、习俗和生活方式等,而“政治的”意味着强烈的建立主权国家的诉求,民族国家(nation-state)的概念意味着建立起了民族与国家的一致性。而“ethnic”更多的是在人类学意义上具有原生性差异的族群。但无论是基于nation,还是基于ethnic的“民族”,从来不是一个中国人熟悉的概念,基于强烈血统关系的民族主义,甚至是种族主义的变体,这更是中国传统所拒斥的。尽管民族或族群的差异有其人类学的基础,在中国历史上,这从来没有成为组织国家的政治原则。“中华”或者“华夏”这些自我认同的观念更多的是一种文明教化的概念,一种文明的归属,而不是族群的归属。中国古代确实讲夷夏之辨,但更强调“夷狄而中国,则中国之;中国而夷狄,则夷狄之”,丝毫没有褊狭的种族意思,更没有强烈的排外情绪,看重的是道德教化和文明程度。这是一种非常珍贵的传统。

传统的“天下”朝贡体系,只是天朝和藩属的关系,这是一种离文明中心远近的关系。中国在近代被西方列强一次次打晕之后,才仿佛知道世间有了“国家”这回事,以至于梁启超悲叹道,中国人只知有朝廷,不知有国家,当然更不知道“民族国家”是什么概念了。这是一个痛苦的转折过程,通过一代代思想家们的努力,终于以中华文明为基础建构了现代“中华民族”这个概念,这是现代中国得以立国的基础,是中国作为现代国家的“民族认同”。

从某种意义上讲,中国是被迫接受“民族国家”观念的,是被迫进入“民族国家”的世界体系。近代化过程由于受到西方强势文化的压迫以及船坚炮利的入侵,中国传统所固有的“天下”观念,在近代逐渐被压缩成一个“民族国家”概念,一种古老的文明被人为地制造成一个民族的概念,这在梁启超等人的著作中看得尤为分明,国外学者对此看得也很分明,“近代中国思想史的大部分时期,是一个使‘天下’成为‘国家’的过程”\footnote{列文森.儒家中国及其现代命运.北京:中国社会科学出版社,2000:87.}。这是一次巨大的文化转型,一次被迫的转化。这一过程有其积极意义,让中国人意识到世界之广大,世界文明之丰富,摆正了中国在世界舞台的位置,这种民族国家的“转化”似乎是现代中国进入现代世界的必由之路。“中华民族的崛起”、“屹立于世界民族之林”、“中华民族的伟大复兴”的说法,都得益于近代所建立起来的“中华民族”这个概念,以及在此基础上建立起来的认同感和爱国情怀。但另一方面,一种以天下为己任的宏大胸怀被迫转化为一种民族意识,似乎又与中国传统格格不入。从“天下”到“国家”的过程,使中国文化价值的伸张受到极大限制,“民族国家”的价值形态在中国遭遇诸多尴尬。源自西方的民族主义和民族国家构成了现代主导性的世界体系,但以这种方式我们不能对自身传统进行有效阐释,这是需要我们进一步思考的。

问题一:民族主义的话语人为地强化了文化的隔阂,狭隘的民族意识得以人为地加强。中国传统中的普世性文化关怀却没有办法得到伸张。中国的面积略小于欧洲面积,文化的丰富性一如欧洲各色的样态,欧洲的民族是在彼此区隔中确立的,而整个欧洲文明则显示出普世主义的特色。中国文化传统亦始有“天下为公”的价值观念,梁漱溟曾说:“中国人是富于世界观念的,狭隘的国家主义和民族主义在中国都没有,中国人对于世界向来是一视同仁。”\footnote{梁漱溟.国人的长处和短处.梁漱溟全集(第5卷).济南:山东人民出版社,1992:980.}但“民族”观念的人为锻造,甚至受到了种族论的影响,其极端表现甚至是一种以革命名义表现出来的狭隘意识。孙中山领导的辛亥革命,伴随民族主义而来的是强烈的排满思想,早年提出的口号甚至连中国境内的少数民族都要加以排斥,中华民族只是汉族而已,这立即会威胁到现代中国的统一和领土完整。于是梁启超在一种更广泛的层面上来定义“中华民族”,使之避免了直接等同于“汉族”的窘境。\footnote{梁启超.历史上中国民族之观察.饮冰室合集·文集.北京:中华书局,1989.}而孙中山在建立了中华民国之后,马上提出了“五族共和”的思想。狭隘民族主义的危险由此可见一斑。

问题二:狭隘的民族概念在文化上产生的自我矮化的作用。在被迫接受了民族国家的概念,进入所谓的“世界体系”后,在文化心态上付出了极大代价。这种心态预设了一个更高的世界标准,凡是中国的,都是特殊的;凡是国外的,则是世界的。于是,就有了“与国际接轨”“融入国际主流社会”之类的说法。不同时代,说法上虽不同,折射的心态却大抵一致,那就是对于自身文化的自卑,把强势文化看作普遍原理而加以认同,自己整合出来的无非是较西方世界次一等级的东西,“中国特色”更常常沦为一种自我掩饰的借口。等到一旦醒悟过来,则又学着美国口吻,满口“国家利益”、“核心利益”的说法。无论是弱势心态,抑或强势做派,究其根本则进退失据,义利失序,没有办法提出新的世界体系,只能迷失在西方人提出的世界体系观念中。胡适曾说过:“以数千年之古国,东亚文明之领袖,曾几何时,乃一变而北面受学,称弟子国,天下之大耻。”\footnote{甘阳.文明、国家、大学. 北京:生活·读书·新知三联书店,2012:457、27.}对于一个自诩天下文明的国度来说,这确实是一种耻辱,这种耻辱不在于做弟子,而在于失去准确的自我定位后,丧失了对于人类文明清醒的认识,模糊了自身对于人类应有的责任。

问题三:无法面对多民族的国家状况。狭隘的民族论完全不是中国文化传统的特质,以这种方式很难想象如何维系一个多民族的国家。当放弃了中华文明的普世性的特质后,就失去了文化的普遍感召力。一种对民族的狭隘理解可能直接导致国家的分裂,中国内部的民族问题就会激显出来,基于单一民族的民族国家观念对于中国这样的多民族国家来说,会形成一系列挑战。因此,梁启超从一开始就提出“大民族主义”和“小民族主义”之分,以弱化民族主义对于中国的危险性。但作为大民族主义的“中华民族”,内在地要求一种民族内部的均质化,这与多民族的现实状况是矛盾的。而作为小民族主义的民族观念在现代世界史上却已经充分显现了其狭隘性、危险性和爆炸性。民族主义的危险性在于它以文化的名义行事,其实质却是政治性的,它以建立主权国家为政治诉求,因此当多民族的国家建立以民族主义为基础的“民族国家”时,这同时意味着在教化其境内的各个民族寻求民族独立的目标。事实上,民族国家的观念对于像中国这样的多民族的文明形态来说,是完全不适合的。按西方路径,这就是一条不断分裂的道路,直到形成文化均质的单一民族的态势。

哈贝马斯在分析欧盟这样的后民族国家现象时,看到了民族国家在近现代不断分化之后,在全球化时代又兴起了超越民族国家的趋势。但在他的分析中有一个明显例外,那就是中国。中国既没有按西方标准以单一民族的方式进行裂解,也难以想象以欧盟的方式建立超民族国家。对此,哈贝马斯也无从判断,他只能说:“目前我们可以观察到最后一个古老帝国中国正在发生深刻的变革。”\footnote{哈贝马斯.包容他者.上海:上海人民出版社,2002:126.}那是一种什么变革呢?他没有明说,也许他的潜台词是在等待作为前现代帝国形态中国的分裂?也许他等待着中国人自己的定位?对中国这样一种多民族国家的形态,他只能感到疑惑。

现代民族国家的概念显然不适合中国这样具有文明特征的国家,这种源于西方彼此区隔,基于族群差异的概念,对于中国这样包容性的文化传统是一种极大的破坏。

二 以阶级为基础的国家论述

1949年以后,中国也曾有过一种新的国家定位,那就是接受了马克思主义的国家理论,以阶级为基础的国家理论。马克思主义的国家理论有其了不起的地方,敢于突破近现代以民族为基础的国家理论,以阶级为基础重新理解国家的本质。在《法兰西内战》中,马克思明确指出国家的本质就是阶级的统治工具,是一种人类自我异化的政治形式,因此工人阶级的革命,要用自己的政府机器去代替统治阶级的政府机器,而不是简单地掌握现成的国家机器,这就必须得实施无产阶级专政。这是一种以阶级为基础的“国家理论”,或者说政权理论。

因为社会主义革命的最终目的在于消除国家组织,巴黎公社的革命“是反对国家本身,这个社会的超自然的怪胎的革命,是人民为着自己的利益重新掌握自己的社会生活。它不是为了把国家政权从统治阶级这一集团转给另一个集团而进行的革命,它是为了粉碎这个阶级统治的凶恶机器本身而进行的革命”\footnote{马克思.法兰西内战.马克思恩格斯选集(第2卷).北京:人民出版社,1972:378.}。尽管在最终实现社会主义之前,公社是一种基于工人阶级专政的政权,是一种“国家形式”的过渡,但它依然有着一种让人民群众获得社会解放的政治形式。为此马克思深深地寄希望于“法国工人”,希望他们为法国的复兴,更重要的是为无产阶级的解放事业而斗争。

列宁在马克思对待国家的态度上更进了一步,在他那里真正发展出一种以阶级为基础的国家理论。他把阶级斗争理论与无产阶级的政治统治,或者说无产阶级专政直接联系起来,更加明确了国家形式和国家政权对于无产阶级的必要性。也就是说,尽管无产阶级国家是走向消亡的国家,但在其现实性上,一切革命的根本问题依然在于国家的政权问题。马克思主义逐渐发展出一种以“阶级”为基础的国家理论来替代“民族国家”理论,非常鲜明地提出了一种取代“民族国家”的新方案。

作为现代国家理论的一种替代理论,阶级国家理论的好处在于其所具有的普世主义取向,事实上,从列宁时代开始,就有了一种有别于民族国家世界体系的全球秩序方案,它超越于民族差别,号召全世界的无产阶级联合起来,试图建立起全世界社会主义共和国的联盟,以后的苏联也一直按这种方案推进它的全球秩序。

这样的国家理论同样有自身的优势和局限。这种具有世界主义的国家追求与中国文化传统有暗合之处。这种具有普世主义感召力的社会主义国家概念,事实上很容易为有着“人类大同”传统的中国文化所接受,最为典型的表现是:梁漱溟对“民族国家”的概念很是抵触,对社会主义国家却颇能接受,因为社会主义国家有着为着人类利益着想的普世主义向度。1949年之后,国家虽处于穷困时期,但世界社会主义一盏明灯的自我期许,在那个艰难时代还是获得了很多人的共鸣。马克思主义的国家理论尽管有着自身的理论指向,但是国家理论的普世主义形态似乎更加适应中国文化的要求,这也是中国文化传统中“天下主义”在起作用。同时,这种强化阶级意识的国家理论也可以淡化现代民族国家强烈而狭隘的民族意识,为解决多民族的共存提供了另一种思路。这一理论也为人民共和国迅速整合边疆少数民族提供了有力的理论武器,但在现实层面上依然有其局限性。

问题一:从根本上讲,这种理论本身强调国家只是一种过渡的形式,其终极目标是要消灭国家,无产阶级的根本利益在于在全世界实现共产主义,于是国家只是一种过渡到消除国家这种最终目标道路上的中间状态,不具有终极性意义。所以在最极端的革命时代,都是要以成立公社来取代国家政权,以表明这种政权理论的人民性和社会性。因为本质上,工人阶级要建立的国家并不继承以往的统治工具,是要建立一种全新体制。但事实上这依然是人类的一种遥远理想,与现实还有很大差距。

问题二:第一次世界大战的现实表明这种政权理论在西方难以生存。这次世界大战被界定为资产阶级的帝国主义为瓜分赃物而进行的战争,原指望全世界无产阶级能够联合起来共同反对这场资产阶级的丑恶战争。事实是这完全变成了一场民族国家之间的战争,无产阶级并没有坚定立场,而是跑去了为民族的资产阶级战斗。列宁愤怒地指责,这些社会沙文主义对于民族资产阶级的利益采取了“卑躬屈膝”的态度。\footnote{列宁选集(第3卷).北京:人民出版社,1975:172.}所以相比于民族意识,卢卡奇要强化的是“阶级意识”。

问题三:阶级斗争是以阶级为基础的国家理论的必然选择,但在现实中,人们在一系列的苦难面前认清了以阶级斗争为纲是无法建设一个健康、富强的国家,无尽的阶级斗争对于国家认同的基础也是一种破坏,社会需要一种和谐的前提。在世界范围内,全世界的无产阶级似乎并没有真正联合起来,反而依然在民族国家的话语体系中挣扎。向无国家的社会状态的过渡显现出了强烈的乌托邦色彩。

所以,当一个国家不再强调阶级分化,不再以阶级斗争为纲,不再以阶级专政为基础来建构国家话语体系时,这种以阶级为基础的国家意识就会迅速衰落,一种民族主义话语就会重新活跃起来,重新侵入我们的话语体系;同时它固有的问题则依然存在,其狭隘性对于一个多民族国家的挑战依然重大。因此,在狭隘的“民族国家”和“阶级国家”之外,似乎还需要某种新的国家定位理论。复兴“文明国家”的概念将有着积极的意义。

三 “文明国家”的概念

基于西方历史经验的“民族国家”和“阶级国家”对于中国人来说明显水土不服,由此形成了学者们在中国的国家问题上的定位失准。事实上,在现代,民族和国家并不是一定要结合在一起,历史上不是这样,现实中也不是这样。但一种非民族的国家理论还缺乏理论上的展开。很多学者敏锐地感到,在中国,将民族与国家结合在一起,尤其不适合。梁漱溟就曾将“中国不像一国家”作为中国文化的一大特征,更引用罗梦册的话称之为“天下国”\footnote{梁漱溟.中国文化要义.上海:上海人民出版社,2011:22.}。而现在比较流行的则是美国学者白鲁恂说的“中国不是一个民族国家,是一种文明,而伪装成一个国家”\footnote{Pye Lucian.China:Erratic State,Frustrated Society.Foreign Affairs,Vol. 69, Issue 4 1990:56-74.}。写《当中国统治世界》的马丁·雅克也有类似说法。确实中国文化传统自有一套关于“国家—世界”的理论,即便是要确立现代中国的定位也依然要顺应一种深厚的传统。从历史上看,在狭隘意义上,“国家”无非是朝廷、皇室;在宽泛意义上,中国的国家观念从来不是一种关于民族的理论,更不是一种关于种族的理论;“中国”这个概念甚至不是一个地域称谓,而是一种文化称谓。我们应该把一种理念和其在历史上的机制化形态严格区隔开来。理念的合理性使其有悠长的生命力,而历史上的机制化形态则有其僵化的一面。也就是说,传统的中国概念尽管是前现代的,尽管有朝贡体系等形态,但是一种以文明自许的形态却依然是宝贵的。甘阳先生2003年提出要把中国从民族国家重新发展为一个“文明国家”\footnote{甘阳.文明、国家、大学. 北京:生活·读书·新知三联书店,2012:457、27.},亦可见这种“文明国家”的理念在中国现代史上依然有其顽强性。对于有着这样一种以文明自许的传统来说,“文明国家”的概念将比“民族国家”的概念更有包容力,更具凝聚力,更符合现代中国的国家形态,也更能承载中国文化传统中强烈的天下意识。

对于西方的现代世界来说,在个体、民族、文明与世界的文化价值的序列中;现代的个体观念与现代的民族概念之间获得了很好的呼应,使“民族”成为建构基于平等个体的现代国家的一个很好的文化中介,并以“民族国家”的方式得到表达,在民族之上则以国际(international)的关系表达世界概念。在这个序列中,文明并没有得到一种实质性的安置。尽管盎格鲁-撒克逊、法兰西、日耳曼人或者斯堪的纳维亚人在民族上分属不同国家,但这并不妨碍他们共享西方文明,共享基督教文明。可是文明在这里并没有得到政治性支撑,只是沦为一个更为模糊的概念,这是西方历史铸就的国家样态所决定的。现代世界体系是西方文明框架下欧洲民族国家体系的放大版,它淡化了西方人共享的“文明”色彩,而强化了在西方文明内部的“民族”色彩;并在西方现代性话语形成的过程中,通过将西方文明在理性层面上的“普遍化”(generalization)而为世界提供了一种“现代文明”。

当中华文明遭遇到“现代”世界,当这个文明要跻身现代国家时,我们面对的是淡化了西方色彩的“现代文明”,是强化了民族色彩的现代国家。于是为了现代国家的认同,我们也依瓢画葫芦制造了“中华民族”概念。但水土不服,一直在理论上面对质疑,在现实中也遭遇困惑。相反,“文明”的概念在国家定位上依然强劲地浮出水面。

“文明国家”就自身而言是一个更为完整的意义体系,作为“文明国家”其价值形态、生活方式、礼法制度、文明理想等各个方面都有自己的系统,是完整的“意义世界”,同时具有一种向善的典范性,其包容性和普世性也是民族或者民族文化的概念所容纳不了的,这些特点恰恰是中华文明所不可或缺的。民族国家在其起源上是以“自我”和“他者”的对峙作为前提,更强的版本则以区分敌我作为其建构的基本机制\footnote{参见卢梭对于民族宗教的论述,《社会契约论》第四卷第八章,商务印书馆2003年版。},差异性和区分异族是建构“民族”的基本路径,民族充其量只能以“文化”涵盖之,以突显文化上的差异性和族群的特征。在文明概念中“天下意识”非常重要,这是一种中国式的普世向度。正如我们所分析的,在“民族国家”中,这种普世关怀的向度得不到声张,甚至完全丧失。在“阶级国家”中昙花一现的普世向度,在重归“民族主义”的话语中再次丧失。一种适宜的方式将是在国家层面上安放“文明”的普世向度。

作为“文明国家”,在本质上有着人类共享的普遍价值,文明为人类建立了普遍的道德和生活意义的标准。现代中国并不只是1844年以后的产物,现代中国承载的是源远流长的文明传统,是“作为人类,代表人类,为了人类”做出杰出贡献的国度。五四以来以摒弃自身传统作为立国基础的思想将得到彻底清算。“文明国家”的定位可以让我们在传统与现代之间找到自己真正的立足点。我们必须承认,文明所显示的这种人类的代表权自近代以来已经自觉地拱手相让于西方,西方哲学家们以理性名义对自身文明的自我确证,更让现代中国人盲目臣服于西方中心主义,于是西方成了普世世界,而中国则只是具有人类学意义的文明类型。\footnote{参见胡塞尔《欧洲科学危机与超验现象学》,上海译文出版社2005年版。}但是,中国作为“文明国家”其普世性维度一直顽强地存在着,文明的整个价值形态决定了它不只是民族性的、区隔性的,文明更呈现其共通的一面。明末利玛窦来中国时为中国人带来了从没见识过的西学,当时的中国人马上明白他不是一个野蛮人,而是一个文明人,士大夫争相与之结交。他带来的工具化、形式化的欧氏几何虽不是中国人熟悉的,但很能得到认可,并迅速掌握,甚至做得更好。同样,中国作为“文明国家”的普世性在于中国人在这种文明的发展中特别发挥出了人类的诸多潜能,并提供了其中好之为好的基本标准,这种文明成果是可以为人类所共享的。在现时代,我们特别需要对此进行系统的挖掘和阐释。在全球化时代,这种共享将更为迅猛,中国饮食文化在世界的流行只是一个微小表征。当然,“文明国家”的界定远不止这些,“文明国家”更是一种自我提醒,提醒对于自身“文明智慧”的意识,提醒自身“天下关怀”的责任,提醒自己对于世界历史的使命。

“文明国家”自然包容着多民族的状况。我们无需再以“大民族”和“小民族”来混淆民族概念,在中华文明下,多元一体格局将继续保持下去。我们将以“文明”而不是“民族”的方式建构国家认同。欧洲过往强烈的民族认同,带给欧洲不断相互争斗的动乱,在欧盟超国家的框架下,欧洲人开始建立其对欧洲文明的认同。美国作为文明国家也不以民族主义作为国家认同,美国建国时的清教主义精神,以其普世性荡涤了民族主义狭隘性,因此在美国,民族主义是一个负面概念,而爱国主义则是一个正面概念。在有些学者看来,多个民族共和一体的国家必须通过一种超越民族主义的,公民式的宪法爱国主义来加以整合,事实上单靠形式化的宪政主义是很难进行整合的,还需要实质性的文明价值来加以凝聚,一如美国的“公民宗教”。中国作为“文明国家”也应该建立其自身的文明认同,以超越狭隘民族主义的藩篱,建设健康的多民族国家。哈贝马斯所疑惑的“古老帝国”在现代将既不是一种传统帝国的延续,也不会以西方的逻辑裂解,而是以“文明国家”的姿态继续存在下去,并且在全球化时代发扬光大。

那么,这样一种“文明国家”在什么意义上可以避免传统“帝国”的形象而依然是现代国家呢?这个疑虑其实大可不必,现代国家有两个基本标准,一是国家的权力来自人民,来自每个公民的授权,保障每一个公民的权利即是现代国家的标志;二是在世界范围内,国与国之间是相互尊重和平等的。中国作为文明国家,同样是建立在对每一个公民权利的尊重上的,尽管其文化认同是文明的,而不是民族的,这并不妨碍它依然是一个现代国家。另一方面,在世界体系中,无论是文明国家还是民族国家,都是一种“主权国家”。在国际社会中,各个主权国家在法律地位上是平等的。因此,“文明国家”的定位并不会超越现代国家的范畴,而退回到“帝国”状态。这里的“文明”更在于一种教化性和示范性,她对世界的影响不是靠武力征服和强力灌输,而是通过一种示范性的价值规范来确立自身影响。

在文明国家的前提下,我们需要建构新的“文明国家”的世界体系,因为不光中国是文明国家,印度是文明国家,欧盟是文明国家,甚至美国也是代表西方类型的文明国家。这里要注意的是西方的两重性。一方面,西方文明以自身理性化的方式建构了“现代文明”的框架,以对每一个人权利的尊重确立了现代文明的基本特征。这是西方文明对于人类的贡献,值得人类所共享。另一方面,西方文明继续保持着其自身的价值观念,有其特殊性。我们必须认识到,西方文明并没有穷尽人类的意义世界,源自中国的文明,源自印度的文明依然有能力对人类共享的“现代文明”做出贡献,因此,我们需要建立一种更广泛的“文明国家”的世界体系理论。这对于理解现代世界,并且促进现代世界变得更加美好,有着非凡意义。亨廷顿在“冷战”之后迅速领悟当今世界乃是多重文明的时代这一事实,但其结论却是基于西方传统的模式来理解的。霍布斯式的个人是冲突的,民族国家之间是冲突的,在亨廷顿笔下文明体系之间也一定是“冲突”的,这是亨廷顿以“民族国家”的视野来理解文明体系之间的关系得出的必然结论。我们亟须从新的视野来理解“文明国家”体系,通过文明对话和共融,建构新的世界体系和全球秩序,这将是一个不同于以往西方所主导的全球秩序。我们需要从“天下”的视野来重新理解文明间“和而不同”的关系。

四 天下意识

建构“文明国家”的国家理论是一项系统的工程,其中最为关键的要素在于体现展现中国人的普世意识,也就是中国文化的“天下”意识,着眼的是人类的“公理”。中国的文化传统一直以极大的胸襟接受其他文明的宝贵资源,如魏晋之接受印度佛教文化、明末之接受西学东渐,现代之接受马克思主义,没有丝毫抵触,一种包容天下的胸襟。作为文明中国,不仅仅是要包容天下,也还必须在全球化的时代,以“天下为怀”的境界,回馈整个世界。

从普遍化、普世关怀和普世主义三重区分来看,中国文化传统在近代化的过程中缺乏通过理性的方式来普遍化(generalization)自身文明的过程,而是将自身传统在现代性的语境下特殊化(particularization)了。由此带来的后果与中国文化传统中特有的普世关怀(universal concern)的情结格格不入。中国文化传统在世界上的顽强生存和流传与这种普世情怀息息相关。西方社会以普世主义(universalism),一种将普遍化理论推向整个世界的方式来表达其普世关怀,其同一化的价值取向会使非西方社会本能性地抵触西方普世主义。普世主义与普世关怀有着极大的差别,但都是构筑文明国家的基本要素。我们需要通过“文明国家”的理论,努力使中国文化传统在现代世界的语境中得以普遍化(generalization),并以此表达出对于世界的普世性关怀,这是一种文明的生机所在,断然不可以狭隘的“民族”概念自我矮化。要恢复中国文化传统中的普世情怀,“天下”概念的激活是必由之路。

“中国”和“天下”在古代“六经”中都可以找到,“中国”的概念与“天下”的概念是联系在一起的,“中国”是一个与“天下”相对应的概念。在“天下”的范围里,“中国”就是一个示范性区域,也就是我们所说的“文明国家”。所以说,“中国”在本质上和西方“民族国家”概念不是一回事。今天我们光把“中国”这个名词留下了,使之成为一个“民族国家”,而把“天下”给遗忘了。没有“天下”哪有什么“中国”。古时候讲“天下”现在来看确实只是有限的区域,天下秩序也不过是区域性秩序。但当初其所想象的范围却是全人类的;就像欧洲人讲人类的时候,大部分时候实际上所能指的也只是欧洲而已。关键是它们各自在历史中形成了理解世界的模式,这种模式在今天依然可以是普遍的,可以成为全球秩序的原则。“民族国家”的概念是一种放大版的霍布斯式个体,其基本动力是自我保全,争夺利益;“文明国家”倡导的是“同天下之利”,为“天下”立法。因此“中国”这个概念在历史上的定位具有强烈的道德涵义和责任意识。因此,中华文明的普世关怀一定要在其国家定位中得到表达,这也就是“文明国家”之于中国的定位要远胜于“民族国家”的原因所在。

事实上,在“民族国家”观念输入中国后不久,像无政府主义、社会主义等世界大同的观念也在中国思想界大行其道,顺带着世界语也在中国流行起来,这是中国文化传统中追求“世界大同”的心态使然。“民族国家”的体制始终与中国多民族国家的现实不合,与中国固有的思想观念不合;“民族国家”背后狭隘的民族主义,甚至种族主义更与中国文化传统中的“天下”观念尖锐对立。西方世界出现希特勒这样的极端人物,在某种意义上讲是正常的,他把民族主义背后的血统论种族主义推向了极致,以极端方式显示了以狭隘民族主义为认同的世界会是怎样一种血腥的体系。犹太哲学家E·莱维纳斯在批判西方存在论传统的著作《别样于存在》一书的题记中写道:谨以此纪念那些死于纳粹大屠杀的600万中最亲密的人,因此暗喻西方存在论传统与希特勒暴行之间的理论关联。确实,即便是康德的《论永久和平》,建立世界和平的前提也只能是霍布斯所说的状态:国与国之间像狼和狼一样,永远处于争斗状态,二战之后的联合国正是这种无休止争斗的无奈妥协,这个问题今天依然没有解决。这样的世界格局与中华文明的“天下”意识是完全隔膜的。“天下为怀”的意识并不是以民族利益作为最高利益的,而是以人类的利益为利益,“天下非一人之天下,天下之天下也。……仁之所在,天下归之。……德之所在,天下归之。……义之所在,天下赴之。……道之所在,天下归之”[六韬]。中国作为“文明国家”就应该是“仁”“义”“道”“德”之所在,而人类应是“四海一家”的大同社会,这就是文明国家的价值指向。

“天下”观念是中国文化最具特色,最具宏大愿景的一面。“天下”意识并不是像西方人那样要强力输出普世主义价值;更多的是着眼于一种文化的整合性、道德性的动力,展现对人类和世界的整体性关切。当一种文化只局限于自身民族的复兴时,它的眼光是局促的,目标是有限的,手段是排外的,动力是不足的。中华文明的复兴首先就是要还原其本来的价值目标。关于中国传统的“天下”意识如何转化为民族意识的论述已有不少\footnote{参见姚大力《“天下兴亡,匹夫有责”的再诠释与中国近代民族国家意识的生成》,《世界经济与政治》2006年第10期。},这里要强调的是一种“反向转化”,也就是从压缩为“民族意识”的中国文化传统中再次释放出普世性的一面,还原其本来的“天下”意识。

“天下”概念蕴含了丰富的价值资源,“天下为公”,“四海一家”都是中国文化传统下普世关怀的价值体现。今天我们讲“天下”并不是要恢复传统的朝贡制度,“五服”制度,这只是囿于当时历史格局所形成的某种机制化表达。今天讲“天下”概念主要是要恢复与“文明中国”相称的“天下为怀”精神,侧重文明的价值理念,侧重精神层面的普世关怀,而不是要膜拜传统的机制化结果。中国文化传统有关于人类的基本预设,那就是“人同此心,心同此理,东海西海,天下一家”。这与现代民族国家的世界体系对于人类的设想完全不同。在这个意义上,我们也许更能理解,为什么中国人对于共产主义的社会理想能欣然接受。这种普世关怀是中国文化传统的一大特色,无论是儒墨道佛家都有“天下”的情怀,这在全球化的今天显得弥足珍贵。

从形而上的层面看,“天下”在全球化的今天有着极大生命力。西方文化传统中,也曾出现过像尼采、海德格尔等呼喊“大地”的哲学家,倒是很能与中国文化传统的“天下”意识相呼应。非常遗憾,也许是与水土和方位有关,“大地”概念的非道德化以及某种地域性种族化的倾向非常强烈,正好反衬出“天下”概念的普世性。中国的“天下”概念可以从三个层面来加以刻画:

第一,这是一个道德化概念。在个体本位的今天,道德似乎只限于个人修为,殊不知这个世界亟待一种整体的道德观念。现代世界的基本法则是在利益的“丛林法则”下建构起来的。中国文化传统中“天下”观念与文明道德教化相关联,不是霍布斯所揭示的争斗和均衡原则。中国文化传统界定人性的标准就是仁义礼智的教化。人性和文明本质上是一回事情,个人的道德直接呼应于天下,在教化中把“天下”的道德关怀呈现出来,由此而风化天下,在政治上则体现为“王道”。春秋战国时,王道和霸道的区别就在于以力胜人还是以德服人。《论语》说,“君子之德风,小人之德草,风行草上,必偃。”[论语·颜渊篇第十二]“风化天下”就是一种道德教化的软实力。“天下”概念展现的不是力的关系,而是教化关系,与人类价值形态息息相关。今天全球化的驱动力,基本上是一种资本和技术的驱动发展,完全不是文明教化的过程。现代世界把更多的物质生活纳入考量范围,于是“天下熙熙,皆为利来;天下攘攘,皆为利往”。现在有越来越多的学者提出“全球正义”概念,但从休谟、罗尔斯的正义观到“全球正义”其实也都是从利益原则出发来理解世界,其范式并没有发生根本转换。在“天下”观念中,我们需要重新思考“义利之辩”。“天下”不是一个单纯的政治概念,也不是一个利益概念,而是以“义”为取向的价值概念,孔子讲“大道之行也,天下为公”,此之谓也。

第二,这是一个整体性概念。全球化时代,人类需要整体意识。“天下”意识与当代地球村、全球化的脉动相吻合。引用老子讲的一句话,我们要“以天下观天下”\footnote{参见赵汀阳在《天下制度》中对此的阐发,江苏教育出版社2005年版。},这是非常了不起的见识,是一种以人类为整体的视野,这在西方观念中是极度缺乏的。西方世界所能考虑的最高原则就是“国家利益”,这是西方化现代世界的法则,也是美国人的口头禅。当我们出于“天下”意识而提出的正义原则时,这个世界是不予理解的;当我们也满口“国家利益”时,我们降低了自身的文明高度。要教育世界学会“以天下观天下”,从世界的角度来看待世界,把人类作为整体来思考。事实上,只有马克思对于资本的批判才达到了这样一个人类的高度。中国传统的“天下”概念,以人类的整体状态为关怀,以文明教化为着眼点,以道德心的提升为路径。在“天下”的观念中突破种族、民族、国家利益的藩篱,从人类角度思考世界前景。这亟需通过“文明国家”的体制传达给世界。

第三,这是一个“和而不同”的概念。亨廷顿非常敏锐地看到,“民族国家”的世界体系其实只是欧洲的放大版,并不适宜对世界作整体描述。是“文明”而不是“民族”才是一种终极的,不可化约的“意义世界”,于是亨廷顿得出“文明冲突”的结论。按亨廷顿的看法,避免冲突只能靠文明核心国之间的协调。看似新颖的“文明冲突”论,背后依旧是丛林原则,是“民族国家”世界体系的固有思路。“天下”意识着眼的“天下”,是一种文明多样化的概念,而不是在同“一种”文明下的“天下”。这里没有用一个原则一统天下的压力,而是着眼于在整体状况下追求“和而不同”的状态,在中国文化传统中,“和”是和谐,是“不同”之间的秩序,一如音律,不同的音,奏出“和谐”的音乐。小人“同而不和”,正是当今世界的写照,为求取世界普世主义的“同”,造成的却是世界的“不和”。

摆在我们面前的有两套理论,一套源自西方,以普世主义的方式在现代世界大行其道,强而有力,中国被迫卷入其中。另一套则是对天下的理解,源自中国,充满普世关怀的道德感,但尚待以理性化方式展开。面对这两套理论,会有根本不同的选择。面对全球化时代的种种危机,究竟该如何面对:源自西方的现代性体系在尊重每一个个体权利方面在人类历史上拔得头筹,因而在世界范围有着极大的道德感召力;但在群体方面,采取的却是武力教化的模式,“冲突”是现代世界的基本线索,无论个体之间、阶级之间、国家之间、民族之间、文明之间,甚至是人类与自然之间,首先是“冲突”,在你死我活之后,再来进行解决。这种模式展现了人类为利益而斗争的一面。另一种是善意的“天下”模式,其对于个体权利的认知是欠缺的,但在整体上肯定多样化的现实,更强调协调、互补,以及善意地理解和学习,着眼于多元要素的圆融,尤其是在人类与自然之间。在这方面,中国的文化传统极具魅力,其核心要素在当今世界中还远远没有展现出来。面对这两种模式,绝不是只能二择一,我们需要更深远的思考。

起码,我们可以先从恢复“天下为怀”的意识做起,就是要从个体主义、民族主义乃至人类中心主义的一己之利中超拔出来。丰富现代人的心灵,抵御各种等级利己主义的袭扰,前瞻超越现代世界观的维度。现代心灵在不断枯竭,虚无主义无休止地蔓延,最后行为的动力只剩下了欲望,这是需求等级中最低级的层面。如果“天人合一”的关系能够重新建立起来,天地人之间,就会有丰饶的成果产生。“天地之心”通过人来产生就是“良知”;“四海一家”就能成为人类的家园;敬重自然、顺乎自然的天人意识更可以成为我们的行动指南。这种天人格局的恢复,需要就是“文明中国”天下为怀的境界,这样一种文明的复兴不只是指向哪个民族,更指向对全人类的贡献。今天讲“文明中国”的确立,一定要以人类的命运为指向,这才是真正的“天下为怀”。

这种“天下”观不仅贡献自身的文明智慧,更力图涵盖、理解、并且深化各种文化传统中的深厚资源,在全人类的层面上指向一种更加完善的自我理解和行为的方式。在生态危机日渐严重的今日,世界上各个国家已经结成了深深的地球命运共同体,即便单个民族国家得到“复兴”,造成的却可能是整个地球生态平衡的崩溃,那么这样的“复兴”还有什么意义呢。地球村时代,任何一个事件都有可能牵一发而动全身。如果中国的复兴不以人类为关怀,就很难在这个时代展现其真正的生命力。中华文明的复兴须以这个时代人类的基本诉求为己任,须对人类的共同命运肩负起使命。一种文明只有以“天下”苍生为念,这个文明在现代世界才依然有自我超越和发展的能力。“文明中国”的复兴也只有在这个意义上才能得到检验。世界正在迈向一个全新的时代,以“文明国家”为基础的世界体系也必将形成一种新的“天下”秩序。

\section{生生不息:一种生存论的分析 2017}

来源:《现代儒学》第一辑,复旦大学上海儒学院 编,三联书店2016年版

时间:2017年10月12日

生存论哲学对于传统的西方哲学来说是一次解放,西方哲学从古希腊到黑格尔(Hegel),一直都在传统的哲学概念和范畴中打转。哲学作为一种理性框架实际上带有强烈的希腊色彩,以至于海德格尔(Heidegger)说,哲学说的是希腊语 。西方哲学想表达自身的生存论经验也深受其苦,于是有了一场现象学—生存论(phenomenological-existentialism)的哲学革命,它通过现象学还原(phenomenological reduction),将各种不同的文化传统通通悬置起来,从而显现出其生存论结构(existential structure)。现象学“朝向事情本身”的口号,使哲学不再拘泥于传统的哲学概念和范畴,使回归生存论经验的研究成为可能。海德格尔关于“此在”、萨特(Sartre)关于“自为”与“为他”、莱维纳斯(Levinas)关于面对他者的生存论描述,在在显示了现象学—生存论哲学的革命性。这些关于生存论结构的描述对于分析一种中国文化传统下的生存论结构非常赋有启发意义,它使我们可以把关于文化传统的论述与一种生存论的经验剥离开来,对一种文化传统进行生存论结构的阐释,实现一种更为普遍化的阐释。

一 “向死存在”的生存论分析

在海德格尔的生存论分析中,他对于“死亡”(death)的论述构成了《存在与时间》中对于“此在”(Dasein)分析的一大篇章。当海德格尔试图对“此在”作出一种整体性的勾勒时,“死亡”问题,或者说“向死存在”(Being-towards-death)成了海德格尔把捉这个问题的重要契机。在“此在”中始终有某种东西是“亏欠”(outstanding)着的,也就是尚未完成,而一旦“此在”不再有任何“亏欠”,那它也就“不再在此”,也就是终结于“死亡”。海德格尔揭示出个体生命不可能无限延长这一“此在”的本真特征,“死亡”正是一个节点,它宣示了“个体”有限的整体性存在,为海德格尔的生存论分析提供了空间。当有人对海德格尔追问“存在的意义”何以要从“死亡”出发大惑不解时,有论者将海德格尔的这部分论述称为“世俗化的神学”(secular theology),这恰恰解答了上面的问题 。宗教之于人生的最大作用便是提供了人生的意义,也就是从终极处对于我们有限的人生给出解答。因此对于这个问题的解答也许并不在海德格尔自己身上,还需要到西方的基督教文化传统中去寻找。可以说,基督教对于生命的理解是西方文化传统中对于生命意义理解的一个原型。这种生命理解的特点在于对“死亡”有着强烈意识,对于末日审判有着强烈期待。在我们日常生活中,“死亡”乃是生命的消逝,是生命从生存到不再生存的关节点。“死亡”在西方的文化传统中始终是哲学和宗教的深思对象,更是基督教信仰的主旨,“上帝要救赎我们的命脱离死亡”(《圣经•旧约•诗篇》:103:4)这一命题表达了基督教的根本愿望。死亡作为生命的大限横亘在生存道路上,必须借助耶稣基督才能予以克服:信基督就是为了得永生。生命在此世必定会结束,那么我们如何关注死后生命呢?灵魂会去哪里?肉体如何消散?在宇宙漆黑茫然的背景中,这是西方文化传统中人对于生命最根本的焦虑。只有上帝才是人们唯一的救赎,是人们死后生命的保障,耶稣基督是人类的救赎者。在这幅生命图景中,生命在死后的延续非常重要,这也是此世生存的意义所在。死亡是一个关节点,更确切地说,是两段生命之间的链接点,死前生命和死后生命之间的桥梁。“死亡”是西方文化传统中一个永恒主题,柏拉图(Plato)说过,哲学就是“练习死亡”,意思是说哲学是在为进入“死后生命”作准备。可见“死亡”主题也正是希腊传统与基督教传统相结合的切入点。

“死后生命”以永生方式超越了生命的有限性,赋予时间中的生命以永恒意义;同时通过“死后生命”的中介(“爱上帝”),道德(“爱邻人”)找到了它扎实的底座。但尼采(Nietzsche)的“上帝之死”振聋发聩,当生命的理解不再有上帝保障,其冲击力可想而知,最为稳固的基座发生动摇,世俗化社会势所必然。因此在上帝之后,重新寻找生命的意义对于哲学家来说责无旁贷。这里悄悄地发生了一种转换:西方基督教文化传统重视“死亡”的范型将转化为现代社会对于“个体”有限性的生存论理解。尽管海德格尔从其生存论本身给出了引出“死亡”概念的线索,但我们可以从更广阔的文化背景出发,把“死亡问题”视作一种理解生命的切入点,只不过海德格尔将着眼点从“死后生命”转向了“死前生命”。海德格尔式的生存论哲学以死亡作为理解生存的根本出发点,并提出了一种对于死亡的本真性理解。通过“此在”对于“死亡”的“畏”(anxiety),揭示出“此在”生存论的整体性存在(Being-a-whole)。

在海德格尔看来,“本真”(Authenticity)的死亡并非人生的一个事件,“此在”并不能在周遭世界中遭遇自身的死亡,因为一旦遭遇死亡,“此在”便不再存在了。“死亡”是“此在”经验的一种存在的可能性,它不是一种现存在手(present-at-hand)的东西,不是一种客观对象。这与动物的死亡不同,动物的死亡只是一个有机体停止了它的生命。而“此在”的特点在于对自己不可避免的死亡有着充分领悟,这将对他的整个生存发生根本性影响。死亡使“此在”的任何生存变成了绝对的不可能,它是我们面临着的贯穿一生的可能性。因此海德格尔宣称,正是在这种关系中,揭示了一种“向死存在”的生存论结构,这是“此在”最本己的(ownmost)关系,并从根本上把人的生存看做终有一死的生命。不再像基督教那样通过“死后生命”来寻找生命意义,而是要在有限生命中确立本真性的生命。

“死亡”作为“不再在此”(no-longer-Being-there)的可能性,帮助海德格尔揭示了个体“此在”的完整性、有限性。因为“本真生命”只有面对死亡,才能真正看清自己的可能性,并向自己或向世界“筹划”(project)着去实现在有限生命中的各种可能性。海德格尔重视死亡,是为了凸显“未来”作为可能性的存在,是为了凸显某种期望和谋划,以死亡为限来求取对此世生命的筹划,这就是海德格尔在论述“此在”的“生存论结构”时所谓的“向死存在”的说法,以此他勾勒出“此在”生存的整体性,是“此在”具体生存的每一个瞬间所构成的整体;但这同时也显现了“此在”的有限性,在从一个瞬间到另一个瞬间的持续中,最终要走向终结。这是“此在”生存论结构最根本的特征。死亡从根本上来说是生命的终结,是不能抓住的,是陌生的、不可预见的可能性,但必然到来。海德格尔的生存论分析强化了生命之流的这种“断裂性”。

在海德格尔的描述中,有几点特别值得我们注意。首先,“向死存在”的基本现身情态(state-of-mood)是“畏”(anxiety)。当然畏死并不是一种软弱的情绪,而是昭示出“此在”生存的基本特点:带着对于人的有死性的明确意识去生活,就是始终面临着生存的威胁,面临着自己生存的空无一物,并需要在此背景下作出选择。海德格尔的“畏”所焦虑的就是在世本身,是作为世界展开的世界本身,而不是具体的存在者,也不是它们的集合体,这样所描述的“世界”并不呈现任何具体的东西。海德格尔在分析“畏”的时候强调的是把“此在”带入世界之中,由“畏”开发出“此在”本己的在世。“畏之所畏就是世界本身。无与无何有之乡中宣告出来的全无意蕴并不意味着世界不在场,而等于说世内存在者就其本身而论是这样无关宏要,乃至在世内事物这样无所意蕴的基础上,世界之为世界仍然独独地涌迫而来。”海德格尔通过“畏”带出了“无”(nothing),带出了存在背后的黑暗(darkness),带出了根本上的一种“不在家”(not-at-home)的状态,一种“无家可归”(uncanniess)的状态。

与之相应的则是海德格尔对于“被抛状态”(throwness)的描述,海德格尔认为“此在”在“畏”中“被抛”掷于此,也开始了“此在”在生存中的“筹划”。在“被抛状态”中,“向死存在”剥离了各种日常生活的规定性,使“此在”不再是沉沦于日常生活中的“常人”,以这种方式完成了对非本真性生活的否定。海德格尔的“筹划”并不是拟定计划的自我设计,而是“此在作为此在一向已经对自己有所筹划。只要此在存在,它就是筹划着”。“此在”的“筹划”可以投身于自身最本己的可能性,也可以投身于世界之中,沉沦于世界之中。

尽管海德格尔极力批判现代性,但本质上他对“此在”的生存论描述与现代性对“个体”的理解是完全一致的。“个体”(individual)意味着“不可分割”(indivisible),很多时候并不专指个体的人,而是用来描述“单个事物”,这是与其他东西分割之后,不可再分割的单位。这既可以是在群体中割裂出的、不可再分的“个体”;也可以是在生命之流中割裂出的、不可再分的基本单位。所以只有和死亡联系起来,我们的生存才能变得是真正的“个体”。“个体”在本质上蕴含了“死亡”对于生命的分割,这种分割体现了“个体”自身的完整性,也无奈地显示了个体的“有限性”,最根本的是,在海德格尔式的经典论述中蕴含了“生命之流的断裂”。

二 “生生不息”的生存论结构

按海德格尔的说法,要回答存在的意义,首先在于“此在”对于存在的领悟,这种领悟是我们借以通达存在意义的途径。“向死存在”构成了“此在”对自身存在方式的理解,这种生存论分析本质上是从“存在的断裂”(break of Being)的角度得出的“此在”的个体性和有限性的本真特征,有着极为深刻的合理性。海德格尔的分析体现了对于个体生命的深刻理解。我们知道,尊重个体生命是现代文明的标准,推崇个体自由是现代世界无限创造力的来源。在这个意义上,海德格尔式的理解依旧与现代性息息相关。但这样一种深刻理解依然有着巨大的局限,并不能充分揭示存在的意义。从某种意义上讲,海德格尔对于“此在”生存论结构的揭示,只发挥了生命阐释的一端。“向死存在”构成了“此在”存在方式的根据,但是对于生命之“生生不息”的体验和领会同样非常重要,基于“生生”的生存论结构不是一种现成事物的运动,同样是生命规定自身存在方式的根据。这恰恰是西方文化传统理解生命的死角,基于“生生不息”的生存论结构也被完全遮蔽了,我们需要通过“生生”的生存论结构来敞亮存在意义的另一个积极面向。

新儒家们也曾看到中国文化传统中对于“死亡”的理解与西方是不同的,但给出的却不算是一个恰当的切入点。他们认为中国文化传统“要人兼正视生,亦正视死。所谓杀身成仁,舍生取义,志士不忘在沟壑,勇士不忘丧其元,都是要人把死之问题放在面前,而把仁义之价值之超过个人生命之价值,凸显出来”。新儒家念念不忘于仁义固然没错,却没有把捉到对于生命的本真性理解,更没有揭示出中国文化传统中的生存论结构。个中问题,不在于用仁义价值来取代或克服个体生命的短暂意义,强调仁义价值超越个人生命,而是要揭示中国文化传统对于生存论结构的根本性理解。

对于生存论来说,虽然“死亡”不可体验,但却揭示了个体生命的终结;相对于死亡,“生”也是一种生存论的事实。“生”不仅仅是生命的开端,也伴随着生命的“始终”,不仅是出生,也是生长,更指向未来,是“生命”中向着世代交替的未来开显的可能性。这与生命的“繁衍”(fecundity)有着直接的相关性,因着“繁衍”更可以演化出一种别样的生存论结构。也许在海德格尔看来,“繁衍”因其过于生物化而淡出其理论视界。其实“繁衍”远不止是一种生物行为,它揭示的是“生命的延续”,而且与动物通过繁衍来延续它们物种的存在不同:正如人可以领悟自己无可避免的死亡,人也可以领悟“繁衍”之于生命之流的“生生不息”的意义所在。这也是理解生存论结构的重要资源,中国文化传统的特色所在。

在这方面,最能体现中国文化传统特色的莫过于《周易》。《周易》把整个世界看作是“大化”流行的生命现象,用存在哲学的话语来说,就是特别重视“存在的连续性”(continuity of being)。杜维明先生曾从中国人自然观的角度论述过“存有的连续性:中国人的自然观”,并特别注重中国哲学中的“气论”。本文无意从这个角度来论述。从海德格尔关于“存在论层次”(ontological)与“存在者层次”(ontical)的区分来看,“生生不息”首先是存在论层次上的问题,强调的是存在的“连续性”。《系辞》称“天地之德曰生”“生生之为易”(《周易•系辞传》),天地间最大德性就是“生生不息”,只有从“生生不息”的道理中,才能把握存在的意义。《周易》对“生生”的重视,为我们提供了理解中国文化传统生存论结构的一条线索,按孔颖达的解释,“生生,不绝之辞。……万物恒生,谓之易也。……有生必有死,易主劝戒,奖人为善,故云生不云死也。”这里从“万物恒生”转到了对人的生命的领悟上,用传统的话来说,是由天道而人道,也就是在某种特别的存在者的层次上来理解“生生”问题,即转到了生存论层次上的分析。强调“生命之流的连续性”如何在人的生存论结构中展开,这就是本文副标题所提示的,一种生存论的分析。在存在论上是重视“连续”还是重视“断裂”会显示出生存论上不同的取向。

《周易》这种对万物“存在连续性”的深刻体悟并非玄之又玄的形而上学,它直接显现为我们日常生活的“预设”,体现在我们对自身生存的领悟上。成语“愚公移山”就透露出这样一种存在论玄机。愚公立志要移除挡在门前的大山,不免被智叟讥笑,但愚公却乐呵呵地说,虽然我会死,可是我还有儿子呢!儿子又生孙子,孙子又生儿子,儿子又生儿子,又生孙子,子子孙孙都不会断绝的呀!他们可以把山挖平。这个成语所表现的最关键之处,不在于愚公自身坚忍不拔的气势,而在于潜藏其背后的对生命的基本预设,对基于“繁衍”的生命之流的理解。这种生命的坚忍不拔,不是靠上帝给予个体“永生”的恩典,而是对存在论上“生生不息”的礼赞。

进而言之,在愚公的生存论结构中,表达了对于生命之流“生生不息”的无限期待。在“向死存在”中得出的是“此在”的个体性和有限性。在愚公的生存论描述中,则体现出生命之流的承续性和无限性。在生命的流转中更表达出对于生命的积极体认和热爱,从“生”作为“天地之大德”,引出人生的基本态度。所以《象传》说:“天行健,君子以自强不息。”“天行健”是宇宙万物“生生不息”的生命流转,是生机勃勃的自然进程,而君子效法“存在连续性”的精神,以之塑造自身的生存,从而刚健有为、自强不息;不仅是自身的刚健有为,这种生命精神还要在子孙万代中传续发扬,以此克服生命的有限性。在这方面,中国文化与犹太文化有着强烈的一致性。莱维纳斯曾批判海德格尔的死亡观,用希伯来圣经中“爱克服死亡”的传统,来构造出基于“爱洛斯”(Eros)的生存论结构,他甚至说过:未来就是有一个儿子的可能性 。

由于生命的“诞生”其来有自,“此在”那种孤独彷徨无所依傍的状态并不是中国文化传统下的人所熟悉的。在“生生不息”所体现的生命洪流中,生命的无限性被展现出来。确实就个体生命而言,“死亡”揭示了有限的存在。“生生不息”的生命现象则表明,生存的背景并不是漆黑的“无”,而是其来有自。生命的诞生在个体身上可能表现为一种偶然,但从生命之流来看,这是生命自身的要求;因此,“此在”的“诞生”并不是一种“被抛的状态”,而是对生命本身的承诺。从“诞生”伊始,任何一个生命体都是有父有母的,在茫茫宇宙中,从一开始就是一种“在家”(at-home)的感觉。个体生命的诞生,始终都包裹在“生命连续体”中的。因此,在中国文化传统中,“家”(family)乃承世之辞,“承世”就是承载“生命延续”的世代相续。从根本上讲,“家”具有一种存在论上的地位,而不是存在者层次上的样态;在这个意义上,“家”远不只是一种社会组织,远不只具有日常生活的价值;它是无限“生命延续”的承载者、保护者。自周以来,中国文化传统中特别强调的“亲亲”也依附于此。完全可以想见,如果我们从“生命的延续”着手来理解人的生命,我们就会描绘出与海德格尔非常不一样的生存论结构。

在“生生不息”的生存论结构中,任何一个“此在”的存在都不是孤立的。个体的生存并不是茫然“被抛”,处于“无家”的状态,而是人人皆由父母所生。中国文化传统中特别强调“身体发肤,受之父母”(《孝经•开宗明义》),或者“身也者,父母之遗体也”(《礼记 • 祭义》) 就是体现了生存论上的连续性,甚至在中国文字中指称自我的“身”,其本义即来自于身怀有孕的意思,隆起的腹部,表示腹内有子。这源始性地表明对于自我的理解并不是个体性的,个体的持存不是个体意志的自由冲撞和个体欲望的充分满足,而是从一开始就持存于某种“家”之中。从这个意义上讲,海德格尔“此在”之“此”(Da)源初就具有强烈的归属感,这个“此”是有安放之地的。美国学者罗思文(HenryRosemont)和安乐哲(Roger Ames)在翻译《孝经》时,特别指出了“家”之于“人”的本真性地位:“我们几乎每个人的生活无论好坏都发生于特定的家庭环境内。……人性的基本单位就是出于其家的那个人,而非分离的个人或同样抽象的家庭概念。”“家”绝不是一种抽象的社会组织。

对于“此在”在世存在的理解,海德格尔借助“现身情态”,也就是“此在”通过此情此境的切身感受,向自身显现自身,这是海德格尔哲学非常高妙之处,一举突破了近代以来西方哲学对于哲学的认识论至上的取向,借助于“现身情态”把哲学拉向了生存论的层面。因此,海德格尔认为这些切身感受在存在论上并非无关紧要,而是“此在”生存论中最基本的环节。“此在”总有情绪,正是在情绪中,“此在”被带进它所处的世界。这里特别要注意的是,“情绪”是把“在世”作为整体展开的,而不只是人内部的心理活动。所以海德格尔要进行一系列“现身样式”的分析。海德格尔对于“怕”(fear)、“操心”(care)、“畏”等一系列著名的分析,都是为“此在”的生存论环节的展开而服务的。在海德格尔的描述中“畏”是最为著名的。这种“畏”不是对于某一种具体事务的焦虑,而是把“此在”从世界中拽回,投入到一种面对自身的“畏”,并由“畏”开发出“此在”本己的存在者。“畏”揭示了“此在”的生存被抛于世的焦虑,这是人面对难以把握的不确定性的体验,揭示了生存论上整体的焦虑;其实这又何尝不是一种“生命之流断裂”的焦虑呢。

相反,当对生命的延续有一种通达的领会之后,海德格尔所谓的“现身情态”中还会流露出一种生命的“乐”(enjoyment),也就是在生存的过程中处于一种“乐”的情态之中。在愚公这里,“个体”在茫茫宇宙中有其自身的来源,也深知自己的最终去向,一切自然而然,因此愚公始终流露出一种乐观的情绪。这是“乐”在生存论上来源在于“达”。“达”作为通晓明白,与“畏”之焦虑正相反对。因此,“乐”也是参透天地万物“日往月来”之后的“达观”。孔子曾说:“天下何思何虑?天下同归而殊途,一致而百虑,天下何思何虑?日往则月来,月往则日来,日月相推而明生焉;寒往则暑来,暑往则寒来,暑寒相推而岁成焉。”孔子的“何思何虑”透露出天地变易中的一种乐观。面对天地日月相推,寒暑往来,生命则自强不息,这正是中国文化传统生存论上“现身情态”“乐观”的存在论基础,所以才会孔子有“未知生,焉知死”的淡定,才会有孔子传递出的“生命之乐”的态度。

在中国哲学中“乐”是非常重要的概念,所谓“一箪食,一瓢饮,在陋巷,人不堪其忧,回也不改其乐”(《论语•雍也》), 可见此“乐”并不和我们日常生活中的贫富和享受有关。那么“所乐何事”呢?宋明理学对此曾有很多讨论,并非常敏锐地把握了“孔颜之乐”的重要性。借用海德格尔的说法,孔颜之乐并不是针对具体存在物的“乐”;也不是“得道”之后的“乐”,程颐曾说,“使颜子以为道为可乐而乐乎,则非颜子矣”。这里的“乐”乃是对于生存本身有一种根本满足,按朱熹的解释,颜回“元自有个乐” ,“元”字表达了生存论上的根基,也表达了在生存论上的源初性 。这种“乐”是源自对饱满生命的根本的“满足”,是对生命生生不息的体悟之后的基本情绪,而不是对生存不确定性的恐惧。经常有学者指出中国文化传统中的“乐”包涵了超越性的宗教维度,也就是这个意思。

中国文化传统中对于生活的“乐观”态度,为许多智者所体察。梁漱溟说仁者的生活就是一团和气,这种和乐之心是从生命深处发出的 。同时这也体现出一种生活态度,因此,在生存的“日用动静之间”皆有乐。林语堂则认为:“人生之目的并非存于死亡以后的生命。因为像基督教所教训的理想谓:人类为牺牲而生存这种思想是不可思议的:……人身真正的目的,中国人用一种单纯而显明的态度决定了,它存在于乐天知命以享受朴素的生活。”这是在温暖的存在之家中体会到的对生命的满足、感激与欢愉,李泽厚更将此总结为中国的一种“乐感文化”说 。通过我们分析,可以看到,这种“乐感”并不只是对于生活的一种态度,而是有着生存论上的深层基础。

三  生存论中的伦理指向

在海德格尔这种完全以与最本己的联系为本真的生存中,他提出了“良知”(conscience)的概念,但是他的“良知”,只是空泛的呼唤,他强调这种良知的呼唤是任何一种可能的伦理思想的生存论基础;但是我们知道,海德格尔生存论思想中的伦理问题一直是一个备受关注的问题,这在海德格尔那里几乎是一个无解的难题。因为他对于“此在”的理解完全是个体性的,“此在”在日常生活中与他人的相互共在被解释成一种非本真生活。这完全是因为他从“生命的断裂”,从个体来理解“此在”的生存所导致的结果。由于在本真状态中,“此在”与他人绝缘,因此“良知”只能在“此在”之非本真的存在与本真的存在之间回荡,于是“此在”既是“被呼唤者”,又是“呼唤者”。但“良知”并不纯在于“此在”自身,此间海德格尔的解释充满了神秘主义式的想象空间。

“生生不息”的生存论结构则展现了中国文化传统下人对于伦理源起的根本性理解,这也是中国文化传统超越性维度的体现。延续生命、哺育后代是生命的自然本能,是人和动物共有的本能。那么人和动物的根本区别在哪里?孟子说:“人之所以异于禽兽者几希。”(《孟子•离娄下》)“几希”是什么?就是要逆动物本能而开发出的文化之根源。在西方文化传统中靠上帝来超越,来实现这个“几希”;在中国文化传统中则强调对于生命诞生的感恩,这主要表现在对父母祖先的“孝思”。《诗经》作为先民最早的自我理解,咏唱道:“父兮生我,母兮鞠我。拊我畜我,长我育我,顾我复我,出入腹我。欲报之德,昊天罔极。”(《诗经•小雅•蓼莪》)这是“孝思”最朴素最根本的表达,父母养育生命之恩如苍天般广大无边,子女报答父母亦是无穷无尽。“孝思”表达了对生命诞生和成长的感恩和报答;基于这种对于生命延续的敬畏,最大的不孝,就是绝先祖之祭祀。因此,孟子讲“不孝有三,无后为大”(《孟子•离娄上》)。这不是简单地表达对后嗣的重视,而是透露出“生生不息”的生存论要求。在这里,对生命诞生的感恩与对厥后子嗣的警示,前后完全是一致的。

在“生生不息”的生存论结构中,“孝思”以最切近的方式使“此在”能够跃出自己生命的个体性,而与最亲近的他者建立起“亲亲”的关系,这种关系包含了“父慈子孝,兄友弟恭”,其中最为关键的是“孝”。“孝”最直接地表达了对于生命的报答;而由此延伸出来的对“祖先”的追思与敬拜,则足以与西方文化传统中对于“上帝”的崇拜并驾齐驱。利玛窦之后的“礼仪之争”,人们惋惜于教皇的蛮横禁令,断了东西方思想的交流,其实无论教皇还是康熙都很明白各自的道理。在中国文化中,祖先配享上帝,这是天主教所绝对不能接受的。《礼记》说:“万物本乎天,人本乎祖,此所以配上帝也。郊之祭也,大报本反始也。”(《礼记•郊特牲》)这是殷周之际以来,中国文化传统建立起来的核心观念。因此,祖先不只是具有生物学、谱系学上的意义,更是一种价值形态,而且是个体超越自我的中枢。儒家很明白其中的联接,因此曾子说:“慎终追远,民德归厚。”(《论语•学而》)在追慕生命之渊源中,思考人生意义,超越自我,社会风尚也由此而敦厚起来。

道德最基本的标志就是要“利群”,就是要超越“个体”利益。在西方文化传统中,克服个人的自我中心和私欲需要上帝对于永恒生命的保障作为中介。耶稣说,最大的诫命是爱上帝,第二则是爱邻如己。这是耶稣对摩西律法的根本概括(《圣经•新约•马太福音》:22:36–40)。在中国古代也有墨家“兼爱”的思想,“爱人若爱其身”(《墨子•兼爱》),简直就是“爱邻如己”的直接翻译。但这种思想传统在中国文化中并没有流传下来,原因何在?中国人不知道“爱邻如己”好吗?知道好,为什么还受到儒家的激烈攻击?知道好,为什么不能在中国流传下来?如果我们拿它与基督教相比的话,基督教“爱邻如己”的背后,还有更大的诫命:“爱上帝”。很多神学家都曾指出,人类正是通过爱邻人来体现对上帝的爱;也就是说,我们超出自我中心的需要和利益,达到爱邻人的道德效果,其实是需要以上帝为前提的,是以上帝对永恒生命的保证为前提的。墨家的“兼爱”思想显然没有能够得到“上帝”的保障。康德的道德哲学想在自身之中建立“自主性”,但最后的福德圆满依然需要设定理念性的上帝。我对他人的爱,亦是对上帝的爱。这里,永恒生命的保障依然是面对“他者”时的根本出发点。要延续永恒生命必须信仰上帝,上帝则要求我爱邻人如己,这是基督教传统中道德发生的生存论机制:在“个人”永恒生命中,以上帝为前提,突破自我,成就道德。

不同于基督教的“爱邻如己”,也不同于墨家的“兼爱”,在中国文化传统“生生不息”的生存论图景中,天地之大德是“生”。因此,生命的存在首先表达的是对自身生命诞生的感激,由此衍生出来的“孝”的观念在中国文化传统的道德生活中具有奠基性的地位。过往人们在谈论“孝”时,常常过于拘泥于历史上的具体做法,也就是这种核心价值理念在当时的机制化形态和表达,如生养、死葬、祭祀、丁忧等方面的规定。究其根本,子女对于父母的“孝”体现的是对生命诞生的感恩,是对父母生养的回馈,是对天地之间“生生”大德的敬拜。“孝”的观念有着深厚的生存论基础,同时通过“孝”的机制,以最切己的方式把人从自我中心中超拔出来,以对生命感恩的方式来超越自我而热爱自己最亲近的人——父母。民间有“百善孝为先”的说法,马克斯•韦伯(Max Weber)也充分认识到这一点,他关于在中国文化传统中“孝是所有其他德性的源头” 的说法,也就是曾子“孝,德之始”的翻译。“孝”是中国文化传统下道德的起点和基础,以“仁爱”为核心的传统道德体系是在“孝”的生存论基础上提升、扩大而成的。以孝悌为基础,在孔子那里发展出了“仁爱”的概念,仁爱是一种比“亲亲”、“孝思”更广泛、更哲学化的表达。但孔子创发出来的“仁爱”体系,其根本仍然是“孝悌”,所以有子深深体悟到孔子仁爱思想的根本:“孝弟也者,其为仁之本也”(《论语•学而》)。孟子更为清晰地表达了这一点,认为“仁义礼智”都与“孝”有关:“仁之实,事亲是也。义之实,从兄是也。智之实,知斯二者弗去是也。礼之实,节文斯二者是也。”(《孟子•离娄上》)及至近代,梁启超也是通过“孝”来比拟群己关系的。

“仁爱”(benevolence)在近代西方道德哲学中也是一个关键的概念,当人们逐渐放弃从上帝身上寻找道德基础,转而从“个体”有限的生命中来寻求道德起源时,“仁爱”的概念便浮出了水面。很多哲学家如休谟(Hume)、卢梭(Rousseau)、亚当•斯密( Adam Smith)等人都讨论过这个概念;他们更加通过诉诸“同情心”作为“仁爱”的基础,并以此来理解如何突破自我问题。但是,他们也看到,人的“同情心”或者“仁爱”是有偏好的(partiality),也就是说,人们总是首先爱自己亲近的人,然后爱熟悉的人,对陌生人的爱则较弱,这就是儒家讲的“爱有等差”。这是一个生存论的事实,东西方哲人都看到了这种“爱有等差”的现象,这也是何以儒家拒斥“兼爱”的缘由。但是一个社会的存在确实需要有根本性的凝聚力,需要有普爱众生的价值观念。对此东西方哲人有完全不一样的解决方案。休谟因为“仁爱心”的偏好和等差,看到了它的局限性,尤其是在更广阔的社会中,“仁爱”之心就显示出其偏狭之处;于是他提出了“人为的德”(artificial virtue)的概念,通过以人人平等为前提的正义观念,作为补救性措施,来弥补仁爱的不足。走到这一步,从某种意义上讲,已经超出了道德和伦理的范畴而进入了政治和法律领域。

相比休谟通过“正义”来补救“仁爱”的等差性和局限性,中国文化传统更强调从最原初的孝爱之心发展成更广大的“仁爱”之心,最终发展至“仁者以天地万物为一体”的思想。“孝”是实现仁爱的第一步,通过“推己及人”的功夫,在教化中把这份“爱”扩大出去,达于“泛爱众”扩大到整个人类。余英时先生曾说:“以文化价值言,中国和西方都有最高的普遍原则,适用于一切个人。这在西方可以‘公平’(justice,一译‘正义’)为代表,在中国则是‘仁’的概念。” 其实造成中西分殊的“正义”和“仁爱”的源头面对的是共同的生存论现象,爱有等差,但为此解决的方案,却选择了不同的道路。在西方,通过人人平等的观念,以计算性的人为“正义”,如交换正义、分配正义来解决“仁爱”的等差问题,由此特别强调“法律”的作用;而中国文化传统强调仁爱,从“入则孝,出则弟”这个源头,用“推及”的方法克服“仁爱”最初的有限性和等差性,正是有了这种“源头”,才有了推己及人的原点,才能“老吾老以及人之老,幼吾幼以及人之幼”。“推及”是儒家思想中的一个重要概念,如何能够进一步“推及”则不是通过法律性的“正义”,而是主要通过“教化”来养成,所以《孝经》讲“教以孝,所以敬天下之为父者;教以悌,所以敬天下之为兄者”(《孝经•广至德》)。可见,基于“生生”之生存论基础的“孝”,并不只是对自己父母的感激和热爱,而是一种根本的道德发生机制,并通过“推及”的教化作用,形成一种普遍的泛爱众人、民胞物与的道德意识。这种结构的道德发生与生生之生命理解是完全一致的,按陈来教授的理解,最终朱熹认识到“仁即生生之体”。

在海德格尔的存在论哲学中,伦理学并不具有源发性的意义,“个体”式的“此在”同样完全不知道那个超越自我中心的基点在哪里,于是只能搬出“良知”的空洞召唤,一种把“此在”从非本真状态中唤回本真状态的声音。于是“良知”非但不是把自我从自身中“召唤”(call)出来面对他人,反而是把“此在”从“常人”中召回到自己面前。良知的呼唤似乎让我们面对本真的自我,但是并不能为我们提供一种实践上更好的方式。更为关键的是:究竟是谁来召唤的?海德格尔只能说:“呼声出于我又逾越我。”伦理问题始终是海德格尔哲学中的一大疑惑。其实,这正是需要我们帮助海德格尔来正视的,莱维纳斯藉着其犹太的传统,勇敢地宣示“伦理学是第一哲学”并以此对抗海德格尔的存在哲学,以面对他者来昭示伦理的源起。在“生生不息”的生存论结构中,“孝”之为根本的伦理经验,为伦理学之为第一哲学的阐发提供了中国文化传统下的另一种阐释可能性。我们需要借助自身的文化传统,审视和批判海德格尔哲学。当我们心中没有自己文化传统的“本体”时,面对西方思想,面对现代性思想,或者是面对海德格尔哲学时就会产生那么多的“障碍”来阻挡自己的生命体认。当我们胸中充溢着对自身生命价值的理解和肯定时,海德格尔或别的西方哲学家的种种理论死角就会暴露在阳光之下。

“生生不息”的生存论结构,对于政治方面也有深沉的影响。近些年国内学界关于“亲亲相隐”的争论充分反映了这一点 。基于“生生”之生存论结构的“亲亲”观念,是否与现代政治相冲突?是否与现代公民观念相背离?我们首先来看一下现代的政治图景。在霍布斯(Hobbes)所确立的现代政治哲学中,个体有着终极的意义,为了保全生命,为了免于个体自由冲突中的暴死,于是决定让渡部分个体的自然权利,通过契约方式来建立国家以实现保护个体的目标。因此国家本质上是为了保护个体而建立起来的。在“臣服—保护”的机制中,隐含的内在逻辑是保护“个体生命”。所以,霍布斯得出结论,在国家的法庭上,“个体”没有义务来揭发自己,而是有权保持沉默。这是西方法律中被告有权利在法庭上保持沉默在哲学上的依据。其实这个问题直接与“亲亲相隐”有关,保持沉默可以理解为一种“自隐”。在中国文化传统中,“父子天伦”确实是道德教化的重要基石,孔子特别设计了非常巧妙“隐”的机制来应对法庭上这种极端情况的挑战。根据传统的论说,人们当然可以搬出各种理据,引经据典为“亲亲相隐”辩护。我们这里则从一种新的视角来理解这个问题,如果我们确定的生命的基本单位并不只是“个体”,而是着眼于“生命的连续体”,那么,霍布斯对于个体“沉默”(自隐)的论证,同样可以顺理成章地演变为“亲亲相隐”的论证。既然基于最初进入政治社会的动机,人们没有义务向国家揭发自己的罪行;那么作为一种“生命连续体”,人们同样没有理由揭发自己的亲人。由此可以,基于“生生”之生存论基础的“亲亲相隐”的问题,在现代政治的结构中并没有违逆之处。事实上,这在法律上也并不是一项难以操作的工作,在台湾,法律就规定父母子女及近亲之间,有权在法庭上拒绝作证。这是现代社会依循中国文化传统生存论结构的一大例证。

四 “学”与“教”的生存—历史意涵

在“生生不息”的生存论结构中,父母与子女有着一种深刻的生命体关联,这需要得到生存论上的阐发。这在西方文化传统中原本就是一个比较薄弱的环节,在强化“个体”的现代性思想中更被弱化了。在洛克(John Locke)看来,子女是上帝的产物,父母对子女的抚养只是对上帝尽到了责任,父母与子女的关系只是一种法律上的契约关系。本质上,他们之间是相互独立的个体。一种自然的“生命连续”活生生被上帝意识和法律意识所切断。在海德格尔的论述中,这种生命的延续性也难以融贯进他的体系,他的生存论分析丝毫没有对“生命延续”的理解,“世代”在生存论的分析中没有任何的位置。

父母与子女的关系除了“孝亲”这个环节之外,基于“生生不息”的生存论结构,中国文化传统特别重视“学”与“教”。对于“生命连续体”的重视不只是表现在生物意义上的生命物种的延续,更成为基于这种生存论的文化模式。“学”与“教”在中国文化传统中的重要地位,正体现了这一点。《论语》首篇即为《学而》,孔子论学的内容非常丰富。所谓“学”,按朱熹在《论语集注》中的说法,“学之为言效”,“效”即是“学”的最初本义。那么所效为何?“效”必有效仿的对象,在“生生不息”之生命长河中,通过“学”和“效”把前人的生命经验拽入了当下的生命之中。“‘学’意味着获得和拥有先辈们投注到文化传统中的意义。这样,一个社群就会拥有一个所有成员以此为基础进行彼此影响、沟通和交流的共同世界。”所以“学”就是对前人经验的效仿和学习,在孔子那个年代更强调对于古代文化传统的学习,这也是孔子讲“博学于文”的原因。在海德格尔描述的图景中,“此在”孤零零的生存于世,“此在”所有的好奇心却也常常使他沉沦于常人世界。海德格尔也曾花大量篇幅论述“曾在”(having been),“曾在此的此在”,及其对于特定世界的影响;但是在他那里,通过“学”这个微观的生存论机制,“曾在”的“此在”如何作用于当下的世界却没有任何的论述,“学”并没有任何生存论的地位。

不仅是“学”,与“学”相联系的是“教”。“教”与“孝”是同源词,就“教”字本身而言,直接与“孝”字相关。《说文》将“教”定义为“上所施,下所效也”。由此可以看出,“教”亦扎根于“生生不息”的生存论结构。透过“学”与“教”,生存论上的世代经验得以在这个机制中得以传承,个体不是完全无依无靠的,不是绝对自由的,他人的存在经验源始地以“学”和“教”的方式传递出来,成为了面对生存进行选择的倚靠。下一世代的成长是与上一代的施教息息相关的。孔子对自己的评价即是“学不厌,教不倦”(《孟子•公孙丑上》)。在中国文化传统中,这个“教”首先而直接地与“孝”相关。《孝经》一开始就讲“夫孝,德之本也,教之所由生也”(《孝经•开宗明义》)。孟子也讲,必须教民以孝悌为先:“谨痒序之教,申之以孝悌为义。”(《孟子•梁惠王上》)可以说,在“教”最为广泛的意义展开之前,“教”最直接的内容就是亲情的推展,一如我们前面讲的亲情是自然的,但“推及”不是自然的,是“教化”的,“教的过程就会通过家庭内部和国家中的模范和效仿的过程而得到最有效的实现”。

在海德格尔的论述中,“此在”之在世的结构中,“筹划”是一个重要的概念。“此在”作为“此在”一向已经对自己的生存有所筹划,更有一种本真的向死存在的生存论筹划。但无论本真的抑或非本真的筹划,筹划都是“此在”的“筹划”。在中国文化传统中,通过“学”与“教”,使代际之间的筹划成为可能。“学”与“教”亦是生命“筹划”的延展,从个体“此在”的“筹划”延展到“生命之流”的“筹划”。在海德格尔“向死存在”的生存论结构中,很重要的一环即是“筹划”,“此在”在各种生存的可能性中筹划自己,当中尽管包涵着对于与自身生存相关的世界的把握,但主要是自我的筹划。在中国文化传统下生存论的“筹划”,就不只是“此在”在此世的筹划,更有一种生命连续体的“筹划”,这更主要地体现在“教”之上。“教”的内容不仅仅是“孝”,也是与生命连续体中的父母子女有关,所以在蒙学《三字经》中,首先就讲“养不教,父之过”,这在中国文化传统的生存论经验中尤其明显。

海德格尔式的“筹划”只能是个体式的谋划,只能是在有限生命中的筹划。可是,基于中国文化传统下的生存论结构中,生命的筹划就会有别样的风采。美国驻中国大使骆家辉(Gary Faye Locke)的故事常常被津津乐道,这可以被解读为美国式个人奋斗的实现;但在中国的生存经验中,他的故事并不只是从他开始的,而是当他爷爷移民到美国,在富人家做仆人努力学英语时就已经开始了。海德格尔虽然强调“此在”是“被抛的筹划”,其实“筹划”是有限制的,这种限制可以从更为积极的层面来理解。中国文化传统对于“教”的重视,就是“此在”“筹划”的一个先在条件。“教”首先在于如何成人,而不仅仅是知识,这也是“生生不息”的生存论机制的一个核心环节。在骆家辉的例子中,并不是说他的成就是他爷爷“筹划”的,但在这种文化传统下,个体的筹划行为并不只是本己的,“家”是最为切近的与他者的共在,事实上,只有通过他人,“此在”才能在自身的筹划中找到某种确定的方式,这种依赖关系决定了“此在”的“筹划”不是独立的。因此,在“生生”的生存论结构中,天然包含着“学”与“教”的位置,这其中正体现了“上一代”对于特殊的他者“下一代”的关怀。这是一个结构性的关系,因此体现教学的“师”,在中国文化传统中占据着特殊的地位,“师”总是与“父”联系在一起的,民间的牌位也总是以“天地君亲师”列在一起。“师”作为“教”的承载者、“学”的榜样,对于生命的延续、文化的延续有着极为重要的作用。这并不是指具体“教”的内容,而是指“教”与“学”应该成为生存论结构上的一环。当我们不只是从“生命的断裂”,而且也是从“生命的延续”来理解生存论的结构时,“教”与“学”作为一种指向未来的“筹划”,是能够与海德格尔式的论述相关联的。

“生生不息”的生存论结构,通过“孝亲”连接生命意识,通过“教—学”形成文化传承,因此在文化上会演化为“为承继祖宗遗志而求文化之保存与延续,以实现文化历史之悠久”的历史文化现象,由此而衍生出中国文化传统中强烈的文化传承意识。历史并不仅仅是过去了的事实,更通过“学”与“教”直接进入当下;“生生”之生存论的指向,不仅仅指向过去,亦指向未来。由追思祖先,寄希望于子孙万代而形成的对无限生命的理解,形成了中国文化传统中“上通千古下通万世”的文化意识,向前追慕,向后传承。《诗经》说:“孝子不匮,永锡尔类。”(《诗经•大雅•既醉》)承继生命的孝子,重视教—学的传统,是文化不朽不灭的根由。

按照卢梭的看法,人的特性在于其可完善性,且不仅仅是在“个体”生命上的完善,更是在类的层面上的完善。但卢梭并没有论述下去,何以人作为一种类可以推进人类视野的完善性。《中庸》结合“孝”的观念给出了答案:“夫孝者,善继人之志,善述人之事者也。”(《中庸》第18章) 甚至这是一种“达孝”。“达孝”不仅仅是对生命的感恩,《孝经》说:“立身行道,扬名于后世,以显父母,孝之终。”(《孝经•开宗明义》)继承先人的志向,努力完成之,才是真正的“达孝”。什么人能做到这样的“达孝”呢?历史上会以武王和周公为榜样,因为他们能够继承文王的志向,完成文王未完成的事业。也就是说,最终个体的成就,也并不仅仅体现在个体价值的实现,而是归于“生生不息”的历史洪流中,这也是“孝”极为关键的一种品质,直接关乎历史的发展,并指向未来。确实,这种延续进步的观念在现代特别深入人心。但基督教原型下的西方文化传统,会为这样一种进步的历史预设某种终极目标,或是末日审判,或是历史终结,或是共产主义社会,以此来表达这种最终的完善性。在中国独特的生存论结构中,虽然有着这样一种文明完善化在微观上的动力机制,但完善却是没有尽头的。与“生生”之生命的“无限性”相应的,则是《周易》六十四卦中最后两卦:“既济”与“未济”。这里最后一卦居然不是类似于黑格尔“绝对”的“既济”,万事皆成,最终实现人类的完善性;而是事情尚未成功,仍须重新开始。“未济”作为最后一卦意味深长:它既蕴含了生生不息的基本意涵,也在生存论结构上指出了事物继续发展的结构,以及“可济”之种种的可能性。

事实上,“生生不息”的生存论结构与“此在”个体的生存并不矛盾。对于“亲亲”文化传统的尊重,并不妨碍我们珍视生命连续体中的每一个节点,重视每一个个体的价值。作为在生命连续体中结出的硕果,每一个生命体都弥足珍贵。“生命之流”可以包容每一个节点,也即每一个“个体”;这种生存论结构也并不试图否定“此在”“向死存在”的生存论结构中所体现出来的“自由”。但是,“个体”化的生存论结构很容易失却“生命延续”的维度,而“生命延续”的维度恰恰是阐释人类“责任”源起的生存论基础。海德格尔哲学缺失伦理的向度,单纯着眼于个体自我,是远远不够的。“生命延续”所展开的维度,融不进海德格尔式的生存论结构;但反过来却是可以的,完全可以在“生命延续”的结构中包涵每一个“个体”此在的存在。作为人类文明不同的代表,西方文化与中国文化从不同侧面来阐释了各自的生存论结构。生存论的结构揭示人类文明智慧多样性的机制,也显露了他们各自提供的伦理源发的原理。

\newpage

\chapter{卢麒元资本论讲座 2021}

出处: https://www.notion.so/35357ab38a0042bcb5e0bfd762d4d495?

\begin{quotation}
    共产主义是信仰,它的源泉源于基督教信义宗,只不过它被一个非常伟大的基督教信义宗的信徒将它进行了某种异化。将信义宗作为共产主义或者是马克思主义产生的基本的土壤,再将信义宗与中国的儒家和孟子结合起来。

    我们都认为只有伟大的导师有伟大的理论指导下,才会有伟大的实践。但你纵观人类五千年文明史,伟大的实践多数不是由伟大的理论来指导的;它是人类、人类智者对现实的一种深刻的体悟来作出的抉择,这种深刻的体悟,有的时候并非由理论家完成。

    同样在六十年代在读《政治经济学教科书》,读的结果,最后由第一次实验——文革,不成,失败之后,很快展开第二次实验,就是1978年、1979年我们开始的全面的改革。这个改革的,“东马”的改革出现了一个什么情况呢?就是我们已经敏锐地比苏联提前二十年意识到国家资本主义不行,国家资本主义有问题。就是那个时候国家资本主义已经出现了官僚垄断资本主义的特征。毛泽东一早就看出来国家资本主义一定会走向官僚垄断资本主义。他为什么说:“走资派就在党内,走资派还在走”;他说:“他们不是胎生,不是卵生,是化生。”毛泽东的历史洞见是非常深刻的,而他这个深刻的洞见,其实我们今天来看,党内并非不认同他的洞见,而是不认同他解决问题的方法。
\end{quotation}

\newpage

\section{两次革命,两位宰相 、对当下经济的看法}

大家好,今天是2021年的4月17号,三月初六,我试一下麦。今天我们正式开启《资本论》的课程,今天是第一讲:两次革命,两位宰相。如果还有时间富余,谈一下子对经济的看法,还有一些其它的杂事聊几句。这两天身体状况不怎么好,我尽可能的咬牙把事情做完。

首先得感谢一下子整理资料的朋友们,这个时间过去的非常快,一直以来大家都在默默地奉献,其实我内心深处是非常感激的,有的时候也不知道该怎么表达这份感激,将来合适的时候,我们用合适的方式,来做一点点表达。当然我说的合适的方式,可能是希望将来有机会,我们共同的做一些事情,做一些事情。今天是2021年的4月17号,农历是三月初六,阳春三月。

这周我接种了那个科兴的疫苗,接种了科兴疫苗以后,没来得及休息,这个,因为这段时间是比较疲倦的,比较累,身体状况出现点小的问题。我尽可能地把这个课程咬牙、先把它挺过来。今天我们开始《资本论》的第一讲,其实我讲《资本论》的压力是蛮大的。因为我知道,一千个人就有一千种《资本论》的理解,我未必是那个最正确的。

如果说,我讲《资本论》的目的是什么?那么就是:我争取能够打开一扇窗,让大家站在我的肩上,在更广域或者是更高的一个位置上,去看一部书、一个人和一段历史。因为我知道这堂课,《资本论》这个课,可能对每一个人都具有极为重要的意义,因为它是已经超越了一般意义的投资的范畴,它可能会改变一些朋友的人生观和世界观,或者是重建价值观。

另外,最近这段时间变化比较大,风云突变。我一直说辛丑年是一个非常凶险的这样的一个年。最近有朋友也让我看《推背图》\footnote{《推背图》是中国古代著名预言书,是唐代李淳风、袁天罡合著。由于历朝历代均严禁此类谶书,该书在流传的过程中又不断被人篡改,将已知的历史改成图谶,加以比附,故其本来面目已渺不可考。现存《推背图》有六种不同版本,且内容各不相同,互相冲突。}的第三十八象,他们也有一些说法。我们不主张搞封建迷信,但是,历史可能会总是这样的巧合(有些契合),所以可能会有一些事情发生。我们今天是正式课,但我们还是会腾出一点点时间来谈一下对经济的看法,对一些事物的理解。

好,我们先言归正传,来谈《资本论》。在进入《资本论》之前,我们要做一些热身,热身大概得需要三堂课到四堂课。这第一堂课是谈历史,因为我必须把大家带回到十九世纪的欧洲去,带回到那段历史去。但你知道就算是你已经读完《企鹅欧洲史》,其实可能对历史的理解,仍然是一种浮光掠影的,不是那么真切的一种理解。可是你不理解那段历史,不能还原那段历史,其实你根本就没办法理解马克思,更没办法读懂《资本论》。

回到历史,其实工作量是蛮大的,我们不想把这个事情搞得过于复杂,变成一种大学的课,其实大学的课也讲不了这么复杂。那么我们怎么来呢?我想把历史浓缩为两次革命和两位宰相。原来想说三次革命,后来想还是两次革命:一次是工业革命,这个事情发生在十八世纪,就是大概在1750年附近(1750年到1775年这段时间),它的代表的事情就是瓦特蒸汽机和珍妮纺纱机的出现。工业革命改造了整个的欧洲,也改造了人类历史的进程。

我们要讲的第二个革命,是法国大革命。如果说工业革命是从物理上、物质层面改变了欧洲,那么法国大革命,则是从精神层面、或者是哲学层面、或者是政治层面改变了整个欧洲。法国大革命应该是在1775年到……它正式爆发应该是在1789年,这个89年真是有意思(1789年),到了拿破仑当政,大概是在1799年前后。法国大革命改写了整个欧洲的政治的历史进程,当然了,它也是一次重要的思想解放运动,这两次革命,

这两次革命,奠定了欧洲的政治、经济的基本生态,特别是深刻地影响了普鲁士----就是德国。在这样的一个时候,马克思应运而生。马克思和《资本论》和整个的社会主义学说和共产主义学说,不是天上掉下来的,它是在那样一片土壤上面,慢慢地形成。它的整个的形成过程,其实是由一个叫马克思的人,来演绎了整个的历史进程。即便没有马克思,也一定会走向这样一个过程。只不过是上帝,有时候选择是……

今天,还要讲两位宰相:一个是梅特涅,梅特涅是奥地利(奥地利帝国)的宰相;一个是俾斯麦,其中俾斯麦跟马克思算是同龄人。梅特涅是对少年或者是青年马克思,有极深刻地影响。而且这两个人恰好是那个时代,从工业革命、法国大革命,一直到后来的德意志帝国崛起,整个历史进程中的两个当事人,它们从另外一个角度让我们来认识马克思和他的《资本论》。

在规划整个课程的时候,我觉得两次革命和两位宰相是不能越过的,它可能让我们打开一个视角。在工业革命和法国大革命这个期间,发生了另外一件大事,就是美国的独立战争。这个事情与马克思《资本论》没有那么密切的联系,所以我们暂时先放一边,先不说它。但是,我要说的是:美国的独立战争与工业革命和法国大革命,法国大革命是在独立战争之后,与工业革命有着密切的联系。

我在给香港的学生讲课的时候,我说过这样一段话,我说:当你们决定走向街头的时候,我希望你们仔细阅读三个宣言:一个是美国的《独立宣言》。你们搞港独、搞台独,连个《独立宣言》都没读过,不合适吧?第二件事情,我说你们应该去读一下子《人权宣言》,法国大革命中的《人权宣言》;第三个宣言是马克思起草的《共产党宣言》。我说三个宣言你们都没读过,你们在这说什么呢?你们既不知道历史来路,你们也不知道历史的去处。如此混沌的状况下……

能做的事情就变得非常有限了。说他们的时候,其实我在想,我在回望我们自己的国家。其实,我们对历史的梳理,有的时候也是不到位的。我数年前,在读俾斯麦的《思考与回忆》。德国人,有两部回忆录写得很好,一个是《思考与回忆》,这是俾斯麦写的;还有一个希特勒《我的奋斗》。其中《思考与回忆》是一部非常有价值的书,它是俾斯麦的回忆录,他写的不好。显然这个俾斯麦在处理文字方面的这个功夫还是差一些。

然而非常重要的是,它是在德国崛起过程中,对德意志民族国家的一种深刻地思考。它其中的一些的想法,对今天的中国有着无与伦比的意义。老实说,在研究德国历史的时候,俾斯麦和马克思是两个极端。如果站在国家的角度来考虑,俾斯麦无疑更具价值。但是站在人类、站在人类文明、站在哲学史的角度考虑,马克思是无比的辉煌的,无比辉煌的。但,历史是这样的构成的,有的时候我们不能执着于一端。可以这样讲:没有俾斯麦就没有德意志的现代化。

好,先回到今天的第一个部分,就是工业革命。工业革命爆发在英国的中部。之所以在英国爆发工业革命,它是由一些条件的,它有先天的条件、自然的条件,也有哲学上的条件。我一直在说英国这个国家确实是很幸运啊,他三百年没有内战或者内乱,八百年没有被外侵。他是一块相对和平宁静的土地,他不像欧洲风起云涌,他相对和平宁静。和平宁静,他就可以完成原始积累或者是资本……

英国,特别是英格兰,在当时,在十八世纪中叶——就是1750年的前后,已经变得非常的富裕了,非常富庶,而且他的教育程度是非常高的。不仅仅如此,他在十八世纪中叶,英国的土地上也成长出了一大批的学者,思考者,既有对大自然的探寻的思考者,也有对人文的探寻的思考者。要知道光荣革命是爆发在英国,要知道第一个实现君主立宪的也是英国。

我们注意到,我刚才说了,他三百年没有内乱、内战,八百年没有被外侵,同时,他是最早的实现了政治上相对包容、宽松的这样一个政治环境。就是在光荣革命之后,在后来克伦威尔之后,他们完成了君主立宪,建立了,初步建立了一个可以包容资产阶级的那样一个现代的政治结构。同时,他们在学术上又非常宽容,另外参与了整个的航海的过程,对英国的工业化其实有的强烈的需求。在这个大的前提条件下……

在一个良好的政治、经济、自然环境下,英国爆发了工业革命,它的标志是瓦特的蒸汽机和珍妮纺纱机,标志是这样。在讨论工业革命的时候,其实里边包含了自然科学的迅速的演进。因为,当我们开始进入机械化时代的时候,其实钢铁、化工就要随之而产生了。那么为什么工业革命对《资本论》这么重要呢?就是因为你没有工业革命,其实就谈不上“资本”这两个字啊。你说有没有资本?有。但只有大工厂的出现,托拉斯的出现,资本……

所以要我说起来,资本产生或者是资本成为一种状况,它是由工业革命来创造的。工业革命产生了三样东西,它在创造资本家、资产阶级的同时,创造了另外一个伟大的阶级,就叫做无产阶级。首先它产生了一个新的阶级,它是旧有的封建结构被工业革命破解了,破解之后形成了新的结构。这个社会结构在工业革命和法国大革命一直发展中,它在整个社会的结构变动中,出现了一系列新的情况、状况,所以,它需要思考。

另外,工业革命也诞生了,也导致了国家,其实国家这个概念在工业革命之前没有那么强烈。就是民族国家成为了一种至少在欧洲、在美洲成为了一种非常迫切的现实。为什么工业革命会导致民族国家呢?工业化大生产为什么会跟政治上的这种国家的结构的建立有着某种的必然联系呢?这就要需要谈到整体的资源配置的问题,就是工业化过程中需要的资源配置和资源的整合,它需要一个……

它需要一种政治的载体。我们将来在讨论《资本论》的时候,我们会讨论到《资本论》的第三卷,就是马克思在谈完剩余价值谈完资本流转之后,其实要准备谈的就是国家与资本的关系。后来在这个问题上,马克思遇到了障碍,他没有写完。没有写完以后,后来后人恩格斯他们把它整理出来的《资本论》第三卷,甚至第四卷其实应该算是资料。它还是在谈资本流转问题,涉及到国家与资本的关系,其实马克思没有做完。我一直在说,真正的《资本论》第三卷是列宁写的《国家与革命》,但这个《资本论》第三卷它也仍然是有遗憾的,因为……

因为国家资本主义毕竟有它的工业化过程的好处,但是它也带来了一系列的问题。苏联的解体,中国在从文革到改革的这样的历史进程中,实际上都遭遇了国家资本主义的一系列的问题和解决。《资本论》的第四卷应该是由中国人来写的,这个在讨论完了国家资本主义之后,我们要讨论一种混合的资本主义,或者是我们管它叫社会主义也行,其实一种混合的资本主义,它比国家资本主义更高形态。其实在经历疫情之后,其实我们对这个事情的看法开始出来了,就是没有国家资本主义,纯粹的社会资本主义发展到极致变成金融资本主义。

当年马克思在《资本论》里边揭示出来的那些规律,在今天我们仍然可以看到。就是在社会资本主义没有国家资本主义混合的情况下,会走向另外一个极端,我们看到了美国的情况。欧洲的情况略有不同,因为欧洲带有浓重的国家资本主义色彩,只是五眼国家相对而言更偏向一种社会资本主义,或者是我管它叫社会资本主义的高级阶段,或者是金融资本主义。这跟马克思的概述、跟列宁的概述有相似之处。实际上它不但处理不了疫情,它也解决不了很多很多的复杂的民族、阶级和一些社会问题。

工业革命它带来的直接的后果就是它使得,除了产生阶级之外,它使得以工业革命为基础而迅速完成工业化的国家进入了现代文明,我们管它叫现代性,也有人管它叫现代化。这个现代化里边包含了双重的含义,一个是物理意义上的或者是物质意义上,一种是精神层面的。在精神层面上的现代性就包括了法国大革命前的启蒙运动,也包括了中国五四前后的新文化运动,它都是一种大革命之前的一种文化上的……

我们在讲《资本论》的时候,第一块,我们今天是要简述一下子两个革命,工业革命和法国大革命;然后简述两位宰相,一个是梅特涅,一个是俾斯麦。用一堂课讲这么多内容其实有点过分,每一个内容都应该可以讲四堂课。我们用一堂课讲四个内容是个什么意思呢?就是我们没有办法把这个课开三年,我打算二十四堂课把它弄完。那么我们在一个课里压缩完了以后,剩下的工作交给诸位,就是你们。我点到了,你们还要回去再做做功课,将来你们自己去欧洲旅行也好,带着孩子去旅行也好,你还是要把工业革命和法国大革命和这两个宰相搞清楚的。

在讨论工业革命的时候,我想就是工业革命的内容、工业革命爆发的原因、工业革命的意义我都不说了,有空大家自己去把它梳理一下子。我是在想工业革命它的正面意义我们是清楚的,它带来了现代化或者现代性,但工业革命的负面意义我们的思考是不足够的,它带来了一些的问题。显而易见,工业革命导致了城市化或者是城镇化,主要是城市化,导致了人口的聚集。它对人类的意义或者是对我们的意义是怎样的?工业革命的极限、发展的极限是怎样的?

在第一次工业革命之后,后来又爆发了其它的被人类总结的第二次、第三次,有很多人把电气化当成另外一次,把信息化当成另外一次。可能未来发生的是生物科技方面的革命,它对人类的正面意义是什么?它对人类的负面意义是什么?作为一个国家应该对这样的事情做怎样的思考?我们应该从物质层面的理解,我们对现代性或者现代化的理解,到精神层面的理解、到哲学高度上的理解,我们做综合,我们考虑的是极限和边际。在欧洲的好多哲学家……

他们也在进行着同样的思考。我们在讨论《资本论》的时候,《资本论》是反对现代化的吗?是反对现代性的吗?《资本论》是一些我国的学者或者是一些西方的学者、反马克思的学者所理解的那样吗?如果我们对现代性和现代化的理解上面存在着某种异议,那么《资本论》可以看成在十九世纪中叶马克思对工业革命、对现代性的一次伟大的历史性的批判。他的深刻思考有助于我们重新回到那段历史,从那段……

从那段历史出发,我们再来看今天的世界、今天的中国以及我们将要走的路,它的意义是非常深远的。工业革命我就说这么几句,我就不展开了,因为要介绍工业革命的内容稍微多了一些,我就不展开说那么多东西。我说几句法国的大革命,法国爆发大革命,它是有它的原因的,当然啦,有它的历史背景、有它的原因。说来也真的是有趣,几乎每一场革命都是缘于税政。

我说过,光荣革命是二十八男爵带剑议政,议政议的不是别的政是税政。就是男爵,就是土豪,就是二十八个类似于像中国的那种、类似于像中国的那种割据一方的军阀,二十八个军阀商量给英国王室多少税收,就是大家带剑议政,形成了君主和地方豪强的这样的一个关系,这样一个博弈的关系,形成了现代政治的典范。就是为了讨论一下交多少税就够你用了,就是这个税你只能建立皇家陆军,但是……

说错了。这个税,你只能建立皇家海军,你去跟法国人打,去跟外边人打。陆军,就是我们这些军阀手上。所以你今天英国的陆军仍然是以地方命名的,什么牛津旅啊、什么这个威尔士旅啊什么的,它是地方武装。这样呢,你的海军,我的陆军互相制约着,谁也不能欺负谁,大家有事坐下来讨论。这个议政长期化就变成了议会。议会后来觉得皇上还是不靠谱,再选个宰相、首相,替你管理打理日常事务,形成了君主立宪的这样的一个体制。法国大革命仍然是因为议税,讨论税政而起。

顺便说一下子,美国的独立战争也是因为税政。美国的南北战争也是因为税政,就是通常大革命都是起于税政。通常伟大的时代的开启也是起于税政改革。好多人说你天天在那儿嚷嚷直接税,是的,我说直接税,其实就是在讨论中国的政治结构和经济结构的进步的方向。如果这件事情做好了,当然不需要革命,这件事做不好早晚还是要革命。不革命解决不了税政的问题。法国大革命就是1789年7月14号发生的。

将来研究经济,在讨论马克思的时候,我们会讨论罗马天主教会教廷的对谷物征收的什一税,会讨论法国当时,因为法国路易十五这个七年战争,打的法国差不多穷了;另外,就是他又参加了,又支持美国的独立战争。这样的话,到了路易十六的时候,法国的财政状况出现了严重的问题。这个你要是看当时的法国财政状况,你难免不会想到今天的美国。今天美国的财政状况很像路易十六,很像路易十六。只不过美国现在,美国国内很难理性的来讨论税政。

说来也有意思,1789年革命,革命的前一年,1788年法国爆发了有史上罕见的旱灾,大旱。然后出现了欠收。那个时候因为大量的法国的农民,有点儿跟英国工业革命的状况是一样的,你要记住这个时间是1788年,就是工业革命还在进行中,大量的农民在“羊吃人”的过程中,离开了他们赖以生存的乡村,进入到城市了,进入到城市了。食品短缺对城市平民,那个压力是可想而知的。

很多人在那个特定的历史时期,是衣不蔽体,食不果腹。我们去读这个当时的法国作家写的小说或者是一些著作,可以看得到那段历史的悲惨的景象。我去到巴黎,晚上睡不着觉,我就遛弯儿。在协和广场,我一直在寻找砍路易十六脑袋的那个位置。我是夏天去的,当我大概确定那个位置之后,我感到背后一丝一丝的寒意,我甚至能看到愤怒的人群,以及与当时的法国的贵族……

法国当时,有点儿像现在的印度的这个种姓制度,它分为三级,教士那是最高等级的,贵族、这个资产阶级和平民,还有就是农民,它分不同的等级。在工业革命之后,资产阶级迅速崛起,但是资产阶级崛起却无法撼动教士和贵族的特殊的权力和利益。所有的苛捐杂税都是针对资产阶级和城市劳动者的、城市平民的。当然农民也遭到了残酷的剥削和压榨,但是因为法国、德国更严重一些,他们还是庄园奴隶制的状况,庄园地主的状况。

相对而言,可能农民,还生活能够略为安定一些,但是城市平民真的是活不下去的,所以这个法国大革命就爆发了。我老觉得这个1789年这个有点意思,因为这个大革命,它爆发之前其实是有一些征兆的。后来托克维尔写了那本书,很多人在读那本书,那本书后来被歧山推荐过,希望读《旧制度与大革命》这本书。我读过这本书,我认为对问题的梳理有它独特的视角,但不是我认为的,那样的一种判断,所以我有不同的看法,在后边我们会慢慢讲。

讨论到法国大革命,其实托克维尔这本书,大家有空还是可以看一下的。今天谈了几本,谈了两本书。一个是俾斯麦的回忆录,一个是托克维尔的这个书,这个对理解法国大革命,对于理解《资本论》是有重要帮助的。另外,在革命之前,其实法国启蒙运动已经进行了很长时间了。这个法国的启蒙运动里边的一些属于哲学史或者是思想史上的一些大事,都是在这个特定历史时期出来的。其实在讨论中国现代化的时候,我们一直在思考就是中国的思想家在哪里?中国的思想家能够起到启蒙吗?

其实整个的现代性和启蒙运动是结合在一起的。当时法国的启蒙运动主要的代表人物是伏尔泰、孟德斯鸠等,其中像狄德罗、卢梭、蒲丰、孔狄亚克、杜尔哥、孔德赛等一大批的哲学家。在谈马克思的时候,我们要谈《资本论》形成的历史和思想史,思想史是下一堂课,我们要介绍一下子当时启蒙运动的一些重要的人物,以及他们对马克思的影响,以及马克思他的逻辑基础,或者是理论的源泉是哪里?我们会讨论到启蒙运动。

启蒙运动,我个人认为现在目前对启蒙运动,中国对启蒙运动的理解,有一些偏颇。就是启蒙运动它的源泉实际上是欧洲的宗教改革或者是文艺复兴,或者是源于中国哲学到西方以后产生了一系列的化学反应。欧洲人不特别的接受我们这个看法。但中国人在这方面的研究显然也是比较慢、比较落伍的。我个人认为《资本论》的重要的源泉。

除了启蒙运动过程中的一系列的思想家,包括康德、黑格尔在内的一系列思想家构成对马克思的深刻影响之外,很大一块是宗教意义的——也就是基督教信义宗。宗教改革,特别是文艺复兴,所谓的文艺复兴,它的对现代性的这种影响或者是它的这种缔造其实是非常深远的。在中国从封建社会走向现代社会这个整个的历史过程中,我们也经历了一定意义上的启蒙运动,虽然没有像欧洲那么灿烂的一系列的大事,我们也……

我要说的是中国的启蒙运动现在才刚刚到了中期,她这个启蒙运动显然远未完结,它仅仅是一个开启、开始。法国大革命基本上将法国的旧有的阶级社会彻底粉碎了,就是由教士、贵族和资产阶级和农民形成的这样的一个社会结构,被一场轰轰烈烈的革命打得粉碎。有的时候我们在思考法国的时候也会思考到英国,英国虽然没有经历这种革命,但是英国在自然的演进过程中也解决了容克地主这样的问题,反而是普鲁士这个事情……

德国的社会结构被彻底粉碎,不是靠自己内生的动能,而是靠两次世界大战完成了对容克地主,对旧有社会结构的彻底粉碎。包括教士、贵族、农庄、庄园主、容克地主以及资产阶级,他们这复杂的关系,法国是通过一场大革命来粉碎的,英国是通过内生的动力、动能来解决的,而德国是通过两次世界大战来完成的。日本也有意思,日本是通过美国占领军,也是第二次世界大战通过美国占领军完成这个政治结构的现代性,或者是现代化的。中国是革命。

中国的社会结构破解和重建是通过土地革命。在中国土地革命的过程中出现了毛泽东、中国共产党——正好今年是建党百年,这场革命破解了旧有的中国的社会结构,建立了崭新的结构,从而为中国的现代化或者是建立现代性奠定了基础。所以毛泽东的意义他不是一个简单的革命者,他是一种现代化或者现代性的缔造者。你必须站在这样的一个角度来理解中国共产党和中国革命,它才变得有意义,不然的话对今天的社会的理解就变得没有那么深刻。同时我们读《资本论》不仅仅是为了一种学习。

我们读《资本论》也是为了利用这样的一个学习的过程,重新审视社会、我们自己和我们国家的未来,这样的话这个读书才变得有意义、有价值。大革命带来了什么?大革命带来了一种崭新的思想,它不光是一个社会制度,它提出了三权分立、君主立宪,它将天赋人权(因为它有人权宣言)放到了极高的高度,它要求打碎阶级或者阶层,人人平等。它实际上是现代资本主义现代性的一次奠基礼。法国大革命的意义远远不止于法国。

关于大革命的起因、过程我就不讲那么多了,像吉伦特派、雅各宾派、热月党包括最后拿破仑的出现,我就不讲那么多了,这个留待大家自己来研究一下子法国大革命吧。你们有空还是要去法国旅行一下子,去找一找革命的遗迹,来理解法国现代化或者现代性过程中所经历的那血雨腥风、经历的苦难。一个文明的进化其实远没有那么浪漫,在看这个革命的时候,我注意到法国的三色旗。

法国的三色旗,蓝、白、红。蓝是民主;白是自由;(民主、自由和人权)红是人权,人权是要通过革命来获得的。蓝、白、红就让我想起了德国现在的三色旗黑、金、红,它(德国的三色旗)也是革命的结果。一系列的革命走到十九世纪的时候开始了,十九世纪的欧洲的迅速地崛起和带领人类进入到一个文明的更高的境界。

马克思是1818年5月5号出生,我的老师是这样教我的——马克思一巴掌一巴掌打得资产阶级呜呜地哭,就是1818年5月5号,这个好记。1818年马克思出生,在此之前,1815年欧洲历史上出现了一个庞大的国家联盟,叫德意志帝国的……可以翻译成德意志联邦吧,它包括了半个欧洲,全部在涵括之内,但它的管理是交给奥地利的。奥地利的总理或者是宰相梅特涅闪亮登场。

好!两个革命我们就讲这么多,这时间过得太快,我简单说几句梅特涅。梅特涅这个人物非常重要,梅特涅他是在工业革命时候出生的,他是1773年出生到1859年离世。这个很有趣,就是梅特涅在奥地利爆发了三月革命之后他辞职离开。而1848年之后另外一个人出现,闪亮登场,天降猛男,那个就是俾斯麦,梅特涅治下的整个的大德意志地区和俾斯麦治下的德意志帝国。

构成了以德国为中心的这样精彩纷呈的近代史,其实可能应该把它当成,还是应该是把它算成近代史,不能算当代史,这两个人物非常重要。其中梅特涅是马克思少年时期,甚至青年时期的一个重要的压迫者,所以他对马克思的思想的形成有重要的影响。因为他是当时实际上的治理者。有意思的是他也是德国人,他是出生于德国科布伦茨。

如何理解梅特涅呢?他更像中国的苏秦、张仪这样的人物,他可能更像张仪吧,他是这样的一个人物。他算是当时欧洲的纵横家吧。讲一段故事,其实梅特涅,年轻的时候也是靠他娶了这个谁呀,他的夫人是一个女伯爵,爱丽诺·考尼特斯女伯爵。应该这么讲,他嫁给这个女伯爵之后跻入奥地利的上层社会。

梅特涅的起身有点像于连\footnote{于连(Julien),法国著名作家司汤达的代表作《红与黑》中的男主角。},他是靠婚姻,靠这个女伯爵上位。这段故事我讲一讲,就是梅特涅跟拿破仑之间的一段故事。拿破仑摄政法国之后成为法国皇帝之后,他就一路攻伐。奥地利、德国都被拿破仑的大军所占领。有一次拿破仑在维也纳参加一个舞会,那个年代喜欢跳舞。女伯爵在舞会上,因为是化妆舞会。

女伯爵被一个法国军官邀请,个子很矮,这个长得这个身材也不是很好,舞步也很糟糕的一个法国军官邀请,就是女伯爵已经想休息了,这个军官来过来邀请女伯爵跳舞。当时的奥地利人对法国人是又爱又恨,但主要还是讨厌,她觉得是维也纳很厉害了嘛,她觉得这个土包子,但她又不好拒绝,所以她跟他跳舞。这个拿破仑跟爱丽诺一边跳舞,拿破仑就拉着她往房间走。当时女伯爵感到很愤怒、很惊讶,但又不便在那个地方发作。跟他进入到房间的时候,拿破仑摘下面具,吓了女伯爵一跳,女伯爵,这么粗俗的人,他竟然是拿破仑。

其实舞会上的人都在看这一幕,他们俩进了房间,但是他们进了房间没有干坏事啊,没有干别的事情。拿破仑只是求爱丽诺这个女伯爵一件私事,他想娶奥地利皇帝的妹妹做妻子,他需要一段婚姻。这个女伯爵很生气,觉得他很无礼,回去把这个事情告诉了梅特涅,当然梅特涅刚开始很愤怒,就是你把我媳妇儿拉房里干什么呀?但听到这个事情,梅特涅就会心地笑了。其实就是这段婚姻把拿破仑带入到另外一个状态。原本拿破仑是想娶俄国沙皇的妹妹被拒绝。

如果奥地利拒绝的话,那么奥法这场战争可能会替代法国拿破仑对俄国的那场战争,因为这段婚姻这个法国和奥地利成为了盟国,甚至在拿破仑大军去进攻俄国的时候,奥地利还提供了一种方便。但是奥地利人有梅特涅这种人在,他不会真心实意跟你结盟的,说三万大军配合拿破仑去攻打沙俄,但是梅特涅的三万大军根本就没出发,甚至等拿破仑兵败回来的时候,梅特涅带领奥地利与英法他们最后做掉了拿破仑。讲一段故事当然是真实的事情。

梅特涅在那个时代的欧洲是起到了一个令人惊讶的这样的一个作用。因为奥地利不是,奥匈帝国不是什么特别强大的帝国,它处在一个夹缝之中,然后在特定的历史时期发挥了如此重要的作用,不能小觑了梅特涅的才干和能力。我看了看今天这个梅特涅的事情,我也不展开了,我想说的是梅特涅是一个非常残酷的保守主义者,因为他治理奥地利和德意志的时候采取了极为残酷而且严厉的手段来维护奥匈帝国、普鲁士的教士和贵族的利益。就是当时英国和法国都在迅速走入现代化,但是由于他的存在,他没有能够让奥地利和普鲁士尽快地走入现代化。同时他以卓绝的外交手腕儿,这个在某种意义上维持了奥地利和德意志的阶段性的和平,这个阶段性的和平后来被打破,后来在普奥战争中,奥地利最后败北。

梅特涅就说这几句。这个课好像一堂课讲太多内容,有一些有些无法铺陈铺展开,我们说几句俾斯麦。俾斯麦比马克思早出生了三年,他是1815年4月1号出生于普鲁士勃兰登堡,读书跟马克思是前后脚,他也进了柏林大学,他也学的是哲学和法律,跟马克思学的东西也像,但他不是犹太人,他是传统的德国……

俾斯麦在大学期间读书期间,曾经27次与人决斗,你就可以知道他多么的好战、高大、威猛、勇敢,就是他代表了雅利安人、德意志人的那种身上那种都特别的东西。你去读他的书《思考与回忆》,你也能看到他的这种朴实无华的风格。在某种意义上而言,当代德国的性格就是由俾斯麦给烙印下的。一个人的评价有的时候很有意思,我们叫他铁血宰相俾斯麦,可是很少有人知……

现代的社保制度创立者,他不是英国和法国,恰恰就是这个俾斯麦,铁血宰相俾斯麦。他之所以创立德国的社会保障制度,很大程度上就是跟他差不多大,比他小三岁的马克思的影响。在处理德国走向现代化的过程中,俾斯麦是个保守主义者,他不是个革命者,他是个保守主义者,他站在马克思的对立面,但他又是一个卓越的政治家,他知道妥协的艺术,他也知道与时俱进。威廉二世如果能好好地去读《思考与回忆》,就不至于在1918年第一次世界大战使德国……

使德国陷入到如此悲惨的境地。其实在《思考与回忆》的时候,俾斯麦了不起,他已经预见到欧洲未来的状况,他甚至已经预见到德意志将卷入的战争,他非常非常反对德国在他之后还卷入战争。他自己喜欢打,而且他通过一系列的战争来建立起了德意志帝国。但他建立德意志帝国之后,他希望有一个比较长的时间休养生息,“比较长”的意思应该是百年的时间。其实这两个国家引起我的思考,就是德国出了俾斯麦,但是俾斯麦真的是很遗憾,就是他的年龄问题,他最后没有办法,他没有办法能够……他活了83岁。

他没有办法阻止这个躁动的国家,犯他所不愿意让它犯的那个低级错误,这个情况跟日本是非常相似,日本也是无法阻止它犯那个低级错误,它最后终于走入战争,并且在战争中成为了另外一个样子。因为今天德国和日本仍然无法建立自己的主体性,因为他们实际上还是在被美国占领军占领的情况下,他们很难形成主权国家所应有的那个样子,或者是应有的那个权利。所以在整个的发展过程中,他们整个国家出现了巨大的,这样的一个波折或者是坎坷。

我想,俾斯麦我也不多讲了。因为大家有时间去读一下子,买那个《思考与回忆》,他的那个回忆录,读一下子。有时间大家也可以简略的了解一下他的这个历史。俾斯麦他是个外交官出身,但他处理战争的能力非常强大,并且他处理战争在某种意义上来讲,它具有极优秀的战略家的水准,他是一个伟大的战略家,不可多得的一位人才。当然他在压迫德国革命者方面,他是,也是足够残忍和血腥的。

整个的在时间周期上面,实际上在马克思进入到青年时代,开始进入到他的革命状态时候,其实俾斯麦也开始进入到他的这个人生的重要阶段。但是马克思和俾斯麦的性格差异是非常大的,马克思是十七岁去追大他很多岁的燕妮的,他就是属于少年早熟。而俾斯麦到了很老的时候,才娶了她的后来的妻子。俾斯麦很有趣,他原来是追两个英国女孩儿,他其中追一个英国贵族的女孩儿的时候,因为没钱,所以他把仅有的钱用于赌博了,他想赌一下子,赚了钱去娶,结果他赌失败了。

读《思考与回忆》其实是比较沉重的一件事情,因为那里边你会看到很多很多的,你会带你去思考很多中国的问题,中国的影子在里边,有的时候会不舒服、有点难受。但其实是有价值的,非常有用的。好吧,我想这个留作作业吧,大家去慢慢的去思考吧。今天的历史的部分我想就讲这么多吧。两次革命,两位宰相其实都没说透,也暂时只能是点到为止,大家再花点时间去看吧,因为这两次革命和这两位宰相……

这两次革命和这两位宰相,都是值得我们去花点时间去整理和阅读的。我们剩下的时间聊几句当下的事情、聊几句当下的事情,因为处在一个关键的档口,因为今年是辛丑年,是贲卦。中国确实处在转身的关键时刻,就是我们需要税政革命,需要一场税政改革。但这个税政改革呢迟迟未能开启,这个我们当代这些人都是有责任的,我自己知道我们虽然起不到那么大的作用,但是我们在积极的努力来做这件事情。如果我们的税政改革顺利的话,那么中国将开……

中国将开启另外一个辉煌的一百年。如果不能顺利开启,那么我们可能会进入到一个麻烦的过程中。因为,我上节课我其实说了一些事情,我说了平成战败,说了天降组,说了1985年美国的双赤字——财政赤字和这个贸易赤字。现在的美国的状况更糟糕了,它的财政赤字和贸易赤字全部爆灯,而他自己目前并未找到解决的方法。只有,不是极有可能,是只有一种可能——就是将祸水东引,而能承载的也只有……

所以,我们处在巨大的危险之中。需要我们自己做出一些的努力,当然我们这里边也有一些事情在积极的想办法,在努力,就阻止一些事情发生。大家见到我比较激动,甚至有时候比较冲动,而且是连续不断的在发一些东西,不光是在微博和微信上发,我也给很多朋友们一些,发一些东西,希望能够在某种程度上能够起到这样的一个作用,能够起到一个相对积极的作用。当然我也知道我们的力量是有限的,但我们能做的事情,我们这代人该做。

我说了这个推背图的第38象,结合贲卦,它是有点意思的。我不主张搞封建迷信,但是我主张借鉴一些事情,完成一些思考、完成一些思考。什么时候过剩的美元资本会涌向中国,会导致中国出现经济上的另外一番景象呢?我必须跟大家说已经涌向中国了,只不过他现在表达为大规模的商品的采购。你们请注意,中国第一季对美出口的增长情况,其实美元……

印刷出的美元完成套现,一则是购买商品,二则是购买资产。只要买到了商品和资产,那个印刷出来的美元就活了,就有了灵魂,就被附加了潜能,就给它确认了它的信用。而商品主要是中国的大量的商品采购,很快开始进入到中国的资产,这个过程跟1985年《广场协议》签后过程大体相同。会不会产生1991年日本的崩盘?我现在还不能说,我只是希望我们一同来努力,阻止所谓的天降组,阻止没有广场的《广场协议》,阻止那一场资本过剩的危机的发生。

对目前的投资的一些事情,我会陆续的讲一下子。好多朋友问会不会出现我们大A,A股和这个黄金的一个交叉,会不会出现一些崭新的机会和情况?我想今天就不讲那么多了,我今天就到此为止,因为我真的有点累,稍微休息一下子。其实今天的时间不短,看明天有时间我再聊几句,或者是我们等到下个星期吧。好,今天就这样,大家多保重身体,注意,还是要注意疫情,多保重。好的,我们明天下午三点钟见。

\subsection{如何跨越官僚垄断资本主义陷阱、最近资本市场的变动}

大家好,今天是2021年4月24号,辛丑年的三月十三日。最近怎么说好啊,麻烦不断,然后群分成了七个,分就分吧。我先试一下麦,我们继续我们要做的事情。今天是聊天,聊天两个内容:一个内容是聊一下子如何跨越官僚垄断资本主义陷阱,这个事情是挺重大的一件事情。因为我可能下周要去学校做一个演讲,今天算是提前给大家做一个先期讲座。

然后另外腾出时间来,我们讲一下子整个资本市场的最近的变动,然后有一个基本的方向上的一个想法和思路。一会试完麦,没有大问题,我们三点钟准时开始。

大家好,今天是2021年的4月24号,辛丑年的三月十三日,阳春三月,时间飞快。确实是让人困惑,困惑不解,新浪将我们的一个群先是变成三个群,现在终于变成了七个群,做了一个比较彻底的粉碎、切割。似乎也能理解新浪的难处,也可能不仅仅是他们的难处——避免群聚感染,可能有一些政治上的考虑,无所谓啊。

我们仍然坚持我们既有的方式继续前行,不管风吹浪打,胜似闲庭信步。我还是希望大家牢牢记住,我们在过往的心学课程里边的那三句话。第一句话,一定要记住:主体性、适应性、创造性。主体性升起之后,其实人是不惧任何的困难的,不是不畏惧困难,是不怕任何的变故无常。当然这里边包括了许许多多痛苦和委屈,包括身体的病痛、记忆遭逢的这个不幸,遭遇种种的困难。

主体性升起之后,其实人是另外一种状态,在讨论学问的时候,讨论学习的时候,没有主体性的人即便是受很好地教育,最近我这个遇到了一些朋友,我的感受是很深的,不管你是多么成名的学者专家,没有主体性,总体上来看,只是一个高级的复读机。因为他不可能有自己的逻辑概念,也不可能有自己的框架,所以他在复述或者是这个重读一些别人的东西。主体性非常重要,其次就是适应性,适应性真……

适应性确实是一件很难的事情。我们说过适应这个:人对自己的适应;人对自然的适应;人对社会的适应;人对神的适应。适应性要求我们,实际上是在主体性基础上,对外部我们也可以叫它外部性的一种主动的接纳和融合。这个过程是困难的,这个过程对好多朋友而言是非常困难的。我自己从大学毕业到财政部,从财政部到中经开到香港。

一生走来,适应不同的环境——不同的自然环境,不同的人文环境,适应不同的气候,不同的食物,整个的适应过程……当你心里边有主体性,你在强调适应性的时候,其实在这里边有很多探索的味道,所以有冒险,有探索,有诸多很有趣的事情发生。很多人认为整个过程可能是,这个经历是沉重的、痛苦的,甚至是疲倦的,但如果你心里边主动地去适应,去享受,去感受,那么有可能适应性其实是你建立外缘的最重要的这样的一个机会啦。

创造性就是主观能动性了。我们在下一堂课要讲到马克思的哲学思想的源泉的时候,会要讲到这个康德和黑格尔,会讲到十八世纪和十九世纪重要的一些哲学家和思想家,他们的思考,对社会对人生的思考。三性非常重要,三性之后就是我们那句话——“正心以中,修身以和”。其实“正心以中”实际上是一种思想方法,就是我们从来不认为什么是对,

我们也从来不简单的判定什么是好。对、好对我们而言,不是那么的重要,我们要找的是“中”。“正心以中”的意思就是我们要找到事物的本质和规律,我们不做简单意义上的是非判断。当一个人正心以中的时候,他不会带偏,无论是做任何事情都不会再被带偏。如果一个人可以做到“正心以中”,就能做到“修身以和”了。其实“修身以和”貌似一个人的修养,其实不完全,是他思想的厚度,他的容纳的宽度。“修身以和”你才能有外缘,

才能够与天地同力,才能把事情办好办成。最后一句是方法论,就是“以无间入有隙”。无论如何,在任何时候都要使自己变得高密度,像钻石一样的高密度。同为碳,当密度达到一定的高度的时候,你就是金刚石,你可以划破玻璃,可以钻透一切的坚硬。如果你松散的、普通的碳,有可能你就是一个煤渣,就是一个木头,都是碳。当然它也是有用的,但是你无法“入有隙”呀,一个人的“无间”的这个功夫,“无间”的功夫对你自己思想而言,实际上是锻炼你的精神和意志。我有时候会举两个例子,我有时候会劝导身边遇到问题的朋友,我总是跟他们说,罗曼·罗兰说过:“我们的心就像是一块土地,铁犁划破了这片土地,划破了我们的心,也同时可以播下种子,让它在划破的土壤里边播下种子,看到孕育新的希望,孕育出郁郁葱葱的收获。”

我还会举一个例子,就是阿拉伯的冰花钢。我喜欢买冰花钢的刀、刀具,我喜欢刀上那一簇一簇的、一团一团的冰花。冰花钢就是钢烧到通红,千锤百炼,再放进冰水里再捶打,再放进冰水里边。心也是这样,坚强就是这样地百炼出来的。精神和意志的高密度可以导致一个人的状态进入到一个别人难以理解和想象的状态。也可以这样的来锤炼自己的身体,也可以这样锤炼自己的团队。我也希望我们,我们虽然分成了七个组,可是我们数千人依旧是“无间”,数千人“无间”,以学问入手,将来以其它的方式凝聚,“入有隙”做出一些成绩来。没什么东西其实可以,没有什么东西可以阻挡学习和进步,没有什么东西可以阻挡,小小的事情其实也就不用放在心上了。我们尽可能地想办法将这个事情处理得更好一些,也无论怎么弄,我们也会坚持把《资本论》和《通论》这两部书讲完。越是拦着,这个劲头越足,我会准备得更好、更充分。

今天讲的两个事情其实跟《资本论》有关,也跟投资有关,也契合当下的形势。第一件事是讲一下子如何跨越官僚垄断资本主义陷阱。可能好多朋友说:“你怎么突然想起这题目呢?”其实这题目是19号,4月19号,一个微信公众号叫“新岭南观察”的它登出来一段话,就是李显龙关于这个官僚主义的这样的一段说辞。这个据说,李显龙说:“中国战胜美国唯一的前提就是克服自身的官僚主义,克服体制内的官僚主义。”到了4月23号新加坡驻中国大使吕德耀在微博上辟谣,指出这个部分中文媒体转载李显龙总理“关于中国官僚主义”的言论是虚假新闻。其实我知道李显龙不会这样说,因为这不属于李显龙的语言范式,但是我也知道李光耀和李显龙会这样想。只不过是这个语言范式显然不对,因为他没有说清楚,所以我觉得这个新闻一定是虚假的。但无风不起浪,那么李显龙看见了什么呢?他想到了什么呢?所以我想念……

简单地说,我跟他的看法高度契合,就是我们想的是一样的。什么是官僚主义呢?显然他说的不是一种习气和作风,不是官僚主义习气和作风,他说的是由国家资本主义延伸出来的官僚垄断资本主义。其实这个问题不仅是在中国日益严重,在新加坡也非常严重。我去新加坡的时候,新加坡的朋友对我说,他说:“李光耀和李显龙两个人是以“国王”、以“王”的身份或者是以“王”的视野来看问题和思考问题,来处理问题的。”

当今世上,比如说像英女王这些王,王是真的,但他们实际上已不具备王的权力,李家父子虽不称王,但是“无冕之王”。最难得的是他们是以“王”的角度来思考社会和思考国家治理,这是极为难得的。因为一旦以“王”的视角来看社会的时候,他自然而然就会与人民站在一起,所以李显龙与家人严重不和,与自己的家人,与自己的部属矛盾重重。当然李显龙是对的,因为他真的是悲悯整个新加坡人民。

不是美国内卷、日本内卷、欧洲内卷、新加坡也在内卷,新加坡也需要克服内卷。当一个高度集权的体制(其实新加坡也是国家资本主义),当一个高度集权的体制,时间久了的时候,它会发生异变,会发生突变。我下个礼拜去讲这堂课,我这个周一直在备这堂课,就是先给大家讲,这堂课就是讲……其实这是给中国当今社会做一次全面的诊断,做一次全面的诊断。这事儿有点大,这事有点大,但这事儿它确实是无比的重要,因为研究宏观看不懂这个是不成的。

马克思写《资本论》,第一卷是价值论。其实所有谈经济学必须从价值论入手,不然的话,你不知道资本的源泉,就是围绕着剩余价值谈的价值论构成了第一卷;第二卷写的是资本流转;第三卷马克思应该写国家与资本的关系。就是这第三卷马克思写不下去了,拖了十来年的时间,基本上没有落笔。不是马克思不想写,是他真不知道该怎么写,因为“巴黎公社”的失败给了马克思沉痛地打击,就是他认为,社会主义者建立的新型的社会主义国家如何处理与资本的关系,其实马克思没有想得很透彻。因为他对……我这么说吧,马克思其实对当时的无产阶级或者是对当时的无产阶级政党有诸多的不满意,所以马克思会对他的女婿说:“我不是你们所说的马克思主义者。”就是他认为当时的无产阶级和无产阶级政党还不够成熟。但是形势发展极快,马克思没写第三卷,那么革命形势发展需要,列宁写了第三卷,那部书的名字叫《国家与革命》。列宁初步论述了社会主义国家与资本的关系——《国家与革命》。

这是第一次有人从理论上来讨论国家资本主义是个什么东东。这就以此为基础建立了我们所说的苏联式的社会主义,那个就是国家资本主义。那个不是社会主义,那是国家资本主义,那和马克思思考的,马克思第三卷准备写的社会主义相去甚远。这个搞国家资本主义的另外一个代表人物就是希特勒,他在德国初步建立了一个国家资本主义的样式,跟苏维埃的国家资本主义不太一样。中国模式的国家资本主义跟苏联模式、跟希特勒模式也不是很一样,他们各自有特点。但是列宁没有解决一个根本性的问题。

也就是说,国家资本主义发展到一定阶段必然出现内卷,内卷之后必然走向官僚垄断资本主义。我重申一遍:国家资本主义发展到一定阶段后就会出现内卷,他不可避免地会走向官僚垄断资本主义。这不是一个喜欢不喜欢,这甚至都不是你用什么概念、逻辑来概述,这是铁一般的现实。“正心以中”嘛,就是我们要说出它问题的本质与规律。那么如何来解决国家资本主义滑落或者是沦落为官僚垄断资本主义?这是一个非常严峻而伟大的课题。

国家资本主义好不好?在落后、相对落后的、后发达的、后现代化的国家,就是由一个农业社会向工业社会转型完成工业化的国家,在没有办法通过殖民掠夺的方式进行原始积累,只能通过国家资本主义,以国家资本主义的方式进行全社会资源的整合和重新配置。这里边包括了土改、革命和公有化整个这个动作。就是国家资本主义这个铁与血的过程,实际上是完成工业化和现代化的一个重组的必由之路。但重组完成之后,因为它是代理人制,因为无产阶级不可能集体去领导他自己的国家,必须选出代理人。

那么经过一个漫长的过程,代理人会经历第一代、第二代到第三代,代理人慢慢地会从理想主义中褪色,开始出现内卷,然后将人民的信托或者是委托(其实是一种委托,没有信托),人民的委托给你的权力僭越。就是由官僚集团以家族的方式、以集团的方式,或者是以个人的方式,完成对政治权力、经济权力和文化权力的僭越,从而掌控了大部分的社会资本和资本收益,形成了官僚垄断资本主义。

很多人对我有些意见,就说我看问题、描述问题和想问题和主流学者完全两回事。我在香港也经常会产生巨大的争议和争论,也是这个原因。当今社会处在一个什么状况呢?就是美国为代表的西方正在走向极端的金融资本主义,就是他由发展比较良性的国家主义与社会资本主义的混合体,正在走向高度垄断的,就是高度金融垄断的资本主义。而以中国为代表的这些传统以国家资本主义为主体的国家也出现了,或者是已经开始出现严重的内卷,陷入到官僚垄断资本主义的这样的一个陷阱之中,正在走入这样的一个陷阱。那么官僚垄断资本主义有什么特征呢?其实官僚垄断资本主义垄断的是什么东西呢?第一,是政治权利,垄断了立法权、司法权和行政权。本来官僚应该将立法权还给社会,还给人民,但由于经历了惨烈的革命和斗争之后......

这个权力形成了某种的血统或者是血缘的世袭,他虽然不是真的爸爸交给儿子,但他有了一种基于关系的一种隐性的世袭。就是立法权、司法权和行政权都出现了这样的一个状态,这非常不好的一个状态。如果仅仅是行政权力的内卷,倒是还是可以的,如果这个立法权、司法权、行政权同时被一个集团垄断并且内卷的话,情形就会变得严峻。但这还不是最大的问题,它最大的问题不在这个里面,最大的问题在经济权力。

经济权利就是财政主权、金融主权和要素主权,要素主权核心是土地。我在香港呢一直在研究香港经济,我也一直在写那本《广义财政论》,其中涉及到的就是经济主权。就是香港的政治主权没有完全回归,政治主权回归的是一个行政主权,他的立法权和司法权是没有回归的。那么经济主权呢?根本就跟伟大的祖国没有关系,他的经济主权是属于祖国的,属于中央政府的,但是经济主权没有回归。因为我们对主权的定义是模糊的,到今天我们也不是很理解主权,所以今天这堂课挺有意思的,我们讨论一下子经济权利。

官僚垄断资本主义垄断财政主权的特征就是垄断了税政的权利,就是征谁的税,征多少税的权力在他们手上,他们不让征直接税,他们只征劳动,针对企业和个人的劳动来课税。财政主权,当然财政主权还有支出的权力,支出的权力就是建立庞大的官僚体系。我们最近注意到东北沦陷了,山东沦陷了,最近要振兴的是华中啊,华中也开始严峻的东北化。你知道官僚垄断资本主义一旦掌控经济权力之后的可怕的后果,因为它是一种将立法、司法和行政匪化的过程,非常麻烦。

控制完财政权力就是控制金融权力。我们如果不知道什么叫官僚垄断资本主义,我们就看一看中国资本市场就知道了。中国资本市场现在大概是六成,最严重的时候超过七成,上市的公司的主要利润由银行和非银行金融机构来提供,上市的主体也是银行和非银行金融机构。我们也包括了很多的地产类企业,甚至我也将所谓的一些高科技企业,如阿里、腾讯,也定义为准金融机构。官僚垄断资本主义对金融权利的垄断,恰恰是官僚垄断资本主义目的和手段的核心所在,因为他确实是不拿下这一块就无法完成……如我们做一个统计,我国正部级或者是正省级以上领导同志80后、90后的孩子从业状况。你可能会惊讶地发现,90\%以上这些的子弟都在从事金融行业。他们没有去做老师,没有去做解放军,没有去从事乡村建设,他们都在从事金融。如果你不理解什么叫官僚垄断资本主义的话,那我们看一看孩子的去处,大体上你就知道什么叫官僚垄断资本主义了,因为这是一个必然的结果。

第三个部分,就是经济主权第三个部分是要素主权。其中核心是土地,依附于土地上面的财产的创造与套现,这是一个非常糟糕的超级地租的套现模式。这个模式并非香港创造或者当代出现,在过往三千年历史上从来如此,不过那时候叫土地兼并。从来都是这套鬼把戏,只不过是换了新衣裳,叫什么按揭,叫什么楼花,换了一些新的状况。然后呢老百姓在里边要生要死、打生打死,在里边沉浮,喜怒哀乐。因为大多数老百姓没有办法正心以中的嘛,没有办法修身以和嘛。

买了房的,买了三套房、五套房、十套房的,当然高兴了;没有买房的,当然很难过了。大家很难站在一个哲学的高度和历史的长度上、阔度上来思考这个问题。官僚垄断资本主义必然会造成一系列的我们今天所看到的所谓的泡沫,因为资本的膨胀导致的价格的泡沫,恰恰是完成他们所谓的财富创造和收割的一个必然经历的过程,必然经历的一个历史过程。这个历史进程就是在特定历史阶段,它可能也是社会发展必须经历的过程。

但是这个过程一旦失去控制,将导致党、人民和国家的灾难,所以不懂官僚垄断资本主义就很难跨越这样一个陷阱,跨越这样一道门槛,很难使国家真的在完成第一个一百年(今年完成第一个一百年),来顺利完成第二个一百年,到1949年那第二个一百年。这还有28年,这28年能否战胜官僚垄断资本主义,差不多是定我党生死、定老百姓幸福、定国家兴旺崛起的最关键的因素,这件事情确实重要。

官僚垄断资本主义控制的第三个权力是文化的权力。文化的权力包括了教育的权力、学术的权力和传媒的权力。这种官僚垄断的结果实际上出现了就是我们现在教育的问题。我们的教育真的非常严重,我们的学术非常严重,这个严重的以至于都影响到香港,整个的香港的教育和学术出现了,这二十年出现了严重的退化。这个退化难以想象,就是世界最顶级的学府全都完蛋了。教育、学术和传媒的退化,传媒的退化我们就不说了吧,都碎成七片了,变成“北斗七星”了。

文化权力被官僚垄断资本主义所掌控,成为了两样东西:第一,它本身仍然可以通过资本化的过程中形成牟利的手段,更主要的它成为驯服工具,它通过文化成为驯服工具。我最近打开微博,我就能看到韩寒做的那个金典奶粉的广告,所以我说我以后肯定不会再喝金典奶粉了。为什么找一个造假的、有毒的人来给奶粉做广告呢?难道仅仅是因为他带有西方的使命,所以他就会获得整个资本的喜爱或者是帮助吗?奥迪请他做广告,我能理解,那是洋人,我们为什么呢?

那么我已经说了,就是国家资本主义发展到特定的历史时期,会出现内卷,卷向官僚垄断资本主义。如何克服这个内卷,走出这个内卷呢?有的时候呢要靠我们思考的力量,有的时候呢还得借助天意,就是天地皆同力。我们也很努力,这样的话呢可以起到一定的作用。那么什么是天意呢?天意就是生产力发展进步的水平,生产力决定了生产关系。

中国终于开始进入信息社会,我们一切都在信息化。信息社会有它的巨大的好处,就是它使得思想的流动,虽可能被一些人做某种限制,但仍然难以阻绝思想家的思考。另外就是数字进步将导致立法、司法和行政权力本身的运作模式发生变化。我将解决官僚垄断资本主义的方法切成了三块,第一个关于政治权力的解决。

我并不主张西方的方式来解决政治权力,就是通过所谓的民主选举来解决政治权力。我主张将政治权力全部数字化,就是:立法权要数字化,司法权要数字化、行政权要数字化。数字化意味着透明、公开,意味着高效率。举例,司法,英国已经出现了这个数字法官,就是有这个电脑,它已经集成了这个所有的案件的大数据,而且它分析系统会做出诊断。A和B两个人发生了官司,两个人不一定去法院,跟这个电子法官把事情输入进去,电子法官做了判决。

因为它是案例法嘛,它根据它所有发生的案例,它知道这件事儿怎么做是最合适的。如果A和B真的去了有人的法院,不是数字法院,判决出来跟数字法官不一样,回来问数字法官,为什么他判决和你不一样?数字法官会告诉B,说A昨天晚上给了他300镑。为什么呢?因为根据既往的经验,如果出现误判原因大概有这么两条或者三条,它认为他是那个第一条原因。这个大数据,数字化政治权力其实意义非常重大,因为这个数字化的政治权力,什么东西放在阳光下都会发生质的变化,什么东西在阳光之下都会变得健康起来。所以我主张中国现在当下最需要做的就是数字化一切政治权力。

第二是社会化经济权力。社会化经济权力的意思就是将征税权还给社会,就是征谁的税,征多少税,这个本身它不是一个行政决定过程,应该还给大众讨论,就是你们需要一个什么样的政府,你们大概需要交多少税。因为议政就是议税,议税就是最大的民主。这个权力还给社会,在今天是可以的了。以前你说还得要讨论或者选出个代表来,现在不用,现在你只要给个议题出来,很快你就知道民意的倾向性就出来了。只是你决定尊重不尊重社会,你尊重它,其实这个东西可以解决了,就是财政权,这个可以解决了。

金融权力能社会化吗?当然,所谓的普惠金融可不就是为了社会化嘛!让每一个人都拥有被金融服务的权利,就是金融服务覆盖到每一个人,这是他的基本人权,这个道理不需要很复杂地讲。你不能在一个实质通货膨胀超过两位数的时候,有些人能贷到无限的钱去买资产获得暴利;有些人还存款,不但不能借款,还得存款,因为不知道明天会怎样生活,没有保障啊。金融的服务不能普惠大众,不能普惠众生,这个它没有社会化,这是不可以的,必须将金融权力社会化。

第三个,必须将所有资本利得收益社会化。我们不共产,但我们要求共享。社会主义最主要的特征就是共享,共产主义才共产,社会主义不要求共产,要私有,但要共享。共享,大家可能说那什么叫共享呢?直接税不就共享了吗?遗产税、赠与税、房产税、数据税,那不就是共享了吗?如果这个过程能完成,这个想法能被接受,那么就不需要缴五险一金了。社会保障是无差别社会保障,人人都有份儿,你就不需要去存钱了,你就有保障嘛,你就可以消费了。

我知道我讲这个课,可能是大家觉得一个新的空想社会主义者又诞生了,那不完全是。数字化政治权力,其实你想做也得做,不想做也得做。其实西方发达国家正在做数字化政治权力、社会化经济权力,其实别人已经做了,只是我们现在要开始做,克服内卷,必须做这件事。第三个就是国际化文化权力。既然是办教育,我们就要请全世界最伟大的、最好的教育家,因为我们给的钱比全世界哪个学校都给得多。那么为什么不请全世界最伟大的教育家来给我们办大学呢?我们为什么不将这个成果购买作为主要的方法呢?

比如说社会科学,我们为什么不买它的研究成果呢?为什么不买它的书呢?你为什么要养一个研究机构呢?你为什么所有的,连县一级都有研究机构?所有的部委都有各种各样的校和机构、所,养那么多东西,然后给那么多的研究经费、课题费,然后什么都不出。其实在文化权力上面官僚主义是最严重的。国际化的意思就是我们用国际上最好的人和最好的方法来解决我们的文化权力问题。其实你这样解决了,可能全世界的人才都会来中国。一会儿我们讲美国可能会陷入史无前例的危机之中,你做好接纳人才的准备了吗?

中国如果能够在重大的历史关头,能够解决国家资本主义向官僚垄断资本主义退化的这个历史进程。这个退化的历史进程,在这个过程中,好多人可能是无知无觉或者是无感的。因为什么呢?因为我国的经济建设确实取得了伟大的成就,在这个伟大成就下,好多事情可能被遮蔽了。很少有人在如此好的一个发展中,能够冷峻地看到潜在的危机。其实一个国家思想家的存在,它的意义就在这里。

一个国家最珍贵的不仅仅是科学家,也需要思想家,其中思想家里边最珍贵的可能就是制度经济学。制度经济学的扛鼎者其实是这个国家的那个制度板块的高度的决定者,因为制度的短板是由它的这个思想家来决定的。如果大家谁都看不到它的短板,那么就完了。其实苏联是怎么垮的?就是垮在官僚垄断资本主义这个陷阱里掉下去了,跟什么军备竞赛,跟什么星球大战,跟什么改革与新思维,跟什么美国特务没有关系,那是形式。

在英语世界,他们通常把官僚垄断资本主义翻译成裙带资本主义、朋党资本主义、权贵资本主义,这个翻译我不是很喜欢。他们叫Crony Capitalism,我老觉得这个翻译有问题,但我也不太接受,我在google上面找着的那些话,那个翻译我觉得也是有问题的。官僚垄断资本主义可能需要给一个更完整、更清晰的定义,这两天还在做这件事情。但我今天就是将择要点先给大家念叨念叨,因为这跟我下边要说的话和聊的天有关。

好,今儿这个官僚垄断资本主义就先说这么多吧,我念叨几句美国。我这两天在读两本书,其实我前两天应该把那照片拍出来,我买了一堆书又...好,我过两天把照片拍出来。我买了一个竖排的、繁体字版的台湾出的《资本论》,精装的,非常漂亮。我这两天读的书是张化桥写的《中国债务危机解密》,他在讲的是中国的次贷危机的进程。还在读一个《散步在华尔街的马克思》,然后还在读一本斯蒂芬妮·凯尔顿的……

斯蒂芬妮·凯尔顿的《赤字迷思》。这个《赤字迷思》刺激到我了,因为凯尔顿的这个书就是关于MMT,关于财政赤字货币化的这个想法,我一直在梳理她们整个的想法的这个逻辑过程和可能导致的政策结果。实际上现在美国政府——拜登政府正在照凯尔顿的这个思想去做实验,在尝试。这个尝试的结果会是一个什么样的结果呢?就是今天我们聊天要聊的内容了。美国,他原来在小罗斯福之后,经历了两次世界大战。

美国在经历了两次世界大战之后,再经过跟东方阵营、跟苏联的这种较量或者是竞赛,所以他在某种意义上,美国建立了一个比较好的资本主义模式,他就是由原来的原始自由资本主义走向了社会资本主义加国家资本主义。所谓的加国家资本主义就是“罗斯福新政”,他用了凯恩斯的一套方法,就是国家还是通过国家政府和公有机构来干预经济,还是有国营的部分还是量很大,那么形成了一个国家资本主义、社会资本主义的混合体。这个混合体运行一直很好,所以美国经济也很好,从上个世纪超过英国。

但这个进程在1971年被打断,1971年主要是越战导致的严重的信用危机,他就是金本位被破解了。破解之后,1971年之后,其实美国开始迅速导向了金融资本主义。这时间只有50年,现在是2021年,1971年到现在50年时间,他的旧有的社会资本主义加国家资本主义的混合体迅速地内卷倒向了金融垄断资本主义。这个金融垄断资本主义走到今天,已经非常非常糟糕了,这个他导致出现了状况,在某些方面跟官僚垄断资本主义很像,只不过是他那个垄断的主体,不是官僚,是金融资本家。

凯尔顿的这本书,她要说的是重建国家资本主义。当国家掌握发钞权的时候,国家控制了资本的源泉,由国家掌控的资本延伸出中央的权力,因为印钞机在中央政府手里边,不在州政府,也不在县政府手里边。当州政府开始履行强大国家职能的时候,就是中央集权的职能,那么MMT就出现了。这个MMT是有前提条件的,不是任何一个国家都能来MMT,就是不是任何国家都可以随意印钱的。

一个主权货币可以不受经济总量的制约,而随意地膨胀。随意地进行信用膨胀有两个前提条件:第一,你所发行的货币必须被市场接受和确认。信用嘛,要有人信、有人用才行。那么这个确认,就是你的货币必须是具有国际硬通货的这样的一个特质,就是你能从中国买到廉价商品,你能到中国去买到优质资产,你的美元的价值是被中国承认的,这是第一条。

第二条,是“三流”可控。三流可控就是:第一是要有人信你;第二是你能管得住;三流可控就是流量、流向、流速,因为流量、流向、流速决定了四矩阵,决定了四矩阵的价格变动。因为钱就是“水”,水火无情,如果处理不当,它会带来灾难;如果处理得好,水可以滋养万物。凯尔顿她看到了水“利”的一面,水“利”的一面。

她对水“害”的一面明显估计不足。凯尔顿和耶伦的经历不一样,耶伦是四进山城,她四次从象牙塔里边进入到政府体制内工作,她有往返四次的经历,她对实操是熟悉的。但是凯尔顿不行,她实际上一直在象牙塔里边,她对现实的残酷我觉得她缺乏足够的认识,所以她不知道这个理论现在的实验已经使美国处于巨大的危险之中了,这个危险应该说是史无前例的。

说来(这个身体还是调不过来),说来还是有些让人感慨。本来是求中国,来为美元背书的,为美元信用背书的,但是,他要揍着你、打着你、骂着你,让你来背书,背书不成还要买单。就是美元的信用由你来确认,确认完了以后,还要将整个的美元信用所延伸出来的附带成本,全部要你来买单。这个事情做得就有点过了,有的时候,他们可能想问题过于简单了。

我最近一直在说三个词:一个词叫资本过剩的危机,说的就是MMT;第二个词平成战败,这个事情大家可能已经开始知道了,熟了;第三个词是天降组。日本是个悲剧国家。1971年之后,美元与黄金脱钩,脱钩以后严重地滥发,印得过头了,到了八十年代,双赤字,严重的贸易赤字和财政赤字,怎么解决?要找个买单的国家。环顾世界一看,只有两个国家有能力买单,都是美国体量的一半,一个是日本。

其中,日本可以自觉自愿地买单。我其实心疼平成天皇,我觉得他是个可怜人。昭和好歹有男儿,平成全都变成废柴了。二次大战结束,1945年结束,到1985年,四十年了,昭和男儿早都退休了,变成了平成一代,平成一代是在美国人训练下成长起来的一代。平成男儿,如果他们还懂得“安全”两个字,那个“安全”两个字也是针对苏联和中国的,他们从来就没有对美国有防备之心。

再加上天降组,记住这几个竹下登、澄田智这几个名字吧。天降组,自觉地履行了,美国人给予的使命。“广场协议”只是一个“起子”,从那天开始,日本的股市用五年时间,攀上了高峰,四万多点,然后房价涨了大概五倍多,股票应该涨了四倍,房价涨了五倍,汇率从三百涨到七十。记着:股楼汇三升,三个都升,厉害了我的国,就是那会儿,信心……

结果到了1991年全部爆掉,房价爆掉、股市爆掉、汇率崩掉,然后进入衰退,三十年过去了,日本没再长大。美国现在二十万亿,日本在五万亿,没再成长,它一直在缩在一团儿,没再长大。中国那个时候是他的五分之一吧,现在中国十五万亿,是他的三倍了,体量是他的三倍。昭和一代还有男儿,平成这一代是真正的废柴,他们从精神上崩溃了。现在的日本的男人、女人的精神状况是完全不对的。

他们第一次,1945年被打倒、被击败,他们只是军事上失败,他们还有再次崛起的机会。1985年这次战败,他们没有机会了,今后三百年之内,日本人都不再有机会,因为精神上垮掉了,非常惨烈。五十年之内被一个国家干倒两次,其中第二次是欢天喜地被干倒的。第一次还是两颗原子弹、血流成河,第二次是欢天喜地。我为什么要强调资本过剩的危机、平成战败和天降组,就是因为1921年这个状态,那边在拼命印钱,财政赤字,这个贸易赤字飞起来,

而我们这边正在金融开放,我们这边……好吧,不说了,其实我该说得都说完了。好,我们回到投资上面,我们还是不关心国家大事,不关心宏观,我们回到微观层面,谈点投资上的具体的事情。初步的估计,在今年的下半年,可能会出现一个我们大家意想不到的局面。现在所有人都认为疫情控制住了、经济开始复苏,所有人都这样认为,

我,不这样认为。首先2008年那场危机没有解决,我说的是2008年美国的那场次贷危机没有解决,只是被掩盖。那场危机的一部分的问题在中国变成了新的问题,大家有空可以看张化桥这本书《中国债务危机解密——一个次贷工作者的醒悟》,看看中国的M2的总量,中国的现在的这个次级贷款的总量,或者是中国现在一个一个爆雷的,就是中国现在这个信用危机,最近可能这个除了永煤之外,今年可能华融、海航等大家伙也在出问题。

中国背负了一些东西,但是中国总体的经济仍然是全世界最好的。那么中国具备了1985年日本具备的一切条件,而且我们的体量比那时候日本还大,只要中国愿意,就可以重演一遍广场协议。就是大量的美元,现在其实在大量购买我们的商品,那个时候也是大量购买日本的商品,就是贸易赤字继续增加、翻倍的增加,然后资本进来开始买我们的资产,什么都买。其实可能重点是股市吧,楼市我们会限制一下吧,因为房住不炒嘛,可能资本市场会有一轮超乎所有人想象的历史性的牛市。

我们现在就可以作观察了,可以做准备了,因为就算不进来,可能也要准备一次像样的牛市了。不要听外边专家怎么说,专家的话不要听,我们自己开始着手准备。那么有很多朋友就会问我,那么短股长金,策略要变吗?我在这里今天明确地讲一遍,暂时不变。暂时不变的原因是我已经说了,今年下半年可能会有一场我们意想不到的危机、经济危机,因为可能疫情获得全面治理这个事情不属实。

我重复一遍,2008年的危机没有解决,只是被遮盖。经济的衰退从2016年就可以见到了,只不过特朗普上台之后,将这个衰退方向做了某种改变。衰退到了2020年以疫情的方式做表达,大家大放水,能解决那个历史性的衰退?能解决那个历史性的遗留问题嘛?两件事情都解决不了,第一,没有解决根本性的问题,那是一场资本严重过剩的危机,这个问题没解决;第二个,用这种MMT能解决经济复苏的问题吗?还是不能解决。两件事情没有解决,那么,意味着什么呢?

我个人认为,中国确定我们的目标是6\%不是没有原因的,我们是精算过的,不是没有原因的。我或者是我身边的一些朋友,其实对下半年保持高度警觉。我们认为资本大量进入中国之前会有一次深调,可能是美国资本市场的一次崩塌式的深调,深调之后才会出现资本大规模的跨境流动。这个时间节点的过程中是否伴随着美元剧烈的贬值和黄金大幅度的升值,我今天不给结论。不给结论的意思就是:我们留出观察的时间,不下结论。

其实在讨论投资的时候,我心里边的压力是挺大的。因为讲《资本论》是容易的,讲准投资的选择,特别是四矩阵的选择,特别是精确的时间选择,真的很难。因为这个事情不是人能做的事情。因为如果我们有人能精准的测算出每一件事情来的话,那他真的是有点像神了。就算我们有主体性,就算我们正心以中,我们也到不了那样高的境界啊。所以我今天要说的话是:那个危机尚未结束衰退,2008年的危机未解决,2016年开始的衰退没有走出来。

如疫情不能得到有效控制,第二轮的危机很快将到来。第二轮的危机到来的时候,市场不是就此上涨,可能还有一次深的调整。如果美国股市出现深刻调整的话,中国股市虽然不会像他们那样深刻调整,还会在通道里徘徊一段足够长的时间,甚至会给出你买货的最佳的时机。所以暂时不动、观察。我算说清楚了吧?再说清楚就有点太过了。就是我们冷静地观察、做功课,,把所有的准备工作做好,随时准备转身,但此刻,不急不急。

我敬爱马克思,因为马克思他以神的眼光来注视社会,来看人民,我做不到;我崇拜凯恩斯是因为他用人的眼睛看社会,并且凯恩斯同时兼具人的,人性的自私。因为凯恩斯除了他的《通论》写得非常好,虽然在制度经济学的角度达不到《资本论》的高度,但它也是确实是大师之作了。另外凯恩斯在投资上面是极为成功的。在经历1929年那场大危机的时候,他是非常冷静,极为清晰的,他这一生都非常地冷静和清晰。

等《资本论》讲完我就讲《通论》,讲《通论》我会讲凯恩斯的故事。我说了我们敬爱马克思,当然我们就变不成马克思,因为我们不打算以神的眼睛来看世界,更不打算像马克思那样的神圣的生活。我们稍微庸俗一点,我们学习凯恩斯吧,我们还是一个正常的,不用颠沛流离的,可以吃饱穿暖的生活,稍微庸俗一点,稍微庸俗一点。今天这个聊天儿聊得稍微地多了一点、重一点,但是这个聊天真的很重要,今天这个聊天非常重要。

顺便说一句,就是这个疫情确实出乎意料。就是那个疫苗,我打了第一针了,下个月初打第二针,现在看来,疫苗对复合的病毒、变性的、复合变性的病毒非但无效,可能还有帮助作用。这个情形可能不是大家想象那个情形,所以全面的恢复在六月底之前是看不到了。也就是上半年一部分的国家有了比较好的恢复,比如说美国;但现在从印度的情况来看、巴西的情况来看,还是非常糟糕的,甚至欧洲的情况也并不理想,而且有可能会伴随其它事情发生。

所以还是请大家多注意自身的安全防护,不要放松警惕。另外腾出空来多读书、多做功课、锻炼身体。原本呢有朋友说,上次我讲的《推背图》今天能不能再讲一讲。我说,后来想了想,封建迷信的东西不搞了,虽然它可能有利于我们思考的多面性,思考的维度可以变得多面性,可以拉长,可以互相地启发。但我想了想,这封建迷信的东西还是不说好,不搞了。那么我们今天就讲这么多,然后分了七个群,工作量变大,辛苦大家了。我真的是非常感激,真的非常感激。

那就这样,我们明天下午三点钟准时交换资料,有什么新的情况我们再随时作沟通,周末愉快。

\section{马克思主义的哲学源泉、当下的疫情}

大家好,非常难得有机会在2021年的5月1号跟大家共同度过这个劳动节。昨天晚上没睡觉,昨天晚上睡不着,因为今天要讲马克思主义的哲学源泉,所以不得不再重温一遍这四百年的思想史。重温这四百年的思想史让人联想到很多东西,所以一边读一边写,就写到了现在。给大家讲完这堂课,我再睡觉。我一早来到办公室,来到写字楼,正好可以有一个完整的思……

好,试一下麦,然后我们三点钟准时开始。今天这堂课可能会稍微地沉一些,但应该是非常愉快的一堂课。好,一会儿我们见。

今天是2021年的5月1号,是辛丑农历的三月二十,今天是劳动节。我们今天开始讲《资本论》这课程的第二讲——马克思主义的哲学源泉。其实要理解《资本论》,这堂课变得非常重要。第一堂课显得紧凑了一些,因为我们第一堂课里边讲了一点点的欧洲的近代史,讲了两次革命——工业革命和法国大革命,讲了两位宰相——梅特涅和俾斯麦。但一个小时容量太大反而显得不够深入,今天讲哲学。

今天我们讲哲学问题就更显得时间的局促。我晚上睡不着觉,我一直在想,必须要用人能听懂的语言,让每个人都能听懂的语言来说一说过往四百年欧洲的思想家们是怎样思考的,这些思想家的思考怎样导致整个欧洲完成现代化,从完成工业化到完成现代化他们经历了怎样的心路历程。在这个过程中,首先我自己痛彻思考五百年来中国的思想家在干什么,在同一个时段里边,在欧洲星光灿烂,

而在东方,除了一个王阳明之外,至今五百年来没有出现像欧洲那样灿若群星的伟大的思想家特别是哲学家,还有构建了整个现代文明的那些科学家。确实是在这五百年的世界历史的长河中我们出了问题,我们在讨论欧洲哲学的时候,我们也在思考东方为什么在过往五百年在哲学上陷入了一个沉沉的黑暗之中,或者是漆漆的黑夜这里边。

今天我们也不能讲太多内容,所以我们围绕着四个人徐徐展开,一个是笛卡尔——勒内·笛卡尔,一个是卢梭——让雅克·卢梭,这两个是法国人,一个是康德,一个是黑格尔,这两个是德国人。两个法国哲学家,两个德国哲学家,他们构成了整个欧洲思想史的一个脉络。为什么不谈英国的哲学家、思想家呢?我想英伦他有独特的地理环境和独特的人文环境,所以它不具备大陆特征,而法德的思想家具有大陆特征。他们过去思考的问题,几乎所有的问题……

他们思考的所有问题,在当下的中国仍然是问题,而且大部分的问题,特别是关于现代性或后现代的问题,实际上一个都没解决,这是一个非常沉重的问题。所以我们今天借着讨论欧洲的哲学史或者是思想史,也检讨一下子我们在哪里出了问题。因为我们的问题现在到裉节儿上了,2021年到了个裉节儿上。我上一堂课聊天的时候讲了官僚垄断资本主义,讲了美国的金融垄断资本主义,东西方都遇到了问题。遇到了问题,需要哲学。

需要思想家来对这些问题做出具有哲学高度的分析、解释,给出答案和出路,但这个工作确实是艰辛的。当我重温这段历史的时候,无论是笛卡尔、卢梭、康德、黑格尔,乃至于马克思都再次给我心灵巨大的震撼。在历史长河中,有许多东西可能、可能有自然的力量,但在很多东西它也有人的力量,我不认为东方人在哲学上弱。

只是东方人一直没有能够从沉重的、东方的、旧有的、哲学历史的包袱中走出来。一会儿我讲笛卡尔、讲卢梭他们是怎样破解了中世纪欧洲的经院哲学;笛卡尔是怎样建立了现代哲学,他是现代哲学之父。而我们中国直到今天仍然走不出儒家的“经院哲学”,我们现代所描述的诸多的理论仍然是儒家经院哲学的现代翻版,我们在捆绑自己的思想,也在捆绑自己的手足。

我刚才,今天因为一直在喝咖啡,我刚才喝了点冷的东西,我想压抑一下我这个激动的情绪,不让我的情绪过于激动。因为我们今天是讲哲学课,所以还是得稍微沉静一下来,不夹杂太多的感情和情绪进去。哲学是一个什么样的学问呢?哲学是关于智慧的学问。那么这个什么叫智慧呢?我记得我上堂课说过,人与自己的关系、人与社会的关系、人与自然的关系、人与神的关系,说清楚这些关系,那就是哲学。

在中国的漫长历史上,中国的哲学家是早慧的。我不知道,我现在没有把握断定希腊历史是不是伪史,但我大体上认同古希腊的哲学家是真实存在的。同时,我也同样地非常肯定中国的哲学家可能比罗马比古希腊古罗马的哲学家更早地完成了在各种复杂关系上的定义——陈述和定义,甚至包括在自然科学方面的陈述和定义。

整个理论的成型、成熟,其实是我们到了汉就已经非常成熟了。成熟定型之后,对旧体系的破解,王阳明算是一次努力,五百年前进行了一次努力——思想解放。但是王阳明的心学并没有构成一次对中国式儒家经院哲学的有效破解,以至于在明末清初一个复杂的历史过程中,糅合着复杂的民族矛盾、种族矛盾、阶级矛盾一系列复杂的关系。整个的明末清初的思想家们并没有给出答案。

甚至当洋人在1840年用枪炮敲开中国大门的时候,中国的思想家也没有给出答案。真正有价值的思考要到五四运动才开始生发,而那个思考至今看来依旧是缺乏哲学高度和历史的深度。所以有可能又过了一百年,现在是2021年,我们又重新来进行复杂的思考。因为我们要解释今天中国的合理性。黑格尔说的“存在即合理”,就是中国成功了、崛起了。

那么我们如何站在哲学高度来解释我们的道路?我们如何来分析和判断我们道路中的问题?我们并且以高瞻远瞩的方法来解析或者是看待我们未来可能还有79年要走的路,或者是三百年应该走的路。我挺佩服欧洲的哲学家和思想家,他们是了不起的一代人,是真正的,真正的精神的贵族。他们确实是构建了一个时代,构建了文明,构建了现代性。

我先从勒内·笛卡尔说起,笛卡尔是1596年生人,1650年过世,他是现代哲学之父,也是近代科学的始祖。那个时代的哲学家没有一个例外,全部都是科学家,其中大部分人都是对天文学,甚至对生物学极感兴趣的。直到今天,实际上一个伟大的哲学家也必须是一个天文学家。笛卡尔了不起,他的著作我今天不一一展开,我只介绍《正确的思维和发展科学真理的方法论》,简称《方法论》。他这个方法论影响了……

影响了整个现代哲学的走向。关于笛卡尔的细节还得你们自己去看,我这里边只说两句话。他说:“我思故我在”。这句话什么意思呢?其实你们都很熟悉,因为我在讲王阳明心学的时候,我提出的第一句话就是主体性。主体性在哲学里边具有根本性的意义,就是“本我”站在什么角度来看世界。这个事情非常重要,经院哲学是站在上帝的视角来看世界,不需要你思考。

笛卡尔和英国的哲学家不一样,和培根、洛克不一样。他从经验主义的反面(我们管它叫理性主义)重新进行思考,用人的眼睛看世界——我思故我在。实际上重新的定义几乎是一场革命,他破解了一直用于解释天主教的那些被教士们垄断的所谓的经院哲学,给出一个生动活泼的、崭新的对现实的解释,对历史、现实、对大自然的解释。笛卡尔说:“一切从怀疑开始”。

很多人觉得一切从怀疑开始,这好像也没什么特别,那你就错了。因为经院哲学要求一切从信仰开始,而不是怀疑。不要小看这两句话,它是打破经院哲学的石破天惊的两块石子,很了不起的。黑格尔说笛卡尔是现代哲学之父,现代的科学,当代的科学家无不将笛卡尔奉为近代科学始祖。如果你要是知道他是解析几何的创始人,你就知道这个人的历史地位该有多高了。确实是了不起,他在自然科学、哲学、文学上面都……

说个小插曲,笛卡尔自幼就多病,所以他在……他是个贵族,他们家是长袍贵族。在四百年前,在四百年前的欧洲,有点像今天我们理解的印度。它分了四个等级,最高等级是教士等级,跟印度的婆罗门是一样的。第二个等级是贵族,跟现在印度的刹帝利是一样的。他们家是贵族制,而且是长袍贵族,也就是他们处在有产并且比较富裕的状态。他的生母过世很早,他1岁的时候他的生母就去世了,然后他父亲把他交给了他的外公和外婆来养,他上……

他去了当时欧洲最好的贵族学校,欧洲四百年前的贵族学校很棒的,因为他那个学习是非常完整的,他受到了非常好的系统的教育。这个教育不仅仅包含了类似于我们的四书五经这种宗教类的课程,也受了系统的自然科学的教育。最难能可贵的,他学习了哲学和法律,他的毕业是以法律毕业的。在欧洲长袍贵族他们的学校,男生天生就要进行这种军事训练,就是骑马和击剑,所以他毕业之后做了军官,曾经在三个国家服役。

而且经历过三场战争。他可能自己本人没有直接参战,因为他是贵族,他是给一些亲王做侍从——侍从武官。然后他离开军队之后回到法国,后来因为思考一些重大问题,所以他又离开喧嚣的巴黎去了荷兰。一生没有结婚,几乎所有的哲学家都不会结婚的,一生没有结婚,跟佣人生过一个女儿。然后有一些爱情,其中有一段爱情是非常震撼的,就是他那封著名的情书,他写给……,我记不得是哪个国家的公主。

那个公主\footnote{公主指的瑞典女王Christina。故事为误传。当时还没三角函数的概念。}期待笛卡尔的来信,而笛卡尔定期会写信给她,而国王会首先拿到信件。拿到最后一封信的时候,打开,里边是一个公式,印象里是r=a (1-sinθ),那个符号应该念sin吧,就是著名的心形函数。因为解析几何是笛卡尔发明的,笛卡尔将代数和几何结合起来,构成解析几何。一如所有的哲学家,他们都是数学大师,因为哲学有时候需要用数学的语言来做解释。这个公主看了心形公式以后哭了。

也知道应该笛卡尔不在人世了,因为哲学家是这样来表述爱情的。说笛卡尔的故事呢,就好像把这个笛卡尔的思想给耽误了、耽搁了。在整个欧洲历史上,笛卡尔的方法论打开了一扇哲学的大门,大家用的就是笛卡尔的这种方法论来一层一层构建了现代哲学的大厦。他是奠基者,研究哲学是离不开笛卡尔的。

我的好多朋友在研究哲学的时候,也情不自禁地去重新学习解析几何。其实我们读完解析几何很久就忘掉了,好多朋友专门又重新去研习解析几何,做数学题。我现在没那个时间了,但是我知道非常有趣,甚至你看解析几何的时候,能够更深刻地来理解笛卡尔。哲学是研究关系的,它是一种智慧。笛卡尔其实他用他的语言和方式解释了人与神的关系,这很重要,因为解放是挣脱枷锁和束缚的一个反抗的过程。

在解释人和神的关系的时候,其实用的是现代科学。笛卡尔并不是要证明神不存在,而只是希望人和神应该有崭新的关系。在笛卡尔处理人和神的关系的时候,他也在关注贵族、教士和其他阶层的人和人的关系,以及国家与国家的关系,只不过他还没有走到那么远。但是他作为一个奠基者,他的方法论确实是一座丰碑,确实是了不起。记住他的话“我思故我在”“一切从怀疑开始”。

我们先说几句笛卡尔,我们再说卢梭。他们的时间上具有连续性。卢梭要晚,卢梭的年代是1712年到1778年。注意就是卢梭去世之后,大概过了二十几年,不到三十年马克思才诞生。卢梭是一个非常奇怪的人,他不像笛卡尔是贵族,他生活在底层,而且有一些不良嗜好,但是绝顶聪明,他的劳动也是让人叹为观止。

我这里边,他写了很多的书,我只念四部书,一个人写这四部书里边任何一部书都足以名垂史册。第一部是《论人类不平等的起源和基础》,你看,他开始关心人的问题了;二、《社会契约论》,这是一个法律问题;三、《爱弥儿》,《爱弥儿》是谈的教育;四、《忏悔录》。其实这些书大家可能比较熟悉,特别是《社会契约论》和《忏悔录》可能都熟悉,《爱弥儿》也应该熟悉。卢梭他看到了什么,他想说什么呢?卢梭在陈述文明的源泉。

人类和其他动物不一样,人类是灵长类动物。那么什么构成了文明的动能呢?卢梭认为文明产生的基础就是私有制,私有制产生了文明。知道这句话对后世的马克思的影响有多么深刻,因为私有是现代文明,我们反对私有,难道不是反对现代文明嘛?其实文明在不同的阶段,有它的局限性,有它存在的问题。将来我们在讲到辩证唯物主义的时候,会完成这个关于文明和私有制的历史辩证,卢梭说的是对的。

卢梭看到了人类天生的就是不平等的,但是不平等的人应拥有相等的权利。就是人天生是不平等的,但人又应该拥有相等的权利,这就是人权概念的提出。卢梭的存在深刻地影响了法国人,法国大革命的哲学基础或者是理论基础就是源于卢梭。卢梭不光影响了法国人,也影响了整个的欧洲,或者是影响了整个资本主义的历史进程。他在提出人权理念的基础上,提出了《社会契约论》,其实这是制度构建的……

《社会契约论》其实是法哲学的起点,到了康德和黑格尔的时候,我们会讲到历史哲学和法哲学,《社会契约论》实际上是法哲学的起点。人与人之间、人与政府之间、国家与国家之间,在处理各种关系的时候,需要有一种方法。当这种方法变成一种在某种形式下形成的契约的时候,可能是通过战争的方式,也可能通过谈判的方式,有可能通过某种交易的方式形成契约,那么就进入到了现代性的这样一个状态。真是了不起的哲学家。

作为父母,将来你们有空可以看一下《爱弥儿》,因为在谈教育问题,实际上是谈人性的问题的。人应该怎样教育?怎样是自然的、天性的?如何在自然天性和人为之间达至人格的平衡?《爱弥儿》这本书有它独到的视角。作为一个伟大的哲学家,卢梭在处理复杂问题上确实是了不起,每一个视角形成的著作都足以登峰造极。他的个人生活,卢梭的个人生活是比较随意的,或者是饱受诟病的,他有很多段的感情生活,有很多段的爱情。

他在活着的时候,批评和诟病比较多,等他死了以后才发现他是不可超越、如此伟大的人物,在各个领域都很了不起,后来他被法国的国王重新安葬。现在你去巴黎,你一定会去这个先贤祠(叫集贤祠还是先贤祠),你可以看到卢梭,看到卢梭。笛卡尔葬在另外一个教堂。去巴黎的时候可以去拜访拜访笛卡尔和卢梭。

我们说一下他们这个层层叠叠的关系。在笛卡尔他们在向经院哲学发起猛烈攻击之后,开始到了卢梭他们这一代人的时候,开始解决社会上的不同分层之间的关系。卢梭提出了人权的理念,这个就不单纯是针对经院哲学,其实这是反封建的号角。开始进入到除了人与神的关系,进入到人与人的关系,甚至初步提出了“社会”这样的一个理念,记着《社会契约论》提出“社会”这样的理念。

这个理念与后边的社会主义、社会主义革命产生了某种的必然的联系,或者是这个概念为以后的社会主义奠定了哲学的基石、基础。让·雅克·卢梭在他的那个时代,1712年到1778年这样的一个时代,实际上是法国完成工业化进入现代文明的转折时期,遇到了诸多的问题,卢梭完成了他的思考,并提出了一个基本的方向,甚至提出了初步的解决方案,这是非常了不起的。

当然卢梭的理论也酿成了大革命。当人民知道人生而平等,应该有权利的时候,他们有可能是通过暴力的方式来争取,来争取一个他们认为更好的社会契约。说到卢梭,其实我是有点小兴奋的,倒不是因为卢这个字,是因为他的思考的角度。他的思考的角度差不多是为后来,包括在马克思前的空想社会主义者和社会主义者提供了一个坚实的基础。

关于他者——第三方的存在,我们一直知道在欧洲的旧有的等级里边,教士、贵族、资产阶级、地主(资产阶级和地主差不多一个阶级),还有就是自耕农和工人(手工业的工人)。在以往的时候,我们只有一个阶层或者阶级的概念,我们并未形成完整的一种社会的概念。那么社会是怎样构成的?社会与阶层里边的关系或者阶层在社会中的关系,它与国家的关系,这是一个非常复杂的哲学问题。如何地理解和定义?如何将阶层之间的矛盾做一种妥协和处理?

其实《论人类不平等的起源和基础》《社会契约论》《爱弥儿》和《忏悔录》构成了卢梭的完整的体系。其中《忏悔录》是一个申辩词,不是忏悔,因为卢梭在他的晚年一直遭到诟病和迫害,他在做一个关于他的理论的解说。我年轻的时候读过《社会契约论》和《忏悔录》,这个对我的震撼是极为深刻的。如果大家有时间,可以看一下卢梭的传记,或者是再读一下他有关的著作。

好,我们进入到第三位哲学家,他就是德国古典哲学的开创者,他就是康德,伊曼努尔·康德,他的时间是1724年-1804年。马克思出生是1818年,就是他是在卢梭之后马克思之前,卢梭是1778年去世,他是1724年出生。康德,我们认识他是通过他的“三个批判”开始的。

《纯粹理性批判》《实践理性批判》《批判力批判》——康德的“三个批判”构建了康德的哲学体系。康德这个人也是很有意思的。康德这个人他在整个生活中,他生活非常有规律,他甚至就没有离开过他生活的小镇。就是如果写康德的传记是没有办法写的,因为他没有生活,他一直局限在一个小地方,他甚至每天下午三点钟准时去到公园散步。他的散步都成为当地人的像闹钟一样的,就是看到他散步,哦,到点了,到三点了。唯一的一次那个时钟被打破,就是因为《爱弥儿》,他读到卢梭的《爱弥儿》的时候。\footnote{“康德是一个生活习惯十分有规律的人,大家惯常根据他做保健散步经过各人门前的时间来对表,但是有一回他的时间表打乱了几天;那是他在读《爱弥儿》的时候。”——罗素(Bertrand Russell)版《西方哲学史》(下卷,P269-270)}

读《爱弥儿》的时候,他被震撼到,然后他想读完,所以没有出去散步。那么康德的贡献是怎样的呢?你如果知道世界公民和世界联邦的提法是由康德提出的,你可以想见他多么宏大的视野。他甚至在他的时代就提出了“不干涉内政”的主权国家的概念。康德和黑格尔具有继承性,就是他们俩的时间也比较晚近。作为德国人对德意志民族的思考,他们开始思考德意志民族和德意志民族国家的……

一会儿我们讲到黑格尔,我会讲一句话,就是黑格尔应该算是德意志国家的精神的接生婆,就是他在哲学上为德意志民族国家进行了奠基。康德在这个时候为什么要进行“实践理性的批判”?其中《实践理性批判》是非常重要的一部著作,他对后世的马克思的影响极为深刻,这“三大批判”:《纯粹理性批判》、《实践理性批判》、《判断力批判》,其中《实践理性批判》对马克思的影响是极为深刻地。康德最伟大的贡献是他将哲学的方法做了一次历史性的……

他说,人要为自然立法。事物的特性与观察者有关,貌似听上去这有点主观唯心主义的感觉,实际上,它打破了旧有哲学思考的角度,他算是一次哥白尼式的革命,他把他者、我者和他者和第三者的关系做了调整。他将英国的经验主义和法国的理性主义做了调和,这个调和构成了康德独特的哲学体系。

那么,我们通常在认识一个问题的时候,经验很重要,就是我们去尝过梨子的滋味——经验。在经验主义的基础上进行哲学的归纳和概括,就变成理性主义了。就是理性主义是通过逻辑来构建的一整套的体系,用逻辑来进行对未知事物的判断。当经验主义与理性主义相结合的时候,就变成了一种比较符合现实的哲学。在某种意义上,康德也正在挣脱固有的逻辑枷锁,为德意志民族国家寻找出路。

经验主义与理性主义的融合的过程,其实是一种剧烈的碰撞。如果你们读过《实践论》,毛泽东写的《实践论》,我们在心学里边也讲过,这里边对经验主义和理性主义有过精彩的论述。这也构成历史唯物主义和辩证唯物主义里边的基本的工具和方法。关于康德的生活和他的主要的著述,我今天这里也不多讲,因为时间不够,你们自己看一下子。这两个德国哲学家和那两个法国哲学家是完全不一样的。德国的哲学家是比较苦的。

这两个德国哲学家也没那两个法国哲学家那么有钱、那么浪漫、那么好玩儿,甚至也没有那两个法国哲学家那么有才华。好,我们讲几句黑格尔。黑格尔生活的年代是1770到1831年,马克思是1818年出生,有交集了,他是对马克思影响最深的哲学家。我将黑格尔定义为“政治哲学家”,因为他在政治哲学上面几乎是开山始祖,或者说,他对德国资产阶级国家的建设做了最系统、最丰富、最完整的阐述、论述。我甚至认为由于黑格尔……

由于黑格尔的历史哲学方面的杰出贡献,导致1840年到1940年,德国的思想家们与奥地利学派进行了百年论战。他的历史哲学要解释的就是德国当时被整个欧洲诟病的那样的制度。其实当时的情况很像是今天的中国,所以他提出了一句话,你们大家都熟悉:“存在即合理,合理即现实”,他这个哲学的命题,是想说德意志民族国家的最后的建立,应该是在他去世之后,大概在一八……

俾斯麦打败了这个拿破仑三世之后,正式的建立了德意志第二帝国。那个国家大体上就是黑格尔心中所想象的那样的一个国家。我们上次讲俾斯麦的时候,我说过,不要用单一色彩来看一个人,因为我们今天所看到的社会保障体系,对劳工的社会保障体系,开山始祖是俾斯麦,铁血宰相俾斯麦建立了社保制度,是不是很有趣?当然这里边有黑格尔和马克思他们的在哲学上的努力,在思想上的努力,最后深刻地影响到制度设计。

黑格尔的书里边《小逻辑》是值得读的。如没有太多时间的话,那么今天只推荐《小逻辑》,这本书要读一下子,当然读不懂,翻一翻也是好的。至于《精神现象学》或者是《逻辑学》,我看可能量太大,《哲学全书》里边的《小逻辑》读了也就可以了。黑格尔的历史哲学和法哲学对现代资本主义国家,构成了非常深刻地影响,或者说现代的资本主义的伦理基础和法理基础,是从黑格尔这儿延伸出来的。你知道我们理解当代(现代)资本主义是必须回到黑格尔的。

黑格尔对国家----因为他是政治哲学家,他对国家有着极为深切地敏感。因为哲学要讨论的关系里边,我刚才说了:人与自己、人与社会、人于自然、人与神,人与社会的关系里边涉及到的就是国家。国家是工具,国家是载体,国家既是道德伦理的载体,也是建基于道德伦理上法理的载体。国家是国家里边、社会里边的人。

国家是民族,或者是这个同属一个民族的不同阶级达成某种意愿----政治意愿、经济意愿的工具。也就是说,无产阶级要实现自己的社会主义理想,也要借助国家这样的工具。因为社会主义无法在一个非国家的体系内建立它完整的法律体系。

马克思在黑格尔的基础上,开始由民族和国家、深入到了阶级。马克思开始进行对无产阶级与资产阶级、与国家、与民族进行了所有关系的思考,其中《资本论》是从经济层面进行的思考,在思考这个问题的时候,使用的哲学工具,大体上用的是黑格尔的工具,从黑格尔来继承的,不管是历史唯物主义还是辩证唯物主义,很多都是从哲学的体系上延伸出来的。

我们在将来要介绍马克思的这个思想形成的过程中呢,可能两个部分非常重要:一个是,他是一个年轻时候是一个黑格尔的崇拜者,并且是一个非常虔诚的黑格尔信徒,这是第一条;第二条,他又是一个基督教新教——信义宗的虔诚的信徒,这两种东西糅合,糅合之后形成了青年马克思。你如果不能理解马克思所处的德国的历史的状况,你不能理解青年的马克思是如何地读黑格尔,又如何地受基督教信义宗的影响,你就无法理解《共产党宣言》。

说到这里呢,我得跑跑题了。我记得我念叨过,就是《资本论》,马克思第一卷写的是价值论,因为谈经济你必须从价值论开始。马克思创造了剩余价值的整个的理论体系,在讲述资本的源泉。第二卷,马克思在谈资本流转,就是资本是怎样流转的。价值论实际上谈的是资本的诞生,第二卷谈资本的流转。我想马克思要写第三卷,应该写国家与资本的关系,或者说无产阶级国家与资本的关系,但是马克思写不下去了。

十来年的时间写不下去。我的看法,马克思对当时代的无产阶级失望乃至于绝望。所以他认为无产阶级在那个时代——就是十九世纪中叶——他认为十九世纪中叶的无产阶级是不能建立一个社会主义国家的。你要如果懂了历史哲学和法哲学,你就知道一个阶级需要一个国家作为他的伦理和法理的载体,那么这个伦理和法理这个构建的过程中,马克思看到了问题。马克思没有完成这项工作,这项工作到今天我们也没……

马克思没写资本论第三卷,但革命发展的速度非常迅猛,需要一部书啊,所以列宁写了《国家与革命》。我认为列宁写的《国家与革命》就是资本论第三卷,列宁在讲述无产阶级国家应该是个什么样子。很遗憾,我今天要这样的说:列宁是一个共产主义者,但他写出来的那本小册子以及他构建的苏维埃政权,那不是马克思意义上的社会主义国家,马克思的社会主义国家不是这个样子。列宁构建的苏维埃社会主义国家是完整的国家资本主义,它是国家掌控一切资本的这样的一个……

苏联构建了一个完整的国家资本主义,它被叫做社会主义。在列宁之后,斯大林将它固化完成一种社会主义模式,或者是叫国家资本主义模式。今天看来,它和社会主义相去甚远,它只不过是由国家掌控资本来形成的一种状态。从哲学的角度来讲,无产阶级建立一个国家作为载体,使无产阶级或者是使无产阶级所处的这个民族迅速地完成工业化,并且开始进入现代,拥有现代性。按照黑格尔的逻辑“存在即合理”,那么它肯定是合理的,有它历史的合理性。

这种模式,被发展中国家的思想家认识并接纳了。中国呢?十月革命一声炮响,给中国送来了马克思列宁主义。马克思主义送到中国来了没有?今天还在存疑,就是因为能读懂马克思的人不多。但是确实送来了列宁主义,就是苏联的苏维埃模式,以苏维埃模式建立一个国家资本主义,这种想法是被中国那一代人接受了。“五四运动”之后的左的这批人就开始思考通过革命战争,三次土地革命,还有就是土改,还有就是工商业改造……

我们构建了新中国、社会主义新中国。但实际上我们构建的是一种与苏联国家资本主义极为类似的,有不同,极为类似的国家资本主义模式。这个资本主义模式也非常非常让我们感到兴奋的是,它在短短的二十八年时间就完成了工业化,又用了四十多年的时间,使中国迅速地完成了现代化,迅速完成了崛起。中国现在已经成为全世界第二大经济体。我们在各个方面发展都非常好,用黑格尔的话“存在即合理,合理即现实”,那么中国是做对了的,只不过是当代中国人说不清。

那么我们中国并没有出伟大的思想家或者哲学家,我们更多的有点像英国。我们不是理性主义,我们是经验主义。我们的经验主义来自于我们内心深处的直觉,这个有点意思吧,这个有点像康德的论述。毛泽东、刘少奇都在六十年代,毛泽东是带了经济学家去杭州读苏联《政治经济学教科书》,刘少奇是带了人去海南读《政治经济学教科书》。还有两个人啊,一个是陈云,一个是邓小平,一定在读,但不一定带人读,他们都在读苏联的《政治经济学教科书》。为什么要读苏联《政治经济学教科书》呢?因为他们认为有问题。

就是说在上个世纪六十年代,我们中国共产党的高级领导同志都在思考,深刻思考国家资本主义的缺陷。他们发现了这个缺陷没有?发现了。发现这个缺陷的解决思路是不一样的。国家资本主义最有效的整合了资源,因为这个整合资源的过程是理性主义。你记住理性主义永远是残酷的,革命就是理性主义,土改也是理性主义,工商业改造也是理性主义。因为它是最理性的,必须经过资产重组和劳动力最有效的配合,就是资源、资本、劳动力最有效的配置才能工业化。

所以那个时代的共产主义者或者社会主义者也是理性主义者,他们将理性主义发挥到极致,以至于我们现在还有很多文学家在搞伤痕文学。比如说方方写的是《软埋》,比如他们拍的电影《芳华》。诸多控诉——控诉土改、控诉文革、控诉革命、控诉工商业改造,就是控诉国家资本主义的大量的文学作品,在东方和西方大量这样的东西出来。但是他不是哲学家,他无法理解“存在即合理”意味着什么。因为没有这个我们没有办法走到今天,但是有了这个,我们又要知道它的局限性和它的问题,这才是哲学家存在的价值。

毛泽东发现了问题,他认为国家资本主义将扼杀整个国家的政治和经济的活力。他看得很准,一点儿错误都没有,而且国家资本主义不可逆、不可避免,这是自然规律,会走向官僚垄断资本主义,国家资本主义一定会走向官僚垄断资本主义,因为这个温床只能长出那个苗子。这是毛泽东看得非常清楚,他说资产阶级就在党内,走资派还在走,好多人理解不了。其实他已经完整地看到了历史的趋势,只不过是解决方案没想好。解决方案应该就是马克思的社会主义,当时他对无产阶级希望是无产阶级来治理国家,但是我们只能通过执政党、代理人来做这件事。

那么如何避免这个被授权委托的执政党或者是执政者不僭越人民的主权而形成官僚垄断资产阶级呢?毛泽东的办法是文化大革命,他让普通老百姓去夺权。那个时候没有想透应该夺立法权,但是人民起来了以后,他就会砸烂公检法——司法权,组织革委会——行政权,他全给夺了。夺了以后呢,人民——马克思的担心立刻出现。巴黎公社的教训就是人民夺权之后无法完成有效治理,所以文革以失败而告终。它是一次伟大的实验,但它实验证明此路不通。

还有一个伟大的思考者,这个思考者叫邓小平。我去江西南昌总会想去看看那个邓小平小道,就是他在那个地方完成了对国家资本主义的思考,他知道国家资本主义这样做不行,所以他觉得需要引入社会资本主义,就是国家资本主义、社会资本主义相融合,才是能够最大限度调动劳动者积极性和最大限度提高资本的运营效率这样的一种方式。请记住,改革就是允许社会资本进入,和国家资本相融合,进入一种混合的状况,允许社会资本进入国家资本主义。开放是...

允许外国资本进入到经济体系,也是社会资本,就是两种不同的社会资本进入到国家资本主义体系内,与国家资本主义体系相融合,形成一种新的混合资本主义。这个混合资本主义仍然不是马克思本意上所说的社会主义,但它已经开始接近了,它具有非常了不起的先进性。不要认为国有企业或国家资本主义一定是坏的,也不一定要认为社会资本主义一定是好的。当它们有机结合的时候,奇迹发生了。

知道我今天睡不着觉,如此激动的原因了吧?因为《资本论》的第四卷该写啦,能写第四卷的应该是中国人,我们需要伟大的思想家和哲学家来完成《资本论》的第四卷。第四卷可以“俗”一些,用新社会主义论这样的一个角度来写,我们现在把它定义为有新时代、有中国特色的社会主义。新时代是肯定是新时代了,有中国特色其实就是混合社会主义。你知道混合,两样东西混合是有化学反应的,它还有个比例关系,孰轻孰重的关系。

它们之间应该采取一个什么方式?讲到黑格尔一激动,就插入了话题插入比较远了。好吧,我们拉回来,今天讲了四个哲学家,讲了一点儿,短暂的讲一点中国的现代当代的哲学思想史,其实我们是空白,到今天也是空白。讲了我们的一些实践,我们是经验主义先行,当然我们在整个的过往的一百年,第一个一百年,我们是理性主义者,但是我们……

我们还是做得不错的。下一个一百年,到2049年,还有二十八年时间。今天讲课的时候我想说一句话,这二十八年什么问题是最大的问题?哪里可能有颠覆性的危险?那么我要说,那就是官僚垄断资本主义。说到这里边其实挺沉重的,马克思的问题我们没解决,列宁的问题没解决,毛泽东的问题没解决,邓小平的实践到现在仍然不能画句号,不能给结论,无法完成理论概括。其实我们这代人是羞愧的。

至少我本人感到羞愧,我们在浪费大量的、宝贵的时间和精力做一些事情,可能很多事情是无意义的,而不可以放下一切,把该做的、重要的事情做完。像笛卡尔、像卢梭、像康德、像黑格尔、像马克思、像列宁,像他们一样。说到这儿感到很惭愧,昨天晚上睡不着觉,在梳理的时候其实也有深深地自责,我们在讨论《资本论》的时候,在伟人的巨著面前感到了自己的卑微,感到了自己的无能,感到了我们自己对自己的要求得太低了,太少了。

今天的课就到这儿,剩下花几分钟说一下子疫情。印度的疫情仍然在深化和严重过程中。在五月初,可能日感染人数会达到五十万或者超过五十万,死亡人数,可能日死亡的人数会上万。严重程度到什么情况呢?就是现在看来不可避免的,在整个南亚和东南亚地区扩散。中国能否守住大门其实正面临严峻的考验。

另外,年初全世界对经济的判断可能要做调整。比如说对印度的判断,年初经济学家们都认为印度今年经济增长是12.5\%,非常乐观,认为美国的经济增长可能会达到创纪录的7.5\%,中国比较保守,是6\%。我个人认为疫情非但没有结束,疫情可能会在第二波更具杀伤力。我们考虑到西班牙流感第二波的杀伤力是非常大的。在两重变性病毒,那个B1617,已经非常严重,现在可能还有三重变性病毒,就是极为严重的一种病毒,它是跨节气...

就是病毒变异的过程中会将问题变得复杂化,可能会穿透我们现在这种疫苗的防线。那么我们的疫苗如果起不了决定性的作用的话,那么年初的所有的对经济的预测都变得没有意义。中国的隔离是非常成功的,但这是一个持久战,这是个时间问题。会出现什么样的情况呢?最好的情况就是可能将整个问题局限在南亚和东南亚地带。即便是局限于南亚和东南亚地带,对全球经济的打击也会非常大,印度是……

棉花生产、纺织品生产、粮食生产、药品生产的大国,现在已经出现了问题,而且供应出了问题。如果延伸至整个南亚,巴基斯坦、孟加拉、阿富汗、尼泊尔、缅甸、泰国、柬埔寨、越南、马来西亚、印度尼西亚、菲律宾,那就不得了。他们整个占据的人口规模差不多占全球人口的三分之一。所以我们现在对经济预判是偏于悲观的,比较悲观的。也提前跟大家说一下子,我们在做投资安排的时候,要有一个审慎的思考。

另外就是节日假期,大家还是出门要做好防范工作,多加小心。虽然现在变性病毒,虽然通过印度人他们已经渗透到杭州等地,但是它还没有失去监控,就是我们现在还都是有源的,没有无源的。但是还是要高度戒备,要高度小心。我那天说了一下子,《推背图》第38象,那个门必须得死死的守住,那个门守住了,到了明年可能情形会好一些,但是今年可能情形并不太乐观或者是不容乐观吧。

今天是劳动节,大家都不容易,节日快乐。然后讲完了我得先睡会儿,我在办公室呢,我得先睡会儿,然后去吃东西。明天下午三点钟我们再打开进行交流。今天这课密度高了一些,但是应该很好的,可能我讲的有点激动吧,你们原谅我。另外就是可能有一些敏感的东西,大家帮我处理好吧。好,节日快乐,我们明天下午见。

\section{马克思生平、信息技术加冠状病毒等于社会主义吗?}

我先试一下麦,然后三点钟我们准时开始,继续《资本论》的第三讲。今天这第三讲算是非常重要的一讲,讲马克思的生平。因为大家都知道的马克思生平和我讲的生平可能差异巨大,所以也算是谈《资本论》里边非常重要的一堂课。好,三点钟我们准时见,我试一下麦,如果没有问题的话我们三点钟准时开始。

大家好,今天是2021年的5月15号,辛丑四月初四。今天我们讲《资本论》的第三讲,第三讲要介绍马克思的生平了。从第三讲开始,我们进入到实质性阶段。关于《资本论》的课程在一定的范畴引起了一些轰动吧,大家就是比较介意我这个课程,尤其是北京的朋友们会把这件事当成很大的一件事情。我知道这当中的份量,我也知道这里边蕴含了很高的难……

我们在第一讲讲了两场革命和两个宰相,实际上在讲一小段的欧洲史。我还是建议大家有时间去买那个《企鹅欧洲史》分段去读一读,那个五大本子全部读完是需要时间的,但其中涉及到我们这个时代的部分可以去读一下子,因为那个《企鹅欧洲史》相对中性。历史确实是这个在不同的角度会有不同的阐释,会产生巨大的差异,一会我们讲马克思的生平你就能看到。我们第二讲讲了四个哲学家,但是确实是太短暂,讲得不透、讲得不深。

今天我们来开始介绍马克思的生平。我知道今天可能会产生巨大的反响吧,因为我眼中的马克思和西方学者眼中的马克思和我们东方学者眼中的马克思都是不一样的。我想告诉大家,我看到了一个怎样的马克思呢?为什么我眼中的马克思和传统史料中介绍的马克思有着巨大的不同呢?这个马克思对中国的意义又是什么呢?对当代中国、当代社会主义的意义是什么?

一会我们把生平简介做完以后,大概花十分钟左右的时间,我们做一个短暂的演讲,就是谈一下子“信息技术加冠状病毒等于社会主义吗?”,做一个小小的演讲吧。因为确实很有趣,信息技术和冠状病毒确实导致这个世界产生了强烈的化学反应。当然了,这里边可能也涉及到未来的经济的一个走势、一个判断,可能对投资也有一定的意义。我们,其实是我将一些的思考和准备的课程——准备的给大学的公开课程先给大家做一些分享。

好,切入主题,我们谈谈马克思的出身。马克思的出身让我想到了在十八世纪、十九世纪的特殊人群,就是在欧洲的,特别是在西欧的那部分犹太人,他们非常非常特别。而且这部分犹太人对当代史或者是对现代性构成了不可估量的影响,其中当然包括马克思,也包括了你们所熟知的那些伟大的思想家和科学家们,他们的确是一群很奇怪的人,为什么会产生这样一种现象呢?

也许马克思的身世里边带有某种的启发性的意义。我一直在说《资本论》是一本学术著作,但“马克思主义”却不完全是学术,如果你非要说“共产主义”是学术的话,那么我必须清楚地再次说明,“共产主义”更加接近一种宗教,或者是它更像一种宗教。因为它是一种信仰,它不是一个哲学推论的结果,它是一种信仰,而且这种信仰可以改变一个人,它可以使人纯粹、高尚,甚至高大起来。

马克思出身于一个非常传统的犹太人的家庭,因为他的父亲、母亲都是犹太人,而且马克思的祖父十几代都是拉比\footnote{拉比(רַבִּי‎, Rabbi),是犹太人的特别阶层,主要为有学问的学者,是老师,也是智者的象征。犹太人的拉比社会功能广泛,尤其在宗教担当重要角色,为许多犹太教仪式的主持 (但他们多有日常正职)。因此,拉比的社会地位十分尊崇,连君王也经常邀请拉比进宫教导。在犹太人的宗教经典《塔木德》,就经常提及拉比的事迹。拉比是老师的意思,是智者的象征,是 “可以去请教的人”,他们经常与常人接触,解答他们的疑惑。他们是一群观察生活,思考生活从而获得智慧的人。},而且是非常了不起的拉比。多说一句,马克思的外祖父也是拉比,也就是说马克思的血液中就渗透着拉比的血液,或者是他身上天然具有拉比的气质。其实你们仔细看马克思是不是更像一个伟大的拉比?当然他也是伟大的思想家,因为拉比必须是拥有思想的人。

马克思出生的前一年(马克思1818年出生),马克思出生的前一年,他所出生的那个地方——特里尔,才从法国人手上归还到普鲁士。也就是说在马克思出生前一年,他们那个地方还是属于法国的。而在那个时代,法国是非常先进的,拥有进步思想、进步理论的那么个地方。所以我们在介绍哲学家的时候,讲了笛卡尔和卢梭,当然也讲了康德和黑格尔,他们对马克思构成了非常重要的影响。而马克思身上确实具有十八世纪法国思想家的那一种的气质。

特里尔这个地方很有趣,它属于莱茵地区,是古罗马帝国西部的首都,它在欧洲的历史上面是一个非常重要的一个地点,而且它也是相当于古都,相当于、类似于像中国的开封、咸阳这样的一个地方吧。特里尔是一个文明交汇、杂合、交融的地方,它又是德法汇聚的地方。由于生长在一个特殊的家庭,所以马克思是熟悉希伯来语,他又从父亲那里受到了很好的拉丁语、希腊语的这种教育。

当然他也熟悉法语和德语,以至于英语他也熟悉。语言非常重要。因为对一种语言的把握其实是非常关键的,因为在很多时候语言代表着一种,它不仅仅是一种知识的载体,它也包含了整个的哲学和逻辑的演进的过程。所以我们注意到所有伟大的哲学家在语言方面都是大师,他们能够很好地理解语言、运用语言来处理极为复杂的概念和逻辑。

马克思的父亲是一个非常特别的人,马克思的父亲因为他不再继承马克思祖父的那部分的、大部分的职业上面的选择。他作为一个非常厉害律师,非常了不起的律师,在西方的文献上,认为马克思的父亲迫于法律上面的压力,他改信了基督教新教信义宗,放弃了传统的犹太教,我不完全这样来理解这件事情。

马克思的父亲思想是非常前卫的,他深受法国的革命的影响,对法国的一些新鲜的思想、先进的文化他有着极为敏锐的观察和深刻的理解。亨利希·马克思,后来改名海因里希,改了一个德国名字。多数人认为他是受制于当时的特定的形势,所以他必须改变宗教信仰以从业,以满足从业的需求。

我不是完全接受这种说法,因为马克思的父亲和他们家一生交往的好朋友——就是路德维希·冯·威斯特法伦,就是燕妮的爸爸,马克思夫人燕妮的爸爸,他属于德国的一个破落的贵族,他们都是非常非常前卫的。在马克思出生之前,他们应该算是一群非常反叛的青年,他们渴望的拿破仑的法国的革命和给欧洲带来的解放。那么如果谈到马克思的话,不得不说一下少年的马克思。因为马克思没有去上小学,他是在家完成小学学业的,而他的老师就是他的父亲。现在记载就是马克思的父亲每晚睡前都会给马克思读伏尔泰的作品,大家应该知道伏尔泰。同时他教授了马克思拉丁文和希腊文和法文和德文,看来马克思的父亲对马克思的教育是非常在意的。(要想)要知道马克思家里边是九个孩子,马克思排行第三。这个教育可能对马克思具有非常重要的意义。

少年时候的马克思除了在家里受教育以外,他还经常去他的邻居家,就是我们说的燕妮的父亲。燕妮的父亲是一个非常活泼而有活力的这样的一个人,他会给马克思他们这帮孩子讲希腊的故事和一个人去演莎士比亚的作品,讲莎士比亚的作品。也就是说少年马克思成长的氛围是非常有趣的,而且是非常幸福的。再加上少年的马克思的家算是生活不错,他们家有一个葡萄庄园,有一个不错的房子。

将来大家有空去德国旅游的时候,记着去两个地方,一个是特里尔,就是马克思的家乡,一个是去佛莱堡。我上次去德国的时候很遗憾,来不及,这两个地方没有去看。等疫情结束了,我们再去走访一下,再去看看。我这里边重点要说的是马克思的中学生活。马克思1818年出生,1824年完成洗礼,正式成为基督教信义宗的信徒,这件事情其实非常重要。少年的马克思乃至于到了中学时代的马克思,其实是一个非常虔诚的基督教信义宗……

我看过另外的一些对马克思的介绍,因为他,当然那个时候的欧洲的好的学校都是教会学校,他就读的学校也不例外,是一个非常传统的教会学校,可能也是整个西欧最好的学校、中学之一吧。在搜集马克思生平的时候,其实两样东西还是震撼到我。一个东西是他的一篇宗教作文,这个题目是“根据《约翰福音》第15章第1至14节论信徒同基督结合为一体,这种结合的原因和实质,它的绝对必要性和作用”。

作文里马克思这样说的:“这个人类的伟大教师向我们表明了从古代以来人的本性一直是在把自己提升到一个更高的道德水平,由此可见,各民族的历史告诉我们同基督一致的必要性。但是在我们研究各个人的历史、人的本性的时候,我们虽然也看到他心中有神性的火花、好善的热情、求知的欲望、对真理的渴望。虽然罪恶的引诱会吞没这些自然的本性,但信徒与基督的一致能够克服这些罪恶的引诱,并提供一种快乐。”

其实我知道,少年的马克思其实在信义宗里边获得了以后他人生的基本的方向,或者是支撑他走完一生的那个力量的源泉。我一向认为共产主义不是诞生于康德、黑格尔、圣西门——跟他们有关系,但它的源泉就是基督教信义宗。请记住我的话,共产主义是信仰,它的源泉源于基督教信义宗,只不过它被一个非常伟大的基督教信义宗的信徒将它进行了某种异化,这可能是颠覆性的一个看法。

我再给你们念一段马克思中学毕业的时候的一份作文的内容,这个作文的题目叫《青年在选择职业时的考虑》,这是很有名的一篇作文。我想你们大家了解马克思的人都知道这篇作文,我把它这最后一段话念出来给你们听:“历史承认那些为共同目标劳动,因而自己变得高尚的人是伟大人物;经验赞美那些为大多数人带来幸福的人是最幸福的人;宗教本身也教诲我们,人人景仰的理想人物就曾为人类牺牲了自己,有谁敢否定这类教诲呢?”你听到了马克思在说什么了,这是一个中学生啊。

再听后边的话:“如果我们选择了最能为人类福利而劳动的职业,那么重担就不能把我们压倒,因为这是为大家而献身;那时我们所感到的就不是可怜的、有限的、自私的乐趣,我们的幸福将属于千百万人,我们的事业将默默地,但是永恒发挥作用的存在下去,而面对我们的骨灰,高尚的人们将洒下热泪。”我想很多人读到这一段的时候会眼眶潮湿的。其实,我每一次去伦敦都会去马克思墓上献一束花,回荡在耳边的就是这段话。

在1848年马克思写《共产党宣言》的时候,其实我更加的多的认为那是一个建基于基督教信义宗的宣言。所以我想今天多说两句信义宗的事情,这个事情我现在还没有足够的证据完成最后的证明,这个事情----我希望我们大家一起来把这件事情完成历史的考证,完成历史的考证,将这段历史把它完整了。因为它对中国人、对社会主义、对我们的信仰有着无比重要的意义,它可能也会改变我们很多人的一生。

信义宗发端于欧洲的宗教改革,是1529年马丁路德创立于德国的。当然信义宗是讲“因信称义”。在1853年他们编成了《协同书》,里边包含了《尼西亚信经》、《使徒信经》、《亚大纳西信经》,算三大信经,它为核心体系的一个信义宗。为什么会发生宗教改革呢?或者我们再问一句,为什么会发生文艺复兴呢?文艺能够复兴吗?宗教可以改革的吗?到底怎样的?

欧洲在经历了希腊和罗马之后,开始进入了严重的内卷——所谓的中世纪的黑暗。中世纪的黑暗被谁打破了?当然不是欧洲本土的思想家——经过了黑暗,然后找到了光明,不是的。如果你看时间的节点,它恰好是明朝,是我们中国也在正好经历蒙古人入侵的苦难,那么蒙古人也同时入侵到了欧洲。是蒙古的铁骑将东方文明载入了欧洲,它触发了欧洲的文艺复兴和宗教改革。

你读中国元朝的历史你会注意到,因为蒙古人将当时的人分为四等。第一等就是蒙古人;第二等是色目人,色目人主要的就是白人,里边包含了一部分的欧洲的白人,也包含了一部分的信仰伊斯兰教的中东的白人,总之他们统一称为色目人;第三类人称为北人或者是汉人,其实是北方较早被同化的……

被同化的,当然是、可能是以汉人为主体,也包括了高丽人,包括了其他的一些民族等等吧,通常用北人或者汉人来称第三类人;第四类人才是南人,就是南宋的人,他们是最底层,地位接近于奴隶。\footnote{“四等人制”是民国书籍中首次提出的。迄今为止并没有发现任何元代有把臣民明确划分为四等的法令和史料。}我为什么要说这四种人呢?因为蒙古人将管理国家的事情交给了色目人,交给了白人,大量从中东和欧洲进入到元朝的治理者大部分都是色目人。他们一方面在进入到中原,进入到大中华地区,另外一方面他们也有序将中华文明带回了……

他们有序将中华文明带回了欧洲。色目人将中华文明带回了欧洲,加上印刷术的普及,他们开始将文明新鲜的“东方文明”开始在欧洲扩散开来。由于蒙古人当时非常崇尚的……当时他们给孔子封了帝,就是非常崇尚中国的儒家。其中色目人最感兴趣的部分——儒家的经典就是《孟子》,《孟子》的“民本”的思想与当时的色目人的想法高度吻合,形成了对文化和宗教的猛烈……

我认为基督教新教信义宗,源泉就是孟子,“因信称义”就是源于孟子。当然欧洲中世纪以后的所有的伟大思想家都对“东方文明”致以极高的敬意,甚至是崇拜。只是到了近现代以来,出现了一种对“东方文明”的否定,甚至撕裂、隔断。我们对西方也有很多的怀疑,比如说对古希腊、古罗马的一些事情,也很多人提出质疑。但我对文艺复兴和宗教改革……

我对西方的文艺复兴和宗教改革给出的答案就是这样。所以,我经常在外边给朋友们题字的时候,就是题“让我们从欧洲迎回孟子”。由于信义宗而延伸出来的共产主义重返东方的时候,它并不违和,原因在于:它其实在某种程度上是儒家精神内核在欧洲的生发。它重返东方的时候,使我们的东方的已经被异化了的儒家,再现出勃勃的生机,所以当毛泽东他们这一代人……

当毛泽东他们这一代人,将共产主义思想在中华大地上重新树立起来的时候,我们从不感到有违和的问题。因为它作为一种信仰,或者是作为一种哲学,它跟我们具有某种同质性。说到这里呢,我们还是需要做大量的工作的,因为整个的考古和考证的过程还需要更加的细致和细腻。当然有一些哲学家和思想家是认同这样一个过程的。文明嘛,本身就是互相地砥砺的,互相地促进的,互相地影响的。

我记得我一直在说,哲学研究的就是关系——人与自己、人与社会、人与自然、人与神的关系。研究人与神的关系就涉及到信仰问题了,我们在马克思身上看到了一种使徒的使命感。我刚才念了两段他的作文,少年时期的马克思的作文,我们看到了强烈的使徒的使命感。其实在理解马克思的时候,要理解三个马克思——少年马克思、青年马克思和晚年的马克思。要理解三个马克思,这三个马克思是不太一样的。共产主义信仰是在少年时期慢慢地生发,到青年时期,到《共产党宣言》形成。

而《资本论》呢,是这种信仰如何在社会、在资本主义社会建立起来,并让它成为一种社会实践的一种缜密的思考。所以我们讲《资本论》,很多人说《资本论》是工人阶级的圣经,其实真正读懂《资本论》,它不仅仅是工人阶级的圣经,它也是投资者的圣经。当你知道资本是如何流转的时候,其实你应该知道了很多很多东西,但当你真的开始学习《资本论》的时候,可能你会成为一个坚定的共产主义者,至少是一个社会主义者,是一个干净的社会主义者,而不是庸俗的资本家。

在讨论马克思的成长历程中,马克思的少年时期是非常关键的。因为我自己从他的作文里边能体会出,无论是西方介绍马克思的那些文献,还是东方介绍马克思的文献,可能没有抓住要害,没有抓住要害,没有抓住共产主义诞生和产生的真正的土壤在哪里。很多朋友说,您将信义宗作为共产主义或者是马克思主义产生的基本的土壤,这个可能已经是已经非常大胆了,您再将信义宗与中国的儒家和孟子结合起来,这件事情就有点颠覆了。

这也不是大胆和颠覆的问题,我个人认为仍然需要足够的证据来证明,而我现在的能力不够,时间也不够。但只要条件允许,我们会把这件事情做完的,也希望大家跟我一起,我们一起把这个事情做完。它不是个大胆的问题,它也不是个颠覆的问题,因为我们这代人必须把这段历史搞清楚。搞清楚这段历史,其实对中国的意义,对中国的道路问题,我们反复在讲道路问题,习主席提出来“新时代有中国特色的社会主义理论”——“新社会主义论”,我们必须把它的始末和未来的方向搞透了。

青年的马克思提出了异化的,劳动异化的理论,他是跟黑格尔的异化论,是一种延伸或者是一种进步,也可以说是一种颠覆。“异化理论”非常重要,其实每一个人在青年时期,都会遇到异化理论的问题。我不知道你们在二三十岁的时候经历过什么,我自己的经历我知道——就是否定再否定的这整个的过程,会非常激烈地进行,整个的这个过程是推倒、重建,颠覆、重建的反复进行的一个过程。

有人问我,你凭什么可以这样的来讲马克思的生平?我说凭我也曾经是一个少年,也曾经历过痛苦的青年,我知道一个人信仰是在什么时候开始生成的,我也知道整个他的信仰、甚至信念、甚至他的思想、甚至他的学问是在什么时候、什么状态下被锤炼,被反复地修理、磨炼而慢慢地生成的。所以我自己的过程,我能够设身处地地去理解马克思的少年和青年,整个的思想的变化的历程。

马克思,大学是父亲送他去波恩大学,用西方人的对马克思的评论,他们就说这个酗酒、夜不归宿,就是不好好读书,还跟人决斗的青年马克思,显然对学法律的兴趣不是很浓厚,后来被他父亲强制转学至柏林大学。在柏林大学,马克思终于放下了法律,开始转入哲学,并且在柏林大学参与了一些激进的社团。马克思用很短的时间——三年左右的时间,就完成了哲学博士的论文,成为最年轻的哲学博士,并且拿着这个博士论文去燕妮家……

马克思的家族,还有燕妮家的这个家族很有意思。马克思的父母,马克思的父亲不算是富有。其实马克思的父亲已经比他祖父强了,祖父生活是比较贫穷的。到马克思的父亲,因为他是律师嘛,他生活得到了改善。马克思的母亲是有钱人家,所以马克思母亲的陪嫁相当于一个普通工人四十年的工资。那个陪嫁的量很大,就是马克思母亲的陪嫁可以买下一个庄园,一个小型的庄园。马克思的母亲的家族是菲利普家族。

就是飞利浦电器的创始人,就是马克思的母亲的家族。不是一般的有钱,是非常的有钱。以至于青年马克思甚至到了、已经到了中晚年的马克思,仍然在向他的舅舅在请求贷款,就是他还得向他的舅舅借钱。马克思的母亲也算长寿,一直活到了好像是1863年吧,也就是说,她应该看到马克思最艰辛的时刻,被一遍一遍地驱逐,生活穷困潦倒。燕妮家也算是有钱人。

但是马克思选择这条路呢,他就不是个挣钱的路子。因为研究完,他成为哲学博士之后,他就只剩下一支笔了,所以他去莱茵报做主笔、做编辑。后来莱茵报太前卫了,又被查封。查封完了以后,他从此以后就失去了正规的、稳定的收入来源,没有现金流了。而作家真的是很辛苦的,我在香港从事写作二十几快三十年了,两个字一块钱。就是你每天写一千字可以有五百港币,三十年前就这个价,现在还是这个价。你要是靠写作为生,那就非常非常的痛苦和艰难了。

后来呢,加上马克思除了写作之外,还积极参加社会运动。其实最早的马克思主义政党就是由马克思和恩格斯指导建立的,因为《共产党宣言》就是为了第一个共产主义政党而写的一个大纲,建党的纲领。以后呢,第一国际的建立也是在马克思的指导下建立起来的,而马克思用他的最主要的时间和精力去从事研究和写作,他没有办法将他的劳动转化为现金流,转化为钱。所以在相当长的时间之内,他要靠恩格斯接济。

被人接济是非常痛苦的事情,是很无颜面的事情。燕妮在自己的孩子——亲爱的长子死后,为了葬这个孩子,她开口向法国的在英国的流亡者去借钱,借了两英镑,给他安排一块墓地,把他简陋地安葬下去。你知道那种借贷过程中的那种辛酸吗?很多人可能不知道,在那种贫病、潦倒的状况下的写作那种状况,没有强大的信仰的支撑,几乎是不可能的。

我自己的体会就是像我这样的人活到这个时代,有的时候仍然无法超越生活的现实,无法摆脱这种俗气、世俗,依然在耗费巨大的时间和精力在商业上面。因为有的时候生活是具体的、现实的,是沉重的,就是无法做到像马克思那样的投入。所以我真诚地热爱马克思。但我很崇拜凯恩斯,因为凯恩斯,就是目前我们知道的投资高手里边可能没有人能超过凯恩斯。他们是两个伟大的经济学家,当然马克思和凯恩斯是不同的。\footnote{凯恩斯:1928 年 -1945 年,18年间相对英国市场的超额年化回报率为8\%。巴菲特:1976年-2017年,41年间相对美国市场的超额年化回报率为11.1\%。巴菲特长期杠杆率约为1.6。}

马克思已经具有某种意义的神性,而凯恩斯还是个人。神性的马克思在钱的问题上真的是一塌糊涂,而人的凯恩斯在钱的问题上就有他独到的地方。我个人认为凯恩斯也是了不起的。如果说到《资本论》第三卷、《资本论》第三卷,列宁写了一个版本叫《国家与革命》,那么凯恩斯的《通论》也可以算作《资本论》第三卷的“西马”——西方马克思的一个版本,因为在《通论》里边讨论的实际上也是国家与资本的关系。当然《国家与革命》讨论的是东线国家与资本的关系,那个是国家资本主义应该如何建立?凯恩斯讨论的是在西方现有的框架下边,国家如何处理资本?

西线的马克思走啊走啊走啊走了凯恩斯之后,到现在就变成了斯蒂芬妮·凯尔顿他们提出来的MMT(Modern Monetary Theory)——新的货币理论,其实这里边蕴含了很多东西。今天我们最后的时间里边会花点时间讨论数字经济加冠状病毒会不会产生导致产生社会主义呢?在整个的历史演进过程中,马克思站在了一个无人可以企及的高度上面,这里边既有宗教的高度,又有哲学的高度,而且还有经济学的高度。那是一颗伟大的大脑。

晚年的马克思其实是经历了更多的精神上的痛苦,除了生活上的痛苦也会精神上的痛苦。因为马克思在巴黎公社失败之后,他开始在思考人类的未来的走向的时候,他得要为无产阶级梳理出一条正确的路来。因为马克思虽然他自己未必承认自己是伟大的导师,但是他毕竟是一个使徒,他把这个事情当成自己的使徒的使命,他一直在思考这条路该怎么走。实际上这条路的走法,就是无产阶级政党必须要获取政权,然后建立无产阶级专政。

那么,真的是建立政权和专政之后会是一样、一个怎样的场景呢?它会不会异化呢?如果异化的话,它会成为什么样呢?马克思的所有的忧虑、晚年的忧虑,都被其身后东线苏联、中国的社会主义革命和实践所证实。就是马克思的忧虑,一件都没有解决,也正在得到解决。一件都没有解决,就是这件事情这个历史进程并未完结;正在得到解决是因为我们今天开始重新读《资本论》了,我们尝试在这个当中寻求它应有的逻辑,它本意……

我们在寻找它本源的逻辑,它可能在今天的现实中,这个逻辑和现实结合以后会产生一个什么样的反应。马克思死了,马克思主义活着,共产主义这个信仰和理想依旧是那么的强大,我相信信仰共产主义的人会被上帝接往天堂的。所以我认同方济各——现在当今教皇对社会主义的看法。我在马克思墓前说过,马克思本人无论如何会想不到天主教的教皇会是一个虔诚的社会主义者,而且是一个伟大的社会主义者。其实人类的文明……

其实人类的文明走到它一个、一个路径必有的路径的时候,它必然是走向这样的一个道路。我必须说整个人类不管是东方还是西方,不管是通过革命、土改、改革的方式,还是西方的国家通过内生性的改革,都在开始迈向社会主义。只不过可能采取的方法不太相同,可能路径不同,甚至说法也不一定相同。但伟大的马克思提前将近了不到两百年,差不多一百七十年已经预见到后边的结果了,了不起的马克思。

读马克思晚期一些的书信是很有用的、很有意义的。当然也有些著作,类似于像《哥达纲领批判》,类似于一些重要的文献,其实对理解晚年马克思在思考的问题是有帮助的。也因为这些东西我们能理解马克思的《资本论》写不下去,因为在1873年,《资本论》的……在《资本论》第一卷发表之后,《资本论》的写作就一直处在断断续续。后来恩格斯将……

恩格斯将马克思的遗稿汇编成《资本论》的第二卷和第三卷。第二卷我觉得还算,因为我想马克思在写《资本论》的时候,一直在跟恩格斯通信,在讲解他写《资本论》的大概整体框架。我想第一卷是没有什么疑问的,讨论价值论。第二卷应该是资本流转的部分。怎么讲?这样的讲是不是有点不尽、就是其实没有说得特别清楚。第三卷的部分算是第二卷的补充,还是讲资本流转。第三卷确实没有来得及完成,就是国家与资本的关系,特别是社会主义国家与资本的关系,这件事情看来马克思还没有想得特别的完整。

那么当然了,马克思还没来得及策划第四卷,就是《新社会主义论》,就是一个接近于完成理想状况的社会主义国家是如何管控资本的,就是在人类已经跨越了资本主义的终极阶段之后,形成了比较完备的社会主义,它应该是一个什么状况。因为即便是到了社会主义,资本作为一个客观现实还是存在的,而且资本这种东西一定会异化的,它让人产生异化,那么社会主义如何克制这个异化?

晚年的马克思是充满了忧伤的,充满了无奈。他在与人交往的时候(其实可能人老了以后都会变成这个样子)开始对人产生了不信任,产生了一种失望、甚至绝望。我的年龄直奔60岁就去了,开始对世界、对人的看法也在慢慢地产生一些变化。因为年纪大了,你知道别人看一个人可能不大能看得很深,到了我这个年龄看一个人,几乎是一眼望到底。

至于说对人会失望吗?我这么跟你们说,我对很多做生意的朋友深感失望,因为我看不到他们身上应有的道德。我对一部分做官的朋友,那就更感到失望,他们缺少那种慈悲,对人民的悲悯之情。所以对做官和做生意的朋友我是感到失望的。因为有的时候越大越让人失望,官越大、钱越多越让人失望。他关注的点,因为权力越大、资本越大,他关注的点反而越来越低,不知道发生什么。所以我对十万人,拥有十亿的十万人……

我对拥有十亿身家的那十万中国人感到深刻地失望。我知道他们需要得到人民的再教育,但是我同时又感到了希望,因为除了做官和做生意的人以外,还有做学问的人,还有一些其他的在普通产业上劳作的广大的无产阶级或者中产阶级和好朋友们。他们不一定有那么深邃的学问,或者是有那么坚定的信仰,不一定的。有些人有信仰,比如说信仰佛教、信仰基督教,但那个信仰它和共产主义信仰不是一回事,我甚至不认为他们是……

我有时候不太认为他们真有信仰,我甚至在很多时候觉得一个真正的共产主义者的那种信仰才是上帝、真主、佛祖所欣赏的,那才是真正的信仰。因为一个真有信仰的人是一个超越了自我、出离了本我的那样的一个高尚的人。而有些人说有信仰,其实将信仰作为了某种心理上面调适的工具,而不是发自内心深处的那种深刻地理解和认同,那种使徒的使命感,不是那种。

我并不希望平台上的所有的朋友都变成一种使徒,有一种强烈的使命感,我并不希望这样,大家都是平凡人、普通人,好好生活就是了。我们这个课程讲的是投资,只不过是后来要讲《资本论》,讲《资本论》不得不讲马克思,介绍马克思不能不介绍马克思主义;介绍马克思主义,不能不介绍共产主义,所以多说了几句话。但多说就多说吧,我仍然希望你们或者是将来你们的孩子成为纯粹的人、高尚的人、有教养的人、幸福的人。我是真心的希望的,但我并不希望他们每一个人都成为一个伟大的共产主义者,甚至我并不希望他们都成为社会主义者,并不希望。

今天的马克思的简介我就讲这么多。其实我一直被晚年的马克思深深地困扰着,我在备这堂课的时候,我经常睡不着觉啊,有时候在查阅文献的时候,我也经常会自己把自己弄得这个眼眶湿润啊,睡不了觉,吃不了东西。有些事情其实不想触碰的,因为一旦触碰的话会带来很多很多的这个感想吧。但这个环节是不能不触碰的,因为你不触碰,你没有投入,今天这堂课就没法讲,但是讲完这堂课呢,我又怕让大家带来一些事情。那么今天就说这么多,我们讲一点儿其他。

数字科技、信息技术将人类带入了一个崭新的状况,这个崭新的状况实际上是人的某种权利的平等,这个权利就是“知”的权利。这让我想到了中世纪的黑暗,中世纪的黑暗就是因为教士。为什么在欧洲分四个等级呢?教士、贵族、资产阶级平民和这个佃农或者奴隶四个层级,最高层级为什么成为最高层级?就是因为他垄断了文化的权利或者是教育的权利,因为他拥有在印刷术未普及的时候能读懂拉丁文的圣经,其实它是一个非常需要学问的事情。

有了印刷术之后,其实文化极大的普及;其实扫盲之后,大家都有了读书认字的权利。但,你懂的,读书识字不等于可以获得信息。信息权是重大的人权。在信息时代数字经济导致了人人皆有权利获得信息,只不过每个人由于能力所限,不能过滤、梳理和整理这些信息。就是他这个整理信息的能力就是高于了获取信息……获取信息这个权利有了,但是能力还没跟上。能力有了,那么这个人就是很厉害、很牛的人了,就是能够处理信息、梳理信息。

信息权的平权,信息权的平权是一个很了不起的事情。当人们知道的时候,当广大的人都知道的时候,其实那些垄断其实就被破解了。因为在多数时候,财富的垄断、财产的垄断是通过信息的垄断来实现的。而信息的垄断一旦被打破,对财产或者财富的垄断、对权力的垄断也就被破解或者是消融了。所以信息时代,社会将越来越拥有更多的知情权,所以社会主义在数字经济或者是信息时代将成为历史的必然。

那么冠状病毒证明了什么呢?冠状病毒证明了一个人的身体和一个社会机体之间的关系。一个人的身体被冠状病毒侵犯,他需要靠整个社会机体整体的反应来对抗这样的一个疾病。社会的机体的组织效能、组织能力表达了这个社会机体的健康程度。所以出现了一种西方目前深刻的反思,就是集权:他们认为中国是一种专制和集权的体制,这个集权和专制体制在冠状病毒面前变得非常有效,就是社会整体机体的反应能力或者是抗体水平非常高。

这导致大家对社会、对社会系统的建立,就是其实对“一个制度体系的建立,什么样是比较好的?”冠状病毒让我们认识到就是中国这种社会机体的整体治理模式下形成的社会机体的反应、整体的抗体的反应,对抗病毒的水平是非常高的。而且在儒家文化圈里边,我们所形成的这种一致性,其实不完全是建立在集权基础上的,具有某种的文化的同一性和这种思想的一致性和行动的一致性上面表达出了一种社会集体、整体的一种健康,一种状况。就是我们社会集体抗体比较厉害,所以能够战胜冠状病毒。在某种意义上来讲,它证明了一种东西。

可能冠状病毒会通过病毒的感染水平来证明社会主义的意义。所以最近大家在思考就是,信息技术和冠状病毒可能成为国家转型的一种动力吧。就是好多国家开始了向社会主义转型的这样一个被动的过程。美国开始税改了,可能很多国家要开始税改。因为最近我们看到美国的数据公布,就是它结论是明确的:就是货币政策无法解决一个国家的经济结构的调整问题,或者是一个社会结构治理问题,它必须通过税政改革来实现这样的一个伟大的过程。

中国呢,最近好像有关部门也在开研讨会,在讨论中国的直接税改革问题了。虽说是,研讨会是不对的,因为直接税的受益者不参加这个研讨,让直接税的受损者去参加研讨会,你什么意思呢?你这个事情不就是不想做嘛?另外类似于像这样的伟大改革,它其实就不是个研讨的问题。因为社会它的民意不是通过专家学者来表达的,专家学者绝不表达社会的主体意识或者民意。

新一轮的冠状病毒在广谱性地爆发。台湾昨天是180例,国内情况有变,日本的情况非常严峻,显然南亚、东南亚,甚至东亚、甚至西亚都处在扩散的状态。欧美能否躲过这一轮是个未知之数。那么它对经济的影响是个什么状况呢?我们在研究《资本论》,那么下一轮的资本流转应该是怎样流转呢?总量增加了那么大,流量在急剧增加,流向能说清楚吗?流速能说清楚吗?

我们讲了四矩阵的理论,那么总量大概可以计算。那么流向为什么会流向特定的领域呢?是什么原因导致资本流向了特定的领域呢?而不流向其他的领域呢?在流向这些领域里边,它的流速会达到一个什么样的水平呢?如果真的是通货膨胀来了,它应当如何做出表达呢?作为一般性的投资者,应该如何做出选择呢?A股是否将开启牛市呢?如果A股开启牛市,那么哪些的类别将在未来的时间内会有长足的发展呢?

课讲到这儿其实都说清楚了,或者有朋友问,就是“你什么都没说”,我其实都已经说得差不多了。再重复一句话:坚持既定方针不变。等待,等待剧烈变动的到来,到那个时候我们再改变既定方针。因为既定方针就是为了应对今天这个局面的,所以还是坚持短股长金。坚守既有的逻辑,不要轻易地做出改变,但要对未来有前瞻性、做出安排。好吧,今天就说这么多。周末愉快。

\section{东马西马和新马、当下的经济}

大家下午好。今天是2021年的5月29号,辛丑年的四月十八,时间过得真快,这转眼五个月过去了。辛丑年的六爻已经走到第三爻了,整个形势的变化还是很有趣的。我们今天是正式课,是讲《资本论》的第四讲:东马、西马和新马。今天的课有意思,也重要,因为这是对马克思主义两百年的一个总结。我试一下麦,三点准时开始。

大家好,今天是2021年的5月29号,辛丑四月十八。刚才说的不对,还是在第二爻,还差十二天进入第三爻。我之所以对这个爻这么在意,实际上是在印证中国的古典哲学对事物判断的这种准确性。将来有空我们可以多讲一些这个事情,其实对时间和空间的理解是哲学的一个基本的框架,但我们今天不讲这个。我们今天是《资本论》第四讲,东马、西马与新马。

在正式课程结束之后,如有时间,我们讲几句经济,谈一下,因为亦如我们的估计,这个耶伦发出第三支箭以后,弱美元以极快的速度在进行比对的转弱。我们注意到人民币的迅速升值,昨天晚上升值的状况超出了我的想象,这个速度极快。那么它会走向哪里?它会对整个市场有一个什么样的影响呢?我们待会儿简单做一个分析,因为好多朋友很紧张,是不是A股的牛市来了?那么我们该做什么?应该怎么做?大家处在一种紧张的状况,不用紧张,我们还是……

我们今天还是先进行正式课,东马、西马和新马,其实这堂课花的时间比较多,备课。因为早就应该对两百年,马克思出生在1818年的5月5号,就是这个月,正好是过去203年。在马克思诞辰两百年的时候,我曾经在互联网上做过一堂介绍马克思的课程。我找不着那个录音了,文字我倒有,找不着那个原来的录音了,那算是一次总结。

今天我们做的稍微细一点,因为我们讲《资本论》的前四讲是做一个宏观的铺垫。第一讲我们讲了两次革命和两个宰相;第二讲,我们讲了四个哲学家;第三讲,讲了马克思的生平和心路历程;今天是第四讲,我们概述一下两百年马克思主义的发展,就是东马、西马和新马。其实这个事情不应该由我来做的,但我知道社科院和中央党校的这个学者们可能他们有所忌惮,所以在处理如此复杂的恢弘的历史题材,可能会遇到巨大的障碍。

那么我今天这堂课分四个部分,第一个部分我想讲一下子分水岭。马克思形成主义,并且开始落实于行动,开始进入到组织和运动状况,应该是1864年。这个时间节点比较重要,这个时候建立了第一国际,当时叫国际工人联合会。这件事情马克思和恩格斯是积极参与的,其中马克思提供了理论支撑和这个重要的指导。它的重要的活动,应该算是1871年的巴黎公社,这是一次……

第一国际在1876年解散,虽然第一国际解散了,但是在欧美一八八十年代,十九世纪的八十年代,已经有十六个国家建立了社会主义政党,实际上马克思主义那个时候已经开始发芽生根,结出了初步的果实。第二个重大事件是1889年的7月14号,国际社会主义者代表大会召开,它是为了纪念巴士底狱一百周年,它的名称叫社会党国际,我们通常把它称为第二国际。这是在巴黎开的一个会……

这是在巴黎开的一个会,有22个国家393个代表参加,这算是马克思主义在组织和这个运动方面的进入到的第二个高潮。其中一些很重要的人物、历史性的人物出现了,李卜克内西、培培尔、瓦扬、拉法格等都出现了。这个时候它就不是简简单单的一般性的组织和群众运动,他们推出了两个重要的东西,一个是《劳工法案》,一个是《五一节案》。第二国际在1914年的8月14号,列宁宣布第二国际死亡,第三国际万岁。

第三国际是在1919年莫斯科建立。第三国际也叫共产国际,这个大家就比较熟悉了,他的负责同志是季米特洛夫。第三国际在一九四三年五月十五号解散,后来托洛斯基——流亡者托洛斯基创立第四国际,但第四国际不被大家承认,所以没有第四国际,就是三个国际,它代表的马克思主义的走的三个阶段。其中,第三国际代表着东马正式的产生。一会儿第二个部分我们讲东马,第三个部分讲西马,第四个部分讲新马。

在这里我想多说两句,就是其实这个事隔了,从1943年到现在,这个过去了又快八十年了,马克思主义其实是在反反复复中不断地进化和演进,其中东马的部分在中国得到了一个非常曲折的发展,我个人认为第四国际应该由中国人来建立,它建立的时间应该是二十一世纪二十年代的中叶,大约在第三国际解散八十年之后,应该有第四个国际出现。

是的,我们曾经向全世界宣布我们不输出共产主义,我们不干涉他国内政。我们一直是这样想的,也是这样做的。在形势的严峻程度往往超出我们的想象和预料。我们现在面对的形势恰好是西方逐渐的联合对中国形成一种围剿之势,这个围剿之势于当年对第三国际的围剿有类似之处。如果我们仅仅是用西方的逻辑和方法来与之博弈,恐怕不是个究竟,我们还是应该站在全球无产阶级的立场上来思考这样一个问题。

当然这是后话,一会儿谈新马,我们会谈国际共产主义运动或者是马克思主义在新时代的发展。因为他马克思主义一向是穿越种族、阶级和国界的。正确的运用马克思主义,既可以解释和解决历史遗留问题,也可以迎接新时代的到来。马克思主义的生命力在于当下,它也解决严峻和复杂的社会问题,包括一些重大的国际问题。它可能使中国站在一个崭新的角度,开辟一个崭新的时代。所以我对第四国际有期待。

好,我们进入到今天的第二个部分——东马的进程。列宁在1918(1917年建立苏维埃),列宁在1918年的时候,他们已经开始在策划全球的各个国家的社会主义革命问题了。所以在1919年这个已经相对稳定下来的苏联开始考虑建立第三国际——就是共产主义国际,以帮助其他被压迫的国家和民族的无产阶级的革命运动。这是东马的起点,我这里边要从理论上说几句。

东马的出现它和第二国际的走向有关系,为什么列宁在1914年,列宁同志认为德国的共产党人这个走偏了,所以他宣布这个第二国际已死,第三国际万岁。他就是认为第二国际深受民族国家的影响,好多无产阶级的政党就是支持本国政府进行这种战争,第一次世界大战。所以列宁在这种状况下表示了一种激愤,第二国际已死,第三国际万岁。

列宁是一个在马克思主义发展史上非常重要的一个思想家,他写的《国家与革命》是我认为真正意义上的《资本论》第三卷,《资本论》第一卷谈价值;第二卷谈资本流转;第三卷谈的就是国家与资本的关系,列宁的《国家与革命》。这本书虽然没有说的那么透彻,没有把整个的框架和体系建立的像马克思、恩格斯那么完整,但他基本上为苏维埃政权的建立提供了一个理论支撑,或者是理论指导,或者是一个基本的蓝图。这本书是东马的指导的一个基本的文献或者是列宁主义就是东马的基本依据。

东马的发展呢?由于列宁在处理国家与资本的关系上面,可能列宁对马克思,特别是马克思晚年的忧虑理解的不够透彻。其实马克思在他的晚年,他已经清楚地意识到了无产阶级建立了政党,由无产阶级政党通过武装斗争夺取政权,建立无产阶级专政,最后的结局是什么。马克思,就是像马克思这样的哲学家,他思考的深度和阔度是常人所不及的,他已经意识到那将是一个国家资本主义的形态,而且他已经意识到即便是无产阶级建立的国家资本主义也必然走向它的反面。

马克思不写《资本论》第三卷不是因为懈怠或者是懒惰,是马克思可能两个原因,一个是他认为可能太超前了,第二个他可能觉得在他那个时代的无产阶级还不能够来实现这样的一个过于前卫和伟大的目标。但是革命的形势的发展非常之快,整个的马克思主义在二十世纪初叶已经成为了一种广谱性的社会思潮。一会儿我们讲西马的时候会讲凯恩斯,我昨天把我去凯恩斯的故居的那篇文章贴出来,就是让大家先熟悉一下。

实际上已经成为了工人阶级或者无产阶级一种迫切的愿望和需求。所以这个时候列宁适应时代的发展写下了《国家与革命》。列宁本人并不知道他建立的不是社会主义,他建立的是国家资本主义,列宁本人也许他在思考的时候他想到了这一层,马克思是非常清楚的知道这一层的,列宁也许想到了这一层,但是来不及为他称之为社会主义的国家资本主义梳理一个百年的愿景或者是未来,所以他可能无法想象他亲手建立的苏维埃政权。

列宁可能无法想象他亲手建立的苏维埃政权苏联在七十年后变质并且轰然倒塌,轰然倒下,不仅这个政党倒下了,而且这个国家也倒下了。所以后来等的苏联解体之后,苏联共产党人包括久加诺夫他们说,苏联共产党再也不可能重建苏维埃政权或者重建苏联了。为什么?为什么不能重建?因为国家资本主义模式它本身不是社会主义,或者是真正的社会主义理想,它是一个过渡型的模式。对过渡型的模式如何完成历史转型这个事情列宁没有做完,斯大林也没有做完。

斯大林的贡献是很大的。斯大林在《国家与革命》的基础上,结合苏联的现实,在实践中创立了一整套的苏联模式,政治模式、经济模式和文化模式。他表达为苏联《政治经济学教科书》,我说过毛泽东带着一群人在杭州读苏联《政治经济学教科书》,刘少奇带着一群人在海南读《政治经济学教科书》,小平、陈云他们在北京读《政治经济学教科书》,其实大家对苏联模式是有深刻思考的,就是中国共产党人对苏联模式是有深刻思考的。尤其是毛泽东,他对苏联模式是有批判的。

当然,刘少奇和邓小平的思考可能更柔和一些,他们可能有一种更柔和的解决方案,毛泽东的方案是比较激进的。所以,其实读苏联《政治经济学教科书》是中国文革的肇因,就是大家都意识到国家资本主义有问题,怎么解决国家资本主义的问题?这个时候在这个问题上产生了历史性的分歧。东马的这个路径主要是苏联和中国这样的一个发展的路径。东马的第一个高潮,就是苏维埃政权的建立;东马的第二个高潮是文化大革命;第三个高潮是中国的改革开放。

东马的实践,到今天也未结束。虽然苏联解体了,但是中国还在;还有一些可能叫社会主义国家的,例如,北朝鲜、越南也都还在。这种社会主义实践,很多人认为已经变质,但我不完全认为,因为国家资本主义的基本框架还在。虽然引入了社会资本主义,但是国家资本主义的框架还在,执政党的理念仍然是以马克思主义为根本指导原则的,就是还是坚持马克思列宁主义、毛泽东思想的。而这个理念可能很多朋友,

这个理念可能很多朋友并非深切地理解它的本质,其实它意味着执政党执政的合法性,这是合法性源泉。所以“为人民服务”那几个字,不是简单说一说的,因为它是一个基本的伦理的原则。一个执政党的执政,它需要伦理基础,需要法理基础,它整个的伦理基础和法理基础,整个的逻辑是由此而构成的。如果不出现像苏联转成俄国这种颠覆性的东西,它这个伦理基础是不能改变的,而它这个伦理基础决定了它的基本的路径和走向,这一点是非常非常重要的。谈到东马就谈一个人,

这个人是福山。福山在1992年他写了本书《 历史的终结与最后的人》,它里边有个结论,就是:自由民主制将成为国家政府的唯一方式,也是最后的形式,所谓的历史终结理论。他说的历史终结,实际上他认为东马终结了,或者是他认为马克思主义终结了,或者是他认为社会主义终结了。福山作为一个思想家,不入流;作为一个哲学家,完全不具备哲学的高度,没有辩证性思维。2014年福山又写了《政治秩序及其衰落》。

2014年福山又写了《政治秩序及其衰落》,这个时候他开始反思了。因为中国崛起了,他开始反思,他说的历史终结到底是谁终结了?所以,刚开始的时候,他们认为是中国是转型完成了,就是中国不再是马克思主义或者是列宁主义或者是社会主义国家,他们不认为了,他们认为中国已经转型成资本主义国家了,所以他提出了一系列的模型和想法。我今天看这个模型,我是不能认同的,这个福山的基本的判断的。其实他反而是对资本主义没看明白,是西方的资本主义正在走向民粹式的民……

而中国恰恰是在逐步的走出民粹式的民族主义。虽然我们强调爱国主义,但是,我们在走出民粹式的民族主义,我们正在走一种全球共享的国际主义,或者是,我们在某种意义上正向的,我们在走向一个新的、共享式的国际主义。在内部,当然我们出现了非常严重的官僚垄断资本主义。一会儿我会讲,为什么马克思预料到国家资本主义不行,就是国家资本主义它的必然的结局就是官僚垄断资本主义,它是个必然结局;而西方的资本主义它的必然结局就是金融垄断资本主义。不管是他这个资本主义,还是我们资本主义,

福山两个事情都没搞清楚。第一个事情是他真的没有读《资本论》。当然,不光福山没读《资本论》,我国的学者专家读《资本论》的人也少。张五常说他读不懂《资本论》,这个我同意。因为受西方经济学训练的人很难读懂《资本论》。不是西方经济学不能成为《资本论》的入门工具,是潜在的意识形态影响了他对一种理论的接纳、容纳。所以张五常认为,他不知道马克思要说什么,马克思为什么要说这件事情?他理解不了。因为他没有这种伟大的拉比对人民的那种悲悯,他没有。他都从经济学算计的角度来思考。

当代的中国人读懂《资本论》的人算凤毛麟角吧。我能收集到的当代中国人关于《资本论》的这种著作,我基本上收集齐了。不要说深刻了,极少有接触到马克思主义本质的人。我说得极端一点,我现在还没见到,反而是欧洲的学者很厉害。欧洲的学者,特别是当代的欧洲的学者,他们非常非常地逼近马克思的本意,这让我感到惊讶。就像中国人很少有人能读懂王阳明,就是中国是一个社会主义国家的,很少有人能读懂《资本论》或者读懂马克思主义,他们经常会把它变成……

就像西方人攻击中国似的,他攻击那个东西,根本不是我们的本质。福山在描绘中国的时候,他说的那个,那是中国吗?那不是中国!所以福山两件事情没搞清楚:第一,他没读懂马克思主义;第二他没有读懂中国,这两件事情导致他整个的判断是误判。但他的误判,福山这种不能算非常优秀的学者,为什么这么有名呢?就是这个历史终结的理论成为了西方的政治上的一种工具,或者是成为美国,主要是美国颠覆其他具有社会主义国家的一种理论工具。它对中国还是产生了巨大影响的。我为什么今天要提一下福山呢,就是中国人好多人接受历史终结。

历史会终结吗?按照黑格尔的逻辑:存在即合理。当俄国爆发了革命,中国爆发了革命,他们建立了国家资本主义,这是一个偶然现象吗?当然不是,这是个历史的必然。这个历史的必然趋势能用简单的是非对错来衡量吗?当然不能。这个现象的存在它不仅仅是我们在文学作品里边描绘的那些惨烈和血腥,它也进行了伟大的建设。难道苏联取得的成就不伟大吗?难道中国共产党人建立的中华人民共和国70年里翻天覆地的伟大成就不够伟大吗?

这样的伟大,福山能终结吗?当然终结不了,尤其是中国,就是“东马”的实践。这个历史有的时候很有意思。就是我们在学习读书的时候,我们都认为只有伟大的导师有伟大的理论指导下,才会有伟大的实践。但你纵观人类五千年文明史,伟大的实践多数不是由伟大的理论来指导的;它是人类、人类智者对现实的一种深刻的体悟来作出的抉择,这种深刻的体悟,有的时候并非由理论家完成。

同样在六十年代在读《政治经济学教科书》,读的结果,最后由第一次实验——文革,不成,失败之后,很快展开第二次实验,就是1978年、1979年我们开始的全面的改革。这个改革的,“东马”的改革出现了一个什么情况呢?就是我们已经敏锐地比苏联提前二十年意识到国家资本主义不行,国家资本主义有问题。就是那个时候国家资本主义已经出现了官僚垄断资本主义的特征。毛泽东一早就看出来国家资本主义一定会走向官僚垄断资本主义。他为什么说:“走资派就在党内,走资派还在走”;他说:“他们不是胎生,不是卵生,是化生。”

毛泽东的历史洞见是非常深刻的,而他这个深刻的洞见,其实我们今天来看,党内并非不认同他的洞见,而是不认同他解决问题的方法。就是通过群众造反,通过人民来夺取立法权、司法权和行政权的方式是不对的。人民只应该获取立法权;不应该获取司法权和行政权。砸烂公检法和这个各级红卫兵建立革委会,这个做法是不对的。显然通过文革的这个实验,证明了社会资本主义不能通过群众运动的方式来做。这样搞,这个社会资本主义搞不了,结论也是非常明确。

在这个问题上,我相信小平同志、陈云同志他们可能在文革结束之后会有共识。就是中国不能再搞国家资本主义了,不能让国家堕落为官僚垄断资本主义。同时他们也不认为西方的资本主义是一个……不是中国想不想选择,中国没法选择。所以呢,开始引入社会资本主义,那个时候我想他们可能并不熟悉艾哈德,并不熟悉德国历史学派,甚至不能从德国历史学派和艾哈德身上来借鉴社会市场经济。只不过这一代人的悟性,这一代中国人的悟性极高,而且他们具有超强的行政执行能力。

很多人不大理解,就是经历了漫长的战争,从1921年建党到1949年夺取政权,经历了漫长的战争过程,我党的一些干部,其中比如说邓小平,他们积淀了或者是磨炼了强大的社会治理能力。你要知道在一个解放区,特别是小平同志负责的晋察冀、晋冀鲁豫、晋绥,就是他负责的区域差不多是三分之一的中国,他是政委,他要管的是当地的财政、金融、民政,是个全面的治理。在全面的治理的时候,他对社会、对政府有着比常人更为深刻的认识。

建国之后,很快陈云、邓小平这些人进入到中枢,陈云是常务副总理,邓小平是总书记,他们开始统御全局。在这个过程中,虽然他们不是扎飞人(注:粤语指负责人),不是最终的负责人,但是他们对整个事情的观察和理解是常人所不及的。他们在战争中锤炼出极高的能力和悟性,就是他们有驾驭整个、驾驭一个政党、驾驭一个政府的能力,他们是足够的。在文革之后,他们表达出他们强大的管治能力,这个福山是承认的。就是中国共产党的领袖有超级的治理能力,这不是一般人所能理解和体会的,他们有极强的驾驭和治理能力。

“东马”发展到改革开放的时候,其实我们并没有成熟的理论。当时一些学者,包括他们在研究东欧问题,想从东欧获取经验,研究南斯拉夫,研究东欧。因为那个时候我开始读书了,那个时代还是很热闹的一个时代,大家在探索这个转型之路。那个时候还到不了一种理论的自觉,就是由国家资本主义引入社会资本主义形成一种新型的混合的资本主义。现在我们就是国家资本主义加社会资本主义形成的一种混合的状态。这种状态达到了两个效果:第一个效果,

它通过强有力的治理,最大限度地调动全部社会资源形成最佳配置,以迅速地实现经济增长。不但在完成,毛泽东完成工业化的基础上,迅速地实现了经济的增长,这是第一个部分。其实不管你是什么主义,主要是要改善人民的生活,在这个意义上面确实是厉害。它既有发挥了国家资本主义优势,也发挥了社会资本主义的优势,这两个主义有机地结合产生了奇异的效果,创造了令人震惊的奇迹。但我要说的是,毕竟没有进行系统的理论的总结,

所以,我们现在依旧面临尚未解决的,就是当年毛泽东思考过但尚未解决的国家垄断资本主义的问题。实际上我们看到的东北病、华北病都是国家垄断资本主义作祟;我们看到中国的新生的十万高净值成为国家的毒瘤,也是这个问题。就是国家资本主义必然产生官僚垄断资本主义,官僚垄断资本主义产生出了一系列的问题,这个问题现在还没有得到有效地解决。我们在看到它的好处、利益的时候,要深刻地洞察到它存在的问题。“东马”还需要进一步地向前去,要走一条新社会主义道路,这条路还需要深刻的思考。

好在这一届的领导群体确实是不错。因为在“东马”的第三次高潮改革开放之后,遭遇过巨大的问题和挫折,这个巨大的问题和挫折其实表达为本世纪初那十年。那十年,我们出现了一系列的问题,出现了一系列的问题。但其实中国是一个非常有趣的国家。因为中国的革命的结束是1949年,七十年;中国文革的结束,如果是以1979年算的话,三十年前,文革这一代人都还在,而且上山下乡这一代人经历过文革洗礼,这一代人很了不起。

虽说是在国家资本主义和社会资本主义融合这个过程中,有诸多诸多的社会扭曲、变态,但是有那么一些人,理想主义的旗帜仍然在心中高高飘扬。我不知道其他人的体会,我们这一代人,我是六零后,我们听到红歌心里边会有一种震荡的、会燃起一些东西的,就是我们听到一些红歌,我们内心深处依旧存在着一种激情,那种激情无法用语言来表述。就是当如果真的提出来为了国家或者为了人民去做一些事情,甚至做出牺牲的时候,我们这一代人不会有犹豫。他是这样的一代人。

另外最为可贵的是,毛泽东走了,但是毛泽东的四卷书还在。那四卷书虽然不像马克思的书那么有哲学高度,也不像列宁那么敏锐地去规划未来的国家资本主义,但是毛泽东的四卷书里边拥有深刻的哲学的积淀。那个哲学积淀不是源于西方哲学,是源于中国心学。它在管理上面的意义大到你难以想象。几乎所有中国的六零后、七零后的企业家都熟读毛选,他们的能力比西方管理学教育出来的人要强大得多。中国的治理能力就来自于那四本书啊。

东马最后的发展可能要仰仗中国这一代人了。新社会主义理论的提出实际上是马克思主义新马的逻辑基础或者是思想理论体系的基础,它将在不久由中国人来完善、建立,并且它可能成为一种新的世界潮流。甚至我今天开篇讲过,我们期待着第四国际的建立。这个世界总是有新的东西产生的,新人换旧人它是正常的,没有什么修昔底德陷阱,没有,他们搞错了。他们面对的是一个用强大思想武装起来的一个……

好,讲一点西马。讲西马我就不能不说一下子凯恩斯了。这件事情我的读者们都熟悉。因为这是其实是我记不得是2015年还是应该是2015年的冬天,去到英国伦敦,然后我就想去找一下子凯恩斯在伦敦的故居,他住在伦敦的布卢姆茨伯里区。这是我的习惯。就是我读一个人的著作,我肯定会把他的传记、他的生平搞清楚。其中一些细节我是会亲自去考察的。对凯恩斯尤其如此。

有一个细节,就是我读《凯恩斯传》的时候,有个细节它让我感到惊讶。就是凯恩斯在上个世纪初,大概是1905年左右剑桥毕业。毕业以后,他们一大帮同学,当时是英国的牛人们,这都是富家子弟,他们竟然去了布卢姆茨伯里区,这是个贫民窟、贫民区。所以我怀疑他是“青年毛泽东带着雨伞去安源”了,这是搞工人运动去了,甚至有可能在那儿组建政党和准备革命的。我有这种直觉,因为他的老师是马歇尔。凯恩斯的著作跟马歇尔相去甚远,这离经叛道的主。

所以我去了伦敦,我就很想去看看,后来大家都觉得那地儿怕不安全,后来我说穿上牛仔裤,什么也不要带,去看看,就找到了那个故居。还好那个故居上面还有个铜牌,上面标志着凯恩斯哪一年哪一年在这里边这个居住过,去英国才发现不光是这个在那儿工作过,那还有一个以地区命名的学派——布卢姆茨伯里学派,而且这个学派的大部分的人最后都封爵了,极有成就。我相信在上个世纪的初叶,一百多年前那个是马克思主义最为……

就是马克思主义作为一种思潮达到了一个历史性的高度,这个高点就是它非常热、非常时髦。就是你要如果在牛津和剑桥读书,你不读马克思主义,或者是你不了解马克思主义,或者是你不悄悄地参加这个相应的组织,那你就out了。相信那个时候青年的凯恩斯可能他也有一些想法,所以他们这一大票人就去了这个地方。至于他们后来是否搞了工人运动,在那建立了什么样的组织,没有办法考证。因为我的英文的情况不是很理想,另外我也无法在伦敦长期生活。但是我大约能够理解当时英国的社会状况和英国青年,特别是知识青年的一些想法。所以我写……

西马,真正的起点还不是凯恩斯,他太年轻了。真正的起点应该是威尔逊——普林斯顿大学的校长,1913年竞选当了总统,是一个平民的总统,也是一个学者的总统。但我自己坚信,我去美国的时候想对威尔逊做多一点的了解,但是来不及。但我坚信威尔逊是一个社会主义者,也应该是一个马克思主义的信徒。虽然他发明了高尔夫球,他这个建立了联合国,就是很多东西都是他建立的,但是他的劳工保护法案、妇女保护法案、联邦储备法案……

威尔逊手上初步建立了西马的制度基础。就是西方的政治在凯恩斯们这种年轻人的推动下,在威尔逊们的这个直接的干预下,他走向了一条与原来的资本主义不同的方向。实际上英美在他们的时代,在他们的手上,进行了没有硝烟、没有革命的社会主义改造。对于西马的路线,学者一般不这样的想。学者会将1923年德国十一月革命失败之后,卢卡奇、柯尔施、葛兰西他们这一票人认为是西马的代表,而我不这样想。

如果你让我来理解西马的代表人物,我会说凯恩斯,会说威尔逊,会说艾哈德、德国历史学派——他们才是真正的西方的马克思主义者。至于卢卡奇、柯尔施、葛兰西,他们可能提供了一些理论,甚至建立了一些组织,但是他们对西方的政治制度的社会主义改造并无太大帮助。如果你研究哲学,那么你就一定能够懂得对立统一的含义。就是事物发展的同一性好多人并不明白,什么意思呢?就是你的对立面倒了,你也失去了存在依据。我们在未来的生活中一定要……

深刻地理解哲学上的对立统一,就是既要看到事物的对立矛盾,也要看到它的同一性。举例,当撒切尔夫人、里根他们毫不犹豫地瓦解了苏联的时候,他们是否理解福山所说的那个历史的终结?不仅仅是终结了社会主义,也同时终结了资本主义。我再说一遍,对立统一的其中一个重要的元素是同一性,就是对立面倒了,你也失去了存在的依据。恰恰是苏联解体之后资本主义……

西方的资本主义迅速地滑向金融垄断资本主义,因为他们失去了外部的约束,内部的约束机制解体。这个时候我们注意到整个的西方世界开始走向它的历史的反面,我们注意到五年前的法国的黄背心运动,我们也注意到2020年的“黑命贵”。这里边有种族问题,有阶级问题。为什么?因为恰恰是苏联解体之后,西方的中产阶级开始集体返贫,重新沦为无产阶级。这不是很可笑、很荒谬吗?

这种可笑和荒谬它就是历史的辩证,它就是对立统一,它就是同一性深刻的表达。有的时候对自己的敌人惺惺相惜,它是有原因的,因为那是你存在的依据。特朗普不懂这个道理,拜登也未必懂这个道理。当他们与中国死磕的时候,或者是动员组织西方世界与中国死磕的时候,他们其实并不知道他们在做什么。当年解体了苏联之后,他们两个事情他们想不到:第一件事情就是苏联解体,西方世界必然解体。

整个西方世界是因为苏联的存在而存在。我一直对里根的评价非常低,演员就是演员,他完全无法理解当年马歇尔、罗斯福、斯大林的神思妙想。我一直说美国真正的盟友只有一个,就是苏联。整个西方世界不是美国的盟友,而是美国的小弟;整个东方世界不是苏联的盟友,而是苏联的小弟。当然有一个伟人他叫毛泽东,他不做小弟,所以出了问题。西方世界亲手干掉他的真正的盟友的时候,就是美国人干掉他的真正的盟友的时候,队伍就散了。队伍散了之后必然会有新的家伙冒生出来,那么就冒生出两个大家伙来。

当东西德完成统一之后,强大的西马,我刚才忘了说西马的高潮,西马的高潮就是艾哈德的——在二次大战之后由德国历史学派艾哈德建立的社会市场经济。它的生命力极其强大,所以当东西德统一之后,东西德经济飞速发展,并且德国人用马克代替坦克统一了欧洲,建立了欧元区,我管它叫“德意志第四帝国”。另外一个就是放出来一个中国,中国在完成国家资本主义与社会资本主义融合之后,形成了快速的发展。这个快速的速度实在太惊人,因为……

东马的头马——中国,西马的头马——德国获得了长足发展。福山怎么懂马克思主义呢?福山怎么叫历史终结呢?福山完全不懂哲学,完全不懂社会学。或者是像福山这样的学者坑害了整个的西方;或者是西方的学者不好好用心读《资本论》、读马克思、读毛泽东,不好好学习德国历史学派,不读艾哈德的著作,也不读凯恩斯的著作,他们完全没有正确地理解历史发展的进程,所以导致一系列的错误。所以我们看到英美整个的体系处于迅速地崩塌与瓦解的历史进程中,能不让人唏嘘吗?

在这里多说两句,当年德国人驱逐马克思,后来比利时、法国也驱逐马克思,是英国人收留了马克思。英国人不但收留了马克思,还给马克思建了墓。你知道英国的思想家,在英国的那个时代的思想家,包括他们的王,他们在哲学上是有高度的。他不但容纳了马克思,他们还同化了凯恩斯,将“青年毛泽东”引入政府,而不是让他走向革命的道路,你就知道那种包容百川的那种雄心壮志。

英国和美国走了一条西马的路线,由于他们对主义对新人类的这种包容、吸纳和整合,形成了英美一百多年的繁荣。看历史就是这样演进的,他们在苏联解体之后开始走向他们历史的反面。当英美瓦解了苏联的时候,其实他们根本没有注意他们迅速滑落的历史进程。所以英国“脱欧”我一点儿也不觉得奇怪,美国的“黑命贵”我一点也不觉得奇怪,他们在走向历史的反面。最残酷的是他们现在对这个历史进程缺乏足够的认识,他们现在没有站在哲学高度来进行反思。

相反,中国人在反思。我不知道俄罗斯的当代思想家在怎样想,我阅读了一些东西但我感到失望。我知道德国人在思考,美国人类似于桑德斯、凯尔顿这样的人在思考,但真正能够再重新上升到马克思这种高度进行哲学思考的思想家凤毛麟角。我们在想就是新马的问题,我们走东马和西马的伟大实践应该走到一起,开始进行融合了。

2021年,我决定自囚,自己囚禁自己一年,囚禁完了以后呢,我们开始进入到工作状态。除了给大家讲课和聊天以外,其实我想了,想在2022年,我们在香港建立一个新马研究会——新马克思主义研究会或者是新社会主义论,当然还是个新马研究会。把这个新社会主义论这个问题,组织大家把它弄利索了。理论的学习和理论的探讨,有的时候是非常重要的。这件事情欢迎大家广泛参加,因为在香港这个地方相对的宽松一些、活跃一些,我也希望在北京也有同样的……

在读马克思的这个著作的时候,我对他晚年的忧虑,随着年龄的增长越来越理解;对毛泽东晚年的忧虑,我也越来越理解。我能理解他们晚年的忧伤,有的时候甚至性格上的那种变态,那种忧伤。因为他们超越了他们的时代,他们无法说服他们的战友,他们甚至不愿意落笔,不愿意写下来,因为写下来就是误会,写下来就是误解,这种状况的确很不好,但是这是他们的宿命。但是值得宽慰的是,马克思晚年的忧伤,毛泽东晚年的忧伤,我们现在有机会来解决了。

新马要研究一条新的社会主义出路,新的社会主义出路的成熟,不是因为理论上的成熟,而是由于生产力发展到了一个特定的阶段,生产力决定生产关系,信息时代到来了。信息时代到来,什么东西最具有社会性?信息。什么是资本的载体?Data。《资本论》是不是可以重新写了?因为资本的载体和马克思看到的和毛泽东看到的完全不一样了,资本的载体发生了变化,另外信息的社会化导致了知的权利的平等,它为社会主义创造了必要的条件。

在信息化时代,立法权的社会主体性可以建立了。我重申,人民立法权是马克思最关注的问题,当人民拥有立法权,可以监督司法和行政的时候,这种社会治理的结构出现,它才是真正意义上的社会主义。新马要从哲学上解释这一现象,要从社会学角度来指导这一现象,要从政治经济学角度来规划或者是规范这样的一个历史进程。其实这是一个非常了不起的工作,非常非常了不起、非常重要的工作,比我们生命还重要的工作。

讲到新马,其实是挺激动的,我在伦敦的那个寒冬,我最后的文章,写到最后的一句话说:“在英伦凛冽的寒风中,我的心一直是滚烫着的。” 我说的是真实的感受,因为其实在精神上,思想上和精神上,其实社会主义者是通的,是神交的,是通的,我读艾哈德的著作也有同感,其实它不是一个简单的主义的问题,它是一种慈悲,一种对人民、对国家的一种悲悯。

新马就不讲那么多了,因为今年自囚,下半年我会把这个东西——《新社会主义论》整理出来。整理出来呢,就算是抛砖引玉,留给大家。然后我想明年,我们有个新马会,新马会可能我们组织起来做一些更为具体的研究工作,因为有的时候还是需要进行哲学思考的,这个哲学思考,当然啦,这也是历史的必然,可能应该在北京完成,但也没办法,我们先在香港开始吧,有些事情总是要有人来做的。细节就不讲了,这个拉拉杂杂一讲就一个多小时过去了,聊几句当下,其实今天跟大家,

今天跟大家神游,这个这时间过得太快,这个好多东西也来不及,有些东西反正作为资料你们可以看,除了我这个在这篇文章,就是《对英美左翼运动的历史性思考》,大家还可以再去读一下子我写的《掠过佛莱堡》,其实那篇文章是写德国历史学派的,那里边也谈了一些我对西马的思考,对东马和西马的思考。我原来写过《新社会主义论(2014修订版)》,但是那个版本我不满意,我要重新写过它。重新写过它以后,可能这个时间过得非常快,因为写这个《新社会主义论》里边最难的部分是信息时代、信息社会,它对整个的这个一个思考。

在很多时候我是觉得我自己的这个能力有限,身体也不行,精力也是不行,所以我真希望年轻的冲上来,我们一起把好多事情做好它。好,说几句当下,上个星期我们聊天的时候说了,就是我们放开汇率,不进行汇率的大规模干预,我们管控资产价格,主要管控要素价格。一如我们对美国的判断,就是耶伦的三支箭全部发了,其中最后一支箭弱美元,它主要是表达为对人民币的弱势,而且这个速度极快,跟我们的预判差不多。

人民币急剧升值带来的问题也是明确的,就是除了我们商品这个就是贸易的结余、盈余之外,很快就会出现大量的资本项下的盈余,因为央行也不希望人民币升值太快,所以它会入市干预,不停地在抛人民币买美元。那么留手上大量的美元,如果让高净值们离境,就是让这个十万高净值有十亿,他们持有的一百万亿,这个人换外汇离境,这可能是最差选择,这就变成了日本的那种平成战败模式了,这是我们坚决反对的,那么就应该将这部分的结余尽可能的变成,

有价值的商品和资产。有价值的商品里边最核心的部分就是黄金和碳排放权,所以我们希望下决心进行所有贵重物资的战略储备,其中核心就是黄金。我们要货不要钱,我们坚持在人民币升值的期间要货不要钱;另外不放人走,不放高净值走;同时呢我们要小心谨慎进行海外资本的投资。我觉得这回美国联手西方围堵中国,对我们一带一路的这个事情进行围堵,从某种意义上,我觉得他们是在帮我们。我不主张进行以美元为主体的海外投资。

我的朋友都知道我的政策建议,我是主张用人民币进行海外投资的,就是允许七个斯坦,允许非洲,允许拉丁美洲到北京发人民币债券,甚至允许他们以人民币为基础发他们的本国货币。比如北朝鲜可以通过人民币来发朝鲜元,这个我不是今天提的,十五年前我就这个建议,而反对我们通过贸易盈余和资本盈余赚来的美元再投出去,我们必须截断美元MMT的闭环,这个不是为了对美国进行打击,而是为了避免“平成战败”。当然建议是归建议,能不能做到,能做到什么程度,我们是不知道的。

那么我们对经济形势做一个基本的判断。现在目前这个情形来看,在资本流动,大家反正听我课的朋友都知道,我平时的分析的框架就是资本的两个流转的层级。一个是区域流转,就是全球资本可能在一个特定的时间会涌入东方。其实日本、韩国、台湾都是大规模资本涌入,他们的货币也非常之强,其中包括中国。主要是东亚区域将成为全球资本涌入的这样的一个核心区域,其中可能中国是主要的一个地方,它会导致中国的资产价格有一轮上涨,甚至可能会构成…

甚至可能在一个时间区间内,会构成一个较长期的资产的牛市。包括楼市和股市,会构成一个较长时期的资产的一个牛市。楼市再牛下去就不合适了,因为我们的泡沫已经达到一个非常高的水平了。另外呢,楼市作为一个再分配工具,这样下去不是个究竟,因为没有直接税嘛,你这个搞法肯定是“平成战败”的这个局。那么就是股市,中国的股市现在非常神奇,就是排第一的是茅台,排第十的是五粮液,中间夹着八个金融。这种结构可能也不是西方的投资者所喜欢那个结构,因为这个东西跟政策的关系关联度太高了。但是中国的经济结构的转型呢,可能要五年之后,2025年以后才有机会转。

所以我们国家处在一个非常特殊的时候,大体上可以确定股市的底部就是现在了。至于说牛市的起点是不是就是现在呢?我请大家再耐着点儿性子,耐性子的原因是我们等结构调整开始,在这个时间呢还是按既定方针办,短股长金按既定方针办。考虑到我们国家可能会采纳我的建议,这也不是我一个人建议,可能会进行比较扎实的战略性资源的储备。其中包括黄金,那么有可能会构成一个意想不到的黄金牛市。

当然所有的事情都是有它的自己运行规律的,我们说了也不算,但是我们有自己的一些想法。我在这里边再次强调一下子,不要把情怀当成投资的依据或者投资逻辑,就是我希望做到的,或者是我批判的事情,可能它会明天就发生。所以我希望我给你们讲的课里边的东西和你们的实体操作呢不要有必然联系,好不好?不要有必然联系。我反对超级地租,但我从不反对同学们买房子,我反对超级地租是必须反对,超级地租——社会整体上是被压榨的,我这个事情是必须做的。

但我并不反对你们解决自己的生活。当然我自己有一个基本的逻辑,就是我们必须解决生活,但是以一个底线就行了。就是我们不去过度的侵占别人的空间,别人的生活空间和别人应享有的资源,我们有个底线就可以了。但我不是反对你,不是想你们走入贫困,那我会看得非常难受的,我希望你们都好。这两件事情要分开。我反对超级地租,这个反超级地租反了整整二十年,但这个时间我也在买房,我也不反对你们买房,但我不能不反超级地租。

有的时候人就是这样的嘛,有的时候人,时代的发展就是这个样子的。下个星期是聊天,再下堂课我们就开始讲《资本论》第一卷了。真正的高潮不是今天,真正的高潮是从第一卷开始,我想会精彩的。另外呢就是可能讲价值论其实是《资本论》第一卷是重返价值论,我们这堂我们讲投资学的时候,开篇是价值论,我们又重回价值论。重回这个价值创造和整个的资产的资本化的过程和它这个逻辑,整个的逻辑体系,我们要重返这个。

这部分可能对大家的理解自身的财富状况、理解投资是有巨大意义的。这个准备的时间时程也不短了,遗憾的是我前面读《资本论》的笔记都留在了北京。我再重新再整理,我这个这回算是给大家讲课算是四读。前三读的笔记札记都不在这个身边。有些事情我还能想起来,一些资料也不在身边。方便的话我得回一趟北京去想办法把这个资料拿回来。我想今天我们就说这么多吧。台湾的疫情开始严重起来了。

国内也有零星的爆发。还是老话,就是大家抓紧时间,能打就赶紧把那个疫苗打了,打了那个疫苗身上有点抗体,会有帮助作用的。另外还是要注意保持社交距离,尽量减少群聚,注意卫生、注意安全,愿大家一切都好。另外夏季到了,这个保持一个好的状态。今年虽然是贲卦,对其他人可能不是很好,但是对于我们大家挺好的。因为我们不是那种装的人嘛,所以当原形毕露的时候,其实我可能是出来的是本色,还是英雄本色。

好,就聊这么多。明天下午三点钟我们交换资料,有什么需要补充的,我们明天下午见。好,再见。

\section{资本论的目的\&方法\&风格、郭树清讲话的评述、当下的金融情况\&形势}

大家好,今天是2021年6月12号,辛丑年的五月初三。今天是《资本论》第五讲,第五讲要讲《资本论》的目的、方法和风格。应大家的要求讲一下郭树清讲话的评述,然后再聊几句当下的金融情况、金融形势吧。一会儿三点钟我们准时开始,我先试一下麦。好,一会见。

大家好,今天是2021年的6月12号,辛丑年的五月初三。真好啊,终于进入到第三爻。因为怎么说好呢?一直是觉得这个辛丑年的最难的时候还没到,不希望进入到最艰难阶段,但又是希望这个辛丑年赶紧过去。香港极热,而且雷暴一个接一个,山雨欲来风满楼啊!

我们今天进入到《资本论》讲课的第五讲,我们一共二十四讲。前四讲算是个铺垫,今天开始进入《资本论》,第一讲是讲的是两场革命、两个宰相;第二场是四个哲学家;第三讲是马克思的一个脉络;第四讲是马克思的出身。今天我们开始进入《资本论》,今天的《资本论》还不能进入到第一卷,今天讲的是《资本论》研究的目的、方法和风格。这个非常重要,这个也是想讲《资本论》的时候将读书写作的一些要点给大家一并介绍一下子,以利于大家自己读书和写作。

通常我写文章或者写书的时候,我会想的第一个问题就是目的。我为什么要写?写什么?怎样写?目的是非常重要的。马克思为什么在十九世纪的六十年代决定落笔写《资本论》?这件事情非常非常重要。他的目的是什么?他要解决什么问题?在经历了一系列的革命和运动,在马克思发表了大量的政论文章,在已经发表了《共产党宣言》的基础上,这个时候马克思认为必须为无产阶级做一些理论上的准备了。

这个理论上的准备,实际上在已经开始展开轰轰烈烈的工人运动或者是无产阶级运动,甚至已经建立了无产阶级政党,甚至已经出现了无产阶级革命苗头的这样一个情况下,斗争需要理论。或者说无产阶级争取他们的权利需要系统的理论支撑,原有的政论是不足够的,必须给予一个系统的解释。这个系统的解释导致马克思开始坐下来写《资本论》了,《资本论》要回答的是三个层级的问题。

第一个层级,是马克思认为资本主义发展到十九世纪中叶,无产阶级争取他们权益的条件成熟了,所以他第一个层级要揭示劳动与资本的关系。揭示劳动与资本的关系,这不仅仅是一个雇佣关系,也是一个剥削压榨的关系。这个剥削压榨的关系为工人阶级争取他们的权利,不管是以运动的方式还是以革命的方式争取他们的权利,提供了合理性,但仍然是在法律上不合法的,但是提供了合理性,提供了一个出于天理和伦理……

出于天理、伦理,甚至部分法理的合理性的阐释,这是非常重要的。一般而言,比如说我,我可能会在剩余价值理论写完之后结束了,但是马克思没有。马克思第二卷写的是资本流转,他第一卷写的是价值论,特别剩余价值理论,揭开了劳工与资本的关系;第二卷他讲资本流转;第三卷,恩格斯将马克思遗稿整理的第三卷讲的是社会生产的全过程。为什么马克思一直在考虑第二卷的资本流转和第三卷社会大生产呢?因为马克思的目光并没停留在“造反有理”。

马克思的视野并不局限于“造反有理”这四个字上面。“造反有理”解释清楚了,造反之后呢?我年轻的时候读鲁迅的著作,我印象非常深刻,就是那个小说《伤逝》,后来好像被拍成过电影。鲁迅有一段话震到我了,他说反抗包办婚姻,反抗不自主的婚姻,去跟心爱的人走在一起,怎么办?私奔。问题在于私奔之后呢?鲁迅说私奔之后的结局往往是凄凉的,私奔之后的男女如何面对残酷的生活呢?

私奔之后,用什么来承纳、来包容和承纳那可怜的爱情呢?所以他写下了《伤逝》。可能爱情发展到私奔进入了高潮,但真正的结果可能是“伤逝”——忧伤的伤,逝去的逝。革命也是这样,造反是轰轰烈烈的,革命也可以是轰轰烈烈。革命之后呢?无产阶级推翻了资产阶级统治,获得了生产资料的所有权,甚至建立了无产阶级专政的国家,那以后呢?该怎么办?

无产阶级获取了生产资料,它意味着马克思在《资本论》里第一卷已经开始揭示了资本主义的根本性矛盾产生的原因是生产资料的私有——生产资料的私人占有。当然马克思认为生产资料的私人占有在法律上是合法的,但并不合理,所以无产阶级才可以革命和造反。那么生产资料私有如果是问题的话,那么解决私有的方式是什么呢?比如说生产资料公有制。那么什么叫生产资料公有制呢?如何公有呢?列宁的《国家与革命》给出的方法是由国家占有。

后来发展为城市的国有企业、农村的公社集体占有,这样的方式行吗?就是城市国有、农村集体所有这样的方式成吗?恰恰是在这个问题上,晚年的马克思产生了极度的怀疑和忧虑,因为他担心无产阶级专政的国家,生产资料一旦变成国有之后,那个国有的那个国家机器可能会产生异化,记住「异化」这个词,而且甚至是不可避免的异化。

这一点也被毛泽东深刻的洞见。毛泽东在六十年代就已经把这事儿看明白、看穿了。当生产资料变成国家的和集体的时候,它是否还是社会占有呢?理论上是社会占有,但实际上它变成了代理人管理。代理人会否发生僭越呢?这几乎按照马克思的辩证唯物主义和历史唯物主义的结论都是明确的,一定会僭越,一定会走向官僚垄断资本主义。就是国家资本主义最后的那条出路,或者是如不进行中间阶段的改向,一定结局是官僚垄断资本主义。

忧心忡忡的马克思没有写第三卷“国家与资本的关系”,也没有写无产阶级专政下的社会主义国家应如何处理资本或者是资本雇佣或者是资本利得,并没有再写这个东西。在革命如火如荼的二十世纪初叶,列宁仓促写下了《国家与革命》,那个小册子并没有完成像马克思般的、系统的、具有哲学高度和历史阔度的详细论述。我们行动先行了,在知与行上,我们行走到了前面,有限的知。

这个事情到了中国的六十年代,其实毛泽东开始意识到。毛泽东说党内有走资派,走资派还在走,说的就是代理人民群众治理国家的那部分人,他们直接触碰到了生产资料,并且他们触碰的生产资料,他们在代理人民管理生产资料、管理资本的时候,他们本身几乎是不可避免的异化,而且异化的过程不会太长时间,是比较短的一个周期。我们注意到了,十月革命到苏联解体就是七十年时间,一代人。一代人的时间,异化就完成了,这是一个非常惨烈的过程。

那么,我们在讨论马克思的写作《资本论》的时候,其实马克思他写的这个《资本论》,它的风格跟以往我们读过的从亚当·斯密到李嘉图到马歇尔,所有的经济学家的著述方式是不一样的,这里边有三个不同。第一个不同,它的目的不一样,它不简简单单是为皇家、为政府提供一个对经济理解的、这样的一个具有某种哲学高度的,或者是某种理论高度的书。它不是为统治阶级服务的,这是第一条,这是根本的不同。

第二条,马克思没有从简单的生产过程入手,而是从关系入手,就是从雇佣关系入手。当然谈雇佣关系必须谈商品、货币和劳动力,但是它的侧重点不在生产流转,而在雇佣关系。讨论关系的这个过程是非常重要的,因为当讨论到关系的时候,实际上三次分配就包含在其中了。所以在研究《资本论》的时候,我们可以注意到就是目的决定方法。我自己的写作也有这个特征,我写《广义财政论》和《新社会主义论(2014修订版)》的时候碰到的都是同一个问题,目的决定方法。

另外,马克思的写作风格也有他个人的特征。其实马克思是受法学教育和哲学训练出来的人。法学教育和哲学训练出来的人,他有两个特征,第一个是非常严谨,第二个特征就是他脱离了一般事物描述,他习惯于走向抽象,走向一般。在抽象和一般的过程中,实际上整个的论述少了故事性、少了激情,它不会像《共产党宣言》那样的,所以《资本论》是不好读的,《资本论》不好读,他一开篇他就直接进行了哲学抽象和一般。所以在读《资本论》的时候,开卷是非常难的。万事开头难,你必须咬着牙走进去。

马克思的《资本论》第一卷发表之后,就被各国的工人阶级接纳和广泛的阅读。它确实是无产阶级进行革命和斗争的理论依据,甚至可以说是工人阶级的圣经或者是无产者的圣经。我们不可以要求马克思在那个时候,那还是在资本主义发展的初期,都没到达中叶——现在才是中叶。在初期就解决所有的问题,事实上也不可能解决所有的问题。马克思只是解决了最初期的问题,就是哪里有压迫哪里就有反抗,造反有理。

至于造反之后的事情,马克思没有全部完成。但是马克思了不起啊,他把资本流转说清楚了,他把社会化大生产说清楚了。如果有真正的无产阶级革命家在实践中掌握了真正的马克思主义,或者是读懂了《资本论》,那么他可以在社会大生产的基础上重建国家与资本的关系。在这个历史路程上,我们注意到了两条不同路线的发展,就是东马这条线、西马这条线。东马这条线是以革命建立无产阶级专政的国家而建立社会主义国家。

当然,我一直将列宁《国家与革命》指导下建立的苏联社会主义和中国在1979年之前的社会主义概述为国家资本主义,因为那不是马克思晚年内心深处的真正的社会主义。东马走的这条路是走的一条国家资本主义的路。那么西马呢?关于西马就是西方的马克思主义有很大的争议。在中国的党校或者是中国社科院,他们定义的西马是指西方共产党或社会主义政党和一部分理论家和政治家,他们在嘴上讲的那个马克思主义。

也就是所谓的“和平长入社会主义”,或者是议会斗争来和平进入社会主义。逻辑和路径大体上是没有错误的,问题在于进程。西方或者是西马是谁在推进,他们走的方式方法是怎样的呢?这里我可能还得重复一段旧故事。不重复也不行啊,因为这个旧故事可以把事情再说清楚一些。这个旧故事就是若干年前,我一直心里边有疑问,就是我读《凯恩斯传》的时候,我对1905年的凯恩斯当时的状况,我是无法理解的,所以我到伦敦要找凯恩斯1905……

要去寻找凯恩斯1905年在伦敦居住的那个地方。我为什么要去找这个1905年的凯恩斯呢?就是凯恩斯的著作,他的《通论》我仔细读过、他跟他的老师马歇尔的经济学,从关心的焦点到叙事风格完全不一样,这个不太像他的那个马老师的风格。而《通论》在关心的焦点,就是目的、方法上面,在很多上面它有点像《资本论》。我被两个人的著作,你如果读了《资本论》,读了马歇尔的著作,你再读凯恩斯的著作,你会发现明显的风格上的问题。

所以我怀疑凯恩斯深受马克思的影响,甚至他在大学期间、在剑桥期间,参与了第二国际或者是第二国际的地下组织。我的怀疑是,1905年他从剑桥毕业之后为什么会到伦敦的贫民窟去?他没有留在剑桥教书,没有进入政府工作,没有进入到金融机构或者大型的商业机构,他为什么要去布鲁姆区?所以我带着我深切的疑问,我到伦敦请朋友带我去这个贫民窟,去找到凯恩斯的故居去探访。

后来在伦敦,我发现,到布鲁姆这个地方的人,不光是凯恩斯一个,大概是三十多个牛津、剑桥的高材生,他们大部分人是出身贵族或者是出身豪门,然后他们都去了贫民窟。他的让我联想就是毛主席去安源、走到工人阶级之间去发动工人运动去了。所以我想知道是否在英国、在美国,在19世纪末20世纪初也有着轰轰烈烈的社会主义革命,尽管他们没有爆发社会主义革命,或者是以推翻资产阶级政权的形式……

在伦敦的研究大体上证实了我的想法,确实有一大批热血青年深受马克思《资本论》的影响,深受第二国际的影响。他们那个时候在各个国家内部,特别是在发达的资本主义国家,包括英、法、美这些国家,都有了工人阶级的组织、政党,而且有了年轻一代的思想家,他们特别活跃于各个大学,尤其是大学里边精英的学生们——我们如果看到五四运动,大概你就可以理解,五四运动比他们晚10年到20年吧。

但英国、美国和法国的情形与德国和俄国是不一样的,因为整个的19世纪中叶,就是马克思《资本论》出现,第一国际结束到第二国际建立这个过程中,资本主义的政府、政治家开始关注到马克思主义,甚至很多人是认认真真读过《资本论》的。其中有一个人叫俾斯麦的,甚至深受马克思影响。他可能算是资本主义国家第一个提出对工人阶级进行社会保障的人,铁血宰相俾斯麦。

德国在整个欧洲资本主义进程里边是一个后来者,在政治体系上他们是集权的。在容克地主构成社会主体的德意志帝国,其实对新生的社会主义政党和社会主义者是缺乏包容性的,否则马克思、恩格斯不可能是在英国。而英国相对宽松,英国在政治上是宽松的。它包容了很多、包括马克思在内的社会主义思想家,甚至很多的政党的活动也是在英国进行。这种包容不仅仅体现在意识形态上,也体现在社会治理上。

我没有太多的时间留在伦敦进行阅读,但是我在香港的中央图书馆调阅了一部分的资料,调阅了一部分的那个时期的伦敦的资料。其实在那个时候,英国的政坛、议会和政府里边已经开始出现了大量的具有左翼思想的人士,而且这些具有左翼思想的人士不仅仅是在牛津、剑桥这些大学里边,而且他们的理论论述已经开始走向台面,不是地下了,并且已经开始逐渐的影响,甚至开始推进英、美、法的立法了。

请注意我的描述,在许多的下议院、英国的下议院的议员之中有左倾的人士,而且这些左倾的人士在某种意义上开始推进对底层劳工、工人阶级甚至无产阶级进行某种人道主义保护的立法的推进的工作。这个事情不但引起了英国政府的重视,而且也引起英国王室的注意。请注意,在王室中的年轻人之中,也有大量的左倾的人士。在这样一个氛围之中,英国的立法机构、司法机构和行政机构有意愿改善他们的治理。

所以在特定的历史时期,政治上的成熟有时候会导致一个国家出现一种和平或者是理性的选择。这个时候政府主动去寻找那些个进入贫民窟的优秀的青年精英,他们其中大部分人最后被请到了相关的政府机构、研究机构,这30多个人大部分被封爵。凯恩斯封为男爵,后来成为财长,为英国做了大量的工作,对英国的政治经济产生了深刻影响。凯恩斯的《通论》,至今仍在深刻地影响着我们全世界的经济发展,它仍然是我们重要的一个理论的依据。

西马的更重要的一条线在美国,在普林斯顿大学。普林斯顿大学一向左倾,那个时候就非常左倾,现在依旧非常左倾,普林斯顿大学。当然哈佛、耶鲁也是左倾的,倒是其他的一些学校,像芝大可能有一些右倾吧,可能是这个是受后来的奥地利学派的影响,这边的淡水学派主要是受德国历史学派的影响。其中普林斯顿大学的校长威尔逊是非常左倾的,基本上阅读他的著作,你可以看到他是一个明确的社会主义者,但他是不是某种具有社会主义组织的成员或者是第二国际的人,我就不知道了,因为我没有证据。但是1913年他成为美国总统。

我们注意到在威尔逊短短的执政周期之内,美国进行了深刻的社会主义改造。德国先于美国开始,这个时候美国也开始,法国也在跟进。也就是说在二十世纪初叶,在俄国革命的前后,可能比俄国革命稍早五年左右时间,十年到五年的时间,稍早十年到五年时间,英、法、美都在开始进行体制内的社会主义改造。并非俄国革命之后其他社会主义政党通过议会长入,不是的。西马的起点是这样走的,我刚才讲了两个人,一个是凯恩斯,一个是威尔逊,西马的进度并不比俄国十月革命慢。

甚至恰恰是由于西马,西方的马克思主义先行了一步,所以使马克思的预言未能实现。马克思的预言是应该在英、法、美这样先期的、先进的、发达的资本主义国家爆发社会主义革命。马克思的预言又对又不对,对的是他们没有革命,但他们进行了社会主义改造;错的是那个革命恰恰是在落后的资本主义国家,甚至农业文明的国家爆发了。但无论如何,《资本论》都是伟大的,马克思主义是伟大的,它改写了历史的进程。

其实我今天讲课我可能有点激动,虽然隔着一条线看不见你们,但我还是激情澎湃,像是在演讲。东马是血与火的社会主义革命,在践行、在实践他们心中的马克思主义;西马在和平、在缜密的讨论中、在不流血的较力、角力中,也在推进马克思主义。西方的社会主义改造和东方的社会主义改造貌似有冲突,但齐头并进,社会的文明、人类的文明、地球的文明在马克思主义整体推进下迅速的进化。

东马的伟大实践终于在走到第三卷的时候,东马遇到问题了,就是马克思没有写无产阶级建立自己的国家——建立无产阶级专政之后的国家如何处理三权,马克思《资本论》没有回答。在马克思的全集里偶尔探讨到这个问题,特别是马克思的晚年。什么意思呢?无产阶级专政的国家解决的是生产资料的公有问题,这是一个经济主权的问题。那么政治主权该如何解决呢?政治主权包括立法权、司法权和行政权。如何叫政治主权公有呢?

马克思没有回答这个问题。苏维埃共产党获得了立法权、司法权、行政权,理论上他们是无产阶级的先锋队,是人民的代表,后来我们概述为三个代表。但是马克思是伟大的哲学家,他早就意识到会异化,毛泽东也意识到异化。而且毛泽东的语言是形象的、生动的,毛泽东说:资产阶级不是胎生,不是卵生,是化生。只要有了资本,有了生产资料在,它就会化生出来。

伟大的导师马克思和毛泽东都看到了问题。马克思由于还没有见到过无产阶级专政的国家,所以他只能在想象中提出一些想法,进行哲学逻辑的推导。毛泽东已经见到了无产阶级专政的社会主义国家,他意图解决政治主权公有化的问题。毛泽东想法是让人民直接去占有他们的政治主权,所以毛泽东发动了那场轰轰烈烈的文化大革命——人民群众起来夺权,夺了立法权,夺了司法权,夺了行政权,砸烂了公检法,在各级政府建立了革命委员会。

显而易见,政治权力或者是政治主权公有不应以直接占有的方式表达,因为人民无法行使司法权和行政权,人民应拥有立法权,但人民群众不应直接行使司法权和行政权。人民群众砸烂公检法并且行使行政权力的时候,由于专业能力的问题出现了巨大的混乱。文革证明毛泽东解决政治主权的方法是不恰当的、是不行的。所以这件事情导致了七九年之后的另外一种状况出生、出现。

那么东马在处理政治主权的时候遇到了问题,东马在处理文化主权上同样遇到了问题。在处理教育、学术、传媒的时候,你们知道为什么毛泽东要让工农子弟进大学?记得《决裂》吗?高举起一双手,看着老茧——这就是资格,不是考试,这就是资格。记得吗?毛泽东取消了高考,甚至取消了我们对学术上的一些的职称或者是级别,在教育和学术和传媒上,毛泽东也尝试着重新夺回这个,重新进行公有化改造,重新让人民夺回。

老茧是否成为入学的资格?这件事情其实我们有很多的例证。屠呦呦女士、还有就是像我们的水稻之父,他们都是那个时代的产物,我们那个时代还有很多很多类似的。但我们也知道文化主权的公有化也不适合由人民直接去占有,文化主权的公有也不能这么搞。所以我们78年之后恢复了高考,又走了另外一条路。东马在正确处理,也不能叫正确处理……

东马在形式上完成了生产资料的公有制之后,在政治主权和文化主权上,在一代人发生异化之后,主权逐渐由公有变成了私有,所以到了二十世纪的末端,我们看到了一个神奇的景观——就是西方生产资料私有、政治权力相对公有、文化权力基本公有;而东方社会主义国家生产资料公有、政治权力相对私有、文化权力绝对私有。这个情况是我们不想见到的,但它确实是这个结果啊,很神奇的一个结果。

苏联是一个非常悲惨的国家。苏联在斯大林治下慢慢丧失了思考能力,一个人在思考,所有人不思考。在经历了赫鲁晓夫、勃列日涅夫之后,这个国家迅速地开始完成政治主权的私有、文化主权的私有、形式上的生产资料的公有。他发生了马克思最担心的异化,而且由于这个国家缺乏伟大的思想家、思考者,他们无法解决,甚至他们连进行文化革命这样的尝试都没有,他们也没有毛泽东这样的导师,也没有非常厉害的思想家、思考者。

所以在上个世纪,在二十世纪最后的十年,苏联轰然倒下。他倒下的原因绝不是因为马克思主义,更不是因为社会主义,恰恰是因为他们的政治主权不能社会化,文化主权不能社会化,我们生产资料的公有制,甚至国家资本主义这样的一个形式也最后变成官僚垄断资本主义,它是彻底的私有化导致苏联的解体和破败。他们必须清醒地认识苏联为什么会倒掉?如果这个认识不正确,我们把它怪罪于马克思主义或者是社会主义的失败,那是我们历史的幼稚。

一个真正的辩证唯物主义者,一个真正的历史唯物主义者,一个真正的马克思主义者是必须对苏联的解体作出正确的判断的;如果这件事情不能解决,那么中国未来的路是不会走对的。好,还是回到西马,西方的马克思主义在苏联存在的时候,虽然生产资料相对私有,相对私有的意思就是即便是在英法德美这样的资本主义国家,他们的生产资料也只是相对私有——相对私有的意思就是社会占有的比例是很高的。

请注意,美国的上市公司大股东的占股比例是4.9\%,香港现在是29\%,在某种意义上,它的社会化程度是比较高的。社会化程度——公家的“公”,公有的程度是比较高的。共有,当官僚垄断资本主义僭越了权力之后,那是一种比社会化程度高更恐怖的一种变相的私有,这是苏联垮掉的根本性原因。西马在苏联垮掉之后,遇到了严重的障碍。反身性理论就是你的敌人倒下了,你失去了存在的依据,这是多么地辩证啊!

什么意思呢?苏联解体之后,美国失去了敌人。一个叫福山的人写了本书叫《历史的终结》,他认为马克思主义终结了,社会主义终结了,那么他们就不需要社会主义改造了。他们理想中的资本主义是金融垄断资本主义。所以政治权力在迅速地被私有化,华尔街旋转门拿走了政治主权——立法、司法、行政;经济主权出现了金融资本通过对数据、对Data、对数字资本的垄断形成了对全社会生产资料的高度垄断,就是贫富分化;文化主权,由于他们控制了教育、学术、传媒……

有没有意思?在苏联解体之后,美国在政治权力、经济权力和文化权力开始了让人感到惊讶的去社会化进程,极度私有化进程。这个去社会化和极度私有化的进程里边,贫富分化迅速地拉大,阶级矛盾、种族矛盾不断地深化并激化,美国为这个国家的颠覆、历史性的颠覆准备了掘墓人。我今天对不起大家,这堂课本来是一堂课,怎么我讲得有点激动了呢?

稍微多讲几句,可能没讲完,没讲完也没关系,反正咱们这个课就是慢慢讲。今天有一点激动。这个我们回到中国问题上来,中国为什么没有在1989年解体呢?因为中国国家资本主义在78年之后,我们一般把它时间放到1979年,1979年之后进行了改造,就是我们在国家资本主义的基础上开始引入社会资本,我们开始从1979年到1989年这十年时间以极快的速度将国家资本主义、社会资本主义进行了混合,到了1989年的时候,我们已经初步形成……

到了1989年的时候,已经初步形成了混合资本主义。混合资本主义的好处在哪里?所有人都记得伟大的八十年代,因为那个时代中国有思想解放运动,不要小看文革结束之后,八十年代我们有思想解放运动,那个时候的政治主权、文化主权有一个反向社会化运动,神奇不? 社会化的过程。所以在中国的八十年代,为什么思想那么活跃呢?就是政治主权在社会化,文化主权在社会化。我们很多老大学生重新回到了体制内,成为立法者、司法者、行政者;我们的教育、学术、传媒那么活跃,许多的精英重返体系内,平民精英。

所以,我们虽然在生产资料上在向私有方向急进,但我们在政治主权和文化主权上面又出现了社会化的倾向。历史就是这么辩证,有的时候反身性又让人很神奇。极度的公有制——生产资料的公有制会导致政治主权和文化主权的极度私有化;而相对的生产资料的私有化又导致了社会主权和文化主权的公有化的倾向,社会化的倾向。我们之所以能躲过那场浩劫,恰恰是因为我们八十年代我们那一场的逆向行走。我们要感谢两个人:毛泽东的文革;另外一个人邓小平。这两件事,

这两件事情可能、可能是历史的巧合,也有可能是历史的必然。在我看来是共产主义跟中国的儒家文明有融合、杂交之后产生的一种崭新的现象,或者是崭新的状态,充满了生机和活力的状态。当然,转眼这个时间就推进到30年后的今天了,我们今天在面对复杂的国际环境和国内环境的时候,我们又开始要重新地梳理我们这三权的分布问题了。

为了很好地解决东马和西马的这种认识,对东马和西马伟大实践的认识,为了很好地来理解中国问题,在今年结束自囚之后,我打算明年在香港建立新马克思主义研究会。东马和西马都走到了一个历史的尽头了,新马应该诞生了。新马要解释新时代有中国特色的社会主义理论、新社会主义论,新马会要做一些更有意义的工作。我想新马应该是在中国诞生了,这个时候中国的思考者应该为中国、为世界重新梳理出一条道路来。

也许冥冥之中是你们要求我、强迫我来讲这套《资本论》的。我今天在《资本论》开篇的时候,被你们触动到激动地情绪激动不已,甚至不能坐。我想我们明年开始系统地来把该写的著作写完,该开的研讨会开起来,该做的工作做出来。我觉得我们这一代人还是非常重要的一代人。我们这代人既有西方的强大的学术的能力,又正好在经历中国伟大的实践,中国该产生群星灿烂的,

该产生灿若群星的那么多的思考者,为一个了不起的中国提供她强大的思想基础、精神基础、文化基础。我前两天做过一次概述在演讲上:第一个100年我们吃的是土地革命的红利;第二个100年我们吃的是税政革命的红利。税政改革将为今后的28年提供坚实的基础,它将导致中国今后的28年保持在5\%以上的高速增长;它将使10万高净值100万亿的资本重返内循环;

它将使已经脱离了,甚至准备逃离中国的那部分资本……我们在讨论《资本论》的时候,将讨论资本流转和社会大生产的时候,我们考虑资本在跨国流动的时候对国家经济产生的影响,资本积累率对国家经济产生的影响。我们将使……可能这100万亿已经跑了20万亿了,3万亿美元不见了,还有80万亿重返内循环。如果我们有好多朋友有乡建的理想,那么就应该让它重返乡村,建立美丽乡村;重返水循环,建立美丽生态;重返高科技,建立中国强大的产业升级能力。整个的论述还需要时间,

对不起大家,今天第二个部分应该讲《资本论》的方法,就是《资本论》的研究方法,第三个部分《资本论》的写作风格。方法呢,我想在《资本论》具体章节的时候我再讲吧,我概述一下,你们知道就行了。马克思写《资本论》的方法跟亚当·斯密、跟李嘉图、跟马歇尔是不一样的。因为他使用的是朴素的辩证唯物主义,就是他进行了一般抽象。马克思写作风格里边有深厚的历史唯物主义,他有着对历史的深刻把握,同时他又是……

在方法的论述里边,将来我会结合我自己写作《广义财政论》和《新社会主义论》的时候来理解、来讲述对马克思的《资本论》……将来我们讲凯恩斯《通论》的时候也要讲,就是他们的方法。因为方法很重要:目的决定方法,方法实现目的。就是这个方法非常重要。以后大家写文章也好,做事也好,方法是非常重要的。目的正确,方法不正确,你完不成。你看马克思的这部书之所以能影响,不是影响三百年,可能影响三千年。好多的著作为什么可以跨越时间空间,影响这么深远?不仅仅是因为它关怀的那个目的是高尚的、是重要的,也因为它的方法是非常非常了不起的。

最后概述一下马克思的风格。马克思本人其实是一个具有……我说过就是他是一个拉比世家的人,他身上具有某种传教士的那种风格,而且他受过严格的法学的训练和哲学的训练,所以他又是非常的逻辑严谨的,思维缜密、逻辑严谨这样的一个人。但马克思绝不缺少激情。他不但不缺少激情,在某种程度上依然拥有斗士的精神,不仅有传教士的精神,他还有战士的那种精神、战士的那种风格、战士的那种勇气,所以文字才能感人。

我想今天这堂课就讲这么多吧,讲得有点……哦,课还没讲完,我一会要讲一点那个郭树清的讲话。我想今天的这个《资本论》的第一堂课就讲这么多。以后我会控制我情绪,这个课讲得有一点像演讲,有点激动了。以后在处理课的时候,我会稍微好一点,哪有一个老师讲课讲得自己激动成这个样子的。

银保监会主席郭树清,他前两天说的,他这个发言里边对问题的揭示是非常深刻的。这个发达国家的金融资产和房地产普遍上涨,通货膨胀如期而至,财政支出很大程度上依靠中央银行印钞。世界疫情诸多不确定性,产业链、供应链局部受阻和断裂,美欧开闸泄洪的溢出效应使新兴国家和发展中国家进入到一种极为扭曲的状况。郭树清对形势的判断基本正确。

(全文)郭树清:加快构建新发展格局 努力防止金融风险再次蔓延 https://news.stcn.com/news/202106/t20210610\_3325282.html

我想,这里边跟我们的区别是他的对形势的判断是准确的,但是他可能对产生这种现象的原因和逻辑可能梳理不出来。不过作为一个行政官员梳理不出来是可以理解的。他关心的问题是非常重要的,就是美国在处理收入与通胀螺旋上升的过程中,可能会对中国构成深刻的影响。这件事情,不光是郭树清在关心,国内的好多经济学家也在关心。其中像何新已经多次在讲人民币升值的问题。多数都在关心。

其实呢,我们比他们还关心。我们为什么讲“天降组”?我们为什么讲“平成战败”?我们为什么讲超级通货膨胀?其实我们讲的跟郭主席讲的,跟现在的经济学家讲的都是一样的。但是我们现在关心的是它的外溢。你知道,4月27号拜登签署的这个文件里边,他是希望建议这个合同工的最低工资从11块钱涨到15块钱,大概涨37%。麦当劳等机构很快响应,他们把最低工资全部调到15块钱,金融机构调到25块钱,最低时薪。美国这个国家是可以调的,因为它调完了以后……

调完了合同工,就是普通劳动者的最低工资之后,可能公务员、军队,其他的这些工资可能会,也会跟进,但是它是有个,一有滞后性,第二有幅度的问题。我们遇到的问题是什么呢?我们敢调最低工资吗?我们调最低工资先要调的是谁呢?当然是先要调公务员了,退休的职工,要调这个解放军、公检法。而且大体幅度比例是公务员100%,那么普通的工人、雇工大概是公务员的100%里的30%,农民大概30%里的30%,大概这个比例关系是这样的,也就是说即便是增加工资也不构成社会消费的……

也无法构成社会消费的结构性增长,甚至构成严重的财政负担。所以中国可能不能走美国、欧洲这样的一条路。我们知道我们要拉动消费,我们要增加劳动者收入,但我们现在处理工资的方法,它无法增加劳动者收入啊,是一个问题。将来我们在讨论新马会,在讨论中国的社会分配和这个劳动收入增长的时候,我们会提出一整套的我们的看法或者是方法,那个时候我们再讨论这个问题。那么我们担心的问题它的溢出效应是什么呢?我们担心的溢出效应是,一共是两个问题:一个是人民币的升值问题,另外一个就是我们的内部通胀问题。

这两个问题可能也是一个问题,就是人民币升值与内部通胀。我们担不担心人民币升值呢?我们担心在强烈的人民币升值预期之中,大量的外部资本进入我们的体系之内,然后在人民币猛烈升值之后,它们撤出,形成“平成战败”的模式,这也是何新反复担心的问题,就是“平成战败”模式。那么人民币要不要升值呢?要升值的话升多少呢?其实,其实,我个人认为两件事情都需要进行深刻的思考。第一,升多少?

升多少?第二,怎样避免升值之后资本外逃?这两件事情都想透了,有预案,处理好了,升值,可能人民币升值是必须要升的,是升多少的问题,升值后不出事。就是不让升值这种事情……我看了,有人很霸道,我就不说他们的名字了。因为你作为领导者,你怎么能说升不升值呢?你怎么来说市场上的事情呢?难道你有真的有操控能力或者是你能战胜市场吗?不恰当。

升值可以部分地缓解国内的通胀压力。用人民币升值的方式来缓解通胀压力,好过学习美国给所有的人增加收入或者工资,那个会导致中国财政压力巨大。我们的财政虽然不至于崩溃,但会导致财政巨大的现实压力,甚至由于错误的方法,可能会导致人民币产生巨大的风险,贬值的风险。因为我们不能学习美国、欧洲去大量地印钞,因为确实是有问题的。而且如果这个印钞的这个钱进入到官僚垄断的这个部分,对国民经济的长远发展不利。

第二个部分是我们关心通货膨胀,就是通胀会到一个什么程度。因为美国、欧洲都陷入到收入增长和这个价格增长通胀这种螺旋式循环的这个,这个怪异螺旋循环的这个状态了,那么我们必须考虑外部输入型通胀的对中国的经济的影响。我们应该将通胀控制在一个什么样的水平?如果通胀发生了,我们如何消弥或者是平抑或者是对冲这个通胀对老百姓生活的影响呢?其实这里边有很多的方法和工具。除了涨工资之外,还有很多方法和工具,不到万不得已,不能涨工资。

好吧,我说一下子我的建议吧。因为我今天发个资料给大家看,就是2021年的立法的内容里边没有直接税。我相信明年二十大的时候直接税立法会提上议事日程。我同时也强烈建议直接税立法,例如离境税、遗产税、赠与税、房产税、数据税,获得的税收,不要进行简单的转移支付,而应抵免人民群众的五险一金。五险一金我一直认为是不恰当向人民群众增加的税赋,那么我们抵免它们,相当于给人民群众增加收入,而且不是增加工资。

国务院办公厅关于印发国务院2021年度立法工作计划的通知 https://mp.weixin.qq.com/s/G6s4niBZD3RgevdUzsmBLg

我们用逐步彻底地抵免五险一金来增加老百姓的真实的收入和获得感,这样可能比任何加工资的方法都好。当然,当然,能否有直接税?能否顺利形成抵免?这还需要我们大家进行大量的工作、努力。不要认为一切都是上天恩赐,一切都有赖于我们艰苦的努力。记住那句话:“孑孓踯躅,砥砺前行”!学问如此,所有的事情皆如此。

关于市场大家不用太担心。我今天也不想再谈论那么具体的事情:通胀会不会来?什么东西会涨?怎么样怎样?我只是想说一句话:请相信如果明年直接税立法开始走向议事日程,并且能够回转,减免五险一金,并且强制那80万亿钱重返内循环、重返乡村、重返水循环、重返高科技,那么中国将进入到一个长达28年的发展进程之中,而且从2022年到2032年,将有十年的超级牛市。好吧,我就说这么多。明天下午3点见。

\section{价值论与价值观、当下的经济形势、巴塞尔协议三}

今天是2021年的6月26号,是辛丑年的五月十七号,隔了一周,上周应该是聊天儿的,结果没有能够聊成。我现在这个微博还是禁言的状态,其它的平台除了微信也全部被禁言了。然后这个平台商议之后,大概讲课是可以的。那么我今天我们不耽误时间,还是正式课,就是《资本论》第六讲——价值论与价值观,这堂课非常重要。然后我们如果腾出一些时间来讨论一下当下的经济形势,讨论一下巴塞尔协议三。好,我试一下麦,一会见。

大家好,今天是2021年的6月26号,辛丑年的五月十七,我们今天还是继续我们《资本论》,今天讲第六讲——价值论与价值观。讲《资本论》,价值论与价值观是一个高潮,是一个非常重的题目。我们在讲投资学的时候,开讲也是价值论,今天再开讲还是价值论。价值论真的这么重吗?这么说吧,所有经济学,特别是宏观经济学的起点,它一定是价值论。

再往前说,就是所有哲学研究的核心问题也是价值论。我在之前聊天的时候我讲过三断,断是非、断大小、断远近,因为我们这个平台上的朋友都懂得三断,三断过了,人生无忧、无碍。断是非是一个道德判断,断大小是一个价值判断,断远近是一个关系判断。貌似三断,其实他们的核心都是价值论,你可知价值论是三断的基础。

马克思写《资本论》的时候,他给了价值清晰的定义——它是凝结在商品中的一般劳动。这个定义是精确的,只不过由于它是定义了劳动创造的商品,所以凝结了一般劳动。我今天给大家另外一个价值的定义,这个定义不是经济学的定义,是哲学的定义:价值是主观对客观意义的一种判断。我重复:价值是主观对客观意义的一种判断。

{\kaishu 学界对此定义有所争议。比如,该定义无法解释存放多年后的葡萄酒价值增加。}

{\kaishu 更广义的定义可以是,价值=人类总需求/稀缺性。马克思的定义是其特殊情况——即人类的时间是稀缺的。}

这里边我要先说清楚了,今天这课又有点绕,我们有时候到了关键时候有点绕。不过大家耐着性子听,不行就两遍、三遍听,绕清楚了事情也就简单了。价值是一种判断,是一种分别,是唯心的。这非常重要,不是唯物的,是唯心的,是主观对客观的一种判断,是唯心的。唯心而非唯物,它就与个人的状况,所处的时间、空间有了某种必然的联系。

那么这个判断是否为公认或者是为大家广泛认同呢?那就未必。当然了,从价值论到价值观的时候,观就是一种识,偶尔也会形成共识,那个时候就可能形成公论。当然了,万事万物皆有价值,只不过是当它没被主观发现的时候,它不做表达;它被主观发现的时候,甚至在市场上实现交易的时候,这个时候它就可能形成某种关于它价值的共识。这里边的含义非常重啊,这里边的含义,因为我们待会会讲到人生观、世界观,因为都是从价值观出来的。

我们通常说意义,就是主观对客观意义的一般判断,这个判断实际上这里边所说的意义,大体上是马克思所用的使用价值这个概念。就是它有使用价值它才变得有意义,它如果没有使用价值,它就没有意义。当然了,使用价值和意义还是略有区别的。我一会在细的时候我们再讨论这个问题。价值判断是唯心的,是非客观的,它是一种意识行为。那么它就是相对的了,这就是……

价值必须在交易中来兑现,这个就是交换价值,我们通常也会把它说成是价格。因为在市场中,两个人、多人进行比对和测度之后,形成了一个共识,这个就变成了一种市场价格,而这个市场价格的长周期、跨地域的平均值就变成了它的价值。我们现在开始做一点点的总结,就是价值的内涵。价值的内涵本质上是主观对客观重要性或者是意义的一种判断或者是分别。

请允许我使用分别这一词,因为当价值论这一章节开始出来的时候,我们已经从自然人上升到社会人了。因为高等动物才有这种复杂的价值判断,主观对客观的价值判断,而低等动物只是一种需求性判断,而不会进行价值判断。因为低等动物的需求主要是一种食物的需求,没有复杂到人类如此复杂的需求,所以...

在价值判断这个问题上,人类有时候比低等动物更愚蠢。因为,一会我们讲价值观会说到,价值观会被误导,会将毫无价值的东西判断为有价值,而将有价值的东西判断为无价值。然后由于在价值论上面功课不够,导致对价值的误判,以至于以错误的价格进行交易。举极端的例子,比如说,美丽可不可以作为一种交易品?那么美丽到底应该如何定价?美丽可以用什么方式来出售?那么她出售换取的东西是否能比得上她的美丽?

在讨论价值的内涵的时候,我想说一下子分别这个问题。因为一旦我们开始三断,就是断是非、断大小、断远近,我们开始三断的时候,实际上我们是起了分别心的。一旦起了分别心,我们就变成社会人了。我们既脱离了自然属性,也远离了神性,我们是自然人。为什么我们在讨论投资学的时候要讨论心学呢?就是即便是我们是社会人的时候,我们也希望我们是拥有主体性的社会人,是我们用心来判断。

“正心以中”的中是对价值的精确把握,这是我们心学功夫里边要求的做到的最基础的东西,就是我们必须“正心以中”。我们知道我们的美丽的价值,我们知道我们荣誉的价值,我们知道我们生命的价值,我们知道我们劳动的价值,我们也知道金钱的意义、其它商品的意义,我们知道那个中在哪里,那个合理的合理性在哪里,那个边际在哪里。这个时候我们的三断就不会出错了,我们不会进行错误的道德判断,我们不会出卖自己的国家,出卖人格,甚至出卖身体来获取那些并无生存意义的东西,并无生存意义的东西。我们也不会……

我们也不会被错误的价值观所误导,然后在市场上进行不对等的交易,不去做韭菜。“正心以中”,从哲学意义上来讲,是我们回归价值本源的一种方法;在经济学意义上来讲,就是找到真正的价值的平衡点,然后我们来判断我们所处的平衡点的相关位置,然后来实现我们对整个投资过程的完成。讨论到分别的时候,我还要再多说两句。在价值这个问题上,如果我们上升到佛学的高度,那么佛学对价值的判断,实际上……

实际上它是佛学里边的在大智慧里边对价值的判断,还是龙树菩萨的中观。实际上这个判断包含了我们讨论价值的第二层含义,就是从有分别、准确分别,到去分别和无分别的过程。就是我们心中懂得价值,懂得分别,但到了必要的时候,我们放下这些分别。我们不对客体只做简单的意义的一般性判断,我们不做这个简单的一般性的判断,并以此来指导我们的取舍,或者是指导我们的行为。

佛学的意义在于认识到这只是唯心的、主观的一种印象,它是空的。当然,它在市场上是有的,但在本质上是空的。中观的意思是在空和有之间、好和坏之间、左和右之间、上和下之间、不垢不净之间。对价值的理解到了这样一个境界的时候,我们将可以超越价值,不为价值所绑架,我们可以予取予求。这个时候大体上可以进入到一种价值论的自然状态,或者是超越的状态,或者是超我的状态,或者……

一切的道德判断都是源于价值判断。好多人可能不是很能理解这句话。一切的道德判断全部是源于价值判断,一切的审美判断全部是源于价值判断,一切的重要性判断全部是源于价值判断,这是价值论的外延。不要小看了我们通常在社会上流行的所谓的意识形态,我们所流行的一种文化现象,他们的底子全部来自于价值观。价值观的外延形成了道德判断、审美判断、重要性判断,他们形成了人们的整体的价值观。

当你用价值观去看人生的时候,就是人生观;当你用价值观去看世界的时候,就是世界观。在这个世界上,万里无一有机会接触真正的价值论或者是价值论的本源,不能真正的接触价值论,不能建立起自己对价值的判断的独立的体系,就是我们的主观、我们的主体性,那么你认为你在判断的那个价值观可能是这个社会强加给你的那个价值观。其实,“三岁看大,七岁看老”,里边说的就是价值观生成的过程。

一旦价值观生成,其实你已经被控制,甚至被奴役了。价值论是我们解放的起点。我们基于主体性的发现,然后重新对客体的意义进行判断,是“我”的判断。这个判断,而且是正心以中之后形成的判断,那么我们就可以超越了社会广谱性的价值观,不被带偏。这个时候,无论是在投资领域,还是在生活领域,我们都有独立的高耸的人格,独立的高耸的人格,有悠远的目光,有精深的、精湛的……

可能好多朋友不相信,我在跟我的老师讨论问题的时候,我看大概有一半时间是围绕着价值观展开的。因为,其实分别和判断,分别和判断,对物、对事、对人的分别和判断,其实决定了整个社会的状态。它在经过整理之后,价值观上升为一种宗教的体系、哲学的体系、文化的体系,深刻地影响每一个人的行为逻辑,甚至构成普通人的悲欢离合。因为确实是源于价值判断。

很多人,现在很多的子女想做官,就是属于二三线城市的孩子可能想做官,大官的孩子想做金融,等等等等。这里边构成他们核心的意愿和理想的原因是在价值观里边形成的东西,因为这是个整体的社会在一个特定时期的形成的整体的意识。观是什么?观就是意识。集体的观,当集体的意识形成了,就是意识形态,这个意识形态将塑造这个社会的伦理、法理。

那么马克思在写《资本论》的初期,为什么要先把价值——使用价值、交换价值、相对价值整个的价值的体系、价值的理论说清楚呢?是因为马克思要准备剖析劳动者劳动所形成的价值的一部分被拿走,就是要剖析剩余价值。马克思在哲学一般的这个功力上面是深厚的,他用的这种方式方法,我也是认同的。我觉得确实是,既然是分别,那么就把它分别到底,分得清清楚楚。你不能格物致知的话,你怎能正心以中呢?这个功法或者这个方法呢,我是同意的,我也是……

但是你知道,事物的发展远比马克思所处的时代复杂了。包括人类的劳动、人类的生产活动都已经变得极端复杂了;包括马克思时代的资本化的过程和当下的资本化过程也都变得不同了;马克思时代还是生产过剩的危机,我们现在所处的时代已经是资本过剩的危机。前两天朋友约我出去吃饭,我们进行了一场争论。他跟我说资本过剩的危机仍旧是生产过剩的危机,我眼睛一亮。他说,因为有些人生产的不是商品,而是货币。

他说的是有道理的。当货币成为了一种商品的时候,那么整个的判断、意义的判断,还是要发生改变的。但无论如何,我们都需要通过这堂课对价值论和价值观有一个系统的、完整的、深刻的认识;也通过这堂课,通过这个认识,我们从“走向分别”到“走出分别”,这个起点也要说清楚。就是我们从进入分别,而且进入精细的分别,比别人更精准的分别,到最后脱离分别、走出分别,无分别。这个过程确实……

在若干年前,我看到一个我尊敬的人写的一篇东西、一个小册子,是关于道德的。那是我目前看到讨论道德水准最高的一个小册子。因为他是哈佛的教授,哲学和社会学功底非常深厚,他谈的道德就是从价值论开始的。我当时很惊讶,就是他从价值论开始讨论道德问题,里边有些内容呢确实震到我:就是无论你自己是否有意识,

无论你是否有意识,在你出生之后,在你接触的有限的范围之内,已经开始有人给你灌输价值论。有可能是你的母亲,有可能是你的父亲,有可能是你的爷爷奶奶或者是其他亲人,甚至有可能是你们家的保姆已经开始给你灌输初步的价值论,甚至是直接灌输价值观。它是一种对重要性、对意义的简单判断:这件事情你能做,这件事情你不能做,这个事情是好的,这个事情是不好的。所以到了三岁的时候,已经有孩子懂得钱的意义了,钱的重要性;到了七岁的时候,已经可以懂美。

那么今天我们的课堂上的人,多数都在二十岁以上,其实虽未受过严格的价值论的训练,但价值观早已生成,而且可能不少朋友的价值观可能是存在一定程度的问题的,那么我们是否可以重新地解构再造呢?这是我备课的时候想了很久的问题。因为在讨论、在开投资课的时候,我们就开始讲价值论,就讨论了这个问题,今天我们再讨论一遍,是可以解构和重建的。

我甚至认为,价值论的问题可能是我们所有社会科学的起点。所有社会科学,研究哲学也好、研究其他学问也好、研究经济学也好,所有学问的起点。那么我们今天回到今天的起点上来,对一个商品的价值,就是它的意义的判断,其实并不难。因为马克思定义它为使用价值,其实这个事情是并不难的。但是,但是,市场上给出的价格往往出现了非常大的背驰,有时候超过很多,有时候差很远。

这里边在经济学上,我们再往后讲就会讲到供需、供需曲线的问题的。供大于求,价格就会下去;供小于求,价格就会上来;有一个供需的问题。仅仅是供需的问题吗?当然不是。因为,价值论是一个科学。价值观是唯心的,就是可以被控制、被操纵、被处理。它既有可能是被制度控制或者政策控制,也可能是被舆论控制,所以形成了以背驰为主的现象。

什么意思呢?在大部分时间里价格是背离价值的。如果物的价格背离价值,我们可能会出现买卖的失误,特别对投资者而言,可能是投资的失误。这个就需要我们有一套正心以中的功夫,就是找到价值的、找到它价值的原点,或者是它真确的价值的点,而将它的“观”的部分、把它清理干净,我们知道低了还是高了,要做的基本工作是这样的。物是容易的,那么事情呢,对事情的价值判断就复杂了。比如说对革命的价值判断,

又比如说对文革的价值判断,比如说对一些运动的价值判断,比如说占领华尔街,占领中环,对一些事情的判断,这个价值判断怎么判断呢?还是要回到价值论的。如果土地革命是资产重组的过程,那么资产重组构成了中国的资源的最佳配置,导致中国的工业化和中国整体的经济的腾飞和社会的崛起、国家的崛起,那么你就知道这个资产重组的重要性、它的意义,那么你能看到它的价值。但是站在不同的角度,它有不同的看法。比如说土改、土地革命,有人受损了,

比如说,革命的过程没有温良恭俭让,有的时候简单粗暴,还有流血,所以就会出现很多的伤痕文学、文学作品对这个进行批判,就是不允许资产重组。因为你不资产重组,家里边是大资本家、大地主、大买办,那么就生活会很好。我们前面这二十年天天歌颂民国,歌颂什么呢?就是反对土改嘛,就是要再来一次那个样子嘛。大体上这个时候的价值论就变得极其地重要,极其地重要。一个没有价值论的人是不能写哲学、政治学的,不能研究社会学的。因为你没有价值论的底子,你谈的那个东西未经精算,毫无意义、毫无价值。

如果说有什么的话,可能是一种文学的价值,一种情绪的东西。而且我也甚至从文学的角度,我也不认同好多人的观点,我不认同对中国的历史的某些的文学作品它本身是有价值的,可能它从一个侧面记录了历史,记录了一些事情,但它本身你没有站在一个哲学的高度上,站在一个历史的长度上、阔度上,你没有在这二维的高度和宽度上来看这个事情,而是蹲在细节里面来用放大镜去观察里边的残暴和血腥,那么你可能就不会成为类似于《复活》、成为《国家与革命》、

成为列夫·托尔斯泰那样伟大的文学作品。因为你反映的不是完整的大的历史,而是碎片和局部。这个是从价值论的角度来看事,还要学会用价值论的角度来看人。对人的价值判断或是对人的意义的判断,是你对他人的意义的一种判断,这个判断决定了你要走的路,也决定了你儿子要走的路,甚至决定了你孙子要走的路,这个意义判断就变得非常非常重要了。因为整体上对事的判断、对人的判断构成了当下的整体的价值观、集体价值观。

其实,构成对人的判断、对事的判断和人的判断构成了一个特定时间节点,一个特定的人群。比如说一个国家或者是一个地区整体的审美、审美观或者是整体的一种意识形态,或者是整体的一种文化取向,它会对这个国家构成深远的影响,会对这个国家或者对这个地区构成深远影响。好多朋友会有体会的,你去到东北的一个城市,你去住上半年,看看大家都在说什么;你到深圳,你再看大家都在说什么;你再到香港看大家都在说什么;你去伦敦,看看大家都在说什么;你能发现非常清楚的价值观差异。

如果你能发现这个价值观的差异,约略你应该知道可以做出判断它的经济为什么会出问题?价值论的意义是非常重大的。因为真正掌握了价值、价值论的人,成为哲学家、成为经济学家,成为各个领域里边的专家,那么他就可以重塑这个社会的价值观,就是集体的价值观。当这个集体的价值观形成了以后,就会深刻地影响这个社会的伦理。伦理影响法理,法理影响制度建设,就会形成不同的面貌。这个不同的面貌决定了一个国家和一个地区的,

决定了一个国家和一个地区的兴衰。二十六年前我来到香港,接触这个社会。二十六年前的香港社会与北京迥然不同。他们这个地方人所拥有的价值观与北京的价值观迥然不同。我来香港是三读《资本论》,来香港之后为了写东西三读《资本论》,再研究价值论,再进行价值判断,从物的判断慢慢升级到对事的判断,最后升级到对人的判断。这个分别,慢慢地你就能看出问题,慢慢地就能看出东西,

慢慢地也就可以形成你自己的理论了。“超级地租理论”、“广义财政学”、“广义税税收”、“新社会主义论”,慢慢地你就会形成你的东西了,其实它们的底子都是这个价值论。我们在对物的判断,马克思说的:凝结着一般的人类劳动。那么我们对事的判断,应该对事情、事的价值给出一个怎样的定义呢?我今天这堂课就是关于价值和价值观的这堂课,是我们进入《资本论》的第二堂课,这堂课非常重要,但我今天不给出更多的结论,因为商品的结论马克思给了;事的结论,我有,但今天不给。

对人的,如果人也是一个商品的话(当然人不是商品),那么人的价值该做如何的判断呢?人的意义该做如何的判断呢?以前我们普遍的朋友的,普遍的中国人是两把尺子:左手一把尺子,是衡量你的官职,官做得越大,你就越有意义,或者是你就越有价值,越会受到社会的广泛的尊重,左手这把尺子;右手这把尺子是钱,你赚越多的钱就能越受到社会的尊重,那就是你的意义。两把尺子,官大也行,钱多也行,两把尺子对人的意义进行衡量。

这件事情,对还是不对?前两天在B站放出我一大堆的视频,其中有一个关于这个“个人价值和社会价值”的这个一个简单的说法,可能引起了很多年轻孩子们的共鸣。社会主义的价值观,社会主义价值、价值观强调的是社会整体的进步、社会整体的意义,而不是特别强调个人的意义,特别是个人的权利的意义和个人金钱的意义。为什么?因为如果社会主义的价值论强调的是个人的官职和个人的钱的话,谁会做烈士呢?

牺牲那么多的人前赴后继,为了整个的国家的前进献出了生命,难道那些不具有很高官职,也没有很多钱的人,他们的生命的牺牲是没有意义的吗?如果我们这样的来思考问题,我们是不是哪里出了问题呢?我们这样的问题会不会影响我们的伦理和法理呢?那么我们要建设的这个社会,是不是马克思《资本论》上所强调的那个社会呢?我们建的这个社会甚至可能不如上帝和真主想建立的那个社会,我们的社会魔鬼化。

价值论错了,价值观就会错。价值观一旦错了,它害的不仅仅是一代人,而是三代人。就是我们错了,我们会带坏孩子,会带坏孙子。所以在价值论这个问题上,如果出现差错,那么道德上必然沦丧。所以我同意老师的小册子上的重要的立论,就是他从价值论角度开始出发,对人的价值的判断或者是对人的意义的判断,我们对他人生命意义的判断,如何来确定它的价值?今天我也不给结论,这两个算是思考题吧,一个是事,一个是人。

因为马克思已经给出商品的判断,就是凝结了一般的人类劳动。其实凝结了一般的人类劳动,它有社会性、有社会意义的。那么事情的社会性和社会意义,人的社会性和社会意义,才是……我又快要说出结论了,才是它的价值的本源主体,而非客体自己所拥有的那部分有限的价值,客体拥有的权利,客体拥有的金钱,并不代表客体的社会价值。价值论的部分其实在我们未来处理投资方面的意义重大,处理人生也是意义重大的。如果你的价值论上是清晰的、是明确的,那么可能会少走很……

在讨论价值的时候,不得不说几句价格,因为一般意义上价格是由需求来决定的,一般意义上。那么我举个例子,一个瓷杯,精美的瓷杯,烧好的青花瓷,在地摊儿上卖,十块钱,如果装了精美的封套,有了大师的证书,这个就可以拿到中艺去卖,可以卖一万块钱。如果再精美一点,金丝楠木的……

如果是金丝楠木的封套,里边有故宫博物院的证书,说是乾隆用过的茶杯,那么它的价格可能是一千万。其实它的使用价值就是喝茶,如果十块钱的话,可能它天天它的使用价值发挥着,如果它是一万块钱的话,我估计它的使用价值发挥不出来了。估计你拿一个大师的茶杯每天喝茶,我估计你很怕碰坏它。如果是乾隆御用的,它就根本不可能再碰茶和水了,因为它的意义已经发生了,悄然发生了逆转。一个茶杯是这样的,其他事……

其他事物何尝又不是如此呢?在我们当下纷繁复杂的这样一个社会生态里边,真正的凝结了劳动的一般商品,其实它的价值常常是被低估的,所以会出现价格扭曲。这里边确实是马克思看得很透,越是一般劳动,越容易被人剥夺剩余价值。而在另外一些商品里边,我刚才说的那些东西,它是另外一种反向的剥夺了。因为这只茶杯……

就算是大师做的,它的使用价值是不是有一万?画一个问号,至于一千万,作为一种文物,它的存在那是另外一个意思了。我想说的是,在我们所处的这样一个色彩斑斓的社会,这种局到处都是,在资本市场里边更是屡见不鲜,到处都是。如果说商品如此,那么事情比商品糟糕一百倍,事物的复杂程度,它的价值判断就更复杂了。那么对人呢?人的判断比事物的判断本身更复杂。

马克思在《资本论》第一卷第一篇里边给了价值的定义,给了使用价值、交换价值、相对价值的基本定义。马克思没有给出全部的工具,就是如何测量,如何精准测量商品的价值。不是马克思不想给出来,是因为通常我们认为这个价值是要在交易中来,一个长时间的不同区域的交易的平均值来做出表达,而非在此之前就能知道它的精准的价值,因为……

但是我们必须站在哲学的高度上,或者是我们用我们的心学的功夫来处理复杂的价值判断问题。要知道我们当时提出来三断:断是非、断大小、断远近,其实就是价值论的运用。就是我们必须学会三断,我们必须进行基本的是非判断,错误的事情我们是不做的。我们必须对事情的重要性做出判断,先做最重要的事情,不重要的事情,不必理它,不必因情绪的宣泄而产生不必要的那么多的麻烦。距离的判断,因为我们的人生命是有限的,时间是有限的,我们必须跟最有意义的人……

其实我很感激,我很感激大家。因为因缘际会,我们在一个平台上,我们走到了一起,如此之艰难,还是不离不弃。原本是本周的课应该再停一下子,过了七一再说,但是我看着大家心心念念,我其实内心深处在备课的时候也很思念,所以我觉得这堂课特重,所以还是要讲的。这是一种重要的价值判断,因为这件事情它真的是意义重大。对我意义重大。对你们意义重大,对我们彼此意义重大,它是一个判断。

说几句价值观。其实“价值”这个词是现代词,我们古代通常不用这个词。“论”这个词倒是古代就有,“观”这个词我们古代也不是很,不会用太多。“论”和“观”的区别在哪里?“论”往往是一家之论;“观”则往往是众人之看法;众人之看法 — 观。价值观的塑造是一个社会的底层或者是高层对社会治理的一种重要的途径和手段,就是塑造价值观是非常重要的事情。

在拥有强大主体性的民族和这个拥有强大主体性民族所领导的国家里边,价值观是不容其他的民族,或者是国家或者是国家机器予以强行的灌输,或者是操纵或者是操弄的,这一点非常重要。我们常常提一个地方,教育出了问题,出什么问题了?就是价值观的塑造出了问题。我们前一段时间满屏小鲜肉,在西方社会里边,是美国英雄,是集体主义精神的英雄。

而在我们东方,无论中国、日本、韩国、台湾、香港,全部都是小鲜肉,毫无集体英雄主义意义的个人的小鲜肉。拍什么电影《小时代》,出什么小鲜肉,好像个人意义——个人有限的价值、有限的意义,就代表了社会的价值和社会的意义,其实价值观的挫败会导致一个国家伦理的沦陷,伦理一旦沦陷,法理不彰,这个社会很快就慢慢地走向衰落。这个事情我们在苏联和日本……

这个事情我们在苏联和日本都看到了,苏联的解体和平成战败,在1985年前后的苏联和1985年前后的日本,我们看到了就是价值观整体的沦丧。价值观整体的沦丧,一个国家不用锤、不用打,一定倒的。今天这堂课的重要性可能不在价值论本身,而在价值观本身。一个社会形成一个正确的价值观,那么这个社会的所有的人才有正确的人生观和世界观。有了正确的人生观和世界观,整个社会才能够去努力的创造价值,不做食利者,以创造价值为光荣,以食利为耻辱。

我在香港这么长时间,我看到香港那个价值观——笑贫不笑娼的价值观,它对人性的毁灭,对青年人的毁灭让人震撼不已。而在这样一个社会,价值观被操控的是它的教育、学术、传媒啊,操控教育、学术、传媒的人却成为了座上宾,成为了最受欢迎的人,最受敬仰的人。好吧,不能再说了。今天开这堂课,好多朋友都反对,说你禁言期间还是不说话的为好。

但是今天正好是讲《资本论》,按照我安排的这个大纲呢,这堂课又不能取消,这个价值论和价值观这堂课。而且价值论这堂课由于我们上次讲过一堂了,可以不展开。但是关于价值观的部分,这涉及到《资本论》的后边的主题呀。《资本论》是写给谁的?是写给无产阶级的,让他们做什么?让他们进行社会主义革命和社会主义改造。社会主义革命和社会主义改造,改造了什么?改造的是价值观呐。通过价值观的改造,改造了社会的伦理,通过社会伦理的改造,改造了社会的法理,通过社会法理的改造,建立了优良的社会主义制度、社会主义政策。

那么如何形成正确的价值观呢?其实在讨论物的意义的时候、商品的意义的时候,就是商品的价值,我们给予它一个简单的定义,就是凝结了一般的人类劳动。但我今天说了,讨论事的时候,就不能用凝结一般的人类劳动来做衡量,我们必须讨论事情的社会意义,也就是说事情的意义不在个人而在社会。如果是个人获得了意义,而社会没有获得意义,那么它的价值、它的意义可能是负面的。

我们对人的评价也是如此,不代表你做到最大的官,你就有意义;不代表你是周永康、徐才厚,你就有意义;不代表你身家亿万、富可敌国,这个人就有价值、有意义;不是的,是你的社会意义。在某种程度上,黄继光、邱少云、雷锋、王进喜、陈永贵,他们可能比有些官、有些富商更有意义。因为我们社会整体的价值观,如果是一种社会观的话、社会主义观的话,我们就会用一种崭新的视角来看待事情和人。

关于价值论和价值观,我们就讲这么多,一个小时整。我们留点时间,讨论讨论当下的问题,因为中间隔了一次聊天儿,好多人觉得遗憾,另外这两个星期又恰好是风起云涌的时候,可能好多人紧张了,我们聊几句。我想这样的聊天,就是我们先不进入事情里边,我们先进入这个分析框架里边,从分析框架出来,然后我们再谈事情。这样的话呢,可能会把事情谈的稍微的透彻一点儿,谈的稍微的清楚一点儿。

先说军事。在2021年,在人类进入二十一世纪的第三个十年这个时间节点,军事上的优势是什么?军事上的优势不表达为军舰的数量——就是海军,不表达为空军——飞机的数量,亦不完全决定于核武器的数量。一个国家,宇宙空间的能量和电磁空间的能量决定了它的军事优势,决定了他在军事上的优势。

如果我们真正的具备了太空上的优势,具备了电磁空间的优势,那么在处理地球上的海、陆、空军的时候,属于降维打击。一个看得见的人,用砖头还是用子弹,对一个瞎子而言其实区别并不大。但是看见这一件事情就需要太空和电磁空间的强大的能力,事实上我国在这两个领域里边增长、进步神速。为什么别人对我们的太空站、北斗那么恐惧?为什么别人对我们的华为、5G那么恐惧?

就是因为我们在宇宙空间和电磁空间取得了相当可观的进步。我们没有证据可以证明我们获得了比较优势,但我们已经取得了相当可观的进步。可观的进步,在某些领域里边可能我们可以势均力敌了。这意味着中国在军事上已经进入到了一种不能被简单或者是轻易打败的境界。也就是说,军事上的欺诈或者霸凌再想像匈奴或者是突厥人——匈奴对大汉、突厥对大唐的招亲纳贡的那种可能性没有了。

既然这种威胁不存在的话,那么中美的博弈或者是中国与西方的博弈,在军事上取得均衡之后,就是在三个流上面的博弈:一个是产业生态、产业链,一个是商品流,一个是资本流。就是“一个生态两个流”这个博弈。在产业生态方面,中国具有无与伦比的优势,因为我们是十四亿人,我们有基本完整的一个产业生态,产业链是完整的,而且我们的量——内需的量和外部这个供给的量都足够大。商品流转方面,我们大体上也、即便是受到了某种的限制……

即便受到了一些限制,但是我们看一下子我们的贸易的往来就能看得到:就是中国第一大贸易伙伴是东盟——记住它不是美国;第二大贸易伙伴是欧盟;第三大贸易伙伴是美国;第四大贸易伙伴才是日本。也就是说对中国而言,在商品流的部分,即便是有贸易战或者是其它的封锁,对我国的影响已经到了不能致命的那个程度了,只能影响不能致命。那么中国现在唯一可能出状况的是资本流。在我们讲课讲到《资本论》第三卷的时候我们会讲资本流转。

那个时候我们会用马克思的方法来分析资本流转。资本的流转以前是围绕着产业链和商品进行流转的,到了二十一世纪,资本的流转主要不是围绕着这个产业链和产品流动、商品流动来围绕着这个流转,资本的流转主要是围绕着资产而进行的。围绕着资产进行流转的那个资本,因为剩余资本太大,就是生产和商品流转根本不需要这么多资本,这个资本流转里边包含了、孕育了巨大的问题和危机。就是所谓“割韭菜”就是这个意思,因为它不断地在全球资产里流动。

那么当剩余资本流入哪一国、哪部分资产的时候,那部分资产就迅速泡沫化;当资本撤离那部分资产的时候,资产泡沫破灭,形成了大规模的经济危机。大体上就是这么个情况。我今天说了,军事上已经没有问题,我们不是大唐,不是大汉——就是匈奴用军事来威胁大汉的那种事情,今天已经一去不复返了;突厥用军事威胁大唐的那事情一去不复返了。我们不需要卫青或霍去病,我们现在拥有太空和电磁空间技术,我们可以足以防备自己,足以完成防御,完成对侵略者的各种的防御,那么剩下的部分就不在军事了,在经济里边。

那么经济里边,产业链、经济生态我们是没有问题的,商品流转没有大问题,虽然存在着小的问题,没有大问题,问题出在资本流转。资本流转一旦控制得不好,就会形成某种局面。请牢牢记住,平成战败就是资本流转带来的资产价格的泡沫和破灭的过程,形成了美元的价值的填充和再确认。苏联和东欧地区的解体也依旧是这个道理,原理是一样的,是资本流转。资本主义的真正的杀伤力在资本流转上面。这个事情,我们今天这个因为我们讲《资本论》直接进入到第三也是个办法,可以先倒着讲也行。

因为我们在讲这个“商品货币“第一卷讲完了以后,讲“资本流转“是第二卷,讲“社会化大生产“是第三卷。其实马克思在“社会化大生产“里边已经谈到了资产价格的问题了,谈到资本流转里边包含的内容,这个除了商品还有就是资产。那么资产这个事情怎样流转,马克思是有初步的逻辑和判断。我们到时候我们会在马克思的基础上再做我们自己的补充和完善,来判断整个资本流转的整个的过程和可能会产生的深刻的、深远的影响,就是它的意义还是个价值论的问题。

回到当下的现实,我们国家在2008年之后到现在,我们也确实是向整个的社会投放了远超商品流转所需的货币,就是我们投入的货币量远超商品生产和商品流转所需的量,就是超过了这个量。超过的量怎么办?那么就进入到房地产,形成了房地产的泡沫,形成了房地产的这个巨大的泡沫。这个泡沫也构成了地方政府的地方债,也构成了家庭债务,也构成了一大堆的社会问题,它也对中国经济构成了一种沉重的压力。

这个压力具有两重性。一重性就是由于泡沫的产生资产价格溢价的速度远超生产企业生产的利润,那么生产企业就倒下了,就走了,然后大家都转进泡沫炒作当中去。这件事情,“房住不炒”压一压以后有一点好转,但现在还不能说这个问题解决了,就是让资产溢价的水平低于企业的利润,低于劳动者所得,这需要财政的直接税。这两天因为直接税的事情捅了马蜂窝,好家伙上上下下、左右都在攻。

我们先不讨论我们该做什么,我们讨论可能发生的事情。现在不光是中国多印了钱,美国印得更多,欧洲印得更多。那么多印出的钱除了进行商品生产和商品流转之外,就必须得进资产里边去。那么全世界现在都是资产荒,那么美国的楼市和股市都炒到天上,那么炒到天上它也不够,它还得出门,像八十年代一样的出门,还得出远门。那么它就对中国构成了一种经济学意义上的悖论。什么意思呢?就是我们肯定要出口商品,我们商品出口肯定多,我们可能也要让资本进来。

那么我们收了大量的贸易项下、资本项下的美元,怎么办?我们没有那么大的需求,就把它花出去。那么我们只有两种可能性:一种可能性呢,是让10万中国高净值,这10万人拥有100万亿,让10万高净值去国,就是他们带着我们换回来的美元,跑掉——跑向全世界。这个2012年到2018年,这个事情已经发生过了,大概走了3万亿的钱;走了多少人我没统计。大概高净值100万亿,走了20万亿吧,还剩了80万亿,现在这80万亿也在准备着走,不让收税就这意思了,想走。

那么如果不让走,不让高净值走,我们手上收那么多美元怎么办?我们的意见是非常简单的,全部转做战略储备,比如说两万吨黄金,比如说贵金属,比如说我们暂停稀土出口。我们不需要赚取外汇了,因为外汇是个问题。为什么?因为我们需要让手中的美元在黄金上沉没,而不是让它变成高净值出走,这件事情对中国具有决定性意义。如果我们做对了,那中国未来的28年,就是第二个100年一点问题没有。如果做反了呢?那真的是不知道该说什么好。还好,有我们,问题不大。

最后说几句《巴塞尔协议》。《巴塞尔协议》的第三部分,《巴塞尔协议(三)》实际上对黄金的意义做了一个重新的评估,实际上是对美元价值的一次深刻的怀疑。我始终认为,我们的判断是准确的,但是在时间节点上、在行为方式上,可能并未如我们的意愿。有些朋友有意见,可能跟着我时间短,其实我们一直是准确的操作的,从两化的股票,所有人都赚钱,到我们在一个重要的时间节点全部转入黄金。

转入黄金,现在可能是没有赚到太多钱吧。当然了,转黄金的股票的,紫金,转股票、转其他产品可能赚得多一点。买纯金和纸黄金,可能这个赚的钱并不是特别的多,或者是特别的明确。但有一条,就是我们转过来以后,避免了在其他方向上的损失。因为在资产四矩阵里边我们不做……当资本流离开其他资产的时候,我们准确离开其他资产就行了。我们新进入的资产是否能够赚钱,可能是需要时间的,这个时间节点可能未如我们预料,没有及时到。其实说来说去,不及时,也就是这半年多的事情,半年一年的事情。

我觉得事情只要是方向对了就行了,方向错了,你细节上的正确有意义吗?所以还是要按既定方针办。这已经是六月底了,九月份不远,九月份才是关键的时间节点。因为美国的财政年度是从10月1日到9月30日,到九月份的时候,川普的财政年度就结束了,将开启拜登的财政年度。如我预见没有差错的话,大体上,全世界都在为这个时间节点做准备,包括美联储也在为这个事情做准备。

全世界都在为这个时间节点做准备,当然了也包括我们这些人。好多朋友耐不住性子,是想去A股里边,重返A股炒作一下子。A股里边有好东西,有些。如果去年买完紫金,今年买完航运,那么整个的节奏就都出来了。但多数人可能在操作上到不了这个程度。我已经提示了,有些东西可能快接近底部了,类似于像一些大型的两化企业,类似于像阿里、腾讯,可能在慢慢地接近底部。但那个调整的过程到底是一个什么状况?今天我没有把握说。两件事情,一件事情,我对疫情我没有把握。

两件事,第一件是疫情,会不会再有最后蹬腿的一下子,我真的没有把握,所以我觉得要谨慎。我是一直不把疫情当成一场危机,而是当成一场战争来处理的,这是第一个问题。第二个问题,是我们在看中美关系的处理。因为拜登走访完欧洲一圈,跟俄罗斯做了沟通之后,中东已经完全放弃;俄罗斯,我看他未必能够准确的拿下;欧洲,貌似拿下,其实没有拿下,因为他不该跟英国签《大西洋协议》。

那么也就是说,他编织了一圈的反华联盟不行,他自己又没有能力进行类似于匈奴或突厥的那种武力突袭,兵临城下、招亲纳贡,他现在没这个能力。那么他唯一能做的就是1985年平成战败模式,大量资本涌入我国,使我国的通货膨胀和资产价格双膨胀,出现一个状况。我们在等——因为这需要中美两国首脑在一个特定的时间节点见面、达成某种协议,然后资本进来,然后形成一个局面,我们在等,等着看是否有这个局面。

如我国政府接受了我们的建议,迅速地将剩余美元全部沉没,全部买金,买战略储备、买碳排放权,那么我知道将会出现另外一个局面。所以两件事,两件事一个是疫情;一个是中美博弈的最后中国的策略,现在还不明确、不明晰。但到了九月底也都看明白了,那个时候我们再调整我们手中的方向,因为我觉得现在在金里边是最安全的,可能不赚钱,但它是最安全的、最稳妥的。其实我心里边中国牛市的起点是直接税,开征直接税是中国十年牛市的起点。未能开征直接税,我看不出来这个牛市如何……

我想,今天我们加上课加上聊天就这样。这个对不起大家呀,上堂课就没讲成,今天算补一点儿,给大家一点点的补偿。我还在禁言,禁言包括头条、包括B站,所有地方好像都封了,封了也挺好,封了我就休息呗。反正只要平台可以给点缝隙,我们的课还正常进行。我想说的是,外边对我的攻击,对我的批评、批判比较猛烈,大家疼我,我知道,但不用去管,也不用去反击。

我们做我们该做的事情,就是不去那些细枝末节上去与他们进行讨论、争论。我其实觉得不是对和错的问题,是没有意义,主要是没有意义。因为如果能进行比较高层级的学术研讨,或者是比较高层级的碰撞,哪怕是批评都好,就是它没有意义。我们大家也不用去理他,不用去理这些人,这些事,没有意义。

另外就是说好了自囚,结果,这个国仁乡建这个全球大讲堂,讲了堂课,结果还是惹了一大堆麻烦,就是还是高调了一些,结果搞得B站很火,搞得其他的网站好像意见也很大,好多人好像很敏锐。我们还是自囚,再安静上这后半年的时间,到了明年我们再开启一系列的事情。无论如何,我非常感谢大家——大家对我的关心,对我的支持,非常感谢,谢谢你们。明天下午三点钟我们见。

\section{价值尺度——资本论里的货币及最新货币理论、货币的意义、货币的两化问题、降准的事情}

大家好,今天是2021的7月10号,辛丑年的六月初一。我们今天还是讲《资本论》的第七讲:价值尺度——货币。我试一下麦,然后三点钟我们准时开始。好,一会儿见。

大家好,今天是2021年的7月10号,辛丑年的六月初一。今天是一个神奇的日子,就在我刚才准备给大家发“谢谢整理资料的朋友”这个话的时候,接到了新浪的通知,然后我又再一次被禁言了。这回禁言的时间可能要更长一些,我不能在微博上再发任何东西了,但现在可能大概讲课还是可以进行的。那么我们就以后就先不发表任何东西,就讲课吧。到底是我说什么了?其实我没说什么。但是,嗯。

可能……好吧,我们为了保住平台,能继续跟大家有一个这样聊天的机会,我们就不再做任何这方面的解释和说明了。另外,也希望大家从此过往在平台上发言不要有过激言论,也不要再去批评或攻击其他任何人了。现在这个情形如此,我们还是安安静静地度过这段时间。

今天我们两个部分,一个是《资本论》的第七讲《价值尺度——货币》。其实这一段非常重要,就是谈货币这一个我们将来在谈资本转化的时候还要再谈一次货币,那个时候可能我们要花的时间稍微多一些,因为货币是一个很大的问题。这个我一直是想在货币这个问题上多花点儿时间,因为我主要是考虑这个货币,如果我要是不讲透的话,其实《资本论》很难讲得明白,也很难讲得有深意,

因为货币的发展的状况,到今天为止,已经远超马克思写《资本论》时候的货币的一些特点。我们今天先从马克思的《资本论》的货币的论述着手,然后我们延伸至货币的最新的货币理论,延伸至内生货币和外生货币,但这不是我们今天要讲的重点。我们今天主要是要讲的不是货币的本质、货币的种类,我们今天要讲的是货币的意义或者是货币的作用。货币的意义的部分可能对所有人都是至关重要的,我们希望有关的事情呢我们能讲得透彻一点。

“币”这个字是分巾的意思,巾就是这个毛巾的“巾”啊,这个巾这个字上面一撇。这个巾,手持巾,巾就是旌旗的意思,执巾者为帅,“帅 ”这个字就是执巾者为帅,执币者为师。什么意思呢?其实,站在高台上擎起大旗的那个是统帅,站在高台上将一部分的这个旗子的条,分发给大家的那个叫师,或者是那个叫德高望重者。

“币”其实它的出现,它不是交易媒介,它是记账符号。“币”它的产生是记账符号,人们拿它作为一个记录,什么意思呢?比如说打猎下来了,大家分东西,可能有些人分到,有些人没有分到,没有分到的,那么就欠你一份,给你一个符号做记录,就是你下回拿这个来取。或者在这次狩猎中你立了大功,但是我们每个人都分一份儿,也不能多给你一份儿。但你立了功啦,给你一条,下回有的时候你可以多领,它是一个借据、借条或者是个欠条,这是古代“币”的含义。

总体上从远古的币,无论是这个金,还是这个贝壳,还是后边的甲骨,其实可能都是一种叫什么好呢,叫记账符号,比较好,比较妥。但这个记账符号又可以流通,它就可以作为钱币来使用了,因为它这个记账符号意味着你可以兑换一定量的东西,它就是它是可以作为媒介来处理商品流转的,所以这个币的本源、起源就是这样。发展到相当阶段之后,人们找到了有价值的,跟其他商品同样具有价值的……

那样一种特殊商品,比如说金、银、铜,就是后边形成的这种金属的货币,我们管它叫一般等价物。它本身就是有价值的商品了,它就不再是一个欠条,不再是个记账符号,它本身有价值,并且他还有除了可以作为价值尺度,还有价值存储的功能。当然了,再往后发展出现了纸币,也就是我们所说的信用货币,这个时候信用货币本身也有一定的价值存储功能,所以我们有外汇储备,就是这个意思。货币的本源与进化的历程,几千年来,

货币从这个巾,从一个欠条一直发展到金银,发展到纸币,发展到现在的数字货币,大概是一串字符串,大概是74个数字形成的一个字符串形成的数字货币。我们通常就把它称之为钱。马克思对“钱”的定义基本上四句话:第一句话就是它是价值的尺度,就是我们衡量一个商品的价值,他是用钱来作为价值尺度;第二句话,马克思上升到哲学高度,说它是一般等价物,一般等价物就是它是一个参照物,是个媒介。

第三句话就是马克思认为货币有存储价值的功能,就是价值储藏的功能。第四个功能,马克思认为货币是一种特殊的商品,它还是个商品。在货币的这种历史的沿袭上面,就是从古代的欠条到后来的金属货币,到最后的纸币,到现在的数字货币,我们看到的轨迹实际上是人类从最初始的信用又回归到最初始的信用。

好像是一个大历史的渐次的回归。后来到了现代,到了当代,这个人们就把这个货币又做了细分,因为在纸币的环境下边,其实我们现在是处在纸币和电子记账相融合的一个特殊的状态下边。所以我们《现代货币理论》又把货币分为内生货币和外生货币。我做一个简单介绍,今天这不是我们的侧重点。

内生货币是指货币存量是由实际产出、利率、物价水平等经济变量的变动决定的。相对,外生货币是指货币存量是由外在于本国的生产过程的某个机构,通常是中央银行提供的。内生货币理论强调中央银行不能完全地、自主地控制货币数量,而实体经济的运行会在商业银行体系内自动生成所需要的货币数量。我估计大家可能不一定能听明白这个意思,我简单再做一点点的介绍。我们因为这个货币理论比较复杂,特别是内生货币和外生货币理论比较复杂,但这个事情是必须懂的。

在货币数量论的两个源头,一个是英国的剑桥方程式,一个是美国的费雪方程式。通常我们讲费雪方程式讲得是比较多一点。这两个方程式分别考察了价值、储藏货币,我们通常管它叫“坐着的货币”和交易媒介货币——我们通常管它叫“飞着的货币”,与货币收入间的稳定关系。前者解释为收入中在一定的既定比例,后者解释为货币收入流通速度和交易值与收入比例的稳定。凯恩斯在《就业、利息和货币通论》中,同时考察了货币的需求、预防和储藏职能,但是保留了存量分析。

就是后凯恩斯理论认为当时的这个凯恩斯的理论保留了存量分析,忽略了流量分析。关于这一点我是认同的,就是不是个动态的,它是个存量分析而非流量分析。这在货币理论就是当下的货币理论里边,它的意义是非常重大的。后来在弗里德曼和卡甘的著作中,外生货币它就进一步延伸发展出表述为货币乘数理论,它就是它的动态的表述开始出来了。

涉及到一些具体的内容,我想今天我们不在这里讲,因为这涉及到整个资本流动,我想在资本流转的时候再详细介绍相关的内容。在现代经济中,货币供给的决定因素主要是考虑货币需求与货币供给两个方面。有一个转折点就是1959年的《拉德克利夫报告》,它是最初始的对货币供给内生性的讨论,后来到了卡尔多和温特劳布发展成为后凯恩斯学派,构成了内生货币供给的重要的理论的体系。

实际上呢,这个内生货币的理论体系就开始与我们当下所讨论的通货膨胀就有了很多的必然联系。然后这里边形成了很多的理论分析框架和模型,其中不少人就是因为这个分析框架和模型获得了诺贝尔奖。但我不认为这些数学模型或者是这些理论分析框架能全部解释当下的现实,因为在理论假设的前提下,只有极少数的经济体系适合那样的假设,而大多数的经济体系有不同的内在和外在的条件,往往那个模型不能解释和说明问题,尤其是对中国不能说明问题。

今天让他们给我这个……我这刚有了发言权才10天,又给我取消了发言权,这10天我好像没说什么呀,没说什么就还是不让说话了,搞不懂是什么原因。看来这个B站也是很危险的。这个……好吧,说好了自囚,自己不好好自囚,叫人家强迫你囚了,变成了外囚,自己不自囚变成了外囚。也好,就是静下心来读书写作吧,把精力放在正经事上,少说点话。

我们讲今天的第二个部分就是货币的功能或者是货币的意义。人一生之中有两样东西的分配是至关重要的:一个是时间;一个是金钱。时间和金钱是由你来分配的,你的时间是有限的,你的金钱是有限的。如何使用时间和使用金钱,它将决定你的命运,它是你生命的意义,是你生命的价值,它也表达了你当下的生存状态。关于时间的分配,我们今天不讨论,将来有时间在聊天的时候谈哲学,我们再讨论时间问题,今天我们讨论一下金钱的分配问题。

举例:一个女生有了一万块钱,她会做什么呢?她其实有多种选择,比如说拿这一万块钱去美容和整容。美容或整容这是一种选择,也可以拿这一万块钱到书城二十三楼,那里边有各种补习班,有计算机的补习班、会计的补习班,有各种补习班,倒不是正规的学校,但那个补习班很好,可以迅速提高你在某个方面的能力。当然了你还可以有其他的消费的选择。那么这意味着什么呢?

意味着你自己出于本能对你自己生产力工具的确认。这说得有点哲学,我这样说吧,就是你可以有三个方面的方向的选择。第一是恢复再生产,就是你作为一个劳动力,作为一个再生产,你要有让它正常恢复,所以你需要的居住和吃饭吧,穿衣、吃穿住嘛——再生产;第二,你提高生产力水平,比如说去学习,这是提高生产力水平;第三,获取资本性收益,就是投资。

多数人,当他开始自己拥有了钱,不管这个钱是继承父母所给予,还是自己挣得的;不管是遗产赠与还是劳动所得,当你有了钱之后,你都面临货币功能的选择。回到刚才举的例子,一个女生她去做美容或者整容,她认为美丽就是生产力工具,她认为美容和整容可以增加生产力工具的功效。如果她去书城二十三楼上补习班,那么她认为她的生产力工具是某种职业能力。

那么我们这里边就有一个问题了,钱和时间都是有限的。再生产的恢复可以有极为奢华的恢复,就是你住十平米和住一百平米是不同的,你穿名牌和穿普通衣服是不同的,你去下馆子和自己煮饭是不同的。也就是说有一个比例性关系,就是你投入到再生产多少钱?投入到提高生产力多少钱?投入到资本性收益多少钱?这个事情在一次性上貌似不那么重要。

然而,当你持续地将时间和金钱进行更为合理和有效地分配的时候,五年、十年就发生变化。你和其他人之间产生了变化,一生过去那么就截然不同。以我为例:因为我可能是偶然,不是必然。我大部分的时间是用来阅读和写作的。我在生活上的要求大体上虽然不是个极简主义者,但是我对再生产的要求不高。

就是我对居住要求就是和平、宁静、衣服足够就行了,可以保暖、满足正常工作需求足够就行了,不追求名牌。对于吃就是如果我自己做的话,就是营养和卫生就可以了,没有特别过高的要求。但我对生产力工具是在意的。所以我的钱在二十多岁的时候,零花用的钱可能一多半是用来买书了。对资本性收益我也是很在乎的,就是当我开始意识到需要资本性收益的时候,我尽可能地拿出一点点钱来做资本性的……

尽可能地挤出一些钱,哪怕是很少的钱做投资性的安排。投资性的安排,我刚开始讲投资的时候我说过,可以是8分钱的邮票,也可以是8万、80万的股票,也可以是其他。投资性的选择,就是资本性收益在年轻的时候是初始阶段,到你老了的时候,资本性的收益应该超过劳动所得,或者是应该远远超过劳动所得。所以它里边有一个分布问题,就是再生产,提高生产力水平和获得、获取资本性收益。我想我在讲货币的时候,讲货币这里边又捎到时间问题了,因为它……

因为很多时候,它不是一个孤立现象。因为多数人对钱的理解,可能是一种刹那间的感觉。它包容了大概这么几个方面。第一,钱在大家的直觉里边,它是可以获取你们所需的东西的,就是实际上是再生产需求,只不过是这个再生产需求里边夹杂了太多的欲望、虚荣心和一些其他东西,会导致人在再生产的部分有的时候不够理性,或者是超出了正常需求。

特别是年轻人由于拥有足够的时间,还拥有可以浪费的金钱,所以对生产力工具不是特别介意。比如说花100块钱去听一个课和花100块钱去吃一餐饭,这个重要性可能不一定能厘得清楚。对提高生产力这件事情上面,很多人其实重视不够。不论是投入的金钱还是投入的时间远远不够。因为,以为小学、中学、大学已经完成了生产力的投入,这是错误的。

真正的生产力提高的投入是毕生的,一直到死的。因为你在人生的不同阶段要提升你不同的能力。这里边包含了无形的一种认识水平、认识高度,有形的、你实际拥有的技能。生产力水平的不断提升是决定你命运的重要方面。它和资本性收益构成了你的命运的两根支柱。两根重要支柱或者是两条腿,它决定你这一生是否幸福。

我们在归纳“钱”的时候,好多人“你说它是价值尺度,大家知道”,价值尺度,对啊!钱可以衡量一个商品的价值,钱甚至可以衡量一个资产的价值。但是,钱却无法衡量一个人的价值。我们在价值论里讨论了“衡量一个人的价值的问题”。衡量一个人的价值,不是左手这把尺子“做多大的官”, 右手那把尺子“赚了多少钱”,不是的。是一个人的社会性程度,社会化的程度或者他的社会性,或者他的公共性的水平。用宗教的语言,那就是他的“业”。

马克思的《资本论》当然不会去讨论人的价值,马克思只会去讨论劳动,因为马克思将劳动商品化了,只会讨论劳动的价值,讨论商品的价值。而劳动的价值和商品的价值都可以用钱来衡量。这里边其实涉及到很复杂的一种哲学和道德问题,就是我刚才讲:一个女生,她去美容和整容,或者是她拿这个去补习班,这是两个非常不同的选择。这两个选择里边,它构成了一个商品性、商品化的想法,一个是资本化的想法。我想你们明白我在说什么。一个是商品化的想法,一个是资本化的想法,是两个不同的路径,两个不同的结果。

所以在马克思的一般等价物、价值尺度、价值存储、交易媒介之后,我们还要把“钱”的意义或者是货币的意义继续往前延伸。除了四个以外,第五个,它是我们恢复再生产的手段,就是我们要活下去,还要继续重复劳动;第六个,提高生产力的手段;第七个,获取资本收益的手段。我不知道大家理解我在说什么了。

在处理钱的问题上,永远要构成三角形思维。就是我们今天这个课讲完了,我就希望你永远记住,今天的这门课就是你必须永远构成一个三角形思维。什么意思呢?三角,第一个角就是再生产。你作为劳动力,你要重复下去嘛,所以吃穿用你必须得有嘛,不然你活不到明天嘛。这是以一个你认为足够就行了,但不能过分的,这种其实它是一种消费,消费性支出就是再生产。第二个部分是生产力工具,就是提高生产力水平,扩大生产力工具。第三个部分是资本性收益。一个三角形。

我们在理解概念的时候,必须将这个概念具象化。三角形的意思就是,你一个月有一万块钱,再生产的部分是多少?提高生产力的部分是多少?资本性收益的部分是多少?你要想透了。我个人认为,越是年轻越要压缩消费,特别是压缩不必要的消费——那些出于虚荣心,出于某种错误的感觉的一些东西的消费——要尽可能地压缩,要提升、要极大地提升生产力工具。

并且在年轻的时候就应该开始进行资本性——就是投资安排,就是资本性收益的布局,而且这个布局,因为年轻,如果你没有获得遗产,没有赠予的话,你自己只是开始,这个开始的原因是少,但是它对你而言是一个重要的学习和经验积累过程。如果你有5年到10年的投资经验,你如果是从二十出头开始,等你到了三十岁、而立之年,你已经成熟了,这个时候你已经有足够的经验和能力来处理财富问题了。如果你持续二十年到四十岁,三十年到五十岁,应该是有足够的……

成功或者是过得不错,里边不完全是偶然,它是由一个人的认识水平决定的。这个认识水平不完全是个道德水平,因为我们今天讲的不是个道德问题,我们今天讲的是“钱”,我们讲的是货币。我今天的标题是“价值尺度——货币”。我之所以用这个标题,它只是衡量商品或资产的一个尺子,这是人们、普通人对货币的粗浅的理解,它只是个尺子。它真的只是个尺子吗?

它是生产力工具,它是资本性收益的源泉。如果你认为我是在讲一个个人,那么可能这样的思考就变得有局限性了。钱,当它转化为资本的时候,我们在思考一个国家的时候,也要有三角形这样的一个分析框架。就是,一个国家的国民财富应该是用在消费呢?还是用在生产力工具呢?还是用在提高资本性收益上面?这个三角如何获得某种平衡呢?如果失去平衡是什么呢?

比如说,有些国家在特定时期,为了使特定人群获得资本性收益,比如说获得资本性收益的主要方法是通过某种资产的溢价,比如说楼市或者是股市溢价,那么有可能向市场投放了过多的货币或者是提供了过多的流动性。并且形成了一种文化,鼓励人们在某些方向过度消费,比如说过度买楼、房地产消费、“六个钱包”。过度房地产消费导致拥有信贷资源的人,可以大量地持有土地资源或者是房产资源,从而获得巨额的资本性收益。

由于极度的扭曲,资本大量地进入到资产的炒作或者是资产的溢价过程中,从而影响了对生产力发展的投入。就是生产力工具在萎缩,就是实体经济在萎缩,而其他的在膨胀。这个时候这个国家就出问题了,这个跟我们个人的情况有的时候很像。就是我们研究货币、研究资本流转,它其实在个人和国家上面,道理大体上是一致的吧。我总说,有的时候我们的母亲就是最好的财政部部长,她不一定学过会计,她不一定是学过出纳,但是她会精打细算,使一个家庭的财政收支平衡。

我特别希望,我们在学习类似的研究货币的时候,我们认清它的本质的时候,我们不仅仅是获取知识,我们更重要的是获取某种的认识高度,从而提升某种的能力。所以我特别特别建议,就是当我们讲完价值论和——其实价值论还没讲完,这是刚刚讲到钱,因为它是价值尺度——当我们讲到钱的这个时候,你开始形成了一种本能的习惯,而且要将这个本能的习惯教给你们的孩子,就是三角形,这个三角模型,就是消费再生产、

生产力工具和资本性收益,这三个角,就是你的收入要进行布局。那么这个布局的比例是个什么关系?年轻的时候是个什么关系?到了你老了是个什么关系?退休以后是个什么关系?你现在做规划并不晚,不管你多大年纪做规划都并不晚,尤其是如果你是个年轻人的话,这个时候就非常非常重要了。我不是要求所有的女生都不要去美容或整容,因为你要对自己做一个定义,我主张增加生产力工具,我主张增加资本性收益,我不主张将自己商品化,可以将一部分劳动商品化。

其实钱这个问题上,你说完全不讲道德是不行的。那么我们在讲到钱的意义的时候,我们最后稍微回一下,回到道德角度来谈一下子。钱的来处是有道德的,钱的去处也是有道德的。如何获取钱或者获取财富它是有一个道德底线的,如何使用财富也是有道德上面的考量的。我再强调,我们所说的道德是社会化或者是社会性,讲的是这个。

就个人而言,我主张将那些不可以商品化和资本化的东西,一定不要将它商品化和资本化。无论是你的美丽、你的权利,还有你的等等等等,不要将轻易将不应商品化和资本化的东西进行商品化和资本化,获取超额的收益,这应作为人生的戒律、人生的准则、人生的底线。我们在处理好三角关系的,这个三角模型之后,应有美好的人生了。

钱的本质,我们今天,钱的本源、进化和钱的本质,我们简单说了一下子。钱发端于欠条,就是因为那个时候巾是最珍贵的。就是织物是最难得的。剪一个角给你,那就是欠条了,就是下回你来拿这个来取东西就行了。而且师,作为一个执币者,是最公道的,德高望重的人,能分钱的这个人,能打欠条这个人,是德高望重、有信用的人。中国这个文字真的是太牛了。所以当我们研究币、研究师的时候,老师的师,发现了很多有趣的事情。

在古代,钱作为那个币的钱其实就是信用。后来有了金属,金属货币就不需要信用,因为它本身它就有价值,后来到了纸币又变成信用。现在到了数字货币,它的信用程度就更高了,因为数字货币本身就更加的透明了。我们今天讲的它的本源、进化,我现在点一下题,说一下货币的本质到底是什么。货币一般而言,是一般等价物,是商品的一般等价物。马克思认为它是一种特殊的商品。

我在这里边多加两句话,这两句话作为学术是没有意义的,就是钱或者货币是我们的生产力工具,钱、货币是我们获取资本性收益的源泉。我希望大家能够记住,后边我附加的这些话。我今天在讲钱的意义或者钱的功能的时候,我们概述了马克思所说的四个,后来我又附加了三个。可能有人觉得你这样讲《资本论》是不是有一点儿超越马克思的本意了。不是超越马克思的本意,不是的。

马克思说的是对的,是我们必须将马克思的理论与我们的生活、与我们的生命相结合。如果我们不能将《资本论》用于自己实际的生活、工作和生活,那我们学来做什么呢?在讨论到钱的运用的时候,我希望大家能够反反复复地记住今天我们讲的三角模型,你的钱不是用来消费的,那是你三角模型里边的一个角,有的时候这个角要很小很小,要尽可能地放在生产力工具上面,提升你的生产力水平,尽可能地放在资本性收益上面。

最后我们今天想讲一下货币。货币,马克思对钱这个问题有着极度的厌恶。列宁对钱也有极度的厌恶,列宁准备用黄金做一个马桶。钱或者是货币会发生异化,这是《资本论》以外的马克思的论述。在异化理论里边就是谈到钱,钱会发生异化。钱本身不是毒药,但是钱可以使人发生异化,因为钱带有的特殊的功能会使人出现迷茫、迷失和错乱,事实上这种现象是广谱性的。

我们今天讲一下子货币的两化问题。他们说卢先生读《资本论》和讲《资本论》总是与众不同,我们将社会学里边的东西带入《资本论》吧。货币的异化这不是我提出来的,这是马克思本人提出来的,只不过是货币异化的具体分类,我做了两化分类,可能并不完整。一个是商品化,一个是资本化。那么商品化的意思就是当拥有了钱的时候,你的钱可以使别人的任何的东西商品化,你可以购买任何东西,甚至人格。

同理,别人的钱也可以将你身上的任何东西商品化,包括你的肾、你的人格、你的道德、你的劳动力,就是钱会使任何东西商品化是一个异化的过程,这个异化的过程有的时候是非常残酷的。第二个部分化是资本化,其实我们看到的腐败实际上就是一种权力资本化的过程,我们看到的好多东西都是资本化的过程。资本主义在资本化这个问题上,实际上是出现了问题的,就是过度资本化。

过度资本化会导致一系列问题和现象的发生,比如说教育,比如说医疗,比如说养老。当某些东西被过度、提供的东西被过度资本化之后,当然这是资本主义的一个特征了,但我们在讨论新时代有中国特色社会主义的时候,要考虑钱的两化问题,一个是商品化,一个是资本化;如果这两化的底线我们能划定,那么在社会治理,在这方面就会出现一个崭新的情况。但我今天要讲的两化不讲社会治理,因为刚刚被禁言,

就算是用语音来谈可能也会比较复杂,比较麻烦。对了,就是整理文字的朋友还是麻烦你们多加小心,这个平台无论如何得保住,就是我以后就展开禁言之后我也不会再转载或者发任何东西,我也尽可能地不再说话,尽可能地把这个平台保留下来。我寄望于听课的这些朋友们,你们自己无论如何要清晰划定你个人的两化的底线,什么东西可以商品化,什么东西绝对不能商品化,什么东西可以资本化,什么东西绝对不能资本化,在钱的问题上你想透,

在钱的问题上,你如果能够想透、想明白,我相信此生无忧。因为如果你懂了我前面讲的那个三角模型,你就知道钱应该怎样最有效率地使用;那么你如果再划定了两化的底线,这个两化不是数字化的那个,不是资产数据化信息产业化和产业信息化,不是那两化,是钱的商品化和钱的资本化,钱对人的异化的这两化。我不能讨论社会、讨论社会治理或者是国家治理,我们今天就讨论我们个人吧。就讨论我们自己,划定我们可以做的部分。

我其实知道这是非常难的,但是今天这个思考,今天这个三角形和两化的思考是极其重要的。我甚至知道大部分的朋友的年纪可能也不小了,你们的两化问题大体上应该是可以解决,没解决好的一定要解决好。因为在未来漫长的岁月里,你懂得可商品化的东西,你就不会犯错误,你就不会丧失你的理想、追求,会保留你的人格品德,不要过度的商品化,不要过度资本化,就是把你可以商品化、可以资本化的东西划出来,其他的底线、高压线是不能碰的。我说这个的目的更多的是希望你们以此来教育和约束自己的孩子。

好多朋友说不知道该怎样教育孩子,他如果还是七岁之前,那么就容易一些,他已经七岁了。那么我想今天这堂课三角形和两化划定就变得非常重要,因为它可以保证这个孩子的生活是有效率的,生命是有意义的,同时,可以确保他们不犯那些低级的、无意义的错误。不要认为有学问的人、有地位的人就可以不犯低级错误,很多人不会守人的商品化的底线、人的资本化的底线。我们生活中的案例具体的东西太多了。

今天,这个讲货币,讲钱这个事情就讲这么多。今天这个情绪被干扰了一下子,本来是要给大家发:“感谢整理资料的朋友”,结果发了半天发不出去,我还认为群发不行,个别发还是发不出去,后来重新查询才发现是禁声了,那么就是可能要到月底了,到二十几号可能才可以有机会解除禁声。所以呢,就月底之前就不发各种东西了,不是不发,是发不了。有什么事情呢?我尽可能地通过微信的群,说完以后,

有微信群的朋友转发给大家。我们腾出一点时间,念叨念叨降准的事情,总体上来看,控制得不错,就是我们对流动性的控制,控制得还是不错。但是下半年的形势其实是比较诡异的、比较复杂的。也就是说在大宗上涨之后,实际上企业需要的运营的资金量是非常大的。另外由于一些环节,主要是跟土地和房地产相关联的环节处在一种迟滞的状况,所以他那个M*V,费雪定律里边的那个流动性,

这个流动性出现了障碍,所以降准呢,算是定点、定向、定量地解决问题,这个事情做的是对的。但现在这种情形大家担心两件事情,就是降准会不会对房地产又产生某种形式的刺激,这个有待观察;另外一个,我们已经看到大宗的上涨,会不会构成大宗的继续的猛烈的上涨呢?这里边有两个因素要考虑,一个因素,不是我们降不降准,而是全球的货币投放量太大了。随着经济恢复,就是M是巨大的,以前是V,

以前是V趋近于零,现在V一旦动起来,加上既有的M的量,它就会变成一个巨大的数字。公式的这边是PQ。Q是商品总量,商品总量增加不了那么多,那么MV增加的量全部表达为P。我个人认为,通胀远未结束,这个P还要上,因为除非你把V停下来,M现在不能缩减,V在迅速地增加,那么这个产量上不去的时候就主要表达为P。那么石油价格其实我个人认为没有太多的意外,除非疫情再次猛烈地、这个再次回马枪。

不然的话,石油的价格我看100美元一定会破,其他的商品的价格也会纷纷的去寻找一个高位,就是我们要准备接受一个比较高的通货膨胀水平了。由于中国的存款利率的水平相比较而言,在全世界相比较而言,算比较高的,如果通胀的水平在5这个水平的话,那么我们还可以保持一个正常通胀的水平。如果超过6了,我们这个正通可能就被打破了,中国开始进入到高通模式,跟日本、韩国、台湾一样了。如果进入到高通的模式,我前两天说过正通、高通……

在正通、高通、恶通和死通这样的模式上有分别不同的投资策略。这一周港股出现了一次800点的大幅下跌,美股和欧洲股市受到影响,但美股昂然向上。从一板斧的角度来看,美股似还有最后一次机会。但从一板斧的角度来看,可能港股和我们这边可能在一个比较长的时间之内面临着一个调整的巨大的压力,大家要心中有数。

可能走到今天,好多朋友开始慢慢能够理解我让大家进入黄金屋的意义了。就是我们看到对海外上市的那些独角兽的清理,现在还没有收官,还没有最后结束,整个的过程还需要时间。滴滴打车的事情,我上次讲简单讲两句,今天再补两句。就是它不仅仅是个数据安全问题,主要是中国最优质的信息化产业全部在海外上市,这个事情本身它不正常,它意味着中国最优质的资产走了。其实在香港上市与在美上市差异不大。

这件事情我们正在深刻地检讨。而我们自己的股市第一是茅台,第十五粮液,中间夹着八个金融机构,与美国的市场的情况截然不同。因为我们的大家伙们都在海外,不在国内,我们有自己的微软、Facebook,有自己的谷歌,有自己的苹果,但我们所有的企业都不在我们自己的资本市场上做表达。优质资产在海外,每年要向海外投资者提供巨额的外汇储备,要把卖袜子、卖衬衫的钱凑齐了给他们分红。确实是问题巨大。好吧,今天也不能说了。

我个人认为有三件事值得我们高度关注,第一件事情老问题了,就是通胀。现在看这个样子,看这个样子,我们由正通转入高通的可能性板上钉钉了。就是我们下半年肯定是高通了,就是正通是不可能了,肯定是高通。现在美国已经是恶通,看来有可能全球很多的经济体都会进入恶通,下半年。我说过,进入恶通之后,最佳的选择不再是……,因为高通的选择是房地产,正通的选择是股票,恶通的选择应该是黄金的。

当然了,死通的选择就是逃命就得了,就不用选了。我们担心可能全球将以美国为代表的近一半的经济体会进入到恶通,所以呢,可能我看早了一年、半年到一年,但我们还是要坚持按既定方针办。因为我们不是看一看而已,我们有我们自己的分析框架和我们的逻辑,大家还是要沉住气吧。

今天课就讲这么多吧。今天给这个,一个月两次禁言,搞得我心情很不好。课就先讲这么多,然后整理一下子,然后我们把它发到其他群和明天下午三点钟我们再做沟通和交流。关于货币的问题呢,在资本流转的时候,我们会再花一个比较长的时间来做讨论。因为货币的转化就变成资本,资本流转的部分我们会花比较长的时间来做一个分析,那个时候跟投资、跟其他的内容就更为密切。好吧,今天就这样,感谢大家,明天见。

\section{劳动、Data、关联公司价格重置、北上深二手房价得到控制、说几句美国}

大家好,今天是2021年的7月24号,是辛丑年六月十五日,今天我们是《资本论》第八讲,谈劳动。在开始今天的课之前,首先我本人以及平台上所有的朋友要问候一下子郑州的朋友、河南的朋友,因为大家都非常非常关心你们,因为我们知道在郑州、在河南的朋友,有好多朋友是帮我们在一起整理音频、整理文字的一些朋友,大家非常关心你们,送去我们的问候,如有需要,我们尽全力做我们可以做的事情。

另外告诉大家一个好消息,从今天开始,我们就可以七个群同步播放这个课程了,真的好啊。然后,我一会儿我们正式开启课程的时候,我就先将七个群的聊天就全部关闭,然后我们到大概课程结束一个小时多、两个小时的时候,我们再将完整的录音提供给大家,然后可能需要明天才能提供文字。感谢在郑州和河南的辛勤工作为大家服务的朋友,我们也真希望,想为你们做一些有意义的工作或者是服务。好,我们三点钟准时见,今天的课会很精彩。

大家好,今天是2021年的7月24号,是辛丑年的六月十五。六月十五很重要,因为这是贲卦,山火贲卦的第三爻,就是六三爻。知道会有事情发生,但没想到这个事情是在郑州,其实心有戚戚然,很担心,真的有不少在郑州、在河南的好朋友,心里边总是在挂念着,虽然我们什么也做不了,但是还是牵挂。希望能帮上一点忙啊。

今天我们讲《资本论》的第八讲:劳动。我讲《资本论》肯定和所有人讲的都不一样。《资本论》立基是四个东西,第一个东西叫价值,我们讲了价值论;第二个部分叫货币,我们讲了货币;第三个部分叫劳动;第四个部分是商品。这是《资本论》的四块基石,这四块基石其实貌似普通,其实不同,因为在普通之中发掘出其深远的东西、深刻的东西才是真正的学问,是在朴实、朴素中看到。

今天我们除了讲劳动之外,还想花点时间讲一下子现下的资本市场的状况。今天在正式开课之前,我发了一个东西给大家,然后我写了一句话,“行到水穷处,坐看云起时”。就是实际上目前在跟数字相关联的这些公司,都在进行大幅度的、严重的减值。这意味着什么?我一会儿会做一个解读,同时我也想把美国经济的情况再跟大家做一点点的介绍,这对我们下半年部署我们的投资、做安排,可能有重要的意义吧。

大家——像马克思这样的大思想家,他做学问的基点,他做学问的脚踏的基石,其实是非常扎实的。其实真正的大家都是从最平凡、最朴素的概念入手的,他能挖掘出别人所看不到的东西。在这个看不到的这个东西里边,其实形成了对历史内在规律的深刻的把握。我想今天讲劳动,算是我们读《资本论》第一卷里边的一个小高潮。因为劳动这件事情太重要了,我们虽然都每天在劳动,但我们实际上我们对我们的劳动可能还缺乏足够的理解和认识。

劳动是什么呢?为什么马克思用劳动呢?他为什么不用工作呢?他为什么不用其他的单词呢?劳动和工作的区别是什么呢?我们要给劳动一个基本的定义。劳动一共可以有三个定义,一个是广义的劳动。广义的劳动是基于生产和生活的一切活动,这个就叫劳动。有些劳动是可以商品化的,可以卖钱的;有些劳动是不能卖钱的,比如说看孩子做饭。在家看孩子做饭,这是很重要的劳动,它也是创造价值的,但它不拿出去进行交易,所以劳动是有偿和无偿的,有区别的。广义的劳动包括有偿和无偿。

那么什么是狭义的劳动呢?也就是《资本论》里边要算账的这个部分的劳动,我把它定义为基于交易的工作或者是基于交换的工作。这个劳动是有偿的,它是带薪的。第三个劳动是什么呢?我们是做一个哲学概括,这个哲学概括也不是我概括的,是马克思概括的,但不是在《资本论》上概括的,它叫创造价值的行为。大家记着劳动,广义的劳动是基于生产和生活的一切活动;狭义的劳动是基于交易的工作;哲学上的概括、抽象是创造价值的行为,这都是劳动,是不同的。

那么劳动的本质是什么呢?它是获得生活资料所付出的脑力与体力之总和,它的本质是这样的。那么什么是劳动的意义呢?我们通常认为它是一种体现生命价值的创造,听上去很文学。这个部分对劳动的定义,大家听一听,先不用去认真思考,但是你记住我今天的概述,因为将来你们会知道我在今天在说什么,可能还需要时间。因为虽然我们都长大了,我们都在劳动,但我们对自己的劳动,对我们的劳动与资方的关系、与政府的关系、与社会的关系没有清醒的认识。由于没有清醒的认识,其实我们并不知道如何来表达。

甚至在很大程度上理解了劳动,才能理解人与人、人与社会、人与资方、人与政府的关系,甚至懂了劳动,才懂得怎样教育自己的子女。因为劳动作为一种狭义的劳动,作为一种商品的劳动、作为一种可交易品的劳动,其实是需要进行某种养成的,它包括了教育、学习、训练诸多因素。今天我们会详细地来解说一下子这个劳动。当然了,我们今天的解说已经超出了《资本论》狭义劳动的那个范畴,因为我们必须把劳动这件事情嚼碎了说清楚。

我们先做劳动的物理特征的分类。劳动分为体力劳动和脑力劳动。孔夫子说,“劳心者治人,劳力者治于人。”孔夫子很早就把劳动做了区隔:劳心者和劳力者,这是一种物理特性的分类。那么我们做一个一般性的分类。什么叫一般性的分类呢?我们把它归纳为“三高”。第一类劳动,是高尚的劳动。这是从它的道德属性上来判断。高尚的劳动不仅仅是造福于自己,也造福于他人,甚至造福于人类、造福于国家、造福于集体——高尚的劳动。

第二个“高”叫高端的劳动,也可以称为高阶的劳动。就是这个劳动,它是有门槛的,就是很多人做不了的事情。它创造价值的能力是比较高的,是经过严格训练的,是有大量的学习训练基础的,是需要有技能要求的,它叫高端劳动。记着,高尚劳动之后是高端劳动。第三个“高”是高效的劳动。什么叫高效劳动呢?就是在单位时间里,你能做10个纸盒,他能做20个纸盒,那个做20个纸盒的人就叫高效劳动。“三高”是想说什么呢?“三高”实际上意味着你对社会贡献的不同状况。

我先前置的说一下子,不是每个人都能提供高尚的劳动的。我不是说普通的劳动不高尚,因为有些的劳动它不能——仅仅是资方给予他、购买他的劳动力。伟大的思想家、伟大的作家,他的作品可能在出版商那里边获得的劳动补偿,就是它的劳动的补偿、它的交易的价格是很低的,比如马克思卖《资本论》很低的,但是因为它是高尚的劳动,它可以获得价外的补偿,就是荣誉。而且那种荣誉是只有这种道德属性的、高尚的劳动才可以获得远超于价值的价外补偿。

高端的劳动我们都知道,就是由于受过极好的训练,你已经进入到一个比较特殊的领域,就是普通人进不了的领域。所以你提供的劳动或者是你创造价值是处在一个非常独特的地位上面的,所以它可能获得常人所不可以企及的那样的一个劳动补偿或者是报酬,所以它叫高端。那么普通人,可能做高尚做不了,做高端做不了,那么就尽可能地提高自己的效率。比如说送外卖,有的人一天送30个,有的人能送300个,这就是高效。他抓住了里边的窍门,又很勤奋,工作时间较长,所以高效劳动也一样可以获得较高的补偿。

其实劳动力的定价问题,就是劳动的分类,我们分“三高”,”三高”里边意味着劳动力的定价,这个实际上是对父母的一种考验。因为小朋友天生下来,天真烂漫的时候,他哪知道”三高”的问题。他未来的“三高”,是高尚,还是高端,还是高效?他不懂啊。那个时候就需要一个孟母,就需要一个教育家来尽可能地将他带入“三高”,由此改变他的命运,使他有一个很好的生活。因为只有人类有劳动的交易和补偿问题,蜜蜂也在筑巢,建筑工人也在建房,但蜜蜂的劳动不做交易。

我快一点讲前面的,因为定义和分类不重要,今天我们的重要的点在劳动定价。劳动定价其实涉及到对劳动力的补偿的问题。那么它是一个劳动的定价,不能是劳动力的定价,就是,我们不拿人去买卖的,我们只是把他付出的那部分劳动——狭义的部分、狭义的劳动——work——工作的部分做一个定价。我们来理解这个社会对劳动的定价,就可以看到这个社会,社会治理的优越,同时也深刻地启发我们每个人重新认识自己在社会中的地位和劳动的价值。

一个社会发展比较传统的社会,在劳动定价方面,它就能看出社会的发展阶段。比如说,有的国家或者有的地区,它这个劳动定价,它涉及到血统问题,出身问题、血统问题。通常是中国封建社会是围绕着官本位的定价逻辑。这个定价逻辑,官本位的定价逻辑,它计算的是官本利得,那么它就是具有血统特征的、非常陈旧腐朽的定价逻辑。当然这个定价逻辑今天在中国的很多地区仍然具有有效性,所以有好多人去考公务员,或者通过各种渠道去到公务员团队。

进入近代和现代之后,其实定价逻辑更加遵守的是金本位定价逻辑,或者是叫资本利得的定价逻辑。它是根据市场交易的结果或者是市场供需结构的结果来形成的劳动定价。这个官本位,它是官场交易的一个定价逻辑;这个金本位,是市场交易的一个定价逻辑。由于我们是在出售我们的劳动,所以它是在市场上形成的。只不过这个市场可能在以前是一个不充分的市场,它就是一个官本位的市场,或者是官场交易的结果;而到了比较现代的市场,它就是一个资本定价的逻辑。

好,在定价逻辑说完以后,我们简单要说几句劳动力定价的构成,这一点非常重要。我们的劳动力的定价,这个劳动力的补偿,分三个方面的补偿。第一个方面的补偿,就是资方补偿。就是你是被雇佣者,叫劳资关系,资方给你的工资或者是其他的东西全部加起来,也包括公务员,公务员的雇佣方是政府,可能给你福利分房、加工资,所有的东西加起来,那个是你劳动力定价的主体。现代文明除了资方给你的劳动力补偿,还有政方、政府给你的劳动力补偿,就社会保障、义务教育。

政府给的补偿呢?现在比如中国9年制义务教育是政府给你的劳动力的补偿;比如说这个公费医疗、免费医疗,就是教育医疗;包括其他的一些补偿。那么这是政方,除了资方给劳动力补偿以外,政方也给劳动力补偿。随着一个国家的文明程度达到较高水平,政方的补偿就是社会保障可能达到一个比较高的水平。现在西方社会,政方补偿跟资方补偿的关系大概是5,1:5的这个关系,就是20\%是政府补偿。通常在资方、政方之外还有一个社会补偿。

社会补偿在西方社会就是表达为慈善机构对社会公益机构,它对劳动力有一个补偿吧,或者它是基于人权的一些东西吧,在中国它就不是这种慈善机构。中国的社会补偿跟中国的传统社会生态有关系,这种社会补偿有的时候不健康,比如说随份子、随礼;比如说家族之间的互相帮助;比如说村落之间的互相帮助;朋友之间的互相帮助。总之,在劳动力定价构成中,你记住它的结构是三块:第一是资方,第二是政方,第三方是社方。这个定价结构非常重要。重要到什么程度呢?重要到你可以写作的程度。

当我们看到不同国家的劳动力定价逻辑、劳动力补偿方法的时候,其实你已经知道文明的程度到何种水平。因为政方就是政府对劳动力的补偿,以广谱性的社会保障,实际上是涉及到基本人权或者是基本劳动力平等、平权方面的问题。如果你再看到社方,那么就是政方加社方,它构成了一个人类文明的眷顾或者是关怀。在这个基础上,资方的这个资助,它属于这种锦上添花,就是不再像过去只有资方一方来处理,才有《卖火柴的小姑娘》和《悲惨世界》。

劳动力定价的这个分类里边,我刚才说的这三个——资方、政方、社方是有形的补偿。就是劳动力定价还有一种是无形的补偿,叫荣誉。对高尚的劳动,它可能资方给予不足,或者是资方无法给予,比如像马克思写了《资本论》,他没收多少钱,资方没给他多少钱。虽然是伟大的劳动,这个劳动,但是它这个交易过程,它这个市场需求有问题。因为它的受益者是人民,人民又不是资方,这个问题非常严重。所以像马克思,像我们国家的钱学森,像好多提供了高尚劳动的人,需要无形的补偿,就是荣誉。

我想在劳动力的定价构成或者劳动力补偿结构里边,可能你们悟出来一些东西了。为什么我们在教育子女的时候必须在“三高”上面下功夫,要让他从事高尚的劳动,尽可能地进入高端,实在是进入不了高端,那么也要高效。在这样的时候,他不但可以完成自己劳动力的再生产,劳动力的补偿再生产过程,可能还给身边的人,家里边或者是更多的人做出有益的贡献。这个人,生命的价值就得到了比较充分的表达,比较充分的表达。

那么劳动力的定价会不会有问题呢?其实劳动力的定价一向存在严重问题。我们这两天看到这个小吴同学——吴亦凡,我们看到了中国的这些小鲜肉、这些演员,他们的劳动的定价就极度扭曲了。都美竹会去找吴亦凡,不管她是被骗的还是被喝醉的,总之她和吴亦凡有一段纠葛。那都美竹会跟在西藏军区上的英雄们有这样的交往和往来吗?不会的。她会爱上他们吗?会陪他们吗?

都美竹会崇拜吴亦凡,不会去崇拜西藏军区的英雄团长或者是英雄士兵,不会。这是都美竹的错误吗?是都美竹的问题吗?你懂了,劳动力定价的异化,它不仅仅是劳动力定价的逻辑问题,它涉及到集体审美,涉及到社会风化,涉及到整个国家的一种精神面貌、一种精神面貌。我国50年代年轻的女孩可能会去爱英雄和劳模。

但今天,都美竹们会去爱英雄或劳模吗?她们会去用心地体会和接受像马克思这样的人吗?思想家吗?更多的时候,她们会去爱吴亦凡,爱吴亦凡,爱这个小吴,倒不是小吴本人的错误,而是我们在特定的时代,我们使用了错误的劳动定价逻辑。这个劳动力、劳动定价异化的结果就是社会的畸形发展。

当某些特殊社会阶层或者特殊社会人士,他提供的劳动价格被极度扭曲的时候,实际上是体现了国家治理的失败,它是表达在税政方面出现了严重问题。其实在演员这个问题上,在很多制度建设较好的国家,甚至不算太好,比如说南韩没那么好,但他对这个事情也有非常明确的规定。为什么呢?因为你不从小吴的超级所得上面有超级的资本利得税做补偿,那么其他弱势人群的社会保障哪里来呢?

在劳动力定价的异化里边,我们要谈三件事情。第一件事情是法外补偿。法外补偿补偿的不是他的劳动,比如说贪污,它补偿的不是贪官的劳动,而是贪官站的那个位置,他僭越那个权力,是对他的位置的定价,而不是对他劳动的定价,我们管它叫法外补偿,这个寻租叫法外补偿。这个法外补偿是不合法的,是犯法的,是寻租的。但这个现象在尚未建立完整税政体系的国家是广谱性的现象,就是腐败,这是寻租。这个寻租叫法外补偿,这是非常糟糕的,这比小吴还糟糕呢。

还有一种补偿叫价外补偿。价外补偿就是工资给你了,但你还获取了一些补偿。比如说获得了爱情,比如说获得了美女的青睐,或者是获得了其他的一些东西,就是价外的补偿。价外补偿比法外补偿好一点,但并不是合适的、合理的或者是合情的。它不合适、不合理、不合情,但它可能没犯法。法外补偿,价外补偿,还有就是价内极度扭曲,说的就是小吴他们。在特定时段,一些特定的人由于在某个领域形成垄断,包括数字独角兽,包括这些演员,包括这些形成垄断。

从而形成了一种不合理的集约定价。这个集约定价实际上是把其他人的资产收益和劳动收益全部集中于一个人身上,形成了极度扭曲的劳动定价。由于税政结构不合理,没有那种对超级资本利得的或是超级利得的累进制的这种惩罚性的税收,所以就会出现小吴这样的情况,都美竹这样的情况,这个对社会治理是非常糟糕的。其实我们看劳动的时候,我们可以对一个地区、对一个国家的制度建设提出一个基本的判断。一会儿再往后讲的时候,

在讨论劳动价值论的时候,在劳动定价的这个过程中,我们要讨论一下劳动价值论。我们在讨论劳动价值论里边,我想建议大家,有空去看一篇我写的旧文章,那篇文章的题目叫做《君子不器》。君子不器的意思是:我们从劳动的价值看到生命的价值,看到人生的价值。我们当然希望我们的劳动,能获得一个好的价格,但我们更希望通过劳动的价值,来表达我们生命的价值,表达我们人生的价值。劳动价值论里边……

在劳动价值论里边,你知道,我最想提到的一个单词,两个字,叫自由。自由是什么意思呢?我有能力,可能我能出卖高尚的劳动和高端的劳动;我没能力,我尽可能地高效地劳动,我出卖我的劳动,仅此而已。我不出卖我的肉体,我更不出卖我的灵魂。所以,我们自己在劳动定价方面,有我们冷峻的思考。有我们对自己劳动的认识,对他人劳动的认识,我们在劳动这个上面,我们看到了清晰的道义感、责任感和牺牲。

我为什么要说自由?我要说的是一个人的劳动定价,基本上也能看出他的人格的定价。我们懂了劳动的定价,我们不会因那个位置做僭越而行寻租,我们不会去做贪污腐败的事情;我们不要求别人做价外的补偿,我们不会去占别人价外的便宜;我们不要法外补偿,我们不要价外补偿。即便是有一天,也可以形成某种特殊的优势、形成垄断,我们也不会像小吴他们这样,进行拿别人的资产和别人的劳动进行集合定价或者集约定价来形成自己独特的价格,我们不占别人的东西。

自由,就是懂得放弃,而不是索取;自由,就是不在自己的背上背负那些不该承受的包袱。有空读《君子不器》,其实我有几篇文章是希望大家能够读的,一个是《君子不器》,一个是《行者无状》。这两篇文章是我对自己的期许,也是对平台上朋友们的一种期许,也希望你们将这两篇文章给孩子们讲解,希望他们长大也能做到君子不器、行者无状。因为人生就应该是大自由的,我们不为一个定价所捆绑,我们不出卖肉体,不……

讲到劳动定价,可能有一点点的感慨,但是今天不会激动,讲《资本论》讲得激动,我们还是回到课程本身。其实,个人对个人劳动的价值,每个人对每个人自身的劳动的价值,应该有一个基本的评价。至于这个你的劳动卖出一个什么样的价格,你是否满意那是第二件事,那是另外一件事情。但我想多数人对自己的劳动的价值,心中无数。我们是在漂泊状态,就是我们正好遭遇到一个单位,或者正好遭遇到一个定价,我们就形成了。

关于个人劳动定价的思考,我希望你们能想三件事儿。第一个是你能否给你的劳动有一个基本的定价,你觉得你干的活值多少钱?第一件事情;不是每个人都会这样思考问题的,那么你同时看一下,现在这个价格是高了还是低了,这是一件事情要思考。第二件事情思考,是你要思考一下边际。每一个人的劳动,都有最低的边际和最高的边际,如果达到了,最低的边际都达不到,你是否应该考虑转移了?就是解除劳资关系。

如果你现在获得的价格,远超你的劳动的价值,就是上面的边际突破了,你想过原因吗?这是你要要的吗?请记住第一件事情,我希望你对自己的劳动的价值有一个评价。第二个,你对你劳动价格的上下的边际,要有一个基本的思考。这个思考涉及到法律、道德等等,就是涉及到道德和法律诸多的问题,还有个人的公平正义的问题。我觉得这是一个重要的思考,每个人都需要思考。第三件事,就是关于个人劳动定价的第三件事。

这第三件事情就是非常重要,我不知道你今年多大,你个人认为,你还有机会提升你提供的劳动的价值吗?你有两件事情可以做,第一件事情,你通过充电、通过学习、通过某种努力提升你的劳动的价值,第一件事;第二件事情,如果你的劳动的价值背离了价格,你有什么方法来让价格更充分地表达你的价值?三件事情,关于个人劳动定价的思考。这三件事情,我只提示你们大家,剩下的事情,你们要自己去思考了。

当然,作为一个学者,我更多的是在考虑国家对劳动定价的存在的问题。社会保障就是国家与劳动者的关系,是一种补偿;是国家通过国家机器,强制性地从资本家那里拿回来一部分,强制性地对弱势的劳动者进行劳动价格补偿、劳动价值的补偿。这件事情是文明的标配,是伟大的社会主义运动的结果,是马克思给人类最伟大的贡献。

如果你学《资本论》,那么我要问你社保、社会保障体系是怎样建立起来的?那么就是从《资本论》诞生之后,风起云涌的社会主义运动、风起云涌的工人运动逐渐地要求政府对劳动者的劳动进行另外一种的非劳资关系的补偿,就是劳政关系的补偿,就形成了这种社会保障。我们今天在这个方面还有很多很多的事情可以做。当你研究《资本论》,研究劳动,研究透了的时候,你必然成为一个作家,因为你看到了那些东西,当你升起悲悯和同情的时候,你肯定要做一些事情。

国家对劳动的补偿是人基本的权利,它涉及人的尊严与人的自由,这是极为重要的。这件事情首创者你们知道是谁吗?是可能是最早读马克思著作的那个德国宰相俾斯麦。并不是类似于俾斯麦的人就会比社会主义者更邪恶,而那些标贴了社会主义的人,有的时候还做不到对人的尊严和自由的尊重,这种尊重必须通过制度化的安排来做表达。

那么站在国家的角度思考,不仅仅是对劳动力补偿,补偿多少合适呢?我们注意到欧美出现的劳动力补偿悖论,什么意思呢?就是随着工会的越来越强大,劳方的力量越来越强大,劳方压迫资方、政方提供对劳动力的过度补偿,导致生产效率的下降,超越了生产效率的边际约束,导致产业外流,实体经济没有了、制造业没了。也就是说真理既不能向下走,也不能向上走,你懂得。

一个国家治理如果能够正心以中,在中观的这个角度达到最优解,该是多么的难呢。当工人、当劳动者争取自己的权益达到一个相当的水平之后,资方和政方变成了弱势群体,它导致的是工人整体利益最后受损。其实我们在欧美看到的这个事情让人感慨万千,唏嘘不已。既要做到社会保障,又不能突破生产力边界,要达到最优解,这是一个极为困难的状态。这是我们站在国家对劳动定价角度的思考,正心以中真的非常……

我们简单地说一下子,因为我一再强调这个算法高于看法。另外,明年我们将在香港组建“新马会”,“新马会”不是进行一般性的理论教条的整理,我们也要建立起一整套的“新马会”的一个逻辑体系。我们如何来理解劳资关系、劳政关系、劳社关系呢?比如说,我们要建立起一种计算方法。比如说在计算劳资关系的时候,我们要将总工资与资本利得建立起一个简单的计算公式,比如叫劳资比,就是总工资及资本利得之比。

比如说劳政比,总工资与国家财政总支出的比例关系。通过劳资比、劳政比、劳社比全球性比较,我们可以对一个国家的治理水平和文明程度做出某种评价。他既不能劳资比太低,也不能劳资比过高。既不能劳政比太低,也不能劳……就是那个最优区间在哪里,我们将来会形成一个动态的图形比较。就我们的研究到最后可以进行对政治制度的一种基本的基于算法的评价,基于算法的评价。当然也有朋友建议我对个人的这个劳动的价值有一个评价的体系。

这个事情我想过,就是就研究完《资本论》,就《资本论》的四块基石——价值、货币、劳动和商品这四块基石,我们今天是讲的是劳动,下一堂课是商品。这四块基石讲完了以后,我把它们全部串起来,你们就已经看到了《资本论》,就是它们构成资本的那个论的基石。把它们串起来这个关系呢,那个马克思要讲的资本的那个意思,那个价值论的体系就呼之欲出了。我们用拆解和粉碎,然后再重新聚合的方法来读这本书,就变得生动而有趣了。在处理复杂关系的时候,我们可能要超越一下子马克思。

不是我们比马克思看得更深或者是更远,而是我们与马克思所处的时代完全不同了。在马克思是自由资本主义的初级阶段,它还没发展到被马克思影响了之后的中级阶段,就是现在这个阶段。因为对马克思深刻影响的现代资本主义这种劳政关系出现,这是马克思始料未及的。你们懂得,马克思为什么要创造剩余价值的理论?马克思对剩余价值的测算意义在哪里?马克思为什么最后说“全世界无产者联合起来”。联合起来做什么?是推翻资产阶级吗?是把劳资关系的那个资方干掉吗?不是啊,是把政方干掉。

马克思要解决的是劳政关系,而不是单纯的劳资关系。所以要无产阶级专政。他认为要通过彻底地改变生产资料的占有方式,通过对政府的彻底改造来获得劳动者或者是无产者的自由。今天我们在讲劳动的时候,你想到了什么呢?你是否想到了生产资料公有制的情况下,未必能达成劳动者的绝对自由,对吗?在国家资本主义形态下,劳动者的自由依旧是受到了很大的限制,劳动者直接持有全部生产资料,那种情况留待《资本论》第四卷讨论。

谈完劳动我们很快就要讨论商品。在这个商品的时候,在劳动与商品的关系里边就出现了剩余价值的问题了。关于剩余价值的定义、测算我们放到商品之后去讲。讲完商品之后,我们就注意到马克思《资本论》的动机、目的和方法都出来了。我们今天简单对马克思当年的思考做一点点的分析。我个人认为在十九世纪的后半叶就是一八六几年的时候……

马克思通过这个对要素的判断和要素关系的整合得出的基本的结论,在他那个时候具有某种合理性。因为马克思当时并没有来思考资产阶级政权可以通过工人运动的方法,而不是革命的方法来得到某种的改变,甚至在某种程度上实现人民立法来实现社会保障。从而达至劳动价格的政方补偿,就是劳资关系、劳政关系、劳社关系它的政方的补偿。

他也没有特别地清晰地能够预感,或者是预设,或者是设想,他的理论在东线以革命的方式,以革命的方式高歌猛进,从德国、俄国到中国,席卷近半个地球的社会主义革命高歌猛进,竟然形成了国家资本主义的那种体系,并非他本人的内心深处的社会主义。而在资本主义的西线,英、美、法、德西线进行了大规模的社会主义改造,形成了今天这样的劳资关系、劳政关系,劳……

所以我们在学《资本论》的时候是非常有趣的,就是马克思没想到,叫我们给碰到了。我们读《资本论》的目的,除了我们想通过《资本论》来认清我们自己,认清当下的现实,也认清国家,也通过我们的努力来寻一条路,寻一条未来的路。我们管它叫“新社会主义”或者叫“新社会主义论”。我说过马克思写了第一卷《价值论》,《资本论》第二卷《资本流转》,第三卷是列宁写的《国家与革命》,那么第四卷就应该是《新社会主义论》了。那么我们在《新社会主义论》里边会重新谈劳动。

时间过得飞快,今天的关于劳动的部分我就不展开,我就说这么多吧。涉及到有一些其他的事情,本来是这个关于这个劳资、劳政的这个关系可以再深一点,但我想其实点到了也就可以了。我们留一点时间,因为最近这个世界发展变化快,已经到了七月份了,在农历是今天是六月十五。这个关键的日子,六三爻,这是一个非常关键的时刻,我们还是要对一些事情做一点点的分析和判断。这个分析判断未必就是精准的,只提供大家一个思考的角度。

有两件事情大家注意:一件事情是基本上跟Data相关联的那些个公司,大部分的公司出现了价值的重置、价格的重置。好多,我今天给你们表格,你们看到了,好多只剩下10\%了,腰斩就算客气了,一直斩到脚趾头。为什么?大部分的与之相关联的公司都在海外上市和准备在海外上市,在国内上市的也有,这个数字公司,这些公司实际上在这些公司里边凝结了高净值。

这样说吧,高净值的财富集中在两块,一块就是这类公司里边,新型的公司里边,他们的这些新型公司的估值存在着某种意义上的扭曲。但是这个公司是因为他们很多在海外上市,是可以完成海外套现的。这个估值本身是海外套现的,而他们在国内的营收是通过某种权力的垄断来形成的。目前的清理,我要点个赞,第一条。当然了这里边好多朋友又会说,你当年不是鼓励我们买两化,那我不是鼓励你又躲进黄金屋了吗?这个两化里边也分着,第一化信息产业化,今天也是好的呀,但是产业信息化里边出了好多问题。

我们让大家躲进黄金屋其实是有考虑的,因为中美在掰手腕。高净值他们的财富一部分在独角兽里边,在数字化公司里边;另外一部分在北上深的房产里边。连广可能都不算,主要是北上深,其中直接境外人持有的中国房地产的部分,就是他们已经成为外国人了,就是已经是英国人、美国人或者是香港人持有的房产大概一半在深圳,就是深圳可能有50\%以上的房产是由境外人士持有的。最近深圳的房价二手房下来,北、上也在调,但没深圳那么狠。

我们注意到,我们注意到中央知道中美斗争到了什么程度,所以在掰手腕的过程中,对高净值以及高净值那八十万亿浮财采取了一些措施。这个也要点赞。我对数字类型公司的减值,对深圳的房价,二手房房价得到有效的控制要点赞。点赞的意思是将这一部分浮财锁定,不构成新的资产泡沫。为即将到来的全球的金融风暴预做准备,这是非常明智的、非常聪明的。要点赞,就是我们的现在的治理水平很高。

这两件事情我们看到了一种趋势、一种苗头,它可能跟2022年就是二十大可能要推出的直接税有某种关联,可能它是前期的一种稳定性的措施。所以我们感到有些鼓舞、看到一些希望,感到有些鼓舞,对未来还是有很深的期待吧。当然事物的发展可能也未必能尽如我们的愿望,它里边还会有很多的波折,其中对一些反弹我们也看到,对攻击我们也看到了。

这是国内发生的事情,可能好多朋友已经敏锐地注意到了这类公司的减值以及最近二手楼交易的这个落寞。其实二手楼交易量的少、量少价跌,它是一个正在形成的一种状况,或者是可能形成一个较长期的趋势,对人民币的升值也形成了某种意义的暗示。我们在观察一个大时代的时候,可能不能从本位主义出发,也不能简单地从个人的职业视角和个人的利益的这种出发,应在一个更宏观的角度来观察。

好,说几句美国。美国的股市疯到什么程度,你们都看到了。我记得我说过它的合理价格,它现在可能已经超过合理价格的三分之一了吧,已经远超它的均值了,远超合理价格了,还在上。另外一个就是美国的房地产起来了,而且起的速度太快。这个我也给大家资料了,你们也看到了。我给了你们美国房地产资料,股市的资料就不用给了,你们自己翻一下就知道了,你们知道发生了什么事情。由于美国流动性泛滥,所以他通货膨胀是通过资产价格来表达,商品价格也在表达——五点七,但主要是资产价格表达。

我已经教给大家进行实质负利率的折算,我们上次折算的结果是大概是-12\%。我们说过四个阶段:正通,1到3;高通,4到9;恶通,10到20;21到99,死通。美国到12已经是恶通了,而且看这个状况没有回头的意思,没有回头的意思。看这个样子,如不能在年底之前找到下家,找到一个背锅侠,就是1985年的广场协议,1991年日本的破灭,以至于日本用自己的全部财产为美元背书。

如果他不能找到背锅侠,那么美国将在不远的将来步入一场深刻的经济危机。这个背锅侠能不能找到呢?他们已经到天津了。能谈成什么程度?不知道。因为美国人又说了,是基于实力。实力这两个词后边是意味着“三美”。美军那是实力,但是美军真的不行了。一个并不具有太空优势和电磁空间优势的美军,就算是有一百条航空母舰又能如何呢?所以,美军已然不行了。

昨天,在与香港的一个老报人聊天的时候,我还在重复凯恩斯的那句话——凯恩斯在一次世界大战结束之后,在英国下议院的演讲,他说:“正是丘吉尔先生的无敌舰队摧毁了大英帝国”。丘吉尔在二战是首相,一战是海军大臣,他认为是丘吉尔先生的无敌舰队摧毁了大英帝国。那么美国的无敌舰队,是不是有点儿像当年的英国的无敌舰队呢?他未必会摧毁他的对手,但他确实可以摧毁他自己的祖国。我们在这个事情上不能犯糊涂,我们必须建立我们自己强大的军事力量。

但是,我们要深刻地认识自己的边界,边际在哪里?我们不做过头的事,“正心以中”嘛,我们不做过头的事。对于目前有好多人在鼓噪的什么下饺子,什么收回南海,什么收回台湾,什么收回钓鱼岛,跟谁谁谁谁怎么一战,我们不这样的思考,我们也不去鼓噪这样的舆论。因为,我甚至在很多时候认为讲这些话的人,可能是想将中国引入歧途的人。请记住凯恩斯的话:是丘吉尔先生的无敌舰队摧毁了大英帝国。我要说的是:基于实力的话,美军排除了。

那么代表美国实力的第二个东西就是美元。美元依旧是强大的,因为它还是全球结算的货币,主要的结算货币。虽然在储备地位上开始下降,在结算上面好像也被欧元超越了,在储备地位上也在下降,但美元依旧是强大的。但这个强大的基础是有一些重要的国家,未纳入美元指数的大型国家为其美元的价值进行价格的背书。因为,我们真的没有买一克黄金。好多朋友说,是不是因为你们买了黄金,让政府买黄金,错!

值得高兴的是:我们终于听到了好消息,上海正在建立储备库。因为你既然是有黄金交易所,就应该有黄金储备库。据说这个储备库的规模是可观的。中国人等这一天等很久了。它不是一个经济学的问题,它甚至不是一个简单的学术问题,它是一种爱、一种责任、一种担当,对祖国、对人民的责任和担当。太难了,太难了。虽说有些晚,但还可以吧。

美元依旧算是美国可以用来说话的实力的支撑,但它已经不是像十年前那样强而有力了。它并不能通过美元对中国经济形成毁灭性的打击,它只能构成某种压制,而不能形成毁灭性的打击了。那么构成美国实力的“三美”,最强有力的就是美媒,媒体的媒。目前美媒真的很厉害啊!无所不在,无所不用其极。你们可能未必感受得到,因为我在媒体这个角度我能感受到那深沉的压力。

唯一让人感到欣慰的是:2015年之后我们进入数字经济时代,我们终于开启了自媒体的模式。中国本土的、草根的思想家在茁壮成长,而且他们在很多领域里边不仅仅可以说话、发出声音,而且在相当程度上可以形成全民共识,这个在某种程度上形成了对美媒的一种强有力的抵抗。请记住:是民间对美媒强有力的抵抗。这个抵抗依旧是孱弱的,有的时候甚至是无效的。

不过呢,我还是觉得不用那么悲观。美军不行了,美元有点还算强,但不像以前那么强了,但是美媒真的是太强大了,实在是无孔不入。你如果生活在香港,你有的时候觉得、简直是觉得没办法说。比如说,你打开电视,他给你一个广告,是一个小女孩大声地呼唤:我想去神户。它其实是个广告,你知道,这就是我们生活的非常现实的一个过程中,她不想去中国,她不想去中国任何一个美好的地方,她准备去神户。

对是非的判别,对价值的评估,全部是错乱的。不仅仅是香港,我们看我们国内的教育、学术、传媒、媒体,他们怎么来判断劳动的价值?什么样的劳动是值得尊重的?我们怎样尊重那些高尚的劳动?我们会给他们做五层的封爵吗?会给他们铜像吗?会给他们荣耀吗?我们给谁荣耀了?我们为什么非要弄小鲜肉呢?一个伟大的民族如果在这个上面的力量始终是孱弱的,那怎么能行呢?

好,又走了,今天谈劳动谈得有点激动,又走了。好吧,我们把美国的事情说完。美国现在最大的问题是美元价值得不到第三方背书。现在它要求的是一种毁灭式的、毁灭性的背书,就是麻烦像中国这样的国家提供足够量的资产和商品来完成它转折时期的背书,跟1985年到1991年的情况是一致的。我相信,我们在这个问题上不会屈服的。如果我们不屈服会怎样呢?美军和美元都会出问题,美媒解决不了现实的生活问题。

我要说的是,在美国资产四矩阵里边,第一个出事的不是美股,不是美房,是美债。美债里边率先出事的并不是美国国债,而是美国的机构的垃圾债券。虽然美国用强硬的手法已经做过两次对美债的救助,但由于美国企业并未能如其预计的那样完全的恢复,所以美债问题在日益严重,美债会出现一次跳脱的局面。美债一旦出大问题,美股、美房都兜不住。

其实我们都在思考美债出问题的时间到底是能拖到什么时间?第二个,它的传导方式会出现何等烈度的危机?会出现2008年那样?2000年那样?还是1929年那样的危机?是一种什么样的危机?就是时间、危机的模式。第三件事情就是危机对中国的影响,对香港的影响、对中国大陆影响,负面的影响有多深?正面的意义是什么?我们的考虑和选择是什么?我们应该持有什么类型的资产来躲过这样一场剧烈的波动?

谈到今天又苦口婆心一大堆,还是想说按既定方针办——短股长金,好吗?好多人认为我执拗、保守、执着,也许是我看到的东西和大家看到的东西不完全一致吧,我觉得我不保守的,当年买两化的时候我们还是比较冲锋的。现在我陷入保守,其实是可能我对危险的敏锐度偏高了。除了疫情的敏锐度以外,对美国现在的目前的治理结构,我是真的是非常的不乐观的。我甚至认为就是中国有意图去挽救或者拯救美国,他都扳不过来,所以形成我对他的一种看法。

但是今天我也要说清楚,我们对一个事物的看法只能是一种预计,对一个大趋势的概述,我们不是算命先生。即便是我们看这个山火贲卦,推它这个六爻,它也并不代表一个时间顺序,因为太多的因素互相扰动,有些事情的发生往往可能不是像我们预定的那样的走,有可能会快一点,有可能会慢一点,时间上面无法推定得那么准确。另外,它到底是一个什么形态的危机的爆发?那个时候美国可以采用的对冲危机的方法是什么?其实我们今天也无法给出完整的评估。

此外,实际上中国经济也面临着巨大的压力,在下半年面临巨大的压力,是大家所不知道的或没看到的巨大的压力,有诸多的问题也需要抓紧时间,赶紧做出解决和处理。并且,我们也在思考如何来应对可能率先在美国爆发的这场危机,因为它这个垃圾债我们怎么看它都覆盖不了了。美国政府如果财政部或者是美联储用什么方式直接下场来扛呢?因为我们注意到我们处理海南航空的方法,我们注意到我们处理资产管理公司的方法。

我们在想,美国人也可能会采用我国的方式、方法。就是我国在处理这个海南航空和华融的时候,使用了一种特殊的方式、方法,就是属于进行国家救助的这种方法。美国将来在出现问题的时候会采用类似方法和手段吗?它会用什么样的手段和方法救助?能覆盖吗?能兜得住吗?还是有待观察的。当然我国现在不光是华融啊,其他的涉房地产的资产管理顾问公司可能都存在着一些的麻烦。还有一些机构,除了数字类型的机构,还有一些机构可能都存在类似的麻烦。

最后,可能即便是没有广场协议,美元和人民币的比价关系,就是在汇率市场上都有可能在今年下半年会出现剧烈的波动。我们已经在上半年领教了商品价格的剧烈波动,下半年商品价格可能会趋于平稳,将会在汇率市场,特别是在金银方面,在一些要素方面会出现一个局面。我也不敢说时间和幅度,因为好多人要根据我的话来做期货的话,那可是真是个问题喽,而且我已经有经验了,就是别说太近,说太近好多朋友可能就是会非常的……

好吧,今天就聊这么多。记住今天劳动的表述,今天的课重要,劳动的部分多听两遍吧。再次感谢整理音频、整理文字的朋友们,还是惦记着郑州和河南的朋友们,愿你们一切安好!也希望其他地方的朋友多注意安全!在这样一个特殊的时期,多注意安全!既要注意疫情,也要注意自然灾害。我们明天下午三点钟见。好,今天就说这么多。

\section{商品、目前的局面}

大家好,今天是8月7号,辛丑年的六月二十九日。今天很神奇,今天是立秋,由于既是8月7号,又是六月二十九日,单日、立秋,所以他们管今天的立秋叫“公秋”,所以今年不会太热,立秋之后会慢慢凉爽起来。今天我们是《资本论》第九讲,今天这课关键,今天是《资本论》第九讲——商品,最核心的一堂课了。我试一下麦,然后三点钟我们准时开始。一会儿见。

今天是2021年的8月7号,农历辛丑年六月二十九日,也就是说农历的上半年终于到了尾声,这个辛丑年的下半年很快即将展开,很有意思。在六月二十九号立了秋,终于立秋了,按照算法这是一个公秋,就是会立秋之后会慢慢变得凉爽起来。不过辛丑年的这个天象总是如此,所以可能总有不断的这个让人震惊吧。

事情很多,我们就先不进入纷繁复杂的事情,我们还是进入学问吧。不管那么多,一心只读圣贤书。我们今天讲《资本论》第九讲——商品。昨天文木先生发帖子,他在讨论大政治家的做派,就是大政治家应该是下围棋,不应该下跳棋。跳棋只能是向前,偶尔躲闪,但毕竟要向前,不管前面是什么。倒也提醒了我,其实学问也是围棋,投资也是围棋,我们只能下围棋。

我想说的是在讲《资本论》的时候,我们《资本论》奠基的四个要素,我们讲这四个要素我是倒着来的,先从价值入手,然后货币,然后劳动,然后今天讲商品。要知道马克思的《资本论》第一篇就是讨论商品的,马克思的第一篇,《资本论》第一篇三个章节,第一个章节就是商品,第二个章节是商品交换过程,第三个章节是货币与商品流通。但我们把商品放在了最后,放到了第九讲来开始进入商品,因为商品太关键了。

今天我们讲完商品化之后,其实自然而然地进入到资本化的过程,其实就进入到《资本论》第一卷的核心了。越是简单的东西越复杂,越是貌似简单的概念,人们的理解越不容易深入,其实我们讲心学讲过同样的道理。就是商品这个概念我们从小就接触,接触甚至几十年,但你懂什么是商品吗?你懂什么是商品化吗?你懂什么是商品定价吗?你知道商品定价的逻辑吗?这一系列的东西下来,其实就看到了马克思了不起的地方,他抓住了要点,层层剖析。

在这儿,我先离开今日之主题,飘一下。我个人的理解,其实我已经不是第一次跟大家说了,我个人的理解经济学其实是三大块,作为一个投资者的经济学一共是三大块:第一块是创造价值的理论。这个我们讲过很多次了,我们在价值论里边反复在讲创造价值的理论,而且我觉得创造价值的理论它应该是人生观、世界观。因为人不懂得创造价值,人是没有意义的,人的生存或者是生活是没有意义的,所以经济学的第一块理论是创造价值的理论,当然这个创造价值的理论包含的内容非常丰富。

第二块理论就是非常重要的理论,就是商品定价的理论。商品定价的理论其实是经济学里最核心的部分了,也是马克思《资本论》开篇就进入的领域——商品定价。商品定价的理论其实是非常复杂的,你如果懂了商品定价的理论,其实投资的技术中的一半的内容就过了。经济学理论的第三个部分才是价值投资的理论。价值投资的理论才是我们最终要摘下的那一颗皇冠上的珠子,皇冠上的那颗明珠。这个其中整个体系里边其实商品定价,或者是叫资产定价、或者叫资本定价的理论,

这是最核心的、也是最难的部分。今天我们讲商品分成四个部分,我原来是想可能一堂课做不完,因为商品太重要。我们争取把它讲完吧,因为我想着已经第九讲了,一共24讲,这个可能节奏慢了。第一个部分是商品的定义;第二个部分是商品化;第三个部分是商品定价;第四个部分是如何理解商品定价的理论,就是商品定价它最后发展成一个完整的理论体系。商品定价的理论体系大体上是我们小到投资者大到治理国家都需要非常非常明白的一个东西,因为商品包含了一切商品,包含了什么劳动力定价、要素定价等等等等等等。

在所有商品定价的过程中,它有一个道德的属性。就是比如说爱情怎么定价?爱情有没有价格?比如说……因为是一切事物,所以它有个定价的,这里边有道德的问题。另外一个就是价格的合理性问题,就是价格按马克思的说法是凝结的人类的一般劳动,但大部分的商品定价远远脱离了劳动本身,它既有稀缺性的问题,也有资本过剩的问题,多种因素影响到商品定价。那么如何来看待商品定价的合理性问题,一个道德的问题、一个合理性的问题?另外就是……

我们在不合理定价之中找到一些重要的发现。马克思发现了什么呢?马克思在不合理定价中发现了剩余价值,你明白《资本论》是怎么来的。就在劳动这个商品的不合理定价之中,马克思发现了剥削的秘密,所以《商品》这章非常重要。马克思发现的秘密我们也要发现,但我们发现的秘密不仅仅是剩余价值的秘密,我们也要发现其他的一些秘密。其他的秘密发现了之后,我们不仅要解决的是投资的问题,而且要对制度进行批判的问题,而且我们可能也要重新对人生、对诸多问题有一个崭新的判断和认识。其实这一点是非常重要的,因为我们在大多数时候是茫然的。

好,我们抓紧时间。今天的第一个部分是商品的定义。为什么还要谈商品定义?好像每一个人都知道商品的本质或者定义,我要说的原因是因为我个人认为马克思和恩格斯对商品的定义,算是一个狭义的商品的定义。我给商品的定义是:用于交易的一切事物。所有的东西皆可成为商品。从哲学上看,商品是人类的属性,是文明行为。就是动物是没有商品交换的,良好的商品交易可以代替战争,所以它是文明的行为。

商品是价值、使用价值和交换价值的统一体,这句话也非常重要,它是价值、使用价值和交换价值的统一体。每一个人眼里看到的商品,对它的评价是基于自己的不同需求,形成了对它的价值、使用价值和交换价值的一种不同的看法,但它是它们三个的统一体。商品,是一种针对人类需求的供给品,商品生产是这种供给的生产行为。大体上我们将商品的定义,从一般意义到哲学定义、到它的分类,大概是这样的一个基本的情形。

在这里边,马克思讨论商品的时候给了一句很重的话,马克思说:“资本主义生产方式占统治地位的社会财富,表现为庞大的商品堆积。”它不仅仅是一种物的堆积,也逐渐形成资本主义社会的一种审美,这种审美畸形发展就是商品拜物教。由于我们通常所理解商品的价值往往是交换价值,我们并不能精算每一个商品凝结的劳动,我们也不一定非常清楚每一个商品的使用价值,比如爱情。

但我们更多地接触到的是交换价值。我们知道一个事、一个物,这些事物,一切,甚至权力,它怎么交换的,它大概什么价格完成交换的。这个交换价值是我们直觉上的商品的价值。但实际上商品的价值、使用价值和交换价值完全不是一回事。当出现严重背离的时候,可能是某种稀缺性,也有可能是某种垄断,甚至有可能出现了暴利、出现了其他的因素,这个时候我们要考虑其他的问题就是制度性的问题、制度设计上的问题。所以在商品的定价过程中,我们看到了诸多诸多的问题。

在商品定义这一块儿,我就不讲那么多,因为其实大家有空会读到《资本论》的第一卷第一篇第一章,其实马克思在这里边花了功夫的。我现在进入到第二个环节,就是商品化。商品化的意思就是,一样东西一旦用于交易,它就被商品化了。其实在现代社会或者是在现代资本主义社会,我们现在很多朋友说很难定义中国现在的社会属性,好吧,在现代社会市场经济的情况下,大部分的事物其实是被商品化了的,本不该商品化的很多东西也被商品化了。

我举个最简单的例子,就是我们现在所说的“新三座大山”,就是居住、教育和医疗,“新三座大山”其中很多的部分是不应该被商品化的,因为它属于公共品,而非商品。那么有些部分被商品化了之后,造成了巨大的问题。我们现在正在改这个问题,就是将不应商品化的那部分公用品退出商品化,使它还原来本来面目。就是我们必须有相同的医疗——公费医疗,相同的平等的教育机会——平等的教育机会,和这个相对满足的,不是绝对满足的居住需求。

商品化是有底线的。我们给商品化一个定义:商品化就是将事物用于交换的行为。这里边在商品化之前有一个非常重要的哲学命题,就是主权。你拥有一个事物,你才能将这个事物拿去进行交易。这个交易里边有很大程度上未必属于合理的交易,因为有些主权你是不拥有的,你只是一个看护者,你僭越了主权进行交易的行为就是非常糟糕的一个行为,比如说基于对某种权力的交易。

在商品化这里边有好多话想说,但是今天我们就不进入到那么细腻的层级。我想说的是:我们每一个人面对今日之社会,必须划下自己可商品化的底线——就是我什么东西是可交易的,我什么东西是永远不可交易的、是不能触碰的。每个人都应该有这个底线。如果有这个底线,你就是一个高尚的人、纯粹的人、脱离了低级趣味的人,是一个可以信赖的好朋友。这话说得……就是这个讲商品、讲商品化的时候,这话说得稍微有点拖了。

商品化为什么这么重要?就是商品化是资本化的前提。我们下堂课开始进入到资本。资本化和商品化的区别是什么?什么叫商品化?商品化就是可交易;资本化是什么?资本化是可交易并拥有的主权,是主权的交易,那个叫资本化。细节我们放到下一堂课去讲,因为这个问题非常之重要,商品化和资本化是不同的。在涉及到商品化之前,我简单陈述一下资本主义的基本上的运转逻辑。

资本主义运转的逻辑,在商品化之前还有“两化”:第一个化叫私有化。它涉及到主权问题,因为不拥有主权就无法进行交易。为什么低级动物无法进行商品交易?因为老虎不能拿一块石头,说是它的,就其他人不能碰。它可以占山为王,但那里边的商品它无法核实、确定它的主权。但人就不一样,人就可以确定它的主权——私有化。私有化是商品交易存在的一个重要前提。第二个重要前提是市场化。私有化和市场化成立之后就有了商品化。

商品化之后就可以有资本化了,资本化之后就会出现资本主义的一系列的问题。例如国际化,例如垄断,例如……等等等等资本主义的问题、弊端就出来了。商品化是资本化的重要前提,所以我们把它放到第九讲,放到《资本论》的开篇之前是有想法的。这是一个围棋,就是我们在包围圈里边逐步逐步逼近它,然后把它拿出来。商品化之后就会出现大规模的工业化,因为你要进行有规模的商品生产,才能满足较高的利润。

大规模的商品生产就出现工厂,工厂需要投融资行为就出现了资本,资本再将一切事物商品化,它是一个轮转的过程。就是你如果画图的话,商品化是居中的,就是私有化通过商品化就是市场化,私有化通过商品化——资本化,私有化通过商品化——国际化,形成了一个小轮之外的、大轮的整个的轮转。商品是核心,没有商品,其它的所有的化不成立,所以商品的含义是如此之重要。马克思写《资本论》奠基的四块石头,其中最重要那块石头就是商品,对商品的理解,特别对商品化的理解……

那么商品化有它的对立面吗?有的,那就是社会化。由于对商品的研究之后,才出了社会主义的理论,商品化对立面是社会化,就是不能商品化的东西就要被社会化,社会化了以后就去掉了商品的属性变成社会的公共品。我想这个好像有点跳,又有点跳跃和跳脱,但不着急,就是因为理解商品的时候,在理解马克思思维逻辑的时候,商品化走向它的另一面就是社会化。社会化与商品化,它是一种商品化的前行之后的一个必然结果。

由于,由于必须有序将商品一部分人民群众需要的商品社会化,所以这部分的安排就产生了,或者是必然产生了计划经济。我们在整个商品这个章节里慢慢来理解马克思的思索的逻辑和逻辑过程。就是商品化里边存在着严重的问题,就是极度商品化是有问题的,无论是个人还是社会都不能过度商品化;一旦过度商品化,这个社会就会走向它的反面。所以无论是哪些国家,特别是发达资本主义国家,在商品化的过程中,他们都会出现阶段性的反思,在这里边我再跳出来……

在资本主义发展的过程中,一共有三种类型:原始的资本主义,我也管它叫工商资本主义;原始的社会主义,我管它叫国家资本主义;超越了工商资本主义和国家资本主义的那种混合模式,我管它叫社会资本主义。现在目前德国和北欧的形态主要是社会资本主义,工商资本主义向社会主义转型的过程中可能会走入歧途,变成金融垄断资本主义。今天的美国的状况就是由工商主义向社会资本主义过渡过程中出现了问题,挪入了金融垄断资本主义。这个可以画一个数学的集合,中间那个圈就是社会资本主义。

中间那个圈是社会资本主义,围绕着它一共还有四个圈有交集:那就是工商资本主义、国家资本主义、官僚垄断资本主义和金融垄断资本主义。国家资本主义在向社会资本主义转型过程中,一旦出现失误就必然走向官僚垄断资本主义。所以国家为什么会失败?是因为走偏了。这个,没有哪一个国家是纯粹的工商资本主义,或者是纯粹的国家资本主义,或者是纯粹的社会资本主义,他们是一个复杂的交集,只不过这个交集之中哪一个比重更大?通常社会资本主义的比重更大,那么这个社会治理就会比较好。为什么中国目前经济表现比较好?中国正好是国家资本主义和社会资本主义相交合的部分比较大。

如果我们画这个交集的话,就是中国现在目前国家资本主义的成分、社会资本主义成分比较高,我们虽然出现了垄断资本主义的苗头,但它还不是构成国家治理的主体。我们要反对我们在这个过渡过程中向官僚垄断资本主义过渡。我们希望我们国家资本主义与社会资本主义形成某种有机的结合,它既有国家资本主义的那种效率,又有社会资本主义的那种均衡、那种活力,形成一种辩证的统一体。现在目前我们所说的新社会主义,或者是这种混合的机制或者体制模型还在摸索中。

事实上,马克思在写《资本论》的时候,他是从工商资本主义想过渡到社会资本主义,但是工商资本主义过渡到社会资本主义过程中需要经历一些的复杂的过程。他没有想到工商资本主义会发展出一种叫国家资本主义的东西;他更没有想象出工商资本主义也可以发展到金融垄断资本主义,就是列宁所说的帝国主义。当然了,马克思由于还没来得及见到国家资本主义形式,他更没有预见到国家资本主义会出现官僚垄断资本主义。我们在研究商品的时候,在研究马克思的逻辑过程的时候,顺便的把整个的制度类型做一个区分。

这个部分非常非常的重要。因为它实际上是一个国家的商品化的底线或者边际控制决定了它的社会化程度。为什么叫社会资本主义呢?就是它虽然仍然是资本主义,但它相当大的部分,它不进行商品化,而将那些商品化的部分改成社会化。在德国、在北欧出现了这样的一个情况。比如说医疗,在香港就能体现到,香港的医疗的社会化,就是所有的人看病一个价钱,这个商品,当然了,你住病房你可以挑单人病房、十六人病房,这是不一样的。

社会资本主义实际上是马克思《资本论》影响了人类的文明的走向。我个人认为,在东西两线的马克思主义的走向上面,东线的革命的这条线路虽然走的极端,但可能由于过于极端,反而使事物走向了反面;反而西线的马克思主义的进程,马克思虽然对西线的影响不像在东线那样是高举的旗帜,但它在本质上反向社会化,就是反商品化的部分、社会化的部分做的相对更为积极、稳妥和恰当。这个事情呢,就是我们明年成立新马会之后,要对理论上有一个重新的探讨,并且……

我们要将这个理论数字化。就是我们要精算:哪些商品不能商品化,哪些商品必须社会化?社会化的程度?哪些社会化的程度来定义一个国家的文明程度。就是商品化划定一些底线,不光是我们个人要对我们能商品化的部分要划底线;一个国家也要划底线。懂得划商品化的底线的个人、机构和国家就是伟大的个人、机构和国家。这个要懂商品化,又要懂得如何遏制商品化;要懂社会化,但同时又要知道商品化和社会化的边际。其实“正心以中”就是我们心学最高的境界是正心以中,寻找平衡点其实是最高境界。如果在寻找平衡点的时候能懂得下围棋,那就是大政治家。

我们在讨论商品化的时候,我们可能要考虑一下子就是整个的人类文明的一个历史进程。其实它在东方的表现是很有意思的,就是谁说中国没有搞资本主义?谁说中国没有搞社会主义?谁说的?中国在战国时期出现的一系列伟大的学术流派,其实代表了不同的倾向。我认为儒家、特别是思孟学派就是中国古典社会主义的学派;我认为中国老庄的古典道家就是古典资本主义学派;

我认为中国古典的法家就是中国古典的军国主义学派;我认为释迦牟尼佛,释或者是佛家,或者是释,他们是朴素的共产主义。其实在中国关于方向、道路与主义之争古已有之。但我觉得非常地惊讶,我们中国聪明地选择了儒家的治理方法。也就是说自汉唐以来,我们的政治上的选择偏向于社会主义。你从每一次的均田,你就能够理解到中国朴素的……

中国一直在追求一种朴素的社会主义,他们对商品化这个问题和社会化这个问题是有着深沉的考量的;中国在税赋上面也有着同样深沉的考量。比方说,我们的黄册、鱼鳞册,黄册是人头税,是这个间接税;鱼鳞册是田亩征税,是直接税。黄册和鱼鳞册往往是在王朝末年一把天火把鱼鳞册烧了,就全部变成了人头税,然后开始进行社会动荡,然后再重建、再重新分田,又建新的鱼鳞册。偶尔,我们会出现内部的治理上的问题,

比如说汉朝初年经济非常地贫弱,所以出现了文景之治。文景之治之后我们就出现了大量的独角兽,主要是盐铁——经营盐的盐贩子和铁贩子,富可敌国,然后他们跟一些地方上的王勾结搞七国之乱,搞什么“泰山会”、“西山会”之类的。后来出现了一个了不起的经济专家,他叫桑弘羊,给汉武帝出主意说:“既然要打匈奴,又要平衡国内的经济矛盾,那么我们就向独角兽征税吧。”于是盐铁专营,将盐和铁国有化,国有化之后变成国有企业之后,

获得了从盐和铁这种商品的这个国有化上面获得了大量的类税性收入,然后成功地击败了匈奴。当然了,这个桑弘羊同志本人他是有边际概念的,但是他碰到了一个大有为之君——汉武帝,他就是战争的规模到底应该多大?税收的规模应该与战争的规模和战争的目的相匹配。我个人认为把匈奴驱赶到一定的程度即可,没有必要赶到贝加尔湖去。如此大规模和长久的战争确实耗损国力。所以《盐铁论》关于这个争论是很有价值的,桓宽的《盐铁论》是中国经济史上一个重要的关于边际的讨论。

我说了,桑弘羊是对的;我也说了,霍光也是对的。就是桑弘羊建议盐铁专营,当时的情形之下跟现在收数据税是一回事,增加直接税和数据税是一回事。用什么来抵补地方财政收入,这是一个课题。既要把独角兽和那些在直接税上面通过垄断形成的高净值做一些处置处理,又要解决国家紧迫的现实问题。所以桑弘羊是老成谋国的,他的想法是对的;但是大型国企、关于盐和铁的大型国企建立起来以后,它带来的弊端也是明确的,也是百病丛生,到最后霍光取消盐铁专营也是对的。

好多朋友说你这说了差不多也是白说。谁说?当然不是白说了。这个事情很重要的。对商品的、特别是对重要商品的商品化还是社会化?或者是商品化中的管理是极为重要的!待会儿我们再讲商品定价的时候,还要再碰到盐和铁,就是《盐铁论》的部分。因为你要从这儿获得税收的话,一旦专营即为垄断,一旦垄断,价格就脱离了它的所谓马克思所说“凝结了一般人类劳动”,它就变成了一种牟利的工具,或者是变成了一种税收的工具,这个时候就会出现霍光担心的事情。所以桓宽的那个《盐铁论》是值得好好读的。

我想,其实中国的书和外国的经济学是不一样的;我想,可能美国人读不懂《盐铁论》。如果美国有一个经济学家读懂了《盐铁论》的话,那么美国人就知道他们在1945年之后开启的这四场战争——朝鲜战争、越南战争、这个伊拉克战争和这个阿富汗战争,他们就会懂。如果他们懂得这个我们《盐铁论》上讨论的边际的问题,他就应该懂这场战争他大概的目的、它的边际在哪里?他就不会把美国经济搞成今天这副模样,他们完全不懂这个战争,或者是跟税收的关系和结构的关系。

超越了自己的边际和能力的任何的安排都会走向反面。其实这里边也涉及到我们自己目前对一些事情的理解。比如说,我们对自己的能力的边际的理解。因为我这里边插一两句,这是我专栏文章上的一些内容。就是我最近在写文章、在讨论:我们到底应该将自己的能力边际放到多远的问题?要解决什么问题?比如说领土问题、领海问题、边境问题,到底这些问题应怎样解决?

这里边就涉及到一个对时间和空间的理解,就是我个人认为我们不能学德意志第二帝国,我们不能在1914年卷入第一次世界大战,我们不应该因为奥地利的皇储被塞尔维亚人刺杀,所以我们就这个打仗。就是如果发生了一些突然的变故,而且这个突然的变故貌似偶然,其实背后有人导演、有人在处理一些类似于像中印边境、南海、台海、东海、甚至一些区域,制造一些紧张的、甚至擦枪走火的事情、擦枪走火的事情,那么我们是否应该理解时间和空间呢?

比方说我们在2031年之前,我们不采取那样的方式来处理这些的问题。比如说我们可以再放20年,比如说在2041年之前,有些问题不做处理。不做处理的原因是,我这个跟好多朋友有剧烈地争论,就是我们不收回台湾吗?当然是要收回,但前提是我们决定收回东亚、收回东南亚,整体纳入东亚共同体之后,解决台湾问题,而不是单独解决台湾问题。我们是在下围棋,不是在下跳棋,不要折腾。如果十年之后,我们的体量——总体量超过了美国,成为世界第一,在20年之后,我们在全球经济中的比例、比重超过了30%或者35%,

那个时候我们就有足够的时间、空间和能力解决所有的问题。而且我们解决那个问题不需要付出某种代价,就是中国人不能在21世纪玩儿丢了自己“中国世纪”,如果这个事情都做不好,那么我们这五千年文明就有点白费了。好,在讨论商品化这个过程中多说了几句话,我们现在回到第三个问题,最重要的问题就是商品定价。其实商品定价或者是我们管它叫资产定价的理论,这里边就是做投资者必须要过的一道门槛,就是商品或者资产是怎么定价的?它的定价逻辑是什么?

我先说一些不搭嘎的话,通常我们对商品定价有一个相对等价模型,什么叫相对等价模型呢?就是我们认为商品是能交易的,如无一般等价物,我们进行商品交换的时候,比如说一袋米换五斤肉,我们就相对等价模型。相对等价模型形成我们对交换价格的一般性的理解。从一般价值形式到货币形式的过渡一旦完成,商品定价的历史过程就完成了,它是否合理其实都不重要,是因为它约定俗成,也就是所谓的市场,所谓的市场定价逻辑就形成了。

我们到了下堂课开始讲资本的时候,我们就要讲资本化会对商品化构成逆向的影响,就是资本会对商品定价形成逆向的影响。因为一般等价物就是我们那把尺子自身也是变动着的,而且那把尺子在跨区域的时候,它具有不同反响的意义。你就知道类似于我们要用美元完成跨境交易或者跨国交易的时候,美元所带来的价值上的、定价上面的一些问题。在讨论商品定价的时候,我们先注意就是相对等价模型,这是马克思提出来的一个重要的思考,马克思真的好厉害。这个相对等价模型,换了我,我是想不到的。

因为相对等价模型进行哲学抽象,才能从一般价值形式完成到货币形式的过渡。其实我们心里边对价值是有一个基本判断的,我们在购买它或者是给别人生产的时候,我们知道它的使用价值,但真正形成交换价值是非常的偶然的事件,就是为什么一袋大米换五斤肉,为什么?你不能精确计算其中的凝结的人类劳动,它是某种直觉或者是某种感觉,并且确认觉得合适,然后货币对定价模型形成反向的影响,因为持有货币者往往代表需求,持有货币者往往构成对需求的……

如我讲课的过程中,这讲课今天讲的有点这个激动,如讲课中可能有的部分掐的掐的忘了抬手,落了一些,少了一部分,请帮着修理的同学帮我把它补上一下去,逻辑上应该没有错就可以了。好,我们讲今天最核心的部分定价逻辑。商品定价的逻辑是什么?商品定价的逻辑基本上是由三个东西构成,第一个,也就是说马克思对它的一个基本的判断就是凝结着的人类劳动,或者是我们管它叫总成本,或者我们一般意义上管它叫底价,就是我们付出这么多劳动得补偿这个劳动。

那么如果是以总成本销售,就少了东西,所以我们会有一定的溢价,这个溢价就取决于它的稀缺性,它越稀缺,溢价水平就越高;越不稀缺,溢价的水平就越低。当它实在没人需要的时候,它可能要低于底价,就是低于总成本来销售。就是溢价水平取决于稀缺性,底价取决于凝结的劳动。那么稀缺性是个什么东东?稀缺性有可能是真实的稀缺,有可能是信息不对称,有可能是资本的垄断,资本通过垄断来形成。

在定价逻辑的部分,第三个就是垄断。凝结的劳动、稀缺性和垄断三样共同构成一个基本定价的逻辑。这个基本定价的逻辑,每一件事情真正的比重关系随着时间的推移,随着资本化越来越强大,随着金融垄断资本主义的出现,垄断变成了定价的最主要的逻辑。记住,垄断才形成了目前对商品最主要的定价逻辑。举例:黄金,它是由凝结的劳动来定价的吗?它是由稀缺性定价的吗?不是,是由垄断,是由资本的垄断形成了定价的逻辑。这个定价的逻辑,

这个定价的逻辑最终会回归它的本源,就是马克思说的凝结的劳动,就是底价和它的稀缺性,但在特定的时间它是由垄断控制的。我们要理解商品的定价逻辑,我们才能进行商品投资或者资产投资。现在我们把它延伸到资产定价逻辑或者资产定价理论。资产的定价逻辑,第一、凝结的劳动。这个现在变得非常之不重要,比如说房地产,房产里边凝结着的劳动,它的总成本或者它的底价,比如说应该是每平方米3千元,但是由于它的稀缺性,它的溢价水平非常高,可能溢价出了3万元。

由于垄断的形成,可能在三万元的基础上又再形成一种垄断的价格,形成六万块钱,我们懂了定价的逻辑,我们在处理我们所投资的内容的时候我们大体上会对它未来可能形成的这三个要素有一个基本的考量,然后形成我们的安排。我再重复一下,不厌其烦吧。我们再重复一下今天我说的三大理论,就是第一、创造价值的理论;第二、商品定价的理论;第三、价值投资的理论。你知道商品是有价值的,我们投资的是价值,价值投资的理论,在商品定价的基础上。

在商品定价理论基础上,我们形成了价值投资的理论。说到这里,我想跟大家说就是我们每一个人都应该自己写一套经济学,你要写你的创造价值的理论,你要写你的商品定价理论,你要写你的价值投资的理论,形成你的信念,形成你对事物的一般性看法。在定价逻辑这,我要讲一段弗里德曼的话,他是了不起的大师,因为他说了一句话:“通胀是货币现象”,什么叫通胀呢?通胀就是涨价,他说涨价是货币现象,他否定了凝结的全部劳动,他否定了总成本——底价,他也否定了稀缺性。

你能理解弗里德曼这句话,你就理解了什么叫货币主义——就是通胀是货币现象,就是涨不涨价钱说了算,资本说了算。而且这句话大体上构成了我们在1945年以后基本的一个经济发展的状况,或者是近二十年来,虽然是MMT的理论推出还是比较晚的,那么他这件事情也已经成为一个广谱性的现象,这对我们来理解整体的一种资产价格的变动,或者是理解货币的价值,具有十分重要的意义,十分重要的意义。

讲完这个定价逻辑,要讲一下子垄断与操纵。目前来看,由于国际化的形成,几乎所有的商品,特别是大宗的商品和重要的资产,全部形成了资本的垄断与操纵。也就是说马克思原来对商品定价理论里边的凝结了全部的人类劳动那个底价仅具一般参考意义,或者是不具备太多的价格指导意义,稀缺性在很大程度上是一个信息不对称的结果,就是是假的稀缺而不是真的稀缺。在房地产这样的事情上,在股票、在美股和中国房地产事情上目前表达的非常非常之充分。

什么意思呢?就是美国的股价并不表达于美国那些股票、那些股权的稀缺性。不是的,它是一种资本过剩之下的对资产的一种垄断和操纵。中国的房地产也表达为资本对一部分资产的垄断和操纵,它垄断和操纵的内在的逻辑已经不是商品定价的逻辑,而是通货膨胀逻辑。什么意思呢?就是你手中的货币即将贬值,你必须把它换成资产,就算这个资产你并不需要,你也必须把它换成这个资产。来完成保值、增值和投机。

在讨论商品定价这个过程中我们讨论了四个问题,第一个问题是定价逻辑,我们说过了,第二个是垄断与操纵。第三个部分是从商品定价到货币定价,因为这涉及到剪羊毛的问题,就是商品定价或者是资产定价不合理是因为货币定价出了问题——MMT,在商品定价与货币定价的矛盾之中,或者是这个矛盾过程,或者是这个波动的过程,不合理的商品定价形成了对我们劳动的剥削,这个剥削远远超出了马克思在,(很快我们讲第二遍的时候)对剩余价值的剥削。

为什么我说商品这章非常重要,商品定价非常重要,就是在商品定价里边有对剩余价值的剥削,对吧,这个定价里边包含了我们的剩余价值、劳动者的剩余价值,但还不完全,他不仅仅剥削了我们的剩余劳动或者是剩余价值,他还剥夺了我们六个钱包,还剥夺了我们的父亲、爷爷的剩余劳动。从商品定价到货币定价整个这个环节我们全部放到下一讲或者下下一讲里边。我们讲资本的时候会讲到货币定价和商品定价的关系,那个时候我们有一个专门的剪羊毛的这样的一个,割韭菜、剪羊毛的这样的一个环节来对这个部分作一个详实的解说,你只是需要知道……

不合理的商品定价构成了人与人之间、企业与人之间、国家与人之间的残酷的剥削,你叫他割韭菜也行,叫他剪羊毛也好,为什么我们将商品定价理论或者叫资产定价理论、或者叫资本定价理论看得如此之重要?是因为我们在这个定价的过程中看到剥削,也看到我们投资的机会。其实学好了《资本论》一定会赚钱的,这个应该是这样的,因为你真的懂了商品定价或者是资产定价的理论,那么价值投资想失误其实很难的。其实是机会很低很低的啦。

今天讲商品定价的最后一个环节就是反对金融帝国主义的压迫,这个好像有点政治了。那个我为什么要把最后一个章节变成反对金融帝国主义的压迫?因为我个人认为资本对劳动者的剥削目前来看我们是有办法的,但是金融帝国主义对其他国家与民族的压迫,其实我们到今天未必找到非常好的解决方案。如果你懂得商品定价的理论或者是资产定价的理论,其实你就懂了价值投资了,可以不被资本剥削和压榨。但国家的问题的解决就复杂起来。

我们想说商品定价权在谁手上?比如说黄金在谁的手上?比如说碳排放权,比如说中国的房地产的定价逻辑和定价权在谁的手上?从石油危机到今天的MMT,我们是否看到了那个定价权那只手,我们从黄金到碳排放权,我们是否看到了那只手?我们从舆论控制,舆论控制不仅控制老百姓,也控制政府,到金融控制,它不仅控制政府,也控制整个的经济运行、控制市场,我们看到了一系列完整的体系的压迫,体系的构造的形成的压迫。

说到这里,我想大体上把今天的商品定价的商品的这个部分就讲到一个阶段了。原本是在商品定价和资产定价里边要举一些重要的例子、案例,就是他们是怎么完成对于一个商品或者是一个资产进行定价的。比如说独角兽的定价,比如说你认为阿里的价格现在合理了吗?跌成这个样子合理了吗?你认为腾讯跌到现在合理了吗?如果这个资产定价此时是合理的,那么价值投资的机会是否已经出现了呢?

你认为黄金的价格应该是多少?如果给你一个时间的推移,一年后、三年后、五年后、十年后,你认为黄金的价格以美元计应该是多少?以人民币计应该是多少?以其他商品的比价关系计应该是多少?比如说黄金与茅台的关系。商品定价它有它内在的逻辑,它有它发展的规律,商品定价构成里边的一些要素需要你自己慢慢地去整理、体会、理解,并且进行较长时间的观察,

它会形成你个人独特的价值投资理论。你的价值投资理论是在整个跟踪的过程中,对一个资产、哪怕是一只股票、或者是一个房地产、或者是黄金,长时间的跟踪、观察,形成你对它的定价逻辑、价格波动的一种理解。这个理解不仅仅是包含了它凝结的劳动,也不仅仅是稀缺性,而更主要的是垄断,或者是对资本控制这个资产或商品的理解,这样的话就是一个完整的画面就出来了。

我请大家务必要记住,其实是三大理论体系合成。这个合成的过程我们争取在《通论》讲完的时候,把它整个体系就完整了。就是我再重申一遍,是创造价值的理论、资产定价的理论或者商品定价的理论和价值投资的理论,是三大理论体系构成我们对整个对价值和投资的一个完整的认识体系。我们必须拥有这个认识体系。有了这个认识体系,那么我们在处理我们个人生活的问题上就没有问题了。另外,我们用这样一个体系来解剖、剖析制度,我们就可以形成我们自己完整的理论和看法,这样的话就有点意思了。

这一堂课我看就先讲这么多吧,因为朋友们对目前的局面都有点儿担忧,花几分钟聊聊。其实我都不太想讨论太细节的问题,但是还是说两句吧。因为说了“按既定方针办”,说了“短股长金”,好多人对最近的波动心理上是有压力的。不要因为某种的操纵而形成对大趋势的看法,不要因为由于某些操纵,无论是舆论的操纵还是技术图形的操纵,而形成对一个大趋势的看法。请相信:通胀来了。

我个人坚定不移地认为,美国已经开始进入恶性通胀的历史进程了,并且我看不出来美国人解决恶性通胀的方法。以前美国人解决恶性通货膨胀是需要他国——上个世纪是日本,这个世纪是中国。以日本为主体,或者是以日本为首的四小龙为主体,上个世纪向美国输出通缩、接纳通胀,以至于日本失去了三十年,接得太狠了。中国在本世纪初,从2008年到2018年输入……

中国输入通胀,向美国输出通缩。我们的测算,总的规模,当然不是中国一国完成的,大概应该接近五十万亿美元这样的一个水平。它也构成我们房地产涨价的原因,就是中国输入了通胀最后全部凝结在房地产里边了,我们形成了几百万亿房地产的资产和负债。那么,此时此刻是2021年了,中国仍然有向美国输出通缩,我们输出低估值资产,滴滴打车这样都是低估值资产,就是向他输出通缩,我们背通胀,输出廉价商品,我们终于知道这个钢应该取消退税了。

我们一直向美国输出通缩来接纳美元的通胀,这个历史进程走到今天,目前看来,中国人并不是特别的心甘情愿。这里边有两个层次的问题:第一,我们不是很开心,就是我们背通胀、送通缩、还挨骂。这个世界怎么可以这样,你求着我,让我把优质资产给你、优质商品给你,然后你印绿纸拿走,然后你还制裁我,这都是什么事情?!就是招亲纳贡就可以了吧?招亲纳贡你还来打,这就有点过了。这汉武帝就不答应了,这唐太宗也不答应了。这不能这样下去,这样下去成何体统?

那么如果没有人接受美国的通货膨胀,那么美国就必须将通货膨胀自己在内部消化。目前美国可用于消化通胀的资产两种类型,一个叫股权,你们已经看到了,它已经上天了,现在还没结束。按照我教大家的一板斧,你们去试一下那一板斧,他美国股市真的没结束。另外,就是美国的房地产,美国房地产差不多也接近历史最高水平了。但,这远远没有结束。也就是说,他现在在用美国两个最核心的资产在容纳他的通胀的泡沫。

如果没有第三方愿意向他输出通缩,并且接纳他的泡沫,那么我个人认为,美国的恶性通货膨胀已经处于倒计时的过程中了。我不否定美股还可以上到四万点,我也不认为美国的房价在现有基础上再翻一倍有什么不可以,我不觉得这个是有什么大的问题。但我在想,难道美国的CPI、美国的生活用品,真的可以保持在一种低价的状况吗?如此廉价的状况吗?要知道现在海上的运费翻倍了,要知道中国钢铁出口不退税。

所以,我个人认为,美国的通胀、恶性通胀的到来已经是板上钉钉的事情,只不过这个通胀何时打爆他的经济,这是我们的第一个判断。第二个判断是什么呢?就是如果这个通胀成为现实,而且成为恶性通货膨胀,如果CPI不表达为现在的5、6、7,上两位数,那美国要不要加息?如果加息会出现什么情况?对其他的资产会产生怎样的影响?MMT怎么往下做?其实这是个大的问题,我们中国目前对可能出现的全球性的恶性的金融危机和经济危机,做好准备了吗?

顺便说一句,我们这些个人做好准备了吗?好多人说,“金又跌了,那么我们是不是应该退出金去转其他的商品?”我的想法非常简单,就是你愿意怎么着就怎么着,我只提供一个思路,我真不能替你思考。我是觉得全世界都在等一声枪响,那么整个的结构就全乱了,就全变了。当然我们不希望这声枪响。今天是辛丑年的六月二十九号,很快就是辛丑年的七月,再过一个半月,就是八月十五了。

好多事情看得长一点儿,看得远一些。另外,在剧烈的波动的过程中,在大的危机的过程中,要学会持盈保泰。我个人认为,既定方针可能是最好的一个选择了。我当然不是给你们任何建议,你们随便,你们随意就好了,我只是告诉你我的一个基本的看法。其实我对我国从2020年下半年以至于到2021年上半年采取的一系列的经济上的举措是表示认同并且赞扬的,我们是有危机意识的,而且我们是有所准备的。

我唯一的忧虑就是,我们可能对时间和空间的理解上不够那么“围棋”,就是文木先生昨天晚上的忧虑。我们可能还真的需要10年到20年时间。就是我说过,德国如果知道1914年不卷入世界大战,他忍到1924年,他已经是世界第一大国;如果他忍到1934年,他已经可以横扫天下;如果他忍到1944年,那么就是德国世纪了,马克将代替英镑、代替美元成为世界的货币,那就是德国世纪。但是他忍不住啊,他对时间和空间的理解是错的,我们希望我们这一代的人……

另外就是疫情的情况有一些这个麻烦,就是好像这个疫苗的拦阻的这个能力是不够的,只是解决了重症问题。目前来看,这个秋冬的这个反复、这个变异病毒的反复会变得非常的强烈、猛烈。我今天在这说两句我的看法:其实我认为疫情或者是病毒已是强弩之末,就是在我们不认识它、不了解它的时候,它的危害是大的;在我们已经初步的对它有了认识、有了一定的方法,虽然它仍然会有巨大的麻烦,但它是尾声,当然是最后一击。

当年,在1918年的那场大流感的那个过程中,其实也是很麻烦的,最后一击也是很残酷的,但它毕竟是接近尾声了。就是它对经济的负面影响依旧会非常严重,但它已经无法再次出现阻断全球的经济运转,到不了那个程度了。另外,看这个样子,今年不会出现去年全球经济负增长那样的一个局面,只不过是今年的经济的恢复可能未如理想。如果是今年的经济恢复未如理想,并且有可能在今年的年底、下半年,或者是在明年或者是怎样……

不排除由于美国的经济出现了一些问题,引爆一场危机。我们不搞封建迷信,其实我对这个辛丑年的这个年尾是有一些忌惮的。但我们不搞封建迷信,不说那些风风水水的话吧,就没什么意思,我们只是做好我们该做的事情就可以了。我想今天我们这个课就这么多,然后明天下午3点钟,我们准时交换一下资料,交换一下看法,如果有没有说到的地方,明天下午再补齐。

好,就这样。另外,希望大家还是要注意个人卫生,注意安全。今天特别特别要提醒到一点,就是希望大家学会调整自己的心理状态,要调整在疫情之中的心理状态,在疫情和在剧烈波动的市场的这个状态里边,个人的心理状态和生理状态要调整好,一定要调整好。没事了多念念我们心学里边那几句话,调整好它。好,就这样,再见。

\section{资本、关于第十次经济工作会议“共同富裕”的问题}

大家好,今天是2021年的8月21号,是辛丑年的七月十四,今天是《资本论》的第十讲。原本《资本论》第十讲应该讲货币,但我思虑再三,决定先讲资本,将来在资本和资产分类里边再去讨论货币以及四矩阵的问题。因为我想先提纲挈领一下子,所以今天我们先讲资本,腾出时间再说几句关于第十次经济工作会议“共同富裕”的问题。好吧,三点钟我们准时开始,我试一下麦。

大家好,今天是2021的8月21号,辛丑年的七月十四号。时间过得像飞一样的,一周过去好像还没来得及备课和准备,就又开始又是个星期六了。在上一周结束之后,我在备课的时候,有朋友来访问我,我们俩讨论到《资本论》的第十讲。我原本打算是讲货币的,后来这个朋友说:“你不如在资本之后放进资产栏目里来讲货币,这样是不是可以大家更清晰一些?”我接受了他的建议,我觉得这是一个好的建议。

另外他认为第十讲太重要了,应该单刀直入资本。因为我们整个的《资本论》的课程里边,第十讲应该是揭盖子的时候了,就是《资本论》核心的问题是资本。揭开资本之后,再将资本的类别、资本的各种形态给大家介绍一下子,然后再将马克思《资本论》第一卷里边的两个重要的部分把它讲出来,一个是马克思关于剩余价值的部分,一个是关于资本流转的部分。把它逐步地、抽丝剥笋地揭示开来,这样的话,可能大家会有一个宏观的、整体的概念或者是印象。

其实准备《资本论》这堂课最难的部分就是资本。最简单的部分永远是最难的,最朴素的部分永远是最高深的。资本,貌似所有的人都知道这个词,但这个词能说清楚,太难了。英文、西班牙、法是Capital,德语用的是K字打头,不是C,是Kapital。马克思将它定义为一种生产要素,一般的学者把它定义为是为用于生产的耐久财。

广义的一般定义,资本是说它是创造收入的累积的物力和财务资源。老实说都不够,不够不是说他概述得不对,是对,还是不够全面,也不够深刻。亚当⋅斯密倒是接近资本的本质了,他的概述是:一个人资产中用于生产利润的部分就叫做资本。这件事情其实我一读《资本论》的时候就思考过,到底什么是资本?我决定使用一个新的概念,我决定使用生产力工具。

我是这样描述的,请大家记住:资本是已转化为生产力工具的一切事物。重复,资本是已转化为生产力工具的一切事物。好多朋友说:“听不大明白,第一,什么生产力工具?”就是它可以作为创造价值、生产商品、谋取商业利益的那个工具,我们通常把它叫生产力工具。是不是这样的概述就已经绝对精准呢?其实我这两天在想的时候,我总是讨论资本问题总会想到毛泽东,还可以再讨论。

在资本的分类部分我今天就不想给大家讲了,如果大家能够上网,可以去看资本的词条,百度的也行,维基百科的也行,维基百科会全一些,上面有资本的各种分类。我个人认为那个资本的分类不重要,我觉得资本是以有形或无形的任何形式存在,有无形资本,有有形资本,它是以任何形式存在。你给它任何形式用来定义资本都没有问题,也不会错。所以那种分类无论是怎么细分、粗分,我觉得都不重要,所以这个地方我就不讲资本的分类了,我要说的是资本的特征,这条非常重要。

我认为马克思在讨论资本里边的有一个部分,我有不同意见,就是可变和不变的问题。其实我认为资本永远处于或者是始终处于动态变化之中的,就是资本是个活物,它一直在动态变化之中的。这件事情是非常重要的,因为它是一个哲学问题,就是在处在一个不同的时间和空间范畴之内,在不同的时间和空间的定义下,资本在那一瞬之间是不变的,但是随着时间的推移,它会变动不居的,它是活物。但当然马克思的可变和不变,他有他的计算方面的需要。

讨论到资本,其实我们就要回到《资本论》,因为我们讲的是《资本论》嘛。《资本论》的第二篇只有一个章节,就是第四章,马克思用第二篇的第四章,第四章就是第二篇,讨论了货币转化为资本,这个题目就是货币转化为资本。我原来想讲货币,也是想在货币转化为资本之前先把货币讲完。不过我想了想,既然我已经把这个课程全部拆了,就是打碎了重组,拆了,所以先讲资本后讲货币也未尝不可。那么我们今天就先进入资本。今天我们给了资本一个定义,是已转化为生产力工具的一切事物,一切事物,不是物品。

资本遇到的最大的难题不是定义,而是测量,资本的量度。在资本的量度方面就出现了严重的问题,因为我们有一套会计体系,我们有资产负债表、有损益表。资产负债表和损益表里边涉及到资本了没?理论上是涉及到的,对吧?我们投入的资本、资本金是多少?资本利得是多少?资本回报率是多少?股东的权益是多少?理论上是可以测度的了,而且我们通常也是这样理解的。然而如果你这样来理解个人资本的量度或者机构资本的量度,可能会出现严重的偏差,这个问题是非常严峻的。

比如说,我们说如何测量或者是度量马云的资本总量,或者是他的资本调度的能力?其实这个测度跟阿里或者跟他的其他的公司几乎没有联系。那么一个人或者一个机构的资本的量度、测量或者是评估、评价,那是一个什么样的事情呢?我想为了今天后续能够解释,在测度这儿,我再给它一个定义,就是资本量度的定义,是指能调度用于商业行为的一切东西。

我这个地方没有用事物,我用的是东西。其实我不想用东西这个词,但是没有办法,因为它包含了人、财、物不同的东西。所以我用的是“能调度用于商业行为的一切东西”。每一个人能调度的能力是不一样的,之所以成为超级富豪,是因为他调度用于商业行为的、调度的能力——调度人的能力、调度钱的能力、调度物的能力都超级强。强到什么程度呢?强到一些人或者是一些机构开始扮演上帝。好,我们顺便再给资本家一个定义。讲资本嘛!必须把一些事情说清楚。

那么什么是资本家呢?资本家就是调度一切东西以获取个人商业利益的人。调度一切东西以获取个人商业利益的人,那个人叫资本家。请千万不要认为资本家就是做生意的。能调度一切东西获取个人利益的人,可能在国家机关——他也是资本家;可能在事业单位——他也是资本家。因为他调度一切东西以获取个人利益,他就是资本家。你们明白了吗?我们对资本家的理解当然不是在商业系统里边那个调度一切东西的人。

我说到这儿的时候,你们可能能理解毛泽东的想法了吧?毛泽东对资本、资本家的认识的确是历史的超前,他是个哲学修为极深的人。他一眼就能看出周永康是资本家,徐才厚是资本家。周永康跟政治没什么关系,徐才厚跟军事没什么关系,他们就是个资本家。他调度一切东西以获取个人商业利益,这就是资本家。不管他的身份是什么,他就是资本家。不用为他的光环、为他的定义而感到有一些什么,他只要在调度一切东西,以获取个人商业利益,这个人他的名字就叫资本家。

我并不想让资本家这个词有贬义的意思。资本家既不是贬义,也不是褒义,它是个中性概念。但是当你动用公权力来调度一切东西以获取个人商业利益的时候,不妥当。所以这个资本家是不被一些社会所接受的、所容纳的,越界了。这些资本家就不如你放下你既有的身份、老老实实回到商业领域里边去,去做一些你可以做和应该做的事情,真正的资本家调度的东西是合情、合理、合法的。

当然,资本家之所以能够在一个社会扮演上帝,是因为他也越界了。他不光是调动了货币——钱,也不光是调动了物品,他也调动了人。甚至我刚才说的是,就是他能安排人、安排事,他甚至能安排一个银行的行长,这不是人和事吗?当你能调用一切的时候,有些人情不自禁地在扮演上帝。其实,我们讲资本,必须看到资本的本质、看到资本家的本质、看到资本主义的本质。不然我们上这堂课干什么?当然我一会儿要讲。

我一会要讲,资本它是一个纯意识、纯意识的一个投射,纯意识的投射。在我们心学里边它非常重要,我一会儿再讲这个东西。因为我们既要用我们学到的东西认识整个的世界、认识社会、认识他人,同时我们也要指导我们个人的实践,我们也要修正我们自己,我们不断地提升我们自己调度一切东西的能力。当然了,我们调度一切获取个人的商业利益,目的不仅仅是为了自己获得比较好的生活,我们应该有更为崇高的理想、更加社会化或者社会主义的理想。

好,该给资本主义一个定义了。什么叫资本主义?你们上互联网查资本主义,是有很多的(定义),那个不是我要定义的。我给资本主义的定义非常之简单,就一句话:资本家扮演上帝的时代,我们管它叫资本主义。当一个社会资本家在扮演上帝,他们开始决定一切的时候,它就是资本主义。我们为什么不要这样的主义,我们为什么不要这样的黄金时代?因为它对社会、对人类、对自然是不友善的。

好,今天开篇讲了一大堆的定义。从资本的定义到资本家的定义、到资本主义的定义。这些定义不是今天讲资本的核心,今天我们讲资本的核心在哪里呢?我们讲资本的核心是在讲一个动词。资本的这个名词,我们开始对它有了比较深的认识。我相信如果不读《资本论》、不上我们的课,可能对资本的理解还是比较浅的,但这是远远不够的。真正地理解资本,必须理解这三个字,我们管它叫“资本化”,就是资本作为一个动词的时候叫“资本化”。

又是我们的老习惯,所有的东西都要先给一个定义。我们怎么来定义“资本化”呢?通常我们是这样来给它一个基本的概念的,就是将事物用于谋利的行为,我们叫资本化。千万不要认为资本化的行为只发生在资本家身上,或者是我们认为的商业领域里边的资本家的行为,不是的。在法治、物权不完整的社会,资本化的行为随处可见,而且这个资本化的行为,它甚至是社会的毒瘤。

请记住以下的话,这个话有点残酷:一切物品皆可以转化为生产力工具,一切物品!有些人将自己的肾卖掉了,肾就变成了生产力工具;有些人卖身,身就是生产力工具;一切物品皆可以转化为生产力工具,有些人将自己掌握的权力卖掉了。总之吧,什么东西都可以转化为生产力工具。非常残酷,故,一切物品皆可以资本化,这包括了人和政府本身。

资本化在资本主义的初期和中期是非常非常残酷的,因为资本化的过程横扫一切。横扫的不仅仅是穷苦人的那些物品,它也横扫了普通民众的精神,它甚至将人格异化。马克思在《资本论》里边反反复复在强调的拜金主义和异化的过程,其本质就是资本化的过程。他们在将重要的东西资本化,而且这个资本化的过程没有底线,没有边际,没有时间限制。

当然,马克思在第二篇第四章里边,他讲的是,只讲了货币转化为资本,就是货币的资本化的过程。其实我反反复复读这个东西,读完了以后,我在想当时马克思的思考。因为马克思必须迅速地完成《资本论》,他不能像我这样的思绪飞扬,可以到处乱跑。所以马克思,我个人认为,在货币转化为资本的时候,他收了,他只是局限于货币,而没有将资本化的全部的过程和广泛的资本化的行为作出概述、表达,而对资本主义进行更为系统的批判。那么这件事情我们再把它做完吧。

我想说的是,资本化必须有边际,资本化有它的空间的边际,也有它的时间的边际。当资本化跨越边际之后,资本主义必然走向邪恶。无论是什么形态的资本主义,无论他是自由资本主义、社会资本主义、国家资本主义,都不能超越底线和边际。超越了,他就必然走向反面,走向覆灭,这是一个必然的结果。另外,我希望大家对资本化这个概念,用广义的视角来理解,而非狭义的视角。狭义就是我们通常会把货币的这个资本化作为资本化的主要的方式。

在这里边说一段毛泽东的看法,毛泽东厉害就厉害在这个地方。他一早就料定了资产阶级复辟,无产阶级专政下继续革命。他知道,一旦掌握了一定的权力、一定的生产资料,那些人一定会进行资本化,他们在资本化的过程中重新沦为资产阶级。所以毛泽东说,走资派还在走,要打倒党内的资产阶级;还说,我党党内有资产阶级司令部。这在当时的特定历史时期,乃至于文革结束之后数十年,我国的思想家、学者仍然无法理解。

直到本世纪2008年之后,我们看到了如此生动的资本化历史进程,大踏步跨越所有边际进行资本化的时候,2008年到2018年是一个非常重要的历史进程,这个时候可能不少思想家、学者才开始慢慢体会毛泽东的厉害之处,厉害在毛泽东对资本化的认识超越了时代。在资本化的认识上,毛泽东的思维直追马克思本人。其实马克思真的很厉害,他写《资本论》这本书可以把人带到一个极其高远的境界,确实厉害。

毛泽东反对资本化,反对无序资本化的过程,并没有进行系统性的理论论述和系统性的制度建设。毛泽东用的方式方法我到今天仍然觉得很神奇,两个字:运动。运动这种东西能解决资本化的问题吗?我想,从短时间的历史来看是不行的,甚至是负面意义的,但是从历史的长河上来看,毛泽东在完成一个基本的启蒙和教育运动、教育过程,这是有意义的。要知道今天来进行反资本化的那些人,都是毛泽东教育出来的人。

你们知道什么是“共同富裕”吗?“共同富裕”就是要反对过度资本化。过度资本化导致的异化就导致资产、资本迅速向少数人归集,出现大规模的财富兼并,跟过往五千年的土地兼并一模一样,进行财富兼并,这是资本化的必然的方向和必然的结果。阻止资本化,让它在一定程度上重新社会化,这就是社会主义的本质啊!“共同富裕”说的就是走社会主义道路,这真的很了不起呀!

马克思的理想与世界的现实 https://mp.weixin.qq.com/s/yN3qVHNpx9vJnyPo0bsPjg

说好了的,就是讲课不能激动,讲《资本论》激动这不合适。我们今天在讲资本化的时候,我原来准备了一些案例。昨天晚上又出去吃饭,朋友说,不要结合当下现实,保住平台,因为大家非常珍惜这个平台。另外他们也是心疼我,就是你没有多少地方可以发出声音了,只要能发出声音的地方,所有人都在那儿堵着呢。这个平台能说两句话,大家也想听,你就别联系现实,甚至不能联系太近的历史,那么就,案例就算了。

关于资本化的问题,我想你们自己去思考,因为每个人都看到、感觉到一些合理的资本化的过程和不合理的资本化的过程。今天讲资本化,我想从一个正面的角度来讲我们如何将我们身边的事物有序地进行资本化?我们身边的朋友们能否成为资本家呢?如果我们能够掌握资本化的技术并且成为资本家,或许这堂《资本论》的课就有了更为崭新的意义。因为我们毕竟不仅打算砸烂一个旧世界,我们还要建立一个新世界。

其实我在本堂课的开篇已经讲过了,资本是一个意识,也就是说它是唯心的,当然它可以表达为具体的物,但它主要是唯心的。心动是我们心学里边的要求的,主体性的出现。当你的主体性出现的时候,当你准备完成一件事情的时候,这个事情可能是要准备实现某种的商业组织、某种创造价值的行为、实现某种收益的时候,这个时候你就变成了资本家,你开始调度一切的资源来完成一个构想。

如此,你个人的资本化过程就已然展开了。我知道多数的人本身在主体性上很难升起,特别是在资本的主体性上是非常难以升起的。我们受的教育是这样的,当我们大学毕业,什么毕业都好吧,不读书也好,我们是去找一份工或者去打一份工,我们自觉地将自己定义为被雇佣者。我们成为资本家调度一切资源里边那个资源,而没有思考如何成为调度者。

我离开北京来香港很多年了,前前后后算起来26年了,接近我生命的半程。在这个漫长的岁月里边,我接触到的各色的资本家,他们的一个共同特征就是他们有想法,有做生意的想法,实际上是主体性升起。特别是做大生意的人,做到非常大的生意的人,他们是有想法而没有钱的。所以我们今天在讲资本化的时候,我想借助马克思第二篇的第四章就是货币资本化的问题,讲几句。

什么样的货币会资本化呢?技术上讲,用于储蓄的货币没有资本化,用于消费的货币没有资本化,只有用于投资的货币资本化了。但是你们注意一个重点,投资的货币是谁的?是资本家的吗?在大多数时间里边,可能在99.99\%的情况下,那个钱不是资本家的,是他调度过来的。可能是借的方法,可能是融资的方法,可能是故事的方法,总之他调度来了那部分的货币,完成资本化。

说到这个地方,开始慢慢地能够接受我为什么将资本说成它是意识,不是物。当然它是物,但是它主要是个意识,就是当你将这个东西用于调度、用于获取商业利益的时候,它就开始资本化了,它就变成了资本。这包括了人、财、物,包括了各种东西。当然有些坏家伙将权力、将别人的器官作为盈利的工具,这个就违反了天理、法理、人伦,整个的这个是不对的。他这个资本化就越界了,但资本化就是这样的。

当我们意识到,当我们意识到资本是什么,我们懂得了资本化的时候,其实每一个人要思考的问题就是:我,就是你,我们,我们作为一个主体性的人,我们是否思考过我们可以用什么样的方式谋取必要的商业利益?思考过没有?我们要思考的问题一共是两层:一个是我可以做什么?一个是我被别人用来做了什么?因为你没做的时候,有可能你是被别人调度的资源,你是别人的资本在谋取商业利益,就是主体性没有升起的时候。

我说了,讲《资本论》,讲到这个章节的时候会令所有的人会感到隐隐作痛,开始痛苦了。我们讲到价值的时候,曾经讲过定价理论,我想你们都记得我,万物皆可以定价。就是你成为资本的时候,你出卖你的劳动力,或者是你出卖人格,或者是你出卖你的身体的时候,你的定价合理吗?当你成为整个商业行为的一个部分的时候,那个价格是合理的吗?是对的吗?你是否曾经想过脱离那个体系,而独立地展开你的整个的一个创造价值的过程呢?这是一个痛苦的思考,我知道对每个人来讲都极为……

好多朋友说《资本论》这本书不仅仅的是照射出、映射或者是映照出一个制度的问题,它在很大程度上开始也映射出我们的人的局限性,或者是我们人的悲剧。这里边看到了一些人性的东西,当然我们不把它叫做人的人性或者是人的……我们更多地把它概述为一个哲学的范畴,就是人的主体性。我再说一遍,我们心学里边讲的三个性:主体性、适应性和创造性,你们知道“三性”的重要……

主体性、适应性和创造性,当我们没有“三性”的时候,其实我们就是被资本家调度的一切事物里边那个物,然后成为别人资本的一个组成部分,而且可能是定价不合理的一个部分。等我讲到马克思后边的部分,我们可能向雇佣我们的资本家提供了剩余价值,而且这个剩余价值不是一般意义的剩余价值,是以生命为代价的一些东西,可能还不仅仅是生命代价,这个还有尊严等东西。这个思考过程其实挺重要的。又涉及到人性问题,这个有点复杂,但我们又不能不思考这个问题。

讲心学的时候,我们老是说三断论,三个性,就是主体性、适应性、创造性。第二个部分是“正心以中,修身以和”,这是做学问的过程,也是做人的过程。为什么是“中”呢?就是我们不能资本化越界。“修身以和”,就是我们跟所有的人,调动所有的资源的时候,我们是为了一个正确的目的和正确的方向。当然方法论的部分以“无间入有隙”,这个不涉及到道德判断,只是一个正确的方法,因为这是一个方法论的部分。但是大家一定要把我们基本的东西牢牢地记住并把握,在处理人生、处理生意、处理一切事物的时候,要灵活地运用于实践之中。不能学完了……

好,讲到资本化的这个部分,我想讲一下子关于资本化里边的一些的负面的意义吧。正确的资本化是类似于毛泽东的……其实毛泽东是资本化的大师,因为不管你是伟大的资本家、经济学家、还是政治家,他都必须有资本化的能力和资本化的工具,只不过他那个资本化和别人的资本化不一样,他调动的不仅仅是钱,他调动的是人的思想和灵魂和积极性、创造性。

一介书生,手无缚鸡之力,身上也未必能有一百块大洋,可以掀起滔天巨浪,他用的资本是什么呢?他有资本吗?没有。我再说一遍,资本是意识!不是货币!如果你非要将资本当成货币,我没有货币我就不能资本化的话,那你学心学做什么呢?“赤化”就是资本化的一种方法,或者是心学的一种方法,请牢牢记住。

神奇的事情的发生,实际上是主体性觉悟之后形成的一种强大的凝聚力,这种凝聚力表达为调度、整合所有资源的能力。当你能有这种调度、整合所有资源的能力的时候,你不仅仅可以创造价值,而且可以引领历史前行的方向。我们在学习的时候,我们有很多的时候要跳出我们固有的、或者是我们从前所做学问里边的桎梏。

资本化的过程,诚如毛泽东的诗词所说:“红雨随心翻作浪,青山着意化为桥。”万物调度皆可为资本,万物调度皆可为资本,就是一切事物皆可为我所用,这就是真正了不起的人创造奇迹的原因,这就是我们在看到的一些了不起的生意,大的、了不起的资本家创造奇迹的历史过程,是主体性的觉醒,是适应性的发挥,是创造性。

资本不是我们想象的或者是《资本论》第二编第四章所说的货币资本化。蒋先生——常凯申同志,他理解错了,他认为他的资本是源于货币。所以在北伐的时候,他用的钱是苏联的钱,苏联帮他武装了一个师的军队;北伐军给了他一笔经费,这是当时由孙中山联俄联共打下的底子;打到武汉的时候,他有想法,这个时候他去了趟日本,与浙江财阀达成共识,从英、美、日本获得了资本。

这个时候他有了资本,因为获得了货币资本,所以完成了对国民党内部的整合、对军阀的整合,获得了江山。他是用这样的方式获得了江山,以货币资本为基础,而且是以西方殖民者的货币资本为基础获得了江山,注定了蒋先生的江山的结构,就是它是一个买办、官僚垄断的资本主义的政府,他注定无法成为中国未来的选择。而毛泽东的资本没有苏联人给,想要都没人给呀,在那样困难的井冈山,在那样困难的延安,他能依托的部分只有两个字——人民!

是这样的一个资本的结构,注定他最后创立的是一个中华人民共和国。所以当毛泽东写《新民主主义论》的时候,其实他想创造的那个事业,大体上已经、已然说清楚了,而且那个路径、方式、方法注定了结果,注定了一个基本的结局。中国在改革开放之后,几乎所有成功的企业家皆受益于《毛选》(四卷),他们是从《毛选》(四卷)中吸取营养,是《毛选》(四卷)提供给他们资本化的方式和方法,这一点真的!

关于资本化的这个事情,我将正面和反面的事情全讲完了。做一点点的概述,就是通常我们谈到资本的时候,请牢牢记住这句话: 资本是意识,资本不是钱,不是货币。一定要记住这句话。资本化是真正的调度一切可以调度的资源,完成你伟大的理想。我们当然可以把它理解为商业利益,创造商业利益,也可以创造价值,但更多时候我们是完成一个伟大的理想,好吗?但是资本化也有可能走向反面,将不该资本化的东西资本化,我们对此要有清醒、冷静的认识。

我讲几句,我这抓紧时间。最后一个环节,就是从资本、资本化,我讲几句资本积累,这里边涉及到一个对经济的一个基本的看法。发展源泉是资本积累率,这个我过去讲过N次资本积累率,资本积累的源泉,可能有货币资本的成分,但也有可能不仅仅是货币资本,而是一个综合型的东西,但资本积累率是非常重要,就是它不断的累积。资本积累率的累积里边必须思考杠杆的意义。大部分的资本家发展不是自己的钱。

当你开始学习经济的时候,那么你就开始懂什么叫价值。因为我一直把价值论放在很高的一个位置上,所以你开始理解空间的意义和时间的意义。当你理解了空间的意义和时间的意义,理解了杠杆的意义,理解了资本积累、积累率,那么开始进行自己的资本化的历程就开始了。在这个伟大的历程上,可能会有许许多多有趣的事情发生。在平台上有很多的朋友本身已经是资本家了,他们是企业主,有些是国有企业的负责同志,他们已经是资本家了,他们在进行这个管理。

其中有一些年轻的、新型的新新人类。资本化的过程或者资本积累的过程是很妙、很绝妙、很神奇的,他们做得非常之好。原本我想作为一种技术做一个总结,在适当的时候或者是将来有空再讲吧。因为今天我们这堂课讲资本、讲资本化或者资本积累的这个技术和方法就无法展开了。我也希望你们顺着这样一个思路,去往深处去、往远处去做思考。因为人生一世,不一定非要被雇佣嘛,我们可以考虑自己主体性的问题了。

在讨论资本的这堂课的最后一个环节,我想简述一下子剥削的问题。只要存在资本,就存在雇佣,不过不管你是雇佣别人还是被别人雇佣,都存在一个剥削的问题。只是这个有一个均衡,就是一个合理值叫均衡点合理值的问题。如果你是一个雇佣者,你是个资本家,那么你应该知道那个均衡点在哪里。如果你是个被雇佣者,你不了解你作为商品的定价是什么吗?你不了解你劳动定价吗?你不了解你持有资产的定价吗?

重复一遍,合理的劳动定价、合理的商品定价、合理的资产定价,是寻求雇佣和被雇佣的一个价格均衡点的一个过程。我们是拥有主体性的人,我们必须对我们所生存、生长的一个环境有基本的认识,这还不足够,因为当你们听完我的课的时候,你们应该对你们的孩子有一个交代,就是他们生长的环境,他们应该知道他们处在一个什么样的状态。我不是说被雇佣一定不好,有好多人不适合做资本家去雇佣别人,我不是说被雇佣,就一定不好,我不是这个意思。

我只是想跟大家说,有一个合理的定价,有一个均衡的定价区间,你要懂得这一个东西在一个合理定价区间里边就可以了。记住今天讲资本的两件事情,非常重要的两条:第一个是资本化必须有边界,不能什么都资本化,因为你不懂资本化的边界,你必然成为罪犯,你必然走向历史的反面,就算不会成为罪犯,你也会成为魔鬼,不管是被法律追究还是被苍天追究,它都是不妥当的。好吧,一定要注意自己就划定边界,不要等别人给划定边界,因为我们有主体性嘛,我们是人。

第二个就是不管雇佣还是被雇佣,要找合理定价的均衡点,实现自己的价值,也给别人实现价值的那样一个条件和机会,这样就比较妥当了。我想关于剥削的这个事情我也不想展开说了,现在社会上对这个事情有诸多诸多的争论、诸多的争议、各种各样的争议都有,看你站在谁的立场上来说,是站在雇佣者的立场,还是被雇佣者的角度来看,但你能找到一个恰当的均衡点,那么你就是拥有主体性的那个人了。

好,这件事情就讲这么多了,我想讲点什么呢?就是这回这个十次会议,对。十次会议强调两件事情:一件事情是金融风险,一件事是共同富裕。今天上午还接受了一个专访,谈这件事情。金融风险,我想平台上的人是最清楚的,因为我们讲过平成战败、讲过天降、讲过广场协议,所以金融风险,平台上的朋友是非常清楚风险是什么,风险在哪里?我们在此不再重复,我们想说的是共同富裕。共同富裕是个什么意思呢?这里边实际上涉及到当下比较残酷的现实。

这里边有两个现实是非常残酷的,一个是资本积累率不够。资本积累率不够,所以有关的专家、学者或者主流的专家、学者,建议实施更为积极的财政政策和更为宽松的货币政策。一句话,印钱、扩赤字,这个实际上是对穷人不友好的行为,这个是饮鸩止渴,它不合适。但它对资本家友好,甚至在某种意义上,可以帮助美国完成美元价值的确认,是主流的。一个主流的看法,其实主流就是首席们的看法。

在这里,我要给本届政府,我们现在的党中央点赞,他们非常了不起,在重大问题上看得透彻,所以提出了共同富裕。共同富裕的意思就是打算进行再分配的调节,再分配的调节就是直接税,这件事情本平台的朋友也是非常清楚的,因为我们也讲了很多年了。其实我们现在讲《资本论》也是这个意思、也是这个意思,毕竟七月份的数据出来了,七月份的数据出来,社会商品零售的这个下调,非常惊人,不能算断崖,但是下得很厉害,就是老百姓没钱买东西。

老百姓没钱买东西或者是不愿意买东西,投资不足、消费不足,上半年依靠出口拉动一下子,下半年出口可能也会遇到问题。所以现在有些学者,你们上网就看到了,著名的,就开始嚷嚷了。但我觉得这回中央的整个的整体思路是非常明确的,就是要共同富裕,共同富裕的意思就是要让那些高净值、高收入人群做出某种的奉献。今天采访我的时候,问我:“共同富裕是不是又要走这种老路?老路,就是这个劫富济贫的老路”。

共同富裕是政治问题,也是经济问题 https://mp.weixin.qq.com/s/a8XRbTEFRSsaljYaoevAOQ

我是这样说的,我说没有劫富济贫,今天形成的高净值,不是高净值通过诚实劳动赚取了一百万亿人民币的财富,而是由于过去二十年,特别是2008年前后,形成的错误的财政和金融政策,导致人、国家、人民群众用极高的杠杆推升了一部分资产的价格,形成了这部分资产持有人的高净值,这是一个错误的制度和政策安排的结果,没有创造价值的问题,调回来是对的。

所以我说这里边有两个方面需要理解,第一个要理解浮财,百万亿浮财它是不合适的制度和政策导致的一个扭曲现象,这个扭曲现象应该得到调节,而不应该让这部分浮财跑掉,导致国家的资本积累率狂降。资本积累率狂降、投资不足、消费不足,使经济出现严重问题,这是错的。正确做法就是共同富裕,共同富裕非常好,这是第一条。第二条其实也应该给现在目前的获益者、高净值们一个投身社会、为国家、为人民做贡献的机会,所以我一向主张是通过二次分配,解决整个社会问题。

有些朋友问我说你为什么不主张三次分配呢?我看了有一次分配、二次分配、三次分配。一次分配是市场分配形成的,由于市场被特定时期的政策制定者给扭曲了,它导致一小部分的人成为高净值,他们是获益者,而国家和老百姓、人民群众背负了杠杆。就是大家集资让一小部分人富了,不是先富,是集资让一些人富了。那么这一件事情不合适,为此做一点修正,这是对的。我们一定要认识到,你这个社会上是没有股神、没有超人的,它一定是哪错了。错了的部分纠正过来,让国家走上正轨,这是非常合适的,但是,没有三次的问题。

只有一次分配和再分配,三次分配——所谓的慈善,我写过一篇长文《慈善?斯恶也!》\footnote{https://www.notion.so/3ea858a5cc154cc7b841d6e643d232a0} 。谁,如果是你讨吃、要饭赚点钱,去救助一个更为饥饿的人,那叫慈善。你通过制度扭曲赚了的钱你还来慈善,斯恶也!不合适。所以我对三次分配论嗤之以鼻,我不喜欢这样说法。因为这个本身就是想模糊一些问题、模糊一些人、模糊一些行为的一些看法,甚至我反对以慈善的名义为某些特定的人和机构减免税赋,反对。

好吧,我坚决支持十次会议的精神。这两件事情都非常重要,一个是防范金融风险,第二个是共同富裕。这两件事情及时、精准,现在只是希望快速到位就可以了。最后再说几句目前的状况,其实在技术上可能美股气势如虹,收不了手,但无论如何它已经走到极致了,它需要的是一个说法。这个说法可能只能由中国人来提供了,就是你那个货币是假的。

美元是伪钞,这句话总得有个人说出来吧。如果美元是伪钞,建立在伪钞之上的资产价格神话没有意义,我们必须为伪钞及伪钞基础上建立的价格神话而做出崩溃之前的最后的准备。这个时间节点不会太久远了,我们在防范金融风险,我们在走共同富裕的道路。我们的共同富裕是跟我们14亿人民中的贫穷而勤奋的人民共同富裕,我们不打算跟用伪钞累积资产泡沫的人共同富裕。

今天在讲《资本论》的时候最后这个结论可能也让大家有点震撼。我再重复一遍,美元是伪钞、是假币,用伪钞垒起来的资产泡沫、资产价格神话必然破灭,毫无悬念,必然破灭,只是一个时间了。我们所有的人因为这个事情破灭做最后的打算,做最后的安排。不要再听那些个崇美的、没脑子的、没品德的那些人、那些专家学者胡说八道,,不要再听下去了。

关于美国的问题,我想接近尾声了,这个美军在阿富汗已经证明了他了。美元可能正在迎接它的阿富汗时刻,或者它的西贡时刻。美元正在接近这种撤退和崩溃的时刻,只是美媒依旧是如此的强大,尤其这个美媒在中国如此的强大,竟然可以控制类似于像我这样人的言论,连说话都受到了严格的管控,真是莫名其妙,真的是神奇无比。

不过,我觉得这个历史就是这样写下的,我怎么看我都觉得拜登是个悲剧人物,他有机会处理一些事情,阿富汗问题是要处理,可以慢慢关电源非要拔插销。美元的问题依旧是如此,可以想办法让它慢慢地冷却,他非要把它推到如此之高、之险峻的地步,等着迎接那个风暴的到来。至于美媒,我不想说什么了,因为再说平台就没了,我们就不说了。

最后,今天课程结束,我是希望大家仍然要注意自己的身体健康,注意卫生、注意安全保护。再次提醒,再次提醒,美元以及美元所带来的整个的一场惨烈的金融危机,因为伪钞和伪钞定义的那个资产价格泡沫会崩掉的,离我们越来越近了,我们要为这一刻做好万全的准备,不要再听那些人的胡扯八扯,不要听他们的胡说八道。好吧,今天就这样,明天下午3点钟见,好,再见。

\section{资本积累的秘密、变革、共同富裕、美国的非农数据以及美国经济遇到了什么}

大家好,今天是2021年的9月4号,是辛丑年的七月二十八日。今天是正式课,我们讲《资本论》的第十一讲:资本积累的秘密。今天这堂课重要,正好结合一下子最近的一些的话题,主要是“共同富裕”这个主题,“变革、共同富裕”这个主题。另外,时间够的话,念叨几句美国的非农数据以及美国经济遇到了什么?好,我们三点钟准时开始,一会儿见。

大家好,今天是2021年的9月4号,是辛丑年的七月二十八日。眼瞅着农历就该进入八月了,其实从我个人的角度,带着封建迷信色彩的角度,我们是期待着这个七、八月份,农历的七、八月份尽快地过去,因为真正的考验应该是来自于农历的九月,也就是我们阳历的十月会面临巨大的考验。

这个山火贲的这个贲卦,可能揭终的时间应该就是在阳历的十月和十一月了,它就应该是把谜底呈现出来的这样一个时候。本来按照我们的教程,《资本论》的第11讲应该开始讲剩余价值。我给你们念叨一下子《资本论》第一卷的编排的逻辑。但我想来想去,做了调整,因为《资本论》的第一篇是商品货币,第二篇是货币转化为资本,第三篇是绝对剩余价值,第四篇是相对剩余价值,

第五篇是绝对剩余价值和相对剩余价值的生产,第六篇是工资,第七篇是资本积累、资本积累过程。我们的这个讲座逻辑过程是这样的,我们的第一讲是两次革命和两位宰相;第二讲是马克思主义哲学的源泉,是四位哲学家;第三讲是马克思生平;第四讲是马克思主义的东西两线,前四讲实际上是对马克思和《资本论》做铺垫;第五讲是《资本论》的目的方法;第六讲是价值论和价值观;第七讲是价值尺度、货币;第八讲是劳动;第九讲是商品;第十讲、上一讲是资本与资本化。

按照我们的这个逻辑过程,其实应该进入到剩余价值的部分。北京的朋友这两天在催我,说你能不能先直接进入到《资本论》第二卷,就是资本流转。因为当下中国的现实需要讨论一下资本流转。就是由于资本从实体经济中溢出,导致中国的去工业化进程加速,现在中央在往回调这件事情。就是还是要重新发展实体经济。那么资本流转的逻辑就变得非常重要,而且为什么会出现资本的逆流转的逻辑,说能不能讲一下《资本论》第二卷资本流转?我思虑再三,也不能跨度太大,跨度太大也不行,

所以我还是决定把资本积累这堂课讲完。因为我觉得《资本论》第一卷的最重要的部分就是资本积累。我们今天是讲资本积累的秘密,这个资本积累的秘密算是大家学习宏观经济学里边的,今天算是开窍的一堂课。就是你听完这堂课,你终于知道资本积累是怎么回事儿了。就是其实了解了资本积累,每一个人都能做资本家。另外,就是看明白资本积累的秘密,你就知道制度和政策上存在的问题。大体上资本积累的秘密已经快要说清楚资本流转的逻辑了,它里边是一个结构性的东西,其实已经互相牵扯到了。

最近有两个人的两篇东西引起社会广泛的关注。一个是李光满的那篇文章,他是说“深刻的变革”,一场深刻的变革;一个是胡锡进的文章:“那不是革命,就是一场深刻的变革,那毕竟不是革命”。没有那么严重,也不可能偏离邓小平改革开放的这个路线和方针,就是还是要改革开放,就是共同富裕并不是否定改革开放的这个路线,产生了激烈的争论和对峙。激烈的争论和对峙,通常我们这样的一个平台或者是我们有信息研究基础的人,我们还是喜欢用算法来说话。

每个人都能感受到,一场深刻的变革正在进行! https://mp.weixin.qq.com/s/j09xDJOnmhoXDhqytGiKzQ

胡锡进:宣扬中国正在发生"深刻的革命",这是误判和误导 https://mp.weixin.qq.com/s/FyfV7yc\_I\_bzohC1zXI7fA

其实在资本流转或者是在资本积累的这个具体的详细测算中,我们可以给予所有关心的人以一个基本的答案,一个基本的答案。如果我们计算了郑爽的年收入,那么郑爽的年收入是中国劳动者平均收入的2000、好像是2400多倍,那么是中国月收入在1000块钱的这样的六亿人,是他们的数万倍。那么数万倍就是一个普通劳动者要干几万年才能干一年的郑爽的收入。

而那个月收入在10000块钱的人已经算不错了,超过了大部分的公务员的收入了。那么也要从春秋战国开始工作干到现在,2000多年才能追上郑爽一年的收入。这样的事情是不是合适呢?我们就要讲资本积累和资本利得和资本流转的这样的一个秘密了。我是非常赞同目前习主席提出来的“共同富裕”这样的一个思路,因为它的意义不在于社会分配的合理性,它的意义在于资本的合理分布。就是资本应该在它合适的地方,这样国家、民族的长远发展才有保障,这跟改革开放的方针、政策没关系。

李光满的文章有一种情怀,就是情怀过重,就是理想和情怀过重;缺少经济学理论的扎实的叙述和论证,就是看法多,算法没有。胡锡进的文章,问题非常严重。因为不光是老胡了,还有一些大家熟悉的人、名人,甚至有些人具有左翼色彩,他们为什么会站在党中央和人民的对立面上去了呢?因为他们曲解了变革。深刻的变革或者是变法,它并非简单地再次革命或者是土改,不是这个意思。所以在聊天上就聊出了巨大的问题。

我禁言了,所以我也不打算说这些事情。我想,这个周末会写一篇文章,可能得放到下下周的《亚洲周刊》来发,谈谈共同富裕的事情。原本现在关于共同富裕的事情,我是接受了凤凰的一个专访,但是好像国内审核通不过,又通不过了。就是我的那些在B站上的视频、南南论坛上的讲座和大部分的文章现在都有问题了,也不知道该说什么好。因为现在处在一个重要的转折关头,可能也是对我的保护吧,给我禁言,不让说话,可能也是对我的某种程度的保护。我从内心深处并无怨恨,甚至偶有感激。

好,我们先进入到今天的讲座。讲座完成之后,我们谈一下美国的经济,这个背后的美国经济的问题也非常得有趣,又非常地重要。今天可能时间稍长一点,占用大家周末时间啦。《资本论》的第十一讲,资本积累的秘密。在讨论资本积累的秘密之前,我们必须对资本进行严格的分类。资本的分类其实在西方经济学里变得不重要,或者它不主张进行分类;在我学的苏联教科书里边,资本的分类倒是非常得细致。通常我们把它分成农业资本、工业资本、商业资本和金融资本四大类。当然,这里边还可以细分。为什么要分类?

因为,当资本在不同的产业里边的资本利得差距是巨大的。就是你投了一万块钱在农业里边,你的回报可能是个位数,甚至是负数;你投了资本一万块钱在工业资本里边,可能能是10\%到20\%;你投了一万块钱在商业资本里边,可能回报是100\%;你投了一万块钱在金融资本里边,回报可能是1000\%、10000\%。就是不同产业的资本利得差异巨大,资本分类的意义就在于这个地方。你如果懂得资本分类,你就知道国家的制度和政策多么的重要。因为你制度和政策不能跟上的话,有些产业必然消亡。

不同的产业资本类型,它的资本构成也是不一样的。通常农业资本一般都是自有资金为主。农民也借不着钱,也不需要借那么多钱,可能也借不着钱;工业资本,就不能是自有资金为主体了,自有资金只能是一个引子,它需要进行集资;商业资本,自有资本的量就更低了,它主要是以融资、以信用证为主体;金融资本呢,就没有自有资本,因为它是一个游戏来的。不同的产业资本具有不同的资本利得和结构。

另外呢,我们也要了解一下,除了资本分类以后,也要了解资本的构成。很多人不知道什么是资本,资本包括以下部分:资本首先要包括你的本金的部分,其次要加杠杆,再其次要加利得,它完整的结构构成了一个资本,总称叫资本。在现代社会,这个本金的意义就变得越来越不重要,除非你是做农业和工业,可能本金需要,而做其他的行业,特别是在数字经济时代,很多人就没有资本。就是一个好的Idea,就变成了一个大资本家,像马云、马化腾他们。这个使得资本的构成让我们有崭新的认识。

资本的构成,它是一个很虚妄的东西。什么意思呢?如果,你没有哲学高度,你对时间和空间是没有感觉的,你理解的资本是静态的,那么你就无法理解赚钱是怎么回事。如果你理解了动态的资本,就是加了时间轴,加了时间轴的时候,由第三维进入到第四维的时候,你一眼就看到那个人赚的钱是我的。为什么是我的?就是我们买房,我们应该买的是什么时候的房?但我们却从银行加了杠杆、去做了透支,去买了这套房,然后构成某个人的利益、利润、资本利得。

这个时候你就知道,某些资本——就是由杠杆构成的资本,形成了特定人的资本利得。他投入的本金没有多少,这就是王健林所说的那个“挣一亿的小目标”,他们完全忽略了它是资本转化为虚妄的资本利得、虚妄的收益。因为那原本就是杠杆,只不过是因为通过金融技术、金融操作变成了一小部分人的资本利得,就变成了小部分高净值人的资本利得,而他们并没有创造价值或者劳动所得,他们创造的是制度和政策。也就是说制度和政策出了问题,所以导致其实是资本变成了别人的资本利得。

说到这里,其实我认为《资本论》非常重要,其实马克思非常伟大,就是马克思确实了不起。整个在《资本论》里边,通过对资本的深刻的剖析和理解,当然了,你不要把目光集中于剩余价值,你要研究马克思相对剩余价值和绝对剩余价值的方法。越过马克思对那个时代的论述,用他的逻辑和方法,对今天的资本结构、类型进行分析,你会得出惊人的结论,为什么要共同富裕?为什么必须通过制度设计和政策导向来实现共同富裕?因为,我们过去的制度安排和政策导向是错的,我们只是在改错,我们并没有否定改革开放的方针政策,没有。

加上空间的这个轴呢,我们就开始能够理解在不同的结构之间、不同的区域、不同的产业结构之间的资本利得的这种变化,它是一种疯狂的变化。比如说,以DATA为载体的资本和以砖头为载体的资本和以农田为载体的资本和以工厂为载体的资本,他们的回报差异是巨大的。这个巨大的差异源自于哪里?不是源自于劳动的差异,就是我的脑力劳动是你的脑力劳动的一万倍,郑爽的脑力劳动是普通劳动者的十万倍。不是这个意思,是错误的制度设计和政策安排导致了巨大的差异。

我们要谈一下子,我按照我的大纲说吧,第一个部分是资本积累的分类,第二个是资本的构成,第三个是中国式的资本积累。中国式的资本积累和西方的资本积累是不一样的,中国人的第一桶金是化生出来的。它不是胎生,不是卵生,它也不是种子生,它是化生。化生的意思什么来?就是偷来的。所以中国的第一桶金呢,主体上是从国有企业转移出来的,就是国企改制啊什么过程转移出来的;第二个部分是由外资带入的;第三个部分是集体集资的或者是集体资产转移,所谓机关办企业、乡镇企业;

第四个才是正常的资本主义国家出现的,叫私人集资,在我们国家排第四。第五个才是通过金融机构,由投资银行提供资本来创业。这个在我们2008年之后的新经济里边比较多了,就是风投之类的东西。中国资本积累的第一桶金结构上是有问题的,所以才遭遇到现在共同富裕的提法上的时候,大家为什么反对直接税?就是我那个钱来得不对——灰的、黑的。现在呢,给你一个解放的机会,让你证明;他不证明,我就逃走;我离岸了,我走了;潘石屹走了,其他的还在的那些人,人在国内、钱在国外。

中国式的资本积累由于是扭曲的,所以它有一个还原的过程,是重新不扭曲。而完成重新不扭曲,所谓的完成共同富裕必须税政改革。李光满没说清楚,他说是深刻的变革,这个深刻的变革的核心就是税政改革。胡锡进反的不是深刻的变革,他反对的是税政改革。他认为税政改革会导致重新财富的再次分配,他认为这是打土豪、分田地。因为其实老胡和一大部分的知识分子有的时候表演出来的时候很左,但是他们已经是有产阶级,或者是他们的朋友们或他们的亲人们都是有产阶级,所以他们处在一个一不小心就把立场亮出来的一个特殊……

那么如果你说,谁创造了中国的资产阶级?由于资本积累出现了大量的资产阶级和规模庞大的资产阶级的资产,那么创造它的源泉是两个:一个是国有、国家和集体所有制企业的财产,这是一个部分;第二个部分就是庞大的居民储蓄。这两个是中国资本积累的源泉。如果你有哲学修为,你拉长时间轴,你进行空间的跳跃,你就看到了资本积累只源于这两样东西,其中杠杆的部分是最近这15年或者是更长一点吧。

可以更准确的一点讲,差不多是20年吧。就是从十七大开始了,这差不多20年时间进行的一个比较夸张的一个过程。这个过程就是通过加杠杆的这个状况,通过杠杆的资本化,然后资本转化为少数人的资本利得这样一个神奇的过程形成了中国的扭曲的资本积累和中国的扭曲的资产阶级。一个特定的历史时期,它出现了一个现象。在今天课程开始的时候,

在课程开始的时候,我讲过资本利得的产业分布是不同的,这个利得的制度设计和政策导向决定了各产业资本利得的巨大差异。那么好多人要问:“到底是谁在制定制度和政策?那么是谁把这个倾斜创造出来了,并且形成了今天这个局面?那么现在谁在改善它,把它扳回来?”如果我们懂了这个东西,那么我们对历史和历史人物就会有一个非常客观、公正的评价。在最需要我写文章的时候,他们关闭了我,那么我也只好保持沉默。我相信,

我相信即便是沉默,那么整个的这个国民的整体的认识水平也会在今年下半年,在这个贲卦,在这个山火贲,在山火烧到山头的时候,让一切还原它的本来面目,现出原形。这是不可避免的,这是不可避免的事情。真正构成资本利得的东西(我说的是这20年)是什么呢?是土地、是环境、是劳动力、是资产定价。其中马克思说的剩余价值主要是在劳动力里边,但我把它排在第三——土地、环境、劳动力和这个定价。

我记得我前面讲过定价理论,我就不在这个地方说了。因为我们如果再重回定价理论,那就会出现严重的问题,不是严重的问题,就是时间不够。你这只要记住定价为什么不合理,你就知道……比如说劳动定价,那郑爽的劳动定价和诸位的劳动定价差了数千倍,为什么?这个定价为什么会出现如此程度的扭曲?这个就是我们要思考的。因为这个定价,劳动定价形成了巨大的不同水平的资本利得,那么是谁形成的这个定价,定价逻辑?那么还又要回到制度和政策上面来看,到底是谁干的?为什么要这样搞?

好吧,我想在资本积累这个地方呢,我想说一下子一个关系。大家可以把这个关系记下来,有时间做深刻的思考。我们要开始熟悉并了解不同产业资本、资本利得的关系。我们要开始了解不同的劳动所得的关系。比如说,郑爽和我们之间的劳动所得为什么会出现如此巨大的差距、差异?这个定价合理还是不合理?不合理是在哪里?为什么?就是不同产业资本定价那么差异大?那么为什么会这样?劳动所得之间差异大,为什么会这样?

第三个关系是资本利得与劳动所得之间的关系。为什么资本利得远远高于劳动所得?那么作为一个好的国家、一个伟大的国家、一个文明的国家用什么方法来遏制和抑制资本利得,使之与劳动所得平衡?能做到吗?如果能做到让资本利得与劳动所得实现某种平衡,那么这个国家一定会成为发展的速度最快最好,一定会成为一个最和平宁静的国家,也是最文明的国家、最伟大的国家、最强大的国家。其实管理资本利得与劳动所得才表达出国家治理的水平。

我为什么高度评价总书记和这一届的政府?就是他们意识到不同产业资本的资本利得差异太大了,不同劳动所得之间的差异太大了,资本利得与劳动所得的差异太大了,必须进行有效的调试。用一句话来说就是共同富裕;用经济学的术语来描述就是让资本、资源形成正态分布,让生产力达到最大效能;如果用李光满的话来说就是深刻的变革;如果用胡锡进的话说就是那就是一场再一次的土地革命,怎么说都行。

其实我们大体上可以得出一个结论了:就是在制度和政策上,我国必须调节资本利得,使之与劳动所得达至某种平衡。就是资本利得不能太高,劳动所得也不能太低,差不多才行。第二,必须调节劳动所得之间的差异。差数万倍、数十万倍,过了。你创造的价值没那么大,你是在抢、在掠夺、在剥削,这不好。一个小姑娘到这个程度,这不好。当然了,她只是一个手套而已,但是真的不好。

第三,我们必须限制资本的流动。不是不让资本自由流动,是限制资本的跨境流动。我们过些日子再讨论《资本论》的流转,资本流转的时候,我们要讨论哲学的问题了。资本的空间流转合不合理,就是资本的跨境流转的问题合不合理?为什么必须进行有效的资本管制?就是资本的时间、跨时间流动合不合理?我刚才讲了,原本,什么叫供楼啊?原本是我们的杠杆、我们的资本怎么就突然变成李先生的资本利得了呢?这里边有一个非常巧妙的制度陷阱和政策陷阱在里边,它是非常非常糟糕、危险、可恨,甚至充满罪恶。

我们今天的这个资本积累的秘密的第四个部分要讲什么是良性的资本积累?良性的资本积累的基本定义是什么?本来按照马克思的意思,资本积累是源于劳动者的剩余价值,但是在现代归根结底还是我们劳动者的剩余价值或者劳动者创造的价值,归根结底还是。但在技术上它不这样表达了而且它压缩了时间,就是通过杠杆它把资本积累的时间压缩了。另外通过不合理定价,它把属于我们的自然环境,属于我们的其他的资源也同时掠夺去了。

那么什么叫良性的资本积累?就是良性的资本积累必须要照顾自然,要与大自然友善,不能对自然资源进行掠夺。第二个良性的一个重要的标志,就是资本与劳动的关系必须是平衡的,资本利得不能过多的剥夺劳动所得。它不是一定要低于劳动所得,它是中间取得一个平衡,如果这个平衡被破坏了,那么这个问题就变得非常之严重。第三个是劳动所得之间的一种平衡。倒不一定说像我们建国初期那样的官兵一致。

我们不是要绝对的平均主义,我们是要的是绝对的不平均,我们反对的也是绝对的不平均主义。我们既不要绝对平均主义,我们也反对绝对的不平均主义。太绝对了,绝对几万倍那可就过了。实际上如果我们不能建立良性的资本积累,那意味着治理的失败。也就是说在改革开放的前二十年,从1979年到1999年,如果我们那个时候还没有表达这么充分的话,在后二十年,我们出现了治理上的问题,它甚至即将导致中国走向一种非常危险的境地,因为治理失败而导致国家衰亡的这种倾向可能会出现。

如果这场深刻的变革在2022年开始,并能够在未来的五年到十年之内顺利完成,那么我们可以这样说:习主席及其领导的党中央再一次挽救了党、挽救了国家。这个说法并不为过。因为如果我们任由恶性的资本积累发展下去,它必然形成对党和国家的吞噬,它已经开始在吞噬人民了,最终会形成对组织和国家的一种反噬,最后使国家陷入到一个非常糟糕的境地。一会儿我们在结束的时候会讲到美国,我们会让你们看看美国遇到了什么样的问题。

当我们给了良性的资本积累一个定义的时候,我们必须给良性的资本积累一个算法。这个算法决定我们的一个标准、决定着我们的一个方向,或者决定我们政策、制度的这样的一个依据。到底与环境友善到什么程度?这个呢我觉得与环境友善在碳达峰和碳中和的这个安排上,大体上给了一个中国式的答复或者中国人对历史和未来的一个交代,这个我看计算起来相对容易。第二个部分就是资本利得与劳动所得平衡的这个关系,我想这个计算,我会在我的文章里边将来提出一个计算的公式和一个计算的方法,而且我们有一个渐进的逻辑。

第三个部分就是劳动所得之间的差异。这件事情,我不认为一个人的工资收入不可以超过另一个人的一百倍,我不认为,超过一百倍也合理。但是超过一千倍和一万倍的时候,如果你没有一个调节机制的话,你如果认为他的劳动是另外一人劳动的一万倍,你真的这样认为?比如说你认为郑爽是另外一个女孩辛辛苦苦劳作,劳作一天的一个女同学的一万倍,你认为这是合适的话,那我就不好说什么了;如果你认为不合适,而事实又这样出现了,那我们为什么不对他们的所得进行调节呢?

我一直认为直接税非常非常关键、非常非常重要。就是劳动定价的事情是由市场来决定的,但是市场弄错了的时候,我们必须有一只手把它调回来。怎样调,其实我们在税政改革里边已经谈了。我这回在岭南大学南南论坛上这个《税改 —— 不流血的革命》\footnote{https://www.notion.so/453e603eb2114916adc0b3ce5c379848},这两个小时的这个讲话现在发不出来了。B站不给发,发不出来了。我其实把逻辑、方法、甚至计算方法全部说清楚了。我也很痛苦,我也很痛苦,现在是没有办法,因为大家现在实际上不是一个人、两个人,几乎是全部,你有没有感受到,你遇到99\%的反对的时候的感觉。

几乎所有的教育、学术、传媒都认为你有问题。其实老胡表达的不是他个人的情绪,是他表达的整个所有的机构,而且他们有的时候还是以左翼出现的。就是不让你搞直接税,这个深刻的变革不让你动。因为满嘴吃猪油吃得正香,你突然提出来这个东西,他就会把它归纳为……我也不知道为什么老胡竟然认为土改、革命是贬义词?不知道为什么?是怎么了呢?为什么这变成贬义词了呢?难道天理、正义、道德、良政在老胡那也都是贬义词吗?中国这是怎么了呢?

依靠市场的自然逻辑而形成的资本积累,必然无法良性,一个良性的资本积累,必须有一套非常良性的制度和政策安排。共同富裕的意思,就是通过良性的制度设计、良性的政策安排,导致资源的最佳分配。这个资源的分配,包括了劳动所得的一个舒适合理的分配状况。这样,大家就心情愉快,既可以做劳动力再生产的提升,又可以使整体社会资源最好的布局,达至生产力的最有效水平。这不正是我们追求的吗?

美国人一直在谈民主,美国人从不谈良政。没有民主可以有良政,没有良政他一定没有民主。就是它两个的关系是:没有民主,它是可以有良政的。无论是李光耀治下的新加坡,还是蒋经国治下的台湾,都不差。但是,有了民主可不一定有良政啊!印度,那是良政吗?今天的美国,那是良政吗?但是有了良政就一定会有民主,就是良政之下,人的权利得到了基本的表达。你能说给八亿人脱贫的这个政,不是良政吗?你能说八亿人脱贫,那不叫民主吗?

所以在考虑资本积累的时候,在我们深刻研究资本积累的时候,反而能将一些重大的政治问题说清楚了,一些非常现实的政治问题说清楚了。《资本论》了不起的地方,马克思一直要讲政治问题,但是马克思就是不说政治,他就讨论资本。终于这个《资本论》写成之后,在东线爆发了流血的社会主义革命,在西线爆发了深刻的变革。就是李光满说的深刻的变革,就是在《资本论》写好之后,在西线发生的类似于像凯恩斯、类似于像美国的威尔逊总统,他们整个在西方掀起的社会改造、社会主义改造运动。这是一个非常好的事。

在讨论资本积累的章节里,我们讨论的第五个问题是什么呢?就是资本积累的效率定生死,资本积累率定生死、资本积累率的提出。关于资本积累我们讨论过很多次了,我也不在这儿谈计算公式,你们去查计算公式,这里边我有两条要说,第一条:没有他们所说的什么,有很多学者说“中国经济发展的体量够大了,所以速度不能快了”,没有。经济增长的源泉是资本积累率,就是你资本积累率在两位数就超过十,你经济增长就不会低过五,你资本积累率在……

中国大跃进的时候,都出现了严重的经济问题,但是我们的资本积累率在大跃进的时候到了40\%,那我们的经济增长肯定是超过20\%的。它就是这么个逻辑关系,资本积累率太高了是不好的,因为它挤压了我们的生活。在投资、消费和这种出口这三个环节里面,投资太多了,那么消费就会少,那我们要取得一个平衡。那么资本积累率,它实际上是我们在特定的历史阶段,来定义我们个人、我们国家、我们的机构,给我们一个正确认识,就是,你决定,你应该维持一个什么样的资本积累率呢?

在你年轻的时候,你要将资本积累率放得比较高,因为你二十多岁、三十多岁你可以少一点消费、少一点奢侈品,少一点……可以住得小一点的房子,而完善你的生产力工具,让你获得更多的资本积累。也可以考虑,向其他人做集资,也可以考虑利用一定的杠杆来扩大你的资本积累规模和提高你的速度。个人的资本积累率是很重要的,机构的资本积累率,因为我做企业我知道,就是你要考虑,你确定发展方向之后,如何来实现资本规模的扩张,扩张到什么规模可承受。国家的资本积累率也是非常重要的。我要说的是,

我国在2008年开始,这个救助美国的过程中进行了这个资本扩张,资本积累率比较高。而到了2015年的时候,就是美国人在退出扩张,我国配合美国人在退出扩张的时候,我们出现了资本积累率狂降。我们资本积累率狂降的原因,不是仅仅是杠杆降低,而是中国在2015年之后出现了大规模资本外逃。这个就是我们将来讨论资本流转的空间的时候要讨论,就是在资本流转的空间限制问题上管理的含义。当大规模资本外逃出现的时候,我们的经济增长一会儿保8、一会儿保7、一会儿保6,现在后来是保5,你资本积累率跌到0。

你资本积累率都跌到0了,你还想经济高速增长,那是不可能的啦。北大那几块材料,这样说是不是有点羞辱?像张维迎这些人,他们根本就不懂经济学,也不读《资本论》,他们认为经济增长下来是正常的。当然了,像林毅夫这样的学者也是很有趣的嘛,他认为中国经济高速增长,但他也不提资本积累率到什么水平,才能实现高增长,就是没有一个算法,没有一个逻辑过程。资本积累率定生死这件事情,真的是非常重要,因为,这才是国家经济治理里边必须关心的问题,我们处在什么阶段?我们应该维持什么样的资本积累率水平?为了完成这个资本积累率,

我们可能在特定的历史阶段,我们还是需要在方方面面有所节制,必须有所节俭的。有些地方我们不能太快,比如说军事增长就不要太快,没有必要。做一些事情,比如说我们压缩公务员团队,我们让特定的人群再稍微的收敛一些,不要那么大的规模、不要那么富有。总之,如何调整一个合适的资本积累率,考验的是治理水平,这个资本积累率与资本的产业分布、与劳动所得的关系,有着一个算法上的逻辑过程。那么今天我们不详细讲,将来讲资本流转的时候,我们提出完整的算法,这个事情非常重。

那么,什么是资本积累率的悖论呢?资本积累率的悖论简称“割韭菜”。“割韭菜”就是资本积累率的悖论。当一个国家的资本积累由高速增长(我们很快就会看到韩国和台湾了)、大量的资本涌入,形成了某些产业,比如说芯片,好多产业发展得非常好,发展到一定程度,一旦形成饱和,或者是有第三方替代,比如说,突然有一天中国开始能生产芯片,并且把芯片白菜化,完了,那么这个时候,资本会迅速从这些国家退出,而这些国家是没有资本管制的,它就会出现资产的暴跌,它不光是芯片的暴跌,是所有资产的暴跌。

跨国金融资本是一种帝国模式,帝国模式它就是一种霸权模式,跨国金融资本就是一个帝国模式,所以它资本进入和资本退出都会形成完美风暴。我国这一年,“山火贲”这一年,我国在制度和政策治理上面出现了一个良好的态势。我们在美国经济即将爆发重大危机之前,我们蹲下来了,我们留出了足够的财政政策空间和货币政策空间,我们在这方面治理上面还是不错的。另外我国在产业资本方面,终于认识到了我国自身存在的问题。

共同富裕,难道只是调个人收入分配吗?当然不是。它最终会导致产业资本由虚拟重回实体,由这个强烈的金融资本生态再重返工业资本,重返商业资本这样一个过程。我要说一下子,我把地产开发商纳入金融资本范畴,我不认为它是工业资本,我把Data、互联网这个商人,以Data为载体的资本也视为金融资本,我不认为它是商业资本,就是我不认为地产商是工业资本,不认为互联网商是商业资本,所以我都把它归入金融资本,它都具有帝国的特征。

对金融资本的管控,对金融资本在国内实行这种帝国操作手段的遏制,这种霸权的遏制,这是非常了不起的,所以我得给这个领导同志最近这段时间的操作点赞。如果你刚才理解了,我们说了地产开发商和互联网独角兽都是金融资本,那么你们就理解了最近对资本的管控是多么的重要,是多么的重要。这个管控的过程中它表达的不仅仅是一种社会主义的原则或者社会主义倾向,它实际上是一种高度的文明,它是一种高度的文明和智慧,它是一种文明。

在这个本次的讲座的最后一个部分,我想捎带讲一下《资本论》第二卷——资本流转。资本的流转,资本的流转分小流转和大流转,小流转是在不同的环节流转,大流转是在不同的产业流转,更大的流转是在空间的流转——国家之间的流转。马克思的《资本论》第二卷和第三卷做了不同层级的资本流转的研究,其实很有价值。我老说马克思没有写第三卷,就第二卷、第三卷我觉得合并起来算一卷。我也是这样安排的,就是我们的前十二讲是讲第一卷,后十二讲,一共二十四讲《资本论》,涉及到的是……

我再重新说一下这段,就是我们大体上安排的一共是二十四讲《资本论》,前二十讲去掉开篇的那四讲,就是讲的都是《资本论》的第一卷,后半部分,还要最后的四讲离开,因为我们要后四讲是回到这个中国和世界的现实。我们大概是用八讲来讲,就是八讲讲第一卷,八讲讲二、三卷,我一再说二、三卷,就是马克思的《资本论》的第二卷,马克思的《资本论》第三卷应该是讲国家与资本的关系,我们在后四讲的时候,我们会讲国家与资本的关系。为什么我们认为共同富裕表达了国家对资本的……

国家对资本治理的一个基本的态度,就是我们本来也要讲国家与资本的关系,就《资本论》第三卷,我们认为列宁写的《国家与革命》作为第三卷的替代物是有历史局限性的,它导致的国家资本主义在国家资本主义的发展历程上,它既创造了苏联式的辉煌、中国式的辉煌,也带来了巨大的问题。苏联由于没有伟大的思想家解决这些问题,所以苏联解体了。中国有伟大的思想家在六十年代就已经看出来《苏联政治经济学教科书》的问题,他们接棒一样地走到今天,今天交给习主席,他们继续来处理国家与资本的关系,就是《资本论》第三卷。现在我们所说的共同富裕和一系列的措施,要做什么?就是要解决国家与资本的关系。

资本的流动里边有它的秘密,貌似资本是逐利的。资本流动是因为不同的空间可能是在生产的环节里边,或者是在产业之间、或者是在国家之间,出现了不同水平的资本利得,所以资本会流向不同的空间,三个层级的空间,这个不同的空间的流动。当然了,我们现在讲的这个流动和我们那个四矩阵的那个流动是两回事,我们讲的那是资本市场的四矩阵,但我们仍然是一种哲学的高度来概述这个事情。要素的流动从来不是市场定价的结果,我们今天把这个作为一个结论讲出来。

我做一个精准的概述,大宗商品的定价从来不是市场交易的结果,要素价格的定价从来不是市场定价的结果。请牢牢记住,资本的流动貌似是市场行为,它的底层逻辑实际上是源于制度设计的失误和宏观经济政策的失败。就是当制度上出了严重的失误——缺陷,而政策又是错了,那么它就会导致资本出现你意想不到的流动。我们今天说出一个基本的资本流动的秘密,

其实也同时指出,我们国家正在经历的危险的转身,这的确是百年不遇、千年不遇的一次转身。什么叫百年不遇呢?十九世纪从1800年到1899年,那个本来应该是中国世纪,但是十八世纪的清廷太糟糕了,那一代的领导同志,就是梳着小辫子的领导同志水平太低,他们不懂资本积累,所以没有完成工业革命,所以十九世纪变成了中国最屈辱的一个世纪,所以中国丢了。十九世纪是中国世纪,没有了,二十世纪就是从1900年到1999年……

二十世纪本应是德国世纪,又碰到个这个糟糕的皇帝威廉二世,他把它又玩丢了,所以就变成美国世纪了。十八世纪是英国世纪,十九世纪是美国世纪,本来是大英帝国和德意志帝国的,那都玩丢了,这回我们会玩丢吗?那就看我们的共同富裕能不能做成。李光满,虽然他写的东西浅了一些,因为他没有深刻的经济学的背景,他也写不出来。他这上有说法、有看法,没算法。但这个说法和看法大体上是方向是对的嘛。至于老胡、以及老胡所代表身后的那些人非常糟糕,值得警觉,因为他们实际上还想走大清皇上、威廉二世那条旧路。

当然了,每个时代有每个时代不同的资本特征:在十九世纪,资本的载体还是农田;后来二十世纪是工业股权,现在是Data 。那么资本的载体在变,资本的流转形式在变、结构在变,但是千古不变的就是好的制度和好的政策,千古不变。那么今天的中国人睁开眼睛看世界,睁开眼睛看历史,当我们登上哲学的高度来重新审视这一切的时候,我们知道二十一世纪中国人不会丢掉。这是我们的世纪,我们必须把它拿回来。

做一个小结,今天这个课非常重要,也不是特别累。这个我们今天讲资本积累的秘密,讲了资本的分类、讲了资本的构成、讲了中国式资本积累、讲了良性的资本积累、讲了资本积累率、讲了资本流转的秘密,大体上我们把资本积累这件事就说清楚了。其实资本积累在马克思的《资本论》的第一卷是最后一篇,最后一篇的篇幅很大,但我们压缩成一堂课。可能将来我们还会涉及到这堂课的内容,但我想我已经把它概述清楚了。

剩下一点时间,我们念叨一下子,美国昨天非农数据出来了,吓坏了大家。因为非农数据预期增加就业应是73万人,实际增加就业23万人,丢了50万人,美国的就业状况急转直下。有没有非农数据之前就预见此事的这个数据呢?有的。高盛在本月初就已经提出来,可能是去,好像八月底九月初就已经提出来,美国经济的第三季度,现在七、八月已经过了,还剩下一个9月没到,第三季度的经济增长已经从经济增长……

诶,我好像又摁得过梭了。好,我再重复一遍,我说在这个非农数据出来之前,实际上高盛已经将美国经济增长从GDP增长9\%第三季,下调为5.5\%,这个下调幅度吓到我。那么就是高盛其实已经早就知道这个非农就业数据实际上会带来的一个基本结果。经济增长大幅度下调,而非农就业不足。非农就业不足在经济学上意味着什么呢?意味着恶性通胀即将到来。就业不足,美国是以服务为主体的国家,就意味着它提供的服务不够。商品不够、服务不够,那么就涨价。

{\kaishu  "经济新闻报道的是全国产品服务的生产总值 (流量),而不是整个全国创造这些产品服务的工厂农场和公司的物理资本 (存量)。但是如果没有看到存量如何通过反馈机制影响相关的流量,一个人无法理解经济系统的动力学和相关行为表现背后的原因。" }

{\kaishu - Donella Meadows}

昨天晚上好多朋友睡不着觉,这个时候发微信给我,想聊会儿。其实我昨天晚上有点累,但是聊就聊吧。那个北京的朋友有的时候他们急切,大家想知道这个就业出现严重的问题是结构性的呢?还是一个非结构性的?所谓结构性的就是它是一个必然结果,就跟疫情没关系,它是个必然结果。如果不是结构性的,是临时性的,或者是短期的,就是你受疫情影响,所以就业出现了问题,那么就是这个数据还可以再还原。我个人认为是个结构性的结果,疫情就算结束了,就业状况也不会好转。

什么原因呢?实际上就是我们今天在讲这堂课的时候,也涉及到今天的两个问题,额外的话题都与今天的课有关,这个共同富裕跟资本积累有关,这个美国经济增长跟资本积累有关。美国的资本积累严重地扭曲了。就是等到我们讲《资本论》第二卷,讲资本流转的时候,我会把美国的资本的流转在时间上一百年资本怎么流转的,从哪些行业进入哪些行业,从哪些国家进入哪些国家,就是三个环节、三个层级的空间的变动,我到时候把它列出来给大家看。你看到一个帝国的衰亡,其实是那个帝国的血液,那个资本流转就看到了。

美国现在资本的构成和它的分布,根本不接受新增就业人口的,这是最简单的结论。你设想一下子,如果中国还持续2015年的政策,所谓供给侧结构性改革,消灭中小企业,然后全部进入房地产、进入到Data、进入到这种虚拟经济里边去,我们也不需要就业啊,我们需要将博士和硕士都变成送外卖。一个国家没有完整的产业政策、产业链和产业生态,怎么增加就业呢?美国面临的就是这么简单的问题,它是结构性的必然的结果,跟疫情没有必然联系。

说到这里,我们两个东西该结合起来看一下了。如果美国的GDP第三季度掉到5.5\%,那么第四季度呢?如果它还要下,看这个样子,Delta这个不是越来越好,可能还有越来越严重。记得我今天开课的时候我说了嘛,十月和十一月份才是山火贲卦烧得最旺的时候,因为烧到山顶了嘛。那个地火终于烧到了山顶了,山火贲。去年是地火明夷,我看去年地火明夷,中国是真的做到了明夷了。美国没有明夷啊,你看拜登老头这糊涂。

所以今年山火贲,山下的火终于烧到山头了、烧到山顶了,所有的粉饰、装饰、掩盖全部烧穿,露出本来面目,十月和十一月份。那么经济增长进一步下滑,叫经济停滞;通货膨胀进一步上涨,叫通货膨胀。那么滞胀的格局,今天,今天九月四号我们已经可以给定义了,可以给准确地精确的答复,就是滞胀已经在美国形成了。这种滞胀可能会迅速地恶化和变得极为严重。滞胀已经开始,这个判断其实非常重要,这个对我们处理投资意义重大。

滞胀环境下,美国的资本市场和美国的楼市会出现一个什么样的情况呢?我们认为在适当的时候,美国的整个地由流动性构成的泡沫、资产泡沫,将因流动速度的衰减,记着费雪定律,流动性等于M乘以V,M不断地增大,V就是流速在减慢,如果这个V减慢减慢再减慢,而M不能再增大,那么P和Q的P就是商品的价格、资产价格,

商品的价格或者是资产价格就不能支撑,那么P就因V的大幅度下滑而下滑。你写一下这个公式,MV=PQ,M就是货币的数量不减,它就是不紧缩,它哪有紧缩的空间嘛?它怎么退出?怎么紧缩?M不能紧缩,V下滑,那么对应的就是P。也就是说在特定时期,美国的资产价格将会出现大幅度的下滑,那么这个Q在资产这个领域里边是没有问题的,是商品的部分会出现一个什么样的变化呢?我们认为与工业相关联的商品的价格也是要下的,唯有定义货币的那些……

唯有定义货币的那些价格和新经济的那些价格,可能P不会下,还会上。因为战略储备的原因,所以我们认为金、银和类似于其他的贵金属,乃至于锂、乃至于铜,都有可能维持一个较高的价格,或者是有些价格会疯狂上涨。我重复一遍吧,重新定义货币价格的那些的东西,那些商品可能会逆势上涨,其他所有的资产和商品的价格都会随着V衰竭而下降。

昨晚北京的朋友说,那么我们A股市场里边的好多这个经济,新经济也好、旧经济也好、Data也好、非Data也好,你怎么看呢?我的意思是说我把价格的事情已经说清楚了,筛选的半径就可以缩得很小了,就是我们选择投资的范围就变得非常小了。如你真的理解了中国共同富裕这四个字的经济政策的含义,那么这个半径可以进一步缩小。那么我都快说清楚了,如果你在听了今天资本积累的这堂课的话,大体上我觉得该做什么,其实在中国资本市场上可以选择的·····

能让你选择的范围就那么一些,既能够抗通胀,又代表新经济,东西不多。好,我们把两件事情的交集画出来。首先必须面对通胀了,而且这个通胀是输入性的通胀,必须面对通胀。同时经济的转型在飞速地进行,就是新经济的部分,资本的载体是Data,不是其他东西。这两件事情的交集就是我们要选的。回去把交集的企业画出来,我们要选的东西就是它了。由于共同富裕直接税一定要上,所以那些不动产的东西要做重新的考虑,好不好?

我就不重复了吧,因为我想我们课堂上很多人都是这方面的专家。因为共同富裕,今天还是一句话,明天、明年就是深刻的变革。李光满说的深刻的变革,这五个字,一个字一个字都没错。那个谁说可能是土改、可能是革命也没错,因为对某些人来讲,它就像是土改,甚至就是革命,也没错。那么我们应该做什么呢?我就不提醒了,我们应该处理好自己哪类型的资产,我们应该如何在一个特殊变革的时期持盈保泰,做好处理,管控好自己,管控好自己身边的人。那我觉得我今天资本积累这堂课就说……

今天的《亚洲周刊》出了一篇文章,是我临时加的,是谈的抗疫的事情,有空到时候我把它转给大家。谈的抗疫的事情——批评了分子生物学家,批评了肺病治疗专家的不正确的观点,就是抗疫是公共卫生的范畴。公共卫生的范畴,它思考的焦点,它不是一个科学和技术问题,而是个财政问题。事实证明这个正确的选择是十万亿级别的正确选择,十万亿级别的正确选择,我们花费只有一百亿。如果算法出来了,其实道理是简单的。

抗疫清零与共存 https://www.notion.so/b9d81c76919f49299ae333c3f4c1f09f

好吧,今天的时间不短了,我今天就讲这么多。明天下午三点钟,我们有什么需要补充再补充,我们明天下午三点钟见。预告一下子,下周我们开始讲风水,下周聊天聊风水,那是我送给大家的中秋节礼物。我也准备得差不多了,好多人很急切,我说不用急,中秋都没到呢,不着急。好吧,今天就说这么多,保重身体,保重身体,样样好!好吧,明天见。

\section{剩余价值、供给学派与超级地租、对美国\&中国\&黄金的判断、恒大的事情}

大家好,今天是2021年的9月18号——九·一八。今天我们是正式课,是第十二讲。这第十二讲,《资本论》第十二讲要讲这个剩余价值了。在这一堂课里边我们除了剩余价值之外,可能还要涉及到供给学派、涉及到超级地租的一些事情,然后末尾留出一些时间,我们讲一下子对美国经济和中国经济的一些判断,包括对黄金的一个判断。如果还有时间,我们聊几句恒大的事情。好,我们三点钟准时开始。

大家好,今天是2021年的9月18日——九·一八,也是辛丑年的八月十二日,这意味着我们距离辛丑年的九月还有18天的时间,两个18。九·一八对中国人来讲是一个永远的痛,我们的想法是三百年或八百年再不要发生这样的事情。我今天发了我以前写过的一篇《关于九一八的再思考》\footnote{https://www.notion.so/395f9614a7f143eea22180db8d715cd7}的文章,那是我关于九·一八的三篇文章里边的其中的一篇,我们希望能够从九·一八的历史事件之中吸取教训,避免今天重蹈覆辙。

我们今天讲《资本论》的第十二讲——剩余价值综述,这个综述里边要加入一些内容。其实中秋节这一讲我花了挺长时间的,因为这个《资本论》第一卷将近三分之一都是在谈剩余价值,就是马克思用了最大的篇幅来谈剩余价值。而我从一读马克思《资本论》的时候,就对这一段、这个部分有不满,有一些求全责备。我们不能对马克思在一百多年前的这样的一篇东西用今天的观点来做出评判、评价,但我们可以进行完善。

后来来到香港之后,我对马克思的剩余价值,我三读之后又有了崭新的认识。三读、四读,现在是五读,又有了崭新的认识。其实读到现在,我对马克思又重新充满了一种崇敬,就是他的剩余价值理论还是具有非常重要的现实意义的。一会儿我们会结合理论跟这个实际做点结合,就是谈一下恒大的事情。看看剥削是怎样存在的,剥削和压迫在新时代是一种什么样式。剩余价值理论其实是一个成长的理论,就是它在不断的发展和成长的理论,是应由我们把它成长起来。

好,我们进入今天的这个第十二讲的第一个环节就是剩余价值的定义。通常我们的习惯是你定义一个东西就要把相比较的定义完整了,那么有剩余价值,那么有没有饱满价值?有没有未饱满的价值?就是未达标或者是未饱满的价值,就是当然是存在了,因为百分之百不剩余、没有剩余的这种可能性是存在的,虽然可能是很微小。百分,不用百分之百,有些人的劳动还达不到工资的、无法达至工资报酬的水平,未饱满或者未达标的这种价值也是存在的。但我们今天讨论的是剩余价值,就是……

超越了资本家支付给劳动的成本的那部分剩余劳动,这部分剩余劳动也可以视作劳动者的劳动结余或者劳动者劳动剩余,它最后最终构成了资本的利润,变成了资本家的原始积累,或者是资本家的超级消费。它变成什么不重要,就是确实是在整个的这个资本运行过程中存在剩余价值,而且有的时候剩余价值的规模极为宏大,宏大到你难以想象。我们现在给出剩余价值一个基本的定义。

马克思在讨论剩余价值的时候,他给出的很有意思。马克思先把资本分为可变资本与不可变资本,不变资本和可变资本它区别就是用于这种物料的,就是生产资料的这种资本投入,主要是耐久材,他认为是不变资本。用于劳动者支付的、购买劳动的这个钱他认为是可变资本。然后他给出剩余价值的定义很有意思,他说可变资本增值等于剩余价值。可变资本增值等于剩余价值这个事情就是我一读《资本论》提出的疑问,就是可变资本增值有没有其他可能性?我跟张复英先生在讨论这个事情,后来张复英先生认同我的观点说:

{\kaishu 在可变资本增值的过程中存在其他可能性,如果武断地认为可变资本增值全部等于剩余价值,这个有可能并不周延。}

我们今天不讨论这个问题,这是年轻时候、读书的时候发的疑问,就是这个定义,一个百分之百的定义是否精准呢?就是我们可以继续讨论下去。至于剩余价值的这个计算公式——剩余价值率,他用的这个公式也是用的这个可变资本增值这样的一个逻辑来计算的,但我觉得我们不要过分强调就可以了。那么剩余价值率其实就是剩余价值除以可变资本得出这个比例关系就叫剩余价值率。

还可以有一个其它的一些运算吧,因为剩余价值除以可变资本,剩余价值可以除以全部资本,得出不同的剩余价值率。这个我们计算的部分今天不打算展开讨论,因为在这个地方其实我也有一些疑问的,因为我在研究超级地租的时候发现了一些问题。剩余价值在很多时候呢是被理解成为剩余产品的,就是我本来、补偿我的劳动我生产8件产品就够了,但我生产了10件,那2件产品就算是剩余产品,构成我的剩余劳动或者剩余价值。马克思在《资本论》的一共是用了三个篇,三篇十几章来讨论剩余价值。

关于对剩余价值的定义、剩余价值的计算这块我想今天忽略过去,因为我们不花太多时间在这个问题上面,留点时间在其它问题上面。我在讨论剩余价值的时候,我得提一件事情,就是马克思对所有劳动者的一个巨大的贡献,就是由于马克思发现了劳动的剩余价值的秘密。所以他为什么要提产品和工作量、工作日?就是为了提出个人权的概念,就是劳工的基本权利,就是工作日的概念。就是劳工他可以提供剩余的劳动或者是剩余产品,但这个剩余劳动和剩余产品极限在哪里?要有一个计算的基础。这样的话马克思提出八小时工作制,你知道这个事情太重要了,这跟我们“996”有关系。

马克思提出八小时工作制之后,在1889年的7月,规定了1890年5月1号为劳动节,这个时候就是这是第二国际的功劳了。然后1917年11月11日,苏维埃颁布法令,确定八小时工作制为一个法律。我国在建国之后也将八小时工作制确立为我国的劳工保护的一个法律条款。全世界是1966年的9月,日内瓦国际工人代表大会根据马克思的倡议提出了这样的一个想法。我个人认为马克思的《资本论》是真正的人权宣言。

在讨论剩余价值、剩余劳动和剩余劳动时间的过程中,提出工作日的这样想法,然后提出八小时工作制,实际上是人类文明进步的一个标杆,这一件事情真的很重要。它实际上是确定人的基本权利,确定一个可计算、可衡量的一个基本权利的一个重要的历史进程。在这个问题上,我们应该向马克思致敬,不光是五一劳动节应该向马克思致敬。所有的劳动者,确实他们确实也是深深的尊敬、崇敬和热爱马克思,也是有原因的,因为在劳工成长的历史上,马克思在重要问题上通过理论论述,他不但是有完整的、系统的思想……

而且他还有很具体的操作步骤,比如说这个八小时工作制、劳动日的这个概念,就是他要将剩余价值量、或者剩余劳动量、或者剩余劳动时间进行精确的管控。当然了,事情远比马克思想象的复杂,在剩余劳动量的计算、剩余价值量的计算这方面我今天就不想展开了,它里边有两个不同的公式,这两个公式大家有时间可以自己去看一下子,它不复杂。因为我们腾出点时间来,我想把剩余价值里边最核心的东西讲一下子,剩余价值里边最核心的东西是绝对剩余价值与相对剩余价值。

什么是绝对剩余价值呢?什么是相对剩余价值呢?绝对剩余价值就是劳动八小时之后,在合理劳动八小时之后又提供了两个小时劳动,那个就是绝对剩余劳动、绝对剩余价值,或者是生产八件产品之后又生产了两件,那就是绝对剩余。什么是相对剩余价值呢?今天我问题的焦点在这个地方,相对剩余价值就是我本来八小时工作制我生产十件产品,但由于我熟练程度不断提升,由于机械自动化程度不断提高,我一天八小时之内我生产了八十件产品,那么剩下的七十件叫相对剩余价值。

相对剩余价值就引起了广泛的关注。这个刚才我给出它们两个的区别,大体上定义也是如此,这里边涉及到什么事情呢?涉及到,第一分工越来越细,分工细就能提高效率;第二机械化、电气化、智能化大幅度地提高这个生产效率。另外就是劳动力价格和剩余价值这个量它出现了剧烈的变化,以前劳动力的价格占的比重、整个价值的比重比较高,比如说一件产品里边有百分之十是劳动价格,那么随着进步、机械化程度,就是一件产品里边可能只有百分之一是劳动价格,那我们是不是说那百分之九就是剩余价值的量的变化呢?

好,我们现在开始做一点点的补充,这也是在研究《资本论》的时候,我研究香港的经济问题想出来的这个超级地租理论。在香港为什么要提出超级地租理论呢?就是我个人认为劳动者不光是被产业资本剥削,还有其他的剥削的可能性。这个其他的剥削是劳动者的劳动剩余吗?是他的剩余价值吗?从宏观上来看,它就是。我是怎么来思考和分析这个问题的呢?我是这样思考的,就是一个劳动者来……

保障他生存必需品,就是劳动者的成本。劳动者劳动再生产的成本,那里边需要衣、食、住、行。那么衣、食、住、行构成了劳动者的成本,就是劳动者再生产的成本。那么这个衣、食、住、行如果不合理膨胀,就会使得劳动者的成本变高。劳动者成本变高,它会表达于产业资本、资本家里可变资本的部分变大。我重复一遍,可能这个课说到这,没有个黑板,有点不太容易讲明白。就是让一个劳动者,我们这个重复的让劳……

{\kaishu 劳动力的价格(工资)由劳动力的成本(供需)决定。}

{\kaishu 劳动力的价值(产出)的上限是劳资双方博弈的结果,因为工作时间可以尽可能逼近7X24小时。}

{\kaishu 剩余价值即两者之差的一部分。}

我们进行劳动者再生产的时候需要衣、食、住、行。如果衣、食、住、行里边,比如说住成本不断放大,那么劳动者的再生产的成本就会放大,它必然表达为产业资本之中可变资本膨胀。因为可变资本给的少了,劳动者是没有办法进行再生产的。那么劳动者的本来应该挣到的工资里边,用于劳动者再生产的部分里边会挤压到其他东西。那么如何来理解这个东西呢?后来我就建立了超级地租的理论,超级地租理论里边有一个重要的假设——现在所有的经济学都需要假设,就是我认为一个劳动者进行劳动再生产

衣、食、住、行的住不应超过其全部收入的、全部家庭收入的30\%,这是我的第一个假设。这个30\%是一个底线原则,就是我挣一千块钱应拿出三百块钱租房,或者是拿出三百块钱供楼,这是极限。如果你是一个治理良好的社会,那么应该是两百块钱或者一百块钱,如果是一百块钱,那么劳动者的这个劳动力的再生产过程就会有剩余,那个剩余呢可以变成劳动者学习的钱,比如说这个买书、进夜校学习的钱,就是增长劳动力水准的钱。甚至从某种意义上讲……

也有可能有机会变成不增加产业资本中可变资本总量的这样一个前提条件。也就是说一个超级地租会导致双重问题:记住我的话,第一,导致劳动力再生产成本大幅度增加;第二,它会导致产业资本中可变资本成本大幅度增加。说到这儿你应该懂啦,其实马克思的剩余价值理论真的是有用的,你说他老人家他那么年轻他把事想明白了。今天我们来再看中国的恒大,你就懂了,他们在干什么?这双重的成本通过某种……

一个是劳动力自身的成本,一个产业资本中可变资本成本不断地增长,从而形成一些控制劳动力再生产要素的那些特殊的资本集团获取暴利,这是一个非常糟糕的事情。好,我们再往前延伸,其实理论的探讨非常重要。这种的掠夺它已经超越了一般意义的产业资本的掠夺,它是双重掠夺,第一重它是金融资本的掠夺。记着我的话:房地产或者超级地租是金融资本掠夺,而非产业资本掠夺。

金融资本掠夺的方式往往是通过货币的操作或者是资本的累积而形成了某种东西的在特定时期的价格的波动,我们管它叫通货膨胀。房子是商品,房子虽然是耐久材,但是它也是商品,房子也是通货膨胀的一个重要的组成部分,它是容纳通货的一个重要的载体。那么,金融资本是如何通过超级地租来掠夺劳动者劳动剩余呢?就是剥削剩余价值的不光是那个产业资本,还有金融资本,金融资本它是用砖头来剥削剩余价值的。

我们看两件事情。第一件事情,货币为什么会多起来?因为,我知道金融在管理上面是有问题的,就是当通货的流量大幅度增加的时候,它必然导致某类型商品和资产的价格上涨,一会我们讲到黄金。这是一个结论,这是不以人的意志为转移的结论,只不过在某些特定的时期,可以通过技术手段做出某种扭曲而已,这是第一件事。第二件事情,金融之所以叫金融,它除了有总的流量之外,还有个人能够操控的部分,那个我们管它叫杠杆。

一个地产商通过杠杆可以控制多么大的金融资产呢?恒大就是个例子,对吧?恒大的,就是像许家印这样的人,他的本事能赚一个亿?但他操控了两万倍的杠杆,通过巨大的杠杆来形成巨大规模的资产,而这个巨大规模资产里边其中的一个部分在短期之内为他们形成暴利。为什么形成暴利呢?是因为我们使用了非常糟糕的这种提前收割的模式,就是所谓的供楼模式吧,六个钱包——它实际上是通过银行信贷提前完成了地产商掠夺,通过金融资本对……

通过金融资本、地产商通过金融资本对劳动者剩余价值的提前的榨取。这是非常残忍的。那么,是金融资本单独来完成这件工作的吗?当然不是,还有一个协同的罪犯,它叫财政。财政在整个掠夺剩余价值里边……好吧,我们今天把它作为一个概述。对工人阶级的剩余价值的剥削来自于三个层次:最低层次就是马克思所说的产业资本、工场主、资本家,这是微观层面的;还有宏观层面的,那么就是通过金融手段进行的掠夺——超级地租;还有更为宏观层面的,是国家自己层面的,就是财政

它通过扭曲的税收向劳动者征收更多的税,而向资本利得者征收更少的税。就是今天这个局面,90\%以上的税都是由劳动者劳动,包括企业所得税、个人所得税,所有都是劳动交的、劳动者交的,而对资产、资本利得不征税。这是一个扭曲的财政,通过第一层、第二层、第三层,财政是第三层了,榨取劳动者的剩余价值。我这样说可能更合理一些:就是当一国经济发展到一个阶段,不管你是国家资本主义、官僚垄断资本主义,还是金融垄断资本主义,这两个情况都存在,无论是金融垄断资本主义还是官僚垄断资本主义,

他们压榨的绝非仅仅是劳动者的剩余价值,他们甚至压榨产业资本的剩余价值。记住:他们还会压榨产业资本的剩余价值。所以你看到了,无论是金融垄断资本主义,还是官僚垄断资本主义,产业资本都会迅速从这些国家消失。它不仅仅是美国现象、欧洲现象,也未来可能成为亚洲现象。一会儿我们要谈供给侧了,供给学派。在整个的马克思《资本论》写成之后,对整个的剥削的过程我们也看到了资本牟利的过程,我们也看到了资本的结构的变化的过程,就是

产业资本不断地被金融资本、被财政制度和政策进行压榨,以至于剥削完劳动者剩余就开始剥削产业资本剩余,产业资本剥削完剩余之后,这个国家开始慢慢走向衰败,然后金融资本撤离、逃亡,大体上是这样一个故事啦。在读《资本论》的时候,其实我们也仔细地在回顾我们国家五千年文明史上面的一些旧故事,其实用《资本论》能解释的通。就是整个的兴衰过程,用《资本论》大体是可以解释通的,可以解释通的。好,我们对这个绝对剩余价值和相对剩余价值里边的一个部分做了一个解说,

也算是对马克思的《资本论》做一点点补充。就是剩余价值的剥削不仅仅是针对劳工的,也针对产业资本。就是当金融资本由国家、由官僚垄断资本主义控制的国家财政都可能构成对产业资本剩余价值的残酷剥削,以至于产业资本消亡、退出。这个事情我们在美国已经看得非常清楚了,所以当特朗普提出再工业化的时候,我们就要提出:你如何确保你的金融资本不再榨取产业资本的剩余价值?这件事情还有更深层次的思考余地和思考的空间。

那就是今天的高科技企业,比如说苹果,比如说亚马逊,比如说谷歌,比如说Facebook,他们算产业资本还算是金融资本呢?他们是生存于金融现象之上还是生存于产品、产业之上呢?如果我们只是在高端,按照供给学派的想法,就是我们只发展高端,低端我们不要了,那么我们这个高端对高端,金融资本会对苹果构成剩余价值的压榨吗?华尔街会压榨苹果的剩余价值吗?

准确地讲,我将房地产、将互联网等新经济全部视同为金融资本,而不将他们视同为产业资本。我非常坚决地支持我党、我国、我国人民对金融资本的约束,甚至对房地产、对互联网垄断的这种打击,我认为是社会发展到特定历史阶段必然发生、必须发生也非常正确的一件事情。因为我们不能允许以许家印等为代表的这种大型企业,以及他们企业背后的金主们再去大规模榨取劳工的剩余价值。

更加的不能允许他们去榨取中低端产业资本的剩余价值。这种榨取如果我们什么都看不到,我们学《资本论》干什么?我们学剩余价值干什么?我们通过《资本论》、通过剩余价值、通过超级地租理论把这事说清楚,说清楚之后,来形成对国家的金融制度和财政制度的这种改变。在这里我非常高兴地要说一下子,就是我国现在虽然在教育、学术、传媒领域存在着巨大的问题,就是我们的专家、学者不站在人民一边,站在资本家一边,鼓吹超级地租,不允许我们进行直接税改革。

但是,我国是一个非常神奇的国家,我们有一个非常了不起的政党叫中国共产党。中国共产党的构成它的确是很神奇的,它和任何国家的构成是不一样的。中国共产党的构成里边有相当一部分拥有一定哲学高度理论素养,又拥有比较高的道德水准的这样一批人,他们可以超越学者、专家的无耻,而进入到另外一种层级的制度建设或者政策逆转。所以我们对二十大有着非常深刻的期待,虽然有人通过地产崩盘、通过互联网企业的价格的崩盘,来玩某种的逼宫游戏,但没有意义,因为我们老百姓是……

广大的老百姓当然是支持中央对现有的这样的一个非常残忍的架构,因为他们通过占据制度的优势、通过财政金融制度的制定者的这个优势、通过对资源的把控、通过对资本的、金融资本的把控,来残酷地榨取劳动者的相对剩余价值。绝对剩余价值、相对剩余价值都被它榨取了,然后还要榨取实体经济或者是产业资本的剩余价值。这件事情我想2021年的贲卦就应有了结,而贲卦到九月份——农历九月份,10月10号之后原形毕露,就大家都看到了。

在这个历史进程之中,会产生许许多多的基于遭遇的情绪化表达,因为很多人是受害者,就是整个这个除了恒大以外,大概有十来间大型地产公司都会出问题。好多老百姓是受害者,不管是他集资了,买了这个金融产品,也不管是他付了预售款买了楼,也不管是他已经买了低质量的楼,总之,很多人会牵涉其中。其中在香港苦主也是非常大的,因为它恒大上市嘛,在香港有上市公司的市值跌去90\%啊,其中有些人是我的朋友,而且是这个资本市场很牛的朋友,他们都投了不少恒大的东西,在郑裕彤\footnote{郑裕彤,香港商人,新世界发展第二任董事会主席兼创办人之一,以及周大福珠宝金行创办人周至元女婿。}的带领下投了恒大很多东西。

如今90\%不见了,当然有些人投的低一点,可能丢掉的是百分之七八十、五六十,但我们身边没有一个人赚钱的,全部赔掉。有的时候不太好说什么,因为其实像我这样的人对恒大一直以来就有负面的看法,所以我身边所有的人,当年去购买恒大不同的结构性的东西的时候,不管是它的股票、它的债券、它的房产,还是买它的什么其他的产品,我都给他们提出过建议。有些人他因为他是专业分析师,他就告诉我怎样怎样,我还是给他出建议:不要碰,麻烦你不要碰他们这些东西。

前两天有朋友再次谈起任泽平这个现象,就是三年前、还是四年前请任泽平去恒大,那么那个时候恒大的结论有三个:第一个是在高点清盘,就是顶级高手上上策在高点清盘,那么就是任泽平去的时候就应该完成在高点清盘套现这个动作,显然他们不是高手,所以上上策没有走;中策,就是及早进行债务重组,这是中策;下策是清盘。中策走不通了现在,因为你债务重组如果做得早,找的目标合适的话,其实恒大这点东西是可以清理出来的。可惜,确实是没有一个高手,所以没办法。

最后就是破产清盘。其实破产清盘可能是对老百姓而言不是最佳选择,但是对国家整体的发展而言是最佳选择。就是他们的破产、清盘会取到三个效果:第一个效果是因为他们将现在的泡沫刺穿了,那么资产价格回归本源、理性,这对其他劳动者的劳动的剩余有好处、没坏处;第二个,实体经济的资本被他们拿走了,当他们的泡沫破灭之后,相当一部分的资本会重返实体经济,对实体经济发展是有好处的。我昨天晚上吃饭还在讨论这个问题,我说我国的经济学家太浑,是资本率决定发展速度,什么体量大了就慢了,这是胡扯……

完全是胡扯,就是“我国经济体量大了,所以经济增长速度就慢下来了”。这跟体量有关系吗?是你的资本积累率到10以上,所以你的经济增长在5以上,资本积累率接近20,你的经济增长还是可以上10的。问题你的资本没积累嘛,资本哪去了?走资了嘛,逃走了嘛。通过房地产、通过超级地租累积巨额财富,然后流走。潘先生不是已经在那儿了嘛,在看比赛了嘛,李先生也走了嘛。如果让这十个地产商都走的话,那你这十万高净值、百万亿财富全走的话,你还有经济增长吗?所以由此你可以看得出来,我党这一年做出的所有的决定是多么的正确!点赞!

在这里呢,因为研究剩余价值、研究超级地租,我就想多句嘴,其实也可以将来在讲凯恩斯的《通论》里边去讲这个部分,但是我想今天简单说几句。因为昨天晚上我们在吃饭的时候还在讨论这个问题,就是供给学派,就是里根经济学。我和一些比较左翼的朋友的看法还是不太一样,因为一些左翼的朋友认为这个世界上不存在供给学派,而且供给学派的立论基础就不存在,就不存在供给学派。甚至他们觉得拉弗曲线\footnote{拉弗曲线是一种假说,设想了政府的税收收入与税率之间的关系,当税率在一定的限度以下时,提高税率能增加政府税收收入,但超过这一的限度时,再提高税率反而导致政府税收收入减少。因为较高的税率将抑制经济的增长,使税基减小,税收收入下降,反之,减税可以刺激经济增长,扩大税基,税收收入增加。}只是一个幻想、幻觉、理想,并不存在减税会导致增税这样的一个局面出现。

我这里边,对里根经济学或者是撒切尔夫人的经济学,我做一个供给学派之外的解释,好吗?因为供给学派的确是个伪学派,我倒不因为它是出自于芝加哥,倒不因为它是奥地利学派的延伸。当然我对这个东西是深恶痛绝的,但我并不认为它里边毫无逻辑或者毫无科学性可言。但我要说明的是,无论是里根,也无论是撒切尔夫人,他们谈的供给学派,他们谈的不是美国和英国本土的供给,因为在美国和英国本土那个拉弗曲线是不成立的,没有那个供给的可能性,难道撒切尔夫人在英国减税就能导致英国经济起飞?

我们在香港眼睁睁地看着香港从1983年12月份起到1997年,香港经历了什么?香港被一种强有力的金融资本以大规模——多大规模?一万亿磅的水平,剥夺了它的香港本土,不光是劳工的剩余价值、实体经济剩余价值,甚至包括香港的所有的价值,一万亿磅被拿走了。这是供给学派的理论吗?供给学派只是作为里根和撒切尔夫人的一张皮而已。那么里根经济学成功是成功在减税上面吗?当然不是,是日本和苏联的解体向他们提供了更庞大的剩余……。

我个人认为,从日本和苏联流往美国的剩余价值的总量,应是香港流出的5倍左右,就是五万亿磅。由此挽救了美国、挽救了英国,一定程度上也挽救了欧洲,也利益了欧洲,这么个意思吧。供给学派能解决真的经济问题吗?那么我问两个人,第一个是特朗普,减税了没?减了;美国经济因为他减税起来了吗?没有!再问一个人,这个人我不能问,我国什么时候?2015年开始供给侧结构性改革,对吗?我们供给侧结构性改革那位可爱的市长,那位可爱的重庆市市长还天天在那讲供给学派,用点脑子吧!

用点脑子好不好?那个供给学派的那个拉弗曲线能解决中国经济的问题吗?能解决中国“三去一降一补”的问题吗?六年过去了,有意思吗?所以我最近写了篇文章,昨天晚上朋友们说:“卢先生啊,这题目有点猛啊!”。我的题目是:“笨蛋!问题出在需求侧!”。供给侧站着都是资本家,需求侧站的是劳工,你跑供给侧改什么革?需求侧需要改革,对吗?需求侧的改革是共同富裕,对吗?需求侧需求增加是需要共同富裕,需要从资本家这儿拿一点儿分给他们,对吗?

每一个极端的右派,不论是西方右翼还是东方的右翼,当他们准备蒙骗老百姓、掠夺老百姓的剩余价值的时候,甚至是掠夺落后国家、弱智国家的剩余价值的时候,他们必拿出供给学派来。什么里根经济学?什么撒切尔经济学?什么供给侧结构性改革?好意思吗?拿计算器用算法来证明一下你的理论成立吗?当我们用计算器计算完了,全部都是掠夺的话,是谁在供给谁呢?是你们从需求侧,这个需求侧可能是你国的需求侧,也可能是他国的需求侧,是从人民的剩余价值搬家来满足了你的供给侧。

我想呢,因为我差不多每周都会给《亚洲周刊》写一篇专栏。上一篇的专栏是引起了震动,关于德国问题,好多人对我也意见都很大,就是你不让解放台湾吗?我谁说不让解放台湾了?我是说我们不能像德国那样将二十世纪的德国世纪变成美国世纪,我们绝不能将二十一世纪再次变成美国世纪,二十一世纪就是中国世纪,你为什么要把它变成美国世纪?因为你提前开枪了。1914年有一个奥匈王子被刺杀,所以威廉二世同时向三个帝国宣战,很牛吧,然后连同自己四个帝国都完蛋,很牛吧。

中国须记取德国世纪葬送教训 https://www.notion.so/065536c914f74695920920e6d8a30cdf)

我今天简单点两句供给学派,希望大家知道,《资本论》是真正的《资本论》,是一个伟大的理论,供给学派那就是一个骗局。至于里根经济学,至于撒切尔经济学、安倍经济学,别扯了,我们2015年开始的供给侧结构性改革,什么事情都让我们看清楚了吧?什么事情都让我看清楚了吧?减税减谁的税?减了劳动者的税吗?五险一金吗?这样的供给,以行政手段去产能消灭实体经济,是在增加劳动者的有效需求吗?很扯,就别扯了,再扯下去,裤子都撕下来了。

最近讲《资本论》讲得经常会陷入到一种很情绪化的过程,请大家原谅。其实,剩余价值理论里边有一个重要的贡献,我今天说了,一个是工作日的贡献,这个非常重要,八小时工作日,这个是了不起的。还有一个重要的贡献就是关于工资。一会儿我们花点时间讲一下工资。剩余价值率的计算公式、剩余价值的计算公式,所有的计算公式我今天都不讨论了,因为没有黑板,咱们也没法写公式、没法念,你们花点时间,不用读《资本论》,上互联网百科或者是维基百科,这个剩余价值上面都有了,这些公式都有了,你们熟悉一下也就可以了,因为我想你不做具体学问的话,其实……

工资是个什么东西呢?工资它是资本家对劳动者进行的货币或实物补偿,通常工资应该是货币,但是有时候也发实物,就是货币和实物补偿。我们把工资作一个定义,就是它是一个完整的定义,就是为了劳动者再生产、劳动力的再生产,而资本家对劳动者进行的一个补偿,它包括了工资、包括了实物,这样一个东西,就叫工资。工资的定义倒是也没有那么重要,就是你大体上你不说这个定义,大家也知道工资是什么意思。工资的分类呢,我看我们也不讨论它,有的是计件工资……

工资有各种方法,有计件工资、计时工资,有的就是职位工资,有的就是……反正什么样的这个工资类型都有,现在工资的分类非常复杂。我自己个人在收入方面就是现在就非常复杂了,所以我就觉得这个分类我也不讲了,反正是你就理解为你个人劳动所得收入就是工资就行了。就是,我是前天去香港税务局把税交了,我就不告诉你们交了多少税,反正交了一大堆的税。那个时候税务局会把你的工资进行分类,而且你哪一笔稿费,哪一次讲课都不会落下。

工资等同于收入吗?不等同。工资为什么不等同于你个人的收入呢?因为你个人有非劳动所得。那就是我们很多人,特别今天听课的朋友里边有很多人都有资本利得。资本利得和劳动所得并不相同,我们不把非劳动所得纳入工资的范畴。工资就是劳动所得,资本利得不属于劳动所得的部分,不能视同为工资。好多人说,那么资本利得也是我辛勤劳动的结果——错!因为那里边有可能是遗产,跟你没什么关系。所以我们在收入的定义里边要把工资的定义和收入的定义做一个区隔,最好……

将来我们在讨论经济制度和经济政策的时候,就明白了:为什么这个劳动所得对工资的这种征收的税赋,建基于对工资征收的税赋要降下来,而对建基于资本利得的征收要增上去。然后我们在后边的课程里边再讲劳动,这个劳动者、就是劳动力再生产的过程中衣、食、住、行,这个衣、食、住、行的成本的降低,实际上相当于被动增加劳动者的这个工资。当然不增加他收入了,是增加他的工资。它会使得产业资本获得更好的发展空间,使得劳动力获得健康发展,也使得产业资本获得更好的发展。所以

我们必须有力的限制衣、食、住、行的成本,现在衣的成本没有问题;食的成本大体上没有问题,因为这都是国家行为;行的成本没有太大问题,你有钱的人买一辆好车,没钱的人这个自行车也能走。现在的问题就是出在住上面。住和医疗、教育三座大山,如果我们任由这三座大山去消灭劳动者的工资、去扩张劳动者的工资,那我们不但是对劳动者本身不仁慈、不给他人权,我们对中国的所有的产业资本和实体经济都是非常残忍的。记着,住房、医疗、教育这三座大山……

它构成了一个国家发展、历史发展的重大的障碍。不是说“不要住”,不是说“不要医”,也不是说“不要教育”,是它必须有效的控制这三个产业资本的资本利得。控制住房的、提供房源者的资本利得,控制提供教育者的资本利得、控制提供医疗者的资本利得,这三大产业的资本利得必须低于实体经济的资本利得,以至于我们消灭超级地租,以至于我们让劳动者获得更多的幸福感,以至于让实体经济重返健康发展的轨道。如果你问我,中美竞争的焦点在哪里?

我会告诉你:不在东海、不在台海、不在南海,就在“三座大山”上。不在海上,在山上。如果这三座“山”我们能把它削平、推倒,中国不是赢在今天,是赢100年、甚至300年、甚至800年,我们如果能够永久、永远地解决这个问题,那我们就是长治久安了。这个,讲到工资的时候,又绕回去。我们还把工资说完。我们工资的定义、工资的分类,收入的定义、收入的分类……收入分类其实重点就是资本利得和劳动所得,两个部分,其中资本利得的部分有遗产和赠予。

就是我们一定要重视这个劳动所得和资本利得的差异性,这个差异性里边表达了一个国家的文明程度和文明水平。我自己将来会在这方面多写一些东西,因为在这个问题上面是我现在新修订的《新社会主义论(2014修订版)》的核心的部分。我们《新社会主义论》谈新社会主义文明的程度的时候,重点的表述就是劳动所得与资本利得的关系。如果我们解决不了资本利得,侵蚀劳动所得,那么我们就没有学懂马克思的剩余价值理论,那么我们在帮助资本家掠夺劳工的剩余价值和实体经济的剩余价值。这是一个什么国家呀?所以这个差异性一定要搞清楚

差异性里边有巨大的文章可做。当然事物都有它的边际与极限,否则它不是经济学。工资并非越高越好。劳工组成了工会,进行斗争,斗争的边际你一定要清楚。就是剩余价值完全没有了,叫丰满价值,这个甚至进入到不丰满的价值,那么产业就会消失了。所以劳动所得也有边际,资本利得也有边际,劳动所得和资本利得的边际又构成某种联系,它定义为三个字:合理性。这个合理性是我们追求的一个最高的境界,就是它们达至一个最高的平衡。

我们是否能在制定政策的时候就解决完所有应该由市场完成的那部分工作?当然不可能。就是一个国家大部分的事情还是由市场进行交易、进行完成。只要有市场存在、有交易的存在,那么就有可能会出现极端情况,就是极端的价格、极端的不合理,就会出现对剩余价值的残酷的剥削,就会有。我们接受出现不平衡,但是由于我们是一个文明国家、文明国度,高度水平的文明国度,我们有制度补偿的能力,就是我们通过制度设计对市场交易的不合理进行补偿,所以我们才有了直接税的设计。

讲到工资,我是想马克思关于工资专门写了一章,但是他那个工资那个内容不如我今天讲的深刻和精彩。就是那个工资,我觉得工资是一个大事儿,应该可以再深一些。但是马克思侧重点在,最后是在绝对剩余价值和相对剩余价值的描述、计算和分类和这个举例方面就没有讲得那么透。那么我们今天剩余价值的部分就简单讲这么多。因为这两天市场出现巨变,黄金又掉了,这两天大家又是惶惶不可终日,所以好多朋友还是建议我聊几句吧,我也正好最近一直在回答北方的朋友的提问。

其实是老生常谈,但是这个老生还是得唱的,不唱不行啊。美国的经济,最近既无内部的重大改革,也无外部的重大变化。那么发生了一个神奇的现象:就是美元开始走强,美国的利率有上升的一个趋势,所以出现了一个基本的现象。我们看到大家预期通胀会可能结束了啊,通胀没那么严重了,然后美元升值了,然后其他资产出现了贬值,出现一个局面。

我不这样看问题。我先说一下子我的逻辑,对美国经济的基本判断。我个人认为影响美国经济的根本性要素,特朗普没解决,拜登没解决。拜登不但不减税,还在加税,加税就是增加要素成本嘛。这个特朗普是想学里根,玩一下供给学派经济学,就是他想通过减税刺激经济,但特朗普的减税只刺激了股市,因为它减税使得美国的企业资本利得、企业所得又大幅增加,所以股市一飞冲天。但是美国实体经济没有寸进。

从拜登这几个月的情形来看更糟糕,不但没有寸进,美国的就业人数在减少。美国现在经济的三驾马车:投资有的,增加了一些投资;出口不行,还追不上以前;消费很厉害,是因为政府撒钱嘛。那么美国的目前经济增长的主要动力源泉是消费,是需求侧,是政府印钱发给老百姓。你觉得这个经济增长后继更加有利吗?还能印更多的钱发给老百姓去需求侧拉动吗?我个人认为十月份我们就见到真章了。

因为这件事情不光是拜登个人的决定,它也涉及到立法机构的一个决定,就是参议院和众院允不允许他再大规模的印钱、发钱来拉动经济?如果不印钱、不发钱来拉动经济,在德尔塔、在冠状病毒依旧如此严重的情况下,美国的经济的动能出自哪里呢?第一个问题,美国经济增长动能在哪里?它不增长就是停滞,这是第一个问题;第二个问题,谁能给我解释,凭什么美国就不通货膨胀啊?美国的物价就会降下来?美国的房价在升,美国的股票在升,股市楼市都在升,油价在上。

那么他的通胀会急剧地掉下来吗?如果增长乏力、停滞,通胀炽热,滞涨的格局形成,强美元的支撑是什么呢?难道中国人买64亿,这个月买64亿美元的美国国债导致美国国债债券的利息出现某种扭曲,这就是我们的看到的希望的那根帆吗?我不这样认为。至于中国人为什么这个时候要去买美国国债,而不买黄金,我有一点想法,但我今天不方便在这个场合把它说清楚。我个人认为

美国结构性的矛盾一个都没解决,美国结构性的问题越来越严重。此次美元的负重深蹲不是历史性的,就是黄金、金银出现调整,有些经济学家认为是历史性的,美元从此以后就转强了,历史性转强了,能不能别这么扯?拜登的经济学是供给学派吗?谁再供给拜登的美国5万亿美元呢?3万亿到5万亿美元,谁供给啊?是潘石屹们能带去的吗?许先生能送过去吗?那几个互联网企业能送吗?3万亿到5万亿美元,能送吗?送不过去。

送不过去,那么就只能收割三个地方,一个叫日本、一个叫韩国,一个叫台湾。收割这三个地方,日本5万亿美元,韩国3万亿美元,台湾2万亿美元,他们加起来大概是不到10万亿美元,然后奉献出他们的一半来——5万亿美元。可能不是一年,比如说分三年,如此可能能缓解一下美国。除此之外,我看不出他有任何办法,因为从中国的收割已于2018年结束了,向中国收割这件事2018年历史性地结束了,这个门不会再次打开。

现在我们要谈一下子日本了,日本能收割吗?因为军洛前一段时间说了,日本会出现日元的剧烈贬值,导致中国的房地产泡沫破灭,结果实情是中国房地产先破灭了,然后日元还没有贬值,没有出现剧烈的贬值。那么日元是升值还是贬值,还是维持原价不变呢?我觉得日本人是吸取了1985年教训的,所以在此轮的过程中,日本人在政治上、军事上极为激进,在经济上非常保守、很冷静。日本人在这个问题上有长进,就是你喊打喊杀没问题,跟中国搞搞事没有问题,我都去做,但再让经济上让我做事情,那就不积极,那就开始怠工了,没有以前那么笨。

能从日本拿走1万亿的……从日本资本回流美国1万亿进入到美国的实体经济中吗?像1991年那个样子,我个人认为不行。因为囤积在日本的美资是有一定规模,但是,但是我看了日本的楼市和股市不经揍,没有1991年那么疯狂,所以他也没那么大的交易量和规模,不行。至于韩国的情况我觉得也不容乐观,虽然韩国的股市和楼市都高耸入云,但是因为韩国本身他的主要的东西在美国人手上,你让美国人卖掉可能还算值钱的东西,

让美国人卖掉,然后回国内接什么东西呢?台湾的情况倒是有点意思,因为台湾的股市如此之高,其中台积电的比例如此之大,出现某种局面,我并不惊讶。因为我不主张采用武力的方式,我主张采用经济的方式,当然这是后话,有些话也不方便在这说太透。我的意思是说,美国的可以有如此规模收割的点,一个是中国这15万亿,一个是日、韩、台这10万亿。除此之外,没有地方可以完成所谓的供给侧的改革了。如果这个改革不能在拜登的前两年铺陈开来,

也就是说,2022年美国仍然无法获得这些钱、这些资源的话,美国将不可避免地在2022年陷入严重的经济危机,再加上2022年是个选举年。我简单再说一下黄金的事情,黄金的比对的关系是什么?就是我们决定黄金涨和这跌的原则是什么?我以前讲过,是实质负利率。因为我们大体上是以美元定义黄金的,那么美元的实质负利率水平决定黄金的价格。美元的实质负利率,是美元的名义利率减去它的通货膨胀率,名义利率趋近于零,而通货膨胀,我说的不是美国公布的通货膨胀5.4。

我说的是计算了股市和楼市还原之后的通胀,美国现在在两位数。我看不出来一个货币每年要贬掉10\%,而黄金对它还贬值,你给我个理由。所以我根本理都不理,所以它跌下来我就继续买,涨上去高点我卖一些,跌下来我继续买。我不认为有什么问题,当然大家可能有在操作上面可能会出现一些技术上的障碍。有些人是借钱了、加了杠杆了,这个我是一开始就跟你们说清楚,我不主张你们这么做,你们自己以为自己真的是高手吗?有些人比较聪明,买了相关联的股票,其中有涉铜,比如说紫金,比如说涉银,比如说涉其它的贵金属的股票,其中也有金,也有保值的,

操作得还是不错的,因为我已经讲过了三种类型来处理这个金的问题。无论是买真金、纸金和股票都可以的,但是可以分别采取不同的手法,其中有些手法是对冲的。我就有时候搞不明白为什么就是一根筋呢?为什么非要在高点倾家荡产去加杠杆去买?为什么低点又斩仓出来哭爹喊娘?这个有的时候就是理解不了。我能理解不赚钱,就是有的时候在风起云涌的重大的年景,你不输就行了。你像今年香港这些我身边这些朋友,50%以上的人都在输,

因为香港你能不买阿里吗?不买腾讯吗?你不可能吧。在香港你不买这些互联网的大家伙,你能不买这个房地产吗?像恒大这些房地产公司在香港那么多的股票,房地产在香港上市公司非常多啊,那么多的股票、那么多的债券,你真能一个都不碰吗?其中有些东西非常之好,回报水平20\%、25\%,你能不……好多人他不能不买嘛。那么今年这些家伙们是不是损失极重呢?有的是一半,麻烦的是百分之七八十、百分之九十都有。那么你说香港人买香港股票可以吧,买汇丰、买长江也……

也要损失30\%以上,持盈保泰是不容易的。至于国内确实是分化得比较厉害。我们也在总结,比如像这些股票我们都买了,比如说比亚迪啊、比如说茅台啊、比如说宁德时代啊,都碰过,但是没有从头到尾这么样的走完。还有一些东西啊,类似于航运类的东西,我们其实知道的,但是有的时候可能做决定受到了一些牵绊,就是处理的速度没那么快、没那么好,包括对当时考虑到金的时候,也考虑到其它金属,比如说铝,比如说锂,铝锂,比如说一些金属。当然,

当然有高手了,如果你在程序上全对,茅台、宁德时代、比亚迪,其中中间又能换手这个航运、再铝,这个操作节奏非常好,抓的点是对的。那么今年翻两三倍也是有可能的。但是那在市场上是少数,反正在香港这边,我身边的人不是全部吧,反正是大部分挺惨的,输得比较惨,能持盈保泰的人就算不错。极个别的,而且不是香港本地人,是国内的人在新经济上面赚一点钱。

我最后结论说两句话,这个结论,我有时候现在不太想讨论投资问题,就是讨论投资问题,有时候会形成很大的压力。但你们都是我的好朋友,我不会因为几个朋友提出问题就不再讨论这个问题。我觉得有两件事情要引起重视,第一个是“胀”,“胀”这个字。通胀会进入到一个非常麻烦的状况,这个通胀不光是美元通胀会非常严重,人民币的通胀在某一个特定区间也会非常严重。所以我们还是要解决这个”胀”字,解决这个”胀”字,我们以前是用的是金的方法,现在看来可以多元化,金是一个根本性的这个依托了,金和银,它具有金融属性嘛,也可以

考虑一个跟“胀”相关联的多元化的一个结构性安排,这是首先要考虑的。第二个部分,看来这个我们也可能需要提前做布局,就是节能减排了,就是碳排放权。就是我们的方向,一个是金,一个是碳,就是这是老早说了。碳呢,以前我们觉得还没那么快,但现在看来可能得加快进程了,加快一些安排。事实上我们看的类似于像中广核呀,一些关于核能、水电,包括一些其他新能源的东西,原材料也好,都有一些变化,其实我们可以多做一点点的关注,不一定非要等到明年、来年再做安排,提前做布局也是可以。特别是,可能,我说的是

可能我们很快会迎来市场的大调整。现在香港这边我们已经开始,大家一直在耳语了,就是美股可能拖不过去了,拖不过去的话,美股可能会出现深调,那么美股的深调到底有多深呢?会不会伴随着秋冬的疫情和经济的急剧的下滑,然后美股出现崩塌,然后引发全球性的这样的一个风暴、经济危机?可能性是有的。那么在危机到来的时候,反正我们也会提示大家,那个时候我们可以开始着手进行最好的选择,补仓。至于你问我,我建议你现在还是持盈保泰,还是按既定方针办。好的,今天就说这么多吧,明天下午三点钟,我们再见。

有什么需要补充的话,明天下午三点钟,我们把它再做一个完善。疫情看来一下子半下子结束不了,身边的在海外的一些人, 又有这个中招,所以有时候常常心理上会有一些说不出的感受吧。大家务必要注意个人卫生,能不出门就少出门,特别是出远门。秋冬到了,安全第一,这是其中一个。另外这个农历的九月十月就来了,热闹非凡,虽说是个吃瓜的群众吧,但是也要多动脑子,多思考啊。好吧,明天见。

\section{资本形变 、《国庆题记》诗解}

大家好,今天是2021年的10月2号,是辛丑年的八月二十六日。先祝大家国庆节快乐,也祝祖国繁荣富强。今天是正式课,我们今天正式地进入到《资本论》的第二卷,今天是第二卷的第一篇——资本形态变化及循环。通常我会把这一篇当成整个第二卷和第三卷的核心部分,就是我管它叫资本形变。我试一下麦,然后我们今天准时开始。

大家好,今天是2021年的10月2日,辛丑年的八月二十六号。时间过得非常快,一转眼从阳历看,我们进入到最后一个季度了,但是阴历还得再过几天才进入到一个最凶险的时间段。我们今天讲《资本论》的一个非常重要的环节,就是《资本论》的第二卷第一篇。其实这一篇就是资本形态变化及循环,是整个资本流转的核心部分,也是我特别喜欢马克思《资本论》的一个部分,我管这个篇叫资本形变。

也不知道是什么原因,今天我在办公室,办公室这个网络非常糟糕,讲完了发不出去,我不行,我就用手机吧。在今天开课之前,我想简单地把我国庆节送给大家的那首诗解一下子。我刚开始写这首诗的时候,我觉得好多朋友应该能读懂,后来发现好多人认为我在感慨、发牢骚,不是的。我明白,可能这里边涉及到一些典故,可能有一些东西不是很容易地就能够理解,可能好多朋友可能想偏了,所以有一位好大嫂说,你过节了,不生气。

我写的这个国庆题记\footnote{https://www.notion.so/c236d70da0344a6b9046c52f9cc7da3c}第一句话是“东坡把酒问君王”。其实我可能跟大家聊过,就是苏东坡和欧阳修他们,他们是当时的北宋的既得利益者,就是他们都是算是大地主了,所以他们是反对王安石变法的。在这个冲突过程中,可能苏东坡被皇上给贬了,但是确实是他们学问不错,文章写得也很好,但在重大问题上出现了严重的立场问题,所以他们一直在反攻倒算,就是想阻止整个的变法。

很多人把那首著名的苏东坡的词理解为给他兄弟写的,就是“把酒问青天”,其实不是问青天,是把酒问君王,那是一首思君的诗。“转朱阁,低绮户,照无眠”,这个天上不胜寒,他这个谈的还是君王事。但是普通的朋友可能理解错了,过于文学了,这是一种这个臣子被贬之后的一种幽怨。后来果然他们成了,最后轮到王安石(王荆公)最后被罢免。我的第二句话是“荆公熙宁黯神伤”,就是王安石的“熙宁变法”最终是以失败而告结束。

北美的朋友,还有国内的朋友都对王安石变法有着一些看法,就是他们认为王安石变法本身是有问题的。其实我将来会写一篇长文,因为我最近腾不出空来,会谈王安石变法的本质和他失败的原因。但是现在没有特别充裕的时间,因为这是作为一个课题留在那个地方,其实值得把这个事情研究一下子。就是前两句是“东坡把酒问君王,荆公熙宁黯神伤”,讲了这个直接税改革的不易啊!是借古喻今。“瘦金不济黎民苦”这是说的是宋徽宗他发明的瘦金体,“瘦金”这两个字是借意,说他你这个金太瘦了,你救济不了黎民的、百姓的苦难。

“宋词风骚引虎狼”,“宋词风骚”是说北宋重文轻武,而且一直对军队、对习武采取一种消极和遏制的这样一个状况,并不是说宋词风骚引虎狼,只是一个借喻吧。第五句是“精忠为民方报国”,这是对岳飞提出一点点的意见,就是精忠如果是报君,那就是不是报国,你也报不了国,精忠为民的时候你才能报国。这里提出一个小小的批评,就是到底精忠为什么?应该是为民,为民才能报国,不为民你当然报不了国,最后风波亭了不是?

“富贵共享真文章”,你写《赤壁赋》,你写那么多东西,感慨了那么多东西,苏轼父子写了那么多篇文章,谈什么呢?谈的不外乎君王技巧。有谈人民吗?有谈共同富裕吗?有谈富贵共享吗?那么好的词、那么好的赋、那么好的文章有意义吗?没有意义。如果我们今天仍然是把那个词章做得那么好,而无视根本性的问题,那算是真文章吗?那有意义吗?你就算是在国外拿十个博士、拿十个首席,有用吗?

“盼得商君废井田,更待诸公再汉唐。”其实我喜欢最后一句话——“更待诸公再汉唐”,这里边“汉唐”是动词。我也知道这些年大家读诗、读词少了,其实我挺得意后两句的,但是我发现大家觉得可能……“盼得商君废井田”,这个“盼得”就是意思我们需要一个这样的人物来帮我们把直接税改革推入进去,或者说我对现在有关部门的这些情况我是不理想的,所以盼得商君。因为没有商君,所以很难废井田的,绕来绕去,他就不去搞直接税改革嘛。所以我才写了这句话“更待诸公再汉唐”。

这是一个期许啊!因为现在这些领导同志还是有理想的嘛,既然是有理想的话,那你就应该、事实上也应该使中国重新成为世界主流,再次崛起。那么中国的一个辉煌的顶点就是汉唐盛世,所以“更待诸公再汉唐”。“更待”的意思是现在还没有,所以希望能够在这个方面能够做好,所以“更待诸公再汉唐”。其实这首诗有史、有意境,也是有气势的。可惜可能很多朋友没读明白,有不少朋友有意见,倒是文木先生觉得是个好诗。

好,诗的事情就说这么多,我们回到今天的主题。今天虽然是过节,我们还得按大纲走,因为时间很快,所以我们今天讲《资本论》的第十三讲:资本形态变化及循环。我管这一篇用四个字概括叫资本形变,大家就记住资本形变就行了。因为资本形变其实是《资本论》里边一个重要的内容、非常重要的环节。在某种意义上,资本形变比剩余价值更重要,或者是剩余价值要说的就是资本形变。我想可能进入到资本形变就进入到《资本论》的一个高潮了,另外对我们的投资可能就开始有一定的指导意义或者是有一定的启发的价值了。

今天我们大概分五个部分——资本形变。第一个部分我要讲资本的循环。资本循环有一个资本循环的公式,就是G—W⋅⋅⋅P⋅⋅⋅W’—G’。什么意思呢?实际上马克思说的是货币资本的循环。今天我们讲第一部分货币资本循环,就是货币购买变成商品,然后商品进入到生产阶段,P是剩余价值创造剩余价值,然后再形成商品就是W’,然后再卖掉形成货币这样的一个公式,实际上是一个流程。这个流程貌似简单,其实里边内容丰富,这个含义也比较深一下子,我简单做一个解读吧。

这里边包括了三阶段:第一个阶段所谓的货币资本阶段,就是拿货币资本将它进行生产力、生产资料和商品的购买,准备进入生产资本;那么第二个部分就是G—W的部分就是货币资本变形形变的部分,P的部分就是生产资本形变的部分,W’到另外一个G’是商品资本的形变的部分,它形成了一个三循环的结构。第一个循环G—W⋅⋅⋅P⋅⋅⋅W’—G’,这是第一阶段货币资本的循环;第二个循环开始是P⋅⋅⋅W’—G’—W⋅⋅⋅P。

什么意思呢?就是在生产资本循环里边是从P开始的;最后商品资本是W’—G’—W⋅⋅⋅P⋅⋅⋅W’,从W开始,它是三个循环过程。其实你大概听一下就行了,因为也不需要知道得那么细。资本形变它对我们理解资本在不同的环节具有重大意义,因为一会儿我们要讲资本形变,要讲一下台积电到底发生了什么事情,就是资本形变的过程中会发生什么。货币资本它部分的实际上是生产资料和原物料,生产资本它其中核心的内容是剩余价值,那么商品资本处理的市场就是实际上商品定价问题。

处于不同的形态的资本,资本利得是不同的。这句话是非常重要的。在资本形变里边,我们一定要知道处于不同形态的资本,就是货币资本、生产资本和商品资本,不同的阶段的资本的资本利得是不一样的。有时候差异会巨大,有的时候甚至生产资本也会被货币资本和商品资本剥削,就是资本家也可能成为被剥削的对象,比如台积电现在遭遇到的问题。因为货币资本要对生产资本加以某种情况的控制,商品资本也要对生产资本加以某种情况的控制,因为剩余价值在生产资本。

记住这句话,就是随着分工细化,资本的分工也出现了细化。剩余价值所在的那个部分就是生产资本的部分,由于它生产资本整个的流转是透明的、是清晰的,并且它受上游和下游的控制,受货币资本和商品资本的控制,所以它实际上是没有多少空间的。以至于随着社会分工,西方国家特别是美国会将生产资本的部分放掉,允许其他的国家来做中间这一部分。为什么我们会成为制造业大国?其实在资本形变里,马克思已经把它说清楚了。

马克思《资本论》的第二卷的第一篇这个资本形变的逻辑,也是指导中国人如何应对现在这个中美冲突的一个重要的理论依据,那么我们就要仔细地研究到底资本利得在哪些部分。因为你知道在中国目前经济学家讨论的是产业生态、产业链,很少有人谈资本生态和资本链。如果你不讨论资本生态和资本链,你讨论产业生态,你会自己主动地进入到误区,甚至会自己主动地跳入到一个陷阱里边,这是非常糟糕的。所以请大家牢牢记住以下一句话,所有政治都是资本形变的哲学解释。

我重复:所有的政治都是资本形变的哲学解释。这不是马克思说的,这是我概括恩格斯在序言里边的一段话。但是恩格斯不是这样说的,我觉得我这样说好像更清楚一点。那么资本形变的目的是什么?资本形变的目的当然是为了获取利益。在马克思的时代,资本形变从货币资本到生产资本到商品资本,通常可以由一个资本主体或者一个资本家来完成。但随着分工越来越细,变成全世界分工的时候,那么就必须有取舍。所以美国主要控制货币资本与商业资本,货币资本的部分它可以控制整个市场资本流量,商业资本的部分主要是控制商品定价权。

谈到这儿呢,我们就要谈一下子为什么在西方世界不是全产业链。他们为什么主动放弃全产业链?因为全产业链的里边,其中特别是中间的部分涉及到了剩余价值、涉及到了剩余价值剥削。由于马克思《资本论》的写成之后引发了整个的西方世界的思考,就是西方在进行社会主义改造,那么西方以英、美为代表的社会主义改造就要求提高劳工待遇,所以有了工会、有了各种各样的工人运动。那么怎么办?那么这个部分不适合再放在他们手上了。

在西方,他们原来把这块放出去的时候,他有两条他认为他们是有把握的:一个就是没有资本供给,发展中国家生产资本没有源泉,所以是可控的。就是我让你发展什么,发展到哪里,你就只能这么多,不然我就给你切掉资本来源,你就死了。另外一个它控制了商业资本或者是商品资本,那么实际上是控制了市场、控制了需求、也就控制了定价权。那么有没有可能发生一些失控的问题呢?现在出来了。失控的问题就是制造业一旦在某个区域形成体系之后会迅速升级,就是现在目前遇到的是中国、韩国和台湾。

有可能我会没注意会切掉,但我尽可能地把讲完整了。什么意思呢?当中国在生产成本越来越发达越扩张的时候,中国的资本稀缺问题已经就解决了。中国资本稀缺的问题彻底的解决,我个人认为是2014年、2015年就彻底解决了。就是中国的资本不稀缺了,中国的生产资本不但不稀缺,还开始外溢。中国开始进入到货币资本和商业资本这两个领域里边去,可以向两端延伸,形成了与西方的在资本环节的竞争。记着,不是生产环节竞争,是资本环节竞争,就是我们开始想要去拿货币资本和商业资本的利润了。

没有贸易战,哪有贸易战呢?是因为当中国资本向两端延伸的时候,要压缩西方在货币资本利得和商品资本利得的空间的时候,爆发了剧烈的冲突。是资本与资本的冲突,不是劳工与劳工的冲突,不是民族与民族的冲突,不是种族与种族的冲突,也不是国家与国家的冲突,是生产资本与货币资本和商业资本的冲突。马克思厉害吧?马克思150年前把所有的事情看明明白白的。好了,我们现在简单回到第一卷剩余价值的部分,看看资本是怎么进行剩余价值收割的。

通常收割剩余价值并非仅仅存在于生产资本,生产资本是主要收割剩余价值的,但是生产资本有时候也是成为货币资本和商业资本的韭菜。那么我们概述一下子,全资本链收割剩余价值的方法:一种收割在价内,就是物价里边包含了剩余价值;一种收割是汇率,用汇价的不平等来收割;第三种叫资本转移,就是高净值,就是你们这个国家赚了,生产资本赚了钱之后,通过房地产、通过资产溢价或者股票溢价形成高净值,再转回到货币资本拥有者手上。记住这个逻辑过程。

就是总资本对剩余价值的收割有价内,价内就包含了生产资本的收割;有汇率,这主要是指货币资本的收割;有资本转移,这还是货币资本拥有者的收割。其实对制造业大国的收割,价内的收割越来越困难。举例,今天,对吧?定价权不在商业资本手上,因为严重稀缺,所以没办法,价内剥削非常困难,那么就通过汇率,汇率收割有的时候也不一定顺手啊,那么就是资本转移,主要是强迫高净值转移。很多人认为让台积电和三星去美国设厂,这种资本转移它是因为高净值转移不顺当,才出现了强迫性转移。

我们注意到就是整个的冲突和危机跟现在他们所制造的什么意识形态没什么关系,这个首先它不是主义之争,也不是民主与独裁之争,它是资本利得之争。资本利得在发生冲突的时候,即便是韩国、台湾都采用的是资本主义制度,甚至很多人都信的是一个上帝,照打不误。因为台湾和韩国是生产资本,生产资本在某些方面由于高科技的发展速度极快,形成了对商品价格定价权,就是对商业资本的这种反控制力,这个是货币资本不能接受的,所以他要采取措施,而且这个措施会非常非常严厉,越来越严厉。

马克思本人就说过,就是资本和资本的冲突必然会导致国家与国家的激烈冲突,就是所谓的这一次大战、二次大战和现在目前可能要延伸出来的问题,实际上是资本争夺的必然结果,跟什么意识形态、宗教、种族没关系,那都是借口。现在这个情形是越来越明确了,所以资本形变的这个里边我们可以得到很多很多有意义的启发。好,我们现在进入到资本形变的第二个内容,我们讲资本形变的含义。资本形变的含义是非常丰富的,有几重含义吧,我把它概述一下子。

资本形变里边呢,你知道资本形变是很有趣的事情,就是原始积累怎么来的就是资本形变的第一重含义,我们叫主体异化。什么叫主体异化呢?其实你看中国就非常知道,就非常清楚,就是所谓私有化的部分它原始积累怎么来的?它将国有企业给私有化了吗?或者是把其他人的东西给私有化了吗?把银行的贷款给私有化了吗?它就进行主体异化,就是跟它没关系,是别人的,将主体异化形成资本形变,这是资本形变的最重要的部分,也是最核心的部分。我们通常把这个私有化分为三个不同的层级:一个叫直接私有,就是直接把国有企业弄回家,这土地算它的,直接将公有的私有;第二个是信贷私有或者是信用私有;

第三个部分才是集资,或者是将集资就是共同拥有的款项私有化。这其实主体异化在恒大这儿,在马云那儿都看得很清楚,主体异化是资本形变的第一重含义。或者是我们开始理解为什么有钱人开始有钱,怎么有钱的?实际上是主体异化。就是资本的主体异化了,这个不是它的,但是它通过某种制度和政策的安排完成主体异化。主体异化是我们研究《资本论》里边非常重要的内容,我们必须懂得主体异化,我们才知道他们所说的神、英雄、了不起的人物、超人,其实是主体异化过程中的一个现象而已,没有什么。

第二个意义是在生产环节的,我们管它叫P部转移,也叫外挂,实际上就是对生产环节增值部分的窃取。在生产环节,由于某些商品,比如说房地产,存在着对货币政策严重扭曲,存在着一种非理性的成长,那么这个环节的增值的部分会被大量窃取。你看恒大你就懂了,他们家为什么包工,他们家为什么把在这个整个的P部环节进行了大规模转移,就是恒大赚的钱,恒大的大股东拿到的钱,很大一部分是P部转移,通过外挂走了。这个研究是非常有意义的,有价值的,它是管理失败。

第三个意义是W部转移,就是商品定价权之争。商品定价权之争有两个含义,一个是稀缺,比如房地产,实际上假稀缺,伪稀缺,但它能通过货币政策或者是行政安排制造出来。香港就是经典的这个假稀缺,因为土地是有的,它假装稀缺,所以形成一个定价权之争,因为土地都被他们垄断了。另外就是另外一种转移,就是所谓的产能过剩。产能过剩了吗?永远是相对过剩,不是绝对过剩,因为还有好多人要这个东西拿不到,它是分配出了问题,所谓产能过剩。W部转移也是非常严重的问题,其实这涉及到商业资本的问题了。

最后一个部分是G部转移,就是股权转移。G部转移、股权转移其实就涉及到了在更高端的定价权,就是实际上是股权的定价权。这里边,我们看到G部转移的时候就知道,为什么我们要终止滴滴,我们为什么要对滴滴打车、对一些海外上市做某种程度的控制,因为它涉及到G部转移。G部转移里边有合法性、合理性两层,合法不一定合理,更不一定合情;合法、合理也不一定合情。这里边有很多很多的问题需要深刻的思考,但我们今天不能展开所有的资本形变。

在讲G部的时候,G部转移的时候涉及到我们,我们在G部转移里边。就是G部转移实际上就是股价波动,就是资本市场的波动。这个波动就是G部转移的基本的工具,就是不波动就没法收割,收割就是波动。那么我们研究的就是G部转移,就是在G部转移里边获取利益,投资学就是这个意思。马克思厉害,老马!我再重复一下,今天我们讲的资本形变的四重含义:第一个是主体异化,这件事情我们不去做它,因为这个主体异化缺德,将来可能好多人也是犯法就犯在主体异化上面,现在这个事情还在清算。第二个是P部转移,就是外挂,这个很多很多包工头他们内部的转让这些东西可能都有大的问题,但P部转移大体上是可以合法的。

W部转移,商品定价权之争。这个事情涉及到政府的,政府治理失败,比如房地产、比如产能过剩,这都是政府治理失败。但这个失败很多时候是故意的,无论是房产定价权还是这个钢的定价权、钢铝定价权,这里边的失败完全是有关政府部门跟货币资本勾结、跟商业资本勾结形成的制度扭曲。一会儿我们会讨论资本形变的最后一个部分,会讨论到这个问题。我们首先开始学懂、搞明白它整个的结构。至于第四个部分,就是它的这个四重含义里边的第四重含义,就是这个G部转移。G部转移我们就是股权转移的部分,这是我们重点研究的部分。这个如果时间够,我将来会专门拿出一堂课讲G部转移。

资本形变大部分的时间,说到这儿,其实马克思挺厉害的。资本形变在东方,大部分的资本形变是不合法的,更不合理,不合法也不合理。但为什么不合法、不合理还成了呢?因为资本积累的过程有它内在规律,有它内在规律。要不就不会说资本是每一滴都、每一个毛孔里都带着血,它确实是一个非常残忍的这个形变过程。在西方,由于法律体系——资本主义的法律体系逐渐的完善,所以西方的资本形变大部分是合法的,但不合理,肯定不合理,是合法但不合理。东方,就既不合理也不合法,甚至是大部分都不合法,不能追究。

这里边就是中国现在正在转型期,转型期就是要解决资本形变中不合法的部分,这就是新时代到来我们面临的挑战是什么?就要处理资本形变。过去不合法部分形成的那部分资本怎么办?另外就是以后让资本形变逐渐合法化,就是四种形态的转变,让它慢慢合法化。这里边我们就提出财政治理失序的问题:第一个部分,国有资产管理的问题,为什么国有资产里边的好多部分,有形的、无形的、直接的、间接的就被私有化了呢?再一个就是信用,信用被私有化、信用主权被僭越的问题。再一个就是定价逻辑,房地产定价逻辑是可以、政府可以干预的呀。

最后一个部分就是股市炒作。股市炒作就是G部转移的这个问题上面,G部转移这个问题上面,国家是应该有系统的、完整的法律使G部转移相对合理、绝对合法,就是要绝对合法、相对合理,不能让它不合理、不合法变成赌场,这个国家治理就失败了。这主要是财政治理的失败,甚至不是经济治理的失败,主要是财政,因为这涉及到财产转移,在美国、在英国都是属于财政部管理的范畴,银行不管这个的。但是我们现在目前财政职能被严重的切割、分割、切碎、撕裂,然后它变成了失能。另外,财政里边最严重的问题就是税政极度扭曲,税政首先它是漏洞百出、极度扭曲,还挺重。

说到资本形变的这个这四重含义,实际上是四种转移的主要方式。有人把它分成六种,我一般把它分成四种,我不想过于复杂。六种的原因是他们把马克思说的资本形变的三个环节,每一个环节里边分成两块,细一点也是可以的,但我觉得太细了反而容易搞得不清晰。我讲《资本论》的目的是将马克思的思想方法和他的基本的逻辑过程给大家,可能有一些分析工具也拿出来,但主要的是你将来可以用。老马这个人太厉害!在资本形变这个问题上,他思考的这个深度确实是所有经济学家、包括现在的经济学家所不及的。

我说下边这个第三个部分这是马克思本人没说,在《政治经济学批判》里边有,恩格斯倒是做了一点总结和归纳,这是我这个把他们结合起来的一个看法,就是资本形变的道德意义。资本形变里边当然涉及到严重的道德问题,在西方为什么——西方的资本主义发达国家都是基督教新教的国家、基督教新教,基督教新教特别是信义宗大体上就是共产主义的发源地,它具有某种天然的社会主义属性,所以基督教新教伦理对资本形变是有某种约束意义的,在道德上它有、它是一个道德对资本主义制度进行约束的这样一个伦理基础。

我想简单概括为三条,这也是马克思自己概括的,就是生产资本的剩余价值具有种族与国家属性。因为,这好像好多朋友可能听不懂我们在说什么,因为西方人对同种族、对自己的国家往往表达出了一种并不那么凶残的剩余价值剥削的态度,他就把这个凶残放在对其他种族和其他国家进行凶残的剩余价值掠夺。我说清楚了吧,这是马克思的原话,就是生产资本的剩余价值具有种族与国家的属性,它实际上是,怎么说好呢,这个我不做解释吧,就是你们知道。

现在的整个西方国家走上今天这条路,它里边是有某种道德原因导致的这种结构性的问题出现。他为什么放掉生产成本、去工业化?就是他更希望剩余价值的制造是在非他种族、非他国家的那些种族、那些国家,比如黄种人、亚洲国家来创造剩余价值给他剥削,这是他更希望见到的一个结果,就是他有这个种族与国家的排他性。其实德文在论述这个问题的时候非常复杂,《资本论》第二卷里边,恩格斯在总结这一段的时候,我觉得讲的也是我想很多朋友可能读不懂吧,我概述一下,道德意义里边第二个部分,就是货币资本仅具有阶级属性。

好多朋友可能又糊涂了,这也是马克思的意思,就是货币资本仅具有阶级属性,它没有别的属性,它就是个阶级属性。因为货币资本(货币资本也就是金融资本)就是阶级压迫的工具,当然货币资本也可以变成民族压迫的工具,那是被人利用了,因为货币资本本身它根本不讨论国家的,也不讨论种族的,它就是阶级压迫。这个第二个部分和第一个部分好像有冲突,那个是讲生产资本,这个讲货币资本,为什么生产资本有排他性?而货币资本没有排他性呢?这是个哲学问题。

第三个还是个哲学问题,就是有没有全资本链共存的可能性?或者说,有没有全产业链共存的可能性?这是一个非常严峻的问题。这个问题既提给今天的西方国家,也提给今天的中国。就是我们所说的全产业生态能存在吗?我们现在所说的产业链和全产业生态只是单指生产资本,我们没有说货币资本和商业资本。如果是货币资本、生产资本和商业资本全产业链共存,并形成历史性的均衡,那是什么意思呢?那个意思才是习主席说的人类命运共同体,那个意思才是《新社会主义论(2014修订版)》的……

讲到这个地方,你们懂得了我为什么要谈资本形变的道德意义,因为《新社会主义论》讨论的不是在生产资本过程中,生产资本的共有或者共享,不是的,是全资本生态或者是全资本链,从货币资本、生产资本到商业资本,全资本链或者是全资本生态的一种共存共荣这样的一种可能性,这可能性有没有?如果有,它是应该是个什么样子?这个也就是写《新社会主义论》里边所涉及到的一个非常严肃的问题,因为《新社会主义论》里边谈的主要是资本的问题,社会主义离开资本就无法讨论下去。所以好多左翼的朋友,因为我相信大部分左翼的朋友没有读过《资本论》的。

我在写《新社会主义论》的时候,从广州专门来了一个左翼的老同志,我印象里是在粤海饭店,我们俩进行了长达四个小时的辩论。他来的时候,是把我的《新社会主义论》全部反复读了,然后里边写了很多的问题,然后我们俩讨论。讨论的过程中,我发现了两个非常尖锐的问题:第一个问题是,他没有读过《资本论》,或者是没有全读吧,他读的应该是课本,就是属于中学课本上的政治课上的东西,那个与《资本论》相去甚远;第二个问题是他没有读过《通论》,他也完全不了解当代的经济学,他是一个从事教学工作的一个同志,书读了不少,但在这两个领域是空白。

他找我来辩论的时候,更多的是一种政治立场或者是政治角度的辩论。后来我跟他说得非常清楚,讨论立场在某种意义上来讲,我说“立场不是一个出身问题,你不要认为你的出身是无产阶级,你就有资格来讨论社会主义,请把自己先放到一个合适的位置上去,你也不要认为这是个资本家,他没有资格来讨论社会主义,因为社会主义有立场问题,但不是立场决定,不是立场决定”,然后我逐条逐条做了解释,为什么社会主义必须有私有化?为什么社会主义必须有资本?

我相信,他当时虽然还有保留,但大体上是听懂、接受。为什么有新社会主义和旧社会主义之分呢?因为旧社会主义是国家资本主义,不是社会主义,真的不是社会主义。所以我不能说它、直接就说它是国家资本主义,所以把它叫成旧社会主义,是列宁《国家与革命》指导下进行的、国家掌控全部资本的、一个历史性的伟大实践,但那个不是马克思《资本论》或者马克思他理想中的真正的社会主义,真正的社会主义还需要时间,所以我们提《新社会主义论》。《新社会主义论》里边的核心实际上是资本的社会化水平。

如何来理解资本的社会化水平?就是今天我们讲的。我们刚才讲哪去了?我们讲的第三个部分,资本形变的道德意义。资本形变的道德意义,就是我们在探讨金融资本或者是货币资本与生产资本与商业资本或者是商品资本,三个资本有没有可能形成一个完整的资本链或者是资本生态,而这三个资本的资本利得达至某种妥协和平衡(而不至于像现在金融资本形成寡头垄断,形成对生产资本的剥夺、形成对商业资本的剥夺,或者商业资本形成对生产资本的剥夺),形成资本的某种均衡,而这个均衡让社会上大多数人获益,《新社会义论》的核心焦点所在。

好,时间过得快,我们讨论第四个部分,资本形变的经济学意义。资本形变在经济学上是具有重大意义的,因为资本拥有者总是希望资本利得最大化,或者是他们的利益最大化。资本利得最大化一定会产生与国家利益和人民利益的激烈冲突,当资本利益侵害到国家利益的时候,国家必然衰落,侵害到人民利益的时候,老百姓就必然走向贫困、贫富分化。现在美国的金融垄断资本主义正在伤害美国的国家利益,正在伤害美国人民的利益。

很遗憾,就是资本形变的经济学意义里边,今天我们讲第一个部分,就是关于金融资本与国家与人民利益的冲突的部分,就是这实际上是一个均衡的过程,经济学讲的就是均衡,但是这个均衡被打破了,它被里根打破了。打破了以后,其实美国在二战之后建立的这个与国家的均衡与人民均衡是不错的,后来被打破了,苏联解体之后彻底打破了,那么就出现了重大的问题,这是一个经济学需要回答和解决的问题。经济学回答和解决了,其实美国的那种优势(道德上的优势)就不存在了,可惜我国的经济学家他不回答这个问题,所以我们每天中国遭全世界,特别是英语国家在道德上谴责,当然了我们也需要解决这个问题了。

经济学第一个意义,就是资本拥有者与国家与人民的利益所得均衡的问题,这是一个特复杂的问题,就是经济学必须回答的问题。什么叫均衡?均衡点在哪里?必须回答。第二个问题,国家总体发展的均衡,因为国家总体发展它有不同阶段,我们在资本稀缺的时候,实际上是我们必须在生产环节,甚至在商品环节,做出某种妥协和让步,这样的话才能使货币资本或者金融资本进入,解决我们资本稀缺和市场稀缺,所以我们必须向两头妥协,一个是向金融资本妥协,一个是向商业资本妥协,因为它有市场嘛,它有需求嘛,那么我们这个妥协和需求需要有一个均衡点,我们什么时候不妥协了?这个历史的均衡是经济学要回答的。

或者说,这是需要我国经济学家给予回答的:就是我们现在处在什么阶段——阶段论。我们还稀缺嘛?我们既然不稀缺了,那么为什么要外汇?我们为什么要出口、要外汇?我们为什么要牺牲?第一个回答;第二个问题就是商品资本,我们为什么要过剩?我们为什么要这个市场的控制?为什么会出现这样的情况?出现一个晶片稀缺,而其他不稀缺,到底我们的问题出在哪儿?就是国家总体发展的均衡,在不同历史阶段要做不同的处理和安排。经济学家不能回答这个问题,你做什么经济学家?第一个是资本与人民和国家的均衡,第二个是发展阶段的均衡,这两个问题是经济学家必须回答的问题。

马克思很厉害,马克思在资本形变里边他提出来,其实这是资本形变,研究资本形变和研究应对资本形变的策略,是我国经济学家的首要工作。我国的经济学家不是在微观里做会计,算哪只股票赚钱或者赔钱。第三个部分,资本形变的经济学含义就是劳动者就业与收益均衡的问题。其实劳动者就业和劳动者收入是矛盾的,就是劳动者的充分就业、收入增加可能跟产品的价格——就是跟商业资本会产生某种冲突,那么这个地方有个均衡点,就比如说我国吧,现在劳动者要达成充分就业,可能就需要低工资,就是收益低工资、充分就业,然后……

低工资、充分就业,然后再匹配一定的社会保障,形成一个均衡。这个均衡点是不断上移的,不断上移的,这个均衡点、均衡曲线有一个不断上移的过程,你既不能过快的——比如说前任总理搞《劳动保护法》,你这个《劳动保护法》有没有道理?合不合时宜?政治正确是不行的,因为实现均衡是需要一个历史节点的,在这个历史节点完成均衡,简单的政治正确是带来很多问题的。所以像五险一金,那个怎么叫保障呢?那个是叫新增税赋好不好?就是经济学家必须要解释,这个均衡点在哪里?如何实现这个大均衡?不是新增税赋啊。

讲到这个地方我又有点略有激动,其实是我对我国的经济学家非常不满意,首先我对我国的经济学家非常不满意,其次我对我本人非常不满意,就是其实这个是应该在《广义财政论》和《新社会主义论》里边,彻底解决和回答的问题,但是我确实是也是做得不好,因为均衡——这三个均衡都需要数学模型和计算,这不能说的,要算。解决算法是我的大问题,就是我的数学不行,模型要经得起考验,所以……。其实问题提出来资本形变的经济学含义,这个含义里边都是均衡,都是找均衡点,这个均衡点找到了就可以测定谁做的对、谁做的错,也可以知道我们国家发展阶段,未来怎么走,但这个算法非常重要,都是非常难的,确实非常困难。

母亲活着的时候总批评我说:你呢,想问题可能快,但尽可能的自己做,就是你发现了问题尽可能自己做,去提出解决方法,而不要对现有的、已成的东西进行猛烈地批评或者是批判,因为肯定是错了,但,你给条路走啊,你那条路算好了吗?其他经济学家不干这个事情,是因为干这件事情是没有好处的,因为这涉及到宏观经济学最核心的内容了,资本形变是宏观经济学最核心最核心的内容,就是马克思花这么大心血研究的东西太难了,真的太难了,就我自己都感觉到实在是力不从心呐。

尤其是到了二十一世纪,进入数字经济时代,其实资本的形变,变得极为复杂了,因为生产者,开始脱离了工厂了,就是比如说像我就是个生产者,这个时候,整个的过程就变得复杂。而这个时候你说经济学解释刚才我提出来三个问题,这三个问题都解决了,实际上宏观经济政策的最佳选择就出来了。但这三个问题涉及到非常复杂的数学模型和算法,而且这个数学模型和算法要具有两个维度的适应性,一个是历史的适应性,就是不是光算今天,要算三千年都是对的;第二个不是算中国,全世界都可以用,这个难度太高了。

好,最后一个问题。我把它这样的叙述的——就是伟大的国家,伟大国家治理就在于合理地、系统地、有效地控制资本形变,就是控制好资本形变,这是伟大国家治理的一个重要的标志。这里边涉及到一些内容,大体上涉及到三个层面的内容:第一个,一定要确保生产资本的廉价与收益,这是一个矛盾,就是我们是处在生产环节。中国的经济为什么会出问题?

2008年之后,我们开始进行了扩张,扩张的过程我们实际上是用两头挤中间,就是货币资本和商业资本都开始挤压生产资本,生产资本的获取变得困难,所谓中小企业贷款难,生产资本不廉价。另外生产资本的利益反复被压榨,所以出现了去工业化的过程。一个伟大的国家当然要确保生产资本的廉价与利得,如果你能确保生产资本的廉价与利得,你这个国家才有机会持续发展和生存下去。做得比较好的就是瑞士、德国或者是北欧国家,西欧部分国家和北欧国家他们做的比较好,确保生产资本的廉价与资本利得。

第二个部分,必须严厉控制货币资本的盈利,其实这个话实际上就是我们以铁腕的政治手段,反对金融垄断资本主义。我们不是反一个国家,就是在我们内部,也要严格地控制货币资本的盈利,就是反对金融垄断。这个,实际上涉及到税政、投机监管和国有的规模,必须税政改革——直接税,必须严厉监管投机,必须将那些涉及到重要的、垄断的部门全部国有。

说到这里可能又要得罪一些人喽。我今天这个最后一个部分不展开,就是第五个部分的最后一个,就是对商业资本的运营进行强有力的均输平准,就是实际上是价格管控,均输平准。这里边,其实最近的这个电的问题就是这个问题,这个问题了。就是外交部、国防部讲政治,他们就是要制裁澳大利亚,但是商业部门是不是要制裁澳大利亚?应该跟外交部讨论,买澳大利亚——廉价进口澳大利亚的煤就是对澳大利亚的制裁,廉价进口澳大利亚铁矿石就是对澳大利亚制裁,而不是不买,好吗?

国防部可以不用算法来确定看法,外交部可以不用算法来确定看法,但是所有的经济部门必须算法来决定看法。就是我认为你做错了,我要惩罚你,我惩罚你是让你受损失。我惩罚你是我受损失,这个外交部的部长要送去学经济学,国防部部长要送去学经济学,因为本来是他得罪你,你要修理他,结果你自己受伤了,这个问题就不好玩,不能这样,所以当澳大利亚惹毛了中国政府的时候……

在疫情期间,我们以低于成本大量进口澳煤和澳铁矿砂,那就是对澳大利亚最大的惩罚,好吗?而不是不进口,让他积货,然后今天高价卖给我们。哎,好吧,感谢卡尔•马克思,感谢恩格斯。《资本论》写得真好,特别资本形变这一堂课,是蛮有意思的。其实资本形变刺激了我很多的思考,就是我写《新社会主义论》的时候、写《广义财政论》的时候,里边涉及到资本的部分,其实大部分是资本形变的部分,因为你剩余价值到了资本形变的时候,你才能有了充分的体会和理解,所以可能这一堂课不足以吧。

另外我也是建议,我甚至建议我公司的员工,因为《资本论》第一卷你可以不读的,因为我们课本里边都教了,但是资本形变,就是第二卷,你还是要下功夫去看一下子的。当然读这个《资本论》第二卷的时候,有可能要读《政治经济学批判》,要结合马克思其他的著作一起读。直接读可能好多人不知道他在说什么,不知道他在说什么,每个字都认识,但是读完了不知道他们在说什么。就跟《通论》似的,好多朋友读凯恩斯《通论》说,这个没什么呀,他有那么大意义吗?其实有的时候跟读经有点像,你说每个字都认识,读完了以后他说这没什么。好吧,今天这堂课就到这儿,然后我聊几句……

耶伦急了,真的真的急了,这个事儿不是她,病根不是她作下的,但是事儿她得处理。就是新的财政年度——这个拜登的财政年度开始了,今天是第二天。因为从10月1号到来年的9月30号,才是拜登的第一个财政年度。第一个财政年度,坏消息就来了。我前两天说过,美国的财政收入大概只能有三万亿,而美国的财政支出肯定超过六万亿,那么就是一半的钱要印了,要发债了。那么现在他如果不取消债务上限,

不光是美国政府能不能开门的问题,主要是美国整个的国家机器的运行都出现问题,甚至如果包括那些基建的安排不能如实、及时到位的话,那么美国经济可能会出现结构性的危机。但经济学都是讲均衡的,这是一个历史性悖论。那么真给你敞开大门,这个放了绿灯,你开始印,那么拜登可能会超过奥巴马,奥巴马上次印了三万多亿,那么拜登真有可能印四万亿啊。那意味着什么呢?

那就意味着美元会出现严重的问题。其时美元现在已经面临历史性的大考。我就觉得我国的经济学家很有问题。我前两天回答了一个问题,回答了一个问题,我在这简单说两句,是北京的一个朋友,就是还是某研究机构的一个研究员说,就是“人民币不能升值,人民币甚至要贬值?”。那么关于人民币该升值还是该贬值的问题呢,在一个群里产生了激烈争论。我的答复是这样的:“如果直接税改革完成了,我们当然要人民币升值,我们人民币为什么不升值呢?如果没有直接税改革,我们当然不能升值,因为我们没有直接税改革,我们升值了,高净值立刻就走“。

请记住,今天我们讲资本形变里讲了,就是剩余价值掠夺的三个方法,最后一个就是高净值。请记住,如果你们国家的这个栅栏、这个门槛、这个围墙没扎好,你升值就是导致高净值出走、资本流失、生产资本全部没了,那么你这个国家去工业化,一夜之间去工业化,就回到农业国,整个拉美、整个的中东、整个的前苏联东欧地区,就是这么个玩法。后来他们玩了日本,现在、最近马上他们就会折腾南韩和台湾,不就是让高净值走资,所以我说这个问题不讨论。

2022年直接税,包括,特别是离境税出来了,所有关于资产类的税、离境税,全部出来了,我们对高净值有了完整的管控工具,人民币当然要升值了。美元与人民币的比价关系1:3,美元1人民币3,不过分的。我们创造价值的能力跟他创造价值的能力差太大了吧。好多朋友说;“那么黄金还有意义吗?”当美元不能再作为公认的价值尺度的时候,请告诉我谁来作为价值尺度?是欧元吗?是日元吗?是英镑吗?是今天还如此不自信的人民币吗?当然是黄金。

有一个很大很大的网红,在互联网上说:“黄金、白银总量有限,不可以作为交易媒介。”这话说的一点都不错,没有人让你拿黄金去买酱油,没有人让你拿黄金去茶餐厅吃饭,没有人让你这样做。它是个尺度,它是一个备兑支付手段,是尺度,是备兑支付手段。之所以这些年1000盎司不能买一套豪宅,黄金被严重低估,是因为美元在大家心目中替代了黄金的地位,如果这个信仰破灭的话,就会出现……

如果美元信仰破灭的话,就会出现一个离乱的时候、离乱的周期。在新的货币替代黄金成为大家的通用手段或者是价值、价格尺度的时候,之前黄金将有一个历史性的周期。我对美国这一代人是失望的,对美国这一届的领导集体也是失望的,除了耶伦以外,其他人真的看不上。他们能否走出美元危机呢?我个人非常非常的怀疑,我个人认为他们走出危机的概率连10\%都没有,90\%美元将在未来的一个时期之内出现严重的危机,我的判断。

那么我们要做什么其实也说清楚了。好多朋友说“短股长金”,这“长金”多长?“长金”可以很长,但“短股”也可以很短不是?其实我选了一些东西,因为我觉得中国的A股快接近调整的尾声了,也可能到2022年二十大我们的直接税一旦出完这个A股就是十年牛市,那个时候之前,我们会提出对行业、产业的一个看法,但我们不会再去给大家提供股票号码,太麻烦。关于金和银的事情,或者是金、银类的投资品种和对货币波动过程中的这种保值和增值的问题,

我想方法也教了,四矩阵也懂,图形的一板斧也懂,大家自己去处理吧。说太多等于没说,其实原则问题说完了,你信就信,不信,你就自己确立你的方向和原则就ok了。我想说的是,我们讲了这么长时间心学,其实目的一个就是建立主体性,建立主体性的意思就是你要用你的眼睛去看世界,要用你的……我上堂课讲风水,要真正的调动起你的真正的意识,

而不是耳识、眼识、鼻识、口识、身识,外部来的东西构成你对世界的一个偏见,那就不好了。还是所有的识都把它抹掉,是你看世界,形成你的看法,这个时候大体上应该就没有什么太大的问题了。我想我们能够在资本形变的过程中,抓住那瞬间的可能性、瞬间的机会。好吧,再一次祝大家节日快乐!另外呢,马上就到农历的九月了,多保重。

在国庆大假期期间,大家放假、出行、玩耍都要注意安全,尽量的注意好疫情防护,注意好个人安全。另外进入九月之后,这还有几天时间,持盈保泰吧,因为我觉得后边这两个月是比较麻烦的,就是还是我们原来说的话,好吧。明天下午三点钟我们腾出时间来交换资料,有未到的部分我们再做补充。好,明天见。

\section{资本周转的秘密与经济危机、求是网《扎实推动共同富裕》}

大家好,今天是2021年的10月16号,农历辛丑年九月十一日。今天是正式课,《资本论》第十四讲——资本周转的秘密与经济危机。今天这堂课,第一重要;第二有趣;第三它可能有实战价值,就是到了一个历史的转折点了,我们说农历九月份的贲卦也就该到了。今天的内容丰富一些,然后可能这个课与现实结合的也比较近一些吧。好,我试一下麦,三点钟我们准时开始。

大家好,今天是2021年的10月16号,农历辛丑年九月十一日。昨夜睡得不是很好,今天的气力不是很够。还好,刚才来到公司的时候,楼下在施工,很吵,我还想转移到酒店去得了。结果现在他们大体上停了,如果他们一会儿再施工的话,我们再想办法。今天我们讲《资本论》第十四讲,这第十四讲其实是一个高潮,它很重要,我们讲资本周转的秘密与经济危机,其实这算是《资本论》的二卷和三卷的核心的内容。

今早出门,夫人问我,说:“今天讲什么内容啊?”我说:“今天是第十四讲——资本的周转的秘密”。夫人说:“资本周转的秘密与当下老百姓的生活有关系吗?”我想了想,我说:“好吧,那我们做一个基本的测算,最近上海的楼租金涨得很厉害,大约涨了20\%吧,看资料,上海的租金一室一厅已经到了五千块钱以上了”。所以我问夫人:“你挣一万块钱,大学生,用五千块钱租房子,你觉得租合适还是买合适呢?”

夫人不假思考地说:“买合适”。我说:“好,如果一个月用一半的工资去买房子,那么就是一年你要用6万、十年60万、二十年120万,请问,120万在上海此时此刻能买到一室一厅吗?”夫人问:“现在上海一室一厅大概是多少钱?”我说:“应该三倍,360万或以上,也就是说买房子是它这个租售比之间的差距差了三倍,也就是现在租比买合适。”

夫人说:“这个租金和这个房价出现了这么大的问题,总理不知道吗?”我说:“总理可能知道的,那你要是总理,你该怎么做呢?是让房价降下来呢?还是让工资涨上去呢?”夫人认为“涨工资太难了,涨三倍,那通货膨胀会到哪儿去呢?”我说:“对,工资是不能那样涨的,但房价能那样降吗?那么这个中间的平衡点如何实现呢?”我说:“这就是资本周转的秘密,这涉及到资本流动和货币定价的问题。”

其实马克思很牛,马克思在资本流转,就是第二卷和第三卷里边讨论了资本流转和货币定价的问题,这里边其实有一些重要的原则是非常重要的:捍卫货币的价值,其实在某种程度上就是捍卫劳动的价值;捍卫资产的价格,其实在某种意义上就是捍卫资本的利益。不学《资本论》,其实你很难懂得这里边的道理,学了《资本论》,可能在投资方面你也就一目了然了。所以今天这堂课其实是意义挺重大的,因为它在我们这个整个的过程里边是一个非常重要的一个转折点。

马克思在《资本论》的第一卷讨论剩余价值——绝对剩余价值和相对剩余价值的时候,已经开始涉及到了资本主义经济危机的问题。在资本流转里边讲的资本的经济危机就看得比较多了,它是一个商品过剩的危机或者是生产过剩的危机。那么今天这堂课为什么有意义呢?就是因为我们正处在整个这一轮的经济危机的前夜,或者是已经一只脚踏入了二十一世纪第二次,第一次2008年,第二次比较大的经济危机之中了。这虽然时间过去十二年,十三年吧,或者十四年吧,但经济危机看来依然不可避免了。

在讲经济危机这堂课之前,就是“资本流转的秘密与经济危机”这堂课之前,我又仔细整理一下《资本论》里边马克思关于经济危机的论述。我想了想,用简单的方法做一点概述,这样有利于大家看问题直白和简单。因为上一堂课讲资本形变的时候,好多朋友说听不懂。我知道不是你们的问题,是我讲课的问题。以后你们也得学我,不要说“你听懂了吗?”要说“我讲清楚了没有?”这不仅仅是个礼貌,这不仅仅是礼貌,因为在多数的时候,你有责任让每一个人听懂,而不是“你听懂了吗?”其实是我没讲清楚。这件事情非常重要,作为政策制定者必须这样。

那么好,我想我们过去讲过四矩阵,今天我们讲一下资本的,上一次讲资本形变的时候,我后来又总结了一下子,干脆我把资本形变的四个部分也变成资本形变的四矩阵。第一个部分是金融资本,这是马克思的分类;第二个部分是商业资本,还是马克思的分类;第三个是生产资本,还是马克思的分类;后来我加了一个现金资本,一共是……。现金资本包括了黄金、金银等这种自然货币,包括了外生货币,包括了内生货币,也包括了现在当下比较时髦的数字货币,形成四个矩阵。我们看资本形变,看当下的经济危机,成因和后果就比较清楚吧。

好吧,回到当下现实。很多人认为大宗涨价是因为疫情的关系,这很扯,跟疫情有什么关系?!很多人认为大宗涨价是美元印发或者是各国超发货币的结果,这有道理,但这个道理说得不明白。那么到底是什么原因导致现在所有的大宗商品、特别是能源类的商品涨价,而且已然形成能源危机,而且开始出现断链?那么我们要说,这并非偶然事件,虽然跟疫情可能有关系,可能跟美元有关系,但它是一个安排。

我其实并不是很认同小宋《货币战争》里边关于共济会的一些讨论,我也不是很认同老何关于梅森、关于华丽家族的一些想法。我去过美联储,也去过美国财政部,我大体上对华尔街也是有些了解的。我在香港蹲了26年,我不认为一定是那样的东西,但我确定它背后有一只手。这只手用马克思的说法就是金融资本加商业资本,注意:四矩阵里边的两个部分,金融资本与商业资本。

金融资本严格地操纵着商业资本。请大家……如果你有时间,将来旅游,你就去美国,你就会去到芝加哥,你就会看到芝加哥商品市场那个地方汇集了几乎70\%以上的品种,或者是90\%以上的交易金额。交易,比如说买铁矿石、买其他的东西、买能源,很大一部分的交易并不是直接两方交易,而是通过芝加哥的交易市场进行交易,特别是期货交易,现货加期货的交易,是以美元在芝加哥资本市场上进行交易。那个地方的商业资本的控制人是金融资本,当然商业资本和金融资本很难划得很清楚。

在马克思的《资本论》里边讨论资本的形变——马克思不是用资本形变,是我用的,马克思用的是资本流转。资本流转的过程中,那么当商品出现价格大幅度上涨的时候,那么商品资本的需求量会成倍增长,比如说天然气涨了5倍,比如说煤炭涨了2倍,比如铁矿砂涨了1倍,那么进行商品交易的资金整个的量就会变得非常的庞大,也就是你要在芝加哥那个地方去买这些东西的时候,你需要更多的美元了。记着这句话:商业资本膨胀必须使计价货币——美元需要大幅度的增加。

这意味着什么呢?这意味着在整个这个资本的池子里边,金融资本、现金资本不变,那么商业资本大幅度增加,必然挤压在矩阵里边的另外一只脚:生产资本必须被挤压。好多朋友问我说,军洛前些日子说的那个东西你认同吗?我说军洛他十多年前来深圳看我,我们是好朋友,绝顶聪明,他是能看到的。就是当商业资本急剧增加的时候,必然挤压生产资本。生产资本在谁的手上?

在德、日、中国等这些制造业大国的手上,那么就会出现这些制造业大国的生产资本被抽取、流出,导致他们的本币出现下滑的这个情况。这不是今天的事情,这是在150年前马克思的时代就是这个样子的。记住,我们讨论的资本流转的这样的一个四矩阵,因为我想这样资本流转的一个四矩阵,可以把马克思的整个的计算就变得非常直观和简单化。就是假设我们金融资本不变、现金资本不变,那么商业资本大幅度增加,那么就是生产资本必须减少,那么就会出现供应的问题,就会出现了一系列的危机和问题。

在这里我想加进来今天我们要讲课的一个内容。在讨论资本的需求的时候,我们要讨论一下子资本需求函数,就是剑桥方程式。发明剑桥函数的这个人叫A.C.庇古,不是我们下半身那个屁股,因为他翻译成“庇古”,我们就“庇古”吧。庇古的剑桥方程式是“M=kPy”。y是实际收入,k是现金除以名义收入,P是价格水平。有时间你们去细细的探讨,其中M和P的关系是这个函数的要点。

它意味着价格上涨,货币需求就会同步增长,一个正比例关系,当然k和y会有扰动。那么这个剑桥方程式告诉我们,就是P决定货币的需求量,也就是说操纵大宗的P意味着货币M的需求量在短时间内大幅度增加。其中这个大幅度增加的是商业资本,它必将对生产资本构成挤压。好多朋友说,什么叫割韭菜?这个拔羊毛?你懂了资本四矩阵,你就大体上明白整个的流程。

在整个讨论M,就是讨论货币需求函数,讨论货币的问题上,我们通常是用两个方程式,一个是剑桥方程式,一个是费雪方程式。费雪方程式大家都很熟悉了,我讲过好几次,就是“MV=PQ”,就是货币乘以它的流速等于商品的价格和商品的总量。一会儿我们会用到费雪方程式,现在我们先不用它,因为我们今天讲《资本论》,我们还是按照马克思的思路走。那么通常当一旦形成大宗价格的剧烈上涨,就会形成经济危机的缘起了,它是一个发起的过程。

其实,在《资本论》写成之后,这些年,这个道理没变过,大体上经济危机的逻辑都是有一只手造成的。所以好多国家、新兴国家为什么最后失败?失败的原因,实际上马克思在《资本论》的第一卷的最后一章也讨论了这个问题,那就是关于生产资本外挂(这是我的词)的问题,就是发达资本主义国家会将生产资本或者是一部分生产资本外挂,就是它放到发展中国家去了。你看他有金融资本、商业资本和现金,但是他会把生产资本外挂。

外挂的生产资本可以对被外挂的国家构成经济增长的这个迅速增长,甚至构成经济增长的奇迹。但是,由于好多国家没有金融资本,没有商业资本,甚至对金融资本和商业资本的理解是有问题的,他只有单纯的生产资本,所以每一次进行周期性大循环的时候,那么美国,以前是英国,老牌的帝国主义国家,就会通过周期性的经济危机,输出之后又收回,就是挤压生产资本,那么使得很多国家就是在经济发展中失败,南美是最经典的案例。

这个日本也算是一个案例,日本和韩国也算是案例。这一回我看美国再一次发起这一轮的经济危机,这回收割重点还是日、韩、台、新,可能会带上欧洲,带上德国这些制造业的大国。但是由于欧洲人对金融资本和商业资本的理解还是比较深刻的,所以可能他们的伤不至于入骨。至于对中国这回收割能收割到什么程度?有多大的问题?一会儿我们讲。因为今天是个很神奇的日子,因为大概在23小时之前《求是》杂志发表了习主席的讲话,那就是中国策略,我一会儿念叨几句。

通常在危机之前会制造一个假象,就是商品过剩、生产过剩、产能过剩。这个在中国我们是熟悉的,在供给侧结构性改革这句话里边前提条件就是商品过剩、生产过剩和产能过剩。要去产能,要这个去杠杆、去产能,要脱实就虚,这是危机的前奏,而且这是一种舆论准备。就是所以中国前些年这个生产资本离开生产进入到房地产、进入到资产类里边,我们这个国家是一个成长的过程

一会儿在今天这堂课的最后,我会发表一通我的感慨。因为这两天有一些朋友对我讲了一些话,因为煮酒论英雄嘛,大家到了一个年龄段总是要论一下人生的成败、人生的得失。有些事情超越了权力和金钱的测量的时候,其实多数人是看不到了,就是迷茫了吧,就是可以理解。我们现在因为我们是课程里边先不讲这个,我怕待会儿时间不够,我们一会儿再去讨论这个问题。最近就是大宗疯狂,运输断链,美元回流。

整个的形式,如果你研究过经济史,其实貌似陌生,但是还是那个熟悉的味道。你要是读了马克思的《资本论》,你觉得他们又来玩儿了,又来开始玩这一套了。虽然每一次会有很多很多的不同的说辞,比如说这一次是MMT,比如说这一次是内生货币资本化,比如说……反正一大堆说法。在世行或者是其他地方也会给中国的管理层提供一大堆的政策建议,从供给侧结构性改革到最近一大堆的建议、一大堆建议。但你回到《资本论》,你一看其实是一目了然。

好,我们进入到今天的环节的第二个部分,我们稍微的进入一点儿理论。就是通常我们认为资本有两个主要的源泉,一个是外生货币,一个是内生货币。外生货币你就懂了,外生货币基本上我们把它理解为央行投放的货币,内生货币通常是指商人银行和投资银行它们创造的货币。外生货币可计算,熟悉,内生货币比较复杂。一般人以为外生货币意味着央行货币政策它有量化宽松、有质化宽松,通常是降息或者是降准向市场投放,它就意味着外生货币增加,这个流动性放松。

通常在经济不好的时候,外生货币采取比较宽松的政策,在经济好的时候开始紧缩。那么我们注意到了没有,这一回美国的外生货币恰恰是在经济出问题的时候开始紧。这个收羊毛的动作跟《资本论》上的看法非常之吻合。那么内生货币就是商人银行、投资银行创造的这个货币,在这个时候在经济危机的时候会增加还是收缩?我要告诉你是急剧的收缩。因为看到恒大你就懂了,越是在经济出现下行的时候,信用越出问题,那么流转的速度就会降下来。

什么意思呢?回到费雪方程式,回到费雪方程式MV=PQ。当经济出现问题的时候,M的总量还没有变,但是流转速度就是银行不愿意贷款给他们了,那么周转房子卖不掉,V开始迅速下降,Q还是原来的量,P哗啦哗啦下来了。看出来了没有?这个M和P不是同步运动的,和剑桥方程式需求函数不吻合,这两个方程式非常重要。第一个方程式是需求函数,那个M不是Money,是函数,它里边有噪音P导致M需求的总量的下行,而这个M(费雪方程式M)就是钱是M0。

好,回到现实。2010年我国在2008年之后增加基础货币大概1万亿人民币,我们增加了M0是1万亿;那么M1增加了多少呢?M1是10万亿;那么M2是多少呢?是25万亿。在一个经济增长非常快、运行非常稳健的时候,这个货币的创造的过程是这样的。就是外生货币的量和内生货币的量的比例关系是大体上是如此的,M0:1万亿、M1:10万亿、M2:25万亿,M2的增速是非常惊人的。请注意经济危机是从M2这儿开始的,这一点非常重要,这不是我说的。我们为什么认为老马很厉害呢?

马克思这个人就是有点聪明,他真的是那么早就把所有的事情看得明明白白的。好,我们既然进入到理论,我们就稍微再细一点。在我们讲的资本四矩阵里边,我们把它用费雪的方程式来做一个表述,MV就是不变,但MV我们把后边分成5个部分:P1Q1我们管它叫商品;P2Q2我们管它叫股市;P3Q3我管它叫楼市;P4Q4我们管它叫债市;P5Q5我们管它叫现金,这里边包括了金银、现金和比特币等数字货币。

好多朋友说:“卢先生啊,你讲完四矩阵讲影响四矩阵的因素,我们也没听懂。”今天我再讲细一点儿,因为今天多加了一个角,四矩阵就变成了五角星。变成了五角星,我再讲一下扰动的因素,你先记下来吧,我估计还是弄不懂,弄不懂不着急,记下来我们慢慢懂。我们这里边有高手,最后终于会把这个立体图做出来的。当高增长、高通胀和高息的时代,那么……我琢磨一下子,我别说错。

好,我说的是一个理论状况,因为它是个动态模型。三高:高增长、高通胀、高利息的时候,商品必然疯狂,P1Q1必然疯狂;当高增长、低通胀、低利息的时候,股市必然疯狂(现在的美国);当高增长、高通胀、低利息的时候,楼市必然疯狂;当低增长、低通胀、低利息的时候,债市疯狂;当低增长、高通胀、低利息的时候,金、银、币陷入疯狂。这个高高高、高低低、高高低、低低高、低高低里边的这个高和低到底应该是怎么来定?

将来我们有时间再慢慢地细聊吧,因为这是一个非常结构复杂的模型。我在这里说,形势正在起变化,高增长疫情之后的高增长正在结束,而且可能在最后一个季度第四季嘎然而止。形势起变化的第一句话,高增长正在结束;第二句话,高通胀已经到来;第三句话,高利息正在路上。这说清楚了吧?

没说清楚,就回到现实中来看实际例子。美国今年的增长高不高?高,原来预期会到8,现在虽然调整预期也是6,在美国历史上数十年,高增长,对吗?但是到了第四季他要调下来,到了明年要调得更低,高增长变成了低了吧?注意这三个字。第二、低通胀变成高通胀了吧?第三、低利息,还得低利息,对吧?不敢加息嘛,他是不是变成了低高低了呢?那么低高低应对刚才我们那个五角星是不是最后一块啊?好,回到中国。

中国的高增长明年会降一些,但不会变成低增长,还会超过4\%嘛,还算比较高。中国的高通胀,中国的这个低通胀明年肯定是变成高通胀,中国的低利息明年还是低利息,对吧?大体上是高高低,美国是低高低。这个对我们的投资意义非常重大,因为我们一直在讲模型,就是资本往哪里去。加一句话吧,我前两天写了篇关于币圈的命运,就是讨论了币圈的问题,因为我个人认为币圈是美国人设的一个套。

有一种走资叫币圈 https://www.notion.so/7fffaab91bf248a6907f92a4e813ee64

我从不认为日本会有一个中本聪的人,我认为中本聪是一个设计出来的,设计了一个比特币,而且这个比特币是耗能源的挖矿,而且这个比特币它有个致命的问题,它就是个算法,因为你要用量子计算机你能打开你的数字钱包。但是比特币现在目前它变成了一个全球性的潮流,特别是吸引华人,70\%以上的交易是由华人进行的,它由小小的1万亿美元的规模,大概迅速上升到10万亿美元的规模。在整个的资本的矩阵里边最讨厌的就是这个货币。

{\kaishu 1/ 比特币有三个地方用到加密算法:

2/ 私钥到公钥用的是 ecdsa, 椭圆曲线数字签名算法。

3/ 公钥到地址,用的是 ripemd160, sha256 两种算法, 依次哈希一遍。

4/ 工作量证明 proof of work,用的是 sha256.

5/ 通常人们担心量子计算会首先反向解密的,是 ecdsa 算法,这理论上可以通过网络上暴露的公钥,反推私钥。

6/ 但是: 第一,如果每个地址只用一次,公钥暴露了也不怕反向解密,所以不怕攻击。 大部分钱包软件都是一个地址只用一次。

7/ 第二,从某个量子计算技术显示有潜力破解 ecsda 算法,到真正能够破解 (花亿万资金在特定实验条件下几个月破解),到真正可以很少钱几分钟破解,每个阶段都是个漫长的过程,至少几年甚至更长。

8/ 而在这期间,这种反向解密的算法技术,可能对于传统计算设施在金融业和政府机构的应用有更大威胁。所以更需要担心的是他们。

9/ 一旦行业有共识量子计算可能对现有加密算法体系产生威胁,可以有足够多的时间,做一次硬分叉,切换到新的可以对抗量子计算的数字签名算法上。

10/ 加密算法的本质是加密和解密计算量的巨大不对称性。在这里,盾的成本比矛的成本要低若干个数量级,使得矛的暴力攻击几乎不可能。

11/ 和传统博弈里面进攻捣乱的成本远低于防守的成本不一样,加密行业的基础在于防守成本远低于进攻成本。

12/ 同理,即使未来最终有新的矛出来,开发切换新的可以抵御量子计算的盾,成本也将会比再开发新的矛要低很多。

13/ 量子计算对 ecdsa 算法的最大潜在威胁是通过舒尔算法,把反向解密的计算量 (对于一个 256比特的私钥而言) 从大约十的三十八次方,降到十的八次方左右。而它对于 sha256 和ripemd-160 反向计算量的减少则比较有限,不足以改变攻守力量对比。

14/ 唯一理论上的攻击时间窗口,是当某个交易被广播到链上到被矿工确认这十分钟内,如果有人有能力把广播出来的公钥迅速反向解密出私钥,再支付更高的交易费用,双花到自己的地址上。但技术能力演变到那一步之前,社区应当早就硬分叉,迁移到新的加密数字签名体系上了。

15/ 理论上说量子计算要破解比特币,需要: 第一,至少 1500 qubit 的算力 (谷歌 2019年的论文里面只有 54 qubit 算力 )。第二,为了抵御外界噪音和辐射干扰导致的量子退相干 (quantum decoherence) 现象,实际算力需求需要再乘一千,就是一百五十万 qubit 的算力。目前看还是非常遥远的威胁。

16/ 现在媒体上很多关于量子计算的消息,第一没有区分通用量子计算 (可以攻击加密算法) 和别的特殊应用的量子计算。第二没有报道实践的错误率,要纠错必须把算力再提高一千倍。第三这必须要在极为封闭和接近绝对零度的环境内才可行。目前看 2030年之前都不太可能对现行 256 比特 ecdsa 算法有效攻击。

——硅谷王川}

币圈规模十万亿美元此点不实。全球加密货币总市值为 2.3万亿美元。算法高于看法。(截止2021.10.11)

{\kaishu 七成投资者来自华人此点,并无数据支撑。因为:

1. 加密货币具有匿名性特征。一个人可以对应多个地址,一个地址也可以对应多个人。因此连参与加密货币的人数都是无法确定的,更何况投资者来源。

2. 中国曾经在挖矿算力上占据过全球70\%的比例,但算力比例并不等于投资者比例。好比北京占据全国银行交易结算服务器70\%的比例,并不意味着北京人持有全国70\%的人民币。

3. 2018年中国央行透露,以人民币交易的比特币占比不到1\%。}

它与黄金、现金一样,它是资本沉没,就是数字货币,因为它不能用于生产、流通、消费,它就是资本沉没的一个东西。有些人故意设计一种资本沉没的工具,并且让你在这个过程中能赚大钱,导致你国家资本的一种外逃、走资、沉没。所以我一直在批这个东西,但是我身边因为有太多的好朋友都在币圈里,而且他们赚了钱,甚至有些人赚了大钱,每次说这个东西会很伤感情,因为他们对我是那样的好,我不忍心,但我还是得说,因为此事涉及国本,不能不说。

回到我们今天的理论的部分,我们今天讲了一些内容,其实外生资本、内生资本、内生资本货币化,讲了剑桥方程式、讲了费雪方程式,好多朋友说这不是马克思说的。是的,这不是马克思说的,但是我们用马克思的资本的流转的结构来定义经济危机。需要具体的计算方法和需要当代的语言来进行阐释和描述,我不得已用外生资本和内生资本和剑桥方程式和费雪方程式,我不得已。其实我挺喜欢庇古这个人的,庇古和凯恩斯是同时代的人,他俩本人也是好朋友。

当然庇古是对凯恩斯持批评态度的,因为我能体会到庇古是一个代表大资本利益的那样立场的一个人,他和凯恩斯的社会主义情怀是不一样的,所以他们俩的想法不一样。特别是在庇古的晚年,他对凯恩斯多有批评或者是批判,有时候还是很尖锐的,但不影响他们两个人的友谊。因为庇古的资历比凯恩斯老,就是凯恩斯初始阶段做研究的时候曾得到庇古的支持。我说的不光是学术上支持,主要是财务上的支持,这很不容易。你在中国人之间,他跟你观点不一样,还给你钱让你做研究,太难了。那观点一样的他都不给你钱做研究的,进行财务上支持,只有这种人,他们这种有信仰的人能做到。

谈到内生货币和外生货币,内生货币资本化的问题,其实马克思和凯恩斯都在他们的著作里间接地谈到了,没有直接,这么直接了当地讨论。因为货币的源泉、货币的成长、货币的流转的过程中,肯定涉及到外生货币和内生货币,特别是对内生货币的研究是凯恩斯《货币通论》核心的所在。凯恩斯为什么认为可以通过国家财政扩张来挽救经济?就是因为他了解了外生货币和内生货币之间的逻辑关系,而且在一个特定历史时期,凯恩斯是正确的,在某种意义上也是凯恩斯的《通论》建立了宏观经济学。

在MMT这个问题上,为什么凯尔顿认为她是新凯恩斯主义?就是他们之所以说是新凯恩斯主义,就是凯恩斯,这样说吧,凯恩斯的主张是通过一定程度的外生货币资本化来刺激经济,而这个外生货币通过与内生货币的放大,我刚才说了1万变10万,10万变25万,放大过程形成对经济整体的推动,这是凯恩斯的本意。那么凯尔顿认为可以更直接了当,直接通过内生货币资本化来完成这样一个历史进程。也可以管它叫新凯恩斯主义或者是新的货币主义,这我看问题不大。

但是要注意受益者。如果这个受益者是广大的创造价值的劳动者,那就是凯恩斯的本意;如果这个受益者是资产持有者、食利者,那就不是凯恩斯主义,那是货币主义。实际上我个人认为内生货币资本化在很大程度上就是今天美国的现象,它会导致资本市场的极度扩张,会导致资本利得的大幅度增加,会导致股市的疯狂上涨和楼市的一定程度的上涨,而不会导致劳动者、或者中产阶级的收益或者权益上涨。

这一点对我国制定财政政策和货币政策,具有极为重要的意义。我闲极无聊、晚上睡不着觉的时候,也会听一听这个观视频的“大佬时光”,听一听我国的经济学家们和各个大行的首席们,他们讨论财政政策和货币政策。当然了,说句重一点的话,就是他们大部分的人的水平还到不了这个层级,因为他们只是在处理公式啊、处理这个数据的过程中,他们还上升不到哲学高度、这个逻辑的层面来对宏观经济进行解释,虽然都是留学回来,但到不了。甚至包括一些著名的经济学家也经常是很扯的。

我国的学者里边有两个极端,一个是主张赤字货币化,或者赤字资本化,就是刘尚希,财政部的研究所的,这是真不懂——既不懂马克思,也不懂凯恩斯。财政赤字可不可以理解成是外生货币?是可以的,那是因为是财政赤字可以变成央行投放货币嘛,因为财政赤字可能会变成国债,变成从央行那儿作为一个支出,但它还是性质不一样。这个,刘同学还是要好好的读书。内生货币则不同,在讨论这些问题的时候,我们必须看到结果。

内生货币资本化是美国贫富分化的根本性原因。我是反对简单的MMT的,这个事情只能美国用,中国不能用。因为当美国人采取这样的方式来对经济进行刺激的时候,实际上他们在走向财政政策和货币政策的极端,甚至我在此时此刻,我认为美国的财政制度正在走向崩溃的边缘,就是当汉密尔顿为美国建立了当下这个财政制度的时候,汉密尔顿可能没有意识到,走到耶伦这一代财长,美国现行的财政制度正在走向崩溃。

刚才我念叨了美国和中国的三个东西:经济增长、通胀和利息的变动,美国就是低、高、低,中国高、高、低。那么欧日的情况呢,其实欧、日包括韩、台、新加坡都在陷入一种同样的困境,由于商业资本挤压、生产资本被急剧压缩,出现了本币的压力——巨大的压力,可能会有大规模资本流出,形成他们的新一轮的经济危机,所以他们可能会陷入低、高、低这样一个状况,或者是低、高、高的状况。

第三世界国家,以印度为代表,都会进入到低、高、高的情况,非常糟糕,其实已经开始进入低高高了,就是不好的状况。但是由于第三世界国家可以薅的羊毛不多,例如印度、例如四小虎,薅的重点应该是欧、日、韩、台。那么我们花一点点时间,讲一下,中国会否被薅羊毛,会到一个什么样的状况?会不会经历一场惨烈的经济危机呢?我们现在的应对策略是否得当呢?我们花一点时间,做一个比较详细的分析,对大家的明年以后的时间的投资,会有一些帮助。

我一直在说,中国目前是以生产资本为主体的这样一个国家,我们有一定量的金融资本,我们的金融资本可能对本国有一定影响力,但与生产资本相比还是比较小的。我们也有一定量的商业资本,因为我国已经是世界上的消费大国,所以我们也对商品的贸易有一定的影响,但我们还到不了芝加哥那个期货交易中心的水平。就是我们的金融资本和商业资本还是比较弱的,生产资本的量比较大,但我们的生产资本与德、日略有不同,这一直是美国人攻击我们的焦点,就是他们认为我们的生产资本很大一部分是在国有企业的手上。

为什么美国人一直攻击我们的生产资本,因为我们的生产资本很大一部分掌握在国有企业手上,那么生产资本掌握在国有企业手上有两个好处,这两个好处可能是救命的好处。第一个,当生产资本被挤压的时候,因为它是国有的,所以国家可以采取某种方式通过注资来解决它的稀缺性,就是国有企业遇到困难了,国家不能见死不救,可以帮帮忙嘛;第二条,生产资本遇到的最大的问题是商品过剩,那么因为你是国有企业,所以国家会对你商品的出处,给予一点帮助,比如说我们在特定的时期,我们做过一次什么——电器下乡。

所以我们对生产资本,有一个上游,就是资本注入的方法,来缓解生产资本的压力,有一个下游,就是对过剩产品有一个舒缓的方法。社会主义制度、国有企业,难道都是缺点吗?当然有效率,有诸多人事、有诸多问题,但是你看到了没有?社会主义就是社会主义,社会主义应对资本主义经济危机的时候,它就是有它独特的力量,有它独特的方法。所以,中国在过往的七十多年时间里边,特别是在改革开放中,从90年代一直到2008年,我们都曾使用我们的方法,来解决生产资本的问题。

马克思在《资本论》里边其实讨论解决这个问题的时候,因为马克思真正的第三卷是要写国家与资本的关系,这部书马克思没写,(列宁的《国家与革命》里边写的国家与资本的关系也没写清楚,斯大林的《政治经济学教科书》里边,因为只有一半——社会主义内部,没有社会主义与资本主义这种大融合过程中的这种关系。)所以中国少一本《资本论》第四卷,这事不能完全怪我啊,也怪大家,就是都跟我一样,不是很努力,所以好多事儿也没办出来。我们一直在讨论生产资本的稳定的问题,就是一个国家的稳定跟生产资本的稳定是有关联的,生产资本的稳定决定了一个国家的就业和国民经济增长,这一件事情很重要。那用什么样的方式来完成生产资本的稳定呢?

我们国家在摸索中形成了我们国家独有的套路。我们国有经济还占据着非常大的比例,特别是在涉及国计民生的重要领域,比如说能源、比如说钢铁、比如说化工,我们在国有经济的比例非常大。在国有经济遇到困难的时候,这个民营经济当商业资本开始严重挤压,以数倍的压力——百分之五百的涨价,数倍的压力压缩生产资本的时候,如果是一个纯市场经济的国家或者是纯民营经济的国家,这个时候生产资本必须投降,甚至被吃掉。但是遇到了中国这样的国家,第一、我们的国家非常强大,就是你这个挤压我们能扛;

第二、我们有广袤的自己的市场。这个市场是容量很大的,这是以往英、美资本主义没有遇到的情况,它遇到的最大的工业化国家就是日本了,德国、日本这样的工业化国家,一亿人口规模的工业化国家,他们从来没遇到一个十四亿人口的工业化国家,十四亿人口的工业化国家有宽广的腹地,这种耐打击能力是难以想象的。所以其实从特朗普进行贸易战,以至于今年这一轮大宗的这种疯狂的上涨、这种商业资本对生产资本的挤压的情况来看,我个人认为:美国、欧美——主要是美国的金融资本……

我个人认为美国的金融资本出现了两个问题:第一个问题,他们对中国经济的理解上可能出了问题,就是中国经济的真实体量——我一直在说,中国经济的真实体量至少是美国的一倍,而不是美国的70\%。就是你两个人打架,七十公斤和一百公斤级,那么一百公斤级具有优势,如果你以为他是七十公斤级,结果他是两百公斤级,这架你就打输了嘛,对吧?你这个账都没算清楚,就上了擂台,就上去就给人一拳,结果人一还手,你不就趴那儿了吗?我觉得第一个是这个账没算清楚。第二个,美国人可能太相信他们在中国培养的“天降组”了,真的,一会儿我发感慨会说这一件事情,就是他们相信他们培养的——我都不说名字了——那些人一定会帮他们把事办成。

问题是中国不是还有我们这批人吗?怎么就一定是那几个留学的孩子一定能把事情办成呢?上周出去吃饭的时候,还有一些人在讨论那些个孩子。那些个孩子,2008的时候,如果你们看过那个时候的凤凰卫视,我们同台登台产生那场争论,多么的热闹。而且在争论的间隙,就是有特定的人过来跟我讲一番话,那是多么的羞辱啊!就是他们才是代表着怎样怎样怎样,你这个就不是,虽然你是财政部的,但是你这个不是,你这个不行的,但是那个——就是洋人嘛,不管他是来自高盛还是来自摩根斯坦利,就是……

值得高兴的是,那个时代结束了,我不知道是不是永久的结束,至少在2021年的10月份,我感到很兴奋,终于不是“黑手高悬霸主鞭”了,再也不是那样一个时代了。再想让中国像2008年那样,甚至再像上一个庚子那样,上一个辛丑,一百二十年前《辛丑条约》那样,想都别想了。当历史掀开崭新的一页的时候,其实可能我们自己并未知觉,但是我相信彼岸那些人已经开始感到无比的惶恐了。

我们花点时间说几句二十三个小时,现在差不多二十四小时了,二十三小时之前“求是网”登的习主席这篇文章《扎实推动共同富裕》,我其实感到非常惊讶——就是果然有高手。在危机即将爆发的前夜,这一篇文章《扎实推动共同富裕》简直就是一个应对方案、应对策略。也就是说我们面对即将到来的经济危机,我们实际上已经有了一个完整的想法,而且这个完整的想法可能构成明年二十大的主题,也就是说它影响的不是今年的年底、明年,可能会影响今后三十年到五十年。

扎实推动共同富裕 http://www.qstheory.cn/dukan/qs/2021-10/15/c\_1127959365.htm

这里边五个主题都很重要,我念叨一下,时间也不多。第一个是勤劳致富:勤劳致富这话你听着俗,我给你翻译成马克思的话——就是劳动创造价值或者是创造价值者最光荣,劳动致富。他所要说的就是我们反对食利,只不过是他用勤劳致富来说反对食利;这里边有政策意涵,也有制度意涵,以后希望你勤劳致富,以后不希望你在家食利,依靠权力或者资本的食利是我们不主张的,我们不主张你躺平,我们主张你劳动创造价值。这第一条真的很牛,这才叫“资本论”。

第二条是基本制度:这个基本制度实际上是一种历史的平衡,我们尊重市场,但我们也要强化管理、强化计划管理,这是市场与计划的平衡,我们不能——这里边它的内在的含义就是我们不能将中国的经济治理主权交给境外资本,我们尊重市场,我们尊重你,你要愿意谈,我们好好谈,但你让我们投降,我们把经济主权交出去,做梦去吧!所以我觉得第二句话也特别棒,你知道吧!我们要坚持基本的制度啊!第一句话是勤劳致富,就是麻烦你劳动,你不要食利。第二句话是我们要坚持我们的基本制度,就是我们要捍卫主权,你想抢主权,去你的!

第三句话叫量力而为:就是我们中国不搞福利主义,我们不搞大锅饭和福利主义。不要认为中国会退回去,不会!我们不搞福利主义,不搞那一套。第四句话是调两头,这句话有意思,其实第四句话讲的是税改——就是我们两头的问题必须马上解决:一个就是有些人赚太多了,而且是不合法、不合理,赚太多了,要调回来,调两头;有些人太穷了、太穷了,处在贫困的底层,生活极为困难,这两头我们都不能接受,赚钱多的人,对不起,“税改”,我们要拿回来了,最穷的人,我们要转移支付。

最后的一个部分,第五个部分是均衡。这第五个部分表达了中国这七十多年或者是中国这三千多年治理上面的一个最高的理想,现在还没做好,就是理想的最好的境界就是均衡。均衡发展包括了区域的均衡、包括产业的均衡、包括了民族的均衡,就是我们都要达到均衡的水平。我们不能让有些地方太落后,我们也不能让有些产业的收入太低,也不能让有些产业收入太高,也不能让东部有些地方这个地方收入太高;它第二层含义就是还是要中产阶级为主体嘛,就是穷人也不能太多、富人也不能太多,中产阶级多一点;第三个就是公共政策要均等。

实际上这句话就是我最喜欢的话:就是无差别社会保障。管你是总理还是乡下的农民,在医疗、教育和养老这些方面是一样的待遇。你认为要好一点,麻烦你去劳动、去努力,你不要告诉我你是总理,所以你必须比他好多少,这个不行,必须公共服务均等。这个很了不起啊。我觉得这一篇文章我看了以后觉得非常喜欢。第三个也很重要,就是我们强调精神生活的建立、精神文明健康,这个我是非常支持的。我们这些年这也太不强调精神文明健康了,实在是不像样子。我对这个课本意见特别的大。

有朋友推荐我的两篇文章,一个是君子不器,一个是行者无状进中学教材,被某部门拒绝,他们却选用了方方写的进教材。我不是说不可以,我也不说我的文章写得比别人好或者比别人更有文采,我是说精神上的追求真的比“伤痕”、控诉重要得多。你做君子和你受了委屈,这是两回事嘛,不像个样子,没文化。但我们也不能批评太多,批评太多就是又关半年到一年,就没法跟大家聊天见面了。

最后一个部分是关于新农村建设,这个我也喜欢,非常喜欢这个新农村建设。应该让知识老年、干部老年、资本老年上山下乡。他们干嘛要在城市里呢?将他们的知识、财富、社会资源全部带到乡下去,为老百姓作出服务,这该多好啊。所以我对习主席《扎实推动共同富裕》的这样一个在求是网发的一个通稿,四千多字,写得极好。我昨天失眠跟这个通稿有关。

最后发几句牢骚,就算是今天这堂课的结束、结尾了。因为我离开财政部,在中经信,南方叫中经开待过一段时间,中经信和中经开本应像中信、光大一样辉煌的金融机构,但是由于历史的原因,中经信和中农信两个机构垮掉了、倒闭了。我现在这个机构的股东还是中经信。这里边的好多朋友一晃都是二三十年的朋友,但是有些朋友不能理解我,他说了一段话,当然他们对我好,说卢先生啊:“你若留在部委或地方造福一方多好;

你若经商,好好地搞一个、打造一些机构出来多好;你为什么要去写文章呢?你为什么要去写文章呢?写文章有意义吗?有经济上的意义吗?而且给自己添那么多政治上的麻烦。”就是话说得没那么糙吧,就是“缺心眼”的那个意思。其实有好多人觉得我“疯”,觉得有点“傻”,夫人也经常用“傻”这个字来概述我。我不这样想,道理其实简单,说出来可能大家会觉得……

我说过,一个人大体上知道自己的被动神经系统,不知道自己的主动神经系统。就是你不了解你的植物神经系统,就是经常你会跟植物神经去捣乱,特别是不了解你的第十套神经系统——迷走神经,但是一个人的身体的健康是由植物神经来决定的。因为它每天在你躺下之后对你的身体进行修复;它在你思维激动的时候、要犯错误的时候,它会阻止你。比如说,你吓尿了,那就是植物神经干的;比如说,你的努力获得了认同,它会给你血红素,你会觉得很幸福,它会给你多巴胺,你会觉得很兴奋。就是整个这植物神经体系或者是第十套迷走神经体系,它对一个人太重要了。

一个伟大的民族、一个伟大的国家一定要有一套了不起的主动神经系统或者是植物神经系统或者是迷走神经;一个伟大的民族、一个伟大的国家,如果植物神经紊乱或者是主动神经系统失效,他就会变成前苏联,他就会变成日本的平成战败。什么是一个国家的植物神经体系呢?就是他的思考者,可能他们是伟大的思想家;也可能他们是平民,他们是独立的思考者;

他们不是那些在互联网上标榜的什么独立学者、独立经济学家。那个“独立”的意思是他没领钱,我们不需要他们;我们需要的是对这个国家的历史、现实进行具有哲学高度的深刻思考的思考者;他们可能是伟大的思想家,也可能是普通的平民老百姓。这些思考者有可能是红二代,有可能是农民,他们是由不同的人构成的,而他们不是一个人,而是一个伟大的群体,而且他们中形成伟大的认同,甚至有非常强烈的共识,构成国家、民族的植物神经体系。

请允许我向在过往这二十年坚持不懈的、努力的这些思考者、思想家们表示敬意;请允许我说《乌有之乡》、《红歌会》、《红色沙龙》他们存在都有他们的价值。有可能很多人认为他们是民粹主义的、社会主义的、共产主义的,这不重要;他们也可能做了很多事情并不正确;说了很多话,并没有那么深刻的哲理;但是他们起的作用是主动神经系统的作用。

请允许我向体制内一些学者,包括铁军老师、汪晖老师等等一大批中国研究社会学的学者以及他们的努力表示感谢;以及他们身边聚集的那么多的学生,也多数都是我的好朋友、甚至是战友,他们在进行艰苦的乡村建设。他们与另外一批的红色的朋友们构成了与当下的资本主义市场经济相对应的一些体系、一套体系。虽然我国的主动神经系统并未达到应有的哲学高度,甚至不成理论体系;

虽然他们是如此的散乱、甚至散漫,甚至充满了内部的纷争,有的时候打架打得一塌糊涂;但是并不影响他们整体上作为一种思考者、思想者,甚至是行动者,对过往这二十多年国家一路走来,他们所做出的杰出的贡献。为什么是二十年呢?因为二十年之前,中国老一辈的无产阶级革命家还在,等中国的社会主义者们从肉体上消失之后,精神的传承就是这些人了,他们构成了中国的主动神经系统——植物神体系、第十套神经——迷走神经。

我今天为什么要说这件事?如果中国可以渡过这场经济危机,我们要感谢的可能是那些并不富有的主动神经系统的这一批朋友啊。因为这些朋友这二十多年来真的是“孑孓踯躅,砥砺前行”啊,充满了辛酸和血泪。没有人正面夸奖他们一句,他们受的委屈是那么的多,作出的工作贡献其实某种意义上也是挺大的。行,我就不在今天诉说他们的名字,其中有一些人是很有名,但是他们可能在社会上负面的名声更多一些;可能会被批评,会被大家觉得有问题更多一些吧。

讨论《资本论》的时候,我不得不讨论社会上的一些现象和人。最后的感慨,请允许我这样的感慨一下子吧。我只是其中的一个沙粒,我们进行我们的努力,我们希望整体社会在一个共同的努力下走向他应有的历史性的辉煌。当然我们会记取各个方方面面的不同的人所做出的不同的贡献,我们珍惜共和国的主动神经系统——植物神经体系。好吧,就说这么多,明天下午三点钟见。

\subsection{默多克与民粹主义、聊几句美联储的变化、房产税的误区和六中全会的改革}

大家好,今天是2021年的10月23号,辛丑年的九⦁一八,真的是很有趣的日子。因为前两天是9月15号,香港的街头都是烧纸的人。今天是聊天。其实这个聊天我做了时间准备的时间长度是比较长的。我可能没觉得我准备好了,但是我想还是先聊吧。今天聊默老爷,默老爷指的是默多克,聊一下子默多克与民粹主义,因为很少人很深刻地去思考英语媒体世界的……

顺便今天有时间,我们也聊一下子美联储的态度的一个变化,因为也是应大家的要求,我们聊一下美联储的一些想法,看一下美元的变动。毕竟是一个大的历史转折关头。如果剩下还有时间的话,我们想聊一下子下个月一一•八——11月8号,这个六中全会,因为这个变法可能是压倒一切的,应该不仅是中国的大事,也可能是全世界的一件大事,此事可能会关乎中国的百年兴衰。所以也是大事,好,一会儿见。

大家好,今天是2021年的10月23号,辛丑年的九•一八。今天是聊天儿,我们聊默老爷与民粹主义,然后有时间谈一下子美联储的变化、美元的变动,如果时间还够呢,聊几句六中,六中全会,11月8号吧。今天这堂课,这不是课,这个聊天,其实我是准备了一段时间的,谈民粹主义,其实是一件比较困难的事情,因为民粹主义现在成为了英语世界的一种潮流。

民粹主义是早于社会主义出现的,民粹主义按照官方的说法,应该是18世纪70年代在俄国出现的,距今大概是两百五十年这样的一个长度,但我个人认为可能更早一些,因为我觉得民粹主义是与宗教改革同时出现;实际上是民本主义的思潮导致了民粹主义的出现,或者说民粹主义应该是社会主义出现的一个前期吧,前期的一个重要的基础。那么,为什么在英语世界会重新出现民粹主义呢?

这个事情,我其实已经想了很久了。何新先生和好多朋友认为犹太人或者是梅森、或者是共济会在控制当下的世界。但我仔细研究了特朗普现象、研究了英国的脱欧和约翰逊,研究了澳大利亚的莫里森,大体上,我觉得英语整个的这个世界,民粹主义的兴起跟一个人有关,这个人就是默多克,当然啦,谈默多克可能应该由邓文迪来谈更好一些,但没有办法邀请到邓文迪,所以还是我来谈吧。

我们先说民粹主义,民粹主义是反精英、反建制、反代议、无政府,这样一种思潮或者是运动。民粹主义在某种意义上,有一些像我国1966年到1976年的时候发生的事情,就是砸烂公检法、造反有理、红司令、这个专政,就是一种革委会的这种样式吧。西方,特别是在英语世界,为什么会爆发民粹主义运动呢?

因为,如果不导入,目前西方的这种极端贫富分化之后,所出现的民间的这种反抗力量,不将它导入民粹主义,那么西方世界将爆发轰轰烈烈的社会主义运动。其实大家可能不太理解,这说来快要十年前了,爆发于美国的占领华尔街,应该是2012年吧,占领华尔街运动,占领华尔街运动是经典的社会主义运动,此事震撼了整个西方世界。当然,2014年香港爆发的占领中环运动也是经典的社会主义运动,他们特征是反资本。

也就是说,无论是占领华尔街运动还是香港的占中运动,它是一种阶级斗争,它具有社会主义性质,它震撼了整个的西方世界。所以西方世界由特定的人,将这场思考、运动,导入民粹主义,它变成了反精英、反建制、反代议的无政府主义运动。请注意,大家注意到了吗?香港的黑衣人,2019年他们冲进了立法会,他们不再占领中环了,而美国最终是冲进了、占领了国会。民粹主义运动里边,背景其实值得我们深思的。

我说了民粹主义应该是反精英的,但它其实在今天英语世界里边,民粹主义其实不反精英,我个人认为,由特朗普代表美国的蓝领阶级是一种讽刺。反建制倒是真的,但他不反体制。请注意建制和体制的区别,体制是什么呢?是金融垄断资本主义主导一切权力,这个体制他不反,就是他不反对资本主义;他反建制,就是现在这个政府,他要反,这就非常有趣,他搞无政府主义,其实民粹主义是双刃剑。不打算解决问题,而导入民粹主义其实是历史性灾难。

民粹主义反对什么呢?实际上他们是破解当下的西方的选举文化,他们是一种逆改革。那么破解的结果是什么呢?破解的结果就变成了一种街头运动,或者是建基于当下信息时代的、互联网时代的一种网络运动,所谓的网红,它是一种逆选举的文化,是一种群众的、集体性的审美的取向,这种审美取向因为是民粹的,所以是非理性的、所以是逆改革的。民粹主义的发展呢?会走向一种极端。

当民粹主义与极端宗教相融合的时候,它就发展成原教旨主义,当民粹主义与民族主义融合的时候,它就发展成为大国沙文主义。其实我讲到今天,讲到此时此刻,你们知道英语的舆论导向,民粹主义为什么会如此猛烈的反华?因为无论是它加上宗教的逻辑,还是加上民族主义的逻辑,它必然形成英语世界的整体反华,并且形成整个西方世界的反华。如果我们不能理解这个民粹主义,我们很难理解中美关系或者是中国和西方的关系。今天讲这堂课,其实我还没有整理完,但我先跟你们聊……

{\kaishu 种族与国家的组合形成民族国家;一种宗教与一个机构的组合形成了基督教;一个宗教与一种文化的组合形成了伊斯兰教。当两种伟大力量结盟时,它可以产生很好的或极坏的结果:一种意识形态与一个机构的组合,就形成了布尔什维克主义;一种意识形态与一个种族组合就形成了纳粹主义。

——Paul Veyne}

民粹主义本质上是反民主的,所以自打民粹主义发生之后,无数次的被右翼利用。右翼利用民粹主义来反对左翼的社会主义,这才是问题的本质。当然,民粹主义历史上从它出现到今天,没有一次不对经济构成系统性破坏。所以我们一直在认真的观察这一场由英语世界开始发起的民粹主义运动——它的走向趋势,因为它必然导致对经济的系统性破坏,基本上没有例外。

我们现在回到默多克。默多克是1931年生人,澳大利亚墨尔本的人,一个农民的孩子,当然显然不是普通农民,是农场主的孩子。所以他二十年后,1951年前后,应该是在英国上牛津大学,在牛津大学读书。在牛津大学读书的默多克没有完成学业的时候,他的父亲就病逝了,他父亲手上有媒体——报纸,有澳大利亚的报纸,于是这个默多克返回澳大利亚处理他父亲的遗留的产业,开始接触报业,接触媒体这个业务。

默多克是一个非常敏锐的人,我并不知道他真实的种姓、真实的出身,我并不知道。因为我查了很多资料,实际上是查不到的。默多克在澳大利亚处理他父亲的报业的过程中,他开始积攒了对报纸、对传媒、对舆论、对商业的理解。十年左右的时间,到了他1961年前后,他已经开始成熟,然后他开始进军伦敦。而且他非常非常聪明和成功,他控制了《泰晤士报》,特别是控制了英国的这个《太阳报》。

可能好多人不知道,脱欧这个事情是谁操纵的。好,可以基本上明确的指出是默多克老儿通过他的媒体帝国操纵英国舆论,特别是操纵底层民众,就是通过民粹主义这种操弄,导致英国完成脱欧。英国完成脱欧对英国民众是好还是不好?明确的讲,在政治上和经济上对英国都是巨大的伤害,但是它极为有利于美国和澳大利亚的利益,说来真的是很讽刺、也很残酷,英国这是怎么了呢?

好像有一段儿发不出来,我不知道是什么原因,好吧,发不出来我先继续吧,到回头的时候我再重新想办法看能不能发送。我刚才讲的那一段是关于英国脱欧的事情,英国脱欧的事情被一个叫默多克的老儿操纵整个的舆论、操纵底层民意——民粹主义导致英国脱欧。脱欧是对的吗?在政治上、经济上都会导致大英帝国历史性的悲剧,但它对美国和澳大利亚是有利的。我们不知道英国人怎么了,英国如此的一个民族怎么就会丧失自己的思考能力了呢?!

默多克随后又从伦敦到美国,他最终控制了福克斯——整个福克斯电视台、福克斯整个系统,控制了道琼斯,由此建立了完整的默多克的英文媒体的帝国。这么说吧,整个英语世界的舆论导向大体上是在默多克手上。你如果认为经济上有这种共济会或是梅森,那么我可以明确地告诉你,在英语世界、在媒体上面、在舆论导向控制上面,就是默多克,这个人在某种意义上在控制全球的舆论导向。

真神奇,发不出去,好吧。他的星空,能不能说,星空这个也是我们的这个凤凰卫视的这个创始股东或者创始人之一,他也曾经想介入中国的媒体界,后来由于一些特殊的原因未能成功。不过他以间接的方式依旧是深刻地影响着中国的传媒以及中国的舆论导向。不要小看默多克,他的影响力远远超出了英语世界。

今天讲这个事情的之前,我跟朋友聊天,他们还是建议我说别提这个人了,这个人的厉害程度可能是远远超乎我们想象,虽然他今年九十岁了,1931年生人,九十岁了,但依旧是手眼通天,依旧是影响力巨大,是一个非常厉害的角色。我一直在思考一个问题,就是默多克为什么如此钟情于民粹主义?为什么会他这样的去思考问题呢?他到底是想达成一个什么样的目标和结果呢?

因为我们自己很清楚,其实这个世界上约略有四种资本主义的模式,一种模式叫盎格鲁-撒克逊模式,就是英语世界的模式;一种叫莱茵模式,主要是北欧和西欧的模式;一种是日韩的模式,所谓的儒家资本主义;一种是当下的中国模式,其实是国家资本主义加社会资本主义。其实英语世界的盎格鲁-撒克逊模式是经典的金融垄断资本主义模式,它出现了严重的问题;欧洲的莱茵模式是社会市场经济,应该算社会资本主义。

而日、韩,甚至可以说包括台湾,是经典的家族垄断资本主义,是门阀资本主义,而中国是国家资本主义正在向社会市场经济或者社会资本主义过渡的一个特殊历史时期。这个世界的这四种资本主义的模式经过这一百、两百年了,反复地竞争和摩擦,它各自的优劣在慢慢的表达,全世界的老百姓也在慢慢地品味和思考,哪种模式更有效率、更好一些,或者是更具有公平正义呢?这种思考,其实在潜移默化中进行着。

有一件事情其实是很显然的,就是英语世界的金融垄断资本主义在21世纪——就是最近这20年出现了严重的问题。其实我们在读《资本论》的时候,我上堂课在讲就是资本的矩阵——金融资本、商业资本、生产资本和现金,金融垄断资本主义必然将生产资本外挂,生产资本外挂到哪去了?外挂到日本、德国、中国、韩国、台湾这样的地方。外挂之后其实金融资本、商业资本和生产资本……

这资本会产生剧烈的冲突,因为当生产资本成长到一定程度的时候,金融资本和商业资本必然对生产资本进行收割,这就是过去曾经反复发生、最近准备发生尚未发生的事情,就是金融资本发展到一定阶段,必须对商业资本和生产资本,主要是生产资本进行收割。可是,现在遇到的问题是:被收割的资本不再是莱茵模式,也不是东亚的儒家模式,而碰到了中国模式,中国模式主权特征明确,收割起来会遇到巨大的问题。

同时美国人感到非常困惑。中国和德国和日本的情形有不同的地方,就是当代中国竟然产生了一大批优秀的民间的思想家、民间的思考者,而且他们这种思考不是民粹主义的,他们并不是反建制、反体制,而是反西方传统思维逻辑的那样一批人。而且这种中国的目前的这个思想界,这种思考它具有广泛的群众性、人民性,就是老百姓同意、支持,甚至在体制内有强烈的共鸣。

什么意思呢?就是如果美国用传统的方式,通过控制财政、金融、控制舆论来进行类似于1985年对日本的收割,那种模式在中国不行,因为有人说了,“天降组”、“平成战败”、“超级通货膨胀”,把事情说清楚了。如果现在还这样做,那么可能会引起巨大的民意的反弹,而且在操作上面可能也会带来巨大的问题,因为中国不是日本,日本是从天皇到所有的学者、机构,除了少部分的日本的思想家都认为美国是好的,是上帝,接受

并且他的执行机构大藏省和日本央行完全接受“天降组”的这种操弄,就是接受整个的美元与日元的这样的一个变动关系,并且接受1985年到1991年的这样的一个收割的过程,而且完美地把这个东西做完。今天这件事情在这儿做不下去。好多人不理解,为什么2021年会谈共同富裕?为什么我们在2022年会进行伟大的税政改革?我们不会重复1985年的平成战败。它里边有很复杂的东西,但是最本质的东西就是我们不接受默老爷的民粹主义。

虽然,我们在体制上面是承接的苏联的、苏式的社会主义,或者叫国家资本主义模式,但是我们在骨子里边也继承了马克思的社会主义的思想,所以中国人对社会主义这个事情是想得明白的。我们虽然现在不讨论、或者是不谈论、甚至不使用阶级斗争这个词,但我们对资产阶级、对资本还是有着全民的警惕和警觉的,所以我们在遏制资本利得这个方面,我们是有全民共识的,比如说对超级地租的遏制、对超级地租的这种厌恶是有全民共识的。

好多人说谈谈房产税,不用讨论房产税了。房产税不是针对房产,也不是针对居民持有的财产,我们就是要遏制一部分以房产为名义的金融资本的超级资本利得。这个资本利得来自于那些属于国有资产或者是共有资产的土地上的超级地租,这么简单的经济学逻辑,二十年前我就把它写好了,在《香港的超级地租》里边就把它说清楚了,用了这么长时间,香港人在用血的教训——黑衣人的暴动,还不能够清醒,那也确实该踢屁股了。

当然,你只要打开所有的媒体,当下中国的所谓的学者、专家、或者是思想家们仍然在反对房产税。他们反对房产税,但他们不敢说他们支持超级地租,他们只是认为它是一个产业,他们不把它当成金融资本的超级资本利得,他们不把它当成超级地租,金融资本的超级利得,所以,在处理这个事情上面,东方和西方出现了巨大的差异,美国的问题、和中国的问题、和欧洲的问题、和日本的问题、和韩国的问题、和台湾的问题具有同质性,只不过是现在在理论上很难用一个完整的声音做出完整的概述。

我个人认为我们是幸运的一代,就是能生长在此时此刻的中华人民共和国,我骄傲、我感到幸运,因为我们现在这批人可以在11月8号为过去的百年做出总结,这个总结是不是足够的好、足够的到位,它不重要,就是我们可以进行一种百年的思考,这是多么了不起。所以这第三个决议具有历史性质,具有历史性质。我们多么期待11月8号的事情,因为它对中国人的意义太重要了。

请不要说什么房产税或者直接税,请相信,这是中国文明史或者是有记录历史以来一次重要的变法,它可能直追商鞅变法、王安石变法、张居正变法,是中国历史上非常重要的一次变法。中国能否完成两个百年?中国是否可以赢得300年无内战,500年不被外侵,形成一个真正的超过汉唐的盛世?可能就在11月8号这个文件里边了。它不是单纯的一个税政改革,它是变法,变法的核心是如何处理好《资本论》上讲的金融资本、商业资本、生产资本的利得的关系。

其实马克思挺了不起的,其实我们讲《资本论》挺有趣的。我们现在进入到《资本论》的第二卷,这个课程讲了十四讲,快要接近尾声了,但是我很留恋,就是是我水平不够,可能没有把这书讲好。但是我是觉得马克思很了不起,其实他把历史看得明明白白、清清楚楚,是穿越历史的一个重要的人。而当代的西方,特别是当代的英语世界是悲催的,因为我觉得他们的学术界还是有人的,只不过学术在新时代、新经济或者是信息时代被舆论或者媒体湮没了。

谁在治理?谁在主导整个的英语世界?所谓的五眼联盟,内在我不清楚,就是掌控资本的那个人我不清楚,但是媒体我看到了,是默老爷——默多克。我想默多克用这样的方式来处理民粹主义,将民粹主义作为英语世界的一种目前的这样的一个方向或者是走向,可能不是经过严格地或者是冷峻地思考的结果。他,在我看来,依旧是一个精明的商人。

我在思考默多克的时候,我一直在思考民粹主义的商业价值,今天记住,大家,我说的这个东西——民粹主义的商业价值。因为民粹主义在数字经济时代、在信息时代,它是非常容易获得大量的订阅的。而且默多克接手他父亲的媒体之后,他向三个方向做了发展:第一个媚俗,搞一些小报、黄色的东西;第二个快餐,搞短平快;第三个民粹,就是要挑动老百姓敏感的神经。

这三样东西使他获得了成功,这个成功被一个叫黎智英的人在香港和台湾复制。所谓的《苹果日报》仍然是这三个特征:媚俗——搞黄色的东西,搞那些狗仔队、新闻,搞黄色的东西;快餐,不要求思考,就是以最短的方式刺激,刺激完了就是快餐;媚俗、快餐、民粹,挑动你的神经,让所有的事情变成政府和体制的问题,然后你开始反精英、反政府、反体制,说错,反建制,不是反体制,反体制就变成了阶级斗争了,反建制。

90岁的默多克,他思考过他整个的商业逻辑对政治的影响吗?我想他可能对政治的思考没有那么深刻。因为民粹主义破坏性远远大于建设性,而且民粹主义很难形成一种系统地、有效地建设,它和社会主义革命是两回事。请大家注意,民粹主义和社会主义是两回事。社会主义会导致体制的诞生,它反的是旧体制,建的是新体制,它不是反建制,而且社会主义并不反对精英,它也是属于精英政治的一种方式,只不过它是红色精英,它是理想主义者的精英,它不是反精英、反建制、反政府的。社会主义也是有政府的,不是讲无政府的。

默多克的操弄其实已经基本上毁掉了英国,我看不到英国的未来,我也看不明白约翰逊在做什么。因为约翰逊治理下的民粹主义的英国政府,不但是自己在反精英,还反对与欧盟的合作、与中国的合作,只是变成了“五眼”的一个附属机构,变成了美国的一只在英国的狗而已。对他的整个的政治上经济上那种破坏,可能要用四五十年的时间才能完成最终的修复。我不知道英国人怎么了?英国的思想家怎么了?

同时准确地讲,我也不知道澳大利亚人怎么了。因为80\%以上的澳大利亚的媒体都在默老爷手上,默老爷选了另外一个莫老爷——莫里森,还有一个印度的莫老爷叫莫迪,三个“MO”老爷。印度那个莫老爷,印度也可以把他理解成英语世界,他们也在玩民粹主义,而且玩民粹主义必然反华,真的很神奇哟。这种政治操弄不能解决根本性的问题,根本性的问题是金融资本利得侵蚀了商业资本利得和生产资本利得,使得国家国民经济极度扭曲,他需要进行社会主义改造,这么简单的道理。

那么默多克与其说是为英语世界续命,或者说(他不可能是再生)他想为英语世界的旧体制续命,而他续命的方法是用的这一味毒药——民粹主义这一味毒药。所以在政治上我不看好英国,不看好澳大利亚,最终我也不会看好拜登治下的美国。如果这个人换成桑德斯,那么可能会走出一条社会主义道路,社会主义改造。而拜登他承接了特朗普的衣钵,向民粹主义极端进一步发展。

我一直在阅读美国思想家的关于政治问题的一些思考。经济问题的思考,我觉得美国人现在很难走向社会主义,就是美国的“咸水派”现在还处于弱势阶段。那么在政治上的思考,我个人认为处在一个转折关头,因为其实美国的思想家,特别是像基辛格这样的思想家,他们已经很清楚这是不归路,这是不归路。如果这样搞下去的话,美国没有英国幸运,也没有澳大利亚幸运,美国一旦出事,很难再完整的一个国家存续下去,以一个完整的美利坚合众国存续下去。

其实批评默多克那个人应该是英国人、澳大利亚人或美国人,而不是我。我来批评默多克,我的目的并非想给英国、澳大利亚、美国旧体制续命,它们再生的可能性现在看不到,得通过革命来再生了,我没有意愿为他们续命。我批评默多克只是想让中国人知道反华的深层逻辑来自于哪里。当我们懂得了反华的深层逻辑来自于哪里,我们就可以从容面对。不要认为欧盟580票对26票通过关于台湾更名的东西,

你就认为是全世界或者是西方在反华,不是的。为什么英语世界拼命地反华,为什么西方世界拼命地反华?我今天给大家结论吧,我们做对了,我们走在大路上,我们走的是康庄大道。无论是11月8号这个六中全会的决议、还是二十大我们将开启的伟大的变革,我管它叫变法,它都是真正的我们从国家资本主义向社会资本主义转型过程中的伟大的实践。所以它对整个西方世界是一次震撼。

他们仇视的不是我国、党、政府和人民,他们仇视的是我们的道路,我们正在走一条正确的道路。实际上整个四种资本主义:盎格鲁-撒克逊的资本主义、莱茵模式的资本主义、儒家东亚的资本主义和中国模式的资本主义,四个资本主义在新世纪的竞争里边可以给出结论了,我们做对了,所以这个事情确实是很恐怖。那么这个民粹主义的发展会怎样呢?我说过当它极端发展与宗教结合是原教旨,与民族主义发展就是大国沙文主义。

简单地说,极端发展必然引发战争,甚至在某种意义上说,当贸易战、科技战、金融战打不下去的时候,热战可能就难以避免了。所以我们看到所有都围绕着生产资本拥有者,无论是围绕着欧洲、无论是围绕着日、韩、台,还是围绕着中国,金融资本和商业资本围绕着生产资本在挑起一次次的事端。我一会儿会简单谈一下大宗的问题,一会儿我们想说一下,把政治上的事情说完,它必然导致一种热战的这种可能性。

理解地缘政治的本质:大国外交的核心动机在哪里? https://mp.weixin.qq.com/s/BJdEdNhZiZwk6AdbDrRnIg

真的爆发热战其实也没什么大不了的,因为中国现有的体制是最有利于战争的体制、有利于抗疫的体制,其实是我们是最不怕的。所以我觉得默多克以及默多克引导的英语世界的民粹主义,正将英语世界乃至于整个西方导入毁灭的深渊。结论:“默多克是一个恶魔,恶魔默多克他们正在使整个的英语世界,或者是整个的西方世界进入到一个历史性的灾难过程中,而我国必须明白这样一个事件、这样一个历史进程。尽可能的让我国避过这场灾难。”

我国能否回避这样一场灾难呢?我个人认为是可以的。我对东海的问题,包括钓鱼岛、对台海的问题、对南海的问题、对西藏的边境冲突、对印度、对新疆的这些问题,我个人认为皆为疥癣之疾,就是不是什么大事儿。即便是欧洲同意台湾更名,更就更吧,更改成“宇宙国”又能怎样?静下心来完成我们的变革。

静下心来完成这次公元2022年的伟大的变法,使中国走上康庄大道。剩下的问题好解决,我们根本没打算解决台湾问题,我们要解决的是日、韩的问题,我们要将日、韩、东亚的、远东的雅尔塔协议未完成、未解决的问题,因为毕竟美国在日本和韩国驻军,这是雅尔塔协议规定的二次大战之后的这个旧体制没解决,2045年马上就一百年了,百年未决之事项我们必须把它解决,解决完了以后,其实台湾问题也就迎刃而解了。

易言之,中国要解决的是东亚一体化的问题,而非某海某海的问题,是东亚整体一体化、东亚元、东亚联军、东亚一体化的问题,而非一个细节的问题。解决整体性的问题必须用整体性的思维,必须具有哲学高度、具有历史的阔度。我们不必急于——好多朋友批评我两个月前在《亚洲周刊》发表的文章就是《威廉二世的教训》\footnote{https://www.notion.so/97bb653276d84b2b94a878c4dc66a8b8}。威廉二世如果不在1914年发动第一次世界大战,那么他可以忍到1924年、1934年,1924年他就是全球第一了,到了1934年其实他已经……

到了1934年,威廉二世已经无需战争,他不需要用坦克,只要用马克,就可以完成德意志帝国所有的理想。我一再说20世纪是德国世纪,但是由于威廉二世,由于德国的思想家犯蠢,他们拱手相送,把20世纪变成了美国世纪。中国人牢记呀,牢记——没有海,没有东海、台海、南海,没有这些事情——牢记“没有,没有!”。忍住了,把自己的变法做好,结束帝国主义强加在远东的雅尔塔协议,完成东亚一体化。

我有的时候可能令到左翼的很多朋友不开心,我有时候批评多一些,我一再说我们不是小粉红,也不是大粉红,我们的爱国主义不表达于对个别事件上做出过激的反应。奥匈帝国的王子在塞尔维亚被刺的这件事情当然涉及到德意志帝国的尊严,但是十年之后不可以拿回吗?而同时向四个帝国开战,脑子进水了吧?中国此时此刻面对整个英语世界的围追堵截,甚至面对英语世界加西方的围追堵截,我们要做好的事情只有两件。

第一件事情,无论如何完成变法,就是税政改革,我们必须要让资本按照马克思的想法,让金融资本利得、商业资本利得、生产资本利得达至历史的均衡,它是必须平衡。绝不能像英语世界,让金融资本利得变成了超级利得,使得生产资本无法生存,乃至于外挂、乃至于劳动人民绝对贫困化、乃至于消灭中产阶级,这是我们绝对不能允许的,不然我们读《资本论》干什么呢?所以我国的变法将使得中国的经济结构更为均衡和健康,使得老百姓生活更为幸福和安康,这是我们要做的第一件事情,而不是那些海……

第二件事情,我们要再长肌肉,我们的骨骼还需要再强硬一些,我们的肌肉还需要更健壮一些。十年之后或者二十年之后,我们的骨骼和肌肉、我们的半径——我们的能力半径将辐射超过三千公里(现在大概是一千公里左右),超过三千公里到五千公里的时候,我们可以说话了,我们需要忍耐呀,我们现在说话可能会早一些。好吧,今天讲默老爷和民粹主义,可能我有点激动,不过我看穿了默多克这点小把戏之后,我内心深处是充满欢喜的,原来西方世界是默老爷在玩,那么我们就放心了,其实我们很担心有更高深莫测的人……

好,讲几句美联储。我一再说、我一再强调,美元以及整个西方世界的滥发货币已经超越了两样东西:一样东西是超越了经济运行本身所需求的流动性——就是你100万亿的GDP需要那么多的流动性来支撑它,远远超越了生产经营活动所需要的流动性,甚至也远远超过了资产交易过程所需要的流动性。比如说你GDP,以中国为例,我们的GDP是100万亿,我们的中华人民共和国的总资产,按照这个……

按照朱云来他们的计算是860万亿人民币,按照社科院的计算是1500万亿人民币。简化——我们100万亿的GDP、1000万亿的资产,我们需要多少流动性来活化、来运行我们的经济和活化我们的资产呢?现在我们大约是300万亿,那么300万亿是多了还是少了呢?我觉得大体上合适。只不过这300万亿里边有结构性问题,就是有一些人通过房地产等金融资本(其实房地产被金融化以后,它就变成了金融资本),通过超级金融资本利得获取了大概100万亿的浮财,而300万亿负债里边大概有100万亿可能是垃圾债,就是这个债是有问题的债。

什么叫改革,什么叫变法?把那个浮财和那个垃圾债要对冲啊,要让那100万亿浮财重返实体经济,那么中国的经济不就没有问题了吗?!况且这垃圾债就是那100万亿浮财造成的嘛,是我们金融治理的失败、财政治理的失败,导致100万亿垃圾债和100万亿浮财,好多人、好多经济学家或者专家可能看不明白,看不明白无所谓,它就是个事实。不用去夸他们,不要认为中国有几个能讲英文的什么什么什么什么,什么都不是!管得不好,就不如在延安的那些泥腿子,不如那四个解放区——晋察冀、晋冀鲁豫、晋绥、不如山东解放区的泥腿子。

所以我对我国的未来的变法、我国的整个的税政改革寄予厚望,因为这个改革可以导致马克思在《资本论》里边讲的第二卷,其实马克思第二卷和第三卷都是讲资本流转的,才能将资本流转到一个正常的状况。为什么资本不能流入生产资本?因为生产资本的资本利得太低了,而金融资本和商业资本的利得太高了。最近的商业资本的利得真是太高了,因为大宗的价格是商业资本控制的,因为商业资本在全球化的时候,在芝加哥市场,在特定的人群手上,它表达的是一种货币的超发。好,我们回到今天讨论的主题上面去,西方的货币的超发,既超过了运营所需,也超过了资本……

西方的货币超过了商业运营的需求——流动性的需求,也超过了资产流转的需求,那么它必然导致恶性通货膨胀。它率先表达为资产通胀,资产通胀里边表达为股市和楼市的价格的上扬,股市和楼市无法容纳如此巨量的、过剩的流动性,它就会表达为大宗商品。大宗商品之后,那么就意味着老百姓财富和生活被剥夺,那么要不要涨工资?如果涨工资不均衡或者是救助不均衡,那么国家将陷入动荡和混乱之中。这一次,谁为美国买单呢?

其实我挺同情日本、韩国、台湾的,这一次还是日、韩、台为美元买单。所以接下来大家可能会看到难以想象和难以理解的事情,没关系吧,东亚也可能只有经过这样的一个历史性的锤炼,这种高峰、低谷之后,东亚才能走到一起去,才能共同、才能形成东亚共同体。1997年索罗斯收拾了四小虎,1985年收拾了日本,这回可能真的是四小龙、四小虎通通要收拾一遍,然后才能开启一个东亚的新纪元。

简单说一下子,什么叫量化宽松?就是所谓QE,收缩QE就是收缩目前流通中的货币总量,就是收那个M,收那个M。什么叫质化宽松或者质化紧缩?就是加息。量化宽松有利于股市上涨,大家看到了股市疯掉就是量化宽松——过度量化宽松;而质化宽松有利于房子上涨,就是低利息环境,比如说中国。现在呢,已经可以明确地讲,美国将进入质化紧缩加量化紧缩,今年进入量化紧缩,明年进入质化紧缩,那么我们知道会有什么结果。

其实,其实目前进入量化紧缩也仅仅是个姿态,因为美国人现在很难进入一种实质性的质化紧缩,因为实质性的质化紧缩必然导致资本市场的崩盘。我不知道拜登是怎么思考问题的,现在开始到明年选举的时候,好像美国股市的崩盘已成为大概率事件,就是鲍威尔选的这个时机,你说他是聪明还是不聪明?你说他是特朗普立场,还是拜登立场?我记得我在2019年的时候已经在我做的讲座上面好像说过了。

如果我是拜登,我上任第一件事情就是量化紧缩,先把股市打下来,房市打下来,第一件事情就是紧缩;然后就去中国,不是去跑到欧洲——去那儿“我回来了“,是跑到中国解决基建的问题、解决财务问题。两件事情都做对了,那么到了2022年的中期选举,他一定赢的嘛,结果他两件事都做反了。你现在开始量化紧缩,明年质化紧缩,你这不是楼市和股市都跌吗?你明年选举怎么赢呢?大基建没开始,就业怎么办呢?

拜登请的什么顾问啊?什么财务顾问、经济顾问啊?无论是地缘政治还是经济问题处理都是一塌糊涂。而且我个人认为美联储最终在处理量化宽松和质化宽松的问题上都是姿态,而不会进入实质。因为美国如果要真的加息的话,按照现在的通胀的情况,实际上可能带来的一种反身性反噬会更为严重。所以我觉得美国处在一个非常困难的状况,这个困难的状况,现在还不好给出太多的这样的一个清晰的推断。

毕竟昨天晚上这个道琼斯又创历史新高了,美股创历史新高了,可见流动性泛滥成什么程度;同时我们看到了人民币在三天前急剧转强,是离岸人民币急剧转强,在岸人民币跟随,这显然不是中国的操纵;而且我们在香港已经感受到有资金进来了,也就是说美国的头部在走。你知道如果是这样的情况发生,我们这边的模拟就是认为5时代,人民币与美元的5时代,1:5的时代不远了。好多朋友说能给时间节点吗?我说不能给,因为这不取决于美国,这取决于央行。

这取决于我国的央行!5时代为什么我一直担心它来得太早?是因为我们没有离境税啊,我们没有直接税,我们没有封死大门,百万高净值要走,央行如果配合的话,真的1985年那个“平成战败”的事情怕是要重演了,就是我们很担心。当然我们也尽了各种努力,我现在是基本上所有的渠道都被封死了,说话的渠道和写文章的渠道都被封死了,通过好多好朋友,可能……反正我们把该说的话和信息清楚无误的表达了。希望这件事情不要发生,就是5时代不可避免,5时代不可避免了,但我们希望它在一个正常的状况下发生。

最后说几句,其实这个话说完了以后,好多人会可能会有不同的感受。我想说,而且明确无误的说,请所有的朋友一定要坚定不移的支持习主席为首的党中央,支持习主席和党中央准备开启的变法,支持伟大的税政改革,为两个一百年做出我们每一个人的贡献。我们每一个人的贡献就是我们用我们明白无误的声音,表达我们的政治诉求和经济诉求,我们不搞民粹主义,就是要走社会主义道路。

热闹不热闹,激烈不激烈?当然很热闹,国内感受不强烈,香港感受很强烈。前两天,美国某报登了一些东西,说了一些话,我后来写了段东西。哎,我现在写东西确实是发哪儿给哪儿添麻烦,我也不发了。因为现在实际上很大的问题是集中在对房产税的理解和解释上面。这里边大概有三个误区,我简单说一下。第一个误区,地产没有、房产有时限,凭什么收我产税呢?我一再强调这是资产持有税,

不是资产占有税,是你持有它,你持有的时间可能是七十年,那么你持有的这个房产有资本利得,麻烦你交税,就这么简单,麻烦你交税,在法理上、在伦理上、在学理上皆无疑问。所以他们在狡辩,这个狡辩毫无意义,就是任志强为代表的,也包括了一些现在网络上活跃的一些所谓的这个经济学家、专家或者小网红们,不要说那种不着调的话,无论是伦理、法理和学理都可以证明资产持有是可以征税的。资产持有,当你持有资产产生资本利得的时候是可以征税的,这是第一句话。

第二句话,它是有起征点的,是超额累进的,也就是说大部分的人民群众是不需要交这个税的,因为大部分的老百姓那个房子住的就是住的,它不是炒的,他没有资本利得,你征他税干什么呀?80\%以上的人民群众不需要交房产税的。这第二条非常非常糟糕,就是我们的专家、学者,特别是一些坏的网红一直在造谣,就是“这是人头税”。当然也不光我国人这样做,在西方世界很多人,比如香港,他就是要把它变成人头税,你有一尺房子都要交税,坏不坏?它没有起征点,没有豁免,不累进,我占一万平米,你占十平米,那你也得交。

就是我知道我国的学者、专家,我国的这些人,好吧……因为一个伟大的民族需要一些对得起这个民族和这个时代的思考者,或者是思想家,他们也可能没被资本或机构或者政府认可,但是老百姓听得懂就行——就是房产税,第一是资产持有税,第二有起征点,有八级超额累进,80\%的老百姓不需要交这个房产税,所以不用担心,不要去给资本家抬轿子,不要!

最后,我想说房产税的用途,将来征收了房产税,房产税用途是什么呢?我个人一直在想说,它不能变成一般性的政府开支,增加公务员或者增加公务员的收入,而应去填补五险一金,让老百姓不要再通过五险一金来减少他们的工资收入。五险一金就应该用房产税替老百姓交,让老百姓获得五险一金,那个重新变现为现金收入,来增加他的支出和消费,让老百姓真正获得好处,形成少数人对多数人的财政转移支付。如果这样的话,那么这个房产税当然会,老百姓会热烈拥护喽。

伟大的改革其实难在哪里呢?就是——因为为什么要支持习主席、支持党中央,就是因为他们是代表老百姓,向现在既得利益者拿回一些好处,而既得利益者正在控制着立法和——立法、司法、行政都控制在手上,为什么这么难呢?就是难就难在这儿。但我们真的很了不起,我们有一个伟大的领导者、有一个伟大的党,还有你们——无数的可以进行独立地思考的思考者,所以中国才可能在21世纪20年代完成他最伟大的变法。

今儿这个聊天稍稍有些激动,所谈的内容可能也有些敏感,但无所谓了,该说的话还是要说的。而且我相信大家,所以我不认为会有很大的问题或者是麻烦,我不这样的想,不这样认为。至于我们的投资涉及的一些细节呢,我希望大家仍按既定方针办,持盈保泰,等待整个的历史巨变,等待!等待!等待!

我最近也在写一些东西,就是关于对收入和人民币的关系。其实《资本论》里边也涉及到,就是我们是应该让老百姓增加工资性收入、增加他的劳动所得,还是通过人民币的升值来实质增加他的购买力呢?其实每一件事情它都有它的边际,就是我们的人民币升值,比如说从6升到5、从5升到3,就是人民币购买力可能会升值20\%,甚至升值到100\%,人民币的购买力变得极为强大。它有它的好处,它有它的问题,就是我们的出口可能会出现问题,所以它有它的边际,在特定时间里边不能走得太急、太快。同时我们也需要增加老百姓的收入,增加老百姓的收入有两个办法,一个是减少它支出。

比如说你不用交五险一金了,国家直接在房产税里边拿出一部分替你处理了五险一金,你不用交了,那么他就增加收入了。另外一个就是涨工资,涨工资是一个非常复杂的过程,因为中国这么多人,劳动者可能在8亿多,怎么涨?还有退休的人。不会均衡的,有些人会涨很多,有些人可能没有机会涨,还有很多的没有职业者怎么涨?所以在处理这三件事情的时候要找到均衡点。找到均衡点,如果再加上税改非常成功,那么我们真的会进入到一个比较好的时代。

最近疫情有反复,我看春节之前还会收得非常之紧。其实现在我们也知道疫苗可以解决部分重症的问题,但疫苗并不能防疫、并不能隔断病毒的传播。其实现在我们正在等,我们在等特效药的出现,这个特效药可能会出现。那么我们自己尽可能的减少公共活动、减少旅行,保持良好的个人卫生习惯,一定不要生病。

好,今天就聊这么多,我们明天下午3点钟,有未尽的事项我们再做补充。好多朋友有很多建议我都看了,有一些想法,也有一些不同意见,有不同意见可以发表,但不要骂人,不要进行激烈地语言冲突。不好,不好!对我有意见可以提,骂也都可以,但建议你在私信里来说这些事情好吗?好吧,明天见。

\section{资本跨境流动的原因与对策、萎胀的问题}

大家好,今天是2021年的10月30号,是辛丑年的九月二十五日。明天就是万圣节了,就是那个鬼节。今天讲《资本论》第十五讲——资本跨境流动的原因与对策。今天这一讲挺有趣的,是马克思资本流转里边一个比较重要的环节,我试一下麦。然后我们三点钟准时开始。好,一会见。

大家好,今天是2021年的10月30号,辛丑年九月二十五日。明天是万圣节——鬼节,西方的鬼节。到了我们说的贲卦的最关键的时候了。今天我们是正式课《资本论》的第十五讲——资本跨境流动的原因与对策。然后课结束之后呢,我想讲一个这个题目吧,就是我管它叫萎胀,萎缩的萎,枯萎的萎,通胀的胀,不是滞涨,是萎胀。我们即将进入一个萎胀的时代。好吧,今天就这么多。

前两天去港大做了一堂课,这是港大今年的开学的课,过两天我会把课件儿放到我们这个平台上。好多朋友挺厉害的,我不知道用什么方法,他竟然进到港大的网站上听了这堂课,我看有传来的照片,好厉害。那堂课其实大部分的内容就是平时我在这个平台上讲的课的内容,其中的一部分做了一些的收集和整理。当然了,因为谈的是比较专业上市的事情,所以角度略有不同。今天我们讲到《资本论》的第二卷和第三卷的结合部了。

讲课之前说一点轻松的,我今天头一直有点胀痛胀痛的。昨天晚上觉睡得不好,昨天晚上看一部书,是这个吕宁思先生翻译的一个俄国女作家写的书,关于苏联解体之后的一些事情\footnote{二手时间-S.A.阿列克谢耶维奇},他这个搞得我几乎一晚上没睡成觉,所以脑子又胀又痛。我刚才喝了一杯咖啡,现在稍微缓和一点,我把这堂课讲完,在讲课之前说点其他的事情。

这是我前两天的一个聊天儿,就是朋友说:“如果你娶的这个女人是王的女儿、将军的女儿,或者是江湖的女儿,或者是商人的女儿,或者是书生的女儿。那么当你遭遇到危局的时候,比如说你被一群人追杀,那么这个女人会做出怎样的反应呢?”如果是王的女儿,她会迅速地跑出去叫外援;如果是将军的女儿,她会拔剑绕到敌人背后与你一起夹攻;如果是江湖的女儿,她会站在你前面替你挡第一剑;如果是商人的女儿,她会向敌人撒出金币;

如果是书生的女儿,她会从背后紧紧抱住你的双臂,让敌人的长剑刺入你的胸膛。我们在说女儿吗?不是,我们是在讨论中国的知识分子。中国的知识分子是王的女儿?是将军的女儿?是江湖女儿?是商人的女儿?还是书生的女儿?我们俩讨论结果很悲催,就是我国的知识分子是“书生的女儿”,在祖国最需要她的时候,她紧紧抱住你的双臂,让敌人的长剑刺穿你的胸膛,好像这样说有点不公道。

其实我昨天晚上想的事情挺多的,就是看吕先生他翻译那本书。吕先生俄文那么好,文笔那么好,我还真是出乎我意料之外。因为我跟吕先生认识这么多年,我知道他是一个极为杰出的电视台的管理者,他也是一个出得镜头的人,但如此才华确实令我惊讶。另外就是这个作者,这个俄罗斯这个女作家她的视角,我不能说她有高度,但是她的视角,她对俄罗斯知识界的这种反思震撼了我们,所以我们才会发出这样的感慨,我们国家需要一个怎样的知识分子的群体?我们需要一个什么样的女儿?

由此我其实想到了毛泽东关于知识分子“臭老九”的这个说法。知识分子真的很重要啊,中国的书生很重要啊,但是一个伟大的民族在什么时候可以产生这样的一个伟大的知识分子群体?她们或是王的女儿,或是将军的女儿,哪怕差一点是江湖的女儿,再差一点是商人的女儿,但你绝不能成为书生的女儿。因为,因为如果你真的是书生的女儿,你不是在救助啊,你是在帮着敌人伤害你的先生,或者是伤害完你的先生,你再用你自己去贡献给强盗。

说这个事情说得有一点点小激动,好吧,我们回到今天的主题,我们还是讲课。其实今天讲的这个第十五讲涉及到《资本论》第二卷的第三篇,就是社会总资本的再生产和流通,也涉及到《资本论》第三卷的第一篇就是剩余价值的转化。我把它做了归纳和整理。我那天在港大的那堂课里边再次讲了资本矩阵的事情,就是资本:马克思管它叫生息资本,其实就是金融资本;商业资本是商业资本;生产资本,我们管它叫产业资本;还有一个就是储蓄或者是现金,它构成了资本的结构的矩阵。

马克思的书写得非常复杂,就是你要想把它看懂或者是把它拎出来,你得需要高度的总结和归纳能力。我把它归纳成资本矩阵,资本矩阵归纳完了以后,它的意义在哪里呢?它的意义不是做一个分类和整理,而是在资本矩阵里边,你要看到资本的正态分布平衡。如果资本在一个资本矩阵里正态分布平衡了,那么国家治理就是一种国家的经济治理就达到了极高境界,就是金融资本、商业资本、产业资本与储蓄、居民储蓄是一种平衡结构。如果是平衡结构,这个国家经济就会非常稳定、均衡。

然而,然而多数国家在治理过程中会出现问题,以美国为例。美国在上个世纪80年代,可能就是从1980年开始吧,他开始迅速扩张金融资本,然后他将生产资本或者是叫产业资本外挂。也就是说美国从上个世纪80年代开始,用了差不多40年时间,他基本上消灭了居民储蓄,消灭了产业资本,当然他的产业资本总量,我那天在课堂上说总量可能并不低于中国。

好多朋友说:“他都没有工业了,他的产业都跑了,他怎么还有产业资本?”我一再强调,你对产业资本的理解可能有问题,请你重读《资本论》。因为产业资本包含的是土地、机器设备、有形的生产要素,还包括科研、知识产权等无形的生产要素。其中这无形的生产要素里边还包括第三个部分,就是有形、无形,还有一个就是对市场的控制力,就是它是实际上产业资本和商业资本的一种混合能力,就是产业资本和金融资本、产业资本和商业资本的混合的部分。其实它还是属于产业资本的范畴,只不过它……

所以我们在读书的时候不能读死书。那中国是产业资本现在是比较庞大,或者中国是产业资本历史上最庞大的一个时期,但我们产业资本达至高峰了吗?结论是远远未达标,没达峰,没有达标,也没有达峰。什么意思呢?就是中国的人均的资本积累水平,人均的资本积累水平是美国的1/3,就是我们还有2倍的成长空间。那么好多人就理解不了,好多人会说:“那你,还有现在已经开始供给侧结构性改革了,就是三去了,就是要去产能了嘛。你怎么还说没达峰呢?”

这里边涉及到我们对产业资本的理解。如果你看华为,你就懂了,就是华为的投入在科研、在知识产权、在市场控制方面的投入是多么的巨大。如果你再看一下联想,你就懂了,所谓的贸、工、技,哪有技?就是一个工,而且是一个代工,是一个OEM、一个组合。如果是这样的话,它这个产业资本呢,它是一种单薄的、平面的,而不是立体的和丰厚的。中国需要在科研、在知识产权、在市场控制方面还需要有一个长足的发展和进步,我们的空间大了去了。所以基于对《资本论》的研读,我们得出一个基本的结论。

中国的资本积累还可以在此基础上再成长2倍,达到美国现在的水平。同时中国这2倍的增长时间长度应该是30年到50年,它意味着中国维持在5\%以上,最高在8这个水平,应该可以持续30年的高速经济增长,应该有这样长的长度。只不过我们要重新理解产业资本、产业资本的构成,我们重新理解,然后我们真正懂什么叫供给侧改革,真正理解什么叫供给侧,这个供给侧不是把工业扔掉,不是这个意思,然后我们搞金融资本和商业资本,不是,绝对不是!

其实我读《资本论》这是第五读,一读在这个大学,二读在怀柔,三读是在部里工作,四读是来香港,这是为了讲这堂课,我是第五读。第五读《资本论》里边有两件事震撼到我:第一件事情是我用的这个版本,我买的是台湾版的,繁体字版这个台湾版的这个《资本论》。这个台版的《资本论》和以前我看的北京的原来我们1953年那个版本的《资本论》不一样,它虽然是这个还是很大的一部分来自我们的马恩列斯——马克思、恩格斯、列宁、斯大林著作编译局,但它附加了很多他们新的资料和内容,而且其中的相当部分是采用的是英文翻译的这种……

就是以书的厚度而言,它是我原来在北京那套书的1倍,它的内容,它字很小,内容的量差不多是2倍。它里边好多资料我还是头一次看到,所以它还是深深地震撼了我,而且它这个编辑非常细腻,做得很好很好,这是第一个触动到我;第二个触动到我,因为我以前对马克思的二卷和三卷资本流转的部分,其实给予的关注度是不够的,这回在流转这儿我多下了点功夫,其实就是个归纳的功夫。其实读东西也就是演绎和归纳嘛,就是归纳的功夫稍微多了一点,今天我们把这个归纳的功夫展示给大家看一下。

在资本流转的这个矩阵里边,它有它的规律和逻辑,正态分布是国家治理的最高境界。其实马克思他原本的第三卷就是打算写这件事情,写国家与资本的关系,他要写这个东西。那么资本合理的分布应该是在矩阵里边有金融资本、有商业资本、有产业资本、有现金——储蓄,按一个比例关系维持一个稳定结构。但是多数国家治理失败和多数国家走向失败,就是他无法进入一个均衡的结构,像美国这样的国家曾经进入一段均衡结构,而且是很长时间进入到一个均衡结构,差不多将近100年时间。

然而上个世纪80年代开始,我们又要说这个人了,就是里根,上个世纪80年代开始,这个结构被打碎。被打碎之后就变得极度扭曲。我们今天最后结尾的时候要讲什么?就是讲萎胀,枯萎的萎,萎缩通胀,就是要讲为什么美国一定会进入萎胀,就是因为他的结构决定了他的最后的结局,他的这个资本构成决定他的资本结局。那么马克思的理论非常了不起的地方就是为什么资本均衡结构可以达致平衡,而多数国家无法达致平衡,那么这里边就一个深刻的主题出来了,就是《资本论》第三卷里边马克思说的剩余价值的转移……

文明,有的时候是有意思的事情。就是其实我们学哲学是知道的,我们从来不把一件事情轻易地进行好坏的辨别,对错的辨别。文明,有的时候对应的应该是野蛮,但文明真的有的时候本身就是一种野蛮,好不好?野蛮有可能用另外一个视角,比如说上帝视角看来可能是个文明。这又说着说着,大家就糊涂了。我想说什么呢?当美国工会崛起,当美国劳工懂得为自己争取权益,当他们开始极度压缩他们被剥削的剩余价值的时候,好像是一种文明,但它带来了问题。

我不是说剩余价值、剥削是对的。在这个问题上,马克思到了晚年的时候,他也是这样看的。就是你不能过度剥夺工人的剩余价值,但你没有剩余价值,你看看?在没有剩余价值的状况出现了两种类型,一种类型就是《资本论》的第二卷的最后一章和《资本论》第三卷第一章很有趣,出现了两种类型,第一种类型就是美国这种金融垄断资本主义,它会出现一个什么情况呢?由于在产业资本中,工人的剩余价值不能剥夺,或者是剥夺不了了,那么,他就将产业资本外挂。这个剩余价值不是不剥夺了,他不剥夺本国工人。

在中国有另外一种情形,这种情形率先在苏联表达出来,后来中国也出现。不过中国采取了改革的方法,避免了灾难性后果。就是我昨天晚上看前苏联这个女作家对苏联的知识分子她算是深恶痛绝,就是她不会思考,但她到了关键时候,她抱住丈夫的双臂,让敌人来一剑穿心。这就是这帮知识分子,本来她是应该是苏联人或者俄罗斯民族的大脑,结果她就变成这样一个人,她就抱住了你的双臂,让你活活被刺死。我这里边没有对女同学的不尊敬,就是我没有这个意思哈,你们娶的妻子都很好,女朋友也都很好。

中国遇到的问题是什么呢?就是我们1949年建立新中国之后,我们开始节衣缩食来迅速增加我们的产业资本,我们的产业资本迅速成长,成长到一定规模之后,由于我们的产业资本是国有资本,这里边有一个非常重大的问题,就是你读《资本论》的时候,国有企业有没有剩余价值剥削的问题呢?嘿嘿嘿嘿,当然是有的啦,因为当资本存在就有资本利得,资本利得的出处它必然是劳动所得。那么,剩余价值是如何表达的呢?

国有企业初始阶段其实工人的工资是低的,所以我们的地租是低的,工人的工资是低的,虽然我们机器设备什么的可能会贵一点,但是我们的资本积累率极高。就是因为我们成本很低,我们地租低,什么原材料、要素价格全都低,除了设备、特别进口设备贵,其它都低。但是我们发展到一定阶段的时候,我们要改善工人的生活了。怎么改善呢?我们工厂办社会,我们得给工人提供住宅、办医院、办学校、养老、送终,这也是从1949年的时候没有发生的事情,到了二十年之后,就开始出现了。

到了1979年,就是建国三十年的时候,其实我们国有企业的这个劳动所得占据资本利得的比例越来越越高了,就是剩余价值剥削越来越剥削不动了,就是没有了,很少了,所以国有企业就出现了效率下降的问题。这个问题比中国更严重的就是苏联,因为他是1917年,所以他到了七十岁的时候,到了八十年代、九十年代的时候,他已经不行了,他已经出了比较严重的问题。就是他的苏联国有企业的剩余价值趋近于零,无法再进行资本积累,也没有效率。

苏联不是美国,他没有能力外挂;苏联也不是中国,他哪有像中国这么多优秀的知识分子和政治家。中国的知识分子可能还没那么优秀,但中国的政治家非常厉害,从毛泽东到邓小平,真的能思考问题的。邓小平,你看,八十年代没到,1976年之后,1978年、1979年,还没到1980年,开始转型了。怎么转?由国家资本主义向社会资本主义转型。邓小平开始,改革——就是引入社会资本,开放——引入国际资本,社会资本和国际资本的进入,导致新的结构发生。

什么叫新的结构?新的结构就会导致中国出现新的产业资本,这个新的产业资本里边,被剥夺的剩余价值是很高的,那么它就又带来了新的活力。在讨论这个资本矩阵的时候,我想举一个例子。因为我在大连读书,我对大连的印象深刻。大连在改革开放之前,直到我读书的年代——八十年代,它仍然是非常牛的。因为那个时候辽宁叫辽老大,就是它比什么上海、广东(那时候广东还不行呢)、比山东、江苏、浙江、上海都大得多,辽老大,大的时候差不多占国民经济的将近一半。

我读书的时候,辽宁的整个的生产总值占全国的大概也将近20\%,大连是辽宁的一个重工业基地,所以它就工业基础极好、自然环境极好、人文素质极高、交通运输便利。那么提一个问题,为什么崛起的是深圳而不是大连?如果,我相信如果你不听这堂课,或者是你不认认真真去读《资本论》的第二卷的第三篇和第三卷的第一篇,你可能不能用一种经济学的角度来思考和回答这个问题,很多人可能回答不了这个问题,甚至不能回答为什么东北整体衰落。

好,我们回到资本四矩阵。产业资本如果没有足够的剩余价值剥削,没有足够的资本利得,产业资本就会流失。马克思在谈资本流转的时候,他说资本为什么会流转?是因为没有剩余价值了,就是剩余价值没有了,所以它就走了,就这么简单。你看马克思是不是比正常的经济学家要厉害得多,他直击问题本质。那么,资本在矩阵中流转的原因也是剩余价值。这样说来,可能作为一个社会主义国家的经济学家或者经济专家,就有点不好意思了。就是你要维持产业资本足够的剩余价值,才能留住产业资本,就是你不剥削不行。这话能说吗?这合适吗?

头痛的厉害,刚才朋友建议我说喝点酒会好一点,我倒了一杯酒。在资本的四矩阵里边,资本四矩阵的均衡,就是金融资本、商业资本和产业资本和存款达至均衡的原因是每一种资本的资本利得都不过分,是资本利得的均衡达至资本总量正态分布,这是我们读《资本论》第二卷第三篇和《资本论》第三卷第一篇得出的基本结论。

我重复一下刚才我说的话,因为这个话是一个基本的经济学原理,当然是我总结的啦,是马克思发现的,不是我发现的,我把它做一个总结,用现代语言来描述:就是在我们讲的资本四矩阵中如何能达至资本正态分布,是因为让每一种资本利得——金融资本利得、商业资本利得、产业资本利得和储蓄的利息达至平衡,如让它达至平衡,就是每一个部分的利得里边所包含的剩余价值水平必须是平衡的,不能太高,也不能太低。那么我相信好多朋友,听到这儿的时候又明白、又糊涂。

明白的地方就是我们把事情说清楚了,这个事情你不能说剩余价值剥削就一定是坏事儿,也不能说一定是好事儿。我们学《资本论》的目的当然是为了消灭资产阶级,当然不是消灭资本,马克思到晚年的时候也不是这样想的,就是这是我们有些极端的马克思主义者可能有时候会犯幼稚病,是这样想的。其实我们读完的结果是什么呢?就是要有一定的剩余价值剥削,但这个剥削是均衡的、是可控的。我每次讲到这的时候,或者想到这我都觉得有点怪怪的,那我也必须这样讲啊,因为这是我们得出的结论,那么就是资本利得的均衡导致资本正态分布,这是国家治理最核心的原则。

我们回到案例吧,为什么当年……,想起来这个事,这个历史真是有意思啊。1978年,谁跟小平同志说你要在中国的南海边画一个圈,而且这个圈的内容是社会资本,他是怎样想到这件事情的?而且为什么划在宝安县,现在的深圳——宝安县。要知道香港在大清也是宝安县的,深圳也是宝安县的,宝安县很大,这个香港、深圳都是宝安县的,这个深圳的地方画了个圈,这个宝安县的最热闹的市集……

宝安县最热闹的市集就叫东门墟,也叫深圳墟,我在深圳的家,就在东门墟,就是从我的阳台上望下去就能看到东门墟最热闹的那个街,步行的话直线距离应该不超过一百米。当时考虑在深圳这个地方设置特区,小平同志是读过《资本论》的吗?我想他没有读过《资本论》,我讲过邓小平小道——就是江西南昌,他在那个地方做过思考,他认为国家资本主义不成,他认为有问题,但是怎么解决这个问题,如何解决这个问题其实在《小平文集》上并没有揭示,我,也只是猜。

其实这里边有一个人很重要,就是我一直在说麦理浩,麦理浩1971年到1981年在香港做总督,他与邓小平的关系极好,他经常跑到中国来。还有一个人与邓小平关系极好,就是新加坡总理李光耀。其他的香港的一些朋友跟小平同志也有来往,他们可能会介绍日、韩、台、香港、新加坡这些地方的经济状况,可能小平同志有所启发。深圳起步——特区起步是代工,一定要了解历史,是代工。

{\kaishu 中国幸亏有了邓小平,扭转了国家发展的方向。1978年他在访问曼谷和吉隆坡后,来到新加坡。他希望我们一起阻止越南进攻柬埔寨,如果越南进攻柬埔寨,就挫败它。我认为那次访问使他开了眼界。他原本预料将看到三个落后的首都。因为这三个国家都是穷国。然而,他看到的三个首都全都超越了当时中国任何一座城市。他在新加坡访问了大约四天。当他的专机在机场关闭机舱门时,我对同僚说:“那些向他介绍情况的人要倒霉了,因为他看到的新加坡跟他们所介绍的完全不同。”给他介绍情况的一定是来自这里的共产党同情者,那是带偏见的介绍。

他在晚宴上向我祝贺,我问他祝贺什么。他说:“你们有一座美丽的城市,一座花园城市。”我向他表示感谢,但补充道:“你们完全可以做得比我们更好,因为我们是中国南方没有土地的农民后代。你们有学者,有科学家,有专家。你们将比我们做得更好。”他没有回答我,只是用锐利的目光看着我,随后继续转向另一个话题。那是在1978年。

1992年邓小平南下广东,敦促领导层继续改革开放。他说:“向世界学习,特别要向新加坡学习,要做得比他们更好。”我对自己说:“噢,他没有忘记我对他说的话。”事实上,他们是可以做得比我们更好。

邓在新加坡看到,一个没有天然资源的小岛通过引进外国资本、管理、技术,能够给人民创造美好的生活。他返回中国后说服人民需要向世界开放经济。这是中国历史上开始兴旺的时期,是一个重要的转折点。中国从此再也没有回过头去。

——《李光耀观天下》}

什么叫代工?就是深圳这个地方,我举例就行了,就是鸿海。鸿海做手机,它来深圳设一个鸿海的分公司,来料加工,加工完了再运走,它用你这儿的工厂,廉价的地租和廉价的劳动力,然后完成它的电器生产,然后拿到全世界去卖。原材料外边来,东西走了就是叫代工。为什么会想起在深圳代工呢?这跟香港的发展有巨大的关系。1971年到1983年是香港的黄金周期,到了差不多1978年、1979年的时候,香港已经达到鼎盛时期,香港有四万家制造业企业,香港在电子表表芯世界第一,玩具、

成衣制造和化妆品等方面都是在世界排名排的上名号的,就是香港的制造业并非浪得虚名,但是香港要素成本在上升——也就是说香港的地租和劳动工资在上升,它有一种外溢的可能性和需求。我写香港当代史的时候注意到了,就是香港从1983年中英联合谈判之后到1984年,重点是1985年,从1984年、1985年开始,香港的企业,这四万家企业就迅速消失了,大概用了十二年时间,香港的四万家企业就消失没了。现在已经四万家都没了,消失殆尽。资本他们转进到……,就是香港的产业资本变成了金融资本,

它去炒股、炒楼了嘛,它就变成了金融资本。金融资本里边的浮财,就是它房价溢价的部分那浮财呢,被英国人拿走了,我原来说是五千亿磅,后来香港这边做研究的朋友们认为可能是一万亿磅,所以撒切尔夫人才能死了之后进下议院。就是下议院只有丘吉尔和撒切尔两个首相的铜像,就是这么几百年来只有他们俩可以进那个下议院,你就知道她的位置。丘吉尔的原因我们能理解,是二次大战,撒切尔夫人的原因,难道是马岛之战吗?不是的,是她偷钱了,而且她挽救了八十年代的英国,就是大英帝国要亡了,她又把他救过来了,撒切尔夫人,那救的钱就是从香港拿走这笔钱。当然了,香港被拿走了很大一笔钱,但是……

这四万家企业的企业家她拿不走,这四万家的企业的企业家北上了。我就说邓小平太牛了,1978年画一个圈儿,1980年开始建设,到了差不多麦理浩离开的时候,1981年左右,差不多深圳开始具备了物理条件和法律条件,就是那数万的这个工程兵把这个深圳圈起来、围起来之后,这个特区大体上有了。到1983年中英谈判的时候,到1984年、1985年香港的工业开始完蛋、消失的时候,正好深圳特区完整的承接了这四万个企业家的转进,可能这四万家企业家里边大概有将近一半……

香港的四万个企业家,大概有一半是转进了深圳,其他的散落在广东或者是江浙。你如果问我资本转移重要还是资本家转移重要,我毫不怀疑地回答你,没有资本家,只有资本的转移解决不了产业资本问题。比如说他们从香港拿走了一万亿磅的资本到了英国,能解决英国的再工业化吗?就是公有化变私有化这件事办成了,能解决他的再工业化吗?他解决不了,因为他不要香港的资本家,他只要香港的资本。那么资本家到了大陆,是否也带去了一部分的资本呢?

这个其实有完整的统计数据,今天我就不在这儿说了,但是那个量比去英国的那个量少了太多了。如果要是把英国那个、转进英国那个资本转进深圳、广东的话,那么共和国的成长速度就太快了。但是没那么好彩,我们在八十年代中国是资本严重稀缺,所以八十年代是无目的、盲目的海外融资;到了九十年代是项目融资;到了本世纪初十年是企业融资。我那天在香港上那堂课的时候讲了这个历史进程,但无论如何深圳是赶上了这一波,所以中国划了那么多特区,只有深圳一家起来。

我们回到大连,大连有没有条件变成一个经济特区、经济开发区?大连有没有条件去容纳社会资本来改造传统的国有资本或者是国家资本主义,引入社会资本主义,使之焕发出生机和活力呢?理论上是有这个条件的,但在实际运作上是没有可能性的。不是上个世纪八十年代、四十年前没有可能,今天还是做不了。这里边有很多很多的问题值得我们去深思,为什么不能容纳有剩余价值、有资本利得的产业资本呢?

其实我们设计了一个体制,我们所说的政治体制,这个政治体制更适合于国家资本主义,而不是更适合于社会资本主义。小平同志的伟大之处就是深圳特区“特”在哪里?“特”在可以容纳和接受社会资本主义,容许接纳资本主义的一些产业资本利得,就是对剩余价值的剥削。为什么全世界的制造业往中国走呢?我们用马克思第三卷第一篇,我们看剩余价值转移的部分就懂了。地租便宜是国外的十分之一;

剩余价值剥夺,我们的劳工是国外的十分之一;地租是十分之一,剩余价值是十分之一;深圳交通极为便利,货柜车两个小时就到香港了;后来又有了盐田港,又有了高速公路、铁路,交通极为便利;电价便宜、水资源丰厚,还能还能、还能怎样好呢?就是这个剩余价值它也太高了。全世界的资本你不来这儿,那你去哪儿呢?我毕业以后回过两次大连,我去考察过,我在想大连的地租能不能便宜一些呢?大连具备劳动力的价格优势嘛?

我说来可能大家不相信,八十年代在深圳代工工厂里打工的打工仔、打工妹的工资和大连国有企业工人的工资是一样高的。我大学毕业在财政部的工资是七十五块钱,加上乱七八糟补贴大概一百零五块钱;而深圳外资代工厂打工的钱就是这个价钱。你说这个剩余价值问题,当然了,大连的工人他有太多的工资以外福利了,这个是也要算进去。他把这个剩余价值是压缩了,但你说,你说它具不具备地租的条件或者剩余价值条件?当然具备;

但你说它具备制度环境吗?那就完全不具备了。不是过去不具备,整个东三省到今天整个的政治设计跟社会资本没关系,跟社会资本主义没关系;它是一个反社会资本、反社会资本主义的一个制度安排。它是中国最顽固的、最麻烦的官僚垄断资本主义的一种形态。这种形态不打碎它,你想振兴东北?别扯了。那个地方容不下任何的产业资本,能容下商业资本吗?容不下。只剩下金融资本,而且这种金融资本还具有官僚特征,这个地方真的没办法发展。但是你不读《资本论》,你连改革怎么改都不知道。

好,今天这个第十五讲我们分三个部分,我先讲第一个部分。第一个部分就是我们对资本利得的分析。资本利得为什么我们要达至在资本矩阵里边的每一个资本利得的……就是经济学里边我们要讲三样东西,第一个,经济学我们要讲的第一个东西叫供求,价格是由供求决定的;第二个讲的是边际,没有好、没有坏,边际是最重要的;第三个我们讲周期,就是它有一个……经济学说来说去就这三件事,就是供求、边际和周期。在讨论资本利得的时候,绝对的时空均衡是不存在的。

但是存在一个相对均衡的、这样一个存在。那么我们无论是宏观经济学还是微观经济学,我们都要建立在清楚的或者清晰的计算基础上的,就是算法高于看法。那么地租应该占产业资本的比重是多少呢?通常我们认为地租应占产业资本成本的部分、就是商品生产成本的部分的0-30\%。就是深圳初期那个地基本上不要钱的,但是你高过30\%,就是你现在地租到这个程度,恐怕真的深圳也是有问题喽。剩余价值的部分、那么就是平均工资那应该占成本的多少呢?是10\%-30\%,低过10\%也太剥削人了。

好多人可能对深圳没什么太深的印象,我最早去深圳是八十年代末。我去参观过工厂,那个工厂印象不是特别深刻,就是跟北方的工厂我看没有太大区别、人多而已。但是你知道宿舍震撼到我了,那个宿舍基本上跟我大学宿舍一样的,不过它比大学宿舍管理的要严格。有一个香港的朋友跟我说,他办工厂的时候,他说,我们八十年代的深圳的工厂的宿舍,它就是一个大集中营,后来我想了想这个行为有点过分。

但是你知道,它是那种工作强度、那种生活状态、那种工资,它才形成那么、贡献出那么庞大的剩余价值被人家剥削。那个时候可能在产品成本里边,这个剩余价值的部分不足10\%,地租差不多是0-1\%,这个工资不足10\%。所以在整个的生产要素里边,地租、剩余价值和其他,其他包括了能源的部分、交通的部分、水的部分这些其他,其他的这个部分的变化有,但没那么剧烈,但是地租和剩余价值的变动就非常的大,所以深圳不好彩,也开始出现了去工业化。

我其实挺惋惜的(我这话又说得可能有点猛)。像深圳这个地方,他的治理者,就是他的市委书记和市长,应该是中国最牛的政治家或者是经济学家、或者是政治家兼经济学或经济学家兼政治家、或者是了不起的经济学家、企业家或者政治家。其实选一个人是挺厉害的,你选对了这个人,他就这个深圳他是有未来的;你要选错了的话,他一天到晚在那炒楼炒楼,把地租炒到天上去,剩余价值现在根本没法平抑,他企业留不下,连华为都待不住啊,他只能跑到松山湖,甚至再远点,走了,这个不对。

至于说大连或者是东北,振兴它,必须振兴它的产业资本,让产业资本重新流入。那么如何让产业资本流入呢?剿匪,这个匪不是土匪,是官匪,就是原来为官僚垄断资本主义而设计的那套政治体制必须彻底改造,要比当年的深圳特区还要“特”,产业资本就自然会回流了。如果你不这样做,你非要像现在这样的搞,你说资本敢入、敢过山海关吗?过了以后就被官匪吃掉,去多少吃多少,他先吃来的,吃不了他就是内卷,开始吃自己人。所以东北的人才、资本、物资……

东北的人才、资本和物资滚滚南下,全部离开了,其实让人感到感慨万千。就是中国,因为我们中国大嘛,所以它像东北出现这个问题,它给人很多启示,它这个启示其实比美国给我们的启示还要深刻。美国是金融垄断资本主义,东北是官僚垄断资本主义,这两个这有一拼的。它对这个产业资本都是一种干净、彻底消灭的这个意思,就是它都不让你活。美国现在这个情形也是让人看的害怕,它这个……好吧,一会儿我们再聊、聊美国,我们把今天课讲完。生产要素在成本中的变动,决定了资本流动,

这是马克思说的。马克思在《资本论》的第三卷的第一篇里,剩余价值的转移里边说了这段话,这段话后来让我翻译成资本四矩阵和资本四矩阵里边流动,并且把这个流动的原因、把剩余价值、将地租和剩余价值加进去,还加了其他,这是我们今天讲资本利得的主体里边的第一个部分。第二个部分,我们举了香港的例子,就是香港的去工业化是怎么发生的?为什么会香港出现去工业化?香港去工业化就是因为超级地租。就是马克思,为什么马克思《资本论》第三卷用一半的篇幅谈地租,其中级差地租谈了很多。原来有好多朋友说,卢先生啊,你写那超级地租是不是抄马克思的级差地租?我说不是,马克思谈的那个级差地租是真地租,我谈的是金融。

香港如何去工业化的?就是地租。为什么要动地租?就是因为英国人要偷钱;同时,英国人非常幸运遇到了谈判对手。我将来写《香港当代史》会说,今天不说他们的名字吧,我为他们感到羞耻、感到惭愧。在整个的谈判过程中,真正的主权、经济主权,其中最核心的主权是财政主权,最核心的是土地主权,狗日们的给放弃了。所以英国人在1984年《中英联合声明附件三》,他们成功了;所以英国人用短短的十二年时间,掏空了香港的产业资本。想到这儿,我都牙疼,恨死这帮混蛋。

当然,随着麦理浩1971年到1981年香港的经济腾飞的过程中,香港的人的平均工资涨得也很快,当时大概是中国人的平均工资的十倍吧,在80年代,差不多是中国人的平均工资十倍。那么产业转移从地租的角度和剩余价值的角度,其实前提条件已经具备了,也就是说它的产业资本是留不住了,那么就出现了产业资本的大溃败,四万家制造业企业都没了,都走了。很多人说这是一个历史必然发生的现象。我有的时候非常愤怒、生气,有时候我不想跟他们辩论,但我自己私下里会,手会发抖,我真的、有时候想,这是什么人呢?这都是些?前两天吃饭,一个国内的朋友,

前两天国内的一个朋友,还他妈挺有名的,他把我气得我又有点抖,他说:我们应该买单,给英国人点钱怎么了?我说:那是一万亿磅啊!给了,怎么怎么着吧,买单,怎么着吧?我们不是拿回来这块土地了吗?我说:你是谁呀?这儿有七百五十万的生灵啊、生命啊,你把他们的未来拿走了,你还问我怎么着?有这样谈话的人吗?他说:好,卢先生,我拿走了,怎么着吧?我就没法儿聊了,要动菜刀了。一直有人,是我国的这些,我也不愿意用脏话,他们一直在替英国人辩护。

我写过两篇文章,是姊妹篇,这姊妹篇非常重要,一篇是香港的超级地租,第二篇是回到一九八三年――关于香港联汇制度的思考。第一篇写的是财政问题、土地问题,第二篇写的是金融问题,《回到一九八三年》。那篇文章回答了前两天惹我生气的这位国内的著名的媒体人、著名的媒体人、惹我生气这个人。为什么1983年香港全面地超越了瑞士,跟瑞士一模一样啊,无论是GDP还是人均都一样,但后来就完蛋了,为什么?瑞士的产业资本没有流失,而香港的产业资本转移到金融资本,并且流出了香港。

我讲了这个里边的国家治理失败的问题。其实我们学《资本论》,从微观上讲应该是个人可以赚钱,我们看到资本流动嘛;从宏观上应该对国家治理作出贡献。就是我们可以避免在产业资本里边的一些要素,比如说地租的要素和工资的要素成长过快,达至产业资本资本的利得为负数,以至于消失,这件事情是可以学习瑞士的。并不是每个国家、每个地区都治理失败,像瑞士和新加坡,新加坡现在产业资本发展的也很好;不是每个国家一定会失败的,只是香港失败。而且香港的失败,跟香港人没什么关系。

这七百五十万人可怜。它原来的妈妈善良而愚蠢,它后来的这个娘太坏,把钱偷走了。你让香港这七百五十万人活了这么长时间,从1983年到现在快四十年了,还稀里糊涂的,他们都不知道产业资本怎么走了。那天、上个星期一,我去香港参加一个活动,见到了特首,见到了他们,其中有一个人跑我这儿跟我聊天,他说请教,我哪里敢请教?我只是跟他说,我说你们不要轻易谈香港的再工业化,你要知道产业资本为什么会来?为什么会走?你才能说再工业化,再工业化……

再工业化是你一句话就可以的吗?为什么香港搞个数码港是个地产项目?香港现在又在搞北部都市圈。我那天跟某人,我说“你不能见报”,我说“我理解不了,香港的北部就是元朗、大浦这些原来的旧的农地、农业和工业区,现在要在香港北边搞北部都市圈”,我说“北部是都会、港岛是新农村吗?”,我说“可不可以别在名词上面玩儿哩哏儿楞啊?能不能读一下《资本论》,做一些有实质意义的事情呢?比如说如何让产业资本回归,我们如何让产业资本的资本利得能获得它应有的……

能获得应有的资本利得,如果不能获得应有的资本利得,那么你告诉我产业资本的资本家凭什么带着资本去你的北部都会圈发展经济呢?是我脑子有问题还是你脑子有问题?怎么这样想问题呢?”但当然我知道现在就是,像我们这样的人,就新浪微博封我半年,六个月;头条封我半年,六个月,一百八十天。能封的都封了,我没说什么,只是因为房产税要出了,可能大家愤怒已极……

另外,就我讲《资本论》这件事情呢,也令到了一些人感到非常的不开心。我那天港大那堂课可能在北京也引起挺大震动,就是我有一句话,我说:“多数的经济学家是以人的视角来看资本流转,只有马克思是用神的视角来看资本流转,因为神不太计较人的是非。”我说完了以后,当时这个在现场,香港的老师看着我就眼睛里边冒着那种诧异的眼光,我知道大家在想什么。其实讲《资本论》讲到今天,其实你们听课的,你们都懂我在说什么。

好吧,我最后,原来是放在最后一个是讲中国的案例,包括了大连和深圳,我今天在课程之中也有所涉及,我在这里边就不想展开讲了,但我在最后我们今天因为谈的是马克思的资本转移的原因嘛,我想说一下子,最终中国如何处理这个国有资本和产业资本的关系,或者是如何处理国家资本主义、社会资本主义的关系,他们是应该以一个替代一个呢,还是应该共生呢?我把我的结论告诉你们,通过我对《资本论》的研究、对马克思的研究,我不能说我是以神的视角来看经济,

但我至少我不会用一些人的视角来看经济,所以我认为国有资本与社会资本有矛盾、有冲突,但是完全可以共存。因为国有资本要涉及的领域和解决的问题与社会资本是有分别的,它是可以交叉,亦可以不交叉。因为国有资本可以承担一部分的政府和社会责任,社会资本,它还是要剥削一下剩余价值的,国有资本就是少剥削一点,但也不能没有。它可以并存,而且事实上,我国的国家资本主义、社会资本主义并存的模式可能是一个理想态。

我想说什么呢?走了这么多年,可能历史以一种巧合让中国尝试国家资本主义与社会资本主义相融合。融合的结果还融合出一种优势来了,在面对疫情的时候,我们看到这种优势,就是一部分的国家资本主义加一部分的社会资本主义,可能是未来全世界的一种理想模式。纯粹的社会资本主义和纯粹的国家资本主义显然都不行,融合——有机的结合、融合形成均衡,可能是最佳状态。我们的进展今天是第十五讲,我们一共二十四讲,还剩下九讲。《资本论》,就我把第二卷和第三卷来交叉,

第二卷和第三卷内容交叉这样的讲,可能大家觉得有点怪异,因为我下一讲是直接进入到地租的部分,可能就是有点跨越有点大,你们原谅我,因为我想就是按照我认为的合适的方式来处理次序吧,可能这里边的交叉跨度有点大。另外我还是建议大家有空你们去买一套《资本论》,再难读也咬着牙读,顺着我这个思路把它做一遍归纳和总结,总归是有好处的,实在不想读的话那也行,就是你就听完,听这个课也行。好,我们进入到今天最后一个环节。

香港今天还是很热,我说不开空调了,因为开空调有声音,不开空调,开一点儿窗还是一身的汗,穿着短袖,倒是头不怎么疼了。这个朋友这个办法挺好,喝了口酒,好像出点汗,头好一些了。我们讲一下子美国经济,美国经济在第三季开始迅速地萎缩,它的经济增长只有2\%,最后一季可能是负的,负值。其实现在美国经济表达的是特朗普治下美国经济的一种延续, 那种强刺激之后的必然反应。美国的经济是一个什么状况呢?实际上是疟疾。

美国的经济是一种疟疾综合症,它就是一种,也是病毒,但是那种病毒比较麻烦,就是它一会儿热一会儿冷,一会儿冷一会儿热,它就抽风,然后它找不着奎宁,它就拼命的打葡萄糖,就是打完葡萄糖它就很兴奋,一会儿就不行了。现在美国准备退出强刺激,量化宽松(QE)退出,质化宽松变相的加点息,虽然这个进程比较慢。但问题在哪儿呢?问题在哪儿呢?问题是这件事情应该是什么时间做呢?应该是特朗普的时候,就后边的特朗普执政的第三年和第四年就开始,或者拜登一上来就应该做。

也就是说在经济上行期间进行,就是中国的供给侧结构性改革,你怎能在经济已经开始下行的时候供给侧结构性改革?你这个撤火呢,你这是不衰退才怪,不萎缩才怪,你这是什么国家治理水平啊这是! 但我也理解现在美国的状况,就是疫情叠加美国的经济的结构性问题,因为美国他就是物理意义的产业资本没有了,物理意义的产业资本全外挂了,外挂到中国去了,外挂到全世界了,那他没有这块儿了,他只剩下金融资本和商业资本,储蓄也没有,他怎么办呢?他的唯一的方法就是剪羊毛。

所以我在最后一个季度说了,就是我们这个山火贲卦、这个最后一爻,它确实会让一些机构,当然也是一些人、一些机构、一些国家原形毕露。我今天开篇的时候讲了五个女儿。我觉得美国他们的知识分子也不像王的女儿,也不像将军的女儿,也不像江湖女儿,且不像商人女儿,是书生的女儿,就是他们的知识分子死死抱住丈夫,等着当胸的那一剑。哎,这是一个怎么样的国家呢?

通胀没办法停下来,通胀的原因不是供需矛盾。我说了,西方经济学三个分析方法,一会儿是供需,一会儿是边际,一会儿是周期。是供需吗?不是。我同意弗里德曼的说法,通胀它就是一个货币现象,你印那么多的钱,它一定会胀,只不过它会找一个借口来胀,找一个角度开始胀,当然开始在能源上面。现在美国出现了三重结构性的通胀:第一重结构,是供应链断了。供应链断的原因真是很可笑,供应链断的原因不是码头不行,不是运输司机少,

是美国多了大概几十万个货柜,就是多出来货柜没地方放,他那个法律很有意思。摆在码头,那么新的货柜进不来;摆在仓库,仓库放不下。那么这多出来的货柜它占了地儿,就是你一个货柜车司机拉了个空货柜,他要放下空货柜去拉那个装货的去运走,现在他要放这个空货柜没地方放。我都不知道美国这个调度怎么会出现这样严重的情况。所以货柜车司机也很愤怒,就是他没地放那个柜,他这要排长队,要等那个放货柜的位置。然后空货柜、几十万个柜的空货柜,因为船来了,卸了柜,没把柜拉走就走,船又赶回去接着拉货去了。

因为曾经一段时间美国的这个码头堵塞,堵塞了以后就货运下来,空货柜回不来,船就空着回,又来了一船货柜。每次空着走、装着来,你可以想见多出多少货柜来。断链的问题现在看来得用采取强制性措施,因为这涉及到法律问题,你往哪儿堆放都涉及到……因为美国是个私人的地方,你那个货柜放哪还是个大问题。所以他可能要采取紧急的法令来处理这个问题,甚至用军队来处理这个问题,不然的话他这个断链的问题年底之前都没办法解决,他这通胀是,断链是一个大问题。其次,是能源价格的炒作。这能源价格的炒作,就是商业……美国的商业资本金大部分都在芝加哥,我们看得很清楚

这个能源价格炒上去对美国的能源企业确实是有巨大的帮助,有巨大的好处、利益,当然也帮助了中东、俄罗斯。但我们说了,在哲学上没有绝对的好和坏,它是一个辩证的逻辑,就是对能源企业好、对老百姓就它就不好。所以它导致能源传导的基本上跟它相关联的所有的价格上扬,而且这个价格的上扬、这个传导到此时此刻还没结束,能源今年冬天可能还会出现更为严峻的问题。因为,无论是新能源替代,还是传统旧能源的替代、对天然气和油的替代,都需要时间,看来是不行。

最后一个通胀,就是粮食问题。粮食问题,我那天在香港,我跟香港的朋友也是讨论。讨论问题很难讨论,有时候很生气,就有些人他基于立场问题,他经常就胡说八道。他就说,“那中国为什么是粮食大丰收?”粮食丰收,这需要造假吗?他说“你要粮食丰收,你为什么今年粮食进口创历史新高。”我说,“你他妈脑子进水了,那就不能做点战略储备?怎么就不能储备一下子?不能为2022年、2023年、2024年、2025年做一点储备?不行吗?你都知道可能会出现粮食问题嘛,怎么多进口点就不行?”所以,我国可能在两年之内在粮食问题上没有任何压力的。

然而,在南半球问题就变得非常大,非洲、南美洲、澳大利亚可能问题会比较大。因为俄罗斯粮食出口可以自给自足,美国没问题,他们顶多是不出口。中国,主粮是没问题的,副食的部分我们估计囤积的不少,应该大体上过得去。南亚和东南亚今年的情况非常糟糕,前年是蝗灾,今年加上疫情、加上这个自然灾害会比较麻烦。以前,你要知道印度和越南都出口大米的,今年全都停了,所以这个粮食的短缺可能会造成上亿人的灾荒,可能是有问题的。

上亿人的灾荒不是不能解决,但是它带来的可能就是粮食价格,以至于所有食品价格的飞涨。就是从能源最终传导到食品,这个时间周期可能需要六个月的时间,但通胀不会停下来,断链、能源、食品整个的通胀。美国的经济是在急剧萎缩,急剧萎缩的原因是它产业资本不能回归,工业是不可能的;服务业,由于疫情和人们生活习性、习惯的改变,服务业在迅速萎缩。那么美国工业没有,服务业萎缩,你知道他经济总量在迅速地萎缩,萎缩的速度肯定会惊掉你的下巴的。它就是,现在所有的这个经济学家预测的美国那个数字,我都认为是不对的。

当然我们自己也有我们的问题,我们不能看到别人的问题,看不到我们自己的问题。我在港大那堂课上也说了一些想法,今天我就不说了,不说的原因是时间太长了,我真的没劲儿了。另外就是下回聊天我们可以接着说。我要简单的说,就是我们有很多的事情遭到了外边的质疑,特别是房产税,但我知道你们懂我的意思。我本人一直遭到非常严厉的攻击,也很不愉快最近。不过我也无所谓,我自己知道我在做什么,我会咬牙撑到底,把事情做好。今天就说这么多,明天下午三点钟我们再聊,再见。好,再见。

\subsection{聊聊李泽厚先生、对中国经济的预期和对美国经济的看法}

大家好!今天是2021年的11月6号,辛丑年的十月初二。今天我们聊一下子李泽厚先生。李先生是11月2号在美国去世了。好多朋友说能不能聊一下子,因为这个人在我的朋友当中争议很大,特别是北京的朋友说不算盖棺定论吧,还是聊几句吧,我说行吧,这个周末就聊聊他,其实我也有兴趣聊他。另外就是谈几句经济问题,看时间吧,我们再聊几句经济。好,三点钟准时开始,我先试一下麦。

大家好,今天是2021年的11月6号,辛丑年十月初二,终于到了农历的十月了。辛丑年的农历的十月,农历还有三个月这个辛丑年就结束了。我们说了就是辛丑年的最后这一阶段会比较吊诡,甚至是诡谲,一会儿我们讲经济的时候,我们闲聊的时候再说。今天我们就先聊一下李泽厚先生,对李泽厚先生的去世我们表示哀悼……

原本不应该在逝世之后这么早就开始做一些讨论,但我注意到了就是这个李先生走了,李先生的所谓的弟子、门生们不闲着,借哭丧的这样的一个机会,就又开始倒腾私货了。我是真心的想纪念一下李泽厚先生,因为凡是我们这一代人,我是六零后,在八十年代进北京的,无一不深受李先生的影响。要知道八十年代我开始读李泽厚的著作的时候,惊为天人,被震到了。当然了,随着读书、学习,随着自己的成长……

到了上个世纪九十年代再读他的著作,已经看到很多很多问题了。到了新世纪——2000年之后,我来了香港,因为在北京凑不齐李先生的书,在香港就比较全,我又买了一些,再读就觉得有些读不下去,有些读不下去。在中国讲《论语》的人很多,其中可能算大家比较认真地讲的一个是李泽厚,一个是台湾那位大师。怎么说好呢?我们先不说、先不做这方面的评述,我们做一个纪念吧,做一个纪念。

李先生是1930年6月13日生,2021年11月2日过世,湖南宁乡人,头衔是中国哲学家、美学家、中国思想史学家,曾担任中国社会科学院研究员、华东师范大学思勉人文高等研究院、德国图宾根大学、美国威斯康星大学、密歇根大学、科罗拉多学院、斯沃斯莫尔学院客座教授(客席讲座教授)、台北中央研究院客席研究员,曾任职第七届全国人大代表。

在第七届全国人大教科文卫委员会委员。李先生是1948年,时间节点非常重要,因为他1930年生,18岁,毕业于湖南省立第一师范,那个时候第一师范不能算大学,算中专吧。1954年毕业于北京大学哲学系,他成名较早,1955年25岁,在美学大讨论中崭露头角。他挑翻了,不是挑翻,其实他没有挑翻朱光潜,只不过是朱光潜那会儿不挑也要倒了,他正好批了朱光潜,出版了《批判哲学的批判》,那个时候他比较马克思主义化,写了《美的历程》《华夏美学》等著作。

1992年,移居美国,任教于美国科罗拉多学院,1999年退休,住在美国的科罗拉多。1988年当选的巴黎国际哲学院院士;1998年获美国科罗拉多学院荣誉人文学博士学位——荣誉博士,1998年获荣誉博士,这大概算李泽厚先生的一个基本的生平吧。那么我想简单的也介绍几句李泽厚的他所谓的思想体系。

李先生他的贡献在整个在理论上的贡献大概可以概述为三个方面,一个是主体性实践哲学。一谈到主体性实践哲学,是不是大家很熟悉?我们讲心学里边讲了三性:主体性、适应性和创造性。其实主体性实践哲学被西方人概述为中国儒家的这个主情论,西方是这样说的,但我觉得不是,它就是阳明心学里边强调的主体论或者是主体性这样的一个东西。在这里边我今天也不想多说,因为哲学问题嘛……

我想说的是主体性、适应性和创造性,其实就是主体性实践,也可以把它概述为主体实践哲学。当然他的用意不在此,他的用意在主体性实践哲学这个的哲学发现是为了解释他的美学,美学中他用的是一个积淀说或者是积淀论,他讲内在自然的人化和外在自然的人化,内在自然的人化是一种美的感觉,外在自然的人化是一种美的现实。他的积淀说是有他的道理的,因为其实我们在九十年代读他的书是被积淀说震到了。

最后的一个他的一个重要就是情本体论,情本体论其实他是想颠覆宋明理学里边的理,就是不是理,也不是性,而是情——情本体论,勉强能说通吧,我不认为这在哲学上有那么高的意义或者那么大的意义,可能在美学上情本体论是有意义的,就是三个东西,主体性实践哲学、积淀说和情本体论,虽然这些东西在八十年代的时候确实是震撼,因为刚刚经历过文革,我们刚刚开始这个进行这种哲学思考、美学思考,打开眼睛看世界,所以有的时候在这个时候出现一些……

在这个时候出现一些事物、人和事,还是有它的意义的。至于有人说,像余英时他们对他的评价,其实是我所不能接受的。有时候说,他,解放了一代人或者是教育了一代人,那就谢谢余英时,很扯的。就像我这样的人,我是60后,他是1930年,我们是60后,差了30多岁。你如果说他本人能解放我们,我看这话大了;教育我们,这话也有点大;有没有启发?是有的,但谈解放那就有点过了。这个我们先放一下子,

因为李有段话——就是他在接受记者专访的时候,有两个东西他让我感到了不适,这个事情不是今天,是很久了,20多年前就感到不舒服。一段话是他说马克思主义哲学就是吃饭哲学,我觉得作为一个哲学家如此来理解马克思哲学,我觉得惊讶,要么就是你真不懂哲学,要么你就是真没读过马克思。事实上我至今也认为李泽厚先生没有读懂《资本论》,他可能读过,但他没有读懂《资本论》。

第二件事情让我惊讶的是什么呢?是在文革时期,他想读康德,然后他把《毛选》放在康德上面,别人以为他在读《毛选》,其实他在读康德。说这番话呢,怎么说好呢?矫情。我不喜欢一个人这个样子,对吧?上面放一个《毛选》,下面读一个康德。好像整个那一代人都以读一下康德、黑格尔怎样怎样,矫情、表演、骚操作。其实他要是好好读那本《毛选》,或者是好好读那本《资本论》,他在哲学上的高度就不会是今天这个样子,他也不会带出那么多的……

他也不会带出那么多的不成体统的学生。我对80年代的中国活跃的一批,包括后来跟他一起写《告别革命》的刘再复等人,也包括还有些像王若望、刘宾雁这些人,也包括他,我对他们始终无法建立起敬意,或者建立起一个,怎么说呢,就是我有的时候对知识分子,对他们的努力和他们的作品表示某种程度的尊重,但无法建立起敬意。有些人,包括那个天体物理学家,他们摆明了是被美国人拿下了,所以他们基本上是想准备搞颜色革命的。

还有一部分人,我内心深处愿意相信,像李泽厚,虽然他最后1992年他还是去了美国,愿意相信他们离开他们的祖国,仅仅是因为他们自己在美学方面、在审美方面出现了错误,而不是人格被收买,或者是他们被收买或者是被拿下,不是。我相信,我愿意相信他们仅仅是犯了每一个少女青春期都可能犯的那种低级的审美错误。昨天有朋友跟我说才子佳人,他说现在的佳人不爱才子,爱财主,现在佳人也不叫佳人,叫“名媛”。

我们80年代的思想家,其实中国建国以后,在哲学上面可以称之为哲学家的人真的是很少,真的是很少。我个人反而认为可能毛泽东、邓小平在哲学高度上反而是常人所难以企及,而研究哲学的,社科院研究哲学这几块料还真不成,一会儿我讲为什么不成。其实《告别革命》,因为我来香港了嘛,《告别革命》这本书我是读过的,我读不下去,就是他和刘再复两个人的对谈录,他们认为对国内产生了巨大的影响,对社会产生了巨大的影响,老实说那本书不像个样子。

或者这样说吧,就是因为这本书使我彻底完成了对这一批人,不光他一个人(李先生一个人),对他们这一批人的否定。这个否定不全是他们学术能力的否定,也包括了对他们的人格的一种否定。我觉得不可以这样。那么在互联网上可以搜到李泽厚的词条,其实我们的国家,我们的党,我们的国家对他们还是非常宽厚的,虽然可能他的一些作品在国内受到了限制,但是整体上他们还是自由的,可以自由地回国、进出,包括刘再复,他们都可以非常自由地进出和往来。

《告别革命》这本书错在哪里呢?好多人说应该可以“告别革命”,刘再复本人一再解释,“告别革命”不是反革命,不是反对革命,只是告别而已。我为什么觉得这本书使他们的所有的哲学家的光环、史学家的光环荡然无存呢?因为一个哲学家、一个历史学家、一个美学家必须能够客观地去解释历史啊,去解释过去啊,要解释革命啊,要解释孙中山的辛亥革命、毛泽东的三次土地革命、毛泽东的文化大革命,你要解释这五场革命。

你解释不了的原因,其实我也懂。你,一个文学青年想解释一场革命,你最后变成了类似于像章诒和这样的人写的《最后的贵族》,这就是她解释的革命。还有像类似于像严家其这样的人,他们在外边出了一些书谈文革。我看了一下子,不行啊,不行啊,解释革命不懂经济学,怎么解释啊?你没好好读《资本论》,你就想解释革命,还“告别革命”,如果不发动辛亥革命,如果让满清君主立宪,中国会更好。我有的时候觉得怎么了?中国的思想家们。

刘再复说:“李先生的意思是我们喜欢英国的光荣革命,而不喜欢法国的大革命。”问题是他们可能真的并不了解光荣革命。真正的光荣革命实际上是外侵成功,就是侵略者打进了伦敦,攻占了首都,这么场重组了政府,是这么场革命。至于法国大革命发生,我觉得我国的哲学家、史学家、美学家们不下功夫读书,老在那表演矫情,表演读康德、黑格尔。

真懂得康德和黑格尔你就应该知道:存在都是现实的,存在都有它的合理性。革命的爆发并不是革命者几个人的选择,并非密谋、策划与暗示就能成功的。革命需要外在的条件、内在的条件全部成熟,它才能一声炮响爆发这场革命,不是你认为某个人不存在。所以我看着他们在说慈禧晚死十年,中国就不用辛亥革命了,或者是慈禧早死十年,中国也不要辛亥革命了。我不知道怎么就成为哲学家了呢?怎么就成为史学家了呢?怎么就成为美学家了呢?

一个伟大的哲学家必是一个伟大的思想家。他活着的意义首先是要向人民解释已经发生了那段的历史,特别是要解释那段历史的它的合理性、甚至它的合法性和它带来的问题——正反两方面的问题,客观的评价,然后在总结历史的基础上再出发。两天之后我们就要开六中全会了,这回六中将总结历史,当然不会总结辛亥革命,是从上个世纪21年到这个世纪21年做一个历史的总结。这个总结可能远比我国的这些哲学家们的水准要高得多,他们可能要好很多。

革命的爆发一定有深刻的经济原因。革命它是一种政治行动,但它是有深刻的经济原因和经济背景。这个经济原因它的潜台词就是需要重组啦,而且这种重组不是一般性的资产重组,是要变更股东啊。变更股东这种事情才需要革命的嘛,不然改良就可以了。如果只是资产重组或者是利益重新分配,那基本上可以走改良主义的道路,但真的需要变更股东的时候,不革命是不行的。好多人,可能是文人吧……

出于某种,怎么说好呢?出于某种肤浅的直觉,他会认为革命流血太多了,革命牺牲太多了。他有一个非常粗鄙的比较,就是你看英国人没有革命不也挺好吗?美国人没有革命不也挺好吗?当然美国是先是独立战争,然后南北战争还是打,还是革了命的,不是没有革命,是革过的。只是英国,英国是非常特殊的国家,因为我最近在重新写这个《广义财政论》,里边我再加进去一些历史的部分,我在讲大宪章运动和光荣革命。

这个事情在英国的发生,它都具有它的特征。因为大宪章运动是25男爵——25个封建领主把这个王摁在地上签了个协议,就是还让你当王,但是预算我们要管了,税收我们跟你商议好,就按这个数字走,就是一个由25男爵带剑议政形成的这样的这个三权分立。到了光荣革命的时候,三权分立彻底地明确,然后王权衰落,开始君主立宪,是走了这么条路。你说一点儿没流血的,当然不是了。我们的一些学者他认为不该进行股权结构的调整。

不进行股权结构的调整,旧的结构,中国人没法进入现代化,没有办法工业化。所以一次革命不行,辛亥革命不行,还得爆发二次,就是连续三次土地革命,中间还夹杂一场抗日战争,一次民族解放。最后辛亥革命做完了,三个土地革命做完了,最后精神上的彻底解放还没能成,所以还有一次文革,做了整整五次。革命对不对?很多时候革命可能是不一定对的,甚至可能革命带来了极为负面的效应或者是影响,但它爆发是有它爆发原因的。

其实我觉得李先生挺可怜的。第一,作为一个哲学家被捧得那么高,差不多中国当代哲学第一人,这么高的一个人对中国历史做了错误的解读。他两次错误,比较严重的错误解读,一次是在开全国人大会上他的发言,还有一次就是跟李再复的这个聊天上他的发言,当然还有一个接受易中天的那个访问。当然了,他和易中天都是那路的人。我不能用右翼这个词,我觉得用右翼这个词可能是侮辱了右翼这个词,他们不是右翼,包括易先生在内,他们不是右翼,好不好,他们不是右翼。他们不能解释是因为他们的学术能力不够,到不了这个程度,解释不了五场革命。

那么解释不了五场革命,能不能预见未来呢?由于不能解释五场革命,所以不能预见未来。你们知道李先生最后九十高龄死于美国。为什么?为什么要离开祖国?因为他没有看到九十年代和新世纪的二十年中国奇迹般的变化。而九十年代他们这批人为什么要去国?除了有可能存在“天文学家”那种可能性之外,“天文学家”他们被拿下了嘛,要逃亡嘛,很多一部分人是认为中国要崩溃的,要乱的,所以他们决定离开。

中国没有灾难,他们做了难民,流亡了。不是一个人,是一大批在八十年代号称最优秀的知识分子,不是一个人,我认识太多了,他们流亡了,以难民的身份流亡了。他们没有能够预见到未来,不要认为中国崩溃论只是一个叫章家敦的人写的,不是,那几乎是李先生他们这一代人对中国未来的一个集体性的预判,他们全部错了。什么哲学家、史学家、美学家!不能解释过去、不能预见未来,要你何用呢?你只是把青年往沟里带。

主体实践哲学,你的主体性啊!如果你只是用一种西方某些学者的视野、视角来看中国历史问题,你的主体性何在?你的实践理性何在?你的主体实践哲学太假了!所以我们在学阳明心学的时候,强调真正的主体性、适应性。适应性是我们对环境的、对自然的一种适应,其实它里边有美学的含义。创造性是主体性自然化。自然化,他说的第二个积淀说里边,就是内在自然的人化和外在自然的人化,这个倒是有点有意思的。

所有的知识分子,包括我本人在内,我觍称知识分子,其实我的学识还差得很远,我们还得要好好学习。所有的知识分子都有一个问题,就是他们一定会对现实进行批判的,这是知识分子的天性,也是知识分子的使命。因为知识分子,我一再强调,一个民族、一个国家,优秀的知识分子就是这个国家的植物神经体系。他做的工作是批判,其实是纠错。那个批判可能有时候激烈了、超越现实了,可能并不恰当,或者是时机不恰当,或者是角度并不恰当,但对现实的批判,是知识分子的责任和使命。

然而,请记住,你的所有的批评、甚至批判,都必须建立在建设性的基础上。如果是没有建设性的、全面的否定,那么你的批评或者是批判,就会导致一代人、一批人、一群人整体的背叛。为什么会有那么多人逃亡、背叛?哲学家们、史学家们、美学家们,请站起来,向你的祖国鞠一躬,你们做了非常糟糕的事情,非常糟糕!有些人是被人拿来做颜色革命,而有些人……

而有相当部分人,因为我经历了那个时代,所以我知道好多人,可能李先生也属于这类人,学问不周严,没有把历史说清楚,不能够正确的预见未来,而将对现实的批判,不是建设性地对现实进行批判,而进入了全面否定。全面否定的结局是什么?全面否定的结局必然是颠覆、推翻,他真的是走向了死胡同啊。后来我看到刘再复解释说,李先生并不认同那场风波,并不认同,那个解释没有意义。

讨论到一个逝去的长者,我们心存,作为人类、作为中国知识分子的,就是我们保持一份敬意。但是此时不把这个事情说一说,其实我心里边也顶得慌。北京的朋友也是想希望我讲讲,就是因为这个事情它变成了一个结,变成了一个讳莫如深的东西,就是好多人不愿意揭开那段伤疤、那段历史。而好多人,特别是在海外,特别是在当下的海外,在美国、在欧洲,甚至在香港,这些人竟然被当成……

这些人竟然可以当成英雄。我曾经在香港的亚洲周刊上发表过一篇著名的文章:《共和国的希望不在街上》\footnote{https://www.notion.so/95abd1bb3a4847bc87c7b5759489ae2e},我写过这样一篇文章。因为八十年代我在红墙内,我们在拼命地改革工作,中国的改革在飞速地前进,但是在红墙外有这么一批知识分子和学生相结合,起来了,折腾,不让你改,他不让你顺利地改,因为他们放弃了建设性的批判,放弃了建设性的对现实的批判,而走上了全面否定。你知道,对任何事物全面否定的结果就是推翻。

原本,我想把李先生的著作一共24本书给你们念一下子,后来我想了想,算了。有空你们在国内现在也能买到一些,因为国内的出版社,后来很多属于比较专业性的书就给他出了,就给他出了。李先生对自己自视还是比较高的,所以他将大脑冷冻了,将来可能还可以再复苏。我不知道最后李先生临死之前,因为毕竟是91岁的人了,临死之前能不能重新解释一下子,他对中国当代史的看法,能否重新来看看、来解说一下子这五次革命。

至于刘先生虽然也能回国,2004年能回国,在广州,我国还是对他们这些流亡的人挺宽厚的,他还是坚持《告别革命》的主张。后来他在香港这边大学任教,深刻地影响了香港这些学生。我没法说中国这点事儿,他们对整个香港的动乱有着不可磨灭的贡献。怎么说这些人好呢?否定五次革命,基本上从孙中山的民国一直到共和国就全部否定完了,否定完了以后,结论就是这个东西不行。因为是全面否定、无建设性的全面否定,所以导致一些年轻人走上了那个路。

他们自己不能预见未来,他们完全在九十年代的时候,他们是绝望的,认为这个国家是没有希望、没有未来的,是一种绝望的状态。崩溃论是他们的底色。我接触很多流亡在美国的学者,你知道他们不是内心深处,接受了崩溃论,是他们在某种意义上、某种程度上,逃走之后,他乐见你崩溃——一种非常的怪异的、非常复杂的一种心态或者是心情。我有的时候觉得就是很不好吧,我觉得为什么呢?为什么你希望你的祖国完蛋呢?你为什么要推翻她呢?你否定她,推翻她,她完蛋,变成苏联解体。

在面对共和国一次次走出危机、一步步走入辉煌,特别是我们最近这五年,非常艰难的这五年的路程上面,急需要我国的思想家,主要是学习哲学的、史学的、美学的思想家,给予我们的党、我们的领袖一个正确的解说:一种在国内的、在国外的一个正确的解说。急需他们的时候,他们在集体沉默。这令我感到非常的惊讶,而且他们获得那么高的荣誉。

如我国不爆发辛亥革命,如我们没有经历三次土地革命,如我国不经历文化革命,那么我们的状态是一个什么样子呢?可能大体上应该是印度的状况,或者比印度的状况更糟糕一些。因为印度就是股权结构没有重组,当年的有钱人现在还是有钱人,还是那么自私,还是那样的胡闹,甚至有钱人都走了,走到更有钱的地方去了,在英国你能看到很多印度人,在香港你也能看到很多有钱的印度人。难道李先生他们希望的是那样的一个结果吗?所以我觉得他们确实……

所以我知道像李先生他们这些人在文革的时候,本有时间和足够的时间和精力去好好的去读那本《资本论》,但是他——包括厉以宁先生、包括吴敬琏先生,他们都在偷偷的在读欧美的思想家的东西,也没全读懂,然后没有办法解释现实、没有办法解释历史,也不能预见未来。不能预见未来是最糟糕的,因为不能预见未来,他们不仅耽误了他们自己,也耽误了他们的学生、孩子们。你知道当你形成一个对错误的历史评价的时候,你对未来的看法根本没有机会是正确的,不可能会是正确的。

关于对现实批判的建设性问题,今天我多说两句。因为其实这也是我对自己的一个痛苦的思考。因为我也是经常进行严肃的批判的一个人,写文章批评多过赞美。它有它的好的地方,好的地方就是知识分子是植物神经,他是负责修复的,他不是负责赞美的,他是负责修复的。但他的目的是建设性的,是修复,而不是否定和破坏,这一点非常非常的重要。我在接手即将破产的公司的时候,我提出的是“三不”。“不批判”,因为批判是没有用的,肯定是前任做错了。

所以“不批判”不是说不总结经验,是不进行批判,总结是要的,“不批判”。“不等待”,等不来了,没有援军的,没有人来帮你的。最后一个“不”是“不抱怨”。为什么有很多人会抱怨?因为他认为他来错地方了,他呆的时间、地点都不对,他觉得他满腹才华、一身的本事,委屈,知识分子的特征,抱怨。“不批判”、“不等待”、“不抱怨”。“不抱怨”的意思是什么呢?是我们是学习过心学的人,我们知道这个时间、这个地点如此之恶劣,这恰恰是你的生门,这是你的机会嘛。

一个地方,那么的好,时间也对,空间也对,你去的时候也对,舒舒服服。你觉得你的意义呢?你觉得你还有前途或者是前景吗?没有嘛。当你面对一个最糟糕的局面、最烂的局面、最糟糕的时间节点,什么都不对,艰难困苦,这个时候振作起来,开始寻找出路、寻找解决方案,就是建设性,这个时候可能创造奇迹了。我想说什么呢?我想说……

李先生在八十年代末、九十年代初应成为中国的寻路者呀,应该为中国寻一条正确的路啊。你是哲学家、史学家、美学家,你在看完中国历史之后,中国后一个一百年怎么走?你要拿着棍子,走在人民的前面、走在时间的前面,去为共和国、为中华民族去寻路啊,一个伟大的思想家,你怎么走了呢?科罗拉多,那儿有中国之路吗?虽然你走了,我不该进行这样的灵魂的拷问,

不过我想我多说两句也无所谓,因为我们这个平台大家都是我的好朋友。今天这堂课不对外,我们不对外,只是我们大家聊天而已,因为今天就是聊天。我想了想还是想说一些话,说一些心里话,虽然我有一点点觉得作为一个知识分子,有一点点觉得不是很礼貌,别人刚走这么几天,未出头七,盖棺定论是不是不合适?发出如此的疑问是不是不合适?但是凑时间嘛,就还是唠唠叨叨几句,这个希望李先生和李先生这一代的知识分子……

希望李先生和李先生这一代的知识分子不跟我计较,别介意。因为我如不跟我的学生、我的朋友们把这段话说清楚,那么我们如何来解说历史?我们如何来面对现实?我们如何来预见未来呢?我们今后将如何进行建设性的现实主义的批判呢?我们如何做好共和国的植物神经体系呢?所以我想今天这个整个的评述可能还是有意义的吧。好,做一点点小的总结。

在我认识的学者里边,像李先生这样的学者具有广谱性,什么意思?中国在八十年代的一大批的知识分子,我就不列举其他的我的朋友了,他们都具有李泽厚、刘再复、刘宾雁他们这些人他们都具有同质性,可能那代人包括文学家像莫言他们,包括像张艺谋他们,也包含了很多人,他们那一代人他有很多的东西,怎么说呢,绝顶聪明的一代人,然而不扎实,学问不透彻、不扎实,

你看他的书充满了思想火花,有才华,然而没有一个问题解决了,都是蜻蜓落上去,踩了一下就飞走了,没有进行深刻的思考。两个方面都有问题:第一个方面,是书读得不深、不透,这一点我还有大家我们要牢牢记取,书读透是非常重要的,嚼碎了、读透了非常重要的,这是第一;第二,对重大历史问题太轻浮了,不深刻,不能胡说。

我想这里边可能是八十年代,那是一个浮躁的年代啊,他不静、他热闹、热烈、他不安静,很难静下心来把书读透。而且像他们这种有才华的人,在当时是明星来的嘛,网红嘛,当时没网,他是明星,在社科院,那还得了,门口都站着人等着签字呢。所以他静不下心来想问题,学问上是有遗憾的,这是第一条。第二条是我不懂啊,就是一个人难道他不该对他的民族和国家有一点责任和牺牲吗?有点责任感和牺牲精神吗?

如果你的学问够,并且你真心关爱这个国家、民族,你是真心的是利用你的哲学去寻路的、去思考,去思考中国如何使中国的股权结构最优化、中国的资产配置最有效率,让中国的政治稳定、经济飞速发展,如何来实现?如果你找到了路了给中国,你又是这样伟大的思想家、哲学家、美学家,多好啊!你学问也成了,事情也成了,难道不是吗?难道不是吗?同学们!我们真的要好好的注意这两个问题,做好学问,承担使命。

我不知道他读过,李先生读过我的文章没有,相信应该有机会读到。因为好多人他们特别是在美国的人,大部分的人都会读我的东西,好多朋友都读我的东西,然后回国来找我。我觉得如果真读懂了《资本论》,那你就会考虑如何让资本正态分布。我们前两天在港大那堂课上说过,必须让金融资本、商业资本与产业资本达至某种均衡,而达至资本四矩阵均衡的方法是在税政之中,所谓的大宪章运动、光荣革命无非是税政的一个安排,无非是税政的一个安排。

至于诸位八十年代著名的先生们所说的那个民主或者是那个自由,虚无飘渺。没有税政上面的那样的安排,八百年前就具有人民主体性的税政安排——大宪章、光荣革命,没有那样的非常好的税政安排,大英帝国凭何而来呀?为什么会有大英帝国啊?你们既然喜欢光荣革命,不要法国大革命,为什么不去好好读呢?为什么不读书读透了呢?读透了《资本论》,怎么就以为民主会导致大英帝国的结果呢?或者是自由会导致大英帝国的结果呢?

我看了刘再复和李泽厚的《告别革命》里边的对话,谈到了光荣革命和法国大革命。里面涉及到的民主和自由的讨论。如果你没读过康德、黑格尔,你如果没读过马克思的话、你没有读过凯恩斯的话,你一定会被他们的话带沟里去的。我就很怕年轻人读他们这些东西,因为你学问不够的话,你会认为是啊,大英帝国是因为他们崇尚自由,所以有了市场他们就有了效率;法国人崇尚民主,所以欧美就现代化了。你一定会这样想问题,你不会理解二十五男爵带剑议政议的是税政,那里边既没民主,也没……

我国今后要走的路是一条什么样的路?是必须通过税政安排将资本四矩阵里边的资本正态分布,让被通过房地产、通过金融手段、通过互联网拿去变成金融资本的那部分的钱回来。回到产业资本里来,回到储蓄中来,回来,正态分布,我国的经济才有机会再次进入高速增长。好,今天这个总结就算做完了。以后我们一定要认认真真的做学问、吃透。第二,我们这一代人不学他们,我们有我们的责任感和使命感,我们可以出国,但我们不是去否定、去逃亡。

好吧,今天这件事就讲这么多,我今天略微有点激动,因为谈到一些人物,因为这些人物和他们的学生、弟子、门生和我们也曾经产生过激烈的冲突,有好多好多事情就不讲了,今天就说这么多。后边花点时间聊聊,一个是聊一聊对中国经济的预期,一个是对美国经济的走势,说几句我的看法。因为大家处在一个特殊的时间嘛,所以都心里边有点打鼓。因为特别是中国的大A这个调的还是比较深,特别是房地产带动下调的是比较深的,所以怎么看中国经济呢?我聊几句。

中国经济实际上从今年第二季开始就进入了人为的深度调整、人为的深度调整,这个调整包含了我们对混乱的金融秩序进行治理,在治理的过程中包含了两个方向,一个方向是房地产,一个方向是互联网金融、互联网。这两个方向都出现了严重的问题。什么意思呢?因为房子如果是住,它就是产业;如果它是炒,它就是金融;互联网也是这样的。这两个领域我个人认为中央抓的非常及时、非常好。而且我刚开始用词是人为的深刻调整,人为的深刻调整就是他知道该调多深,调多长时间。

现在调整完了吗?没有呢,现在还仅仅是初始阶段,特别是房地产的问题还是初始阶段。互联网的调整倒是进入到中段了,但还没有到全部结束。因为我觉得可能对数据的管理还需要更全面、更系统的一些立法工作。房地产的事情我觉得需要一个理性的价值回归吧。理性的价值回归就是房地产的调整包含了三重含义,我们表面上看的是一个房屋的价格,这只是个表面现象。第二层含义是在房地产这个产业里边。

房地产这个产业里边获利方——就是在这个产业里边获利方利益结构的调整。这里包含了地方政府、包含了地产商、包含了贷款买房者,贷款买房者里边有很多是炒楼者,还有一些是可怜的集资买房、和租客、和租赁者,这四方的利益结构要重新调过来,要调整一下子。这个调整的过程中呢可能会有一些风风雨雨。第三个方面就是我国在通过调整的过程中,通过建立完整的税收制度,因为税收制度实际上是一个统计学。好多人不理解财政,就是财政的基础是统计。

什么意思呢?房子需要实名登记呀,统计的结果实际上就是一个财产登记、财产统计。实际上是一个表面上是一个房产税的征缴,实际上是一个净化过程,财富的合理性净化过程。这个过程会很痛苦,但是他又是共和国走向高度文明的过程中必须跨过的一个门槛。我挺佩服这一届领导人的,这一件事情只要能试点就行了。这个收多少收的方法在其次,因为统计开始了,一旦建立起这个统计体系,那么财产公开的事情就已经是呼之欲出了。好。

说说对中国经济的预期。中国经济在第三季是4.9,第四季如无意外,应该还得向下。有不少人认为可能会见到击穿4见到3,我自己认为可能是在4附近停下来。因为,我们最大的问题、最严重的状况大体上应该是过去了。我不是说人为调整结束了,人为调整远未结束,但是叠加内外的一些的经济上的一些问题,最困难的时候已经过去了,包括能源的紧张,包括这个煤炭供给。

第四季能够,如果能够稳在4附近、4以上就很好。明年,我对明年是有比较大的期待,因为今年地方的换届,六中之后地方的换届就已经开始了,各个省、市的新领导就陆续的要走上岗位了。新领导到位,那么新政就要慢慢铺开,所以明年可能是调整的最后的阶段,调整的最后阶段什么时候结束呢?我觉得可能在二十大之前就算结束了,然后这个明年下半年可能会进入到一个新的增长过程中,新的增长过程中。

我坚持我的看法:一旦直接税立法逐步铺开,那么中国的资本的回流,就是资本在四矩阵里边重新由那些高净值持有的那种准备外逃的金融资本被扣下,回流实体经济,从金融资本、商业资本回流产业资本和储蓄,那么中国经济就会进入到新的增长阶段。我在港大讲课的时候我说了,中国是可以高速度增长的,原因是中国人均的资本积累水平是美国的三分之一,我们还可以再涨两倍,而且这个两倍涨的过程中是可以比较高的速度,所以我们维持在5\%以上的速度完全是可以的、合理的、正常的。所以我个人是这样的认为的。

我依旧坚持认为中国经济会在2022年到2032年,重新重返一个高增长的过程,而且这个高增长的过程中,其他资产的价格,比如说房地产那些价格会保持一个相对的平衡与稳定,而股市将会有一个超级大的牛市,而且是比较长周期的,10年周期的、超长周期的大的牛市就会出现。在这个牛市里边,数字经济就是两化——信息产业化、产业信息化和碳排放权、碳经济会成为主体,会成为新的主体、新的资本的载体,给大家带来非常丰厚的投资收益。

好,讲几句对美国经济的看法。昨天晚上金涨了,当然美股全部拉高创新高,它出现一个什么情况呢?它出现一个要货不要钱的这样一个状况。就是不管美国财政部或美联储采取什么样的政策,即便是开始缩表,而不是——怎么说好呢?由1200亿变成800亿,还是在往外面撒钱呢不是?就是他现在还在往外撒钱,这个局面已经到了一个边际——失控的边缘,所以我自己认为美元,

美元可能在特定周期面对欧元、英镑和日元可能不会有太大的问题,但它面对人民币可能在2022年会出现深调。这么说吧,在2022年人民币与美元的关系进入5时代,几乎是可以确定的事情,就是现在是6.4,6.4可能会进入到6之内,2022年应该能看得到,甚至不排除在调整的过程中,如果有突发性事件、突发性事件可能这个调整的幅度会比较大。对于美国的资产价格包括了股市和楼市的两个方向上的资产价格,我们都认为……

我们都认为目前美国无论是通货膨胀,也无论是资产价格,表达的都是美元通货膨胀,它具有合理性。就是美国股市高、美国房价贵,它具有合理性。因为你觉得高和贵是因为你没注意尺子变长了、尺子变了,就是他那个货币的尺度变了,所以你以为高了其实没高,因为如果你……好吧,如果数月之后你用黄金折算,它并没涨多少。好多朋友说中国房价涨了多少多少多少,我说在特定的时间内,你如果折算成黄金的话,你就知道那个价格其实也就那么个意思吧。

因为涉及到三样东西,所以谈美国经济我是比较谨慎的。一个是涉及到币圈的朋友,就是这个币圈。中国的好多朋友都在币圈里边,特别是那些优秀的孩子们、优秀的孩子们,还有一些,还有一些人、朋友他们都在币圈里混,因为这里,这些年币圈里跑的这些孩子确实赚到钱、赚了不少钱。对美国经济的看法,特别是这一段的看法,涉及他们的生死,涉及到他们的未来,所以不敢说的太多,因为方向你是对的,但是时间节点你能说准吗?况且这是一个非常非常疯狂的一个领域。所以谈美国经济的时候,

另外就是美股、美国股市。我觉得今天讨论美国经济,如果讨论基本面或者讨论技术指标都已经没有意义了。我今天在说这个李泽厚的时候,他那个哲学里边它有一个情本体论,情本体论用在讨论美国股市可能是有意义的。我觉得它不是个理性的本体了,它是个情本体,它是个情本体。这个美国股市在技术上、图形上来看,它根本就没结束呢,他还、那个通道还可能还要走,而且可能时间长度可能还要,就是如果没有突发性事件,

甚至我的好多朋友还在建议我买美国股票。你知道最折磨人的就是这个时间节点,因为你知道了一个趋势,然后它不来,这是个就是,黄金这个问题上就已经非常折磨人了,因为这个时间节点如果你错一个月、两个月,其实大家觉得可以原谅,你错一年试试,大家都熬得够呛,都急了。现在美元面临的问题就是这样一个问题,就是美国政府现在也是打了死结,就是回头太难。要说爱你不容易,回头太难。但是难道不回头吗?难道一条道走到绝处吗?

至于美国的房地产,我倒是觉得,既然我们已经认为2015年之后资本的载体不再是砖头了,所以我倒是觉得没有必要太执着。好,对美国的这个问题我们做一点点的总结,就是我个人认为无论如何在时间概念上它进入到最后环节了,最后的一个周期。我不是建议你,我是说我自己的看法,不构成对你的建议。如你持有的是美元或类似于像美元兑换券——港币这样的货币,

如你持有的是这样的货币,是现金的话,我建议你转换成其他的货币或者是其他货币计价的资产,这是我给你的——不能算建议,因为好多事情你说完了以后,朋友们做完了以后,一个月、两个月他看不到结果,他就跟你急的。所以我不给你建议,我给你说说我的看法,就是这个时间到了避险的时间了。因为我的好多朋友、同学们都希望我在每一次关键节点做出提示,那么我们就提示一下子,如果你持有美元或类似美元,比如说像港币这样的货币,现金的话,麻烦你把它转换成其他货币或者是其他货币计价资产。这是第一句话。

第二句话,个人认为币圈的朋友,因为大部分的币圈的朋友都是用美元做交易的,还是要考虑择机部分的或者全部的离开,转入到实体中吧。也建议,就是离开币圈之后也不再存留于美元或美元计价资产。第三句话,就是现在仍然在市场里边的,无论是在股市还是楼市里边的,我指的楼市不是说你自住啊,好多朋友投资了很多的美国的房地产。

向资本持有课税、向资本利得课税、向资产持有课税,将成为美国的政府的唯一选择,Only one,没有别的路了,所以我也不建议你持有大量的美国的房产。我对你持有的美国的股票, 在比较好的时候择机减持,或者是全部减持,然后也考虑转换成非美元或者非美元资产,当然不一定非要转换成黄金或黄金计价的东西,你随意。我重申,我刚才说了三句话,没有一句是建议,我只是作为一个朋友、一个老朋友,讲几句话提醒,如此而已,好吧。

任何事情,无论是我们讨论中国经济还是美国经济,一个正确的决定都是由时间、空间构成的,脱离了时间节点,脱离了空间位置,毫无意义。如果三年前你拿我这句话说,这是你说的,那么你不对,因为2017年开始我跟大家一起,我们做得很好,尤其是2018、2019,甚至包括2020年,我们都还是做得非常好的。再过两年你再说我这句话也是不错的,每一句话的正确的意义有它的时间、空间的节点,脱离了时间和空间,没有哲学意义了嘛,好不好?今天呢就说这么多,今天所有的内容,都不对外公开,好不好?谢谢你们大家,谢谢!

\section{地租的分类及演变}

大家好,今天是2021年的11月13号,是辛丑年的十月初九。今天我们是正式课《资本论》的第十六讲——地租的分类及演变。这堂课真的重要。其实马克思理论里边的第二个法宝其实就是地租,除了剩余价值之外就是地租。然后我们今天的在这儿可能也得要多下一点功夫。另外,结合今天这堂课,留出一些时间我们评价一下子六中全会。六中全会不是一次评价完,我们先开始吧。

好,开始今天的课程。今天是2021年的11月13号,辛丑年的十月初九。我们先讲这个《资本论》的第十六讲——地租的分类及演变。然后,我们腾一点时间结合今天这个课讲几句六中全会。在开课之前我要表示一下子我对一些我尊敬的领导和老师们(在这个群里边,我才知道),我要向大家表示一下子敬意和感谢吧,这无论如何都是一个很大的这种鼓励,我尽可能的把能做的事情做得更好。

好多人对《资本论》是感兴趣的,包括了国内的朋友,也包括了在境外的、特别是在北美的朋友。一些朋友在听,而且他们也提供了宝贵的意见。每个人自己都有自己心里边的一个马克思,每个人可能有一个自己理解的《资本论》。这堂课能不能起到那么大的作用呢?我没有把握,因为《资本论》一共有三个读法。第一个是把概念读懂,把字认识了、把概念读懂,用马克思的语言来重新解释世界,这是第一层的意思。

第二层的意思是真正的学懂了马克思的方法。其实《资本论》因为写的时间久远,当时所处的环境、所使用的语言和今天有很大的变化,但是马克思研究问题的方法是绝了的,是非常重要的。如何掌握马克思研究问题的方法,用他的方法来进行我们今日之研究,这就进入到第二个境界了。如果能进入到第二个境界,就不必执着于个别的结论和观点。如果执着于结论、执着于概、执着于观点,就容易陷入教条主义,容易陷入教条主义。主观主义和教条主义都是非常有害的。

第三个部分是在马克思《资本论》的概念、逻辑基础上,在马克思的方法论的基础上,逐步形成自己的系统化的理论。这样的话就进入到第三个境界,就是最终形成你自己的东西、你的体系。这样《资本论》就算是有一个境界,读到一定的程度了。我们都在向这样一个方向努力。我也希望在平台上,这些老领导、老师和亲爱的朋友们多提宝贵的意见。因为这本书不仅仅是我的,这堂课不仅仅是我的,也是大家的,我们可能在一个平台上完成一次历史性的飞跃。希望这堂课能起到这样作用。

好,我们今天先进入《资本论》的第十六讲。说到第十六讲,可能好多朋友觉得你怎么又穿越了呢?因为这是《资本论》的第三卷的最后一篇。就是第三卷分为上和下,下的部分就是《资本主义生产的总过程》里边的第六篇“超额利润转化为地租”。我之所以直接进入到《资本论》的最后一章,是想把要点按我的逻辑重新穿起来,因为地租这一章太重要了。所以

恩格斯在编纂《资本论》的时候,把它放到了《资本论》最后一篇,其实我觉得不妥。如果我编《资本论》,应该把它放在第二卷里边,因为地租其实在资本的构成里边。它的意义重大到什么程度?在某种程度上是超过了剩余价值。在我们当代重新回看地租的时候,可能这种理解会更为深刻,因为我写了《香港的超级地租》,建立了一套关于超级地租的概念、逻辑或者是理论体系。超级地租在国内不为一些主流经济学家所接受,因为他们不大接受建基于土地上的这种剥削、食利和资本积累。

这个不能全怪别人不接受、不理解,很大程度上是我对理论的阐释、建立还没有到马克思这样的一个程度。但我们尝试使用马克思的理论体系来解释发生在我们身边的现象。其实在1995年我来香港之后,我一直在研究香港的结构性问题。结构性问题用宏观的视野是看不明白的,因为你打开香港的预算案,我连读了26年,那是一部非常完美的、甚至是堪称经典的预算案,因为它小政府支出很低、税收很低,怎么看怎么好。

你转入微观反观宏观的时候,你就发现了一个严重的问题:就是真正的剥削藏在价内——房价之内、房租之内。而真正的剥削不是传统《资本论》意义上的马克思的工业主、工厂主、资本家对劳动者剩余价值的剥削。它还是对剩余价值的剥削,但是这种剩余价值剥削依托于土地,而非工业的劳动、劳作、工商业的劳动。不是在你劳动价、创造价值中直接剥夺,而是转了一个弯儿,通过一种地租的方式来进行剥夺。这种剥夺更为隐蔽,它避过了《资本论》,避过了我们对剩余价值剥夺的那一套东西。

非常经典的是,香港的土地在港岛和九龙界限街以南的部分是英女王的,当年鸦片战争割让给英女王的,界限街以北的部分是满清政府租赁给英国政府的。一块是拥有主权的,一块是租赁权的。但这两块土地它在使用上面,关于这两块土地的地租方面,其实狡猾的英国人是明白的,而新中国之后的中国的管理者是不懂的,他们大多从战争中走来,走上了领导岗位。

然而他们不知道地租是什么、地租的本质是什么、地租的工具意义是什么,他们不明白地租与权力与资本结合之后是一个什么样的状况。所以1983年当确立中英谈判之后,撒切尔夫人从北京飞回去,第一件事情就是将香港的联系汇率建立起来。1983年12月份,第一个动作是将香港的金融主权送给美国,金融主权——经济主权的一半送给美国。从此以后港币失去了主权货币的地位,变成了美元兑换券。香港到今天,金管局副总裁——一个可以看一切资料的副总裁必须是美联储任命。

第二年中英谈判开始并于年底结束,在人民大会堂发表《中英联合声明》。英国人巧妙地在《中英联合声明》和《中英联合声明》特别是附件三中关于土地问题的规定,他们做了套。这个套导致1984年开始,特别是从1985年开始一直到1997年,12年时间,土地的地租变成了房屋——工商业房屋迅速地涨价、成长,10倍到15倍吧。这样呢,因为大部分优质的房屋、土地和土地的使用权在英资手上,成功完成1万亿英磅的套现。

同学们,不能叫同学,好多是领导。朋友们,不懂地租能治国吗?如果你不懂地租,你就不懂得为什么东北会衰落,你就不懂得为什么深圳会兴起。深圳的崛起靠两样东西:第一个是地租,是出让地租;第二个是剩余价值。其实你读完《资本论》看深圳非常简单,因为土地以极便宜的方式,以开发区的方式,它不需要在香港或台湾付出那么高的地租,并且有比香港和台湾廉价10倍的劳动力,工厂就来了。因为是代工,是转口贸易嘛,所以迅速地进入……

前些日子我跟一些做乡建的朋友们聊天,我说如果读懂《资本论》第三卷最后一篇,你们就知道资本为什么不下乡或者资本为什么可以下乡,或者资本应该如何下乡。资本流动的那个引子,最主要的第一位的是地租,第二位的是剩余价值,其他的要素在后边排列。以后我们在《资本论》做总结的时候,会把这件事情说透,说清楚。但你知道、你知道,多数人读《资本论》第三卷的时候会懵掉的。因为我个人认为恩格斯在整理马克思遗稿的时候,有些地方有些乱。

我们先进入《资本论》第三卷里边地租的定义和地租的分类。在《资本论》的定义里边没有太多的问题,就是“土地使用者缴交土地拥有者的超过平均利润以上的剩余价值”,这仍然是马克思经典的提供的一个概念的语言范式。但这里边有两个东西我们要注意了:一个是土地拥有者,他说的是土地主权;土地使用者,他说的是土地使用权。在这里边包含着一个剩余价值,在这里边。

在事实上,土地的主权、土地的使用权有着多种变形。举例,香港的土地主权现在是归中华人民共和国中央政府的。记住!是归中华人民共和国中央政府的。一定要搞清楚,香港土地的地租应该全部归中华人民共和国财政部。这两句话一定要记住:中国的所有的土地主权都是中华人民共和国中央政府的;她的土地的地租应该归中华人民共和国财政部。我本来应该重复三遍,但我觉得累,我就重复一遍,让你们知道谁在这里边做什么了,该管的人为什么不管。

好,土地的主权和使用权会变形的。因为在有的时候土地会以其他形式……比如说1949年之后,中国进行了土改,将土地分给了农民,然后在人民公社的运动中,在50年代末又集体化又把土地收回来了,就是又变成集体所有。农村的土地是集体所有,而非农村的土地是国有,国有就是中央政府直接持有。那么这个主权的变化里边有什么问题吗?当然是有的,比如说香港的主权,现在是香港的所有主权、土地主权都是中华人民共和国中央政府,对吗?这没有疑问吧?

然而,然而中央人民政府成立了一个香港的土地管理委员会吗?没有。她为什么不成立呢?她不打算管。她为什么不打算管呢?噢,话说到这儿的时候,你们知道我跟好多人的冲突就是由此而起。因为香港的土地主权那个主人不来管理的时候,就会有人帮她代管,代管的那个人叫香港特区政府吗?仔细听着,是香港特区政府在管理香港的所有的国有土地吗?也是,也不是。如果是,那么土地就不是今天这个样子。

如果不是,谁在处理由中央人民政府拥有的香港这块土地上的主权呢?我们或者说是使用权僭越,那么谁拿使用权变成了主权了呢?这里边到底是如何由使用权——本来是主权才有租金的,结果使用权就可以收租了,哪儿出了问题呢?读《资本论》对微观有帮助,但更主要的是宏观管理,对国家管理是有巨大的意义的就是。你国家你拥有这个主权,你谁来管?国资委?财政部?哪个机构?专门委员会?中联办、港澳办替你管土地主权?

回归26年,《基本法》发布,1984年到现在快38年了,主权和主权收益有法条吗?有执行条例吗?没有。在重大问题上,涉及到真正国家主权的问题上,啥都没有。所以我有时候觉得为什么呢?为什么呢?为什么要关掉我呢?要关我180天,不光微博关,头条也关,通通给关了。关了也好,关了我能读书写作。

激烈的争论,从1995年开始直到现在,从未停过。这个争论与我国的领导同志和学者的争论,到香港的政府官员与学者的争论,其实我是挺痛苦的,因为大家没有办法腾空三万尺,站在一个高空、高度上来看待历史和现实,大家每一个人都从利益的角度出发。比如说这一次中国的房产税,反对的声浪可以到滔天的程度,都是站在本位主义、现实的角度来看,没有人思考历史性的问题。

其实我对六中全会是高度评价的,一会儿我们会讲。其中在涉及到经济问题的时候,可能有一个历史性的反思和纠偏,可能文件上没写那么明白,但我看到了那里边的斑斑点点,我们先把课讲完。地租的定义我刚才给了,地租里边的定义的重要的部分——产权。产权的部分和变形,有产权以及次期租权,使用权也有不同的分层,同时地租有不同的缴交的方式。

让大家注意一点,在不同的历史发展阶段会有不同的主权、使用权的关系。奴隶社会为什么要强迫劳动,他不雇佣呢?在奴隶社会,人才是第一生产要素,因为土地并不稀缺,没有产权问题,你开荒就是了。一旦农业高度发达、高度文明之后,土地才是第一要素,人反而变成第二要素,所以人自由了,变成了自耕农或者是佃农,那么土地的产权就变成稀缺资源了,地租就出现了。奴隶社会是不需要地租的,到了封建社会才开始出现地租。那么同学们,资本主义的地租是什么样?

社会主义国家有没有地租呢?如果有的话,那么国有土地上地租是怎么回事?集体土地上地租是怎么回事?地租是以什么样的形式在社会主义国家出现?我国改革开放之前有没有地租呢?改革开放之后有没有地租呢?地租是什么方式呈现?它是否是合理的?是否是合法的?是否是合情的?如果不是的话,我们该怎么办呢?第一个部分定义我就先说这么多。第二个部分是分类,分类我就不想讲那么细了,我希望你们大家上百度百科或者亲自去读一下子马克思《资本论》的最后一篇。\footnote{马恩全集第二十五卷――第六篇 超额利润转化为地租 第三十七章 导论 https://www.marxists.org/chinese/marx-engels/25/038.html}

马克思按照产生的原因和条件,将地租分为级差地租、绝对地租和垄断地租。这个好多朋友说,您是不是因为研究了级差地租,所以才分析出超级地租?我说不是。因为这个级差地租它是根据农田的好坏分类出现级差地租。这个级差地租对当代治理是有启发性意义的,就是我们在胡焕庸线以西的部分,那个土地的农用价值并不高,它有些矿可能有意义,农用价值或者是工商业价值并不高;在胡焕庸线以东的部分它就价值非常高了,而且有着不同的价值。

级差地租对中国未来重新让“三农”振兴有着极大的意义。其实振兴“三农”,级差地租上是一篇大文章。那么什么是绝对地租呢?就是由土地私有权的垄断产生的,租种任何土地都必须缴交一样缴纳的地租形式。绝对地租,绝对地租还是在封建状况下的一种情况。今天绝对地租这个概念可能要发生一些变化了,要发生一些变化了。我今天这个地方先不展开。

垄断地租就是在资本主义农业中,不同级差地租和绝对地租的垄断地租是在特殊有利条件下,由农产品按高于其价值的垄断价格出售形成的地租。这一点是容易理解的,因为一旦,任何东西一旦形成垄断之后,它就会形成一种单边的霸权,它就不,没有讨论或者,或者其他的东西没有了,它就是一个非常、非常残酷的单边霸权。至于地租的缴交,有实物地租、有劳务地租、劳役地租、有货币地租等等吧。我想涉及到地租里边,

我想涉及到地租里边的基本概念和分类,大家上百度百科搜一下子,自己去看一下子,因为如果把这个细分的话,我怕今天要讲的主要内容可能就讲不完了。还有按时代划分,分为封建地租、资本主义地租、社会主义地租。但是我觉得按时代划分,这个社会主义地租的部分需要补,因为马克思只分了封建和资本主义这两种形式,社会主义那时候没有,马克思不知道社会主义地租长什么样,所以没有说。

在这里边,我想讲地租的本质。很遗憾《资本论》的第三卷里边没有细述,但是在马克思其他著作里边对地租的本质是有详细讨论的。地租的本质是商业运行的成本,它在农业上是构成粮食的生产成本,在工业上构成工业生产的成本,商业上构成商业成本,在居住上面构成了居民生活的成本。它是,地租的本质构成商业活动的成本,这个成本合理不合理是个大问题。

因为它是主权收益(刚才我说了是主权收益)。作为主权收益,它又作为使用权的成本,它是主权收益、使用权的成本。如果这个主权的收益过高,就导致使用权的成本过高。那么使用权成本过高意味着什么?这是我们今天要讨论地租的关键所在。地租的本质是,第一是主权收益,第二是使用权的成本。使用者比如说当它成为工业成本的时候,它向主权拥有者提供的那个地租,提供的那个地租如果远远超越了合理的范畴,

如果工业使用的地租,向主权交付的地租远远超过了合理范畴,什么叫合理?合理就是人类或者地球平均地租水平,超越了这个地租水平,那么必然出现去工业化。知道先进国家怎么完蛋的吗?就是主权拥有者不断加大地租的收益,它是主权收益,而增加使用者成本。问题来了,真正懂治理的人要解决一个问题——均衡,主权收益边际在哪里?使用者成本边际在哪里?算过吗?想过吗?用制度来做安排了吗?没有。

我们为什么取消农业税?就是主权是国家的,税收是使用者的、是农民的,收农业税就是租子,我们取消了这个租子让农民种地收益稍微高一点。原来这个租子为什么不收了呢?就是收的这点三五两的地租,还不够养乡村干部的,收了也没用,还给农民。但还给农民,乡村干部也没减少,还得从工业进行转移支付,这是后话。关于工业成本,我在香港,因为我写香港当代史,对香港的这个工商业租我做了一些研究,触目惊心呐!

在1997年之前,主权收益方界限街以南的地方是英女王收,界限街以北的地方是英国政府。因为那时候我们租给他了嘛,我们租给他的时候,我们收了他的租金,他是二房东他又接着租?他们收取了超额的收益,这个超额的收益导致工商业成本过高,所以出现了香港四万家企业,12年消失殆尽,去工业化短暂快捷——基本上就完了。本来是比瑞士还牛的一个地方,没多久就变成了一个娱乐场,马照跑舞照跳,什么地方?

当香港的地的主权的收益以模糊的、超级地租的方式被一部分或者主要的部分被英国人拿走,还有一部分被美资和在香港的港资拿走。于是产生了震惊全世界的什么世界首富。排名,那个时候全世界一百名的富豪里排名几十名香港的地产商和商人——震撼全球。那个时候中国大陆的所有的经济学家为什么那么安静啊,为什么都沉默呢?我1995年来,1999年开始发表《香港的超级地租》。

然而人微言轻,终于不能阻止二十年之后爆发的那场黑衣暴乱、那场灾难。所以我有时候回看我的人生、回看我的研究、回看中国发展的历史,感慨万千。因为不能说什么,你无法去抱怨他人或者抱怨这个时代,你只能检讨自己哪些地方做的不够,做得还不够好、不够努力。我在想若是马克思、若是列宁、若是毛泽东,他们就有办法解决这些问题,但是我们还真的不行。

使用者成本上升到一定程度,导致使用者放弃,这就是去工业化的原因,这就是商业资本撤退的原因,这就是高科技整个的整体衰败的原因。国家治理治理什么呢?当土地主权是你的,你收的租啊,你为什么不收租?或者是你为什么让别人收租,或者是你为什么不把租——你知道这是使用者成本,为什么不把它控制在一个合理的范畴呢?!边际!中国的房地产疯一样的涨,沿海特别是深圳,像疯了一样的涨,是国家收了租吗?为什么不思考《资本论》第三卷呢?

如果不是国家收走了这部分租金,那么你们在让谁替国家收租,僭越了国家的土地主权,向使用者收以暴租、暴利,租税猛于虎也!像深圳为什么迅速地去工业化,连华为都待不下去,使用者成本太高了嘛。你不知道边际在哪儿吗?好意思说你是北大、你是清华、你是教授、你是经济学家,好意思吗?好意思说社科院吗?好意思说党校吗?你们研究什么呢?一个政府收租的边际都看不到,一个使用者收租的成本的边际都看不到。在赞美什么?在批评什么?

好,不能激动。说到地租的本质,我们说到了是主权收益,但是主权会变形。我说过,一再强调,有的时候行使主权者那个叫中华人民共和国政府,政府必须具体到哪一个部门。比如说国资委或者是香港土地管理委员会,哪怕是财政部,你得有个部门来管理这块土地的主权。主权的租金的归属,既然是归属中华人民共和国财政部,财政部就应该对整个的租——谁替你收了?跑哪儿去了?你应该收多少啊?怎么收啊?土地财政应该怎么建立啊?要有完整、系统的想法、思路。

最为要紧的是关于土地与地租必须有宪法层面的制度设计和制度安排,这不是政府、地方政府可以做的事情,这是中华人民共和国中央政府必须完成的非常重要的一件事情。第二个一百年,1949年我们建立了政权,那个时候由于非农村的土地都是国有的,主权是明确的,没有人用其他方式僭越这个主权去收租的嘛。农村集体所有,虽然有“地富反坏右”,但是集体所有之后,集体和国有的性质差不多的,所以也没有这个问题,然而——

改革开放之后,特别是1994年分税制之后,建立了土地财政,却对土地主权和土地的使用权和土地的地租,并没有上升至宪法层面,有一整套的制度设计。没有,到今天都没有!所以几乎出现了学香港学到车毁人亡的程度。感谢六中全会,十九届六中全会。这件事情总算是刹住了,但还没有扳回来,刹住了但还没有扳回来,只要能刹住就有扳回来的希望,中国才有希望。

说几句地租。一旦出现强烈扭曲,怎么办?好多人讲历史,特别是讲中国史,中国史里边风起云涌的,热闹得很。但你看,每一次革命或者是农民起义干什么?均田。看见了吧?人类历史的发展之道,始终围绕着土地与地租进行的,土地主权或者土地使用权,然后地租,当土地主权剥夺使用者剥夺得实在太离谱的时候,革命就爆发了。革命当然不好。

但是,你必须解释革命的正当性啊。从辛亥革命开始,到三次土地革命,到文化革命,五次革命围绕的还是这个东西。革命,对不对呢?因为革命是重组,产权重组,就是让股东结构重组;还有就是资产重组,大规模的资产用于到工业、工商业、城市建设、利润分配的安排。不革命变不了啊。比如说印度不能重组,不能大规模重组,就没法进行现代化和工业化,因为大地主太坏了,他拥有土地主权,他拼命的收租子,然后挥霍,没法发展。

大宪章运动和光荣革命限制了谁?限制了国王?限制了国王的什么?限制了国王的土地主权的收租权,不然他收得老百姓民不聊生,天天打内战。读史别沉浸于华丽的历史叙述,不管是站在统治阶级角度进行的叙述,还是站在平民角度的历史叙述,它的本质,特别是在整个的封建社会的本质,都围绕着土地的主权和地租进行的。看明白这一点,算《资本论》读通了。

除了革命之外,有没有别的办法?有的。古代叫变法,商鞅变法,变什么?废井田、开阡陌、奖励军功,是不是土地主权呢?是不是关于地租呢?当然是了。王安石的改革,王安石的变法,关于什么?青苗法,土地租金,不是土地主权,如果王安石知道该动土地主权,而不仅仅是动租子,可能效果会好一点,不至于变法失败。张居正的改革还是关于地租的,涉及到一点点土地主权,主要是地租的,半成功,没有意义,还是大明亡了。亲爱的同学们、朋友们,我们2022年的改革……

我们2022年二十大的改革,那才是一场真正的变法,它涉及到土地主权以及土地主权的收益、地租和土地使用权,重新规范化。不要认为直接税或房产税是一个简单的事情,为了捍卫五次革命的成果,必须有一次变法,你把它当成第六次革命也成。但是,无论如何,这是中国历史进程中的一个重大转折点。土地问题永远是最核心的问题。我说了,马克思当时怎么想,我现在不能很深切的体会,就是他把地租这个地方,

就是马克思把地租(这不是他的本意),是恩格斯把它放在第三卷最后一篇。其实我觉得地租和这个剩余价值,它是《资本论》里边最核心的两个东西。在讨论地租的时候,我们现在要进化到有形土地和无形土地了。无形土地就是虚拟空间的那个地方,有没有主权和地租问题?现在已经涉及到数字主权了,那么数字主权这个主权的地租该怎么处置呢?这才是新时代《资本论》所需要面临和解决、解释和解决的问题。下周我们聊天,我打算系统的给大家聊一次元宇宙,谈谈另外一个空间土地和地租问题。

在讨论土地和地租的时候,我不得不说一下子,关于“地租资本化”的问题。这是我继“超级地租”之后,提出来的第二个比较重要的概念,就是地租资本化。地租资本化我有一些想法,但是还在写。这个部分,我将来会在新社会主义论(2014修订版)和广义财政论里边都会涉及到,因为无论是《新社会主义论》里边关于地租的问题,还是《广义财政论》里边涉及到财政收入里边的地租的问题,都涉及到地租资本化。因为地租资本化是人类进入资本主义社会之后的一个经典的特征。因为地租资本化了,所以我们看不见,所以我们被地租剥削的时候,我们忘了这就是剩余价值,忘了,可能是故意忘了。

地租资本化的过程中,它出现了主权僭越,这个僭越,往往不是由资本僭越,往往是由政府,就是官商勾结,政府与资本达成媾和形成的主权僭越。政府加开发商加炒房者,他们集体拔高了居民或者是工商业者的使用成本,这种使用成本往往没有人精确计算它的边际,没有人精算。比如说,中国居民的居住成本极限在哪里?我在“超级地租理论”里边给出一个基本的结论。

中国居民的居住成本,不应超过中国居民平均收入之30\%。如果我国的居民的收入是1万元人民币,年收入是1万元人民币,那么他的年居住支出不应超过3000元人民币。如果超过,那么土地拥有者你必须检讨谁在向使用者收取超级地租。

藏在价内,不管是房价、还是租金之内的超级地租是有去向的。研究经济的经济学家,最好你不叫马光远,最好你不是叶檀,最好你不是这种人,你不要成为这种人。你要问一下子,租子去哪儿了呢?主权的地租没有进财政部啊,地方政府拿走了一部分变成土地财政,但大部分的钱,建基于这个上面的钱,算下来账可能会吓死人呐。我国的如果总资产是1千个亿,70\%是……

如果我国的总资产是1000亿,70\%是跟房地产相关联的资产,那么,这些年在这个成长过程中形成了超级地租,可能是我国现在GDP的三倍——300万亿,去哪儿了呢?去哪了呢?地方政府用于建设拿走了,假设1/3,还有2/3去哪儿了呢?去哪儿了呢?如果他们投入到实体经济、投入到工商业中,谢谢!如果他们转移了,甚至他们消费了都好,如果他们转移了的话,国家治理失败。我们已经说过,他们已经转移过相当一笔钱出去了,导致美国、导致五眼联盟的房价暴涨,导致五眼暴涨。

我要说什么呢?地租资本化以后,第一个我们要看主权的僭越;第二个我们要看地租的转移。租子作为一种收入,它决定着国家和民族的前途与命运,因为那是老百姓付出的剩余价值,是血、汗。然后如果它转成了发达国家的收入,变成发达国家的财产,国家治理失败。说到这儿,你能理解十九届六中全会多么伟大吗?多么伟大吗?了不起!但细节还得明年二十大逐步地看到,还有很长的路要走。

地租的资本化导致,地租的过度资本化,资本化没有错,过度资本化导致一个国家的金融体系整体的堕落。金融体系的堕落是什么呢?这里边包含两层含义,第一是政策上的堕落。什么叫积极的财政政策和宽松的货币政策?政策能用积极和宽松吗?它实质是实质负利率的创造,创造实质负利率导致价格的上涨,其中使它产权的租金收益暴涨,非常坏,坏透了!这是一种制度性掠夺啊,它是倚仗着……

主权拥有者是中华人民共和国中央政府啊,它依仗中央政府进行私人的、残酷地掠夺。他们是金融的,我国整个的金融体系没法再说了,整个的金融体系在做啊,在某种财政政策和金融政策的媾和之下,形成一种非常惨烈的局面。还好,终于开始控制、截住、开始改变。好多人对房产税、对现在的现行的制度和政策有很大意见。我其实想爆粗口的,但我这么文明的人不爆粗口,不爆粗口依然压抑不住我内心的那种愤怒。我就觉得我们的老百姓要懂啊、要明白啊,要学《资本论》呐,要知道超级地租啊!

最后一条。这个地租,其实有很大的文章可做,将来我们的平台上的朋友,还有这个国仁乡建这些朋友们,可能我们要出一本书啊,新的一个地租理论要出一本书。我最后一段要说地租理论的当代管理运用。当代管理运用有几个东西,第一个是必须要释放农村及偏远地区的地租,就是国家持有这些土地是你的主权拥有者,你要释放它的地租出来,这些土地就变得非常有价值,就可以吸引人才和资本的进入。释放地租是一个主要的方式方法,也被三千年历史所证明

这需要非常高的学养和政策制定和制度规划的能力,这件事情现在可以做了。地租理论的当代运用呢,必须确立“极限压制”沿海中心城市超级地租的完整的理论说明和制度安排。请注意我的用词是“极限压制”,“极限压制”沿海中心城市的超级地租,因为特别是在深圳、上海这样的沿海地区,超级地租迅速地形成了高净值人群、食利阶级,很扯!

因为沿海中心城市是我国经济高度发达的地区,也必须是制度文明的地区,或者是制度文明高度发达的地区。在这些地区,这些地区就必须务必干净、彻底地消灭僭越国家主权而食利的那些人们。如果做不到这一点,请不要再使用新时代有中国特色社会主义这样的话语,因为这句话说的就是必须完全彻底消灭僭越国家主权的食利阶级。必须的!

中国与美国与西方的竞争,优势在哪里?就在这个地方。请注意我的用词是“极限压制”,因为不能“极限压制”,谈“共同富裕”真的是不行的。第三个部分,放松胡焕庸线以东非沿海区域的工商业地租,必须让工业由沿海进入到二线,但不跨过、不跨越胡焕庸线。过度的集中于沿海的大城市,将深圳、上海、北京、广州、南京、杭州搞成五千万人口的城市是不明智的

应允许胡焕庸线以东非沿海区域释放出一定的工商业地租来,让这个地方工商业得到飞速发展。我们算是现在一线城市的战略腹地,这件事情早该如此,但一直做不出来,原因是我们对地租的、对土地主权和地租的处理上面,始终缺乏足够的哲学高度、经济学高度、管理能力的高度,这个事情非常重要。第四个部分,我们必须在学术上,我指的是在哲学或经济学上确定地租与剩余价值的均衡。

说到这里,大家可能有点犯糊涂,我是说什么呢?我们既要遏制地租资本化后资本在地租上取得的暴利,我们这件事情是肯定的,但是我们有的时候在偏远的农村和西部,又必须释放出一定的地租的利益让资本进入。在胡焕庸线以东的部分,还要在一个特定时期提供工商业的地租,这是一个辩证的过程。剩余价值问题也是这样,我们要尽可能的减少资本对剩余价值的剥削,同时我们也不能没有剩余价值。

我们绝不能通过过度的社会福利使剩余价值负值化,像美国这样的。如果剩余价值趋近于零,地租趋近于零,这个地方就没有资本进入了。要记着我们还是在一个资本主义社会,不管是国家资本主义还是社会资本主义,资本进入是主要的经济发展模式。如果不懂这样的一个逻辑和理论的话,那我们学《资本论》干什么呢?今天总结了这四条,原本还有一些内容,行了,今天就先说这么多。我可能地租的部分看看吧,因为我们一共二十四讲,我只留了一堂课,就第十六讲是讲地租。如果还有时间的话,我们可以再补充一堂课。

这十六讲差不多完了。这套书出来应该挺有趣的。需不需要加更呢?我想在春节前后我们再考虑一下子,看看要不要加更。有些事情好像也没说清楚,今天地租这个事情说的有点激动,但是大体上就这样吧。我聊几句六中全会。六中全会有三个地方非常重要,第一个地方是关于毛泽东的部分,六中这个文件的起草很有功力,这个对我有启发。就是你看看人六中全会的文件里边全部是赞美,没有一句批评,但起到了比批评更有力的、更有效的作用。比如对毛泽东的五个肯定。

五个确认、五个肯定意味着什么呢?意味着推翻了这四十多年对毛泽东的一切的否定。太棒了,用肯定来解决问题多好。肯定了毛泽东不仅仅是肯定了他在新民主主义革命时期的贡献,也肯定了在社会主义革命时期的贡献。这五个肯定很有逻辑性,他肯定的不仅仅是毛泽东,他肯定的是中国共产党、是新中国,是中国革命与中国建设的合理性、正当性、正确性。了不起了不起。我们的敌人一直在利用我们自己的人来否定毛泽东,通过否定毛泽东、否定共产党、否定新中国,以期颠覆和改朝换代。

这个肯定太重要了,了不起了不起。第二个部分,我非常喜欢的部分,马克思主义中国化。马克思主义中国化既有在新民主主义时期的运用。那个肯定是什么意思呢?那个时候肯定毛泽东意味着中国曾经在中国共产党内部形成的本土派与苏联派两派的激烈争斗。我如果把改开之后的三次斗争放进去,就是我党历史上十三次路线斗争斗什么?三个字:主体性。前面那十次大体上都是跟苏联有关的,因为我们留苏的孩子太多,所以有的时候,他就不要主体性,他背“本本”、背《资本论》。

改开后新自由主义出来了,所以就出现了一系列的问题。对毛泽东的肯定,对特别是对马克思主义中国化的肯定。肯定马克思主义就是用一种成熟的方式,严厉的驳斥了新自由主义。这份文件了不起,非常完整和系统地驳斥了新自由主义。不是新自由主义不好和不对,是因为丧失了中华人民共和国国家主体性和人民主体性的新自由主义相当于是资本的主体性,那是我们不能接受的,了不起,文件写得好,再赞一次。

第三个部分,今天有朋友问我卢先生,你关于威廉二世、关于台海问题的忧虑,是不是可以放松一些?我说是了,这个文件里边谈到了这个问题,我感到很高兴。不在某些局部和这个局部的区域和一些问题上、现实的问题上犯颠覆性错误,再点一次赞,了不起。这些事情不要听左边和右边人的煽动。台湾是个岛,不是个船,它开不走。十年之后解决就可以了,不用急,急什么急。至于说周边的这些困难、问题、风险,有没有?

当然有,但你在上升的势头嘛!你不急的嘛!是别人没时间了嘛!别人要完蛋了嘛!你急什么呀?所以我觉得在这些问题上不犯颠覆性错误,不上别人的套,少给我们来奥匈帝国王子被刺的垃圾故事,我们也不会上你这个当。我们有办法来处理相关的事务,但我们不用向四个帝国同时宣战,有必要吗?所以我们会沉雄坚毅、咬紧牙关,解决我们今天说的超级地租问题,彻底的解决超级地租问题。地租问题才是核心和关键。

今天讲的热血沸腾的。这个好多朋友让谈一下子当下的这个经济问题。这个当下的经济问题呢,就放到下周聊元宇宙的时候一起聊吧。至于说美元升值、黄金升这个市场的变化,这个我觉得还没到转折关头呢,所以也没有什么需要叮嘱或者是提醒的。我唯一提醒的就是比特币、币圈儿的朋友要考虑收手,要考虑收手。币圈的朋友还有就是持有大量美元资产的朋友要做一些考虑,其他的就没有需要提醒的。另外我们下周聊元宇宙,顺便把一些事情讲一讲。

农历十月初九农历十月天气急剧变化,然后疫情的情况看来还没到一个可以彻底解决的时候。其实最消耗人的耐力的就是这个时候了,最考验人一个耐力、信心和耐力的时候就到这个时候了。希望大家尽可能的保护好自己,也少出门、少添麻烦。挨过辛丑年,你看看都十月了,十一、十二月。辛丑过后就是既济,我们既济再进行大规模的活动。辛丑持盈保泰、平安为上。谢谢大家,我们明天下午三点钟见。

\section{资本积累和扩大再生产、南非变异病毒可能对市场构成的影响}

大家好,今天是2021年10月27日,农历辛丑年的十月二十三,农历十月就要过了,辛丑年大概还剩下最后的两个月,我说了:辛丑年的最后的这两个月会非常的困难,大体上现在应该已经开始了。我们今天的课是《资本论》的第十七讲——资本积累和扩大再生产。其实资本积累的课我们讲过一次,但应该不是很透彻吧,因为资本积累和扩大再生产不仅涉及到一个企业,也涉及到一个国家的战略选向,所以我今天把这个课再展开一次。

另外,考虑到不讲资本积累和扩大再生产,无法对华为和联想进行比较,这堂课也顺便就将这个问题解决了,余下点时间我们讲一下子新的南非变异病毒可能对市场构成的影响,我们还是从资本流转的角度来讲一下子。我先试一下麦,然后三点钟我们准时开始。今天的课可能理论的部分会比较枯燥一些,最后的部分会有趣。但不讲理论是不行的,因为今天是《资本论》第十七讲,我们还是按着大纲走吧。好,一会儿见。

大家好,今天是2021年10月27日,农历辛丑年的十月二十三日。亦如我们年初的一个基本的预判,今年的最后一爻会是变得非常的麻烦,或者昏暗或者是惨烈。另外,整个一年的时间,资本市场也确实走在一个通道里边,我们经历了很多的折磨、经历了很多的事情,不过我略感欣慰的是所有人大概还是持盈保泰,总算是挨到了最难的时候。这个最难的时候一旦过去,明年就是另外一番新天地了。

今天我们讲两个部分:一个部分是《资本论》的第十七讲,这个第十七讲实际上是马克思第二卷第三篇的第二十一章,它的原标题是“积累和扩大再生产”,我将这个章加两个字作为我们今天的课:资本积累与扩大再生产。如果你们记得的话,我们在第十一讲讲过资本积累的秘密,那个时候讲资本积累的秘密是从微观的角度讲的,实际上是讲我们如何进行资本积累。今天这个资本积累和扩大再生产呢,我们是从宏观上讲,可能更大的意义是讲国家治理和国家的一个治理的哲学逻辑和国家治理的政治经济学吧,稍微高一点。

另外我们腾出点时间来,因为如拜登所愿,这个“光棍节”双11那一天南非产生了新的冠状病毒的变异,病毒变种。已经传入香港,香港发现了一个印度裔的香港人从南非带来了这个病毒。不好的消息是它已经来了,好的消息是我们中国有机会获取这个病毒并可以迅速地研发新的疫苗和有针对性的治疗方案或者是治疗的药物。现在所有的国家都在跟时间赛跑。

我们最后的部分讲这个冠状病毒对经济的一个影响。其实这个,撇开阴谋论的角度看,这个冠状病毒来的这个时间节点是非常有趣的。它再一次向世界证明这件事情没完,就是冠状病毒这个事情没完,它会导致一个什么样的结果呢?会导致经济衰退乃至于大萧条吗?我们今天留点时间来讨论这个问题。如果是衰退,整体的衰退或大萧条会导致一个什么样的景象?我们最后的部分留出时间来讨论这个问题,这个问题也确实是变得非常重要了。

好,我们开始进入《资本论》第十七讲:资本积累和扩大再生产。讲资本积累,我们就必须得讲三个概念:工业化、去工业化及新型工业化。我想呢,因为马克思是从宏观角度来出发、来理解和讨论这个问题的,所以我们不讲(这个第十一讲是讲“资本积累的秘密”),我们这回是讲资本积累它作为一种哲学的逻辑,它到底是个什么?我们把它说清楚。我们先进入第一个环节,就是工业化、去工业化和新型工业化。我们讲的一般意义的工业化概念就是马克思定义的工业化的概念到底是一个什么样的东西呢?

马克思理解的工业化实际上我们可以简化为机械化。实际上是瓦特发明了蒸汽机之后,开始由机械能代替人的劳动,极大地增加了这个人的能力,它是通过机械能扩展了人的力量。并且由于机械化生产需要的是一个规模的经营、需要标准化、需要分工与流程、需要庞大的市场。实际上第一次的工业革命也奠定了资本主义的基础,资本或者是资本主义与工业化是相伴生的,或者是工业化导致了资本主义。

通常我们将讨论工业化围绕着这样的三个标准:第一个标准就是一个地区或者是国家的工业生产总值超过GDP的50\%,我们认为这个国家已经开始进入到工业化了,或者是初步完成工业化;它第二个标准是工人人数超越了农业就业人口(就业人数);第三个标志是城镇化率超过50\%。从这三个标准来看的话,我国早已经完成了初步的工业化,或者是我们已经进入到工业化的较高级阶段。就是我们虽然比西方国家晚了150年左右的时间,

但是我们在1949年之后的30年左右时间就初步完成了工业化,并且在改革开放之后的40年时间将工业化水平提高到一个崭新的高度。工业化之后会出现两个方向:一个方向就是传统的工业化向新型工业化的升级。这个升级过程做得最好的当然是美国了,美国的新型工业化做得非常之好。那么我们要给新型工业化做一个定义,什么是新型工业化?如果说传统的工业化是机械化,那么新型的工业化我们概述为信息化。信息化我们通常又把它说成是“两化”。

“两化”就是我们唠叨了一年多的时间了,就是信息产业化、产业信息化。当然信息产业化、产业信息化这个过程中不仅仅是新型工业化的,它不是全部内容,它是其中一个内容。新型工业化还包括了分子生物科技的工业化,是分子生物科技的实际运用以及工业化,其实这个领域可能是新型工业化的更高阶段了。在分子生物科技这个领域,美国也是走在了前面,就是信息产业化、分子生物科技,美国都走在前面。那么我们是处在第二梯队,就是新型工业化我们处在第二梯队,在信息产业化、产业信息化以及分子生物科技领域,我们都在进行努力的追赶。

当然追赶也发生了一些,其中也发生了一些异化的问题,一会儿我们会讲到。在讨论工业化和新型工业化的时候,我们必须要注意一个概念,就是去工业化。去工业化这个情况在每一个国家,每一个已工业化国家都会发生。其中有些发生比如说香港,香港在1983年的时候还有40,000家制造业企业,其实工业化水平、工业化的程度是非常之高的,但在短短的十多年时间完成了去工业化。它去工业化是非常彻底的,就是它既没有传统的工业化了、工业了。

香港既没有传统的工业了,也没有新型工业。它没有完成新型工业化,那么去工业化的结果它就变成了一个,我们讲它是第三产业占主体,或者是将服务业占主体,其实主要是指金融、旅游和其他,或者是房地产——所谓的香港的四大支柱,反正是跟工业没有关系了。香港是去工业化的典型,而且是一个极糟糕的典型。因为它丧失了创造价值的能力,所以这个地方出现2014年的“占领中环”和2019年的“黑暴”是自然而然的事情。所有去工业化的地方如果出现了异化或者是异变,必然导致社会动荡。

我说了,就是所有已工业化国家必然会面临的去工业化是因为传统工业必然伴随着一系列的问题,比如说污染——对自然环境的污染;比如说劳动保护——就是社会保障等等一系列的问题。那么低端的制造业它就需要出现转移,中国也面临同样的问题。转移不是个问题,转移也是正常的现象,问题是你能否进入到新型工业化的时候完成对传统工业的转型,或者是某种程度的转移或者是放弃。如果你不能完成新型工业化而放弃传统的工业化的话,那可能是一个悲剧。准确地讲,在美国、欧洲都出现了类似的问题。

在这里我们简单地讲一下子马克思关于资本积累的一些定义或者他的分析框架。马克思真的是天才。马克思通常把资本分为不变资本与可变资本,其实他说的是资本的结构。不变与可变的区别在于不变资本中不包含剩余价值,而在可变资本中包含剩余价值,但它们两个资本又是哲学的辩证,就是可变资本构成不变资本的积累。我说清楚了没有啊?我再说一遍,不变资本中不包含剩余价值的生产,可变资本中包含剩余价值的生产,可变资本中剩余价值的生产再回转到不变资本构成资本的积累,大体上是这个意思。

传统的不变资本和现代意义的不变资本已经开始出现了巨大的差异。传统的不变资本主要是指固定资本或者是生产资料。但是原始意义的不变资本是指土地、厂房、机器等等。现代意义的不变资本或者是固定资产已然变了,现代意义的特别是在新型工业化之后,这个不变资本的含义里边包含了更多的无形资本,主要是围绕着专利等知识产权形成的新型的不变资本。

而可变资本的部分,由于可变资本属于流动的资本或者是流动的资产,那么这里边不仅仅包含了企业对自己工业工厂或者是公司,或者是集团内部的剩余价值的剥夺,也开始加入了对社会上其他劳动者或者是商品购买者的剩余价值的剥夺。它是以一种类似于价内税的方式,价内剩余价值的方式进行剥夺,从而再将这部分剩余价值转移到不变资本,它构成了新时代不变资本和可变资本的结构。马克思讲不变资本与可变资本讲的是个结构问题——注意是结构问题,千万别搞错。

马克思在讨论资本积累里边的第二个框架讲的是固定资本与流动资本。好多人说既然谈了不变资本、可变资本,还有必要再搞出固定资本与流动资本吗?请注意,固定资本和流动资本讨论的是资本积累的速度,而不是结构。注意是速度而不是结构,你搞错了以后,在这里边会出现问题。固定资本的积累的速度是非常重要的,固定资本在一定程度上是需要进行大规模扩张的,在它处于这种膨胀期和发展期是需要大规模扩张的,而固定资本的扩张有的时候不止于固定资本本身,这就出现了我们讨论的杠杆的问题。

固定资本的扩张,有的时候需要一些不仅仅是资本家或者是股东的自有的资本的原始积累,它有的时候需要多元化的资本的源泉进行资本积累。但是它的前提条件是你是在某种垄断性的行业。这个垄断有两种可能性:一种是国家特许经营;一种是由于高科技新研发出来的新的商品导致你形成了某种情况。在这个时候固定资本扩张它就需要一些其他的方式方法,现代的固定资本扩张和传统的资本积累有着巨大的区别。

我国在理解高科技企业的时候,由于对《资本论》可能理解得不透,所以我国对高科技企业的固定资本或者是原始积累,我们没有像美国那样给予打开方便之门。就是我国对高科技企业就是所谓创业板或者是这种初创企业,这种风投就是固定资本的原始积累没有提供必要的条件,以至于中国在这个领域里边的头部企业都跑到海外去做了融资,这既有他们的问题,也有我们治理的问题。这个将来在合适的时候,我们花一个比较长的时间来讨论这个部分。

至于说流动资本的部分,由于处在一个我国的金融——金融的制度、金融的结构或者是金融的机构与新时代不匹配,就是我国是一个非常落后的金融体制、金融机制或者是金融结构。所以我们对特定的企业,比如说对高科技企业,不光是固定资本的提供上出现问题,对流动资本的提供也出现了问题,就是我们对中小企业这个贷款难。由于速度的问题,固定资本和流动资本在特定时间的速度上的问题,导致我国在这个领域里边出现了卡壳的问题,就是我们在治理上可能出现了一些问题。

虽然我国存在这样或者是那样的问题,但是由于我国在整体发展之中处在一个不断地适应和改造的过程中,所以我国在新型工业化的过程中还是走得比较快的。虽然我们确实存在一些问题,比如我们对固定资本的理解,固定资本是什么?固定资本,如果你要是跟银行申请,说我是要买厂房、机器,那么银行可能会给你,如果说是购买专利、或者是自己研发或者是知识产权或者是构造生态系统,那么我们的金融机构就无法理解,当然这里边有个定价的问题了,就不太容易理解类似于像一些这个高科技企业的一些做法。

由于我国的高科技企业与外部的金融机构形成了某种的结合,用贬义的话就是媾和。类似于像孙正义软银这样的机构,我国改革开放之后,金融开放是幅度非常大的,所以他们大规模介入了我国的科技企业的头部的企业的投融资,所以我国在这个领域里边资本稀缺的问题至今还非常严重,但似乎也不是完全没有解决方案,它还是有一定的解决方案,只不过可能它的成本和代价偏高了一些,我说的这成本和代价不是融资成本,而是股权外溢,就是大股东的结构可能出现问题。

理解我国科技企业,甚至理解可能不是那么科技的企业,比如说联想,他们为什么会美国化?为什么会出现一系列的状况?这里与我国的他们自身的这个认识有关,也与我国的整体的一个制度设计和政策安排有它的一些必然的关系。就是我们对一件事物、一个企业、一个机构、一些人的判断要放在一个大的历史背景中去来理解,这样的话才不会偏颇、不会情绪化。当我们在一个大的框架里边来分析和理解的时候,可能事情会变得更为客观、更为切合实际。

马克思的扩大再生产的整个一套理论是建基于机械化的。就是第一,这个传统工业化情况下的扩大再生产,所以他对新型工业化的这个扩大再生产这个《资本论》就无法提供更详细的描述。这件事情不怨马克思,是我们的事情。我们如何理解不变资本的积累?如何理解固定资本的积累?如何理解不变资本与可变资本的关系?如何理解固定资本与流动资本的关系?如何理解固定资产与流动资产的关系?其实这个关系就是扩大再生产的核心的逻辑基础。

关于扩大再生产呢,今天也不放开讲,这里边我想提供一个概念,就是资本积累率。资本积累率其实我们说了很久了,这个资本积累率如何计算,实际上是将股东权益转回为股本,你可以将之理解为将可变资本中剩余价值的积累的部分重返不变资本,增加不变资本的总量,也可以将它理解为继续增加固定资本的投入,资本积累率。资本积累率实际上是决定一个企业或者一个国家长远发展的动能,一旦资本积累率没了,那么不管是旧的工业化还是新的工业化就都没了,香港就是这样的。

这里边我们可能有些东西要对马克思的这个章节,就是马克思的这个第二卷第三篇第二十一章做一点补充,就是科技投入属于不变资本,知识产权属于固定资本,这个我们要有个明确。明确了这个是非常有利的,因为我们鼓励科技投入,就是我们鼓励科技企业增加科技投入这个不变资本,鼓励这个知识产权——固定资本知识产权的增加,这涉及到会计政策以及国家治理的一个基本的逻辑。不要小看会计政策和国家治理,这涉及到哲学的高度,甚至涉及到一个国家、一个社会的集体审美,说得简易一点,涉及到国家治理者的智商水平。

资本积累的一般逻辑,它在处理逻辑问题上,我们有时候需要进行反向思考,就是如果你是鼓励知识产权的,你是鼓励科技投入的,那么政府在税收方面、可能在一系列的政策方面要有某种的倾斜,就是它涉及到国家在制度设计和政策设计上面的一些的倾斜,另外它也涉及到社会的一种整体的这种审美和气氛。就是要知道我国在八十年代的时候,所有人都想这个学理工的,“学好数理化,走遍天下都不怕”,理想是当工程师。

然而进入到21世纪之后,变了。我国的所有的……我在B站那个演讲,在岭南大学那个南南论坛上演讲,后来B站发了以后反响极大。后来这个新浪就因为那个B站那个节目就封了我的微博,后来头条也封了我的账号,都是180天。因为我那里边说了段话,惹毛了大家,就是领导同志的孩子,他就什么都不想干,他就去了金融机构,基本上几乎是百分之百都混金融机构去了。那么为什么大家觉得做科技不光荣呢?做金融光荣呢?这是一个集体审美的问题。

今天你要是去吃饭,饭桌上聊起来你孩子在做什么呢?你要说做金融——高大上,你要说做工程师就有点问题了,你要是说做公务员好像还凑合吧,就是大家觉得你有一个稳定的饭碗,你要做工程师让大家就觉得那怎么能去做工程师呢?就是整个的一个社会审美结构是出了严重问题的,因为不认为工程师是可以更能创造价值,或者是由于国家在制度设计上出了问题,所以金融机构的工资要高得多,而工程师再牛的工程师能高过某大行的CEO吗?它是个问题。

这个问题在北欧、西欧的一部分国家,比如说瑞士、德国,处理的也是比较好的,就是劳动者的收入差异呢,它实际上是跟你创造价值是有关联的,美国处理得不好,英国处理得不好,中国处理的是最不好的,就是你投胎或者是你找关系进入到一个金融机构,并且熬到一个中级管理层,大概年薪就到了一个相当的水平,几十万是一定得有了。现在光城商行做一个行长的大概都是数百万年薪,甚至上千万年薪,当然你要在大的银行做的话,那千万年薪可能是个起步价。

他们的贡献真的比倪光南院士、比这些科学家的贡献更大嘛?那显然是没有的。为什么会出现这个情况呢?那就是我国的整体的定价逻辑——劳动力的定价逻辑出现了严重问题。定价逻辑是由谁管的呢?定价逻辑是由谁管的呢?我写过一篇文章,那篇文章是这样说的:吏部出道德、出审美,也就是组织人事部门来定工资的时候,这个地方出道德;户部——财政部,出公平,就是税收,它导致一个社会的公平,而工资,其实组织人事任命里边就包含了工资,它出道德,道德就是集体审美,问题就出在这儿了。

不辖两部无以治天下 https://www.notion.so/05fb6953865245b49b29b2b592d3c1fc

这件事情,我想可能到了二十大会出现一点点的改善,可能是开始改善,就是什么呢?共同富裕。共同富裕这里边对劳动者的定价就会出现一个深刻的调整,当然这个深刻的调整是两重调整:第一重,是定价本身就要调整,你之所以在“信达”或者是“华融”,像赖小民,你坐一个这样的位置,不一定完全是因为你的能力和你创造价值的能力,而是因为某种人事结构、人事关系导致你在这个位置上,而你在这个位置上获取了超越正常劳动者数十倍、数百倍的收益。

也就是说,我们在一次分配里边,现在我国的劳动力的定价体系必须进行深刻地改变,不然的话它对新型工业化是一个悖论,就是没法进行新型工业化,因为我们最需要照顾的是那些非常有才华的科学家、工程师、技术工人,而不是由于某种人际关系而在某些金融机构或者是类金融机构类(什么叫类金融机构呢?比如联想就是类金融机构,比如说阿里就是类金融机构,它不是我们理解的那样的机构。),在处理这个问题上面呢,我们不可以继续深化我们固有的问题,就是劳动力定价上的问题。

这不是我说的,是马克思说的,所以这回说错了,不能再关我180天,要关也得关马克思180天,好不好?第二个部分涉及到劳动力定价的二次调节,就是税收,税收的调节主要是针对通过直接税进行调节。这个事情可能二十大也会有新的东西出来,但我们是希望尽快开始,因为中国在这个第二次工业化或者是新型工业化的过程中呢,出现了一系列的问题和麻烦,如果这个问题和麻烦如果究其根本,可能就在这个地方。所以最近关于华为和联想的讨论变得极有价值,因为华为它对它的科学家和工程师和技工的照顾,达到了一个比较合适的这个状况。

而我国目前对处在这种非常尴尬的——例如金融机构,金融机构就算有些是民营金融机构,它也是由金融工委来派干部的,有些国有机构是由国资委派干部的,这个派来的干部,他就是……,你说他的主要的依托是某种人际关系中的一种特殊位置,他就是……,而这个东西导致一个定价上出现的这种巨大的差异和问题,其实是需要调整、是需要思考的了,因为它会导致不变资本向可变资本的转移出现严重的障碍。

好,这个部分不展开,我想呢,今天我讲的内容可能涉及到的量比较大,我也不想那么细,你们有空可以自己做一点点阅读,懒得阅读呢,听一下或者听两遍大概也就清楚了。我们第三个部分想讲一讲新型工业化中的积累问题,今天第一个问题讲工业化和这个新型工业化和去工业化;第二个部分讲一下是马克思关于资本积累的定义;第三个是新型工业化中资本积累的含义。我为什么要讲新型的工业化的资本积累呢?因为新型工业化和这个传统工业化的资本积累出现了巨大的差异。

新型工业化的资本积累,有没有物理意义上的传统工业化的资本积累?有的,一定有的,因为它需要土地、需要厂房、需要基础设施,作为国家在这个物理意义上的积累就是“铁公基”。我们这个企业的积累,这些基本的要素,你就算是高科技企业,你也需要一层写字楼的,它有物理意义上的积累,但这个积累在固定资本的构成里边越来越小,而它在固定资本的构成里边主要的是知识产权和生态系统构造。知识产权和生态系统的构造呢,它就不完全是物理意义的啦,因为它更多的是一种无形资产。

前段时间我们在一个晚餐上面对美国的大公司,主要是对微软、苹果、谷歌、Facebook等一系列公司他们的定价进行了讨论,定价是否合理,因为这些公司都已经是轻资产了,但是它不是轻固定资本,它固定资本(请相信马克思真的很了不起)庞大到你用现有的会计方法无法理解,所以你就无法理解它的定价,因为你很难理解苹果构造的那个生态的价值。现在你知道华为在做什么?它构造的是一个生态,这个生态它的价值到底是多少?无法估量。

而联想呢?OEM。系统集成的不是生态,而是一部电脑、一个机器,它仍然属于传统工业化范畴,它没有进入到新型工业化。这说来呢,有些人看上去很高大上,但是他搞的是一个家族企业,他构造的是传统工业化里边的一种东西,而传统工业化之所以能做大、做强的原因,是由于借助了某种方式或者是某种权力或者某种关系,形成了一种垄断,当然也不能说联想一点科技都没有,有的,至少收购IBM,它还有一个品牌含量、品牌价值在,但是生态这个方面让人感到失望——它没有构造自己的生态。

新型工业化除了第一层物理意义上,第二层生态构造,第三层就是基础科学的研究。构造生态的基础就是基础科学。你没有基础科学,你拿什么构造生态啊?因为它是一个大的、完整的系统,机器在这里边只是一个部件。它真正的价值是生态体系本身,而不是那里边的一部手机、一部电脑、一部汽车,是一个生态。几乎美国上万亿的企业都构造了自己独特的生态,而我国现在可以构造生态的企业凤毛麟角,可能是因为……

我们在基础科学的研究方面还存在着巨大的差距,或者是我们对基础科学向实际运用方面这个转化过程中还存在着一些壁垒和障碍。第三层是基础科学,那么第四层就是哲学和政治经济学。新型工业化积累的含义一共包括四层,重复一遍: 第一层,是物理意义的传统工业化部分,因为你必须有个地儿嘛;第二层是生态;第三层是基础科学;第四层是哲学和政治经济学。其实老任在哲学和政治经济学的高度上面是老柳所不能比及的。

这和你可能的学历、经历没有什么关系,因为哲学高度有的时候真不是读书能读出来哲学高度的。老任是处在一个特殊的时期,他有特殊的经历,他可能是读过一些,他是可能,他的哲学还更多的是从《毛选》上获得的。不管从哪儿获得的,但他的哲学高度在中国的企业家里边确实是难有出其右者。他对政治经济学的理解也非常到位,这是非常难得的。马云、马化腾、马明哲“三马”,他们在某种意义上也拥有了一定的哲学高度,有一定的政治经济学水平,但还需要再努力一下。

我们回头来看美国,不管是比尔⦁盖茨,不管是已经逝去的这个苹果的原来的创立者,也不管是扎克伯格、也不管是马斯克,他们都拥有我们感到惊讶的哲学思维和哲学高度,而且对政治经济学的理解也是非常到位的。其实在某种意义上,这个才是新型工业化最要紧的东西,或者最要命的东西。其实老柳知识结构或者是个人的能力都不差。至于说人品,我们不对企业家人品做出过多评判,但就是这儿上不去,怎么都上不去。

在新型工业化的过程中,这四层,其中核心的部分我们注意到了,就是固定资本已经完全异化了,传统工业和新型工业在固定资本的内涵上全部异化了。以前是个产品,现在是个生态。创造生态的企业才能叫头部企业,才能叫独角兽,才能够形成全市场、全方位的垄断优势。而构造生态是一个极为漫长、复杂的过程中,它不像卖产品那样容易获得发展。好多企业,最后卖产品,卖着卖着就转金融投机去了。

因为创造生态的基础是科学技术进步,创造生态的基础是科学技术进步,它需要哲学的高度,需要对政治经济学深刻的把握;最主要的是人性,你在这个过程中要有极为强大的那种韧劲、耐心,要一点一点地将生态体系攒出来,这个太难了。所以单从我们对《资本论》这一章的理解,我们不得不向华为这样的企业和企业家表达我们的敬意。因为我们能构造生态的企业还是太少了,在新型工业化过程中,我们需要的是这样的企业家和企业。

好,我最后综合起来,不管是传统产业还是新兴产业,我们对资本积累的要素做一个概述。资本积累大概有哪些要素?第一个要素是秩序。这个秩序的构造当然不是个人和企业,是国家来做的,就是你要有好的制度环境、好的政策,制度环境里包括了一个集体审美。其实我们不仅仅需要劳动保护法,我们更需要资本保护法。我们对资本的积累,特别是资本由可变资本向不变资本转移的过程中,要给予充分的制度保护和政策保障。

资本积累的第二个要素就是生态。构造一个完整的生态体系, 这个生态分三个层级:第一个层级是国家的产业的生态,就是产业链。各种产业的配套的体系,完整的生态,在这一点上我国做得不错的;第二个层级是企业构造生态,就像华为构造的生态,类似于像苹果的生态、像谷歌的生态。能构造生态真的是很了不起,如果看不到生态,而集中于理解某一项科技或者某一项产品是不行的。如果你看我国的航天,你就明白我国的航天是一个完整的生态体系;你看我国的航空,完整的生态体系;你看我国的船舶,完整的生态体系。

第三个部分,我说的资本积累要素是地租。对地租的理解我们讲过了,就是如何控制一个国家提供给企业作为不变资本或者是作为固定资本的地租,怎样的水平是合适呢?一个国家的制度的制定者和政策的制定者,你必须理解地租,必须理解地租合理的区间。不是低了就一定好,也不是高了就一定不好,区间在哪里?边际在哪里? 如果你不懂这个,那么你就变成了今天深圳的治理者。深圳的治理者完全不理解地租,所以深圳在迅速地去工业化,而不是升级。

我们看到深圳的这个状况让人感到非常痛苦,在重复香港的旧故事。在要素里边,秩序这个要素深圳可能略好一些。能好多少?我不认为它一定比杭州更好,但我想它可能比东北好一点。在构造生态这个领域里边完全不自觉,它不知道什么叫生态。有空有关方面请深圳市委书记和市长谈一谈深圳在构造生态方面的理想,我想,他们还没有这样的哲学高度,无法讨论生态。第三个是地租。地租的构成,地租将构成企业不变资本或者是固定资本的核心要件,那么你给企业这个东西,合适了它才能留下。

为什么你会放纵资本炒高地租,以至于导致一个地区去工业化?为什么会放纵资本?为什么?为什么?为什么?是政治治理、经济治理上面出现了严重的思考上的障碍,这是一个哲学高度。思考上出了问题了,所以对政治经济学的理解就会偏差。地租能不能便宜地提供给制造业呢?我们看看瑞士的例子、看看新加坡的例子就懂,这是完全可以控制的,但这个需要治理者极高的哲学高度和对政治经济学的非常深刻的理解。当然大部分人是不会读《资本论》这样的东西的,不会去体谅这个不变资本。

那么以深圳为例,它如何使新型工业化的地租构成处于合理区间呢?那么就需要通过税收——直接税的调节,将吃租的人、吃超级地租的人,将他的租子拿回来。吃租的人有可能是政府——卖地,也有可能是二手的房东,或者是那些拿到廉价土地炒地皮的家伙们。如果这两件事情你办不到,你说你还想搞新型工业化?我是个非常文雅的人,我看到深圳的衰落,我说不出话来。湾区在做什么呢?我们讲地租讲了这么长时间。

当然!很遗憾,深圳市委书记和市长不在这个课堂上面,所以他这个错误还改不了,还继续炒高地租。房价好像掉下来了,但是工业用的不变资本或者是固定资本的部分仍然在不断地攀升。一个月前,我见香港的,我参加特首的一个活动,我就提出来,他们提再工业化,我就提出来知道再工业化的前提条件吗?或者是你知道资本积累的要素吗?如果你不知道,你就起个北部都会区的好名字,行吗?

香港搞数码港的教训不深刻吗?数码港在干什么?在炒地皮呀!是个房地产项目!它根本就没理解,它把地租炒到天上,不是要搞现代工业化、搞新型工业化,它是去现代工业化或者去新型工业化,拦都拦不住啊,所以我们第三个要素是地租。第四个要素是剩余价值。剩余价值重不重要?当然重要。你在这个地方它必须有大批的、优秀的、又不是很贵的工程师和技术工人。工程师和技术工人为什么不那么的贵?和你综合社会成本有关,对吗?

如果你这个也不懂,地租和剩余价值全都没了。这个要素,第一个要素——秩序,东北已经没了;第二个生态,没有生态;第三个地租很贵;第四个剩余价值,很高。不是地租、剩余价值没有啊,就是你太贵了,所以剩余价值就是企业构成不变资本的那个东西没有啊,不变资本无法增值,不变资本无法增值转化为,不是,可变资本无法增值转化为不变资本,那个企业怎么发展呢?所以,如果这四项都没有,我们大体上可以宣告一个城市、一个地区进入了去工业化。没用了,离开它。当然我们不希望深圳这条路再往前走下去,太悲惨了。

还有三个要素,就是其他要素成本,包括水、电、气、物料等这个其他要素的成本。这个不是最重要的,最重要前四个。第五个是其他要素。第六个是配套能力,就是你这个地方有没有可扩展或者可扩容的空间。如果有巨大的空间的话,就可以形成更广域的体系和生态。第七个才是交通运输。为什么说后五、六、七不重要呢?就是好多地方,比如以色列,他没有什么其他要素,但他发展得很好;比如瑞士,他也是内陆国家,要资源没资源、要配套没配套、要交通没交通,但他发展得很好;比如说台湾。

最近,有一个台湾的朋友将深圳和台湾做了比较:深圳前40年发展这么好,现在不行了;台湾前40年发展得一般,现在很好了。后来他跟我说,他说我们实际上看不明白深圳发生什么事情,怎么就哗啦哗啦就不行了,而台湾现在涨得很快。我说你如果读《资本论》的话,懂得资本积累和资本流转,大体上这个事情就可以看明白,并且能够说清楚。顺便说一句,就是我在《广义财政论》里边会建立一个资本积累的评价体系,对一个地方、对一个企业的资本积累作一个评价。这是一个非道德的评价体系,就是我们不对人好不好、坏不坏不作评价,只是看他做得有没有哲学高度,对整个结构的理解是否到位呀!

本来我还有两个题目。第五个题目是“国家资本积累与国家资本主义”,就是国家资本主义好不好,国家资本主义对资本积累是正面的还是负面的。我觉得这个课题其实非常重要,因为它需要构成一个什么呢,就是国家资本积累在传统工业化期间它的意义是非常重大的,因为它能集中所有的就是将可变资本变得放大,甚至是很大。比如说国家可以将地租压缩到零,将剩余价值剥削到很大,然后转入不变资本形成高资本积累、资本积累率,迅速地完成工业化进程。从这个意义上,国家资本主义是好的。

但是,国家资本主义因为有“国家”两字,所以就有了吏部和户部的问题。吏部和户部有的时候会处理不好劳动力定价,会处理不好这一系列的矛盾,所以国家资本主义在进入新型工业化的时候暴露出严重的问题,走不动了。这个时候,需要社会资本主义对国家资本主义进行某种的补充,社会资本主义通过市场、通过各种方法来调动全社会的创造力,特别是构造生态这样的一个问题。那么,社会资本和国家资本主义如何相互和谐处理好它们的结构而达至最佳呢?

其实我国的经济学家要研究的就是这个问题。我国的经济学家不用高举个旗帜,像张维迎这样的就是“国家资本主义要不得,社会主义必须死,必须依靠市场”,你那哪是经济学家!经济学家讲究的是边际,没有对错;讲究的是结构均衡,讲究的是边际。你结构不均衡、没有边际,你死乞白咧地吴市场、厉股份,张维迎的“冰棍”,张冰棍,你这个玩法,你这个北大、清华,人家不能这样折腾嘛!你这样折腾在哲学上就是完全是无法成立的,政治经济学上幼稚嘛。所以这个课题就不说了。第六个问题就是“新社会主义的资本积累问题”,原来我是有些想法的,反正将来我会写在《广义财政论》或者《新社会主义论(2014修订版)》里边,将来大家看去吧。

做一点点的小结吧,今天这堂课其实挺——我不敢说今天——我觉得这堂课重要。马克思很厉害,马克思他就是、他就是哲学家,他一弄,他就立刻进行结构性设计。所以不变资本、可变资本、固定资本、流动资本,结构性变动每个资本的结构里边又有不同的结构。整个的结构的变动和异化的过程中,我们就看到了资本的流转、资本的积累,哪一种方法更有效率、更对,就他给你一个分析框架,你套到今天的新型工业化来看就行了。大体上读到这个程度,《资本论》算是就是能用了、读活了。好多人读《资本论》说一点意思都没有。谁说的?你好好读啊!

好,这件事就今天就一个小时可以啦,我们就讲这么多。剩下点时间我想讲讲南非这个病毒可能对经济的影响。其实我们封建迷信年初的时候就讲了这个辛丑年这六爻最后一爻它就是这个样子了。我们一直在警惕警惕,一直是持盈保泰,一直是躲进黄金屋,其实说的就是到最后这儿了。这个病毒会产生什么样的影响呢?大家已经看到了,油价在掉、各个国家的股市在掉、汇率在掉,股汇债三杀。发生什么事情?为什么?如果我们从今天的课程来讲,大家就懂了,是……

资本的流转速度突然又停下来了,它跟上一轮的疫情是一样的。就是由于疫情造成的恐慌,导致人、财、物的流转速度再次降速或者是失速。用费雪定律来定义,就是MV=PQ里边的那个V失速了,又开始降。M不变,甚至有可能要缩表,那么M还会减少,M不变,V收缩,那么你想想P、Q呢?我们把P、Q分解为P1Q1+P2Q2+P3Q3+P4Q4,P1就是商品,P2就是资本市场,P3就是楼,P4就是现金、等价物、其他东西。

请问V掉下来了,如果再加上美联储或者其他的央行去掉QE,M也降,V也降,那么请问,假设、假设Q不变的情况下,哪个P降?P1、P2、P3、P4,哪个P降?显然、显然P2资本市场、P3楼它都必须得降。那么从P2、P3里边逃离出来的资本会去哪里呢?那么只能是P1和P4。P1就是商品,因为商品需求是刚性的,你越不能提供它越需要,所以通货膨胀还得上。

为什么会去P4呢?P4里边有现金,有金、银、贵金属,还有元宇宙,还有数字货币等等。它为什么会去这儿呢?简单得很,因为如果存在货币贬值的可能性,也就是如果通货膨胀不可控,那么必须得考虑资本的保值问题。不是增值是保值,因为增值肯定不能进这个领域,无论是现金,现金的利息那么低;也无论是金,金没有利息;也无论是数字货币,它不能制造实体的财富,但是它可以保值。现在要思考的问题其实也挺简单的,并不复杂。

我们通常分析,可能好多人觉得我执着、执拗,就是不食人间烟火,从来不看别人怎么想、怎么说。因为原因是通常我分析问题是有我的分析框架,我只是获取信息、摆进分析框架得出一个基本的结论。我有的时候还挺自信的,我认为我的结论大体上是对的,只是时间节点和幅度能到什么程度。有的时候会出现偏差,但结构在这儿。所以我不是听谁说什么,或者是听到什么,我听不到。而且好多人的分析我看完以后我个人认为很可笑的嘛。

现在要考虑的就是总量,到底V这个速度降下来之后,相当于MV这个总量会缩减到一个什么程度?因为市场上的资本一旦开始紧缩,会有一个回流效应。就是全球资本一旦V掉速,掉速之后大量的美元回归美国本土,它就导致美元强势。它道理是非常简单的,这个回归的美元会进到哪里?P1当然是最主要的,是因为要吃饭、要穿衣、要活嘛,所以P1是最主要的部分。那么有多少会进入到P4?所以我们要想的是一个量的概念。

其次我们要考虑一个结构,不!我刚才说错,要考虑第一个是量的概念,第二个是时间的概念,就是什么时间达到这样一个水平。好多人认为美元的升值周期开始了,我不这样认为,因为骤然的紧缩会导致美元回归美国本土。骤然的收缩使得美元与他国货币的比价关系出现某种新的现象,但这并不决定美元自身的价值。为了重视,我重复一遍。由于骤然的通货紧缩,就是V降速了,市场流通的货币、流通的资本没那么多。

在骤然的通货紧缩的状况下,囤积在他国的美元会迅速回流美国本土,导致美元出现强势。这个强势不是这个星期出现的,已经持续了一个月了,美元指数从底部已经持续了一个多月的时间。实际上这个情况的产生包括油价的调整,它不是今天,它今天是发生了,它已经我说了11.11就是光棍节那天它就已经爆发了这个南非的变种病毒,就是大家不注意。但它是否意味着美元的价值出现增值呢?是美元的价格上扬,但不代表美元的价值出现增值。原理在哪儿呢?

原理在美元的真实的价值取决于美元的实质负利率。现在的通货膨胀6.2,现在的美债的孳息率减去这个通胀的水平应在5以上的水平,但我们估计这个通胀水平可能还不完整,甚至还要、就是6.2还要上。而国债的孳息率或者收益率正在迅速地收窄,也就是说这个负值,实质负利率的这个值会迅速地扩大,它将意味着美元对贵金属的贬值不可避免,

好像我确实有点执着了,好吧,我只是将我的观点,说给大家听。所以我不太建议大家,我已经说了很久了,不太建议大家,高手可以在资本市场和楼市、股市和楼市里边晃悠,我不太建议你们在那个地方晃悠,你非要晃悠,又是顶级高手,那就晃悠也没关系,但我真的是觉得不建议,这是今天要讲的第一个部分。第二个部分,2022年,2022年我上堂课聊天儿的时候已经说了,就是明年会进入衰退,中美两国都进入衰退,如果中美不能联手采取断然措施遏制这个衰退,可能进入大萧条。今天我还是坚持这个判断。

我们这一代人,我经历过衰退和大萧条,我在香港经历过,所以我知道那个时候资产价格是个什么样子,你要想赚钱应该是个什么样子。大家可能没有经历过,但我相信,你到了这个平台上,不管是聊天也好、听课也好,听到今天大概也应该明白一些了,我们给了一些分析框架、给了一些结构,应该懂的。我们说了,2022年二十大之后,中国有可能进入一个十年的牛市,但进入牛市之前的这个调整过程中,就变得非常的有意思了,其实是在做基本的准备的工作。其实准备的工作是非常辛苦的……

一个人能赚钱在于预见性,而不在于就是你在,就是刘伯承说的“吃着一个、夹着一个、看着一个”。吃着的这一个,我们现在短股长金,吃着一个;夹的那一个,大概是重点是新能源,还有就是一些信息技术,其实是新能源和信息技术的结合部,可能是最为核心的部分,夹着一个;看着一个就是未来的东西。因为你转身的原因,不是因为你觉得这个价位到了,就是我摸了个顶,我最讨厌别人摸顶,因为一个人一旦要是抄底和摸顶,他的精神就分裂了,精神分裂的人就赚不到钱。

什么人是不会去摸顶呢?是因为,他发现他要夹着的那个进入了理想的地步了,所以吃着这个到顶没到顶就不重要了,他把这口咽下去,要到底部去拿那个夹着的那个,所以他转身了。顺利完成转身,不再、不追求顶的原因,是因为他眼睛看的那个东西在底。这是个非常重要的哲学逻辑。好多人为什么老是要摸顶?因为他不知道夹的那个东西,他说我卖了以后我去哪儿呢?当你知道你去哪的时候,你就卖掉了,你就转身了嘛。所以,前瞻性非常重要,就是“吃一个、夹一个、看一个”,这个非常重要,在军事上如此,在投资上也是如此。如果,你要是夹的那个东西你没搞清楚是什么的话……

夹得那个东西你没搞清楚是什么的话,我今天也不能告诉你,我今天要告诉你,又惹来一大堆的麻烦。因为早,时间还早,适当的时候我们会以适当的方式告诉你,夹得那个东西是什么,另外看的那个东西是什么,这个时候,我们在合适的时候就完成我们的转身。其实赚钱的道理不在于你赚多少,而在于你不赔、或者是你少赔,只要你延续不断的进行这种平稳的发展和积累,其实那个效果是非常惊人的。好吧,今天就讲这么多,技术上的事以后再聊吧。明天下午三点钟,我们接着聊。好,再见。

\section{现代殖民理论与后殖民主义、中央经济工作会议、一收一放}

大家好,我试一下麦。今天是2021年12月11号,农历辛丑年十一月初八,今天是《资本论》第十八讲——现代殖民理论与后殖民主义。如有时间呢,我们聊几句中央经济工作会议和“一个降准,一个升准”,聊几句经济。好,一会儿见。

大家好,今天是2021年的12月11号,农历辛丑年十一月初八。转眼就已经到年底了,我们的课程进展进到第十八讲。今天是《资本论》第十八讲,是现代殖民理论与后殖民主义。这堂课非常重要,因为这是马克思《资本论》的第一卷的第二十五章,也是《资本论》第一卷的最后一章。马克思《资本论》写作的过程中都是比较平静的,唯一写到这一章的时候,马克思使用的语言开始文学化,并且非常激动。我在思考这一章的时候也下了一点点的功夫。

备这堂课,因整个这一周都是一个会议周,忙得要死,所以断断续续在备课。直到今天呢,我还没有把这个课备完,那也时间到,我们就继续讲吧。因为这一章算是在《资本论》里边的一个非常重要,它不长,但这个章非常重要。因为在马克思的研究过程中,《资本论》以外的研究里边大量地涉及到了殖民地的问题,而殖民地的问题也是我非常关心的问题,因为我来香港以后一直受殖民地这个问题的困扰,所以我对这一章也比较重视。

在开讲这一章的时候呢,我们做一点前期的铺垫。其实也很多人在问我:“就是你说说《国富论》、《资本论》和《通论》这三部顶级的、具有传世价值的经济学著作,它们的区别是什么?”今天我把这个区别讲一下子。《国富论》是研究资本利得的,《国富论》整体的框架结构是在讨论资本的利得,当然它讨论的不是个别资本利得,是以国家为单位的国家总资本的利得,这非常重要,所以它叫《国富论》。实际上《国富论》算是在宏观经济层面最伟大的著作。

到今天我们在讨论中国问题的时候,依旧需要认真地来研究国家总资本的资本利得。前两天有人在网上对我进行了一些批评或者是攻击之类的,就“讨论资本积累”。因为我讨论的不是一个企业的资本积累,我们讨论的是一个国家总的资本积累率。国家总的资本积累率出现下降是因为总的资本、社会资本里边出现了大规模的迁移,不是迁入,而是迁出的时候资本积累率就会下降。总资本利得是《国富论》要讨论的核心问题。那么如何增加总的资本利得,使国家迅速完成资本积累和完成工业化?这是《国富论》要研究的问题。

而马克思的《资本论》它研究的是劳动所得,它希望将劳动者的劳动所得最大化或者是合理化,在合理化的基础上能够比较好地来维持劳动者的本身的再生产。《资本论》主要是研究劳动者的劳动所得的,是以这个为主题展开的。如果说《国富论》是一个国家进入工业化、完成工业化的一个重要的一个逻辑基础,或者是一个理论依据的话,那么《资本论》则是一个国家重视民众、重视社会、重视均衡发展的

一个重要的逻辑基础或者是理论依据。它也是一个国家从君王或者是从君王专制、资本专制走向社会共治的这样一个重要的经济学理论。《资本论》的价值其实直到今天,大家的意识或者认识上面仍然是不彻底的,它是一本关于解放的著作,关于解放的书。曾经社会主义国家将《资本论》作为无产阶级的“圣经”来面对它、来读它,它是有它一定道理的,因为它一直在研究劳动所得的合理性以及它的最大化的问题。《资本论》的所有东西围绕着这个地方展开。

那么《通论》呢?《通论》是研究两“得”(一个是资本利得,一个是劳动所得)获得均衡的一种方法。它主张在政府干预下,通过某种政府的干预,使得资本利得与劳动所得获取到某一种平衡。我一直认为凯恩斯是一个伟大的社会主义者,而这个伟大的社会主义者是英国人,他既熟悉《国富论》,他也熟悉《资本论》。所以《通论》的立论基础并非单纯地希望国家资本利得最大化,而是希望国家资本利得与劳动者所得达至某种均衡。如果你这样地来理解这三论,并且去阅读它,那么效果就不同了。

如果你非要让我概括一下这三个理论的是什么的话,我可以这样说:《国富论》是一本经济学,当然是古典经济学了,是一本重要的经济学著作;而《资本论》其实是政治学著作,或者我们管它叫政治经济学著作;而《通论》是经济政治学。好,我重复一遍:《国富论》是经济学;《资本论》是政治经济学;而《通论》是经济政治学。研究经济要考虑用政府的方法来处理它,它们为什么这么重要?因为它们都是在这个领域里边的开山之作,达到了极高的高度,而且在相当长的时间,很多人……

很多人认为自己怎样怎样,那么多获诺贝尔奖,其实无论是在哲学的高度,还是在经济学本身,还是在数学层面能达到他们这种水准的人少之又少。如果达到如此开山的程度,那真的是非常了不起的。所以这三部著作非常之重要。这三个人有一个共同的特点,就是悲悯。因为即便是亚当•斯密,他最后写了《道德情操论》,他不光是《国富论》,写了《道德情操论》。其实他们的悲悯,对人民的这种悲悯是可以感同身受的。

在讨论殖民、现代殖民理论的时候讨论这个问题,其实殖民理论——马克思的现代殖民理论里边涉及到超越了劳动异化和资本异化的更重要的话题,因为殖民的整个过程其实依旧是资本剥削和压榨的域外延伸或者是境外延伸,只不过是它以更残暴、更粗鲁、更野蛮的方式进行。所以我们今天讲这个,其实是要回到当下的现实,好多人认为中国早就结束了殖民地的统治,我们原来是半封建半殖民地,半封建半殖民地这个事情已经被毛泽东解决了。

然而,历史怎么会那么轻易地完结呢?如果你看到了香港黑衣风暴,如果你看到了台湾倒行逆施,如果你看到了日本平成战败,你说殖民的这个过程结束了,但殖民主义结束了吗?后殖民主义结束了吗?后殖民主义在中国结束了吗?难道莫言的出现不代表后殖民主义的文化现象吗?莫言、方方不代表后殖民主义的文化现象吗?我们在经济上的一系列问题难道不是后殖民主义的滥觞吗?

所以,读马克思《资本论》第一卷的第二十五章——现代殖民理论,是重要的,今天这堂课也是重要的。我只是担心我对这个事情的把握还不到足够的高度,或者是处理得不好,没有能够讲出马克思的本意来。其实马克思将“现代殖民理论”这一章放在《资本论》第一卷,它是《资本论》第一卷第七篇的最后一章,也是整个《资本论》第一卷的最后一章。你知道马克思《资本论》的第七篇是资本的积累过程,他讨论资本积累,资本的积累过程里边殖民这一块儿是一个重要的源泉,所以他把它放在这一章。

另外,马克思的我说的是第二十五章,如果你看第二十四章,他谈的是“所谓原始积累”。其实马克思对资本的原始积累是表达了他自己的内心的一种愤怒,和一种对成为原始积累的殉葬品的那些人、那些国家的一种悲悯、同情。因为原始积累的过程中,所谓的殖民——在古典殖民主义里边所谓的殖民,是原住民的几乎全部的种族灭绝。它是非常非常残忍的过程

美国人在说我们的新疆是种族灭绝的时候,滑稽。因为真正的、真正的这个古典殖民主义,因为我们今天讨论的马克思这一章的标题是“现代殖民理论”,这当然不是今天的现代,是马克思时代的现代。那么马克思时代的古典殖民是什么呢?最经典的就是现在我们知道的这“五眼”,“五眼”里边北美包括了澳大利亚、新西兰,美国、加拿大、澳大利亚、新西兰,甚至包括南美都存在大规模的种族灭绝问题,就是原住民基本上消失殆尽了。一会儿我讲“现代殖民理论”马克思对这个事情的看法。

现代殖民理论里边涉及到的重点其实是我们中国。我们管自己叫半殖民地,其实这个半殖民地“半”到多少程度,主要是跟印度比较,印度算是全殖民地,我们算半殖民地,由于我们还有一个政府嘛。满清、民国还是个算是这个半独立的政府,就是还不是英国人直接派总督,所以叫半殖民地。但殖民的特色或者是我们给西方原始积累作出的贡献远比印度大。如果你问我英国、美国之所以能有那样的成就,其中,资本积累的主要贡献者就是中国。

好,我们进入今天的这堂课的这个阶段,我念一遍马克思在本章开篇的这段话:“政治经济学在原则上把两种极不相同的私有制混同起来了,其中一种以生产者自己的劳动为基础,另一种以剥削他人的劳动为基础。它忘记了后者不仅与前者直接对立,而且只是在前者的坟墓上成长起来的。”其实你知道我第一次读《资本论》的时候读到这一段话的时候我就被震到。

我想说,很多人在对中国的近代史的时候、读中国近代史的时候,我其实对中国当代的史家存着很深沉的愤怒。我不但看了中国的史家我感到愤怒,特别是近代史和当代史;我看了台湾、香港的史家,我感到更加的愤怒,是愤怒加倍。原因在哪里?原因在于对英国和美国的解释,特别是对美国的解释的部分。美国迅速崛起为现代化工业强国的资本积累,到底有多少是美国人自己通过劳动积累来获得的呢?这是一个问题啊!

我一直希望,能够对鸦片的问题做一个经济学的、完整的、具有统计学价值的一个结论。就是从外国人、从英国人开始向中国推销鸦片,一直到这个鸦片变成中国遍地开花,到底向中国卖了多少鸦片?拿走了多少白银?其中谁拿得最多?可能惊人的结果是,美国人在中国卖的鸦片远远超过英国人并且是最多,而且正是这些鸦片在上个世纪末,不是,在19世纪末——上半个世纪末和世纪初成为美国迅速工业化的资本源泉。

我们向英、美和欧洲乃至于日本提供的白银,构成了整个世界迅速工业化和现代化的重要的资本积累的源泉。很遗憾,马克思这一章里边没有完成统计学意义上的论述,它只是,刚才我念了这段话,你听到这段话你就会愤怒。马克思是愤怒的,因为他用了“只是在前者坟墓上成长起来的”,他用了这样的严厉的话语,然而殖民地的人们,包括我国的学者、专家们能把这件事情讲清楚说明白吗?没有,始终……

如果我们对历史看不明白,我们对现在能看明白吗?我们对未来能看明白吗?中国的原始积累是来自于哪里?中国未来的发展来自于哪里?中国可以持续一百年、两百年、乃至于三百年屹立不倒,不出现大规模的资本迁移的原因在哪里?如果这些事情都说不清楚,那么你怎么去面对那《国富论》、《资本论》和《通论》呢?你的学问在哪里呢?哦,我今天说的说的又稍微有一点点激动了,其实我对关于鸦片的这件事情呢,我一直在想,我也给我的学生出了题目,我说希望你们写一本“鸦片经济学”。

好,说一下子马克思,是马克思的“现代殖民理论”和“古典殖民理论”和“后殖民理论”,它们三者的区别。“古典殖民理论”是什么意思呢?是当时以英国为代表的欧洲的国家在工业化过程中出现了一个严重的问题,就是马克思说了“以土地为根本要素的生产资料不足,而劳动力过剩”。那怎么办呢?向外输出过剩的劳动力和设备。他们刚开始进行殖民的时候,他们认为劳动力是多余的——就是劳动力是廉价的、是不需要的,而土地是重要的。

所以你看到了一个经典的特征,就是他要杀人——要将原住民全部干掉,然后要他们的土地、要他们的矿山。要土地、要矿山,这是“古典殖民主义”的基本特征,它就是“古典殖民主义”的基本特征就是“种族灭绝”。因为它根本就不是让你来参与,它就是让你就全部消失了,通过疾病、通过散布流行疾病——“天花”之类的,来让你迅速的灭绝;你要是不灭绝,它就来屠杀;你要不行的话,它就将你迁移。以至于美洲大陆一亿人口最后只剩下“一百万”——印第安人一亿人口剩下“一百万”。加拿大、澳大利亚、新西兰,不遑多让

其他欧洲人在美洲、南美洲、中美洲、南美洲,以至于非洲干的也差不太多,不过没有英国人做得那么丧尽天良,没有英国人做得那么丧尽天良、丧心病狂。所以马克思那段话呢,我是能理解的,因为“古典的殖民主义”它就是要搞“种族灭绝”的,它只要生产资料,别的不要。马克思写的这个第二十五章“现代殖民理论”在说什么呢?他现代殖民呢,就比如说对印度和中国的殖民,他就不搞“种族灭绝”了,他搞的是什么呢?生产者与生产资料分离。你看马克思看问题他就非常的深邃、简单、清楚,他让生产者和生产资料分离。

什么意思呢?就是英国人在印度,他基本上是把你的生产资料,就是你的土地和矿山拿走,但我不杀你,你回来成为新的雇佣者。因为到了这个时候,劳动力又变成了一个非常重要的东西。因为美洲当大量的土地被占有之后,劳动力不足,所以开始大量地贩卖黑奴和中国的……。从中国贩卖人口到美洲,从非洲、中国贩卖大量的人口到美洲,成为奴隶、成为苦力。到了印度的时候,他们已经不这样做了,就是他只要生产资料,而不杀劳动力,他只让劳动者与生产资料分离。

“现代殖民理论”的特征就是“生产者与生产资料分离”,那么“后现代殖民理论”——就是现在,它是什么状态呢?它连生产资料也不要了,生产者与生产资料不需要分离。“后现代殖民理论”是建立在金融垄断资本主义基础上的,它通过对资本的操纵和市场的控制,通过资本与市场的控制,来控制具有某种主权意义或者主权特征的国家与地区,而它这种控制更加的隐秘,主要在宗教、文化、意识形态、资本、财政、金融。

“后现代殖民理论”到今天,其实没有一本像样的著作——经济学著作或者是政治经济学著作,来讨论“后现代殖民主义”原理,我也很遗憾。其实我看《资本论》的时候,我觉得马克思用如此之短的篇幅讨论殖民,我觉得是有一点不恰当的,就是因为这个殖民地问题是个大问题,尤其是……当然了,作为一个白种人、犹太人,可能在体会上他体会不到我们这种人——我们曾经被殖民过的国家、民族的人,那么痛彻地感受了。

在这一章里,马克思将资本积累与原始积累他做了一个区隔,因为一般意义的资本积累实际上是通过剥削剩余价值完成的,而非一般的资本积累或者是原始积累往往是通过生产资料的劫掠来完成的,一个是剥夺,一个是劫掠。也就是说,资本主义的原始积累相当的成分是来自于盗匪一般的劫掠,当他们今天道貌岸然地开什么民主峰会的时候,非常的可笑,因为他们之所以今天可以讨论民主,是因为他们是盗贼。

马克思的这章节里边,他说了一般意义的资本积累或者《资本论》里边研究的资本积累是剩余价值的剥削。他用了这一章告诉我们一个浅显的真理,就是资本主义原始积累里边相当部分的原始积累与剩余价值没关系,因为它是直接生产资料的掠夺。无论是消灭了生产者,把人家生产资料占有,还是生产者与生产资料的分离——就是“现代殖民理论”,“古典”和“现代”它都直接把生产资料拿走,只有后殖民主义的时代它不直接抢了,他通过资本与市场的操纵来完成财富的转移。

大家也可能对1840年之后的中国有一点点了解,就是“现代殖民理论”里边,他第一件事情就是他要控制中国的关税。可能是三个东西吧,第一是关税;第二个是中国的这个国债,满清政府向英国举债;第三个部分就是银行,就是金融;就是关税、国债和这个金融。就是后来大英帝国在中国开的这个银行代替了中国传统意义的票号,这个是“现代殖民理论”里边的经典特征。我的时间,总是不够用

我们平台上的朋友、有经济学基础的、有一点文笔的朋友们,我们可不可以组织起来,有些东西不能等。例如我今天讨论到的关于鸦片的问题,就是鸦片经济学。因为你没有写一部鸦片经济学或者鸦片经济史,你没有办法知道中国的近代史和现代史,你说不清楚的。为什么积贫积弱?为什么我们的白银不见了?为什么换中国白银换走最多的是美国?而中国人竟然什么都不知道。美国崛起了,发展迅速成长了,原因是什么呢?当然美国人的经济学不会这样讲,中国的经济学也不这样讲。这不行啊。

我最近在互联网上比较惨,就是能封我的所有地方都封掉了,我的名字甚至都不能出来,一出来就有问题。可能主要是香港问题,就是我谈得比较敏感。难道香港问题不是经典的现代殖民理论里边描述的问题?更是经典的后殖民主义理论的经典的案例和现象吗?好吧,我们说一下,我们研究的三个重要的课题。第一个应该写一部鸦片经济学,对吧?必须把这个事情写一下子。第二个必须

写一部中国现代殖民原理的实际的一种状况。好多人说你要说什么?在讨论中国问题上,无论是中国台湾和香港都很难统一一个看法,就是毛泽东和共产党是中华民族的解放者。但也有人特别是台湾和香港,很多人认为蒋介石可能在某种意义上就是也参与了中国的这个历史进程。但我写过文章介绍1927年10月份蒋先生做了什么,实际上他是现代殖民理论里边的那个中国代言。

现代殖民理论里边,殖民的过程需要在地精英的完美的配合,因为(生产资料和)生产者和生产资料都保留了,其中生产资料由西方占有,像中国半殖民地生产资料也没有完全占有。那么殖民地的管制需要经营代理人的。这个代理人是1927年10月份蒋先生以娶宋美龄的名义去到日本,实际上与英、美、日达成了一种共同的契约。到今天我们见不到那张纸,看到《蒋公日记》有什么意思?我们想知道在日本签署的那个密约的内容,其实那才是中国现代殖民史上的经典的文件。

第三部书应该是讨论一下子后殖民主义在中国。后殖民主义在中国大陆,比如说莫言和方方的现象、比如说伤痕文学、比如说这些东西,以及后殖民主义在经济上的表达。前两天有人给我讨论VIE这些事情,离岸的问题,实际上就是后殖民主义在中国经济上的表达,政治上有没有我们也需要一个深刻的思考。那么多的学者在研究问题没有上升到一个哲学的高度,所以所有的问题讨论讨去都是讨论一般的事物看法,或者是一般性的道德判断,没有上升到马克思《资本论》第一卷第二十五章现代殖民理论的高度。

对不起大家说得有点激动,我去方便了一下子。现代殖民理论实际上可以作为一个宏大或者是庞大的问题,它不光是一个理论,因为它涉及到我们用这个理论来看今天的现象。我说几句话可能会得罪一些朋友,但请原谅我,我还是得说。因为在我国的不同阶段的人有不同的教育背景。我是六零后,五零后、六零后、甚至七零后在后殖民主义问题上比较的没有那么大的麻烦。

我们生活在毛泽东时代,所以我们是反封建、反殖民这样的一个意识形态起来的。当然可能在文革的过程中有一些过的东西可能在身上也有烙印,但主体性是在的。八零后出小小问题,九零后又好一点。九零后我们大批的孩子出国留学了,他直接站在西方来回看中国或者是看西方会好一些。但八零后的问题非常大,我在香港接触到八零后的孩子们,特别是女同学女孩深沉的后殖民主义的烙印,因为她整个的审美过程。

我有时候不知道该说什么好,甚至我本人有的时候说不出话来。因为它是一个审美逻辑,就是当你站在哲学的高度审视欧美、审视日本的时候,可能在他们看来你这是在做什么?你这是在做什么?今天早晨有人还给我发帖子,说我写的关于欧日的那篇文章,就是用了非常糟糕的语言来评价。我知道他们没有经过思考,因为在他们的Matrix里边那个是最好的东西,你怎么能批评人家呢?我们的是糟糕的,那是最好的东西,你怎么能说它不行呢?

我一直是希望我国的高等教育部分应该把《资本论》放在一个比较重要的位置。原因就是重建我国的青年、特别是知识青年的主体性,这一点非常重要。因为学问本身如果不是建立在主体性基础上,就如同一把刀你是交给一个好人还是交给一个罪犯。如果主体性丧失了,这把刀交到他的手里边,可能就是变成杀人利器了。最近发生的好多事情,我们看到的很多事情里边,难道不是后殖民主义吗?

有朋友问我,你怎么看联想这件事情?今天我们讲这堂课其实都不用我说了,你们知道这是后殖民主义在中国的经典的案例或者现象。柳家的两个女儿,一个在滴滴,一个在Uber,然后在此之前是在美国读书,然后进入高盛,然后垄断了中国的我们衣、食、住、行的行的部分。你看不到后边那只长长的手的控制嘛,如果你看到了,那你说这是什么呢?甚至在某种意义上都不是后现代了,还直接控生产资料去了。

好,我们抓紧先把理论的部分过完。这个后殖民主义理论,不是,现代殖民主义理论主要揭示了三个东西。第一个部分,殖民意义上的民族压迫或者叫种族压迫,仍然是阶级压迫的一种形式。我重复一遍,因为这是重要的政治原理,殖民意义上的民族压迫或者是种族压迫仍然是阶级压迫的一种形式,或者说民族压迫或者种族压迫依旧是阶级压迫,请牢记这一点。它意味着民族解放不等于阶级解放,它虽然是一部分

但是民族解放了,阶级不一定解放。请牢记,阶级解放必须建立在民族解放基础上,就是这个原理,正的原理是成立的,逆的原理不成立。什么意思呢?任何的阶级解放它的前提是民族要解放,民族不解放,阶级解放不了;但是民族解放了,阶级未必一定能获得解放。我不知道大家知道不知道我在说什么?这不是我说的马克思说的,这是现代殖民理论的一个重要原理。就是阶级压迫,就是民族压迫这个解决了,它不代表,贵族来了嘛,可能阶级压迫还在,民族压迫不在。举个例子,比如说西藏,虽然是英国人走了

但西藏的领主们,那并不会解放农奴的,但你农奴要想获得解放,一定会把压在领主上面的外国人也通通赶跑,对吧?这是一个非常重要的原理,我再重复一遍,重要的东西我重复三遍。殖民意义上的民族压迫(或种族压迫)仍然是阶级压迫的一种形式,因为殖民压迫必然是最后剥夺最穷的劳动者的、生产者的,它不会剥夺精英的,精英是狗来的嘛,它只是剥夺那些最底层的人的嘛。所以它意味着两句话,第一句话,民族解放不一定代表阶级解放,阶级解放必须民族解放。两句话,第一句是民族解放不一定阶级解放,阶级解放必须民……

由此你读到这第二十五章,知道毛泽东和中国共产党的伟大,他们完成了对外国人完成了民族解放、完成了阶级的解放。这双重解放是中国进行工业化的前提条件,是中国的现代性必须具备的基础。待会儿我讲后殖民主义,我会讲为什么他们要《告别革命》。这堂课讲着讲着就有点儿不太像经济学了,但是也没办法,我们只能按照马克思的逻辑讲,因为我读第二十五章的时候,这事儿我就觉得跟经济学已经关系,当然有重大的关联,因为原始积累嘛,但它已经出离了经济学本身。

现代殖民理论强调的第一句话讲完。第二个也是一个重要的原理,就是现代殖民已经形成文化、教育以及制度的标准模式,就是现代殖民理论,在殖民地一定形成文化、教育及制度的标准的殖民模式。请牢记这一点,因为你要如果不知道这个东西,你就无法理解香港和台湾,你理解不了,因为它已经形成标准的文化、教育及制度模式。你知道我们谈“五十年不变”在谈什么吗?我们谈“五十年不变”,谈的是现代殖民理论里边讲的那个殖民的模式五十年不变,非常严重的问题。

又讲得激烈了一点儿,整理资料的朋友们、同学们,那个我们把这个事情处理好了。我不是说“五十年不变”不对,前两天那个小朋友找麻烦,就是又挑我的毛病,就是这变成了又否定和怎么着怎么着,我不是这个意思。就是当你觉得这个“制”,“一国两制”这个“制”,你认为它是一个好东西的时候,这个认识本身具有浓厚的后殖民主义的偏见,因为香港的制度是现代殖民已经形成的文化、教育及各种制度的标准的殖民地模式。

它是不是具有西方现代文明的特征?具备,当然具备。但你要分清楚,这个“制”里边有殖民文化、有殖民教育、有殖民制度安排,如果这里边文明之内包含的那些不好的东西,你看不懂、看不明白,你要付代价的。中国人一定要冷静!香港的回归的过程中,我们可以学到一些重要的东西。当然如果1983年我能讲这堂课、讲二十五章,也许我们在香港中英谈判上不至于犯那么严重的错误,不至于导致后边的后果。

好在今天我们讲这堂课讲《资本论》第一卷的第二十五章,它可以为将来解决台湾问题打下一个重要的理论基础,打下一个重要的逻辑框架。台湾的后殖民主义跟香港的后殖民主义不一样,因为台湾2100万人口里边有700万人是日裔或者是日系,日裔的意思就是他在血缘上跟日本人有关系,日系是他们相当一部分人原来是日治时期的精英,或者是有千丝万缕的联系,它属于那个时代所谓的现代殖民理论或者模式框架的一部分,他不愿意回归你不完全是关于领土的问题哟,这个事情要看透。

现代殖民理论的第三个问题,就是殖民地精英往往具有深彻骨髓的奴性。问题在于这个殖民地精英深彻骨髓的奴性却成为殖民地人民审美的一个部分,就是殖民地人民在审美上面将殖民地精英的奴性当成一种“美”来赞扬、赞美、歌颂,而不知道羞耻。他们不知道用鸦片将中国白银换走,这件事情是错的,他们只看到了劳力士、看到了爱马仕、看到了宾士、看到了那套豪宅,可悲呀!

我概述一下子啦,这堂课虽然没备好,但是讲的时间有点超长。现代殖民理论揭示了三个东西:第一个,殖民意义上的民族压迫或种族压迫仍然是阶级压迫的一种形式,民族解放不代表阶级解放,阶级解放必须民族解放,第一句;第二,现代殖民已经形成文化、教育及制度的模式,这个模式深深地嵌入到所有的生活当中,包括了精神和物质的所有生活当中了。第三个部分,殖民地精英的彻骨的、深彻骨髓的奴性,竟然成为殖民地的一种审美。如果这三个东西就是马克思现代殖民理论提到的这些东西,我们对之没有深刻的反思的话,我们还会犯历史性错误。

今天这堂课,无论是录音或是文字,请大家都不要外泄,你们知道就行了。因为我想这堂课可能重了一些,可能好多人可能受不了。虽然把我封死了,但是他们还是不想让我的声音出来,你们不要把它扩散,不要给平台添麻烦,不要给我添麻烦。好,最后一个部分我们谈一下后殖民主义。后殖民主义这个概念是1978年萨义德在《东方主义》或者叫《东方学》中提出的。我讲过,我在好像平台上,新朋友不知道,老朋友知道,我讲过萨义德, 讲过《东方学》。他在这个地方率先提出了后殖民主义的概念。

因为马克思第二十五章写的是现代殖民原理,现代殖民原理,到了二次大战之后,殖民地都通通解放了,大体上都解放了。但是殖民地解放了,殖民地的人民那颗心还没有解放,所以萨义德1978年写下了《东方学》。后殖民主义或者是后殖民理论或者后殖民主义理论,其实在萨义德之后,在很多的殖民地国家开始兴起,有了深刻的反思。然而,请注意时间节点,萨义德的《东方学》是1978年,我国正好赶上改革开放的时候,我国呢,正在对我国的……

我国正在对我国的革命与解放甚至文革进行深刻反思,所以我们这个时候,错过了在国际上进行后殖民主义的反思。我们那个时候正在拥抱西方,正在改革开放,所以我们不认为这是个问题,所以我们不认为这是个问题。请注意,香港的中英谈判是1983年开始,1984年《中英联合声明》。如果我们那个时候注意到萨义德的《东方学》,注意到后殖民主义,注意到这些问题的话,那么1984年《中英联合声明》就不会是现在我们看到这个版本,也不会导致2019年的黑衣动乱。我们希望,为什么反复谈这件事情,就是希望我们在处理台湾问题上做得更好一些。

后殖民主义有哪些特征呢?我先说经济的特征,后殖民主义的经济特征包括了三个方面:第一个是金融垄断资本主义。金融垄断资本主义很“棒”,它并不直接持有生产资料,它是通过控制资本流动和市场定价来完成对劳动者剥削的,他依旧是对生产者剩余价值进行剥削。我说了,种族压迫从来和阶级压迫是一回事儿,它仍然对劳动的剩余价值进行剥削,但它并不直接掌控你国的生产资料,这是后殖民主义的经济特征,这一点非常非常重要,我们好多朋友可能认为解放了就没有压迫了。

殖民主义的压迫,在后殖民主义时代的特征表达的时候,我们发现,后殖民主义的剥夺,有时候比古典殖民主义,没有那么残暴了,没有种族灭绝,比马克思所谓的现代殖民理论里边用鸦片换白银差不太多啊!很残酷的。你在香港你能感觉到那一万亿磅啊,没了!非常残酷的。呼……,请允许我透口气,因为我国的、我国的当代精英仍具现代殖民地的精英特征,所以我国的专家、学者在类似问题上做得太扯了!不是一般的差劲。

后殖民主义的第二个特征是政治特征,它就是代理人、代理人制,实际上有一种毫不掩饰的政治操纵。例如日本的“天降组”、台湾的“台独”、香港的“港独”,他们具有经典的代理人特征。当然在一些重要的国家,例如像中国、俄国这样的国家,代理人会做某种隐蔽,隐蔽性安排。虽然隐蔽了还看得到吧,对吧?不然你一个人读书,突然去了高盛、突然做了董事、突然去了“滴滴”、突然去了Uber,然后他们就都通通地获得了资金上市了,当然还有体制内的

我们说最后一个特征是文化特征,后殖民主义的文化特征,从伤痕文学开始。我为什么老批张艺谋?我为什么特别讨厌那部《菊豆》?为什么写知识分子,把知识分子描绘成一个嫁给一个老男人的女人! 而这个老男人没有性能力,然后老男人拼命地折磨这个女人,他描写我国知识分子跟我党的关系用这个《菊豆》来描写。然后这个女人没有办法,所以她采取的通奸。通奸,我国知识分子没有别的出路。通奸,什么狗屁的伤痕文学啊!

然后是莫言,然后是方方;然后他们文学不够,还要开始进行美学批判,从《告别革命》慢慢走向了反革命。香港这个地方很有趣,我曾对某个姓徐的教授感到了非常的愤怒,就是他们的张爱玲热和张爱玲研究。为什么丁玲战斗在太行山上,你们认为不好,那么好、那么美的文字!人生经历都不对,非要捧一个跟汉奸睡觉的张爱玲!而且她的那个东西我不能说她的文学上没有一点点的东西,为什么呢?!

难道我国的殖民地文化到了21世纪仍然如此的猖狂?!可以登上各大卫视,可以登上各种媒体,为什么《色戒》都能上演?为什么?为什么这个张爱玲热一直到今天演变成各种各样的殖民地文化模式,并且大行其道?!战斗在太行山上的丁玲不美吗?其实到今天我们看到的政治学著作、经济学著作、文学作品仍然具有非常清晰的后殖民主义文化特征,我一讨论到这儿有时候怒不可遏,受不了。

有没有进行反对后殖民主义的斗争呢?有的。我觉得德国历史学派算是在这个问题上最早从哲学高度来思考问题的。除了马克思之外,德国历史学派算是很牛的。我为此写下了我非常喜欢的一篇文章——《掠过佛莱堡》。这篇文章一万字,我差不多花了将近一年的心血写下来。写完这篇文章,我蹲在厕所哭了差不多三十分钟,哭得我的邻居都出来了,还以为发生什么事儿呢。其实我是感慨,我不是感慨我自己,我没什么好感慨的,我是感慨我们这个民族、我们这个国家,这个路走得太难了、太曲折了,而且,还没走出来,怎么走?还没走出来呢。

除了德国之外,苏联对后殖民主义的理解是非常肤浅的。苏联到了他的末期,就是八十年代,萨义德1978年写《东方学》的时候,苏联的政治家、学者对此不敏感,他们不懂什么叫后殖民主义,苏联终于被后殖民主义拿下了,悲剧啊。我们中国正在进行伟大的实践,这个伟大的实践,需要中国伟大的思想家、伟大的思考者,为他们把定方向,这个把定方向的核心是主体性。我们讲心学:主体性、适应性和创造性。没有这个真不行。

好,就讲这么多吧,虽然有一些事情可以谈,但是我知道一说可能就过了,一过了就太敏感。那么我们就讲这么多,关于这个、关于这个,其实我讲的比马克思那二十五章还多了。反正你们是我这么好的朋友,我们这个课也不需要那么拘谨,我们就尽可能地畅所欲言吧。另外,这个平台希望将来有可能,能够再升级为一个学术性的平台。我今天谈了三部书的问题,希望自告奋勇吧。需要资料我会提供,需要理论框架,我们可以讨论,好不好?我们把一些该做的重要的工作做完它。

好,留点时间讨论一下中央经济工作会议。中央经济工作会议,我的评价极高,因为“五原则”清晰,这“五原则”都是我们现在当下最重要的原则。第一条原则,坚定不移地做好自己的事情。这条原则好多人可能就是我写文章里反复强调中国赢在哪里——我自岿然不动。就是你们爱玩什么玩什么,你愿意南海就南海,你愿意台海就台海,你在中印边境也行,你愿意新疆也行,你玩儿就是玩儿了,我自岿然不动。就是我们不要被人打断,我们自己的这种逻辑或者自己的运行的这个状态。所以这五原则中央提出第一条就是坚定不移做好自己的事情,太棒了,就是要这样。

第二条,第二个原则,不断做强经济基础。好多人不知道这句话在说什么,不断做强经济基础,那么我们就要问:什么是经济基础?什么是上层建筑?什么是生产力发展的根本?做好经济基础非常非常的重要。实际上真正的工业基础、真正的经济基础在第一产业和第二产业之中,我们要把我们的经济基础夯实、夯牢,这一点抓得非常的准,我们不盲目地去搞所谓的升级换代、腾笼换鸟,这一句话回答了这个问题,非常好。不断做强经济基础,不要腾笼换鸟,不要华而不实。

第三、第三个原则,增强科技创新能力,太重要了吧,看看人家逻辑性还挺强的。第一、坚定不移做好自己事情;第二、不断做强经济基础;第三才是增强科技创新能力。这跟以前我们的提法不一样了吧?以前要腾笼换鸟,现在是要做强经济基础之后增强科技创新能力。第四条,多边主义,高标准国际贸易、高水平对外开放。多边主义谈的是中国与国际的关系,多边,这里边包含了我们单独的、跟东盟的、跟欧洲的、跟非洲的、跟南美的关系,多边主义,这一条非常好,非常重要,做得非常对,高标准的国际贸易。

好多人不一定知道,什么叫高标准的国际贸易,高标准的国际贸易就是按照现代的国际最高的标准,我们原来有一道道的门槛,按照最高标准来进行国际贸易,实际上就是中国政府主动放弃一些补贴、一些不平等的贸易,进行高标准国际贸易,她是个负责任大国了。高水平对外开放,高水平对外开放就是以前的开放,现在进入到比较高的层级,这里边高水平里边包含了两个层面的含义:一个层面,就是我们开放终于开放到金融了,就是我们以前开放的是比较低端的,现在最高端的也开放给你。第二个是我们不光是把我们的市场让出来,我们也要开始进入你的市场,我们也要进你的金融,就是高水平开放。好。

第五个原则,深层次改革。这个中央经济工作会议在深层次改革上没有点透、没有琢磨,比如说税改没有提,但我知道,这个会议上不提这些事情是对的。看五个原则全部涵盖,抓住了,非常好、非常好。这是中央工作的会议,我看了两遍。我觉得现在这一届政府确实是中国最近这一百年来最棒的,而且很朴实,抓问题抓得很准,执行力也非常强,所以我觉得非常踏实。这个会开的,就是我看的中央经济工作会议,我去年也解说过,今年再次解说,是非常好的一次这个中央经济工作会议,很棒!

第二个给我的印象就是历史性的,就是中国从2021年开始,我们从跟节奏到带节奏的过程,就是完成了从跟节奏到带节奏的过渡。我们现在整个的发展思路也非常清晰,好多朋友说方向,方向就是四个,绿电、AI、分子生物科技、太空科技,四个主要的阵位,传统的东西行不行呢?我待会儿到第三个方面再说,到第三个部分再讲,绿电我们说清楚了啊,AI我们也说清楚了啊,分子生物科技这个事情我会留一丝儿功夫,梳理一下子

也可能会介绍一些中国分子生物科技发展比较好的一些公司吧。太空科技、太空科技现在大家可能还没有觉得它的重要性,太空科技会变得极为重要。所以在绿电、AI、分子生物科技和太空科技方面,我们这个四个方向都是走得不错,并驾齐驱。其中可能在AI方面我们有缺,比如说晶片,分子生物科技方面是中国最弱的部分,现在正在抓紧时间赶上。其中我参与的,好吧不说了。我觉得我们正在抓紧时间赶上。正好这些东西都是我们明年要重点关注的内容,重点关注的内容,如果金在某个时候完成的话,我们可能会开始转型,现在必须开始“369”,盯上去。

好,最后说一下子一收一放。就是我们玩了一次经典的汇率操纵,星期四,中国先是降准——存款准备金;然后又提准——外汇的准备金;内部降、外边升;一降一升,好多人说看不懂了、看不懂了,为什么中国要玩经典的汇率操纵?简单说一下吧,就是全世界的资本,正在蜂拥前来,全世界的资本正在蜂拥前来,印度已经开始将人民币作为它的这个锚定的基础吧,就是印度的卢比,要锁定人民币,跟人民币共进退。

在香港感受比较强烈,就是人和资本进入的量是开始了。就是经历了2019年黑衣风暴、2020年的这个病毒,2021年开始人和资本进入。当然他们不是来香港的,他们是要进大陆的。这里边有三个原因。第一个原因呢我们国家的疫情控制非常好,以至于我国的经济没有受到重创,我国的生态,产业链、生态越来越完整,而且我国的企业经过一轮一轮的整合,冒出了一大批具有创造价值能力的,潜能十分优异的企业。

第一,疫情。第二,我国的生态已经形成了,这个生态里边必然冒出一大堆的好东西。第三,目前全球的货币的估值,美元、欧元、日元这些货币的估值整体上偏高,人民币的估值偏低。资产便宜,货币估值偏低,会发生什么情况?那只能是资本大规模涌入。中国虽然用强制性手段提高了成本,但我个人认为效果有限。提到9,从5提到7,从7提到9有意义吗?可能不行,还会大规模涌入,记着我说的话。

第一,这一回的通胀是结构性通胀,不是周期性通胀,回不去喽。第二句话,全球经济开始逐渐进入衰退,中国是第三季度开始进入衰退,第四季度明显,美国第四季度开始进入衰退,全球经济受影响可能在明年进入衰退周期。第三个部分,第三个部分,各国的货币的估值必须进行价值回归,必须进行价值回归。我们今天赞扬了中央经济工作会议,但是我们在宏观经济政策上面还有可以商量的地方、可以讨论的地方,不管吧,作为投资者,我们要开始密切注意国际金融资本进入之后所导致的现象。

其中,其中香港,已经开始蠢蠢欲动,蠢蠢欲动的部分不完全是新经济,旧经济可能成为大资金进入的基础,比如说金融类、银行、保险,比如说一些传统,因为结构性通胀导致以前我们不喜欢那些东西,比如说什么钢铁啦什么的,这些东西可能都陆续的开始和以前不一样了。当然最有价值的还是我刚才说的绿电、AI、分子生物科技和太空科技。今天拉拉杂杂的又讲了差不多80分钟了,我也没劲儿了,就聊这么多,然后下个礼拜我们聊天。因为到年底了,我得准备点年底的东西。

顺便说一句,又到12月份了,明年春节很早,好像2月1号,所以大家得抓紧时间写总结喽。养成好习惯吧,养成好习惯。会非常有用的,会终身受用的。今天就说这么多吧,这个气候变化得挺大,这个疫情还有不确定,大家还是要注意,注意卫生、注意保暖、保重身体健康。好,我们下周见。

\section{价值规律、 苏联解体十年祭}

大家好,今天是2021年的12月25日,农历辛丑年十一月二十二日。今天是《资本论》第十九讲——价值规律。其实这堂课可以说是我们整个《资本论》里边的一个非常重要的裉节儿,就是它是一个坎。这个可以把它当成一个解释《资本论》里边的几乎所有重要问题的一个底层逻辑,可能会有很大争议,我们今天试一试吧,因为必须得迈过去了。好,3点钟我们准时开始。

大家好,今天是2021年12月25日圣诞节,农历是辛丑年的十一月二十二日。与去年的情况类似,去年是圣诞和元旦是在隔离期间,我们继续地上课和聊天。今年的圣诞我就呆在写字楼里了,不管是圣诞夜还是圣诞,我都在写字楼里跟大家一起度过。其实挺有意思的,原本是他们约我晚上去热闹热闹,后来我想了想,因为这是第十九讲——价值规律,我怕我没有弄……

我总怕是,这个事情我怕有太多太多的歧见,大家会有一些分歧,所以我昨天晚上还是再加了加班,没有出去折腾。因为我想了想这个“价值规律”这堂课在我的大纲里边算是重头戏。另外要结合当下的现实,它实际上是中美博弈的一个重要的底层逻辑,也是MMT理论的一个重要的底层逻辑。所以我们在这个问题上不敢大意,要做更细致一点的努力。

马克思《资本论》里边并未有专门的章节讨论价值规律,我们的依据是恩格斯《资本论》第三卷增补,也就是说马克思《资本论》的三卷结束之后,在第三卷第七篇第五十二章“阶级”这一章结束之后,恩格斯补了一个《资本论》第三卷的增补。增补了两个内容:一个是价值规律和利润率,一个是交易所。大家应该都知道,恩格斯这个人心极细,而且他有洁癖,他不愿意在马克思的原著上做任何的增减,他之所以把它挑出来是因为这事儿太重要。

恩格斯的这个增补其实内容并不是特别的多,只不过由于整个《资本论》三卷里边断断续续始终涉及到价值论的问题。价值论里边最核心的部分就是价值规律,价值规律里边就有利润率的问题,而整个的交易又是围绕着交易所进行。这件事情恩格斯在最后的增补里边把它单列出来,显然恩格斯对《资本论》的理解早已经入木三分了。这是问题的关键,所以我们把它放在第十九讲。到了第二十讲的时候我们可能会讨论阶级问题了,会讨论一些涉及到资本的一些政治上的一些结构性的问题。但第十九讲是个关键,我尽量努力吧。

因为,我说过了,好多重要的朋友在这个平台上面,所以也会在平台下有一些争论,这是非常好的、这是非常好的。因为往往高手过招有的时候才能把一些重大的、哲学性的问题或者是涉及到的历史重大问题搞搞清楚。其实这件事情就是关于恩格斯《资本论》第三卷增补这件事情,其实我一读《资本论》的时候已经注意到了。但其实越是大家认为熟悉的东西——类似于价值规律,其实里边的内涵和深度往往不一定能摸得到。

这一件事情重要在哪呢?重要在恩格斯似乎、似乎在确认马克思关于价值规律的判断之后,又完成了对马克思价值规律的一些补充或者叫补白。什么意思呢?就是通常我们认为价格和价值总商品的时候是一致的,长时间的时候是一致的,它是有一个约束性条件的,但价格和价值背离它是永恒的,它要不背离它就不正常了。背离,那么背离的逻辑是什么?这才是《资本论》要回答的关键性问题。

如何背离?是什么原因导致背离?背离对劳动所得和资本利得会构成何种的影响?所以我注意到恩格斯真的很厉害。恩格斯在这个增补里边,他用非常简洁的语言非常清晰地把这件事情做了一个概述和总结。所以《资本论》第三卷这个补白、这个增补就显得极为重要了。我们还是按照讲课的习惯从定义和这个意义来出发,我们第一个部分讲价值规律的定义和它的意义。马克思给出的价值规律的定义是什么呢?就是价值规律是商品生产和商品交换的基本经济规律。

这里边有一个非常重要的定义,就是商品的价值量取决于社会的必要劳动时间,价值量取决于必要的社会劳动时间。这是马克思给出的一个具有哲学高度的一个关于价值的一个定义——就是价值是必要的劳动时间。他的哲学概述的前提是总商品、所有的商品。总商品它是符合这个规律的,就是总商品里边的那个总的必要劳动时间和它的价值定义、决定了商品的价值,在总商品这个概念下它是成立的,拉长时间它也是成立的。举个例子吧。

北京,古都,一千年之前就有房子、房产,那么一千年的房产的这个价格,剔除它中间的一些波动,它的基本的价格和它的价值是吻合的,大概平均在五根金条这样一个水平,大概是这个意思吧。当然我们不必要去钻牛角尖。总商品和历史长河,它是一个空间概念和时间概念,因为哲学必须解决这个问题,所以马克思的定义是精到而准确的。但这一件事情它被现代经济学带到沟里去了。为什么呢?因为价格。

价格永远是变动不居的,而我们一般意义上接受的经济学是,价格是由供需决定的,供给与需求决定的。当供给大于需求的时候,价格就会向下;当供给小于需求的时候,价格就会上行,就是供给与需求——供需决定商品的价格。貌似、貌似合理,可能在某种特定条件下也成立,如果我们就到此为止,其实马克思——哦,恩格斯就不会写“增补”,增补的意思是他们发现了一个重大的问题,这也是《资本论》要说的问题

在资本主义条件下,供给与需求是由资本决定的,那么结论也就出来了——价格在资本主义条件下是由资本决定的。我念一段马克思的原话,这是马克思《资本论》第三卷上的一句原话,

马克思是这样说的:“商品不只是当作商品来交换的,而是当作资本的产品来交换的。这些资本要求从剩余价值的总量中分到和它们各自的量成比例的一份,或者它们的量相等时,要求分到相等的一部分。”其实你读完这段话,你知道马克思已经说清楚了,他不是把它当作商品来交换,而是当作资本的产品来交换的——资本决定论。价格是资本决定论,而价值是劳动决定论。请注意到它们两个的区别,价值是劳动……

请一定要注意到问题的本质:价值是劳动决定论,价格是资本决定论。我们的好多聪明的朋友也发现了这个问题,例如何新,他也提到了金融决定论、金融决定价格。金融决定论或者是资本决定论,大体上对供需决定,就是当代西方经济学里边的供需决定是一次——我觉得这个不是推翻的问题,也不是挑战的问题,是一个完善和补充,因为它基本上趋近于揭露了问题的本质,甚至现代理论MMT里边,实际上……

实际上恰好是在用《资本论》里边的价值规律,因为“资本创造需求”这句话非常重要,资本创造需求,所以资本创造价格;因为资本可以截断供给,注意贸易战、科技战、金融战,资本可以截断供给,所以资本也可以创造价格。想说什么呢?刚才马克思的那段定义里边涉及到两个内容,一个是关于价格的问题,另外一个是关于剩余价值的问题。

马克思那段话我翻译成中文的白话文就是说,资本在整个的交易过程中,是要参与这个剩余价值量的分配的,也就是说资本利得里边它要涉及到如何分配劳动所得,整个的剥削在价值规律里边已经做出了比较明确的表达。整个的这个价值规律的过程中,我们深切感觉到了资本主义发展的不同阶段,特别是资本主义发展的比较高级的阶段,到了垄断阶段,它基本上形成了对资产价格或者是商品价格的一种绝对的控制,垄断的阶段是一种绝对控制。

如果,你在香港观察、连续观察香港的房地产,你就懂了马克思和恩格斯要说什么。房地产的价格、香港房地产的价格,恰恰是在资本垄断之下形成的一种特殊的价格,它超越了价值规律了没有?它当然是这个价格远远超越了必要劳动时间,远远超越了必要劳动时间,但就是因为它极度扭曲超越了必要劳动时间,所以通过超级地租形成了对劳动者剩余价值的占有、剥夺。我知道这堂课不好说、不好讲,因为讲着讲着就绕里边去了。

为了说明这个问题,我们回到这个基本面来。大家可能记得我不只一次谈过研究经济的三大理论,第一个是创造价值的理论,第二个是资产定价理论,第三个是价值投资理论,三大理论。三大理论是构成我们研究投资的根本性的理论,创造价值的理论我就不说了,我们讨论第二条资产定价的理论。资产定价的理论的底层逻辑就是价值规律,你说重要不重要?如果你不知道资产定价的理论,你就无法理解价格与价值背离的原因,你就无法理解房价、金价是怎样出现的。

资产定价的理论其实是价值论里边的非常重要的部分。就是我们讨论资产定价,首先在哲学上它是一个必要劳动,是劳动定价,因为所有的资产也好、商品也好,是由必要劳动时间来决定的。但它是一个总商品的概念,或者是一个长时间的概念,在单个的商品、特定的时间节点,它的价格是可以严重背离价值的。而这种严重的背离,如果你关注到资本对需求与供给的扭曲,你就可以理解发生了什么。这一点非常非常重要。

恩格斯在《资本论》第三卷第七篇第五十二章《阶级》结束了之后,他把这个价值规律做一次概述,这个价值规律和利润率做一次概述。这个我看完了以后我其实挺惊讶的,这个价值规律讨论的是资产定价理论,利润率讨论的就是价值投资理论。资产定价理论是第一个部分,第二个部分是价值投资理论,因为利润率嘛。后边他又加补了一节就是《交易所》。

交易所就是资本扭曲资产价格或商品价格的那么一个特定的工具。讲到这个地方大概齐我们开始可以理解价值规律在微观层面的含义,在微观层面我们可以能够看到一些东西了,如果加上我今天重复的三大理论的运用,可能对大家来理解资产定价有着重大意义,甚至是决定性的意义。我们要说的是……,在这里边要注意的是两个重要的结论。

第一个重要的结论是:资产与商品的价值决定于必要劳动时间,就是劳动决定价值,这是第一个结论;第二个结论是:资产与商品的价格由资本决定;就是劳动决定价值,资本决定价格。而资本为了获取资本利得,必然通过扭曲资产或商品的定价来剥夺劳动所得,或者恩格斯在增补的部分给整个《资本论》三卷做了一次总结,这个总结微观的意义大体如此,宏观的意义在哪里呢?

它宏观的意义就是:如果你是一个伟大的国家,那么你要治理这个国家,那么你就必须遏制资本对价格的垄断或者是超级垄断。因为如果放纵资本对价格——资产价格和商品价格的垄断,导致资本利得严重侵蚀劳动所得,必然导致社会陷入到一种悲惨的状况,或者是一个混乱的状况,或者是一个无效率的状况,所以社会主义的存在就是要对冲资本对资产和商品价格的垄断和操纵。

现代资本主义深受马克思《资本论》的影响,深受社会主义运动和社会主义思潮的冲击,所以他们在一定程度上(包括英、美、法等发达资本主义国家)都在某种程度上通过立法对资本对商品和资产价格的垄断进行了某种程度的约束或者控制,有的约束控制得很好,有的约束控制得比较差。在宏观层面这个控制是源于财政政策,其中核心是税政和金融政策,他们通过这两个政策进行某种程度的调控。

易言之,社会主义国家一定要尊重价值规律的,资本主义国家在某种意义上也一定要遵从价值规律,如果不遵从价值规律,那么国家治理必然走向衰败、走向历史的反面。所以大体上你可以理解价值规律在整个微观层面和宏观层面的重大的现实意义,因为价值规律在资本主义的社会条件下,它会形成历史性的悖论,因为一旦资本控制了供给与需求,那么也就控制了价格。

这涉及到我们对事物的一个认证和理解。按照马克思的价值规律,那么所有的商品和资产都存在一个合理定价的问题,这个合理定价的逻辑就是劳动——必要的劳动时间,如果这个合理定价能成为交易现实,那么资本利得和劳动所得必然达至某种公允,但事实上,无论是我们所谓的社会主义国家和那个所谓的资本主义国家,里边大部分的时间合理定价都是不存在的,都是不合理定价或者是扭曲的。

我们在社会主义建设的初级阶段,为了迅速完成工业化,对农产品定价就偏低了,对第一产业的劳动所得进行了某种程度的剥夺,以至于形成工业积累。请注意,那个时候农业的劳动所得、甚至农业的资本利得都被扭曲了,所谓剪刀差、城乡差别。剪刀差是为了使工业资本利得能够得到某种程度的放大,或者是迅速完成积累,这是所有的社会主义国家初期,大部分的国家都会出现的问题,它是不是不符合价值规律?是的。

是的,它不符合价值规律,但它又是国家治理所必需。同时由此我们开始能够理解为什么有些国家商业资本利得会那么的高,为什么有些国家金融资本利得会那么的高。当一个国家的财政主权和金融主权不在老百姓、人民手中的时候,那么整个的倾斜就会导致第一产业——就是农业资本利得、工业资本利得迅速被蚕食,而增加商业资本利得,特别是增加金融资本利得,从而形成地方发展的悲剧,它经典代表就是香港。

香港的扭曲就是扭曲资产价格,其中最核心的资产就是房地产,就是通过扭曲房地产价格,为香港的金融资本带来超级金融资本利得,它源于超级地租。其实你要是读懂了《资本论》,真的理解了价值规律,那些小动作在你面前洞若观火、一清二楚。如今在美国发生的事情依旧是如此,由于美国的最主要的外部约束力量——苏联解体了,没了;内部的约束力量,美国的左翼受到了残酷的打击……

所以在某种程度上,犹太金融资本掌控了美国的财政制度和财政政策的制定权,掌控了美国的金融政策制定权,以至于他们开始长时间的扭曲美国的资产价格或者是商品价格。2008年的金融危机就是扭曲房地产价格的杰作,它使得少部分人的金融资本利得实现了大跃进,而使得美国的国民的利益受到伤害,并且使得美国的经济出现严重问题,其后这个问题并未得到解决。

在今后的,我们现在还剩下五章,我们可能会留出一节课来讨论阶级问题。阶级是个立法权问题,立法权里边就是《资本论》里边的资本决定还是劳动决定的问题。这里边就是说到美国2008年之后是劳动能决定的吗?不是,仍是资本决定,而且这个资本决定升级到更高的程度,就是它不仅炒房子了,它还炒资本市场。所以你们注意到美国的资本市场的价格也出现了一次非理性繁荣,它又导致美国第二轮的超级资本利得的出现。当然美国资本市场的超级资本利得的出现不仅是掠夺美国的劳动者的剩余价值,而是掠夺全球劳动者的剩余价值。

恩格斯在第三卷增补的内容里边谈到利润率,其实他谈到的是资本利得的决定的这个价格以后导致的利润率的出现,它决定了资本流动的方向。我们一直在研究投资的时候讨论资本的三个问题,一个是资本的流向,一个是资本的流量,一个是资本的流速。它决定了在某一个特定领域里边的资产价格或者是商品价格,所以我们在四矩阵里边就看到了它们的价格的变动,其实这里边决定性的原理、底层逻辑就是价值规律,就是价值规律。

还没有来得及谈阶级,但是我们今天可以将一小部分的结论给出来。何为定价权?这个定价权应该是由供需决定,但这个世界上供需可以决定的事情很少的,还是资本决定。资本决定资产的价格和商品的价格。这个定价权涉及到什么呢?涉及到基本的人权。对了,它就是基本的人权。定价权涉及到基本的人权。资本决定权,重新定义了人权。当西方世界在讨论我们的民主、自由和人权的时候,

极少有人说定价权才是最基本的人权。如果这个定价权控制在少数资本家手上,特别是犹太资本家手上,这何尝不是一种野蛮的奴役过程呢!?这个奴役虽说不是用刺刀,没有那么血腥,但它仍然是赤裸裸的奴役,剥削、压榨和奴役。如果我们回到《资本论》的层面来重新讨论人权的时候,我们所看到的可能他们认为的集权、专制反而可能形成一部分基于定价权的人权。

讨论的有点复杂了,好吧,我们进入到今天的第三个环节,就是MMT与价值规律,凯尔顿的MMT的理论。其实凯尔顿是一个优秀的经济学家,我相信她也算是在美国或者是西方能读懂《资本论》和《通论》的不多的经济学家之一。MMT的理论的依据就是价值规律、就是价值规律。只不过它在价值规律的基础上它升级了,所以MMT认为“需求创造信用”。

请记住这句话:“需求创造信用”。好多朋友说为什么是需求创造信用呢?实际上这是一个复杂的逻辑结构,拥有资本者可以创造需求,而需求决定商品或资产的价格,而价格又决定了本币的信用,这是一个非常复杂的逻辑结构。但这个逻辑结构在特定条件下不但成立,而且是真理,所以才有MMT。今天这堂课整理的同学可能要辛苦一些,因为这个逻辑过程整理出来以后,它有它重大的理论意义和现实的指导意义。

其实MMT理论的成立的前提是SWIFT,就是全球的美元结算系统、全球的美元结算系统。因为在这个系统下,美国只需要创造资本就能创造需求,通过创造需求就能创造信用,这个逻辑链是这样形成的。实际上凯尔顿在她的著作里边也把这个事情解释清楚了,不过这里边有一个可能并不是永恒存在的前提条件,那就是生产的过剩、生产的过剩,不管是绝对过剩和相对过剩,这个前提并非永恒存在。在特定历史周期

由于中国这样的世界工厂出现,她的能力极为强悍,而且除了中国以外,可能整个南亚、东南亚它都是以生产来出现,它就是创造供给,而且这个供给一直处于相对过剩的阶段,所以创造需求就变得必要了,因为不能创造需求,那么这个供需矛盾实际上就失衡,那么它通过创造需求来平抑供给,形成了MMT。但这个世界它并不是按照凯尔顿的逻辑走的,所以美国人选出了一个总统,他叫特朗普,特朗普他决定打击供给——贸易战、科技战、金融战,他来打击供给,而这个世界还有一些特别疯狂的人,

他们将粮食作为武器、将能源作为武器,他们时不时的通过掌握这些供给与需求来制造恶性通货膨胀和恶性通货紧缩。如果你今天听到这堂课,你就知道了什么恶性通货膨胀、什么恶性通货紧缩,它的底层逻辑依旧是马克思的价值规律。只不过价值规律正向的讲是劳动决定价值,逆向的讲就是资本决定价格,是资本操纵价格后出现的某种波动,这个波动恰好是资本获得利得的一个重要的源泉,或者是我们讲的这种有规律的波动,正好是资本收割韭菜的过程。

这里边有很多的事情,就是价值通常在我们现代,特别是现代的经济学里边,已经大体上把它作为一个没有用的概念忽略掉了。没有谁去计算一个茶杯、一个手机、一辆汽车它所达到的必要劳动时间,也就是说价值量的表达没有那么精确了、没有那么的精确了。偶尔我们用一个概念叫成本,就是成本来表达一下子价值,但真实价值这里边应该蕴含的劳动所得……

真实的价值表达就是实际上隐含的必要劳动时间和应给劳动者的劳动所得的部分的精算。其实已经不是那么地被重视,而大家关注的主要还是价格,而价格由于供需理论、供需曲线、供需决定论导致了资本剥夺的合理性,或者叫资本利得合理性。因为价格是由市场决定的,那么就是上帝决定的,所以那么它就是合理的了。我刚才第一段的时候念过马克思的那段原话,在结束之前我会第二次再念一遍。恩格斯对这段话有解释。

如大家将来有时间呢,我还是希望你们买一本《资本论》,可以去读一下子恩格斯《资本论》第三卷增补的这两节。一节是价格规律和利润率,第二节是交易所,交易所这一段很短。恩格斯没有说那么多的话,我想这个增补放在这,其实已经把他的意图交代清楚了。读《资本论》,马克思的每一卷、每一卷的这个卷首的说明、前言、介绍和他最后这个增补其实是非常重要的。因为没有人像他那么去了解马克思本人是怎样思考的,恩格斯是最逼近马克思真实想法的那个人,而且他有足够的理论素养。

价值表达既然如此重要,那么作为国家治理——一个负责任的大国的国家治理就必须尊重价值规律,就必须非常清晰地来计算价值表达、必要的劳动量和必要的劳动所得,这是一个文明的标志。也就是说一个负责任的大国,你不管叫资本主义也好、叫社会主义也好,叫有中国特色的社会主义也好,这是一个基础性的工作,就是对资产与商品的价值表达应有一个精算。这个精算内容里边,特别是要包含对劳动所得的。

如果我们知道了它的价值,我们在价值与价格的关系里边就知道如何做某种形式的对冲、对冲。因为定价权是基本的人权,如果我们对定价权不做管理,实际上我们在某种意义上,以香港为例,放弃了财政主权和金融主权之后,将定价权直接交给资本,那政府或政党对人民就显得不是那么的负责任了。一个负责任的政党或者是政府必须非常清晰地处理价值规律问题,价值表达非常重要。

或者很多人认为难道交易的过程、市场的过程没有表达吗?请一定记住,它表达的是价格和价值有内在关系,但没有直接关系。而价格的表达要么被扭曲为极夸张的,就是价格严重背离价值;要么它就被压缩,所以形成了某种意义上的对剩余价值的剥夺。马克思说它拿到它合理的份额,而不是拿到它不合理的份额,拿到它不合理的份额,它就必然导致资本的大规模出逃,类似于香港1984年到1997年所发生的事。你说它是交易结果吗?好像是,但其……

在这里边我要提示一下子,随着产业不断的升级,随着科学技术的发展,我们已经在2015年进入到数字经济时代了。数字经济时代,资产或商品的定价开始发生快速地漂移。漂移就是什么?资本的载体由传统的资产,比如说砖头在迅速漂移向Data数据。所以在新的时代,资产定价正在发生猛烈的变化。以美国为例,现在苹果已经是2.8万亿美元的市值,整个中国资本市场大约是……

以前2015年还是6个苹果,现在变成5个苹果。整个的定价过程它既有它的合理性,它毕竟是市场交易的结果,它也有存在内在的制度和政策上的原因——财政制度和金融政策上的原因导致出现这样一个局面。这里边有三件事情提醒大家注意,我们开始要做研究了。因为信用也在转移,需求、创造需求,创造出一个什么新的需求呢?碳排放权,它变成了新的信用的载体,甚至可能构成货币发行的依据,大家要高度注意这档子事儿,这件事情非常非常重要。

另外由于高科技的突飞猛进,无论是资本的载体还是国家的安全更依赖两个空间。第一个是太空——外太空,就是太空具有对地球安全的决定性意义,它当然不仅仅是个安全意义,它也具有资本决定性的意义,具有价格决定性的意义;第二个是电磁空间,电磁空间就是苹果它们现在正在抢的,苹果啊、谷歌啊,它们美国的科技公司正在抢占的电磁空间——太空、电磁空间、碳排放权。在整个的价值规律里边,随着时代……

随着时代的不断的发展,资本的载体在变化,定价的方法、定价的逻辑也在做某种意义的成长、延伸。我们懂价值规律的时候,也要随着整个时代不断地向前走动,学会前进。我们今天第一个部分是定义以及意义。第二个事情是讲了这个悖论,就是定价逻辑的悖论,就是价值规律的悖论,就是在资本主义社会里边,到底价格是个怎么回事。第三个部分讲了MMT与价值规律。第四个部分,我们开始讲第四个部分就是新的博弈论。

我国在整个的成长过程中,今年是建国72周年。在成长的过程中呢,我们经历了两个阶段吧。第一个阶段是受苏联影响的那一套苏联《政治经济学教科书》下边的一整套体系,那里边儿呢我个人认为相比较而言是比较尊重价值规律的。同时政府在定价权的问题上还是有全民族考虑、国家整体思考、对人民的负责和对国家整体的负责,这个是有的。同时由于我国那一任领导同志具有极高的觉悟和水平,包括毛泽东、包括邓小平。

所以他们比较早(在上个世纪六十年代)就已经发现了国家资本主义的问题,就是国家资本主义必然走向官僚垄断资本主义。我今天在开课之前呢,还在我尊敬的这个李慎明先生,李慎明院长他发的一篇文章上面,我做了一点点的这个评述就是:苏联的解体,因为这个圣诞节正好是苏联解体,三十年前圣诞节苏联解体了。那一年我二十八岁,所以记记忆犹新呐,因为正好是一个特定的历史时期,就是决定我们每一个人命运的那个特定的历史时期。

好多人在讨论苏联解体的原因,我说你们不如倒过来思考,中国未解体的原因,为什么苏联解体,中国没有解体?实际上苏联的解体是源于赫鲁晓夫,也不要一定就当赫鲁晓夫是个坏人,不要用好坏来鉴定。赫鲁晓夫可能比我国的领导人更早发现国家资本主义的问题,只不过赫鲁晓夫没有那么高的哲学高度,也没有能力来解决国家资本主义必然导致官僚垄断资本主义的这样一个结论,没有解决。最后,到了1991年解体了,而我国的领导人一直进行深刻的思考。毛泽东、刘少奇、邓小平、陈云都在想一个问题,所以他们在六十年代。

分别带着人去读苏联《政治经济学教科书》,毛泽东读完了发现官僚垄断资本主义是个大问题,要打倒他们,所以发动了文革,后来也没怎么打倒。但这个事情整个的过程积累了非常重要的、奠定了非常重要的理论基础和实践经验。实践经验是失败的,理论基础——无产阶级专政下继续革命,这个理论也是不成功的。但它为改革开放奠定了一条路,就是它试过了,试错试过了,这条路——人民直接代替官僚垄断,这条路走不通。那么怎么办?必须向国家资本主义引入社会资本主义,所以改革开放顺理成章。

1978年我们开始改革开放,改革就是让社会资本进入国家资本主义,开放就是让国际资本进入国家资本主义,这两个进入之后形成对国家资本主义固有矛盾的某种程度的对冲。同时,它在某种意义上也解决了国家资本主义走向官僚垄断资本主义,这条路被堵上了,所以1989年我们过去了。中国在苏联已经完成了官僚垄断资本主义的那一刻,我们跨越了那道坎,历史的幸运。今天我们回到价值规律,回到定价理论,回到这个基本的价值规律里边来。

我国在改革开放之后,曾经有一段时间,可能用两个事情来定义时间节点,我一向对时间敏感,大家都知道我每天念时间、念时间。1994年的分税制和1995年的联系汇率算是两个重要的事情,就是我们放弃了一部分的财政主权和一部分的金融主权,然后换得了美国,特别是美国和西方对中国的资本和市场的一个(就是我们是一个严重资本稀缺和市场稀缺的国家),我们放弃一部分主权,换得了资本和市场。这件事情持续到什么时间呢?是2014年和2015年。2014年我们取消了分税制。

2014年我们取消了分税制,央税、地税合并,2015年我们取消了联汇,二十年时间。其实我国的政治家,我不知道大家注意到没有,还是非常优秀的,他们对价值规律是懂的。这个定价权放出去这段时间出了好多事情,后来拿回来了,拿回来的过程中剧烈冲突、矛盾和冲突。好多人问我是特朗普要打贸易战吗?我说不是,换任何一个人上来,在2014年央地税合并、2015年放弃联汇之后,贸易战必然上,因为这涉及到定价权的争夺,这是价值规律决定的,它不以人的意志为转移,不是哪个人要干哪个人就不干,这是价值规律决定的。

当我们重新拿回来关于中国资产的定价权、关于中国商品的定价权的时候,冲突将不可避免。既有内部的资本利得与劳动所得的激烈冲突,又有源于国际上的资本利得与我国劳动所得、我国劳动者剩余价值的激烈冲突,这个冲突几乎不可调和。所以好多人说怎么看中美关系呢?大国博弈,老二不能起来。其实我们可以不用写小说的手法、不用文学的手法、思路,还是回到《资本论》的角度来看中美博弈就显得比较清楚了。所以我们把第四个部分叫新博弈论。

那么关于定价权的争夺,实际上就是主权的争夺。其实第一个部分,我觉得我们从十八大以后,就是以习主席为首的党中央在这个问题上做的是非常好的,不容易。2014年央地税合并,2015年取消联汇,我们实际上在争夺中国的财政主权和金融主权。在整个过程中虽然有诸多诸多的还需要进一步的完善,但整体的思路是明确的,这一点特别特别的重要。而且在整个的这个过程中,记着2016年特朗普当选之后,很快就开启了贸易战,就是我们在2014年、2015年刚干完这个事情,连三年的喘息时间都不给。

就发起了贸易战。然后贸易战之后一系列的动作,包括香港的黑衣暴、整个的南方的、西藏方向的压力,一系列的问题开始出现,但我们大体上顶住了。我们虽然并没有获得在国际意义上的资产与商品的定价权,但你们看到了我们在做什么?我们为什么要处理恒大?我们为什么要处理互联网企业的垄断?我们为什么要处理教培?所有一系列的动作关乎资本利得、关乎资产与商品的定价权,关乎对价值规律的认识和理解。

初始阶段,初始阶段我们的主要工作还不到争夺或者对决的过程,这需要一个历史过程,我们能做到的事情四个字:“均、输、平、准”。我们“均、输、平、准”的意思就是有些东西我们需要跨越时间周期,跨越时间周期。所以我一直在强调,过两天可能大家能看到我那个凤凰的专访了,那个上面我说了,美国人正在进行逆周期调节、调整,而中国进行跨周期调整,它的区别是什么呢?美国经济调整的最佳时期错过了,就是美国经济好转之前就必须得调。

但是由于政治上的因素,所以特朗普和拜登把好时间节点错过了,现在调就是美国经济将陷入衰退的时候Taper、缩表、加息,真的是不想活、不想活,自己要往坑里跳。逆周期调节是非常糟糕的一件事情,因为这不是战争,这不是什么大型的,这不应该进行逆周期调节的。那么中国为什么也不进行顺周期调节呢?美国在逆周期,中国也不是顺周期,中国是跨周期调节,中国必须得把以前存留的问题在阵痛中把它解决了。其中最核心的部分就是房地产定价——价值规律,把这儿解决了,让资本利得压下来,让劳动所得增上去。

在整个的博弈里边,时间是个非常重要的东西,跨周期调节就是需要将这个时间节点跨过去,而跨周期调节需要一个政治上的高度稳定,因为正好是十九大和二十大,就是这个跨周期调节的时间节点,差不多需要四年左右的时间,这个非常重要。所以当下的一系列事情我个人认为都非常好,做得很对、很好,跨周期调节要坚定不移,即便是经济增长失速、掉一点速度下来,也要完成跨周期调节;不要听那些个机构经济学家的建议,就是要积极、要宽松,不让跨周期调节,还是要资产定价,还是要资本逻辑,还是要资本决定,那是不可以的、那是不可以的。

“均输平准”只是一个治疗的过程,不是根治,根治的过程还是在立法层面,将来通过直接税来彻底解决问题。现在我们需要争取时间做一个基本的“均输平准”,这是一个古典思维、古典思维。在新博弈论里边,第一句话是“均输平准”,第二句话是“信用悖论”。“信用悖论”实际上就是需求决定,我刚才讲MMT——创造需求、需求决定信用,这个事情也可以在中国人手上把它改成供给决定信用,将这个悖论调过来,因为我们有跨周期调节能力、有对冲……

供给决定论或者是劳动决定论,要依靠的是什么呢?要依靠的必须是国家资本主义,全国一盘棋,倾举国之力与国际金融资本进行“均输平准”,一种平衡,来完成跨周期调节。这堂课可能讲到这个程度就有点儿议政的意思了,其实我根本就不想议政,一点儿议政的意思都没有,因为我国很快就面临着大规模的财政基建和金融基建了。财政基建是什么呢?就是税政改革,就是要建立直接税体系,那才是解决我国定价权……

那才是解决我国定价权——长治久安之策呀!那个时候我们才能由劳动决定和资本决定达至某种平衡,不是现在纯粹的资本平衡,我们现在只能跟它做对冲,而不能解决这个问题。所以财政基建将二十大之后进行吧,可以解决这个问题。另外一个是金融基建——金改,金改的核心实际上是人民币的问题,主权货币的国际地位的建立、如何建立,这个将来作为一个话题,我们可以稍微长一点的来讨论它。但非常幸运,就是我国保留了国家资本主义的一些非常优秀的部分。

我国在1978年改革开放之后,将国家资本主义、社会资本主义达至了某种条件的妥协与平衡,所以我国现在是国家资本主义、社会资本主义相融合的这样一个状况。好多人说起中国发展优势的时候,他们看到了中国国家资本主义的这个厉害的地方,它有优势的;他们也看到了我国社会资本主义的优势,它也有厉害的地方。纯粹的社会资本是无法对抗国际金融资本的,有了国家资本就可以对冲国际金融资本,纯粹的国家资本没有社会资本的积极性,也无法对抗国际金融资本。你们知道六十年代中国的那场激烈的思想运动,导致了我国达……

导致了我国达成今天这样的一个局面。所以我常说,不要轻易评价一场运动,不要轻易给它结论,它一定给一些人带来了很大的伤痛,但它未必没给历史打开一扇窗;不要轻易地评价一些历史人物,比如说赫鲁晓夫,难道五十年代批斯大林的赫鲁晓夫没有看到国家资本主义的问题吗?只是由于他个人的哲学高度和能力,以至于苏共整体上的哲学高度和能力不够,所以五十年代、六十年代他们没有完成我国领导人那么痛彻地思考,不但是思考……

而且进行了痛彻的、甚至惨烈的实验,它最后终于得出一个大家公允的、同意的一个结论,那就是改革开放,就是国家资本主义与社会资本主义相融合,创造出一个崭新的模式,这个模式既有极高的效率、效能,又可以保障国家与人民的某种程度上的安全。甚至,今天我们课程讲到这儿可以给结论了,甚至从某种意义上也高度契合了马克思和恩格斯所强调的那个价值规律,这个价值规律我们今天从自在走向自觉,再深刻地来对它进行一些认识。

讲到这儿,快接近尾声了,我再念一遍吧,马克思关于价值规律这段描述:“商品不只是当作商品来交换的,而是当作资本的产品来交换的(马克思写东西真有意思啊!),这些资本要求从剩余价值的总量中分到和他们各自的量成比例的一份儿。”你听马克思说什么了?这些资本要求从剩余价值的总量中分到和他们各自的量成比例的一部分,或者他们的量相同时要求分到相等的一个份额。我总是觉得马克思考问题的方式方法确实独特,而……

有的时候,我们不知道马克思、恩格斯在说什么,但是有时候你要一旦知道了,你就知道这两个德国人真的厉害、真的厉害!另外呢,人家说这个价值规律,说这个价值规律其实是不存在的,因为总商品、什么时候有个总商品呐?没有的嘛!什么时候可以一千年讨论一个价格的波动是围绕着价值规律来波动的呢?也没有嘛。背驰、背离是永恒的,而背驰的过程中所有的故事、悲欢离合就展开了。当然我们这些庸俗的人投资的那个道理正好在悖论里边也产生了,当然一个大国治理的道理也在这里边了。

好,今天的课就讲这么多。越讲到复杂的时候呢,越容易这个绕,有一点绕,绕就绕吧,慢慢就习惯了。原本是想说两句这个当下的情况,后来想了想,也不急吧,也不急。那么今天正好是苏联解体三十周年,在十周年的时候我写过一篇《苏联解体十年祭》\footnote{https://www.notion.so/60eb9cdade564b57b015e27f7e61df18},后来那篇文章我印象里应该是发到《信报财经月刊》吧,后来是好像被全都给删了,我后来又找也找不着。那个时候对苏联的认识,还到不了今天这个高度。

大体上我们可以给这个逻辑上梳理清楚了,就是列宁同志在面对纷纭复杂的政治变化的时候,来不及,来不及进行马克思、恩格斯的,马克思或恩格斯式的(方式的式),式的思考,来不及。他必须迅速解决当下的问题,所以他写了《国家与革命》。其实马克思一直在想写“国家与革命”,写这个国家与资本的关系,社会主义国家与资本的关系,但没写出来。他后来到了晚年的时候,他就写不出来,不是,也可能是……,总之没写出来。写作的人都知道,有的时候千言万语就是无法落笔。《国家与革命》这部书指导苏维埃的建立,

但是列宁忽略了一个重大问题,就是国家资本主义不是社会主义,因为社会那个主义或是社会没有充分的表达出来。马克思的本意让劳动所得与资本利得做出均衡,他这个国家资本主义起不到这个作用,或者是在特定的周期里边能起到一部分作用,但最终由于它滑向官僚垄断资本主义,官僚垄断资本主义、官僚垄断资本利得一定会侵害劳动所得的嘛,走向反面是必然逻辑。所以好多人不从这个角度来思考苏联亡党亡国的原因,会更多的是“是不是男儿”这个角度,或者是文学的角度,那是不可以的。

我们从苏联解体看到了什么呢?我们看到了两个方向都有问题。第一个方向,官僚垄断资本主义一定会走向历史的反面,为什么全体苏联人民会抛弃呢?是因为大家不喜欢,因为劳动所得与官僚资本利得之间形成的那种矛盾和冲突已然不可调和。所以结论不是谁好谁不好,谁是男的谁是女,这跟性别跟觉悟,跟这个没有什么关系。因为这一件事情最好的时间节点我说了,上个世纪六十年代,如果中国领导人和苏联领导人在这个问题上的哲学思考能同步的话,那么苏联的悲剧是可以避免的。当然这个历史不是这样的,历史不是这样的。

另外一个我们注意到的危险的倾向正在美国发生,就是金融垄断资本主义。金融垄断资本主义控制了财政主权、金融主权和当年的苏联官僚垄断资本主义,你看上去不一样,本质是相同的。它开始将金融资本利得无限放大,去吞噬美国劳动者的劳动所得,吞噬全世界劳动者的劳动所得。虽然他有那么那么多的文学的、政治学的解释,民主、自由、人权等等,但本质上的剥削肉眼可见,所以美国的当下的金融垄断资本主义没有未来。悲惨的是,我们还看不到他有改变的机会

虽然美国的左翼,民主党里边的一些左翼也提出了一系列的思考,但显然美国的左翼特别是民主党内部的左翼——桑德斯、沃伦,他们已无能力在可预见的未来获得执政权来改变美国当下的现实。所以我想起了一首歌:“一出悲剧正上演……”。看到前苏联,看到当下的美国,其实我们更多的是思考我们自己、我们可爱的祖国,我们如何跨越官僚垄断资本主义和金融垄断资本主义这个陷阱,我们走向一个国家资本主义、社会资本主义高度融合、均衡的……

好吧,今天这堂课,价值规律这堂课算是送给大家的节日礼物吧。我在准备元旦那堂课了,因为我们下一堂课是元旦,是聊天,又是新年伊始,那次聊天挺重要的。不管怎样吧,不管怎样,祝大家圣诞快乐、保重身体、一切都要好好的、样样好。那么明天下午三点钟我们再见面、再讨论,再研究一些其他的问题。好,谢谢大家。

\section{阶级}

大家好,今天是2022年的1月8号,是辛丑年的十二月初六。辛丑年的最后一爻果然凶险——香港第五波疫情爆发。昨天晚上六点开始,所有的食肆、酒吧,所有的公共场所全部关闭,相当于西方的宵禁吧,可能会形成一次比较麻烦的疫情的爆发,现在还不知道,大家都在观察。我一早来到写字楼,准备今天的课程,今天是第二十讲,《资本论》第二十讲——阶级。

大家好,今天是2022年的1月8号,辛丑年十二月初六。昨夜赶稿子,很多年没有这样的连夜写东西了,写超过一万字的文章,还是挺兴奋的。早晨起来,大概六七点钟算是封笔,结束了,一高兴呢就写了两句“卷帘金光轻抚面,拍落一身五更寒”。我自己有时候觉得生活其实还是挺美好的。

反正也睡不了觉了,所以我就背着书包步行到公司来,准备今天的这堂课。我每次一到星期六就特别的兴奋,有一种幸福感,我以你们为借口,允许我在星期六喝一瓶无糖的可乐。我每天早上起来背上书包出门,想着今天可以喝到一瓶可乐,其实我特别的高兴。当然,可乐是个借口,我其实很感激大家,因为《资本论》这个课又让我沉浸在一种状态,我每一次备课和准备这堂课,又学到好多好多知识。

出门前夫人问我:“如果周恩来活着的话,他会如何评价当下的中国?”。我说:“这是个好问题,容我在路上去想想看吧”。一晃周总理都已经离开了四十六年,那代人会怎样看时下的中国呢?如他们还在,他们想说一些什么呢?亦或者,我们可以在这个课堂上延续我们前辈的思考,延续他们对未来的一种渴望。

好,不浪费时间去抒情了。我们今天进入《资本论》的第二十讲——阶级。其实《资本论》讲到现在,确实是进入到最后阶段,我也不敢说它是高潮,但是我不知道听课的人感觉如何,但是讲课的人确实有一点高潮了,其实我挺激动的。阶级这堂课是非常难准备的、难讲的。为什么?因为马克思写《资本论》只写了一卷,二卷和三卷是由恩格斯编纂的,其中“阶级”这一章是第三卷的第五十二章。

可以说三卷《资本论》的最后一章,结尾这一章是“阶级”,这是不是马克思的本意,我真不知道,但我能看出这是恩格斯的深思熟虑。恩格斯作为一个伟大的导师,他对《资本论》最终的落脚点放在了这两个字上——“阶级”。但你读这一章的时候,你会感慨万千,原因是这一章只有一页半纸,我数了,大约在六百字左右,恩格斯在这一章结束的时候用括弧加上了六个字“手稿至此中断”。其实我看到这儿的时候,扼腕……

马克思不想谈阶级吗?他只写了六百个字,他当然要谈,他想谈,但他没谈,他没谈吗?从马克思开始写作,从《共产党宣言》到《资本论》,哪一章写的不是阶级啊?但最后归纳、总结阶级这一章的时候,却只有这区区六百字,这六百字完全没有说我们理解的那个该说的阶级。这是一篇残章、残篇,也许这个残章和残篇留给了我们继续创作《资本论》的这样一个巨大的空间,所以我们今天来讲一讲试试看。

我们每一个人一出生就已经烙印了种族、种姓、血统,一句话——我们是带着阶级来到这个世界上的,阶级是我们出生那一天就带着的烙印。但我们可以有选择的机会吗?以前可能有人跟你讲命运,所以就没得选。但伟大的马克思告诉我们,我们还有权利可以选择阶级,所以马克思写下了《资本论》。《资本论》是无产阶级的圣经。

说到阶级,其实阶级问题是一个非常复杂的问题。就社会意义而言,阶级是一个伦理问题、是一个学理问题、是一个法理问题、是一个管理问题;就个人而言,阶级是一个物理问题,阶级是一个生理问题,阶级又是一个心理问题。如果马克思曾经试图解放无产阶级的话,那么他要解放的不仅仅是一种物质、物理、生理,也要完成最后心理和精神的解放,而心理和精神的解放也许不能在《资本论》中完成了。

亦或者,最终完成心理和精神的解放,要依靠我们自己。我们讲了王阳明先生的“心学”,重建我们的主体性、适应性和创造性。其实,心理的解放——心里挣脱对阶级的束缚,可能才是我们一个一个人心理解放的结果,才是我们这个民族和这个国家完成一个根本性的解放。我们在半殖民地之后经历了,如果从1840年到2040年,我们大约要经历两百年完成一次民族的、国家的彻底的解放,当然我们首先必须完成个人的……

当然我们首先必须完成我们个人的心理的和精神的解放。其实包括我本人在内,包括我身边的许许多多的人,我们一开口、一写文章、一做事,我们就能看到阶级的烙印。好多人有基于阶级的自卑,基于阶级的……也有很多人有基于阶级的傲慢。有时候这种自卑和傲慢会表达为伦理、学理、法理和治理的各个层面,我们不敏感而已啊。

我无法完成对《马克思和恩格斯全集》中对阶级的叙述和这个整理。我尝试着将《阶级》这一章做五个方面的概述。今天这堂课其实挺重要的,我只是觉得我准备得不够充分,我怕讲不好啊。阶级是如何形成的呢?阶级,我们通常看、讨论阶级会讨论这样三个问题,第一个问题就是血统。我说了我们一出生就打着烙印,它有种族的、种姓的、宗族的、家族的,总之我们一出生就烙上了阶级的烙印。

另外还有出于对职业的一种阶级的划分,比如说我们通常使用的统治阶级、被统治阶级。人分三六九等,自古就有基于职业的阶级划分。通常在西方国家,最高的是传教士、是神学士,然后是贵族、是官吏,然后是有产者,然后是普通老百姓。当然这里边是在同种姓之内的一个定义,还有涉及到种姓、什么其他的问题,那就更复杂了。但是职业算是一个划定阶级的一个重要的方法。第三种,对阶级的区分,

第三种对阶级的区分就是资产。其实《资本论》要讨论的主要是这个部分,所以就有了无产阶级和资产阶级。不谈阶级,《资本论》没有落脚点。因为从《共产党宣言》到《资本论》,它的结论是无产者联合起来进行无产阶级的革命与斗争,最终建立无产阶级专政的国家,实现无产阶级的彻底的解放。当然马克思不写第三卷我是理解的,因为他对最后的这种结论,无产阶级革命与斗争的结果——无产阶级专政的国家,他产生了深深的怀疑。他的怀疑当然是有道理的。

事实证明,列宁仓促之下写下的《国家与革命》,并以此为理论依据建立的第一个社会主义国家——苏维埃政权,其实不是马克思本意上的无产阶级专政的国家。因为没有想到更好的办法,如何让人民真正地拥有财产,就是无产阶级怎样变成有产阶级。无产阶级专政的国家——苏维埃出现之后变成了国家资本主义,而国家资本主义天然地走向了官僚垄断资本主义。这个问题,毛泽东是60年代就已经发现了的问题。

毛泽东曾经用了十年时间尝试解决这个问题。虽然毛泽东没有亲手将这个问题解决掉,而且那十年至今仍然有巨大的争议。因为他的解决方法可能并不符合中国的当时的国情,所以那次激烈的实验以失败而告结束,然而那一次激烈的实验它也有它深远的历史意义,它为以后中国寻找另一条解决国家资本主义道路提供了正反两个方面的经验和教训。以至于到了上个世纪90年代,苏联解体了,而我国挺过来了。

这样说吧,因为我昨天在熬夜写那篇长文里边再次谈到了《资本论》:我说整部《资本论》要义在哪里?要义是马克思希望能够最终取得劳动所得与资本利得的平衡。这不是终极目标,终极目标是共产主义,就是没有资本。但中期目标,我说的不是终极,是中期目标,必须获得劳动所得与资本利得的平衡。这是列宁《国家与革命》没有想透的问题,亦或者在上个世纪60年代,无论是苏联的社会主义者和中国的社会主义者……

虽经过了那么长时间的论战与思考,也终于没有找到标准的答案。上个世纪80年代开始,中国尝试用另外一种方法来解决劳动所得与资本利得的关系,所以在国家资本主义里边引入了社会资本与国际资本,形成了混合资本主义。混合资本主义有社会主义意义吗?有的、有的、有的。所以中国在混合了十年之后,躲过了那一场苏联解体的那一场危机。因为我们在混合的过程中释放了中国巨大的生产力潜能,使中国经济迅速增长,解决了当时的问题。

然而阶级的问题解决了吗?没有。今天的第一部分应该给阶级一个定义,但我觉得阶级这个定义太难了。你不能说是阶层,你也不能说是……不太好说。所以在第一段里边,我只是讲了一下子,大概血统形成的、职业形成和资产形成的。完整的定义,大家去思考吧,我想不一定每一次由我来给一个定义。我不大同意我所见到的那些定义,我觉得都不是特别的周严。第二个部分我们简单念叨了,

我们简单念叨了阶级的生成、血统的生成,这种资历的生成,包括教学、履历的生成,包括最后的物质决定论就是资产决定论。这里边我想说的是,在阶级生成的这个过程中,蕴含了我们对阶级解放的思路。阶级解放的思路,如果是血统的东西,那么我们就要在伦理上、学理上、法理上、治理上面,将这种出于种族、种姓,出于等等,这个东西,慢慢的把它抛弃。

那么,出于这种后天的资历而形成的这种阶级当如何理解呢?我个人认为要正视、要接纳。我们反对出生的不平等,但我们要承认努力、承认劳动创造价值,承认这个努力,它形成的这种阶级要如何面对呢?我讲在社会主义阶段是必须正视的,共产主义阶段有可能也不需要正视这个问题,大家都是神嘛,就没有阶级问题。但在社会主义阶段,资历这种事情我们要承认、要接纳。

至于资产决定论和物质决定论,我想它是一种事实,它是一种事实存在,劳动创造价值是有差异的,所以物质决定或资产决定是一个事实存在,我们不需要否定这个事实,但是,我们要给它划定边际、划定底线。划定边际、划定底线,使得每一个劳动者不因他的劳动的累积,而再次形成血统的继承,形成新的食利阶级,这是我们的制度文明的要义所在。

我想,中华民族如果真的是一个伟大的民族,他就应该能够翻越阶级这道门槛,翻越阶级这道门槛,真的是非常重要的。我们能不能砸烂那个反动的血统论,不让那个血统的阶级在中国人身上出现?我们不允许那种阶级固化,不允许那种社会固化,不允许社会的严重的固化而内卷、而导致青年人的躺平,这一件事情我们必须做到。第二个部分,我们虽然承认个人努力的成果,承认它。

然而,每一个劳动者的光荣给予肯定,不代表对劳动的结果进行千秋万代的继承,我们不允许新的劳动者再次形成血统论,我们不允许这种阶级的轮回在社会主义的新时期、在21世纪的20年代,仍然让它继续的存在、甚至发酵和深化,甚至形成新的食利阶级,我们不允许这样的事情发生。至于第三个,物质决定论或者资产决定论,我们必须划定边界,通过伟大的制度建设——税政……

将这种资产决定论,终结于个人的有限的生命周期中,而不使它形成阶级的轮回,我想这也应该是马克思、恩格斯的本意。当然我所说的只是社会治理层面的逻辑,我昨天晚上写这篇东西用了阶级的这个观点,我们只是在讨论直接税立法、讨论税政改革的当中,如何来体现阶级的平等,人民的平权。

我说,这是社会治理的层面。但我同时想说,我们每一个人的心理层面、物理的约束或者是社会治理的约束,有太多地方需要改善,但首先需要改善的是我们每一个人心理上的阶级约束或者是阶级压迫。好多朋友说,因为我们的出身是平民,或者是我们没有那么多的资产,所以我们有心理上的阶级,出于阶级的这种自卑,这是一个心理上的障碍。那么那些出生在豪门、贵族、富豪家的孩子们的那种阶级的优越感,是一种压迫吗?

那当然也是一种疾病、一种不健康。今天在想到总理的时候,我在想,中国的那一代的伟大的革命家、无产阶级的先锋队、伟大的共产主义战士,多数都是出身于有产阶级啊,他们并没因为自己的出身的傲娇、出身的好,而放弃对人民的、对国家的责任,什么是伟大的无产阶级革命家,什么是共产主义战士呢?所以我们能看到他们有着真正的悲悯。

我曾经说过,这是一种无缘大慈、同体大悲,他不单纯是对黎民和苍生的悲悯,对无产阶级的悲悯,也出于对整个生命的悲悯、同情,所以他们为这件事情去做了努力,他们才是我们心目中真正的伟人,甚至,他们就是我们心中的神。至于那些基于种族、种姓、血统、财富、地位而形成的大牛们,例如薇娅,当然也例如那些股神们、超人们,他们有这份悲悯吗?

都那样了,还琢磨着偷税、漏税呢。在我看来,无论是血统决定、资历决定、还是物质决定,即便是站在高处,很多人依旧是有病的,是有阶级心理障碍的,是有病的,所以有病就得治,心理健康才能谈社会治理呀。当一个民族在心理上是如此的强大,它的主体性、适应性和创造性全部生发出来,还有什么力量,可以阻挡这样的一个伟大民族的复兴与崛起呢?对阶级的认识非常非常的重要。

说到这里呢,讲两个小故事,我个人的故事,一个故事发生在深圳,一个故事发生在香港,都是在二十年前的事情了。那时候我还年轻吧,我懂、我懂,但我还未必能超越我心理上的阶级障碍,所以我还是,在那个时候,这两件事情还是深深触动了我,留下了挺深的痕迹。一件事情是在深圳,周末去喝酒,某个二代,正国级的孩子、正国级的公子。

他很开心,请我喝酒,他就拿着一沓(约略是一万块钱一沓的那种)人民币站在大门口,谁敬卢先生一杯酒就给一百块钱。那场子里边那么多的工作人员,还有……,所以就绕着柱子排队,一人上来喝一杯。我知道、我知道也许不是恶意,但我感受到了一种东西,其实我包里边也有一沓,但我不会这样做。后来我离开,以后不想再和他们喝酒。

还有一次,在回归前,在香港的君悦酒店,某富豪(已经离世了),开家宴,十桌,我算是嘉宾吧。在酒桌上他的孩子跟我一桌,然后趴在我耳边说了一大堆的英文,我没听懂,但我出于礼貌在点头。然后,舞池当中那个十桌饭中间有个舞池,舞池有个小乐队,还有一个约略身高两米的外国美女在那里边唱歌。他们请那个美女,

他们请那个美女呢与我跳一支探戈。我没听太明白,我点了头,点了头就应该下去跳这一曲。两米高,我这个身高他就不太合适;另外我那时候初到香港,衣服----我那个破破烂烂的西装,和我那双不是那么合适的皮鞋,在那一曲探戈里边那洋相百出,整个十桌人笑的前仰后合,我也知道他们很开心,但我咬着牙把一支曲子跳完,当然后来这个富豪过来道了歉。其实我当时的感受是非常的……

那个时候,我其实有一些基于阶级上的一些心理上的一些问题的,后来就不会这样了,我甚至都不会再往心里边去,好多人跟我有时候在网上打笔墨官司,这个问题是我根本不往心里去的。但那个时候我突然意识到好多的问题。人生你总是要经历诸多事情的嘛,一档子一档子的事情,其实我们自己有我们自己在心理上的这种阶级上的、这种心理上的关于阶级的障碍。其实他们那些人,

有的时候不仅仅是障碍,甚至是病态。所以N年过去以后,当大家再见到,他们会感到惭愧,不应该,一再的解释、道歉什么的,其实已经过去了,已经过去了,只是成长的过程中有的时候都要经历这个过程。我讲我的故事,是想讲给你们听,你碰到的这种事情,关乎阶级的敏锐、敏感,其实没那么重要,可以轻轻放下,因为它对你无碍、没有障碍,其他人可能出于病态的表达,就让他表达吧。

我们,看到了阶级关于物理、关于生理、关于心理上的问题,我们要走出来,走出来之后,我们都走出来了,一个民族走出来了,我们就可以开始伦理、学理、法理和管理上的改造与建设。其实,1949年,我们大体上完成了革命,三次土地革命完成了土改、完成了城市工商业改造,我们在社会制度层面进行了阶级的平权,也可以叫做一种物质上的解放。

然而,这解放就算结束了吗?所以我一再的强调,请大家去看美国那部电影《紫色》,那里边妹妹对姐姐说的话很重要,总统的一纸《废奴宣言》,只是从身体上解放了你,你不再是人身依附的奴隶了,但是心理上的解放仍旧是漫长的,心理上的这种对自己阶级的否定、平权的过程,是一个漫长而艰辛的过程,这不是美国黑人要解决的问题,这也是半殖民地、半封建社会,中华民族需要不断地解决的问题。

我昨天在文章里说了,革命可以解决平权——阶级平权的问题吗?No!中国历史上无数次的农民起义、无数次的均田,最后呢?轮回了,又兼并了,更残酷!革命可以解决问题吗?苏维埃建立之后,短短的七十年,就又退回去了。终极的解决,有没有方法呢?有的。所以革命是一次性解决,改革则是不间断的解决。

好多人说:卢先生啊,你一直在提直接税、直接税,那个直接税是针对财产的课税,是直接针对财产的课税,而不是针对劳动的课税。那是要解决什么问题呢?当然是要解决阶级平权的问题呀,解决血统的问题呀,难道遗产税不是解决血统的问题吗?难道所有的针对资产类的课税,不是为了转移支付吗?不是为了解决平权问题吗?我们承认劳动创造价值,我们承认阶层甚至阶级的存在,但是我们需要通过制定来划定底线、划定边际,使我们这个民族,

使我们这个民族,不再陷入革命与兼并的轮回,不再陷入那种阶级的轮回,不要再重复已经五百年的殖民历史了。其实,讨论到阶级的时候,我一直在想,马克思为什么只写这一页半——六百字?可能马克思思考的,比我们今天思考的更深、更远,但是马克思可能无法用周严的语言来说出,否则马克思本人可能在,在19世纪末就成为了反革命。

我们今天讲的第三个部分是关于社会分配的,社会分配是解决阶级差异或者是解决阶级轮回的一种和平的方法。实际上我一直在强调税政改革,税政改革特别是直接税立法将成为中国的大宪章运动,它才是中国的光荣革命,它也必然是中华民族复兴、登顶人类文明历史的序曲。这里边的重要意义,我无法用朴素的语言来完成,我想,在欧洲出现了《共产党宣言》,出现了《人权宣言》,

在欧洲曾经出现《人权宣言》《独立宣言》《共产党宣言》,那么中国人要不要有个“宣言”呢?要的。我没取好那么好的名字,我想应该是一个“平权宣言”,这是一个关于阶级平权的宣言。这不是简简单单的新儒家,也不是新马克思主义,也不是简简单单的思孟(儒家思孟)的复活或者是复兴,它是我们综合了人类一切文明走进新时代的一个标准的、标准的制度配置或者是制度建设。

如果说马克思列宁主义、毛泽东思想是关于解放的学说;那么我们现在需要的是平权的学说。一个平权的学说才能将马克思《资本论》最后一章第三卷第五十二章这个“阶级”写完它。这章的后半段,不是,它只是六百字,这章全部写完,不是不应该称之为阶级,而应该称之为阶级平权。马克思没有见到真正的社会主义国家建立,所以他可能也没在财政部工作,所以他可能对财政和税政也没那么熟悉,所以他可能没考虑到通过税政、通过直接税立法来解决,

没有考虑通过财政改革、税政改革来解决阶级平权的问题。是的,是的,直接税的本质就是不允许任何人凭借权力与资本偷盗属于人民的地租与剩余价值。整理文字的这个朋友可能会辛苦一些,因为今天这里边有一些话挺重的。直接税的本质是不劳动者不得食,直接税的本质是劳动光荣、食利可耻。

直接税的本质是要求一个伟大的国家建立真正的人权、民主、平等与自由的完整的制度安排,其中这个制度安排的立足点就是直接税立法。我们在很多时候要揭穿境内外金融资本给我们灌输的那些宿命论的东西,那些基于他们在境外受教育而形成的某种阶级或者是学问上的优势,更加的不应该让食利者光荣。

我其实对我国的教育、学术、传媒意见很大,我们应该对他们进行系统的批评与教育;我们怎么能够允许我们的教育、学术、传媒弘扬海外金融资本?弘扬种族、种姓、阶级这些东西,血统的东西?不要这样。好多人号称左翼,但是他总是在血统里边出不来。不要这样、不要这样,我们应有更宏大、更高远的事业;有哲学的高度、有历史的纵深,来重新解决这个问题。我刚才将直接税的本质概述为四条,等到整理出文字这个时候,大家仔细去看、去思考和……

今天的第四个部分想简单说一下社会主义的阶级与阶级斗争。我们必须说,在社会主义依旧存在阶级,共产主义没有了,社会主义依旧存在阶级,存在着阶级斗争。如果你说“告别革命”,我觉得可能有一定道理;如果你说没有斗争,我觉得问题非常之严重。如果我们接受种族、种姓、血统,接受这一切,就是我们生而带来的阶级的话,那么人类岂不又重返了黑暗,你愿意回到印度的种姓制度中吗?

承认社会主义存在着阶级与阶级斗争是《资本论》最后一章所要说清楚的,对中国而言,这里边有两重的含义。第一重含义就是我们还是要坚定不移地反对后殖民主义压迫,这件事情没完。我们争取在2049年完结它,走出后殖民主义压迫,并且历史性地终结人类长达五百年的殖民历史。这件事情理应由中华民族来完成,完成五百年殖民历史,让它终结,也完成对后殖民主义的批判与清理。

第二个部分,我上一堂课发给了北京的朋友们,有同志说这两座大山,这个第二座大山后殖民主义是好办的,第一座大山是比较难办的、也比较敏感,就是社会主义的阶级与阶级斗争里边的第二个问题就是官僚垄断资本主义的问题。好多朋友说不要用官僚垄断资本主义好不好?我说请你给我一个比它更能概述国家资本主义导入官僚垄断资本主义,比这个更好的、更清晰、更准确描述的话语,无论是概念、定义和逻辑,请再给我。我其实也不太想用这样的词语。

好多朋友可能都知道,我的微博、头条关180天,现在B站也上不去,哪里哪里都关了。惹麻烦的可能就是这句话——官僚垄断资本主义。我也可以不说的,好吧;但我们怎能不说呢?我们不可以把这些问题留给子孙。我想,若四十六年前走了的周总理在的话,他也会认同我的看法。

今天讲阶级这堂课的最后一个部分呢,我想讲一讲文明的标配。最近美国人在逐渐地丧失道德制高点,他们在丧失道德制高点。丧失道德制高点之后,就是在伦理的解释上越来越不能自圆其说了。到底什么是文明的标配?文明的标配不是一种形式上的民主,而是一种真正可以解决阶级平权的制度设计。

美国解决了阶级平权问题吗?没有。美国解决了美国与其他国家的种族、种姓的问题吗?没有。美国的美国中心主义难道不就是后殖民主义吗?美国滥用战争与制裁难道不就是后殖民主义的一种精准的表达吗?美国对外的后殖民主义、对内的阶级压迫难道还不够严重吗?难道那不是美国最现实、最严苛的问题吗?

当下的美国的思想家、美国的政治家其实正在逐渐地丧失他们在伦理上的高度。美国在疫情上的表现不是美国人在伦理、学理、法理和治理上优越的表现,它是上苍,我不愿意用上帝这个词,它是上苍、是神对他们的启发;我也不愿意说那是惩罚,都是老百姓,那是一次暗示,是一次揭示,是希望你懂得错了,而不要将这个错误强加给其他人。这个错误难道仅仅是……

这个错误难道仅仅是一个政治上的问题吗?阶级不能平权会导致经济上的不可避免地衰落。要知道一旦阶级固化、阶层固化、资产归边,资产归边向少数人聚集,它的结果是整个生产率急剧的下降,最终会导致多数人不满、反抗、革命。难道不是吗?难道美国距离他的经济危机还远吗?我们已经看到了不可避免的政治危机,难道这个政治危机不会导致经济危机吗?

当下的美国的思想家、政治家竟然、竟然能攒出一套理论,想用印钱的方法——MMT来解决眼前的现实的政治矛盾和经济矛盾,能不能行?曾经有成功的经验,比如说1985年的平成战败,比如说1991年的苏联解体,为他们的操作提供了财产、财富,使他们渡过了他们那个时候的危机。今天还可以再来一遍吗?

下堂课我们是聊天,聊天要聊柯立芝繁荣了。因为上堂课后殖民主义把柯立芝繁荣又推了两周,下周讲柯立芝繁荣,讲柯立芝繁荣是为了讲上个世纪的那场危机、大危机。然后我们对即将到来的美国这场危机提前做一个预判吧。因为新的一年开始了,虽说是既济,但既济这个卦,是先甜后苦、虎头蛇尾,所以我们可能对未来还是要有一个冷静的判断。

我想在这里说:文明的标配、文明的标准必须是阶级平等。文明的标配就是阶级平权的制度性安排,而这种制度性安排它最经典的特征就是非常优良的、适合本国国情的直接税税政的体系。直接税立法和直接税的实施,就是直接税税政的体系,这一套税政体系才是一个国家文明的标配。不要拿你那个选举政治说话,没有这个标配,其他你所陈述的民主,

你所陈述的民主、自由、人权毫无意义、毫无意义。因为有了这个标准配置——直接税立法的标准配置,其他的附加的东西才有意义。好多朋友说,特别是左翼的朋友又来给我杠精,又来说因果关系。我已经解释了无数遍,我也不想再解释,800年前大宪章运动是税政改革引发了宪政文明,是税政改革引发宪政文明,不是先有了君主立宪,才有了直接税立法,好不好?这么点事就说了这么长时间,怎么就说不明白呢?

我甚至这样说:新时代,新时代就是中国经历了1949年的这个解放,经历了那个“十年”,走到今天,我们如果再次解放的话,它的本质是什么?它的本质非常清晰和明了,就是直接税立法,直接税立法,而非其他,请不要再讲其他。当然直接税立法是多数人对少数人的某种意义上的约束和剥夺,甚至是中国人对外国人的约束或者是剥夺。在某些时候这个少数人拥有一定的话语权、甚至立法权、甚至司法权、甚至行政权,

有的时候,文明的进步难就难在这里。然而中国当代仍然有思考者,虽然这些思考者是如此的孱弱,动不动就给关180天,但是他总归是有的,而且很多很多如我一样的人在思考。这个思考一旦形成全民共识,将排山倒海;它不再是革命,它是温和的、和平的、强有力的历史性的改革和进步。有空大家可以再去读我写的那一篇老文章——一万字的《掠过弗莱堡》。

经历了第一次世界大战、第二次世界大战,两战之后,终于将容克地主——一个德国非常重要的一个阶级消灭了,敉平了。艾哈德得以在废墟上建立了当下的德国的社会市场经济,使德国在人类社会主义道路上走得比较的平稳,也比较靠前。好多朋友说,你一说社会主义东线和西线,那你非要认为他们搞的是社会主义。请相信无论是英美还是欧洲,都进行了非常深刻的社会主义改造,虽然它叫资本主义。美国主要是威尔逊,

美国主要是威尔逊,德国主要是艾哈德,他们都进行了社会主义改造,而且改造的这个成果是中国人应该学习和借鉴的。我们既要看到东方社会主义革命的重要意义,也要学习西方社会主义改造的重要意义。因为我们现在不再需要一场轰轰烈烈地革命,我们只是需要在平和、安宁之中完成这一次伟大的制度建设——直接税立法,这是我们的“大宪章运动”,这是我们的“光荣革命”,我们由此而让我们伟大的祖国走向、迈入新时代,我相信它将给我们伟大的祖国以……

以新的经济增长的动能,确保我国在未来30年甚至50年之内仍然高速增长。并且我国有机会为全人类创造出崭新的制度模式,甚至有机会解决人类所面临的那些超越了人类自身阶级平权的外部性问题,比如说环境问题等等、等等、等等可持续发展问题。讲了阶级,讲了阶级平权,其实这有一点像中国历史上的变法——废井田、开阡陌、奖励军功,让每一个人有机会发挥他的能力,使我们的国家更加美好。

阶级这堂课就讲这么多吧。仓促之间,所以我想可能我那篇文章,我的文章写完了,但我今天不能给你们,那篇文章发了之后,可能要到春节之后你们才能看到,因为我其实觉得那文章写得挺有意思的,虽然写得未必就是完美,但昨天晚上写得我热血沸腾的。现在的经济状况的确是让人感到有一丝丝的忧虑,有一丝丝的忧虑。

我个人对我国的经济的看法还是持有乐观态度的,我觉得我国这一代领导人非常优秀,对政治、对经济的认识精准而不是不精准,处理得恰当而不是不恰当。我们要做的事情不是中国经济增长要五点几、要六点几,不是,一点几我看也很好。我们要做的事情是安安静静、安安全全的走过这段路,走过直接税立法这段艰辛的历程,重新激发中国的经济增长的动能。

我们不必理会外边的事情,因为我们已经可以预见那场风暴——无论是基于政治上的、还是基于经济上的、甚至是基于军事上的风暴将会到来,我们只是安安静静做好我们自己,做好充分的准备就可以了。在这个时候,速度是没有意义的。好多朋友说,比如说应该去挽救这个产业挽救那个产业,如何激发什么土地财政之类的,我笑了,因为多数的,这样说是不是有点武断?多数的专家和学者并不真正懂得财政。

马克思说得很透彻,资本积累的三个源泉:第一源泉是殖民,第二源泉是地租,第三源泉是剩余价值。中国没有第一源泉,我们不能殖民;中国只有两个源泉,一个是地租,一个是剩余价值。中国为什么可以在改革开放的这后边、后段创造经济增长的奇迹,是因为我们完成了工业化,第一代人完成了工业化,完成了工业化就要开始城市化,城市化一旦开始,我们就可以收取地租了。地租为中国带来了巨量的资本、天量的资本,它使得中国经济出现了崭新的动能。所谓土地财政、所谓的分灶吃饭里的地方……

所谓的土地财政、所谓分灶地方财政,实际上是地租套利,只不过在一个特定的时期,这个地租套利不均衡。因为地租是什么呢?是土改的红利,是人民的财产。地方政府进行地租套利,可以、可以说没有太大错误吧。但是如果伴随着其他的比如说境外资本、比如说机构、比如说个人一同套利,甚至是主要套利者,你以为我会认为这件事对吗?我是学财政出身的,当然地租套利不均衡,有问题呀!

我为什么写长文?地租套利存在三个问题。第一个是公不公平,为什么让那些个人,甚至让境外资本进来进行地租套利,然后去捐给哈佛,这事对不对啊?第一个问题,地租套利不均衡。第二,地租套利是有边际的,地租套利是有边际的,不能无限套利,它跟工业化、城市化是有一个逻辑关系的,不可能永远套利的。你这样套下去,无止境、无边际的套利会出事的,因为它是一个期权,你全都提前套了,

你把子孙饭都吃光了,将来中国的老百姓的消费在哪里?中华人民共和国经济增长的动能在哪里?不要耍小聪明,不要以为土地财政创造什么奇迹,马克思说清楚了,那就是个地租套利。在特定历史阶段,财政出现困难、出现问题,采用或者放纵地方政府做一段时间地租套利是可以的,但不可以如此下去,更不可以将这件事情美化乃至于神圣化,这是胡闹啊!不读书怎么能行呢?第三个,地租套利的本质是套劳动者剩余啊!

是劳动者未来的剩余价值,套了。这件事情要仔细的想一想哦,这是供给侧还是需求侧呀?消灭未来的需求侧造成今日之供给,造成今日的财富的造富运动,造成高净值人群的百万亿浮财,于心何忍呐!我们今天的课讲得是阶级啊!是阶级平权啊!不可以再进行阶级压迫,不可以再进行这样的从土改到兼并的轮回,不可以啊!我只是简单说几句吧,原本是想说一下经济又激动起来,好吧,将来在我那篇长文章里边,你们会看到我完整的想法,今天就说这么多吧。

香港第五波疫情暴发,大家在封城,大家其实很紧张,深圳也发生了,国内的情形还是比较严重,务必请大家多多的保护好自己,尽量地不给家人、不给单位、不给国家添麻烦。清零至少到今天来看,都是一个非常正确的决定。在我们还不知道疫情会走向哪里的时候,我们就坚守吧,坚守是非常重要的,坚持最后一刻,直到胜利。我们明天下午3点钟再见,有些话没说到,明天再补吧。好,谢谢大家。

\section{地租套利与土地财政、说几句市场}

大家好,我试一下麦。今天是2022年的1月22日,辛丑年十二月二十,还有10天过年。今天是《资本论》第二十一讲:地租套利与土地财政。这第二十一讲又是特别重要的一节。好,我试一下麦。一会儿三点钟,我们准时开始,一会儿见。

大家好,今天是2022年的1月22日,辛丑年十二月二十,还有10天过年。又高兴又感到有些许的忧虑,高兴的是再有10天,辛丑年就结束了,最凶险这一段就过去了。有一些忧虑是还真的比较凶险,这个情形倒也大体上类似于我们前期的估计和判断。今天我们念叨念叨市场,然后今天的主题是《资本论》第二十一讲:地租套利与土地财政。

我这一周的周一,周一还是周二?总之是这一周,周一或者是周二我记不得了,所以正式发出对美股的预警。此前我正式发出对数字货币和元宇宙的预警。这个事情对平台上的朋友可能影响不大,平台上也有一些朋友在炒美股、在玩数字货币。但我知道人数不多、量不大,但是香港的在本港的朋友多数都沉浸在这里边。我记得我说过了,就是每一次发生事情之前,我一定会预警一次。

{\kaishu 2022.1.18 卢先生认为美股到了顶部。https://weibo.com/1245732825/LbmDAb2la

读者也需要注意

2021.4.25也认为美股见顶。https://weibo.com/1245732825/Kcxtd6nR5

2020.4.11也认为美股见顶。https://weibo.com/1245732825/ICNOXEtml

2019.11.5也认为美股见顶。https://weibo.com/1245732825/IeIrFv3id

2019.10.2也认为美股见顶。https://weibo.com/1245732825/I9GOrfvax
}

香港这边的情况不是很理想,币圈的这些朋友多数、我指多数大概是六七成退出来了。这个比特币已经跌到三万多了,从六万多跌到三万多,元宇宙里边的一些资产也是在狂降。我在关于比特币和元宇宙的事情我已经谈过三次了,几乎每个月都谈一次。我也知道在香港的这些小年轻们可能不愿意听我的,但一些非常要好的朋友我是非常简单、明确地跟他们讲要退出来,所以他们也就退出来了。

美股还好吧,因为我身边这些炒作美股的基本上都是专业人士,都是基金经理,或者是有些人是行家了。当然这回损失也是比较惨烈的,几乎是股、债、汇三杀。所以好多朋友、香港的朋友还是对我有一些意见,他们就说、就是预警没留出来时间,他们认为应该留出三天的时间。我也不好说什么,因为我自己认为如果是个人的户头的处理不需要这么长时间,当然你要处理一个基金的话可能需要一个时间吧。

另外作为朋友嘛,答应了的事情就会履行承诺,就是我会在可能有风险的时候提前做出一个预警。因为这个事情,2015年我就做过,2015年那次股灾之前我是提前一个月,后来到了出事儿的前一个星期,连续的做出预警,好多朋友还是躲过那一劫。美股这一次是非常明确,所以我们也做了非常清晰而准确的,因为你不能太早,太早它可能再去摸高点呢。所以好多朋友说这是不是有点太……,其实有的时候没有办法,因为我也得看图啊!所以反正这一件事只能这样。

美国股市虽然我们都知道它需要调,甚至走到去年年底的时候,我在这个平台上跟大家已经讲过了,应该结束了,但是我们呢、我们呢有的时候不敢过于武断地去看待人性。有的时候人是很疯的一种动物、一种生物,他就有可能会做到你不信,所以就不敢那么精确地发预警。我为什么这一周周初就发出预警呢?是因为星期五的时候我们又做了一次复盘,就是复盘了一下子它的成本,就是美国这些上市企业的成本。

美国上市企业,就是美资、美企的成本里边,劳动力成本是构成40\%到50\%的,美国现在工资涨幅超过了10\%。此外所有的要素成本都在上涨,而美国这些企业提供的劳务或者是商品却不能这样的上涨,也就是说我们可以预期在美的这些上市企业都面临巨大的问题。同时像苹果,我上次已经说过了,像苹果,英国已经开始动手了,法国、德国也在陆续动手——就是反垄断。中国比较厚道,还没有开始收拾美国的头部。

所以大体上我们觉得对美国上市公司而言,一个特别的时代结束了,一个特别的时代结束了。是以疾风暴雨的方式结束呢,还是一种缓慢的这种调整来结束呢?到今天我也不敢说,因为虽然连跌四天,虽然这周他们只有四个交易日,星期一是假期,虽然只有四个交易日,但是我们非常非常明确地看到,在技术上非常明确地看到美国的主要的金融的大的金融机构在出货,连包括我本人。因为我也有在美资行的户口,他们也一直在提醒我要赶紧走。

这次调整意味着什么呢?时间长度是多长?对中国的影响是个什么呢?我个人认为这是一次历史性的调整,甚至,现在先不下结论,我只是在平台上跟你们——我的朋友们,你们是我最好的朋友,聊几句,不作数的。我个人认为这是一次历史性的调整,因为2008年那场危机其实美国没有完成调整,这一次愿意不愿意都要进入一次历史性的调整。会不会陷入大萧条?我说了,上一堂课我就说了,我们现在先不给结论,但我个人认为比较麻烦,非常接近陷入大萧条的前夜。

对中国会有多大的影响呢?肯定会有影响的,怎么会没有影响呢?现在全球经济是一体化的,就是如果美国的经济陷入一种状况的话,至少消费端我们少了一块,消费端少了一块之后,我们自己可能也会种种的问题牵扯,也会有一些麻烦。而且不排除美国人他经常是用其他方式来输出,不能用金融手段输出通胀的话,可能会采取战争的方式来输出一些通胀或者困难,导致美元资本的回流,有可能会出现极端的事情,都会对我国构成严重影响。

稍微略感欣慰的是我国提前蹲下啦,就是我国在2021年的下半年,率先对房企、对互联网、对教培等泡沫型的东西,进行了提前的泡沫释放。现在好多人在批评我,就说你非要喊直接税立法、房产税等等,而且我写的一些东西可能、可能……,所以有一些人可能对我意见很大,看了我的东西之后,特别是有关的部门、机构,甚至一些部门和机构的一些人通过自媒体来表达的一些意见吧。

其实我自己心里有数,我在说什么,我在做什么,而且好多朋友冤枉了、冤枉了我。我不是今天写的超级地租,更不是今天提直接税,我从来也没有单独提房产税,我写超级地租已经是二十年前的事情了,我提直接税也提了差不多十五年了。它有一个渐进的历史过程,今天引起了的领导的重视,而且可能成为立法行为,或者成为将来、未来的制度建设,这绝不是我一个人的努力,也不是什么影响到谁、挡谁发财之路,没这个意思、没有这个意思。

今天我们讲的这个《资本论》这个课程,跟这个内容略有联系。今天我们讲的课程的题目是《资本论》第二十一讲,题目是“地租套利与土地财政”。其实这是我给某位领导的一个作业吧,因为他们一直在要这篇作业。我交了以后,后来我不是很理想、很满意,因为大家更需要一个在学术上更清晰、更准确的东西,因为这不是一个政论,这也不是一个议论,它是一个非常严谨的学术论证的过程。

另外,香港这边的朋友呢也很急切,我其实都理解,因为香港这边,香港有一个富豪,喜欢玩女明星的富豪,叫刘銮雄,他的公司与“皮带哥”合作,他所以他买了好多皮带哥的股票,大概还有一些债券,去年亏了上百亿,几乎把它他亏出问题来。这个人是绝顶聪明的一个人,好多香港朋友问我,因为我虽然也跟他有见面,但不是好朋友,不熟悉。但我知道他绝顶聪明,当然我也知道他的问题,他不像那几个老狐狸。

香港的老牌的那四大家族,是非常懂得,非常懂战略的,所以他们身边会养门客,会请一些大师级的人物吧。这个刘呢自视聪明,所以他比较猛、猛人!所以他也发财,发财也快,但他往往就是到了这种褃节儿上出问题,他出的问题跟皮带哥、跟小任他们具有同质性。皮带哥不聪明吗?绝顶聪明。小任不聪明吗?绝顶聪明。而且他们都生活在那个位置,都在北京,甚至,基本上是接近最高了,接近……

香港的朋友总在问我,就是这年底、年初,大家在思考和总结,总在问我:刘先生——刘銮雄错在哪里?我说他错在看人,不是错在对事物的判断,是错在看人。我跟你说一遍,你听听看:中国的房地产在中国GDP中占什么样的比重?三成以上,三成以上GDP源于房地产,中国、中国财政收入半壁江山源于房地产。就不用去讨论土地财政问题了。

用当下网红,引述某网红----他其实不是经济学家(是规划专家)的话说,连我们的国防建设的费用都是源于房地产,这个民生改善诸多诸多的事情都来源于房地产,甚至我们的货币,货币发行的依据都是房地产。那么房地产在中国经济总量中的分量到了一个什么程度呢?香港有一个分析师,也现在是,几乎是最牛的基金经理吧。好吧,我不能再说了、再说这不合适了。

他们的数字虽然是不准确,他们认为是四五百万亿,就是中国的房地产的存量是四五百万亿,其实他们计算上是错的,远不是这个数,比这个数可能大一倍。这四五百万亿里边,如果跌20\%,就100万亿不见了,就相当于一个GDP丢掉了。每年呢?卖地收入八万到十万亿,土地财政,没了呢,地方财政怎么活呢?土地、地租,围绕着地租已经形成了一个完整的生态,我们的政府、我们的金融机构是活在这个地租生态圈儿上的,他们想说什么?你听,你听到了没有,你听到了没有?他们想说什么?

皮带哥聪明,小任也聪明。大刘——刘銮雄,香港人管他叫大刘,大刘也聪明。他们看到了一个什么问题呢?就是这个事情碰不得,所谓大而不能倒,为什么他们认为大而不能倒?就是不是地产商不能倒,是地产不能倒。地产一倒,围绕着地租形成的土地财政、土地货币、土地金融,甚至包括国防等一系列的事情都要受到影响,你够胆动吗?还什么直接税立法,还房地产税。所以,为什么大家敢于那么公然的去说那些话,甚至攻击,有些人跳出来攻击我。

是因为他们知道这里边的分量,所以他们认为没有人够胆碰这个东西,至少没有人敢现在就碰这个东西。击鼓传花嘛,传给别人就是了嘛,你干嘛要碰它呢?所以他们老神在在,在食租而肥。然而、然而,他们不了解我国这一代的领导人,他们不了解他们。他们想的问题不是他们个人的利益或者是某些人的利益,他们想对历史有一个清楚的交代,而且这帮人经历过上山下乡、经历过文革,他们足够的坚强。

香港的朋友说:“这也太不专业了”,我说:“如果是按你的专业理解,那这个泡沫就不应该刺破,地产也不应该限制,互联网也不应该整顿,那么教培继续鼓励”,那么他们把地租(我一会儿讲马克思的地租),他们把地租生态发挥到极致”。问题是地租生态,围绕着地租生态形成的这样一个车,它是一个加速型的,它越来越快、越来越快,最后肯定是车毁人亡;而大家都认为自己可以有机会从中间跳车走了,百万亿高净值都认为自己有机会,这不是这回看错了。

有没有人跳车?有,还得说那四个老狐狸,他们2015年就开始跳车了,到2018年,主要的全部跳完了,走了。好多人说他们蠢,跳早了。跳早了没有?当然跳早了,但是够了呀,赚够了,十倍到十五倍还不够吗?比在香港赚的都多,所以他们走了。而后边这些人认为从2015年开始还可以再有十年,这事儿就想的有点猛了,五年啊!2020年,人家2015年跳车,你2020年之前就该跳了,你不跳,还说那些话。

好,到了2021年,你想跳车,这车车速太快啦,跳的下去吗?有一个军人出身的同志——姓王的同志,他跳车跳得稍微早了一点,他是2019年跳车,虽说受了点伤,但人家不是完整跳车了吗?对吧,这个王健林同志(万达)人家跳得好啊、跳得早啊,大体上能活下去吧,对吧?但有些人还在嘲讽,这有什么好嘲讽的?你自己对人、对一个党、对一个国家没有承担,没有担待,做出了严重的误判,好玩吗?

其实读《资本论》读到这个程度,感慨万千。其实马克思对地租分了几个部分,做了比较深刻的剖析。地租是完全可以写一本专业的学术著作的,你可以写叫“地产经济”、“地产生态”或者地产什么吧……,或者叫“地租生态”,总之它是一个可以写一部书的,而且是一个严肃的经济学著作的这样一个东西。因为《资本论》的年代太久远,《资本论》没有遇到过土改之后的一个社会主义共和国地租生态圈发生的事情,他不可能想象的到,而我们正好处在这个时候,而我们所有的专家、学者、经济学家不愿意解释这个问题。

躬逢盛世,给了无数的机会,让无数的年轻的学子有机会登顶诺贝尔经济学奖,其实把一件事情吃透,就足以、足以青史留名了。不过多数的学者专家,包括小任他们,过于、过于在乎那点利了,比如说1500万年薪之类的。很多人认为小任不懂经济学,所以急功近利,看不懂,其实我不是这样看,因为我觉得皮带、小任、大刘都是顶级高手,他们是看人不行,看事看得准,但看人错了。

同时由于他们的错,让我感到无比的欣慰。如果他们都对了还得了?因为历史的经验证明,所有的发展中国家,甚至像日本这样的早已经完成发展、成为发达国家的国家,都没有跨过后殖民主义这道门槛,唯一的希望就在中国人身上。而中国人能跨过后殖民主义的门槛这一代人,必是出于文革,因为那是一次重要的思想解放运动,恰恰是那一次思想解放运动塑造了最后一批人,他们的使命就是将中国带出殖民主义,特别是……

特别是跨越后殖民主义。我应属于那一代人,因为我1963年出生,三岁,1966年进入文革,文革结束1976年,十三岁,进入中学。我进入中学的时候还戴红卫兵袖章,我很荣幸是我们那个年级的(一个班是一个小队、一个年级是一个中队),我是我们那个年级的红卫兵的中队长,也是学校大队副,因为我那会儿表现不错,所以我对那个时代有我自己的体会和感受,当然了,没几天那个红卫兵袖章就不戴了。

我入学之后第一批的共青团,另外1976年我们成立了第一个马列主义学习小组,那个时候开始读《毛选》,一个特殊的年代,我生长生活在山西太原,那个时候很多人觉得很可笑,就是这么点儿小孩,你能读懂那个东西吗?我不但读了,还写了一些诗。好多人认为读不懂,可他们不知道真读懂了、真读懂了,读懂了以后就很麻烦,就是这一代人赶上那个时代,并且读了一些东西,而且读懂了。这是皮带、小任、大刘他们无法理解的。

好,不做感慨,回到今天的课程,《资本论》第二十一讲“地租套利与土地财政”。我们先讲第一个部分关于地租,我就不再详细进入理论的部分,我提供一下子《资本论》第二卷第三篇是社会总资本的再生产和流通。社会总资本的再生产和流通里边涉及到了地租,其中第十八章第二个部分货币资本的作用里边就涉及到地租,第二十一章积累和扩大再生产又涉及到地租,《资本论》的第三卷第六篇

超额利润转化为地租,算是专门谈地租的。不过这一篇反而不重要,就是超额利润转化为地租这一篇,马克思偏重于农业租。因为地租我把它分成农业租、工业租、商业租、房租,分成四租。我们现在讨论的地租生态不包含农业租,农业租在当下的中国意义不是那么大,将来会意义大,暂时我们现在不重点讨论农业租。马克思在讨论地租的时候,由于他处在一个特殊的时代,所以他讨论了两个问题。但是马克思还是很厉害的,马克思、恩格斯在讨论经济学问题上,他确实有哲学高度,你不管隔多少年看。

这里边,首先在讨论地租之前,首先要讨论地权。地权与地上物权,无权哪来的租呢?所以地权和地上的物权,地权就是土地的产权;地上物权,可能土地不是你的,但地上物权,物权里边包含了产权和使用权。产权这种东西在很多国家是没有的,比如说香港,你只有个地上物权,这个物权是使用权,不是产权,因为地全是中华人民共和国,以前是英女王的,现在中华人民共和国的。地上物权里边包括了房子的产权和使用权,但这个事情在法律上还是存在诸多争议的。

你如果现在在英国,比如说有栋小楼,一栋楼里面有六套房,你买其中的一套,你知道你获得的是什么呢?是六分之一的地权加六分之一的产权,就是地权你拥有六分之一,产权你拥有六分之一。但由于这套房子原来的主人诸多的法律约束,可能你只能获得六分之一的地权和产权,却无法获得地权和产权使用的决策权——就是投票权。因为可能这个产权的股东或者是董事只有三个人,而你不是,所以他们可以决定你的事情,你得听着。

这个物权又有可分割之物权,它是,就是我刚才讲英国那样的可以带土地产权,但是在中国这种情况不存在;而大部分类似于像香港和大陆这样的,是带有有期限之物权(有时间限制的,五十年不变,就是到了2047年以后,这事儿重新讨论)。地权之上产生的地租、地权和地上附着物产生的地租,而这个地租还是一个广义的地租,还没有细分。

广义地租就是包含了地权、地上物、物权、它们的产权与使用权构成的全部的收入,这叫做一个广义的地租。所以马克思在讨论这个问题的时候,我估计很多人是看不懂的,看不懂没关系,看不懂没关系,但我们要把它的整个的、对整个事物的概念、逻辑要搞清楚,这是分析整个这个地租流转模型的非常重要的东西。我先说一下子地租产生的这样的一个逻辑链。地租产生的逻辑链是这样,农地的地租和今天我们理解的地租套利不是一回事儿了。

今天我们讨论的地租套利,它是工业化、城市化、土地资本化、地上物资本化形成的广义地租循环体系,形成的那种广义地租,而是对这种广义地租的套利。好吧,一共四化:第一是工业化,没有工业化,它就是个农业地租,其实那个地租是有限的;第二是城市化,因为有了工业化就需要工人的聚集、大规模人群的聚集,它需要城市化,城市化是一个复杂的东西,因为它有公共服务,它有医疗、学校等公共服务,它城市化,这个时候那个土地就变得有价值了;第三是土地资本化,这件事情大家能理解吧?

第四是地上物资本化。这四化一旦完成,那么广义地租的生态体系就开始出现了,一定要把这个搞清楚,一定要把这个事情搞清楚。然后,广义地租出现之后,就会形成广义地租的流转模型,它就开始是一个流转了——就是土地、开发商、金融机构形成一个完整的环流。这里边又涉及到土地财政、涉及到金融、涉及到劳动者的预支劳动者收入,一个完整的循环就开始了。我先不讲生态和循环,我先还回到理论上去。

我先讲一下子我国地租的性质,马克思没法讲我国地租,这个我来讲一下我国地租的性质,当然还是用着马克思的分析方法。我国的地权是人民所有,在某种意义上,人民委托国家、委托政府代持形成的所谓的公有或者是国有。准确的讲,土地地权是人民的,由国家(代为管理)代持或者代为管理,我说清楚了。因为马克思说了地权非常重要,地权没搞清楚,不能讨论地租,第一个部分说清楚了。第二个部分……

第二个部分,所有土地之上的广义地租收入,理论上都是人民的,很重要吧,地权以及地租。我说的这个地租收入不是指所有的收入,而是净收入。什么叫净收入?地产开发商开发它有成本的,去除成本之后的那个收入都应该是人民的,它既不是政府的——它土地财政拿走了,也不是地产开发商的暴利,也不是金融机构的。这不大家把人民的租分了,人民呢,

人民,用三十年的劳动——未来的劳动,提前预支了一份地租,这份地租被大家给分了。地权说清楚了,地租是属于人民的,第二个部分。第三个部分,地租的源泉是劳动者的未来的收入,按马克思的话说是劳动者的剩余价值形成的地租,被一个一个一个abcd给分掉了。地租套利的过程有没有正当性?有没有伦理问题和法理问题?这需要好好地讨论。

为什么我说这是一本书呢?是因为我们讨论问题的时候,有的时候会非常非常之困难,非常非常地困难。其实围绕地租形成了一个生态圈,这个生态圈——我说的是中国的地租。首先是土地财政。因为地权在谁手上?在政府手上,是因为不管是国有土地还是农村的集体所有土地,都可以通过政府的征地获得,然后政府把它卖了,卖了,卖地收入构成土地财政。它有没有伦理和法理上的问题呢?有一点点,但大体上……

因为我们的东西他拿去卖了,这个在法律上需要一个授权,需要一个授权。但是你说我们在委托他们的时候,这事算委托了还是没委托?信托了没信托?现在这是一笔糊涂账。但这件事情呢,讨论这个问题,马克思之所以要讨论地权的问题,实际上就是讨论制度和政策的伦理和法理逻辑问题了,这也是我们对土地财政进行改动的伦理和法理依据,要高度重视,因为我们讨论的是学术和理论问题,这个不能错。第二个部分,生态的第二个部分是土地货币。在联系汇率制度下,就是1994年分税制搞了土地财政,1995年联系汇率,我们的货币跟美元挂钩了。

如我们的货币不与美元挂钩,那我们货币发行的依据是什么?我们的依据一共是3个东西。第一个东西是税收,那么我们税收的主体是土地财政,我们的货币发行跟地租有关系;第二个依据是根据共和国的资产总量,我们资产总量里边大宗的是房地产,还是土地;我们的依据,无论是税收、无论是营收还是资产,三样东西,税收、GDP和……

税收、GDP和总资产。人民币发行的不联汇之后的3个依据全跟地租套利有关。所以在不联汇之后,我们现在货币发行的依据是依据地租来的,所以我管它叫土地货币,土地财政、土地货币。土地货币由于是为了活化地租用的,所以与之相伴生的就是土地金融。我前一段时间说,你打开中国上市公司,第一是茅台、第十是五粮液,中间夹着8个金融机构,这8个金融机构最近这15年在忙什么?土地,地租套利,它对工业的支持是越来越小了,它们都在忙着地租套利呢。

于是就形成了中国特有的现象,叫土地金融。土地金融上附着了两个东西,一个叫地产商,一个叫炒房客,他们都是通过土地金融来发达的,寄生于土地金融上面,土地金融寄生于地租套利上面。这是中国的地租套利、地租生态。地租生态第一个是土地财政,土地财政伴生的是我国大规模的基础设施建设、城市建设以及国防等等等等,就是中国的建设是来自于土地财政,它有一定的合理性,我刚才说了伦理和法理问题,也有一定的问题,因为……都搞这么多年了,也没有人写本书把这事说清楚,也不知道大家在忙什么。

土地财政、土地货币、土地金融,土地、地租套利生利圈的顶级部分就是土地工业。围绕着土地建设形成的钢筋水泥、形成的家电,形成的一系列东西,就是土地工业,还有这个大量的工人就业,它形成一个完整的地租套利的生态圈。这个生态圈对中国的经济发展是不是有重要意义呢?太重要了,重要的程度可以说是中国腾飞和崛起的一个非常重要的一个过程,因为我们不依靠这个地租就无法完成财务的整个的这样的运转和确认。

问题在哪儿呢?问题在整个这个过程是一个自发的过程,而非自觉的过程。就是从1994年分税到1995年联汇之后,然后到2002年我们开始发展土地,就是从胡温时代开始,土地到现在20年,我们始终无法完成对地租理论、地租生态的一种系统性的描述和论述,就是它应该怎样,这个地租生态圈应该怎样分布就是一个合理的布局呢?大家都来讨地租的红利、地租套利,那么套多少是合适的?现在合不合适?如果不合适,我们该怎么办呢?

更为严重的是,就是地租套利过程它是一个物理运动,它会不断加速,加速到一定程度就车毁人亡了。而且因为它这个加速的过程呢是一个大量制造财富和劫掠财富的过程,在整个这个过程中非常非常残酷。因为我在香港嘛,我看到了嘛,这个残酷的过程呢,会使得资本和持有资本和可以通过权力来参与整个套利行为的人陷入疯狂。不要认为赖小民、孙力军他们只是一个偶然现象,他们是在中国特殊的经济大潮中被资本搞晕了头的一部分的人而已。

我为什么这么高兴?就是整个这部加速度的车在2019年开始踩刹车,2020年再踩刹车,2021年大体上刹住了,这个车大体上刹住了,我们可以预见的结果是不会车毁人亡了。但不会车毁人亡会不会有附带的效应呢?当然会有。为什么那么多人批评我,明着批的、暗着批的,一大堆人批我,批我的原因就是他们的想法跟皮带哥、跟小任、跟大刘是一致的。那你不让这样弄,那中国经济就会出现问题呀,要衰退啊,那怎么办呢?

具体事物具体分析,中国的GDP一年100万亿,中国的财政收入20多万亿,地产的贡献8万亿,就是卖地收入,直接的、直接的大概是8万亿左右,间接的可能到10万亿。这块儿不会全没有,但会少一些。有问题没有,我个人认为没问题呀,怎么可能有问题呢?因为我看到的是,已经累积起来的,在过往20年之间通过地租套利累积起来的那部分财富还摆在那儿,我自己的计算是百万亿级别。

可能,可能有20万亿走掉了,包括李先生他们几个。一些老狐狸和小狐狸可能走了20万亿,跑了,套利走了。但大部分还在啊,至少还有80万亿,甚至可能还有100万亿摆在那个地方。他们拿这笔钱不愿意去工业化嘛,他们不愿意工业化升级嘛,也不愿意参与国防建设的嘛,也不想给老百姓脱贫致富、共同富裕。那么这个时候就需要重新使用财政工具,而且要使用得非常好,让他们都开心、高兴地重新参加社会主义建设。这样……

在一个特殊的过渡时期,比如说2022年下半年我们来不及,那么我们用水循环的建设来启动经济,重新拉动、以工代赈,老百姓的问题先解决。但2022年完成直接税立法之后,我们陆续将那漂浮在现实生活以外的百万亿浮财,请它们回到实体经济中来与人民一起进行伟大的社会主义建设,完成第二个100年,这不好吗?你不要跟我说没有钱,没有了。有的,有的,这部分钱从伦理上和法理上本就属于人民嘛,不小心被拿走了。

话说到这儿呢,可能好多朋友不一定同意我的看法。我也不是要打土豪、分田地的,没这个意思。甚至我根本就不想让他们把这个钱无偿地送回来,没有必要。只是由于特殊的历史情况发生了,所以好多好多人——三代人六口拿30年的钱、30年的未来收入去买一个本来就属于他的东西,这个整个的这个操作逻辑和操作方法上是有问题的,只是我们现在要把这个时间和空间做一点点地挪移而已。如果我们这代人都做不成的话,我可以明确地说……

在拉丁美洲发生了什么?50年代阿根廷已经追上西班牙了,为什么出问题呢?不就是个地租嘛,地租套利嘛,没解决好嘛。北非、中东不就是个地租套利嘛,发现油了嘛,这个地租套利出了问题嘛。前苏东地区还是个地租套利嘛,这个地租套利过程肯定很难正态分布,不能正态分布在伦理和法理上没有解决,那么我们通过伦理和法理上重新梳理,我们让它稍微正态一点,不要太变态就可以了。如果这个道理能说清楚,我想这马克思如果是今天活着,他会说得比我清楚。

治理逻辑,特别是基于地租和剩余价值的治理逻辑里边正、反两个方面的辩证是非常重要的,必须给工业资本提供地租和剩余价值,不然你工业化升级做不出来,你不干这件事情,你就变成香港了,你就会变成明天的美国。去工业化,没有了之后,重新返回第三世界。你以为发达国家就不会返贫吗?不懂得这个理论,读不懂《资本论》,你就想敢谈治国理政,是不是稍微有一点点的……好吧,这话大了。

好,我们回到今天最后一个环节。我想说工业化升级的前提条件是什么?工业化升级的前提条件我把它归纳为六条,这六条跟马克思、恩格斯归纳略微不同,跟列宁的就完全不同。但我觉得这个归纳是有必要的,我们必须在理论上有所提升和创新,不提升、不创新是不行的。我那天听了汪晖老师的一段话,我觉得他说的有道理。他说:“不能用西方的经济学或者是社会学的原理来解释当下,越解释越乱,也不能用原来我们的固有的语言来解释当下,越解释越乱,要用新的体系和新的语言来解释当下。”

好吧,我尝试建立新的一个分析框架,我管它叫工业化升级的前提条件或者是必要条件。主要是个前提条件吧,必要条件有点太武断。一个地方能否完成工业化升级,必须具备六个基本的条件。第一是秩序,一个国家一旦丧失了正常的秩序,出现了混乱,工业化升级就中断了。美国正在陷入失序的状态,这个零元购、抢运货列车,过两天就该打劫豪宅了。就是你一旦失序,工业化升级就没得谈了,所以秩序是第一位的。

这一谈到关键的时候这信号就不行,讲完了发不出去。第二就是制度成本,制度成本必须压缩在合理范畴之下,制度成本一旦高速扩张,工业化升级是没机会的。如果你不理解秩序和制度成本,这是第一条和第二条,那么我告诉你,“东北病”就是这两条。第一,秩序出了问题。秩序出问题不是“座山雕”回来了,是有些同志变成“座山雕”了,所以秩序乱了。第二是制度成本太高了,都在吃,不光吃地租套利,什么都吃。

我们注意到俄罗斯,注意到几个斯坦,注意到一些国家,他们经济发展不起来,有些国家地方秩序还凑合,不算太乱,但制度成本高到你难以想象,制度成本太高了。秩序和制度成本是排第一和第二位的,那么谁排第三呢?地租,地租排第三。有些地方将地租套利上升到一个无法理解的高度,乃至于开始去工业化,工业化升级结束,最经典的案例是香港。无论是工业租、商业租、房租都高到天上去,任何工业都无法生存。

一个月前某个场合,大家在讨论这个,特首也在。我就说麻烦大家回到1971年去,看看香港工业化的起点在哪里,为什么那个时候可以四小龙腾飞?不要站在2021年,不要站在2021年说香港的再工业化,不要谈什么北部都会区,你不懂地租,你说这些事有什么意思?你能为工业资本重新提供廉价的地租吗?你不能提供,你说这个干什么呢?新加坡比你地方还小,填了块海,造出一片工业区来强行提供工业地租。

从海外雇佣了150万廉价劳工,强行提供剩余价值。有了地租和剩余价值,新加坡竟然有了工业。你不学《资本论》,你啥子都不知道,你还说这些话,还号称……,不批评他们。好多人都是以专家和学者的身份,是港府顾问呐,就是没法说、没法解释。就是他们如果不在了,香港就还有一丝机会发展,他们全都在,这绝无机会发展。因为他们说的全错,不是一点错,就没一个是对的。如果照他们的逻辑去发展,除了灭亡就只有灭亡。你能怎么办呢?

工业化升级的第四个条件就是剩余价值,你必须向工业资本提供劳动者剩余。新加坡的办法,150万海外劳工圈在那个工业地上,然后形成了他的工业租和剩余价值,解决了这个问题。香港能不能解决这个问题呢?当然没有问题的嘛!我们是基建狂魔,推倒交椅洲、推倒喜灵洲那几个岛,你是要建居住环境有居住,要建工业有工业。香港具有天然最好、全是最好条件。是需要你懂啊!你懂、读懂一下《资本论》。至于剩余价值的提供,香港本土也有,外边请也行。

在这儿想多说一句,大家提醒我不要说不要说,但我不想,不能不说,就是深圳。深圳在秩序和制度成本上凑合,真算不上好。秩序和制度成本,秩序这两年稍微好一点,制度成本还是高啊。但是它竟然在地租和剩余价值这两个问题上犯了跟香港同样的错误,所以迅速去工业化、迅速衰落,非常之糟糕。不读《资本论》,当然读了《资本论》也不一定就行。这个你这个治理,治理一个地方,你怎么来理解地租呢?难道你就是来加速套利的吗?套完利呢?然后就去工业化,然后就这个城市死亡。糟糕透顶!

一个地方第五个工业化升级的前提条件才是电讯与交通——基础设施。电讯与交通这种基础设施其实是容易处理的,它是个物理建设。但说是物理建设,它也需要一个比较长的周期。比如说印度,你没有电讯交通等基础设施配套,你工业化的条件其实是不具备的。工业化升级的第六个前提条件是其他要素配套,就是水、电、气等要素配套,配套条件。其实五和六这两个配套条件是容易做到的,因为它是具有一种物理性。

第一、秩序;第二、制度成本,它是考验政府的能力和水平的;第三、地租;第四、剩余价值,它考验的可不是政府治理水平,它考验的是这个地方立法者的悲悯——道德水平。如果这个地方的立法者的道德悲悯没了,他来了就是慌忙地进行疯狂的地租套利,对剩余价值进行残忍的剥夺,他用短短十年时间就完成了一百年的套利,这个城市就变成一个悲剧了。希望不要再有。

讨论地租的时候,我们讨论了地租生态圈、讨论了一个模型、讨论了……,大体上先说这么多吧。因为我是打算写完之后花一次聊天的时间,把这个事情聊给大家听,因为除了理论之外,它也很有趣。另外它对好多投资者是有重要价值的。我香港的朋友、好朋友、一些基金经理,他们就很喜欢听这个东西,就是他们觉得如果你能给出一个完整的分析框架,我们其实知道去哪个城市投资和生活、去哪些地区投资和生活。我说这是对的呀,这是对的呀。我们批评归批评,是不是也该表扬啊?

我想表扬一下合肥,合肥做得好。合肥不光是吸引了很多的高校、高科技企业进驻,而且他对工业租的理解、对商业租的理解、对房租的理解是到位的。也不是特别完美,他到位,他看得远。他不会把高科技企业,像面板这样的高科技企业、像京东方这样的企业叫来像训小煤窑主一样的训的嘛,不会这样训嘛,他不会赶你走。他请你进来,甚至你有困难帮你解决,甚至他还作为资方、作为风投方参与你,让你成长,很了不起的。不知道读过《资本论》没有,很了不起。

上海、杭州做得也还凑合,不算特别好。我还以为上海先东北化呢,我哪能想到是深圳呢,率先完成东北化。太扯了、太扯了。关于地租的部分,今天也讲不了那么多,就先说这么多吧。我想如果将来有时间,可能应该写一本小册子,如果没有时间的话,我会把完成的那篇文章,因为春节之前必须完成,春节前完成之后,还有十天吧,我把那个东西给大家。

简单说几句市场,因为不聊也不行。虽然美股出了问题,这个港股仍然在通道里,大A也在通道里,国内的股市和这个港股都在通道里。我们因为“短股长金”,“短股”,好多朋友可能也没炒,有些炒股的朋友可能也没跑。“长金”的一些朋友们对我有意见也不好说什么,反正是没输嘛。因为这是一个特殊的时候,除了疫情以外,还有诸多诸多的问题,比如说我国正在对地租问题等诸多问题在调整,经济可能会在一个特殊的时期进入衰退。

会不会陷入大萧条呢?这些日子我们一直在讨论,看这个样子,疫情还得持续到这个上半年结束,可能下半年会好一些。也不排除有一个新的、更要命的病毒出来偷袭,如果是那个样的情况,大萧条基本上是可以确认。即便是没有这个,那么美国经济会否进入深度调整呢?

我的看法是,如果疫情不能走出,如果美国出现更为严重的失序,那么大萧条将不可避免。是否会叠加军事冲突?比如说乌东地区的军事冲突,比如说南海或者台海。反正我的个人建议是非常清楚的,就是我们不希望在台海、南海发生任何事情,因为我们需要忍10年,到2032年以后再说,这十年之内不管发生什么事情都必须忍。有太多太多人想让中国做点事情,但我们不会学威廉二世,我们不会葬送一百年,没必要。

在香港的时候,我很激动,他们在讨论台湾问题。我说我们中国人从来没说要解放台湾,我们要解放的是日本,好不好?我们要解放的是韩国,好不好?我们要东亚走出二战的雅尔塔体系,东亚整体上走出后殖民主义时代。你为什么非要我们解放台湾呢?我们要解放日本、南韩,好不好?所以好多人非逼我们钻别人那个圈套,我们就不钻。就好像乌克兰,摆明了美国人和俄国人设个圈套给德国人、法国人的、老欧洲的,但你知道,现在德国选了一个绿党,八零后的一个女生。

我突然感觉到,这个民族怎么了?两次世界大战都没打够啊?还这么笨!还这么弱智!这事儿有意思吗?!但我也理解美国,收割谁呢?三条航母、四条航母来中国南海,问题是我们不理你,我们要解放的是日本,好不好?是南韩,台湾问题不着急。家里的事情你急什么呀?对吧?他又跑不了。把那两个、把那边的事先办了,办完了以后回来再说嘛。所以,我估计,收割中国是收割不成的。日本和韩国那个样子,只能收欧洲。

所以,今年中国人可能是既济。美国人和欧洲人能既济吗?壬寅既济,况且既济的最后一爻依旧是凶险的。所以呢,我们的策略还是要持盈保泰,我自己主张还要坚持一段时间“短股长金”,我们耐心在等,让通胀充分地表达。表达之后,然后我们等经济相对稳定一点之后,因为中国的调整包括对地产的调整、包括对互联网的调整、包括对教培的调整也没结束呢,美国呢刚刚开始进入到深度的调整过程。

我们给自己一点空间,在这段空间里边我们写总结,春节到了,写总结。然后整理,对整个市场做出整理,整理出我们要做的、观察的那9个家伙,把它列出来。然后开始仔细地观察,寻找时机,把那9个家伙弄利索。高手呢可以考虑在美国市场出现了剧烈的折腾,这个里边会有一些机会,顶级高手可以出去玩。普通老百姓就别折腾,老老实实地听话,持盈保泰。不过无论如何,2022年都应该是我们赚钱的一个年,这是一个好的年景,我们应该今年是有比去年更好的斩获,因为2019年、2020年我们都不错,2021年好像……

好吧,2022年我们做得好一些。至于大家说一些细节,我想春节前不急吧,春节之后我们进入一些细节的讨论。其实细节大家也都清楚,我们一向关注的一些问题,能源、两化、生物科技都差不多是清晰的了,我们只是再把它捋一捋,省得大家……有好多朋友可能说,我们没有9个观察点,那么提供一些观察点,让大家平时养成每周看东西、观察,然后寻找自己切入点。其实最近大A也有好多不错的机会,都出来了一些的机会。你知道每一次狂跌都是很好很好的机会,不要认为互联网的整顿不是件好事。

你不是终于给你挑你喜欢的货的东西出来了吗?有些调整过程其实就是一个机会嘛。不要认为房地产的调整不是好事,不是也给了你足够多的机会吗?对吧?好吧,今天就讲这么多,拉拉杂杂。今天有些事情敏感,我们就不外传了,大家听一听,消化消化就可以了。分析问题嘛大概是个这样的分析就行了,知道、记着地租套利生态圈,地租套利、地租生态圈,以后你再理解地租问题和房子问题就很清晰,这样就好了。好吧,明天下午三点钟见。大家一定要注意安全,注意保持社交距离,不要乱跑,注意安全。好,明天见。

\subsection{关于后殖民主义的思考}

今天是2022年的1月1号,农历年辛丑年十一月二十九日,恭祝大家新年快乐!我们进入到2022年了,今天聊天是关于后殖民主义的思考。好,我试一下麦。3点钟,我们准时开始。

大家好,今天2022年的1月1号,新年快乐!农历年还在辛丑年,今天是十一月二十九日,还差一个月就是春节了。今天这个聊天思虑再三,按原来的想法是柯立芝繁荣,讲一次柯立芝繁荣,但因为今天是1月1号,说好的今天是一个元旦的礼物,那么我觉得应该把最重要的内容在今天把它讲了。昨天晚上参加了亚洲周刊的跨年的……

昨天晚上参加了亚洲周刊举行的跨年的一个派对,从9点钟到凌晨1点钟。其中:台湾那边是马英九、张亚中,还有这个苏起,北京这边有这个温铁军、汪辉和宋阳标,香港这边有一些都是我熟悉的朋友。我昨天做了一个简短发言,谈的也是这件事情。这件事情其实也引起了很强烈的共鸣,我倒是没有想到香港的一些老人家对殖民问题那么敏锐。

这件事情要从春节说起,应该是说辛丑年的春节,是2021年的事情啦。春节的时候我们聚会,大家在讨论:百年未有之变局,是什么变局?讨论走入新时代,什么是新时代?什么是变局?什么是新时代?另外一位老同志提出来说:建党一百年(两个一百年,第一个一百年)走过去了,第二个一百年——建国一百年,中国最大的挑战是什么?并要求脱离具体问题,

尽可能地上升到哲学高度,尽可能地拉长历史纵深来思考问题,这就一直算是在我心里边的一件事情吧。关于这个后殖民主义,关于萨义德,我们在这个平台上讲过好多次了,但我想了想,还是要把它系统化的讲一次,作为新年献礼给大家。因为这件事情对我们每个人都很重要。当然了,从更宏观的意义上来说,可能对我们的国家也很重要,因为我们必须跳起来,跳起三万尺才能看得清楚,看得明白。

我们讲了《资本论》的一共是19讲了,接近尾声。其中我们讲了马克思《资本论》第1卷第7篇第25章,也是《资本论》第1卷的最后一章——现代殖民原理。我记得我当时说过,马克思是真的了不起。他对问题的理解地深度,可能有的时候要反复地回望,我们才能发现他深邃的哲学的视角和他历史的阔度,确实非常人所能及。要记着第7篇——《资本论》第7篇谈的是资本的积累过程,而第25章——最后一章谈的是……

而最后一章——第25章谈的现代殖民理论,它是资本积累过程的一个重要的构成,或者是马克思讨论《资本论》或者是讨论资本主义,或者是讨论社会主义非常重要的一个节点。我相信很多读《资本论》的人可能对剩余价值是可以理解的,剩余价值的理论是可以理解的,对资本流转的过程也是可以理解的,但对现代殖民理论能理解到什么程度,那就见仁见智了。难得的是在110年之后,《资本论》是1867年发表的,萨义德的《东方学》是1978年发表的,就是在110年之后,有一个阿拉伯裔的……

有一个阿拉伯裔的美国人竟然洞彻了《资本论》的这一章,他读懂了这一章,创立了后殖民主义。这个概念呢,中国人是生活在后殖民主义当中,但对后殖民主义缺乏理论的自觉,就是我们缺乏对自身后殖民主义现象、后殖民主义问题的深刻的认识,更加缺乏系统性的批判,也没有办法自觉完成伟大的历史性超越。如果,如果你问我:“什么是走进新时代?”

那么我要说:与其说走进新时代,不如说超越后殖民主义。当然超越前提是必须认识它,走出它,完成对它的历史性扬弃。也许今后的28年,我们要做的工作就是这个工作,所以这件事情天大地大。平台上的朋友也许会问:“这可能是对国家的意义,对个人的意义也是如此吗?”当然。国家的命运和个人的命运是一体的,况且在我们每一个人身上都有后殖民主义的烙印,是我们浑然不知而已。

殖民……好,我们还是按老规矩,先从概念和定义入手。殖民是什么意思?殖——殖是滋生的意思,是繁殖的意思,也可以理解为植入。殖民,如果翻译得再白一点,就是移民。殖民主义难道仅仅是移民主义吗?那就不用用“殖”这个字,因为“殖”这个作为一个动词的“殖”,和“移”还是不一样的,它是插入、是入侵者、是后来者。这里边不能简单地用贬义和褒义来理解这个“殖”字,但是这个“殖”字非常非常重要,“殖民”这个词非常重要。

好,为了理解什么叫后殖民主义,我们必须先说什么是前殖民主义。人类开始殖民到今天约略是500年的历史,如果我们把哥伦布发现美洲大陆作为一个起点的话,那么人类的殖民历史、现代殖民历史应该是500年。当然马克思并不这样认为,马克思认为有了人出现以后,人开始形成社会之后,殖民现象就一直存在。这个我也同意,但我们不能把它定义成5000年或者是一万年,我们把这种工业化进程中的殖民过程,大规模的殖民过程,

定义为一个可以作为一种哲学上描述的殖民史。我们把它的起点放在哥伦布发现新大陆开始,那么是500年。那么你就可以理解马克思写《资本论》第二十五章的时候,《资本论》发表是在1867年,1867年发表这篇文章的时候,发表这个《资本论》的时候应处于殖民史的中段,所以它叫现代殖民理论。萨义德读懂了马克思,所以他叫后殖民理论。现在我要说的是前殖民理论,前殖民、现代殖民和后殖民它们的区别是什么?前殖民……

前殖民,殖民者不要殖民地的生产者, 所以前殖民主义的经典特征就是种族灭绝、大规模的种族灭绝。它不是一个简单的植入,而是替代,它把原住民或者是占据生产资料的生产者进行种族灭绝、大屠杀,北美、南美、澳大利亚、新西兰,非常经典,盎格鲁撒克逊人、盎鲁人在前殖民时代玩儿得最经典的就是种族灭绝。所以当他们谈到新疆种族灭绝的时候,请不要惊讶。

那么,前殖民的种族灭绝发展到后来,进入到亚洲区域的时候,这个前殖民的种族灭绝就不好用了。因为当进行到亚洲、特别是东亚的时候,这个地方它和印第安人不一样,印第安人是部落,没有统一国家,灭绝是很容易的。进入到亚洲,像中国这样的国家、像日本这样的国家,甚至像印度,他都有统一的政权、有国家,所以大规模的种族灭绝不现实,所以就是马克思所定义的现代殖民。现代殖民的特点是什么呢?将生产者和生产资料分离。

他们不再屠杀,也不是没屠杀,也没有少屠杀,还是在大量屠杀,但不再进行种族灭绝。他只要求生产者与生产资料分离,因为他要生产资料,让生产者沦为殖民者的奴隶,或者是沦为资本主义状态下的雇佣劳动者。马克思之所以写现代殖民理论,就是在殖民地雇佣关系是什么样的呢?是殖民者占据了生产资料,将原生产者重新变成奴隶或者是雇佣者。在非洲、在一些不成为国家的部落,包括南美,很多人沦为奴隶,但在亚洲,以中国为例吧,就是重新成为……

重新成为被雇佣者。现代殖民理论,现代殖民的这个现象持续到什么时间为止呢?持续到1945年第二次世界大战结束。我想对三段殖民做历史性的划分,前殖民起点我们把它假设为哥伦布发现新大陆;那么现代殖民的起点应在何时呢?我们觉得应该将英、美进入东亚或者是我们用一个……

我们用一个粗略的时间节点吧,约略是在1775年这样大概的前后,把它作为一个时间节点,然后它的终止的点是1945年;后殖民的时代是从1945年开始。二次大战结束之后,反殖民的运动风起云涌,大部分的殖民地独立了,当然现在还有一些殖民地没有独立,但是即便是没有获得解放和独立的殖民地,现在也改善了以前的那种现代殖民时期的那种生产者和生产资料分离的那样一个状况。那么后殖民是个什么状况呢?后殖民……

后殖民它就不再进行生产者和生产资料的分离,生产者和生产资料没有分离,而殖民者通过控制文化,主要是教育、学术、传媒,通过意识形态、通过文化的控制、通过资本的控制、通过市场的操纵,形成对后殖民国家的系统性控制。这一点非常重要。要命的是什么呢?1945年之后,进入后殖民主义时代的原殖民地国家包括半殖民地国家,

所有的反抗、反对后殖民主义压迫的国家,至今无一成功,没有一个成功的,全部都失败了。最后的战胜后殖民主义或者是终结殖民时代500年的殖民历史的这个历史使命就交给了中国人。所以你理解我刚才说的那句话:与其说走进新时代,不如说走出后殖民主义。你知道这个份量,也就是说到2049年,用28年的时间,中国人将第一个成功地完结后殖民主义。

如果,如果我们能够完结后殖民主义,我们将终结500年的殖民历史,我们将为这个世界寻找出一条新的道路。如果我们将殖民与资本主义,把它的关系说清楚的话,或者是殖民时代是资本主义必有的一个特征,因为马克思在《资本论》第一卷的最后一章讲现代殖民理论,实际上是将它定义为资本主义的一个经典特征的,资本原始积累的来源就是殖民、是殖民主义,是经典的资本主义原始积累的特征。这个时代一旦结束,那么人类整体上将进入到新时代。

我们简单的介绍一下萨义德,萨义德这个人他还是很了不起的。萨义德,他出生在1935年,他的生命的长度并不长,这个萨义德2003年9月走了。有意思的是,他是著名的文学理论家与批评家,是巴勒斯坦建国运动的活跃分子。注意他参加了政治运动,巴勒斯坦建国的活跃份子,成为了美国最具争议的学院派学者。

同时他也是一位乐评家、歌剧学者和钢琴家。好多朋友说,为什么后殖民主义理论是由一个文学评论家来完成的,并不是由哲学家或者是由社会学家,或者是由经济学家完成的?这才是今天我们讲课,今天不是讲课,今天是聊天,是我跟大家汇报我的思考,这才是我们要讨论的重点。因为,“萨义德”这是他的姓,他的父母都是巴勒斯坦人,他的英文名叫Edward、爱德华,然后他是美国籍,大学普林斯顿,研究生哈佛,在哈佛任教。

因为是巴勒斯坦人,因为是深受巴勒斯坦文化的影响,所以他对巴勒斯坦人有着深刻的体会。我们每一个人对出生的地方、对祖籍都有一种莫名的情感,我们对我们出生的那个地方的人也有着深刻的了解。虽然我们可能长大之后远离了,但整个那种关注是不言而喻的,萨义德也是如此。比较有趣的是,他受完整的西方教育,并且他是基督徒。这有意思吧?在双重的文化背景下来思考巴勒斯坦问题,他看到的东西要比常人更深刻。

作为一个文学评论家,他首先从文学的角度、从文艺的角度、从文化的角度发现了后殖民主义的问题。好吧,我们回到现实中来。新东方,新东方在教英语。我们最近对教培的事情——教培里边核心的问题是两个,一个是奥数,一个是英语。奥数的问题我们暂时不讨论,英语的问题,我想简单说几句。因为昨天晚上香港的发言的人,三个人谈到了同一问题就是香港的中文大学没有中文了。英文,香港的中学、好一点的中学全部是英文教学。当然好一点的小学也全部是英文教学。

昨天有个艺术家提出一个问题来“艺术应该用什么来教?是用英文来教,还是用中文来教?科学应该用英文教还是用中文教?”我们在很多时候没有将语言问题深化为文化问题。语言问题是什么?当你的一个社会,比如说香港,比如说新加坡,也比如印度,当你的全社会将英语作为一个主体性语言,或者是工作性语言、或者是生活性的语言,那么它的信息源泉就源于英语世界。

在意识形态上可能是一种天成,在意识形态上可能英语世界就变成了一种主体性的东西。这里边没有简单的“好与不好”的问题,萨义德只是将它概述为欧洲中心主义或者是英美中心主义。因为长时间的一种熏陶,实际上殖民地、半殖民地的人经过若干代——都不是两代、三代——若干代之后,他已经将一种洋人对世界的理解、特别是对殖民地的理解,变成了一种意识形态的本能。其实我们今天看到的年轻的男男女女

如果你简简单地认为他是崇洋媚外,那么可能就浅了。在语言问题上,我感到惊讶。我昨天在我的发言里边,我表扬了香港的这些老的艺术家和思想家,他们看到了问题,他们看到了问题。语言问题是文化的根,是文明的根,是伦理的根、是法理的根。如果根没了,是你接受了殖民者给你的游戏规则,这个游戏规则中其实你可能未必能够理解和看到:你仍然是被殖民者,是被雇佣者。是被剥削和压迫者。

现在说到这儿,我们的朋友们大概可以理解为什么改革开放之后,后殖民主义回潮。后殖民主义在中国回潮之后,形成的民国热、张爱玲热,形成的莫言、张艺谋等对革命和文革的控诉。什么是民国热?什么是张爱玲热?好吧。我现在来解释萨义德发现了什么,什么是后殖民主义的特征呢?后殖民主义的特征与三个殖民时代有关系啊。

在后殖民主义时代,其实是通过代理人制来实现殖民管理的。因为前殖民,殖民就代替了——移民代替了原住民,所以就不存在代理人的问题。现代殖民理论确实是直接,人家就在这儿来管你,就把生产者、生产资料分离了。只有后殖民主义或者后殖民时期才需要代理人,代理人三重,三重请记着。第一重是派遣的代理人,虽然他是留美归来留英归来,但是他是派遣来的,有可能是政府,有可能是机构派遣来的。

请记住,大部分派遣来的代理人在我们这儿是拥有崇高地位的,你不会把它理解成为Spy,在法律上他也不是特务,但是他是代理人,请记着“派遣代理人”。第二个,自发代理人。自发代理人是所谓的本土精英自觉的、自发的成为殖民地文化的代理人、政治的代理人、经济的代理人。请记着,是“自发代理人”,不是派遣的,他自己愿意干,而且呢因为他处在社会里边的精英地位,他正好可以承担这样的一个责任。第三个是盲从代理人。他并不知道,只是他不小心做了这件事儿,但他不知道他在做什么。我想90%的代理人是盲从代理人。

你说我国的一些“艺术家”类似于像莫言、张艺谋、方方,你说他是自发代理人或者自觉代理人,我看都是一种盲从。他们不知道他们为什么批判共产党,他们不知道他们为什么反革命、反文革,他们不知道;他们只是本能地认为他是精英,而精英在整个那个过程中受了伤害,他并不知道这是后殖民主义。所以其实在香港这些大学里边,好多的大学教授,可能是大部分的大学教授本能地就是这样的,为后殖民主义申辩。

我在香港,1995年来,26年,今年开始到27年,当我提出超级地租的时候,你知道我的孤独。因为多数人认为英治时期是好的、没有问题的、不容置疑的、不可以怀疑的;不光是他们认为,我们在中英联合声明上也是认为50年不变。没有人去思考后殖民主义问题,更没有人完成对后殖民主义的系统性地批判,包括文化的、政治的和经济的,没有人做这个事情;你做了,石破惊天,伤害到代理人的利益了,所以他们团灭你。

好多时候好多事情呢,你说到这个程度的时候,其实我们有些词、有些词可能大家现在用的很少很少了,对吧?用“代理人”这个词就是怎么说好呢,确实是存在一种派遣,你管他叫特务也行,确实存在被拿下的所谓的汉奸,确实存在买办,但特务、汉奸、买办这样的词其实在当下的语境里边是不合适的,因为在法律上,特务、汉奸、买办在法律上是不能不追究法律责任的。

你举报一个人是特务、是汉奸、是买办,你有举证责任,你举证了以后还要有相关的法律来完成对他们的制裁。但是在现代的语境里边、在后殖民主义的语境里边,在法理上不成立。虽然你明知道他是派遣代理人、他是自觉代理人、他是盲从代理人,但在法理上并不成立。我为什么要讲,这不是讲的,我为什么要今天聊这个天儿,就是希望我国的青年能够懂,我们要用28年的时间,在2049年的时候,结束后殖民主义时代,结束500年的殖民历史。

由于后殖民主义经典的文化特征,不敏感的思想家是看不到的。而萨义德在哈佛,他反观中东、反观中国、反观印度,他写了《东方学》,他发现了后殖民主义问题,他从文化角度发现这个问题,并且从文化角度进行了论述。我将我的想法讲给我的一些朋友们听,我的朋友们告诉我,你现在上维基、上百度,上维基百科、上百度,你去调后殖民主义、特别是百度的后殖民主义那个定义,你看一眼,你会感到极度失望。

因为既没有人真正读懂萨义德,而把萨义德的后殖民主义进行了文化概括,很大程度上是文学概括,而这个文学概括没有那么深刻。所以当中国的老百姓去批评莫言、方方、张艺谋的时候,去反感民国热和张爱玲的时候,其实他们没有找到思想的武器、理论的武器,对殖民问题认识是不深刻的。所以我们在处理重大政治问题的时候,依旧是存留于英美主体性的滚滚洪流之中,被裹挟,泥沙俱下。

反对后殖民主义,必须建立中华民族的主体性。这里边的主体性包含了两个层面,就是我一会儿讨论反对后殖民主义的两个“性”问题,第一个是民族、国家的主体性,第二个是人民的主体性,国家主体性和人民主体是双重的主体性,完成之后才完成对后殖民主义的压迫。一会儿我讲逻辑关系,为什么国家主体性建立不能反对后殖民主义,为什么人民主体性建立才能反对后殖民主义?就是人民主体性建立,社会主体性出现之后,人民主体性是反精英的。当然这里边也有它存在的必然的、一种逆反的逻辑,我们先按照我们的逻辑顺序把它讲下去,不着急吧。

后殖民主义的问题在政治上表现是非常明确的,它是一种审美。后殖民主义政治问题涉及到的核心问题是立法权问题,因为在司法权层面、行政权层面不存在后殖民主义或者是现代殖民或者前殖民主义的区别,因为司法权和行政权本身不具有殖民特征,它是工具属性而不是政治属性,这恰恰是主席——教员同志在文革中忽略的问题,就是立法权才是人民主权,司法权和行政权是人民需要监督和管理的主权,而不是你要替代的。

当砸乱公检法、在政府建立革委会的时候,其实人民主权向前多迈一步就是民粹主义,有的时候是极端民粹主义,必然导致政治走向反面,所以反对民粹主义非常困难。萨义德认为反对民粹主义搞不好就变成极端民族主义。比如说伊朗,萨义德对伊朗的问题研究地非常透,就是伊朗巴列维国王被推翻了,因为他是买办嘛,它是后殖民主义的代表人物嘛,那么回来的是谁?是霍梅尼,是原教旨主义。注意,阿富汗塔利班也是!注意,土耳其也是!注意,今天北非、中东国家的同一个现象也是!甚至南美洲也是!甚至……

甚至连东欧也是这样的一个情况!我们有没有出现这个情况?有的。极端民族主义这种情绪存不存在?有的,存在的。我今天不想在这儿讲现象,因为讲现象可能会伤到一些国内的左翼的一些朋友,因为任何的一种形式的存在都存在边际,过一步就是谬误,虽然你的立场是对的,但过头了就是谬误。所以极端民族主义和极端民粹主义是存在的,所以我们在政治上、在讨论政治主权的时候,我们要知道我们要什么。

后殖民主义的立法权有一个有趣的特征,就是明星立法者。我们将运动员和电影演员都放到人大去参与立法,甚至我们将港、澳、台的这个明星也放进来参与立法。其实这是后殖民主义经典的,不是文化特征、是政治特征。因为他不是人民,他完全不能代表人民,但他可以代表代理人呐。代理人呐,说到这有点心疼。请看一看构成吧,如果两亿农民工只有一个代表、两个代表,而金融机构竟然有两百个代表的话,你知道发生了什么事情吗?

今天聊天之前,好多朋友千叮咛万嘱咐,说点到为止、点到为止,别伤害到平台。我说好,政治这不谈了,那我们谈谈经济现象吧。经济现象其实不用我来说,因为在改革开放之后,在经济上出现了一系列的问题,经典的买办机构我就不说了,还有一些不是派遣代理人,是自发的代理人,类似于某想、某想,还有一些盲从的代理人,我也就不举这些具体的这个机构的名字了。我只是想说,请大家多多关注我国金融机构的构成,我说的不是股权构成,我说的是管理权的构成。

代理人僭越之后,形成了后殖民主义经典的文化特征、政治特征、经济特征。这个特征非常之明确,如果你没有听,如果你没有听今天的聊天,或者是你没有一种主动的思考,你很难对这个事情产生一种敏锐和敏感的判断。你知道因为不敏锐、不敏感,就在这样的一个气氛中,香港就衰落了,台湾就离了。就在这样的气氛中,崛起中的日韩可能正在走向深刻的灾难。后殖民主义问题,难道对东亚……

后殖民主义问题,难道对东亚不是压倒一切的重大问题吗?在拉美、在北非、在中东,反对后殖民主义的轰轰烈烈的运动最后全部失败,因为他们不可避免的走进了两个陷阱,一个是激进的民族主义,一个是激进的民粹主义。我刚才讲了民族主义最后变成原教旨主义,民粹主义就变成了恐怖主义。所以激进的民族主义和激进的民粹主义最后的发展的过程中,使他们本是正义的争取变成了一场惨烈的东西方的文化冲突。

我们中国能否走出这两个陷阱呢?我是有信心的,原因是因为我们在、我们在,我们可以把后殖民主义的事情说清楚,把它抖搂清楚,抖搂明白,把它搞明白。我们既坚决的反对后殖民主义,我们也不踏入后殖民主义的陷阱,我们走出那个千难万险中间的那条正确的路,完成对后殖民主义的超越,终结五百年的殖民历史。我们不会重复拉美、北非、中东的错误,我们也不会重复日韩、台湾、香港的错误。

那么如何重建民族国家的主体性呢?重建、重建民族国家的主体性,我刚才说了是两个主体性,一个是国家的主体性,一个是人民的主体性。国家的主体性我们在1949年的时候大体上建立起来了,我党领导中国人民的革命第一个百年解决得最好的是国家主体性。人民的主体性,有一个人尝试进行,把它制度性的建设,这个人就是教员。教员为此发动了十年轰轰烈烈的“那个”,后来证明失败了。

然而,任何一件事情都是有意义的。教员想为人民争取到最高的政治权力——立法权、司法权、行政权。但是什么事情走过头了,当砸烂公检法的时候,当司法权和行政权争取的时候,反而把这个事情走向了反面,连立法权最后也丢了,所以后来又变成明星了不是。但是它是一次挫败性的经验和教训,所以它对未来的发展有十分重要的意义。你能说巴黎公社没有意义吗?你能说十年没有意义吗?每一次都是我们深刻理解后殖民主义的、有益的……

每一次都对于我们理解资本主义、理解殖民主义有着深刻的意义。所以在这个基础上,我们再来审视我们的国家、我们的人民、我们的历史、我们所面临的东西,才有意义。我甚至认为正是由于那十年的尝试,我们今后可以避免了两激,两激就是激进的民族主义和激进的民粹主义。我们有过尝试之后,我们知道有些路是不能走的,这就是历史事件的重大的历史意义。这是我们做出的第一个思考,就是我们必须尝试建立国家主体性和人民主体。

第二个思考非常重要。我们是读《资本论》的,我们是学马克思的。生产力决定生产关系,生产关系决定上层建筑。生产力决定生产关系是什么意思呢?你的人、你的普通的人民、普通的劳动者,人均收入不过一万美元的话,你没有物理条件或者物质条件反抗殖民、反抗后殖民主义。也就是说首先要发展经济、发展生产力,由无产阶级变成中产阶级之后,你才能系统性的反对后殖民主义。

否则很容易沦入激进的民族主义和激进的民粹主义。因为当人民认识到后殖民主义的问题的时候,他们会极其的激烈、暴躁、盲动。我们在那十年见识过了,它是一种剥夺,一种剥夺的方式,而剥夺的方式并不能解决人民整体上从物质属性上面的升级,就是由无产阶级变成中产阶级这个升级。请注意,这是非常重要的物理条件或者是物质条件。物质条件不具备,精神条件具备了是走不出来的,我们是唯物主义者,这一点非常非常重要。所以邓小平的改革开放是有重大意义的。

不要轻易的否定,既不要轻易的否定那个十年,也不要否定后边这四十二年、四十三年。在冥冥之中,我国,我们这个伟大的党,我们这个伟大的国家走的是一条正确的路,这条路非常之艰难,还剩下二十八年,一点儿都不能错,所以我们今天这堂课,我们今天的聊天就变得极其重要。中国的中产阶级的崛起,才意味着中国新社会主义者的重生与崛起。难道无产阶级不能成为中国的新社会主义者吗?结论是,

我的结论是非常明确的,是中国的中产阶级将成为中国的新的社会主义者,他们才能战胜后殖民主义,并且走入新时代。无产阶级要解决的问题,首先要解决的是贫穷,而不是简单意义上的、简单意义上的、简单意义上的,因为那个简单意义上的革命已经被教员做完了,所以两次,一次是武的,一次是文的,做完了。而后边的,1949年之后,更主要的工作是完成物理意义上的建设。我知道很多左翼的朋友,是不同意我的看法的,慢慢讨论吧。

许多左翼的朋友一直没有终止跟我的辩论,直到现在还在辩论。我展望一下子未来吧,请大家注意时间节点:1994年我们进行了分税制改革,我们将一部分财政主权交给了西方;1995年我们联系汇率,我们将一部分金融主权交给了西方。就是我们后殖民主义再次崛起的标志性事件就是:1994年的分税制,1995年的联汇。我反复重复,就是我们要有哲学的高度和历史的纵深来看问题才能客观,不然的话肯定是一种错误的看法。

1994年的分税、1995年的联汇,错了吗?没有错。中国是一个资本稀缺、市场稀缺的国家,不进行分税制改革、不进行联系汇率,我们无法集中足够的社会资本与国际资本,进行迅速的物理建设,这两件事情是对的,在特定历史时期就应该做正确的事情。所以历史的看、从哲学的高度来看,1994年的分税制和1995年的联汇是对的,然而十年之后,2004年、2005年我们大体上解决了资本稀缺和市场问题之后,就应该进行反思了。

还好、还好,我们遭遇了,不是遭遇,我们有幸遇到了的习主席为首的新一届的领导集体、党中央,他们在2014年将分税制结束了,央地税合并,请注意时间节点,2014年央地税合并,2015年取消了联系汇率。注意这个时间节点,很了不起的,如果你懂什么叫后殖民主义,你知道2014年并税、2015年脱钩,完成了我们特定历史阶段的那个、那个、那个向西方的主权的让渡,把它主权收回来。

由此你能理解2016年之后,不管是希拉里上台还是特朗普上台,必然与中国翻脸。能理解了吗?因为并税是财政主权,脱钩是金融主权,汇率是脱钩是金融主权,两个主权拿走,是你或者是说我们,在从西方那个地方、主要是从美国讨回主权、讨回主权,当你讨回主权的时候,是你率先脱钩的,所以别人会激烈的反弹。所以我们看到后边的贸易战、科技战、金融战,一系列的战,这是一个必然的结果。我认为2014、2015年,都晚了,但这个时间节点还好吧,没有到翻……

没有到翻车的时候转身。所以我们感谢习主席、感谢这一届的领导集体,了不起呀。这件事情饱受西方诟病、饱受我国精英诟病、饱受殖民者诟病、饱受代理人诟病、饱受买办诟病,说什么难听的都有,包括现在对房地产的问题,包括对新东方这种语言的问题,包括等等,包括互联网问题。我们能丧失国家主权吗?我们能丧失人民主权吗?我们难道不需要将主权拿回来吗?你拿回来以后,西方的反应太正常了。好多朋友问我,为什么美国和整个西方反华呢?

请问,当你的党、你的政府,旗帜鲜明地反对后殖民主义的时候,殖民者的反应应该是什么?难道美国、五眼联盟,甚至包括欧盟的反应不很正常吗?确实是厚颜无耻,包括对新疆问题、香港问题和台湾问题,殖民主义者的嘴脸确实厚颜无耻。但整个的反应,如果你放在这样一个哲学框架之下,放在这样的一个历史纵深中来看的时候,不是一目了然吗?是不是一目了然?所以我们现在,在2014年并税、2015年汇率取消联汇之后,我们那是进一步的完善和强化财政主权、金融主权。

我国终于吸取了我们在香港问题上的教训,就是我们1983年12月,放弃了香港的金融主权,1984年放弃了香港的财政主权,回归之后我们没有经济主权,没回归啊,在人家手上啊,政治主权、立法权不在手上啊,只有一个行政权。行政权,昨天晚上好多朋友说得很精彩,董伯伯带着秘书、助理、司机四个人去接管的公务员团队啊!行政权在你手上吗?所以我们懂了,这件事情貌似小事,其实惊天动地啊!它是大事,所以我们生长在一个幸运的年代。

我对2022年有很大的期许,我认为2022年,特别是二十大,中国将开启中国式的大宪章运动与光荣革命。我讲过大宪章与光荣革命,今天我再多说两句。大宪章运动就是二十五男爵带剑议政,他们没有杀掉国王,只不过是跟国王商议,跟国王商议,这个二十五个男爵交多少税,让国王建立一个海军来保卫英国。大宪章运动实际上是英国的税政改革,是议税。议税运动,最后变成了议会,后来觉得国王的管家不靠谱,给你派个管家——就是首相。

“大宪章运动”之后,那国王又折腾,后来就是“光荣革命”。“光荣革命”就正式的将现代的“君主立宪”形成了英国的现在的宪政体制。请注意,请注意,无论是“大宪章运动”和“光荣革命”,讨论的都是一个事情——税、税政。税政为什么重要?就是军政、民政全是在财政基础上,那财政的基础是税政。税收的意义为什么这么重要?因为它决定资本的流向,因为它可以最大限度地将资源进行最优配置,提高一个国家的生产力水平,使一个国家强大起来,所以大英帝国的强大就是税政改革来的。

我已经不再跟左翼的许多朋友进行理论上的沟通,因为他无法理解,他认为税政改革怎么能是宪政改革呢?税政改革为什么会推进政治文明呢?他理解不了。因为不读历史嘛,所以理解不了。他认为就得先革命,就得要先造反,就得要先把政治制度改了,才能去改财政和税政,否则就改不了。糊涂啊!二十五男爵带着剑谈的是税政好不好。再选一个新国王吗?新国王可能比那个更烂。所以我在有的时候觉得此时此刻的中国人要保持高度的哲学高度和清醒的历史纵深,所以2022年这二十大非常重要。

我们终于进入到中国的一个主体性重建的伟大的历史进程中了。因为如果2022年二十大我们开启的直接税的立法工作——就是税政改革工作,我们将是开启中国式的“大宪章运动”与“光荣革命”,它是具有极其深远的历史意义的。其实在这个基础上,可能将中国……,因为通过税政将中国的什么问题解决?我们用《资本论》的理论来概述一下:我们将解决劳动所得与资本利得的根本性的关系。

如果中国的税政可以均衡的解决劳动所得、资本利得,甚至可以更细化解决生产过程的资本利得、商业流通的资本利得、金融资本利得、解决不同的行业的资本利得的水平,使它均衡。平衡劳动所得与资本利得,同时再平衡所有不同行业的资本利得,我们将构建一个极富效率、极高文明程度的崭新的社会形态。这一点太重要了,要不学《资本论》做什么呢?如果你问我什么是社会主义,我刚才说了,那就是我们想追求的新时代有中国特色的社会主义。

在处理这个问题上,我还是主张从欧洲迎回孟子或者是接回孟子。我们主张就是孟子去了欧洲,将基督教改为基督教新教,最后衍生出共产主义,衍生出马列主义,主要是马克思主义。我不愿意把它概述成列宁主义。列宁主义怎么说呢,这是一个非常矛盾的问题,因为我觉得国家资本主义有它存在的合理性,但它真的不是社会主义,不是马克思本意上的社会主义。我们从欧洲迎回孟子,重新梳理人类文明的最高境界。

昨天香港的朋友说,我们这个民族哪都好,就是没有信仰。其实当我们从欧洲迎回孟子,我们不但是处理信仰问题,我们还要将基督教新教伦理所形成的西方的现代工业文明与东方的关于儒家的思考来一次历史性的大融合。这个融合才是我们进入新时代、超越后殖民主义所必须依归的那个新的理想主义或者是理论,否则我们没办法解释“新时代”。

中国的努力将创造一个新的范式,在超越后殖民主义之后我们将建立新的范式。这个范式其实不仅仅是符合共产主义理想的,是符合马克思主义本意的,也是与基督教新教伦理特别是与基督教新教信义宗完全吻合的一整套体系,我所以才说从欧洲迎回孟子。在讨论这个问题的时候,我们不得不说一下子就是我们因为我们一直在,中国在走出后殖民主义的时候,必须处理现在压迫中国的《华盛顿共识》——就是所谓的“新四化”:就是私有化、市场化、资本化、国际化。“新四化”……

“新四化”对不对?当然对呀。好多人说麒元你疯了吗?没有私有化,如何社会化呢?社会化是社会拥有,私有化是社会化的一个部分,国有化是反对社会化的,好不好?但是每件事情都有它的边际、有它的形式,这很重要。国家资本主义存在是有意义的,国家资本主义、社会资本主义与国际金融资本相融合,它是有它意义的。私有化要有正确的理解,市场化对不对?不市场化行吗?难道易货吗?但市场化是有边际的,我为什么批评厉以宁、吴敬琏呢?没有边际,将它当成宗教信仰是错误的。

资本化对不对?不资本化怎么行呢?必须完成对资产,特别是对生产型资产的资本化。当然资本化是有边际的、有底线的,完成这个资本化使之能够提高生产率,在生产过程中顺畅地流转。这也是《资本论》第一卷、第二卷,特别是第二卷谈资本流转里边反复强调资本化这个过程是非常重要的。中国现有的资产我估计超过三百万亿美元,可资本化的进程远远没有完成,还早着呢!所以中国经济增长为什么我有信心?一方面是我们创造价值的部分,还有一部分是资本化的部分,最后是国际化。当然是国际化了,因为中国很快就……,已经是世界上第一大的生产体系和贸易体系了。

国际化是必须的。我们反对《华盛顿共识》,不是反“新四化”。请牢牢记着,我们是在“新四化”的内容上设立它的底线和边际,这才是水平嘛。所以我们强调国家主体性、强调人民主体性,我们强调在新的税政体系里边建立起劳动所得与资本利得的平衡;我们强调在资本利得的不同的类型里边,比如说生产资本的利得、商业资本的利得、金融资本的利得,建立起均衡,使资本、使资本流入它该流入的地方,形成极高的效能,使中国能够顺利的完成今后二十八年的使命。

哎,一不小心就讲了七十分钟,我想今天就讲这么多吧,有点累。关于经济的问题、经济形势的问题,我想下一堂课第二十讲《资本论》的时候,我们花一点时间再去讲吧。今天我就不谈钱的事了,不讨论投资了。新的一年开始了,总结是要做的,我们每个人要认认真真做总结的,总结里边得加一条——你身上有没有后殖民主义的痕迹呢?当我们国家完成超越的时候,你、我是否也应该完成历史性的超越呢?

好吧,最后强调一件事情,就是关于冠状病毒。因为这个……,我原来想把资料发给大家,后来想了想,不发吧,就是冠状病毒解剖结果发现:有很多冠状病毒这个已经好转了、复阴的人,最后在脑子里边又发现了;而且大部分的已经好了的人其实死了之后(已经过了二百三十多天了),解剖之后还是发现了有冠状病毒的存在;而且以前认为大部分的病不是死于冠状病毒,后来的结论是百分之九十以上的还是死于冠状病毒,因为它在大脑和心脏形成了血栓,形成了一些问题。

埃博拉病毒给我们很多的教训:就是埃博拉病毒的第二次的兴起是因为一个埃博拉病毒的病人好了,但是那个埃博拉病毒藏在他的是脑干还是心脏的地方,后来五年之后复发,复发以后又出来。因为我们担心的就是这个冠状病毒,它这个、它这个真是跟人类作对,它可能会比较麻烦。所以,我们要以极大的耐心这个先坚持清零;另外我们每一个人要真的还要坚持戴上口罩,减少活动范围,尽可能的避免出现这个感染,我们等待有更好的疫苗和药物出来再说吧。

再次恭祝大家新年进步、新年健康、万事如意。好,今天的聊天就聊这么多。好,我们明天下午三点钟再见。

\subsection{对新冠病毒、美国资本市场和中国经济的一些看法、柯立芝繁荣}

大家好,今天是2022年的1月15号,是辛丑年十二月十三日。今天是聊天,按照原计划呢,我们今天是聊“柯立芝繁荣”,这已经推过一次,所以今天不能再推了。我们今天就聊“柯立芝繁荣”,但会加进一些对当下的一些重要事情的判断,包括对冠状病毒这个防范策略,以及对美国的资本市场的一些看法。看看时间吧,反正我们今天是聊天儿,尽可能把大家关心的事情一起聊一下子。好,一会儿见。

好,大家好!今天是2022年的1月15号,辛丑年十二月十三日,距离春节还剩下17天。辛丑年就快要结束了,这个辛丑年还真的是在最后的尾部形成了一个巨大的震荡。这个震荡确实值得我们自己好好的揣摩,我们虽然对此有一定的预见,但是是以这样的方式到来,确实也让人感到震撼。我们主要是了解一下子壬寅年情况。

今天也不知道为什么公司的信号非常差,总是发不出去。今天我们讲“柯立芝繁荣”,然后我们讲一下子这个冠状病毒和这个美国的经济的情况。其实关于冠状病毒的事情呢,我们在课上或者是在文字上断断续续也有一些说明。最近这个比较影响大的是美国评出未来一年最大的一个不确定事件,第一项就是中国的“清零”,美国人认为中国的“清零”,给全球经济造成了巨大的不确定性。

难道真的是这样吗?难道是中国“清零”的政策给全世界造成了巨大的不确定性吗?我看不是这个意思,真实的意思是由于全世界都处在非常麻烦的对抗疫情的过程中,只有中国处于清零状况,所以他维持一个完整的产业链体系、完整的一个经济生态。这个用美国人、欧洲人的视角来看,这不公平。但他们又做不到清零,所以他们希望中国采取一致化的行为。关于这一点,其实我们的看法也有分歧,就是由于我国国内也有网红赞同清零……

我国国内也有一些专家、学者或者是一些网红不完全赞同清零,认为应该选择共存。我对这件事情的解释是基于对财政学的理解,从财政学的角度,我们算过账,全国清零,2020年、2021年的费用都是百亿级别,可承受。如果不清零,中国人的感染数量应该是千万级别,而重症的级别可能会非常之高,而这个死亡的数量可能会超越美国。美国现在86万,因为中国人口基数大,那么可能也是百万级别。

我们认为,如果是采用共存或者是所谓躺平的做法,那么中国有两个不可承受:一个不可承受是财政,用于处理和救治的这个费用保证是万亿级别,这个财政部这边有测算,保证是万亿级别,那么就是百亿级别的一百倍,懂财政的人都知道,这事儿玩不起呀;第二件事情是,如果我们是千万级别住院、百万级别死亡,那意味着什么呢?意味着中国经济陷入全面的衰退,甚至是严重的经济危机,那导致的经济增长的下滑可能就是负值的情况。

易言之,财政支出增加一万亿,经济增长减少三万亿到五万亿,这主意能听吗?这不是一般的馊主意啊,这是害我们的主意啊!我不认为那个张先生——上海那张先生,因为作为一个传染病专家,他对防疫如此复杂的系统工程真的理解吗?他懂得财政吗?所以有些时候有些人的话听一听就算了,因为他们不用去承担最后的结果。我国的选择好多人认为是智慧,其实不完全,无奈呀!你不这样选择的话,你扛不起呀、你受不了的嘛!这是一个非常简单的问题,但现在变得很非常复杂。

另外,就是现在,无论是美国的各类型智库、媒体,也无论是整个西方的各类型的智库、专家、学者和媒体,都在对目前这个疫情做出一种判断。当然他们对我国的策略意见很大,这个我们能理解,因为我国的策略的成功印证了他们的一个失败。但是各类专家的这种传媒上的这种解读呢,我看完以后非常惊讶。这周我跟香港的朋友有一次聚会(虽然不让聚会,我们是不吃饭的那种聚会),其中有个学数学的,他讲了一番话震到我了,我听完以后觉得非常的震撼。

他说抗疫策略是一个运筹学问题,它是个纯数学的问题,它是个最优解的问题。抗或者不抗,抗到什么程度、抗多长时间、在多大范围抗,他说这是个运筹学问题,是个数学问题。他说,目前中国可能不自觉地处于最优解。他说这个事情呢,你如果要是把这个账统一算完以后,他说会非常非常震撼,就是你用运筹学的角度来看中国的抗疫策略,他说会非常非常的震撼。因为无论是从人道主义、人权的角度来考虑,还是从经济利益的测算的角度来考虑,这是一个非常聪明的选择,他说应该点赞的呀,而且必须点赞。

这个朋友说完全无法理解美国、欧洲。他们发生的是一个智商的问题呢,还是个治理的问题?如果是个智商问题呢,他说那么多的学者、专家、数学家,难道不知道什么叫最优解嘛?如果是个治理问题,那他说就可以理解了,因为整个的疫情的发生,防疫、治疗存在着巨大的、难以想象的经济利益。当然这个经济利益可能是一些机构的、个人的利益,它与国家的整体命运可能是反向的,也因为存在巨大的经济利益,有可能导致真正的学者和专家声音出不来了。

其实我们现在也是走在一个非常关键的时候了,因为这里边有三个问题。第一个是人是有一个耐力极限的,就是你采取一个状态,时间过久了以后会超过人的耐力极限,人会产生一种疲倦,甚至反抗——本能的反抗。“清零”这个策略可能对中国人的心理和生理的压力都是巨大的。同时,一种事情让所有人都理解或者所有人都受益,这是困难的。他说:“至少现在很多很多人,特别是中国比较高端的家庭,

中国比较高端的家庭通常都会有境内、境外这样的问题,所以他们会遇到不能相见或者是困难、诸多困难,所以会产生一些怨言。”同时,在承担巨大的筛检过程这个工作也给相应的这个机构和人群带来了巨大的压力,也可能带来巨大的问题。第三个呢,这个疫情有受益者,也有受害者,其实那些受害者会采取激烈的意见或者是反弹,甚至人们会把中国目前出现的一些企业状况、甚至失业状况,把它导入到防疫策略里边来,所以还是挺难的。

我在这里想说两条。第一条首先我们国家采取“清零”,我说了是无奈的选择。目前不是我们懂得从财政计算或者是运筹学角度,它是最优解,是无奈的选择。因为我们放弃这个选择,我们可能面临我们完全无法承受的、残酷的经济危机和社会危机乃至于政治危机,这是一个无奈的选择,不要去想谁更聪明。第二条就是如果在全球博弈的角度来看,我们坚持得越久,可能我们获得的可能性就越多,机会可能就越多。所以我们必须……毛主席说的“胜利往往在于再坚持一下的努力之中”。

至于说目前这个疫情会怎样走,一共是三种可能性。最坏的可能性是再度变异,一旦出现一个新的传播速度和毒性都极高的新的病毒,人类将面临一场浩劫。不是没这种可能性,所以我今天为什么把它放在头里,还是建议大家要有足够的心理的准备,同时可能也要做一些生活上的考量。不是不可能穿透你的防线的,“清零”是一个非常残酷、危险的边缘作战,它的难度之高是可以想象的,所以仍旧做最坏的打算。

其次,如果这个病情,就是第二种可能性,最坏的可能性说了,第二种可能性就是它真如现在西方专家所言,它的毒性是弱的,那么它就是一个大号的感冒。在毒性弱到一定程度,那么我国随着疫苗的普及、推广,可以采用医疗或者中医的方法能够降低它的死亡率的话,那么在适当的时候,也许是2022年下半年,也许是2023年,封建迷信一点讲,2023年这个水的这个状况结束,开始进入到火的状况,可能病毒就会好一些或者是彻底消亡。

第三种可能性就是出现特效药,出现特效药之后这个病就不是个问题了。这个特效药包括两种可能性:一个是特效的疫苗,打完以后就没事了;另外一个就是特效的治疗用的药。这两个事情有一个事情出现,那么这件事情也就结束了。三种可能性,现在我们期待第二种可能性,争取第三种可能性。不希望,我们祈祷不希望是第一种可能性,因为这场浩劫如果爆发的话太残酷了。因为按照香港的这边风水师讲,这个处在一个长达12年的蛊卦的周期,所以他们认为是非常麻烦的。

好,封建迷信这段就讲这么多,我们聊几句美国市场吧,最后再讲柯立芝繁荣,因为柯立芝繁荣是一个知识性的东西,它没那么急迫,当然它很有趣。这个美国的这个市场很有趣,因为鲍威尔当选了,美联储这个天一亮就是“鸽声”,天一黑就是“鹰声”,就是一会儿“鸽”一会儿“鹰”的,所以搞得市场有点乱,搞得中国的资本市场也一片萧条。这个事情我觉得至少我们平台上的朋友应该懂得,

平台上朋友知道两年前我们已经开始“短股长金”了,因为知道处在一个特殊的历史时期。处在一个特殊的历史时期,其实那个时候还没有疫情呢,我们已经说了它是一个通道,它在一个通道之中。特别特别厉害和聪明的人当然可以博弈一下子,因为整个的这个这两年中也存在着大量赚钱的机会,比如说你可以考虑买比亚迪,比如说你可以考虑买完比亚迪以后再买这个航运股,比如说你可以买台积电等等等等等等,就是机会确实是存在的。

但你知道不是每一个人都那么厉害的嘛,那么大部分人这两年在通道里要做的什么事情呢?就是别损失,不输,当赢。“短股长金”在很大程度上就是为了这么个意思,你蹲在、躲在黄金屋里边,你没事儿不就挺好了吗?如果发生一些我们预测的事情,那么这个金可能会给你一个回报。如果没有发生,你是安全的嘛。所以好多朋友批评我,我说你这个可能是要求过高。你要求我带着你在通道里边还赚个3倍、5倍,这不是不可能,但这不合适吧,这不合适。

我有一个好朋友,在香港在这个投资界的地位极高,属于那种在香港快要登顶的人了。去年呢,他们输得比较厉害,输得比较厉害。其实这个事情也是可以理解的,因为他们管的东西大,所以他们有时候投资的时候,你说他能避开腾讯、阿里吗?他能避开科网吗?他避不开的。你说他能提前躲过“教培”吗?躲不过的。你说他房地产类的东西一点儿都不沾不碰吗?不行的嘛。所以他们输得很严重,所以跟我说的时候,他说:“你们聪明,你们聪明。”

我说:“我们是笨,不是聪明,是真的笨,因为不知道会发生什么。所有的选择,在短线上翻倍的选择,包括在比特币和在元宇宙里边的选择都是超短线。而我的,我要面对的平台上面的朋友,短线的技术还到不了炉火纯青啊,万一陷进去,其实可能就是一个悲剧。所以我们一直以来是持盈保泰。”他说:“如果我们也按照你的说法去做的话,那么就可以避过这个邪恶的周期了。”我说:“嗨!”我说:“各有利弊吧,各有利弊。”

看美国问题,别看报纸、别看电脑、别看机构学者年底的胡说八道,千万别看,看完了你想不上当都难。就我这身边这朋友都已经是顶级的啦,他已经远比国内的机构的首席们强,不是强一个百分点,是强很多很多,就不是一个等级。因为在此之前,在这两年之前,他的年的回报水平一直在50\%以上,是非常厉害的。2020年还能说得过去,这2021年一个跟头倒下,非常麻烦,但也不至于到了不可救药的程度吧。

我想说的是,我看美国经济通常是使用财政分析的手法,我不用金融分析。我读美国预算超过,连读超过26年,不光读联邦预算,也读州预算,大体上我知道它走到哪儿了,它这个病怎样、走到哪儿了,我是清楚的。好多人、好多人病态,在那等着美联储加息、不加息、点阵图,在守这个东西,有意思吗?当然没意思啊。你们认为那些很厉害的人经常到我这儿来,想了解下我们对财政的分析对不对,我就不说名字了。

财政分析的结论是非常清晰的。我这样说吧,美国财政决定美元的价格,美国的金融决定全球资产的价格,就是它决定美元,而美联储决定各国的资产的价格。你看美联储有没有用?有。但那不是美元的命运,美联储貌似在决定美元的命运,不是的,决定美元命运的是美国财政。现在我们看完2021年,其实非常明确、非常明确,美国的税政完全不能支持美国的财政了。

到2022年,它是从10月1号开始拜登的财政年度到明年的9月30号,这个区间里边,美国税政能贡献出来的收入是三万亿美元。因为现在看拜登是没有办法加税,现在看耶伦并未采取我认为需要采取的堵漏洞行动,就是美国税法的漏洞,她没堵漏洞,也没增加新的税。那么旧的税在现在这个情况下能提供多少的税收呢?大概是三万亿美元。而我们可预见在这个财政年度里边,美国的财政支出,

美国的财政支出几乎没有悬念要过六万亿,这是一个非常残酷的现实。我说这个的意思是说加息、缩表不影响这个基本的判断,这是基本的格局。加息和缩表是技术上的处理,它不影响结构性的。结构性问题和技术问题是两回事,结构性问题决定美元的价格,所以我个人认为美元对其他的货币,包括欧元在内的货币在特定时候会完成它自身的价格重置。

而我国在处理人民币与美元的关系上,我看到他们的犹豫、犹豫。有很多很多的中国人,包括专家、学者、从业人员,甚至包括各个机构的首席都认为人民币升值一定会影响中国的出口,那么出口影响就影响中国GDP增长。这个逻辑成立吗?我想说的是这个逻辑不成立,不但不成立,而是逆反的。难道人民币出现土耳其里拉那样的局面、阿根廷货币的局面,会大幅度增加中国的出口吗?

那么很多朋友会问我:“人民币坚挺为什么会增加人民币的出口呢?”我想说的是,在讨论人民币国际化的这个问题上分四个层面:第一个层面,是人民币在国际贸易中的定价权,什么东西是以人民币计价;第二个是在全球贸易中人民币交易的份额,这很重要,计价和份额;第三个是人民币在各个国家外汇储备中的比例;

第四个是人民币他国国家,就是政府、机构持有的人民币债券、人民币资产的比重。我说清楚了,是四个方面。这四个方面,为什么说人民币越强,中国的出口会越好,中国的金融开放会做的越好?是因为你越强的时候,别人才拿你的货币做计价依据、做贸易的使用媒介嘛。这道理,谁都知道美元要贬,他就不太愿意用它了嘛,包括印度都与人民币挂钩了嘛。

当然,你说我国央行的先生们一点儿都不懂这件事儿吗?也未必就一点儿都不懂。你说他们全都是坏人故意在捣乱吗?我也不这样看、我也不这样看。但你说这件事情中国的金融的管理层,就是来制定政策的人,都是心明眼亮,并且能够按照正确的规律做,就像我们抗疫一样做最优选择吗?我也是觉得没有那么踏实的。因为看到最近他们在处理相关政策,包括人民币的政策、包括中国的储备政策、包括对黄金和碳排放储备的政策、包括一系列的政策,其实我们还是忧心忡忡的。

但即便如此,其实人民币升势依旧是锐不可挡。如果,反正我教过大家一板斧,你看MACD人民币的月线图,其实你明白是不可逆的,就是人民币的升值趋势是不可逆的。我们现在讨论不是讨论它升不升,是讨论什么时候升,如果出现升的话在哪里停下,停下的那个点。就是它一旦进入5时代,是今年进入5时代,进入5时代以后在哪里停下,这是我们要考虑的。不然你怎么配置资产呢!讨论问题只能这么讨论好不好!

年底年初好多因为照以往的惯例,这个时间我应该回到北京。北京的好多朋友我们这个时候该见面讨论问题,但今年回不去,好多朋友是通过微信或者是通过电话来讨论对宏观经济的一些看法,交流一些看法。以前我们可以飞来飞去,他们飞过来或者我飞回去,现在只能是这样讨论。香港这边也比较紧张,所以在年里年初,我们尽可能的花一些时间做一些沟通。现在因为不允许群聚嘛,所以我们只能是在办公室喝喝茶,把事情沟通一下子。其实对整个的大势的判断是清晰和明确的,特别是对美国的判断是清晰而明确的。

拜登错了。其实拜登这个事情给大家很多启发,因为拜登是一个老政客,一个老牌的,他们都认为他是懂外交的一个政客。他绝对不是政治家,因而他们认为他应该采取一条,可能(因为他八十岁了),可能比较稳妥的路线。但事实证明他完全不懂外交、他完全不懂外交。更证明一条他对经济理解到一塌糊涂,就是他对经济理解是完全错误的。如果说特朗普耽误了美国的四年,那么几乎我现在可以给结论——拜登将再耽误美国四年。

好多朋友说:卢先生你总是那么的执着。我说是啊,中国的问题多简单呢!直接税立法就这么就这么一件事儿。美国的事情多简单呐,税政改革。它不仅仅是税政,它不仅仅是直接税立法,它是必须进行深刻的税政改革。可是你知道这个世界上所有的人都不喜欢去碰那个最难的事情、费力不讨好的事情,所以我们也在直接税立法上面遇到了一些障碍、一些困难。美国税政改革,当桑德斯、沃伦等提出之后,你看给压的——民主党内部就过不去,共和党就更不可能,那么它税政改革就上不来。那么只能粉饰太平。

愚蠢的美国人连以工代赈都不懂,非要直升机撒钱。撒钱的后果非常严重,撒钱的后果非常严重。如果是以工代赈是在创造价值可持续的过程中,而撒钱的过程中它不是以工代赈,它是用赈亡工啊。它这个整个这个赈是毁灭,对整个的工业的、工商业的一种毁灭。这个代价现在还没表达,这个代价要到今年的下半年才能够明确地表达出来,所以美国的经济的问题的严重性现在还没表达,这也可能是共和党人做的一个蛊啊。因为民主党照这个样子是一点机会都没有了。

美国不仅仅是我们理解的字面意义上的7\%的通胀,通胀是很高很高了。不仅仅是我们理解字面意义上的通胀,因为最近我也在写一篇东西,给香港的一个刊物写一篇东西,他希望我把通胀问题说说清楚。以前大家知道3\%以下那个通胀叫合理通胀,当然低过0就是通缩了。3\%以下通胀叫健康通胀、合理通胀、好的通胀。3\%以上应该分两个,一个叫高通胀,一个叫恶性通胀。所以好多朋友问我7\%是高通胀还是恶性通胀?我不太愿意用CPI来陈述这个问题,我通常会用、

我通常会用实质负利率来陈述这个问题。所以我在给你们的,我记不得是聊天还是课程上我们讲过这个通胀问题,就是用实质负利率、还原后的实质负利率来讨论这个问题。那么我们认为今天的美国的7\%仍然属于高通胀范畴,没有进入双位数,如果进入双位数的话就是恶性通胀。我们一会儿讲柯立芝繁荣的时候,我们会讲到最后的结果,它就是越过了高通胀的那种最后的门槛。越过的原因是机会主义,就是大家认为可以再看看、再等等,跟今天情况非常像。就是今天美联储在缩表,什么时候加息、什么时候加息、什么时候加息?它在缩表,讨论如何加息、什么时候加息的问题。时机在一点点的流逝。

一如既往,我们讨论这个问题根本就不是为了,就是好像互联网上好多人说你说对了,你猜对了,我们不是为了对错,我们是为了为未来一个较长周期可能发生的事情做一个预判。然后相关联的投资安排能够做好了,为未来五到十年我们能有一个比较好的成长吧,做一个安排,我们的目的是在这儿。我们不去跟别人逞能,说你看我什么时候说了,你看它到了。我们不去玩这个,这个没有意思。因为其实嚷嚷声音大的人,其实晚上回家躺床上的时候,他可能心里会不舒服。不嚷嚷的人,这数完钱年底他可能会心里边很舒服,其实嚷嚷有什么用呢?

美国的资产的峰值、美国股市的峰值已经到了。今天我们可以确定的讲美国的资本市场的峰值已经到了。什么时候开始狂泄,什么时候开始泄或者是开始进入狂泄的阶段呢?我个人认为六个月之内保证见到了,六个月之内可以见到。原因是我们对支撑美国股市的那几根支柱做了比较细致的分析。这个年底年初好多朋友聚在一起我们在讨论苹果,过了三万亿美元这个苹果它到底值多少钱?难道它能变成四万亿和五万亿吗?

特斯拉到底值多少钱?特斯拉明年的,不是明年,已经是今年了,今年特斯拉的销量潜在的增长幅度是多少?中国的电动车市场或者中国电动车在国际市场的份额,今年应该是个什么样的状况?其他几个重要品牌德国车和日本车在这个市场上,传统车市场上缩减的比例是多少?新增加的电动车的比例是多少?其实大家都在算账、都在算账,这里边特斯拉的份额大概能有多少?美国这几间头部公司未来决定着美国股市。最后,

其实挺有意思的。讨论这个问题,有一个熟悉这个电子的,熟悉电子,就是熟悉数字经济、知识产权的一个朋友,他跟我说了一段话,我觉得很有趣。他说:美国没钱,那么就会向苹果等高科技企业开刀,知识产权问题、垄断问题。他说,现在欧洲、甚至包括英国在内,以法国为代表的政府肯定会向美国的高科技企业动税的,就是他肯定会,无论是数据税、还是反垄断税都会动手。

现在最客气的、最斯文的就是中国。中国政府对美国的高科技企业采取的一个百般呵护的,那么的政策态度、政策态度,就是。但你能保证美国在向中国科技企业下死手的时候,中国政府一点儿事情都不做吗?如果美国都做,中国也跟着做一点,欧洲再做,那么美国的头部的高科技企业今年的政策环境当如何?要知道这些企业是靠生态赚钱,不仅仅是产品,是靠那个APP赚钱的。

此外,在技术上、在技术上,像苹果、像微软、像特斯拉,在技术上领先程度在与追赶者的距离在迅速收缩,而不是拉大,不是在拉大,而是在收缩。也就是说后边在硬件上的追赶的速度很快,甚至在APP上面的进展速度也很快。而他们遇到的是前后的夹击:一个是后来者的追赶,一个是前面政府设置的政策障碍。所以,所以我们在聊天的时候,他们说,可能2022年将是美国头部企业最辉煌的最后一年。

那么美国的房地产还有空间吗?这个我请教了在美的朋友,因为确实美国的房价涨得很疯,因为原来我有与美国同事、美国这个一起做董事的同事、朋友,我在听他们的意见,因为他们对房地产非常熟悉。他们的看法是:如果依靠美国劳动者的薪资成长来支撑美国的房价增长,这是不可能的;因为去年好像是35左右吧,薪资增长没这么高;如果利息趋近于零,增加杠杆,

那么美国还可以,美国楼市还可以再繁荣一段时间。但事实上按照现在通胀水平,减息的可能性是没有的,加息的可能性是有的,加息的速度可能是快的。那么美国的普通劳动者能买多少房呢?最残忍的是:马斯克他们都把房卖了,其实他们知道,未来美国经济不行的时候肯定会税改,税改里边最核心的肯定是直接税,所以有钱人在转向其他领域,而穷人正在疯狂地进入,但这个极限也将在2022年冲顶而结束。我们基本上把美国的事情讲明白了吧。

我并不是说,我的分析就一定是对的;我只是提供大家一个不同的角度的看经济的一个方法;是不是一定对,我也不知道。我们尽可能地把一件事情讨论清楚、讨论明白。说结论吧,美国在今年可能出现的局面是:美元维持一段时间的坚挺,相比较而言,维持一段时间的坚挺;就是美元没有我们原来估算得那么早出现下降和崩溃,而美国的资产市场,包括股市和楼市,可能率先出现深刻的调整。

我刚才可能摁着说话的时候没注意,就是过了六十秒我还在那讲呢,好吧,我重新说一下子。就是我们认为美国的资产价格会先于美元的贬值提前到来,就这么个意思,简单的判断。因为其实你看到我刚才讲美国的股市里边的头部,无论是政策的天花板,还是它的技术的天花板都到了,它走不上去了。同时美国的楼市,无论是利息还是杠杆都不能再支撑了,所以它都是极限;而硬撑资产的话,美元将迅速地崩溃。这件事情,美国是承受不起的。

好,我们继续。美国的资产市场的调整会出现一种什么样的局面呢?就是美国资产市场的调整,会导致美元资本有方向性的外溢。好多人认为美国加息的周期会导致全球资本涌向美国,这个判断以前成立,现在不成立,原因非常简单,发展中国家,发展中国家类似于像土耳其,手上持有的美元或者是可逃跑的美元的数量并不多。持有数量最大的,是三坨:第一,中国;第二,日本;

第三,就是中东。你们认为他们手上的美元会去美国购买美国的债券或者是股票或者是房产吗?结论非常明确。因为沙特的钱、中东的钱一直在往香港走。沙特阿美在香港上市之后,现在相当多的中东富豪生活在香港,那么你们觉得中国的钱会去美国吗?我想2019年之后就开始慢慢关门了,到2022年恐怕,不能说想都别想,至少没那么容易吧;还想跑啊,跑不了了。

至于日本,渡边太太会放弃日本的股票、债券、房产去美国购买苹果、房地产或者是美国债券吗?我觉得想都不要想了。所以这一轮美元的回流可能只是一个美好的愿望,而不是一个美丽的结果。相反在美国的美元资本可能会纷纷地外泄,这也是我对美元资产下跌之后,美元依旧会下跌的一个基本判断,就是美元资本会大规模出走。

美元资本会向哪里去呢?其实我们在香港感受是明确的,你看那个港币和美元的这个关系,照理说黑衣风暴加上疫情,香港应该非常非常糟糕,没有啊,没有啊。你说香港是因为南下的资本、大陆南下资本来支撑香港吗?没有啊,没有啊。哪里的钱在支撑香港?当然,这个俄罗斯、中东、印度有一些钱来,但最主要的,我觉得还是英美资本。他们嘴上说《香港国安法》之后不行了,但是下体很诚实,他们还是回来了。当然回来香港不是为了香港,还是为了进入大陆。

所以我觉得2022年,我原来在2021年说:2022年到2032年将是中国的十年牛市的起点,2022年将是中国十年牛市的起点。所以我们现在可以进入到牛市的起点里边,开始进行挖掘、整理。我们这回要走在老外的前面,我们该把一些东西布局,完成布局。在合适的时候,我们会有一堂课把这个结构性的布局再说一遍。以前我们陆续说了一些,但是我们大家说:不清楚,那么我们就将来再花一堂课把这个事情说清楚,说说清楚。总之、总之,

总之呢,我们的判断,(这确实也是没有什么办法)可能还是一个大趋势吧,一个国运吧,可能还是会有大量的资本在未来的十年,特别是美元资本进入中国。为什么要进入中国?好多朋友说中国有那么好吗?她不是好不好,它是一个定价偏低的资产,就是我们中国大陆的几乎所有资产定价仍然是偏低的,不管是资本市场还是房地产市场,还是等等等等,定价偏低了。我说的是,不是今天用我们的理解来计算,是十年之后的那个溢价偏低。

其次,如果不来中国购买商品,如何输出通胀、买回通缩呢?只有两样东西——中国资产和中国商品帮你消化通货膨胀。通货膨胀就是金融现象,通货膨胀就是超发货币的购买力。通货膨胀必须有一个国家帮你背,以前可能是日本,是谁谁谁,是四小龙,今天只能是中国,Only one。那么我们要不要背呢,要背多少呢?该怎样背呢?这是一个大问题,这是一个非常大的问题。

我们必须要感谢特朗普,就是特朗普做的事情总是那么的有水平。就是自打美国跟中国打了贸易战;自打美国开始制裁中国的公司;自打美国开始驱赶中国在美的上市公司;其实美元购买中国资产这扇门差不多让特朗普关了半边,关了半边,不然的话中国优质资产还是滚滚的西去啊。还好,就是特朗普关了半边。而新上来这个老爷子拜登不明白、不懂经济,他不知道印出来那个东西,印出来那个绿纸不赶紧换成中国资产和中国商品的话,过期就馊了,过期就作废了。

我国呢,我国极为配合,我们金融开放都已经喊到……我们金融开放快要喊到一个什么样的高度呢?就是金融开放差不多快成了中国金融人唯一信仰的真理。就是金融开放就是真理,信仰它就是信仰上帝,差不多是这个局面吧。所以谁要敢反对金融开放,那还了得,就是金融开放。要不要开放呢?当然要啊。是我们出去,好不好?!是别人对我们开放,不是我们把门打开,都来,这个事情要想想清楚的,要想想清楚的。我们那么多的优质的股权一定要给人家吗?一定不能给我们平台上这些朋友吗?

我们平台这些朋友买一点股、股权,跟共和国一同成长,老了以后享受点红利,不行吗?一定要给美国的退休者吗,退休基金吗?这件事情前段时间议过,就是跟北京朋友也讨论过。就是不是不能开放,是开放什么?开放到什么程度?我们不是不可以消化一点点的绿纸,不是不可以帮他们背一点通胀。问题是你能背多少?你也准备通胀吗?你也准备恶性通胀,让通胀水平到10\%吗?我们输入性通胀可背的重量是有极限的。

其实做金融的朋友比做财政的朋友更聪明,他们啥子都知道,啥子都明白。但是你懂得,经济学不是关于科学的学问,经济学是关于立场的学问。一个人站在……

一个人当他站在敌人的立场上,他的聪明是很恐怖的。你非要把一把锋利的菜刀交给一个歹徒,你说他的刀法好,所以你给了他了,你对你、对你家里人很负责任吗?你把菜刀磨快了,交给一个歹徒,然后你想欣赏歹徒宰你家里人的刀法。我们现在很多人对我国的一些首席经济学家们的态度差不多就是这个意思,就是这个意思。把刀磨快给你,去我们家杀人吧,你要欣赏一下他的刀法,大概是这个意思吧。

我主张学习特朗普,我们开始学习特朗普,我们要制定一个清单。就是特朗普说了:中国不能买的东西的清单,就是你不能买我们的高科技企业、垄断性企业,什么这,都不许碰。中国想去美国买东西那么容易吗?什么都不让买,对吧。那么我们呢,我们的资产呢?我们金融开放是没问题的,我们的资产呢,要不要有一个清单呢?比如说,我们就欢迎他们来买恒大的股权,恒大的股票;但他想买华为,华为也不上市,他也买不了。但我们那么多高科技企业在上市,我们那么多涉及到国家重要资源的东西在上市,能不能有个清单呢?

另外,我们国家也不是没有钱的吧?我们财政空间还是有的。我们也不能让老百姓失业呀,比如说稀土,我们就采矿、生产呗,但国家统一采购,收起来吧,那不是比黄金还值钱吗?又好储存,能不能国家购买收起来呢?我们不一定着急,有些股权不一定着急卖,有些商品也不一定非要卖的嘛,不急着换外汇的嘛。因为现在用不了那么多外汇的嘛。你看看我们现在外汇都存哪去了?存商业银行去了嘛,对吧?

好多朋友说外汇储备没增加。外汇储备统计的数字是外管局的,麻烦你去看看,中国所有的商业银行的外汇,好不好?你再计算一下海外流通的人民币,好不好?有些、有些金融人小动作,我们不说,但不好看好不好?其实在好多好多问题上,在中国这14亿人里边,有很多很多的人看得是清楚的,有在岗位上的,有不在岗位上的。我想在这儿提醒的是,今天和以往历史任何一个朝代都不一样,因为不是人人在记账啊,不是分布式记账,不是区块链,是人人在写史啊。

人人都在写史,那一篇一篇的文字,甚至不发表的札记都是历史啊。不要晒小聪明,没有意思。对你没有意思,对你的孩子——如果你还真是挺重视家族,对你的家族没有意思,一点儿意思都没有。好,我们今天回到主题。如果美国会发生美元资本的外泄,外泄的量大概是多少?其中有多少进入中国?中国应如何处置如此多的——以前我们是非常需要的、境外资本进入,现在我们可能不需要,那么我们该做如何的处置呢?

同时,我国该如何地计算输入性通胀我们的边际和极限呢?我们输入通胀极限是哪里?我们要心里有数。在2022年,我们在这个问题上,尤其是要有一盘大账。还好,我们是计划经济的体质,我们还是有这个计划经济的东西在,我们大概知道我们需要每一种商品的总量是多少。我们在特定的时期,当某一种商品达致一个非常麻烦的物价水平的时候,我们如何熨平通胀周期、如何做对冲,是有办法的。

在很多时候,在很多时候,其实真正的民间的智库是非常重要的。上一周我参加香港的——还在封闭之前参加香港的一个活动,你知道香港的各种智库、论坛多如牛毛、多如牛毛,但没有一个智库——我说没有一个智库是不是有点过分,但大体上我没见过——没有一个智库是提供“智”的、是提供“智”的,但很多很多智库却从政府和国企那儿获得了大量的资金,很有意思啊。

没有“智”的意思,就是他没有人给你算账,就是美元资本将来出现外溢,通过香港走资的时候,香港这个地方会发生什么情况?香港应该做什么?是没有人去思考这个问题。后来我建议做一点点的安排,甚至考虑到更长远一点,就是美元一旦进行剧烈调整的时候,港币的,怎么办?或者是现在在香港上市以港币计价的资产怎么办?再多说两句吧,就是我和你们——平台上的朋友,我们好多朋友通过两地的这种沪港通、深港通也买了在港的股票,我们应该做怎样的安排和选择?是买对?还是不买对?深沪港……

深沪港三地如果汇率激烈波动,应该做如何的对冲安排或者是如何投一次机?都已经是2022年的1月份了,写完总结之后就得写计划,那么这个计划是不是应该提前6个月做好,然后我们蹲在门口等着?我们总不能等来的时候我们不知道该怎么办好吧?好多朋友找我说“能不能、能不能给一点点的建议”。我知道有的时候太好心不行,不是因为钱,是因为你要错的话、你就等着吧,你要对的话,其实大家说“不一定”、“不一定会”。

其实我们都不要别人感谢,只是他会觉得“你看,我那时我就说了应该怎样怎样”,但是输了的话,“你看……,都是他说的”。所以有的时候不是不愿意奉献,是太久了,我知道这个市场是个什么样子,我也知道人性。但我们作为平台,我们自己要有自己的一个完整的预判和规划,因为难得我们处在这样一个时代,难得我们准备进入2022年了,不是准备进入,已经进入了,所以我们好多事情要提前做准备、要做的精到、精密,而且要有回旋的空间和余地。

好,今天讲不了柯立芝繁荣了,柯立芝繁荣其实挺有意思的。柯立芝他是一个非常沉闷的总统,但他做的事情呢一点也不沉闷,他做的有些事情很有趣。就是柯立芝他是出生在7月4号独立日的那天、美国独立日,他是美国唯一的在独立日那天出生的总统。然后他又是一个命特别好,拣的总统,就是哈定,他是选了副总统,结果那个总统哈定突然就病倒了,就是他们说哈里斯可能就有柯立芝的这个命。然后他就当了总统。然后他当了总统呢他又很受欢迎,又当了一任总统,然后呢大家让他当第三任总统呢,他打死都不干,当时大家谁也不理解。

后来,大家开始理解了。因为那是1928年,他打死都不当总统。到了1929年,美国的大萧条开始,经济危机开始了,每个人都在怀念柯立芝,都在恨他的那个、后来是他的商务部部长叫胡佛当了他的下一任总统,所有人都在骂胡佛,都在赞美柯立芝。这个人呐有的时候他……当然这里边有可能是巧合,但你看他就是、他就是这么个有运气的人。有一天晚……就是哈定死了以后呢,他不好意思直接搬进白宫,所以他就住在隔壁不远处的一个小酒店里边,不是一个特豪华的酒店。晚上小偷来光顾,光顾以后开始偷他的怀表。那个年代怀表挺值钱的。

柯立芝这个人呢很有意思,他在黑暗中对小偷说:“你能不能到窗边看一眼那个手表上刻的字。”你知道小偷这个思维逻辑呢,他不是一个特别有智慧的人,小偷竟然拿那个表凑在窗前光线下看那个表上刻的一行字,是美国的大法院赠送给柯立芝先生的。他那会还没当总统,然后小偷就问:“你就是现在要当选的柯立芝总统?”他说“是啊”,他说那个表呢你最好不要拿走,你去拿钱包吧。小偷很听话,把表放下去拿了钱包。然后小偷解释说:“我最近遇到什么事,我需要32美金”。柯立芝就说:“那么你就拿32美金吧,但你记得要还我”。

后来呢小偷拿了32美金,很快呢又把这个钱还回给了柯立芝总统。柯立芝有好多很有趣的故事,我讲他跟小偷这一段,你就知道为什么会有大萧条了。他对偷东西的人是如此的放纵,他对小偷是如此的放纵。那么那些金融小偷在柯立芝繁荣、在伟大的柯立芝繁荣,正好是从1918年到1928年,10年最伟大的繁荣期间搞得一塌糊涂。这段历史对美国和对中国都有重要意义,我们下一次聊天再说。

补几句,补几句对中国经济的看法。这个事情是我最不想说的,但是好多朋友说最近股票上输了很多钱,这个好多行业也很困难,问我有怎样的看法。对A股上输钱的朋友,其实我一点都不同情你,我就想打你屁股,不听话,真以为自己本事很大,真以为自己可以投机倒把,胡闹。但已经、已经输了的朋友,我想对你说,你输的那些东西,不管是教培也好、互联网也好、甚至房地产也好,都接近底部了,别跑了。

至于对中国经济未来的看法,我写了篇文章,就是关于水循环建设的,每年两万亿,一共10万亿,每年拉动两个点的GDP,大体上可以将房地产过剩产能和过剩劳动力消化掉,问题应不大。至于说好多朋友自己的新的就业方向和创业方向,我在这儿再说一遍,但我不能负责任,就是我不建议你们还是去考公务员,去做一些这样的事情。2015年我们已经进入数字经济时代了,那么去做跟数字经济时代相吻合的事情吧,如果是找新的企业。

如果你一定要找新的企业,那么你就找一个做数字经济的企业,如果你在传统企业,那么你就动员那个企业完成数字化改造,总之以后资本的载体是Data、是数据,不再是砖头、不再是其他,那么你得跟上来。如果是孩子的话,那么也应该向这个方向努力,向这个方向努力。就是我们要跟上时代、跟上节奏,而不是走反方向。我们不要去固守在夕阳里边,固守在夕阳产业,应该跟上时代。

好多朋友说遇到一些困难和障碍,2022年这样一个年景它是正常的,它是正常的。不过今年壬寅年水火既济是一个好年景,至少前四爻都会非常好,前四爻都会非常好,所以不用太多的担心,花出点时间来多读书、多长本事,如果疫情没那么严重,多出去走走、多去看看。多读书多看看,好过瞎琢磨。

好吧,今天柯立芝繁荣用不上了,等下一次聊天。下一次我们聊柯立芝繁荣肯定会很精彩,顺便呢也对未来即将到来的大萧条做一个伏笔。好,谢谢大家。明天下午3点钟,我们再拾遗补缺,周末愉快。

\subsection{柯立芝繁荣、聊聊当下市场的情况}

大家好!今天是2022年的1月29号,辛丑年的12月27日,终于就要结束最后一爻了,凶险的最后一爻啊,果然杀伤力巨大,再有几天就熬过去了。今天呢是聊天儿,今天聊柯立芝繁荣,然后呢聊聊当下的这个市场的情况。好,我们三点钟准时开始,一会儿见。

大家好,今天是2022年1月29日,辛丑年12月27号。这个已经到了最后一爻,已经到了这个最后,到了贲卦的最后一爻,应该最凶险的这个时间就过去了。我们将迎接壬寅年了,壬寅年的既济,既济的初爻,前四爻都是不错的,后边稍微有问题。所以恭喜大家新年快乐。

香港这个天气真的很诡异,我今天还是来写字楼。因为疫情的关系,所以我最近早上不跑步,每天步行上班和下班。我仍旧是穿着短袖的T恤,然后步行走到办公室,温度大概在20度左右,走快一些还是要出汗的。今天晚上呢大概会降温,降到12度左右。香港的疫情呢扩散了,香港这是,怎么说好呢?就是香港的防疫的这个想法啊、控制啊,令人一言难尽吧。

先不说杂事,因为今天聊天虽然是聊天,也不想扰动了我们的主题,今天我们是讲柯立芝繁荣。这个聊天呢就是可能比讲课还重,就是因为我自己判断:一次世纪性的大萧条不远了,要到了。那么这次大萧条对中国意味着什么呢?我们让历史告诉未来,所以我们先聊柯立芝繁荣,然后再聊大萧条。可能一堂,不是一堂课,一次聊天可能聊不完。无所谓,我们一次聊不完就聊两次。

谈柯立芝繁荣呢,通常大家会直接去聊柯立芝,其实上上上上一次聊天我们简单聊了几句柯立芝,柯立芝这个人很有趣,我讲了他的故事,他的故事还挺多的。但今天的要点,说柯立芝繁荣却不在柯立芝本人身上。我要再谈一次威尔逊。其实很多人知道我对威尔逊的评价是极高极高的,就是威尔逊是伍德罗⦁威尔逊,他是1856年12月28日生,1924年2月3日去世,他在柯立芝。

威尔逊恰好是在柯立芝任上去世的。那么威尔逊这个人为什么重要呢?我用我的语言先做点简单的概述吧。一般人会认为美国的总统排名,就是在那个美国那四个石头山上的总统,但我个人对美国总统的排名是这样的,美国总统里边的前三,第一是华盛顿,第二是林肯,第三就是威尔逊。在某种意义上,奠定现代美国霸权的人是威尔逊。

如果说华盛顿对美国建国是有贡献的,林肯对统一美国、完成一个美国完整的版图是有贡献的,那么建立一个强大的美国,那么那个人就是威尔逊,或者说在美国的成长历史上威尔逊的贡献,绝不亚于华盛顿和林肯,其他的人跟他不可同日而语。理解威尔逊是非常重要的一件事情,因为好多人可能觉得。

好多朋友可能认为懂美国历史,如果是一个读美国史的人,你要说懂美国历史的话,我就说你可能上当了。我们今天要讲的柯立芝繁荣,涉及美国当代史最重要的一个部分,其实就是威尔逊总统的一系列的制度建设。威尔逊这个人很有意思,他是律师,有律师资格,但他是普林斯顿大学校长,他又做过新泽西州州长,就是他拥有严格的法律训练,又是哲学博士、校长,州长。

从结构上来讲,在美国历史上,哲学博士、大学校长、律师身份做总统的人,唯此一人。他能做出来的事情呢,请不要惊讶,因为我以前说过就是威尔逊留下太多东西,比如说联合国、比如说高尔夫球等等。我们今天呢主要是讲威尔逊到底在他的任期1912年到1920年这8年他做了什么?为什么他这8年奠定了柯立芝繁荣,乃至于以后的强大的美国,又为什么美国开始步入衰落。

美国自建国开始就是仿照英国模式,因为美国的主要的建国者都是来自于英伦的,所以美国实际上是庄园奴隶主共治的一个,庄园奴隶主共治的共和国,华盛顿建立那个共和国叫庄园奴隶主共治的共和国。林肯南北战争期间对这个庄园奴隶主的共和国进行了某种程度上的改造,但它仍然是庄园奴隶主共和国。好多朋友可能不知道我在说什么。

我想说的是直到威尔逊1912年当上总统的时候,美国四十七个州仍然是州政治经济为主体,联邦非常弱,是国弱民强的状态。国弱的状况谁改变的?那么就是威尔逊。威尔逊做了几个大手术,一个就是我以前讲课时候讲过,就是建立了联邦储备系统,或者是叫联邦储备法案。这件事情因为好多朋友读过《货币战争》,好多人对小宋的那本书呢……

我十多年前在清华有一个演讲,就是关于这个美元的问题。这个演讲的基础是,我记得是1994年我们访问美国财政部和访问美联储的时候,我在美联储提出的问题就是关于这个联邦储备系统和联邦储备法案的问题。实际上美国真正建立中央银行,真正的控制私人银行和华尔街就是这个联邦储备法案,1913年通过的这个联邦储备系统。实际上将权力上交给了议会、立法机构,这也是后来保尔森为什么要单腿跪在参议院。

保尔森在2008年金融风暴的时候,他单腿跪在参议院,请求参议院给一次机会、给一次机会,挽救美国的金融体系,挽救,主要是救一下那些银行。建立联邦储备系统的意义在哪里呢?就是你不要小看威尔逊,威尔逊干了两件惊天动地的大事,一个是这个联邦储备法案,它才使得美元成为未来的国际货币,没有这个联邦储备系统、没有这个联邦储备法案,美元怎么完成国际化呢?!没可能的嘛。是这个法案使美元成为国际货币,开始国际化的。请一定要记住这件事情,我们要做的就是这件事情。

第二件事情,威尔逊改革了美国的财政体系。实际上,威尔逊开始以消费税为主体,建立了联邦的、联邦的税政体系。实际上在整个的过程中也刺激州与联邦在税政方面进行了系统性的改革。这个时候开始建立比较系统的直接税体系和完整的联邦财政收入体系。好多朋友不知道我在说什么,那么我一个结论告诉你,没有联邦的财政收入的迅猛增加,就不可能有强大的美军。也就是说除了美元之外,威尔逊组建了一支……

威尔逊组建了一支可以横扫欧洲的美军。在此之前,美军的实际的军力在全世界的排名,我估计、我估计可能进不了前五。不要说与英国、法国、德国比,可能跟俄国都比不了,奥匈帝国比不了,就是美军是非常弱的,美元是非常弱的。虽然美国经济总量已经超过英国,但是美元不行啊。结算是以英镑,全世界的整个贸易结算是英镑结算,因为美元不可靠、不可信呢!没有法律基础的一个货币,能相信你吗?就是人民币发行你不立法,怎么行呢!你的发行权要从央行转交给全国人大才行的嘛。

由于有了联邦的财政收入的巨额增长,才有可能组建美军,因为联邦的军队在美国实际上是民兵组织凑合的。不要听我们现在后来知道的这个美国名将录里边那些名将,那个时候他们就算是地方武装的头头,包括麦克阿瑟、包括就是二战这些名将们都是,是因为联邦拥有了强大的财政能力,才可能建立起一支前往欧洲参加第一次世界大战的那支美军。

金融改革就是联邦储备法案,财政改革就是新的税制。你们看到了什么呢?这是柯立芝繁荣的基础啊!但这是全部吗?当然不是。最要命的或者是最厉害的就是他的《反托拉斯法案》,又叫《克莱顿反托拉斯法案》,是克莱顿提出的一个这样的一个法案。美国是一个,我一会儿再讲这个谁,因为大家一般意义上认为威尔逊是一个哲学家,其实我不把他定义为哲学家,我定义他为政治学家,而且是可能是人类史上少见的优秀的政治学家、政治学专家。

他什么时候开始盯上这个政治、政治学问题的呢?是他在当律师的时候,年纪轻轻的他,他是年轻的时候曾经仰望过这个李将军,就是南北战争的李将军,看过李将军的这个面容,他的父亲又是在南方是具有极高政治地位的一个牧师,所以他对政治问题极为敏感。他发现了一个问题,发现了一个重要的问题,这个问题到今天仍然是困扰我们的一个重要问题,就是政治与行政的关系,他认为必须让政治脱离行政。

这话听上去你可能听不懂,政治和行政脱离。那么我告诉你,四十七个州就是四十七个庄园,四十七个庄园奴隶的奴隶主才是治理这个国家的那个人,而不是美国总统。所以如果让美国强大起来,必须强化中央集权。他做的事情一共是两条线,一个是强化中央集权,一个是维护劳工权益。我说过威尔逊是伟大的社会主义者,到今天可能跟我一样看法的人,还很少。

说个插曲,建立美联储之后,美联储发了一张钞票,是十万美元面值的金圆券,印的肖像,就是威尔逊,是为了纪念他。现在这个钞票因为它是金圆券,它不是美元,现在这个钞票就变得极为珍贵,就是美国的顶级富豪,以拥有一张这个钞票为荣,这证明你是牛人嘛。它现在到底值多少钱呢?我也不太清楚,因为这个没法估算,但它在银行家和美国的大企业家心中,这张钞票像神一样的存在,它有无限法偿的能力。

反托拉斯法被美国工会领袖——当时的工会领袖塞缪尔–龚帕斯认为是劳工者的大宪章。你知道他不光是反托拉斯,威尔逊规定了8小时工作制,就是一天工作8小时,废止童工,要求最低工资保障,建立社会保障体系。你懂得,反托拉斯法案,如果说联邦储备法案和税政改革是美国成为强国的重要的经济基础,反拖拉斯法案是美国进入发达国家的一个根本性的措施,因为它真的建立中产阶级并拉动了消费。

威尔逊作为政治学的一个专家,他们说他是哲学家,但是哲学和政治有时候分不了太清楚,但我理解他,他是一个伟大的政治学家,同时他把他的理论自己亲自做了实践,他也是伟大的政治家。一个可以既建立联合国又建立高尔夫球运动的这样的一个人,是不是很神奇呢?我要说作为基督教信义宗、基督教新教伦理的美国,在某种意义上他具有这样的能力,就你如同当下的中国,共产主义源于信义宗,拥有共产主义理想和信仰的一代人才能干这种惊天动地、划时代的这种事情。

威尔逊他,怎么说好呢?就是他对政治的理解是清晰的,他知道美国的这种47州的这种以州政治和州财政为主体的模式,对于美国的国家总体在国际上的这种扩张和发展是不利的。所以在政治上,威尔逊提出的是威尔逊主义,请注意,请注意,威尔逊之前是门罗主义,门罗主义用中国话来表述叫做闷声发大财,关起门来搞建设,这就是门罗主义。威尔逊主义是什么呢?叫共同富裕,人类命运共同体。这个想法是不一样的。

威尔逊主义在威尔逊死后成为美国的政治的主体思维,直至今天。谁在质疑威尔逊主义呢?那个质疑者就叫川普——特朗普,他就是要关起门来搞建设,不搞威尔逊主义了。但是很不巧,这个特朗普下台,拜登上来,重新端出威尔逊主义。威尔逊的一个特征呢,就是他要将民主的理念铺满全世界,他当时就是支持弱小国家民主运动,就是你们就是把你们的国王赶走吧,你们开始进入现代化。一路走来呢,就是灯塔国的这个肇始者。

灯塔国的这个“肇事者”呢应该是威尔逊和威尔逊主义,它是一种政治上的一种主张,特别是在国际政治中的一种主张。威尔逊的这个主张难道仅仅是一种理想主义吗?是美国人所说的威尔逊主义就是一种理想主义吗?当然不是,威尔逊主义也是现实主义的,它是为了美国经济整体的扩张做的一个铺垫。只是作为一个伟大的政治家,他看得比其他政治家、比庄园奴隶主看得要远一些。所以没有威尔逊主义怎么可能有柯立芝繁荣呢?要知道、要知道,一次大战之后,全世界的资本,全世界的资本家和资本涌入美国。

好多朋友可能理解不了柯立芝繁荣,柯立芝繁荣自然有内部的原因,也有外部的原因。外部的原因两个,第一个是第一次世界大战,威尔逊是1917年决定参与一次世界大战,1919年就结束了这场战争,你看看他厉害不厉害?他1912年当总统,1920年离开,在这中间呢把这个世界大战打完,而且是辉煌的胜利。一次世界大战,还有一件事儿就是西班牙流感,这个人也很厉害。全世界西班牙流感,特别是欧洲惨不忍睹、惨不忍睹,唯有美国受西班牙流感影响最小,因为威尔逊实施了比较严格的公共卫生政策。

关于戴口罩的问题呢,今天呢我们都不理解特朗普是怎么想的。但是你要看威尔逊时期、西班牙流感进入美国时期,青年男女约会时戴口罩的样子,你会觉得很有趣,他们戴着口罩悄悄地在树林里约会。历史就是这样的,历史看上去好像很巧合、很有趣,但历史就是这样的,像车轱辘一样转来转去,故事呢也不是那么复杂。如果你理解了他控制好了西班牙流感,如果你理解了他打赢了第一次世界大战,你就开始能够理解欧洲的资本家和资本、人才为什么滚滚涌入美国?

实际上在一次世界大战进行期间,大量的欧洲的贵族、资本家、优秀的学者,甚至有能力逃亡的产业工人都开始离开他们的国家。这不仅包括了德、法、俄、奥匈等帝国,甚至也包括了一部分的英国。英国的资本家、资本,一些老牌的贵族,包括了产业工人,因为他们觉得美国的更为民主、先进,更灯塔一些,对吧。你看这个他的法案。

哦,我刚才漏了,反托拉斯法案和劳工法案之外,还有一个妇女,就是这个谁还建立了母亲节,然后给妇女选票,就是这个威尔逊。你无论从任何一个角度来讲,那个时候的美国都是人类理想中的乐土,人们尤其是对欧洲有着无法抵抗的吸引力,所以大量的资本、熟练工人、资本家涌入美国,它构成了美国的二十年代、1920年之后的柯立芝繁荣,这个柯立芝繁荣的由头,准确地讲是威尔逊给的,这个柯立芝……

我总说柯立芝呢有点像克林顿,就是这两个“柯/克”。就是柯立芝接手的是威尔逊缔造的一个崭新的美国,强大而崭新的美国;克林顿是从老布什手上接过的,经过里根和老布什干掉日本和苏联之后重新吸血、满血复活的美国,所以克林顿1992年到2000年的克林顿繁荣,他就是那么一个情况,跟这个柯立芝繁荣有一点相似。这个人的命好有的时候是没办法的,你知道柯立芝其实他没做太多的事情,他连小偷都懒得抓,我上次讲了他的这个跟小偷那个故事。

威尔逊的改革是双重的,请记住是双重的。第一个,他建立了美国从未建立的强大的中央集权。这个中央集权表达为美元和美军,强大的中央集权,因为终于将联邦储备法案推出之后,实际上一个权力是从华尔街和金融资本家、银行家手里边拿回来的;另外一个权力实际上是从这个中央财政建立一个新的税收以后,开始建立起强大的美军。社会主义改革则体现在他的反托拉斯法案和妇女法案、劳工法案方面。这个整体上形成了一个完整的框架。

甚至威尔逊之后,美国成为灯塔国。他1920年离开,这一百年,我更愿意定义美国这一百年为威尔逊时代,就是从1920年到2020年,我愿意定义他为威尔逊时代或者是威尔逊百年。柯立芝繁荣只是威尔逊打下的底子之后的第一次开始的一个繁荣,后边它还有一共是,就是从柯立芝繁荣、二战之后的繁荣到克林顿的繁荣,一共是三次,重要的这种美国经济的这样的,当然了,现在是不是可以定义认为美国的这个繁荣周期结束了呢?我们将来有空再聊这个事情。

威尔逊只是很多人很羡慕他,就是他长得很帅。然后呢,当总统第一年媳妇儿就死了,然后就娶了一个新媳妇。这个新媳妇儿有印第安血统,后来呢,四十年后才死,但她一直守着这个威尔逊那个老宅子,后来威尔逊这个老宅子成为了现在一个文物古迹。下回你们去华盛顿的时候,记着去参访一下子威尔逊的故居,他现在已经成为了他们的美国的文物保护了。威尔逊本人,放在华盛顿座堂,他也是华盛顿座堂唯一安葬的一位总统,你看他的地位跟林肯那个纪念堂差不多。

所以去美国可能华盛顿你要看一看,林肯去看一看,这个威尔逊要去看一看,去华盛顿可能要看看这三个人,其中重点要去看一下威尔逊。因为奠定当代美国的就是他。如果说我们中国人需要理解美国或者需要学习美国,学习谁呢?学习威尔逊,可不能学柯立芝和克林顿。这个威尔逊也是一个可遇不可求的这样的一个人物。就是我们中国是不是也会产生这样一个集法律专家,哲学、政治学于一身的伟大政治家呢?将来或许会有吧。

好,威尔逊就先说这么多,我们来聊几句柯立芝繁荣。柯立芝繁荣的标志是什么?柯立芝繁荣就是在柯立芝的任上,美国由债务国变成了债权国。当然这个原因还是威尔逊的第一次世界大战,威尔逊打第一次世界大战,就美联储成立了,美元开始好用了,就向全世界提供了大量的美元的贷款,各种美债,就是,然后呢,强悍的美元、坚挺的美元,这个从全世界购买了大量的物资,然后全世界资本涌入美国,就是我刚才说了,由于战争和瘟疫影响。

我又忘了看,可能摁过头了。就是威尔逊吞吐欧洲的过程中,美国从全世界最大的债务国变成了债权国。因为全球资本涌入美国,这个时候美元变得无比强大,那么各国的货币变得极为孱弱;同时,同时,由于欧洲的产业工人大量涌入美国,欧洲的科学家、产业工人大量涌入美国,美国的科学技术进步速度惊人。这里边有两件事情是需要大家注意的,我们在学管理学的时候一定会学到泰勒制和福特制,就是所谓福特模式。

泰勒制和福特制都是产生于柯立芝繁荣期间。这里边是什么问题呢?就是由于劳工法案使得美国中产阶级就是蓝领工人变成中产阶级,并且有钱买汽车了,买房、买车。买房、买车这种消费给美国经济的拉动是无法想象的,我就不在这里背数据了。所以到了1929年的时候,美国的工业生产总值已经占到全世界的50\%了,50\%!不是……,是一半呢!就是有点惊人了。

美国在1913年的时候,黄金储备占全世界的19.24\%,1924年的时候是44.99\%,就是45\%吧。从20\%到45\%,差不多全世界一半的黄金被美国买走了,美国不但向全世界输入资本,而且把全世界的黄金全部拿回来了。工业生产指数,1921年是67\%,1928年是110\%,1929年126\%。爆灯啦,爆灯啦,就是太好了,好到难以想象的程度。但请注意另外一个数字,

1920年到1929年,全美生产率增长55\%,全美工资增长2\%。你懂得,他一定会陷入大萧条的。就是经济高速增长,老百姓的收入增长很慢,2\%,其中农业和农民的收入是下降的,农产品价格下降。因为美国大量地进口其他国家的农产品,因为美元强嘛,使本国的农业大量破产。可能大家很难理解,就是1929年10月29号发生了什么。

美国的1929年10月份股市崩盘,最后崩了多少呢?是70\%的市值不见了。那么美国的楼市跌了多少呢?是50\%不见了。那么美国的企业是什么状况呢?是美国超过60\%的企业破产了,银行、金融机构和企业破产了。那么美国的失业人口的增长的水平到什么程度呢?大概美国有30\%的人没有收入,需要乞讨或者是救助,就是这么个状况。柯立芝这个人很聪明的,柯立芝到1928年的时候,别人让他选,他说什么都不选。然后,柯立芝有两员大将,有两员大将……

柯立芝的财长叫梅隆,柯立芝的商务部长叫胡佛,胡佛就选了这个1929年的总统,一上任就是大萧条,这个倒霉蛋,后来他修了一个大坝叫胡佛大坝。可能大家去美国的时候有机会可能会去看那个大坝,我上次去拍了个照片,打了个卡。你说柯立芝是否意识到荣景不再呢?我想美国的政治家、顶级的政治家是明白的,知道不行了,所以他拒绝,因为他当了哈定的(就是1920年的那个)第二十九任的总统的一年,然后他当了第三十任总统的四年,一共他干了五年,他还可以再干,但是他拒绝了,他不选了。

柯立芝繁荣里边到底是哪些伏笔留下来成为1929年大萧条的问题呢?我们是不是要真正的理解一下繁荣的危险呢?因为经济处于超过8\%甚至10\%,高速增长超过十年甚至二十年,一定会有一些问题留下来的,我们也要理解和总结中国经济高速增长之后可能已经出现的问题。我们今天简单聊几句问题的生成,就是大萧条的那个肇因都是些什么?

美国在威尔逊的时代,其实经济增长是一种非常好的状态,就是它高速、平稳,工业化升级速度非常快,在迅速地追赶欧洲。当时无论是英国、法国和德国,在工业能力上,虽然总量上是美国超越了,但是美国是低端,中高端在欧洲,所以他是在进入到一个加速赶超的这个过程中。到了柯立芝繁荣的时期,美国的工业的水平已经开始上来了,就是他造了福特汽车,虽然不如德国车、不如英国车那么好,但是他能造,而且他的成本低、产量大,就是老百姓都能开。

在整个的危机构成里边,后来分成不同的学派,对美国的这场危机有不同的表述。我们知道的就是弗里德曼的表述,这个弗里德曼的表述他认为是在金融管制方面出了问题,这也构成后来伯南克任上和到现在这个鲍威尔任上,他们在处理美国经济时候采取的策略。因为他认为是货币供给出了问题,就是弗里德曼始终认为是货币供给上出了严重的问题。什么意思呢?就是货币供给的量过大,而在收水的时候……

在收水的时候又过于猛烈,就是大家都把这个问题责怪美联储,就是不该提供那么多货币的时候,你提供了太多的货币,然后出现问题的时候又迅速的收水,导致金融体系不能支撑美国的经济,以至于最后经济出现崩溃。这个供给以供需来讨论,就是供给侧、供给学派的这一套逻辑到今天仍然在深刻地影响着美国的政治和经济政策,就是认为只要直升机撒钱,其实就应该能够不至于爆发大萧条,不撒钱就一定大萧条。

所以在特朗普的最后一年、在拜登的第一年,就是2020年和2021年,美国政府都执行了宽松的货币政策——所谓QE,执行了这个接近于零的这个利率的政策来刺激、来应对美国经济和疫情,有点像西班牙流感、有点像西班牙流感,来应对这个疫情。其实,其实,这个政策可能带来的问题远比它解决的问题更多、更大、更严重。

我在接受凤凰专访的时候我说了,中国是跨周期调整,美国是逆周期调整,大家都没有在进行顺周期调整。顺周期调整什么呢?就是经济热的时候,我们就金融、货币政策要冷、凉,财政、金融要冷,给它降温,这个经济不行的时候,财政金融要热,给它升温,就是用财政和金融政策熨平经济危机。在大萧条之后,1945年一直到现在,这就眼瞅着可能快要八十年了——七十多年,再没有爆发特别大……,有经济危机,但没有大萧条那么惨烈的状况,就是因为……

就是因为采用了比较理性的财政和金融政策,在某种意义上熨平了经济危机。也就是说从1945年一直到2015年,我们大体上认为整个这个历史周期都是一个顺周期调整,既没有进行逆周期调整,也没有进行跨周期调整,是顺周期调整,所以它大体上平稳地度过了这么个漫长的过程,没有出现大萧条。有经济危机,但是没有进入大萧条,没有再发动这个世界性的战争。那么这一次,我真的是不乐观,因为这一次我看到中国是跨周期调整,美国是逆周期调整。

什么叫跨周期调整呢?就是中国不以当下的经济现实为依据出牌,就是中国现在经济在第三季开始陷入衰退,第四季开始明显衰退,2022年的第一季可能会非常惨烈。理论上讲,中国不要采取太大的结构性的调整,但是中国考虑到长远发展问题,所以对房地产、对互联网、对教培等,进行了非常深刻地调整,而且在这个过程中,我们持有一个非常审慎的财政和货币政策。我们是跨周期进行调整,就是我们不在乎眼前这个周期的这个波动,我们决定跨越这个周期,为长远十年、二十年、一百年做安排,这是很了不起的。

我有时候跟香港的朋友说,我说我对这一代领导人感到非常的钦佩,在他们身上我能看到威尔逊同志的优良品质,包括“共同富裕”这些提出,那跟反托拉斯法案、跟劳工法案、跟妇女法案异曲同工啊!共同富裕、人类共同体、建立联合国,这跟威尔逊的想法是一致的,只是我们现在没有联邦储备法案,就是我们人民银行发人民币这件事情,应该是全国人大出一个人民币发行法案,就是这个事情念叨这么长时间,就是变成了一个敏感话题了,就是弄不了。另外就是财政法案——就是直接税立法,怎么都出不来,这个大家就是在重大问题上还是缺少信义宗的那种情怀。

好,我们回到今天主题。危机产生的原因是不是弗里德曼、伯南克他们认为的这个金融过快收紧、过急的收紧所导致的呢?我个人认为这是事情的表面,不是本质。事情的表面是金融收水过快、过急,所以导致大萧条的出现,整个的股市、楼市的崩盘,这是一个表面现象。不收水、放水只能使问题更为严重。当然可不可以稍微缓和一点呢?是可以的。我们讨论“柯立芝繁荣”,为什么不直接讨论大萧条,要讨论柯立芝繁荣?就是我们想知道这10年,威尔逊那8年交给柯立芝之后,柯立芝年这8年干了哪些坏事,这些坏事可能是

就是柯立芝办的这些不漂亮的事儿,可能才是大萧条真正产生的原因。我对柯立芝的工作做了一点总结,我原来是想多讲一点柯立芝的故事,后来我想了想,刚才由于时间也不够,以后有空再说吧。大家也可以上网去查一些关于柯立芝的事情。我个人认为柯立芝作为……

1920年再次选举的时候,共和党人哈定和柯立芝搭班子,哈定和柯立芝,共和党的总统。美国的这两党政治很有意思的。因为这个谁,普林斯顿大学这个校长威尔逊,他是顶着民主党的头衔当选的,其实他不能算是一个地道的民主党人。在他8年执政之后,共和党人终于夺回了总统的宝座。夺回总统宝座之后,他们认为,当时共和党认为威尔逊太左了、太左了,他们要向右进行修正,提出来一系列的简政放权的……

就是把把威尔逊的那一套左的做法、社会主义的做法全给废了,然后开始搞共和党这一套东西,就是放纵金融资本、劳工保护,就是要放松一些……所以纵容这个华尔街,就什么不管。因为有这个威尔逊打下的底子,美国经济发展的速度是那么的好,所以他那种放纵在某种程度上也刺激了美国的经济增长,但里边最核心的问题是刺激了金融资本、金融资本资本利得的暴利。而整个我刚才说了,劳动所得9年之间只增加2\%,而经济总量增加了55\%,资本利得……

就是资本利得吞掉了大部分的利润和好处,而资本利得并没有能够带来崭新的、新的消费需求。那么,工业的迅速增长,需求的不增长,特别、特别是一次大战结束之后,英、法、德,包括俄国都在恢复他们的工业生产能力,不是一般性地恢复,是迅速地恢复。而亚洲像日本也在迅速地增加自己的工业生产能力。那么全球产能的相对过剩早已出现、早已出现,消费整体不足、产能严重过剩,而对这个事情美国的当政者……

对于产能过剩这件事情,美国的当政者并不敏感,甚至缺乏应有的觉悟。所以在产能一向都是生产过剩的危机,就是经济危机大体上是生产过剩的危机,有效需求不足、供给严重过剩,需要进行供给侧改革。但是柯立芝同志没有这样做,他把所有的事情交给市场,不负责任。这个时候已经埋下了整个大萧条的隐患,加之在整个的柯立芝最后的执政的那一两年,美国股市和楼市的繁荣爆灯了、爆灯了。大量的金融资本脱离了实体经济,进入到资本市场炒作,跟今天美国的情形一模一样。

关于美国的这个大萧条,除了弗里德曼的看法以外,还有凯恩斯的看法。马克思主义者就是社会主义国家的学者,对大萧条也有一些看法。就是在分析美国这场大萧条的时候,我们不认为它是一个简单的供需失衡的问题,不是个简单的供需失衡的问题。这里边我们要注意到几个方向,第一个方向就是关于对资本的管控的问题。你放纵资本自由地在市场上来进行整体上的一个操控的话,其实大萧条是必然的结果。这与今天我们看到美国对资本的态度如出一辙。

在讨论柯立芝繁荣的时候,有些内容可能就是涉及到——因为柯立芝他不是像威尔逊这样的有思想的人,他只是共和党、一个共和党推举上来一个总统,他在很多时候受制于共和党的这些财阀们的一个威胁和压力。我说到这儿,我想表扬我国,就是我国的领导人,我国的财阀想绑架我国的领导人,目前做不到。但1920年之后的“柯立芝繁荣”,实际上柯立芝是一个被牵线的木偶,那段繁荣的时候柯立芝本人实际上对美国经济没有做出他应做出的一些调……

嗯,4点了,我想呢留点时间吧。这个大萧条我们放到下一个章节,下下堂课去聊吧。我花个十分钟左右的时间聊一下子对美国市场的看法,因为大家都很紧张。昨天晚上,远方的朋友、远在远方的朋友给我发帖子,质疑我对美元和人民币的看法,就是他们认为美元应该很强、人民币应该很弱。我为什么认为人民币会很强而美元会弱呢?认为我在逻辑上可能不能自洽。就是这件事情因为很重要,对好多人来讲。

我是怎么来体会美元和人民币的关系呢?我个人认为、我个人认为,人民币是一个非常特殊的货币,就是它现在资本项下没有开放,那么人民币就有一个离岸和在岸的关系问题。在岸人民币整个的汇率的情况其实是被严格管控的,只有离岸的人民币会处在一个比较麻烦的状态。就是离岸人民币是可以被操控的、被操纵的,我方也可以操纵,对手也可以操纵,所以离岸人民币就比较有趣。

而我呢,身处香港,可以近距离观察离岸人民币的波动,而且因为我有一大堆的就是做这个行业的、行当的一些朋友,我们经常会进行一些讨论。我昨天本想写个帖子回复,后来我想了想,不急吧、不急吧,因为有些数据也不方便拿出来,不方便给大家提供。那么离岸人民币发生了什么情况呢?我就给大家念叨一下事情。至于数量和精算的事情呢,可能隔一段时间会有数字出来,但我不方便提供,因为大家有个约束的嘛,就不能这样。

我国在处理美元问题上呢,2018年之后,外管局大体总量控制在三万亿美元的外汇储备,不会太多也不会太少。那么我国这些年,这个进出口贸易项下边的盈余去哪里了呢?我们没有买黄金,我们甚至没有进行战略储备。有一些吧、有一些储备,但是跟我希望的量差太远了。它在哪里呢?它散落在金融机构(包括投资银行、商人银行)和涉外企业手上。

规模有多少呢?可能你们听了会害怕,接近两万亿美元。就是我们外储之外还有两万亿的,不在外管局手上的、不纳入统计数据的外储,没有结算的。这个外储、这批美元和人民币构成了某种离岸的对应关系。好多人看不懂,你跟他说,你不跟他说,他说你……反正总之说我什么都有;你跟他说,他又很惊讶,但这是个事实好不好?这是个事实,它是真实存在的。那么离岸的美元和人币的关系……

离岸的美元和人民币的关系,正确的做法应该是什么呢?我个人认为,为什么要飘在外边那么多美元呢?你知道有风险,你为什么不处理风险呢?包括香港,还飘着4553亿美元的外汇储备,而且都是美元资产。你知道有问题,外管局、央行,亲爱的同志们,你们要尽责任呐。可以收回的部分,比如说收回一万亿,把它变成战略储备。就是那些美元不要这样的抛在那儿,抛荒。收回来,变成金、银、碳排放权、尿素、钾肥,变成什么都好,好不好?稀土好不好?

而囤积在商业机构和企业手上这批未结之外汇,其实构成了中国巨大的金融风险。这里边好多人,实际上他们对人民币、对中国经济是因为持有怀疑态度才囤积美元的。囤积美元,这个囤积的过程中,形成离岸美元和人民币的交易,不是在对手盘上,是我们自家人手上在折腾。有些人认为人民币会升值,所以在抛美元。抛美元,以至于前一段时间人民币涨得极强,就是因为好多人认同我的观点,所以他们抛美元换人民币。因为美元可能会出现兑人民币的一个5时代,所以好多人在抛。当然也有一些人对我持质疑态度。

远方的那位朋友在质疑我,远方的朋友,因为他们是在欧洲,他就觉得,卢先生,你看我们都是这么好的朋友,我就信任你,但我不能理解你,你在说什么,我们认为美元才是好,一个好的货币:美国经济增长6.9\%(第三季),比中国经济增长好很多很多很多很多,为什么呢?美元不能强,而人民币还要继续强呢?有的时候讨论问题,有时候挺伤感的、挺伤感的。因为你跟行家说话呢,就简单一点;你跟媒体人讲话呢,有时候就是困难一些;你跟非行家、非媒体人讲话,

有的时候还会带一些问题出来,因为很多朋友,他质疑你的、他质疑你的专业能力、他质疑你的专业能力。其实这里边是三层:第一层,第一层最核心的是获取信息的能力,你得知道,多数人是没有能力获取信息,这不是他们不努力,是因为所处的那个位置得不到,你得不到那个信息,你看不着,你怎么做判断?第二层,是专业能力,你有了信息,有没有把它综合进行分析的那个逻辑框架,有了这个信息源,有了专业能力,才有可能得出一个初步的结论。

但这个初步的结论,还要与宏观经济水准、就是宏观经济分析和判断相结合,就是金融专业的专业能力,信息、金融专业的专业能力和宏观的(宏观的就是出于对制度和政策的判断的把控能力),三者相结合才能获得一个初步的结论,不然的话纯属胡说八道,胡扯八扯,就是属于夺眼球吧。因为现在互联网上大家都想语出惊人嘛,要夺眼球,所以可能会……,真正的、真正的是害怕说话的、不敢说话的,因为怕误导大家。另外一个,没有绝对把握,你说出去,这万一错了呢?这个碎一地……

好吧,说点儿结论性的东西,今天也不能搞太久,今天还有点儿事。说点结论性的东西:第一,对美国经济的判断,美国的荣景就是美国的全年经济增长5\%点多,不是,这个第三季增长是6.9\%,全年经济增长差一点就追上中国,全年经济增长中国是8\%,它是5\%点……,忘了啊,5\%点多吧,追不上,但是最后一个季度是超过中国的。实际上它全年经济增长如果是不扣除通胀因素的话,也是超过中国、超过10\%的,比我们的8\%厉害的,但是他因为要扣除通胀嘛,他通胀水平很高,扣除通胀之后,剩下的这个数也是很惊人的。

可是昨天有朋友深夜打电话问我说:“为什么第四季美国经济增长这么高呢?”我说:“你知道美国经济的结构吗?”他说:“你想说什么?”我说:“美国GDP构成里边81\%是第三产业;1\%是农业;11\%是工业,是制造业,第二产业工业还不是11\%,制造业纯制造业11\%。农业还凑合,制造业完全不成、或者是第二产业完全不成,第三产业大增长。第三产业的构成里边最核心的:是金融和房地产。”我说:“还需要解释吗?”

我的朋友恍然大悟:“哦,懂了,原来是这样的GDP增长。”我说:“是的。”我说:“非常明确地告诉你,这个增长,这个高速增长不但没有意义,而是问题。”我们将来讨论柯立芝繁荣的时候,会讨论它的经济数据,到时候你就看出来了,那个繁荣有问题。美国的第四季经济增长6.9\%是有问题的,中国经济增长的下浮反而可能是健康,反而可能是一个正确的选择。风来了,飓风来了,蹲下,趴下可能是对的。

美国股市在2020年深调之后,出现了一个极为动人的深V,这个有些,这个就是比较,比较这个……就是……算了不说这件事,他们一见到深V会这个春心荡漾啊,这是个深V。美国两次出现深V,深V在股市里边意味着什么?意味着有一只大手生生把这个坍塌的东西拉起来了。这回2022年1月的深V能否拉得起来呢?

在远方的朋友提问的时候,我是没有办法回答他的,因为朋友说:“现在不是70\%的交易都是由机器进行的吗?”我说“这个可以这样说,但是我想说的是:无论是通货膨胀,还是股市、楼市的价格都是货币现象。它与美国的实体经济没有关系,它就是个货币现象,甚至它就是一只手拉着的,一只手画出来的,包括黄金的价格,皆如此。”

然而,然而,“操纵”——任何的金融操纵,都存在边际和极限,这个边际和极限一到,可能这个“操纵”会带来更为猛烈的灾难。我不认为拉第一个深V是一件好事,就是美国2020年一方面向老百姓撒钱,另一方面以迅猛的力量将股市拉回并且再次进入到这个历史的一个最高点。好多人跟我说:“你看苹果的业绩了吗?”我说:“是看了,美国头部被拉起来,这个业绩我是看了,那又能怎样呢?”

我知道,这里边的逻辑今天在这个场合也不能说。好多人说:“财政部,美国财政部没有买股票啊,美联储也没有买股票啊,你这个拉是谁拉的?”我不能说,好吧,以后在将来可以说的时候,我们再讨论这个问题。好多人认为中国才救市,美国不能救市,能不能?能不能不这样幼稚呢?能不能不要这样的幼稚呢?但你知道每一件事情代价是非常高的,强行将股市提拉到这个程度,将美国楼市提拉到了这个程度,代价是非常之高的。这个代价“谁、来、付?”这是个大问题。

今天的聊天,只能到此为止。该怎么做呢?今天也不想多说了。因为春节到了,先过春节吧。春节之后,我们讲完《资本论》,下下次聊天再说这件事情,今天呢就说这么多。我要说的唯一的一句话,最后叮嘱的话就是:“在美元资产上的朋友们,请你们撤退。”再一次报出预警。上周发出预警以后,好多人不甘心,是不是还会拉出一个深V,可能有机会。但我还是劝你们撤退,不要去等待最后的一次,假的……不要等待最后一个假的“破顶”,不要等了,走吧,好吧。

疫情仍然严重,隔离、阻绝,我也回不了北京。隔离和阻绝可能对好多人是构成一个巨大的问题和压抑,但无论如何坚持住,保重自己、保重家人、保重他人,身体比什么都重要。先提前预祝大家春节快乐,很想念大家,也总是觉得仍然不够啊,就是我为大家做得仍然不够,希望你们过一个愉快的春节。当然春节的那一个课我会仍然继续,而且那也是《资本论》的最后的三堂课。

《资本论》还有最后的三堂课,越到最后可能越精彩,我也希望你们能喜欢这三堂课。这三堂课结束之后,我想休息一下,休息一周或者是两周时间,然后开始准备凯恩斯的《通论》。然后还是大概约24堂课吧,这样就是到了2023年了,那个时候我就想可能我们就终止一段时间学习吧。因为其实我虽然学了很多,但是也很辛苦、有些累。另外呢,我想今年的九月份开始正式地执教鞭,去大学讲课了,精力可能也不完全够,因为还没有彻底退休嘛。

好吧,今天就拉拉杂杂聊这些东西。明天下午三点钟,有时间的话,我们再继续沟通和聊天。提前祝大家壬寅年春节新春快乐、健康、进步、样样好!好,明天见。

\section{马克思主义的中国化与现代化、聊几句壬寅年经济可能出现的变故}

大家好,今天是2022年的2月5日,农历壬寅年正月初五,先给大家拜个年。祝壬寅年健康、进步、愉快!今天我们讲《资本论》的第二十二讲:马克思主义的中国化与现代化。今天这堂课算是一堂比较重的课,也是对《资本论》开始进行做总结了。然后如果有时间的话,我们聊一下子壬寅年可能出现的经济上的一些变故。好,3点钟我们见。

今天是2022年的2月5号,农历壬寅年正月初五。又是一年,我们还在一起过年,我到了写字楼。香港已经扩散了,有两百多例源头不明的个案,分布在各个区域,看来香港要成为一个实验皿了,来证明一下子奥密克戎是否能够完成群体免疫,我估计这个月香港的情况会非常糟糕。从欧美的情况来看不理想、非常之不理想。

原以为奥密克戎大面积传染之后,再加上疫苗可以形成超级免疫。但事实上是:现在已经出现了奥密克戎one、奥密克戎two、奥密克戎three三种的变异,而且可以反复交叉感染。最近这几天呢,整个欧美的感染人数在下降,死亡个案在急速攀升。感染人数在下降可能是因为开放之后不纳入统计,而死亡数量急剧攀升倒反映了一个真实的状况,尤其是以色列情况近乎危急,所以大家还是要多加几分的小心。

好,我们先进入今天的课程,因为这是第二十二讲,本来这个第二十二讲我是打算放到最后的,把它放到第二十四讲的。结果有朋友催促,说你能不能放到二十二讲来讲,因为可能有些朋友要用这里边的一些内容。我想了想也可以吧,因为提前来讲一下子也是可以的,因为最终我们是要讲新马的,就是新马会,就是马克思主义的中国化和现代化的问题。这个课题其实非常之大,放一堂课,可能无法承载。

那么这件事为什么这么重要呢?是因为其实涉及到了,涉及到一个关键的褃节儿了。因为在讨论马克思主义中国化的时候呢,我们大体上把它分为五个阶段:第一个阶段,实际上是中国创党时期,主要是二十年代——创党时期,实际上走的是经典的列宁路线,其实用的是演绎法,就是那边有理论了,我们把那道理拿过来推演就行了;第二个阶段是创军,是建立红军的一个阶段,主要是在三十年代;

第三个阶段是建政,建党、建军、建政。其实在建军和建政时期,这个毛泽东做出了卓越的贡献,他已经在修改列宁和斯大林的一些做法,已经开始了中国化的过程。只是在毛泽东进行的时期是归纳法,他没有来得及进行理论的升华和总结。五十年代进入到第四个阶段,是建国的阶段,这个时候用的还是演绎法,就是我们是照抄了斯大林模式。六十年代毛泽东开始反思,所以在六十年代、七十年代做了某种实验,我指的是那场轰轰烈烈的事情,做了某种实验。

第五阶段是邓小平开始的,就是从八十年代一直到现在,这个时代非常重要。为什么这个马克思主义中国化难点在邓小平?邓小平、江、胡、习,经历了四代领导人,就是他们做的这个事情、这个时代是不是马克思主义?因为用演绎法无法解释了,但是现在也没有办法完成归纳,这一件事情非常重要,但是无法完成归纳。完成归纳,也就是现在我们要做的事情,那么就开始要进入新时代了。进入新时代就要给马克思主义一个更新……

更新的、更现代的一个新的解释。我们真的是幸运呐!因为我们处在一个大时代,我们处在中国这样一个神奇的国度,处在一个大时代。最有趣的是我们处在一个理论虚无的时代,就是它是个空白,我们说是有一套一套的东西,但实际上在理论上没有人能够把它串起来。所以肯定可以确定,二十一世纪的这个二十年代,我们这个时代将会产生一大批的真正的大师,因为这个过程很了不起。

我们回到今天的主题,第一个部分呢,要讲一讲什么是马克思主义。关于马克思主义呢,如果你上维基百科,他对马克思主义有个解释;如果你上百度百科,对马克思主义有另外一个解释。那么我们如何来理解马克思主义呢?讲个小故事。1935年,红军开始长征了。二方面军,贺龙同志,长征的时候从湖南抓了一个传教士,这是一个瑞士籍的英国传教士。抓了他以后呢……

抓了他以后呢,(可能)按照他的说法,因为他的书现在在香港这儿可以买到。国内将他后来写的书摘了一个部分,摘了一个部分,编进了有关红军的史料里边。这位仁兄被贺龙抓到之后,贺龙提出来就是要他拿钱来赎人,所以跟那个教堂商量,要……按照他们的说法是十万两白银。那个教堂哪有那么多钱呢?后来好像是交了一部分钱,但是达不到红军的要求,所以红军就带着他开始长征了。从湖南一直走到云南,走到云南。

这个传教士后来写书上说,他很惊讶。因为他是传教士,他觉得红军比他更像传教士,衣衫褴褛、吃不饱饭,整天在打仗,但是他们每一个人心中都存有一份信仰,意志坚定,又是那么得有才华、那么得勇敢,一群年轻人。他觉得非常惊讶,所以他在想办法理解他们的信仰是什么。所以这个传教士在长征过程中对红军产生了非常好的好感,甚至一度想跟着走下去,不想离开。后来红军考虑到后边的事情……

红军考虑到后边艰苦卓绝的战斗,就请他离开。后来他回到英国写下了这部书,并且他后来又重返中国,他仍然放不下这支红军队伍。我想说什么呢?好多人问:“什么是马克思主义”?我想说的是,马克思主义首先是一个信仰。红二代里边有个代表性人物,我就不说他了。他们重走长征路,过了雪山草地的时候,有人问他,如果当时是你,你能走过去吗?他说,我走不过去、我真的走不过去,他们太难了,他们简直是神呐,海拔3500米的高度、严重缺氧、没吃没喝。

马克思主义是包含了马克思主义哲学、马克思主义政治经济学和科学社会主义。我们在谈哲学的时候,很难想象马克思主义是一个信仰。但你如果全部读完马克思的书,或者是马克思和恩格斯的书,特别是在1942年到1948年这个期间写的那些东西,你会慢慢地建立起来对共产主义的信仰。信仰,是不是宗教呢?和宗教的区别在哪里呢?就是哲学与宗教这一道门槛在哪里呢?

后世人攻击社会主义国家无非两条:一条就是政教合一,就是他当我们这是一个宗教,所以叫政教合一;一条就是独裁专制,他当我们这个不分权。事实上,马克思主义真的不是宗教。虽然我一再说马克思身上有拉比的气质,但他整个描述的历史唯物主义的这个进程,对一个时代的一个阶级有了深刻的影响:当他们知道他们应该向哪里去的时候,他们充满了坚定的信念、前赴后继、流血牺牲。

在讨论中国问题的时候,我说,从李大钊开始到1921年中国共产党的建立,是马克思主义给中国一部分的年轻人送来了灵魂。而这个灵魂随着中国共产党嵌入到整个中华民族的体内,使我们由浑浑噩噩的状态开始重新有了灵魂。同时,它又是一股精气,灌输到中国共产党、中国军队的身上,使它有了一种精神。

当然,马克思主义是哲学。但是当哲学形成某种信仰的时候,它变得又超越了哲学,它很重要。你知道儒家——大儒,是有自己对人生看法的。你可以将大儒们面向人生的时候的看法当成是一种信仰,它不是宗教。当然与共产主义者很类似的是佛教徒,度己度人。度己,例如我,先要解放自己,然后尝试着解放别人、解放天下人。度己度人,当然其他……

当然其他宗教也有同样的叙述。马克思主义对中国的意义是非常非常重要的。当你将一个民族的一个基本的信仰推翻的时候,这个民族会丧失它的灵魂、会泄尽它的精气,这个民族必散、必死。如果我们不能理解到这个程度,如果中国还是想像改革开放初期——我这样说,可能北京的一些学者和朋友会生我的气——因为在改革开放初期80年代中国兴起了“新儒家”。为什么搞“新儒家”呢?

为什么搞“新儒家”呢?其中代表人物也是我的好朋友。“新儒家”的意思,他们就是想用“新儒家”代替马列主义。当然,他们可能不懂什么是列宁主义,我始终将马克思主义和列宁主义分开的,所以我始终不将马列并立。因为我们当成信仰的是马克思主义,不是列宁主义,大家一定要搞清楚。想用“新儒家”代替马克思主义,甚至他们在天安门广场想立孔子像。当时我们都写了文章,坚决反对这件事情。我不是反对“新儒家”,我不认为“新儒家”可以确立一个替代马克思主义的信仰。

如果有人想夺我中华民族的灵魂、魂魄,泄掉中华民族的精气,那么我作为这个民族的一份子,我就不能不说话了。其实讲马克思主义最核心的部分、最核心的部分在这里——由于有了这样一个信仰,由之而产生出伦理、法理、学理和治理。我想说什么呢?好多人将哲学进行教条化,然而他们不知道是一个理想生发出来的东西。如果你懂心学,你就会懂这里边的道理。

我那天说了两句话。我说,马克思主义所形成的共产主义的理想,他给了我们一个无主之天国、无邪之信仰。我的意思是说、我的意思是说:没这个东西,没魂了;没这个东西就不会有气贯长虹的精气,你什么都干不成。所以我对马克思主义的评价非常高,我说它是最智慧的哲学、最文明的主义。因为无产者是劳动者,又占据人口的大部分,对他们的改善,解放或改善难道不是最基本的人权吗?难道不是最伟大的文明吗?

阳明先生在哲学上达到了极为高的高度,所以他讲究的是化育与生发、有根而无限,化育与生发有根而无极,这个充分体现在毛泽东的思考中。我惊讶的是小平同志——邓小平。他在江西的这个小平小道之上完成思考之后,在整个80年代和90年代他做到了化育与生发、有根而无极,他做到了。我必须在这里边说,邓小平理论(好多人认为邓小平没理论,是我们没归纳,我们没总结)就是马克思主义中国化的成果。

邓小平的80年代、90年代的实践,甚至包含了本世纪初20年,一共是40年的实践,是马克思主义中国化的伟大实践。好多人认为这是新自由主义,那么我觉得你要如果这样想的话,可能你书读少了。如果你看所有的西方国家对中国的评价,没有一个人认为中国走的是新自由主义道路,虽然奥地利学派代表人物从古典到当今都对马克思主义持严厉的批评态度。

然而,我认为他们批的不是马克思主义。因为他们没读懂马克思主义,他们批的是列宁主义、是斯大林主义,他们批的根本就不是马克思主义。至于他们批判列宁主义和斯大林主义中的一些东西,对我们很有启发。我甚至在某种程度上认为这个批判或者是这个批判才是使我们重返马克思主义本源的一种重要的借鉴和推动力量,这是个很重要的一个组成部分,但是他们批的不是马克思主义。

大家可能很难想象,马克思晚年的时候是主张议会制的。我今天要将马克思主义分成四个部分:马克思主义哲学、马克思主义政治学、马克思主义经济学和马克思主义的社会学。其中马克思主义的政治学的部分算是马克思的弱项,但是马克思和恩格斯晚年同时考虑到无产阶级专政下如何解决专政问题,他们同时考虑到了议会政治。注意、注意、注意,议会政治。后来的苏维埃和我们的全国人大在某种意义上是落实了马克思的初衷的。当然,现在的好多内容和形式,未必是马克思希望看到的。

马克思最伟大的部分是哲学的部分,他给了我们灵魂和精气,一点儿也不过分。马克思主义第二个最牛的地方是经济学,他说清楚了资本的源泉、资本的分布、资本的流转,他让我们既要公平正义,又要有效率,真的了不起!至于马克思主义的社会学,我们讲过现代殖民原理和后殖民主义,一会儿我会——时间够的话,我会把这事再说清楚。那么马克思主义有没有问题呢?我必须说,我们最后的课的总结的时候,我会说《资本论》的问题,就是到23讲我们会讲《资本论》里边的问题。首先,这不是马克思的错,也不是恩格斯的错。

任何人都无法脱离时间和空间的限制,所以马克思主义具有鲜明的时代的局限性。时间的控制是因为马克思生活在十九世纪的这个中段吧,这个中间那么几十年时间,这是时间限制,他不可能超越他的时代。第二个是空间限制,马克思的活动范围是德国、法国、英国,在英国的时间比较长。这个空间范围除了地域的空间范围之外,也包括了阶层的控制。就是马克思没有当过公务员,没有领导过一个国家,马克思大部分的时间是在写作,大部分的时间是在写作。

我说了,第一个问题就是时代的局限性,而且这件事情不能怨马克思,好吗?第二个事情就是他在政治学上有巨大的缺陷,这个缺陷呢也不能全部去……我们不能对前人有苛责。就是马克思对于政治设计的形式的多样性与进程的复杂性预见的是不够的,但马克思本人并未教条、并未教条,只是他对形式的多样性与进程的复杂性没有足够的预见。这里边包括他认为英美先爆发社会主义革命,他没有想到英美是社会主义改造而非革命,而他认为不可能革命的,像……

类似于俄国和中国这样的地方,他竟然爆发了革命。所以我一再强调,在政治上面,马克思主义理论的政治学必须考虑形式的多样性与进程的复杂性,形式的多样性是包含了有各种各样的可能性来进行社会主义建设。威尔逊在美国的社会主义改造,难道不是社会主义建设吗?艾哈德社会市场经济难道不是社会主义建设吗?邓小平的改革开放之后引入社会资本和国际资本,难道不是社会主义建设吗?形式的多样性不能教条与进程的复杂性是迂回曲折的,不是一天就能成的,不是把大家凑一个锅里边……

不是给大家圈一个食堂里边共产主义就实现了。所以在政治学上呢,由于马克思可能他独特的性格,所以他留下了一些遗憾,这些遗憾导致以后的马克思主义者或者社会主义者走了许许许许多多的弯路。我在讲中国化的时候我要讲中国人是比较幸运的,就是中国人在处理马克思主义中国化这个问题上,我们由于拥有深厚的儒家的哲学观点和我们深沉的历史经验,所以我们没有进行机械、教条的简单的模仿和演绎,所以我们走过来了。

第三个部分就是马克思的这个局限性的部分,就是他在政治学上面有一点点的问题,在社会学方面。关于社会学的问题呢,我想马克思虽然是一个坚定的唯物主义者,但是马克思可能对机械化时代理解得会深刻一些,但是马克思还没有来得及经历电气化和信息化的时代,所以对历史的进程的理解,就是社会在大历史进程中的这种变迁,以及变迁中的社会发展,可能在最后的思考的部分可能有一点点的偏差。

我去马克思墓的时候,对马克思墓上写的“全世界无产者联合起来,以及无产阶级专政”,就是他对社会的这个理解,可能马克思没有预料到,除了社会主义革命,还有社会主义改造。社会主义改造以美国、以威尔逊为标志,他将社会变成了一个中产阶级为主体的社会。那么中产阶级为主体的社会要不要马克思主义?要不要走社会主义道路呢?该如何走社会主义道路呢?还需要无产阶级专政吗?如果还需要专政的话,应怎样专政呢?

我说了,这三个问题无论是局限性,无论是形式与进程的多样性与复杂性,以及对社会学的看法,都不是马克思本人的问题。因为作为一个伟大的思想家,他能做到那个程度就可以了,剩下的事情应该后马克思主义者和新马克思主义者或者新社会主义者来把它完成,但不能失去它的灵魂,不能失去它的根本。我想,如果是马克思活到当下,他会对苏联解体以及中国的社会主义建设做出一个崭新的评价。

马克思会怎样评价苏联呢?他会接受我国关于修正主义的描述吗?比如说,我们总体上会认为是赫鲁晓夫、戈尔巴乔夫葬送了苏联社会主义。可我不这样认为,我认为列宁一开始在缔造苏联社会主义国家的时候,就种下了结构性的问题,只是因为仓促之间写下《国家与革命》,对无产阶级专政的政治学的解释偏颇太多了,所以……

所以我想马克思活着,列宁活着,马克思会跟列宁说:你不是个马克思主义者,你才是修正主义者,因为你要做的那个事情并不是我想做的,并不是理想中的模式。可不可以作为多样性和复杂性的一个过渡?当然可以,包括战时共产主义都可以。但,如果这个东西一旦固化为教条,那么苏联的最终的结局在十月革命都注定了、注定了,七十年的寿命其实是注定的。讨论这些问题对中国的意义非常重大,是一面一面的镜子,我们反观我们自己,哪些是对的,哪些是错的。

马克思主义哲学有伦理学的价值,你知道它对中国的伦理冲击是多么的大!中国的伦理实际上一直,儒家没进化,我们是建基于血统的伦理——君臣父子。新的马克思主义的伦理学的意义非常重大,同志——这两个字非常关键,就是同志超过了亲人。其实你如果不能理解马克思主义的伦理学的价值,你就不能理解当下的社会现象。举例,最近北大那个报告,说,我就不说数字了,因为我不认为他这个数字是对的,就是说大部分的留学生不愿意回国,其中就是研究半导体,就是这些最尖端的人不愿意回来。

为什么越有本事越不愿意回来呢?因为,他们的解释是:国内的机构——从政府开始到商业机构,特别是国有企业,没关系你是进不去的。你虽然学得不错,真有本事可能也会埋了。而国外相对而言,你真是金子,是有机会闪闪发光的。为什么我们在结构上出了问题?就是我们现在是伦理,因为伦理是法理的基础,所有的制度看似偶然,其实都有伦理基础。我们半部《论语》治天下,就是以伦理作为立法基础,所以我们讲究的是君君臣臣、父父子子,还是一个“亲”字——血统。

马克思主义的伦理观是同志而非血统,这一点非常之重要。如果我们真的是学懂了马克思主义,特别是他的哲学观用于制度建设,制定法律、法理,那么那些孩子呢更愿意回到这儿来,为了一个共同的理想一起奋斗,而不是依附于权力或者是资本。你懂马克思主义哲学的重要价值,不光是信仰,它涉及到伦理、法理、学理和治理,整个系统这个哲学都是个根。我国研究马列的人多,

但我国用现代的哲学、经济学、政治学、社会学来重新解释马克思主义的有分量的学术著作,不多。不能说是零,还是有些人做研究的,不能说是零,但的确是让人失望的。其实你对照治理的先进程度,你不得不佩服马克思主义在那个年代就是最先进的文明,到今天依旧是如此之先进的文明。我们反血统论多长时间了?我们反得动吗?但马克思主义就可以解决这个问题。

关于马克思伦理的问题、学理的问题、法理的问题和治理的问题,今天不能展开讲。因为将来写这个《新社会主义论》里边,这是重要的内容,这是非常重要的内容,那个时候我们会把它讲的清楚一些。在马克思主义的哲学里边,除了唯物主义,就是辩证法,辩证法的部分也是非常重要的。比如解释当下的一些事件,比如说是否与病毒共存。按照西方,特别是供给学派的说法,与病毒共存是理性的经济行为,那么用辩证法来解释就很简单和很清楚。

理性的经济行为,它的假设是自由。生命重要还是自由重要呢?生命高于自由,死了还要自由个屁啊!但是它这个假设就是我们可以让一部分人去死,这是理性的经济行为,与病毒共存。其实我们用马克思主义来解释中国的防疫的合理性、合法性,就很容易解释了。美国已经死九十多万了,很快到一百万,看这个样子死亡还有可能再一次加速。我真的对新自由主义感到绝望,就是这是什么主义呀?!

今天讲第一个部分就是什么是马克思主义,讲几句马克思主义的政治学。马克思主义政治学在马克思主义整个的理论体系中是比较少的,也可能是由于马克思在这个方向上面没有的那么多的切身体验或者是实践经验,所以他在处理这个问题上局限于理论更多一点,这个也是后来列宁、斯大林他们犯的错误最多的地方,就是到底马克思主义政治学是一个什么样的叙述,是一个什么样的解释。这里边我想从对手的角度来看问题。

通常政教合一是一种比较原始的政治现实,就是欧洲在宗教改革之前是政教合一的体制,我国西藏在解放之前也是政教合一,今天的伊朗也是政教合一。通常我们认为这是一个没有进化好的政治系统、政治体制,西方攻击中国——攻击当下的中国,攻击当年的苏联,攻击大多数社会主义国家,都认为是政教合一。是政教合一吗?其实不是,其实不是。首先我们没有“教”,是信仰、是理想,

是信仰、是理想,是以这个信仰和理想为伦理,然后以此伦理来建立法律体系的。但是由于我们在解释现实的时候,没有说清楚,在处理具体问题的时候比较混杂。在社会主义国家,通常政治领袖也一定是极富水平的思想家,甚至本身就是哲学家,这样呢貌似教主,其实不是,其实不是。有痕迹没有?有的。共产国际就是个痕迹,共产国际就是个痕迹。后来斯大林把这个解散了是对的。

如何破解整个西方对这件事情的错误看法呢?其实我们不仅仅是要做学术上说明,我们可能也要做未来政治结构设计上的一些安排,才能把这件事情处理得恰到好处、恰到好处。在古田会议上面,我党确立了一个重要的军事原则——就是党指挥枪,党理政、党立法。其实这个原则,我们一会儿在讲政治上的时候,这是一个非常好的原则。

然而这个原则不是“教”,不是“教”,甚至这个原则不涉及信仰,不是“教”,不是信仰,是非常先进的一种分权制衡,党指挥枪是一种非常先进的制衡。所以我们看到无论是在抗日战争还是在解放战争,国民党军队成建制投降,我们的军队不可能,它就在这个问题上。好,解释第二个就是对手攻击我们的东西,就是独裁、专制。我们是否独裁或者是专制呢?其实不是,解释中国今天的政体,其实是一个非常艰难的任务。

毛主席很了不起,他将马克思的无产阶级专政变成了人民民主专政。人民民主专政不是无产阶级专政,人民民主——人民比无产阶级大了更多,因为人民里边包含了中产阶级、小资产阶级。毛泽东真的很厉害的,就这一件事情你就看到他在政治上面是比马克思和列宁看得远的。人民民主专政是可以包容今日之中产阶级崛起的,这是很厉害的一个。它不是一个简单的心血来潮,它是一种非常重要的政治设计。

关于政体内的分权问题——立法、司法和行政。分权的问题,毛泽东在文革做过尝试,就是让人民群众来参政议政。但是我已经说过了,就是人民群众应拥有立法权,而不应拥有司法权和行政权,尤其是司法权。司法权是一个非常专业的东西,它跟你的阶级没有必然联系。好多人一直理解不了这一点,就如同我在说香港问题的时候,我说:“请你们不要轻易碰那些大法官,不要!不要这样!不要这样!”

在这个问题上可能还需要一个漫长的理解和讨论过程,因为什么是真正的人民代表问题,毛泽东尝试解决,没解决好。后来呢我们搞到今天就出现了好多体育明星、演员、资本家,反正什么人都有,其实这是马克思最反对的东西。马克思说无产阶级专政里边,对无产阶级专政的代表是有详细论述的,这一件事情列宁不敏感,斯大林不敏感,毛泽东非常敏感,因为他谈到了资产阶级法权,就是当……

就是毛泽东他的视觉的深邃程度是追得上马克思的,他认为虽然公有、国家代持,但是代持的人是官僚、是干部、是官僚。当官僚代持资产的时候,它一旦碰到资产,它就会异化,这也是马克思的看法,它就会异化为新生的资产阶级,这是毛泽东的结论,对不对呢?他真的是厉害,是对的。苏联为什么解体呢?就是那个异化的过程神一般的快速,大家很快就异化成新生的资产阶级,并且将前苏联吞并了、出卖了,毛泽东看得太准了。

那么如何让人民民主专政里边的人民代表具有真正的人民的代表性呢?这件事情,我们将来在讨论新社会主义论的时候会提供一个系统性的建议。我坚决反对选举的,我是坚决反对选举的。不管什么样的选举,它实际上都意味着资产阶级专政,没有办法。谁在管理盎格鲁-撒克逊的选举呢?默多克,默多克的这传媒帝国。他为什么要这样做呢?钱。

其实马克思在整个的论述里边,他最难的点也是这个地方,就是他想不明白怎么办好。想不明白也就想不明白,但这事儿可能难不倒中国人了,我们在这个问题上可能有办法,当然我们不会公车举孝廉,也不会再玩科举制,我们有我们现在新时代的、信息时代的新的方法,我们能解决这个问题,但我们今天暂时不说这么复杂。政治学我就念叨两句、念叨两句,我就先不多说,因为这里边涉及到一系列的问题,比如说威权主义、比如说一党制、比如说国家主义,

比如说我定义的国家资本主义、我定义的国家资本主义发展出来的官僚垄断资本主义等一系列的关于政治学的概念。西方是金融垄断资本主义,东方(我们不是,我只是说可能)我只是说前苏联是官僚垄断资本主义,它跟社会主义已经相去甚远,这是一个政治学问题。马克思当年没有说的特别清楚,列宁给理解偏了一点点,我们经历了好几代人的实践,该做总结了,该给它一个结论了。当然政治学的部分敏感,但这一件事情谁做下来谁就为中华民族立大功了。

关于马克思主义的经济学,正是我们《资本论》讨论的东西。这个经济学里边(马克思的《资本论》真的很厉害的),他讲了资本的源泉,其实源泉讲剩余价值不是那么重要,虽然也是开山立论,他讲的是个公平正义问题。第二个部分是资本的分布,资本的分布意味着一个效率问题,分布就是资本的整个的流转和分布,它意味着一个公司、一个地区、一个国家整体的效能问题。公平正义是第一的,第二个是效率问题,至于第三个部分,就是第三卷强调的资本流转的部分其实涉及到……

其实涉及到大国博弈啊!如果你《资本论》读懂了,其实无论是个人问题、机构问题、还是国家问题,我说是经济类的,其实就通了。有一条就是对马克思早期的论述和晚年的论述产生了巨大的歧义,就是马克思在的时候就有很大的争论,马克思死了以后就有更多的争论,这里边争论涉及到关于无产阶级与中产阶级的看法,因为中国跨越一万美元之后,将来是以中产阶级为主体的社会,这个还好,毛主席说人民民主专政,就是不再是无产阶级专政,那么这个时候这个主体性的阶级应该怎样实施他的这个政治权力,

实施他建基于产权的政治权力,资产阶级法权、资产阶级法权,而且它这个法权如何能够不像今日之美国那样,已形成的中产阶级和中产阶级的法权被剥夺了,美国很快会陷入严重的内乱,一会儿我们结尾的时候会讲几句壬寅年的那个第三十九象,讲几句。不能老是这么认真的讲马克思主义和《资本论》,我们有时候稍微搞一点封建迷信、搞一点封建迷信。在经济学里边,马克思并未提出消灭资本和消灭市场,对吧?马克思是在认识资本和认识市场,没有提出消灭资本和市场。

消灭资本、消灭市场这件事情是留给共产主义的,不是社会主义者要做的事情,但在一个特定历史时期,我们急呀,跑步进入共产主义,我们就提前把资本和市场都给干了。这个大家有空去读这个西方人对马克思的批判,其中以哈耶克为代表的,其实奥地利学派对马克思的批评是最多的,他就说:“马克思说的剩余价值有意义吗?如果劳动是个商品的话,劳动定价是供需定价,对吧?货柜车司机没了,他的工资会涨到天上去。”那个怎么来衡量剩余价值呢?他的意思是有资本和市场在。

当然我们将来在讨论这个时候,在马克思主义经济学里边的时候,会系统的驳斥奥地利学派、特别是供给学派的这个对马克思主义经济学的、对《资本论》的一些攻击。是我们在行走、在落实马克思主义与社会主义的时候,有一段路走偏了,所以我们消灭资本、消灭市场,后来我们发现没有资本、没有市场,没办法最有效的配置资源,并提高资本的流转的效能、提高创造价值的能力,反而会影响经济的整体发展,这样对建设社会主义是不利的,后来才认识到这一问题,其实这个问题比较尖锐的发现的还是邓小平,真的了不起的。

经济学讲的是辩证法,我们从无产阶级消灭资产阶级开始,到无产阶级变成资产阶级,这是不是、这是不是出乎马克思、列宁的预料呢?第二个,我们从消灭资本到利用资本,到有效管控资本,这是不是也是挺有趣的一件事情呢?我们到批评剩余价值,到存留合理的剩余价值,我们到消灭地租,到存留合理地租,以利于资本积累,是不是在经济学上有了更新的、更切实际的、更深刻的认知呢?在马克思主义经济学的部分……

在马克思主义经济学的部分,可能要做的工作量也是非常大的,其实从哲学到政治学,到经济学,每一个部分都可以写一本书的。现在这个时代太好了,你在这个领域里边完全是像一个枪手进入西部,插旗子就是你的地,因为没有人来做这件事情,西方人不会做,我国的学者全都在研究西方的经济学,他怎么会研究马克思主义呢?我国的马克思主义者,我不说他们了,但你们可以的,或者是将来我们组织起来把这件事办利索了。因为它太好了,它就是蛮荒之地,没人来。就是我们跑这儿晃悠一下子,我们决定种不种这块地。

关于马克思主义的社会学的部分,其实我觉得这个社会学的部分涉及到我国的对内对外的一个解释,将来我们可能在这个事情上有很多的事情要做。我想说的是前两天发生的一件事情,前两天德国海军司令去印度说了一番话,他认同、认同普京,认为克里米亚丢了就丢了,他里边的话很重,他说:“我们都是信仰上帝的,那是东正教,我们是基督教,信仰上帝,我们都是白种人,所以我们的共同敌人是中国。因为他和我们不是一个宗教,也不是一个种族。”作为一个当代的德国人,他表达出……

作为一个当代的德国人、海军司令中将,他表达出了一种非常残酷的种族歧视和信仰歧视。如果、如果……,当然他现在被撤职了,他撤职的原因和中国没关系,就是全部白种人或者是全部的西方世界注意力是他说普京是好人,这个事情大家接受不了,所以把他撤职了。而不是说他说中国,而且他竟然预言十年之内必与中国一战。那个新上台的绿党的小姑娘,八零后还准备用核武器进攻中国。这是个社会学问题,好不好?这是个社会学问题。

马克思《资本论》,我讲过殖民原理,马克思当时写的叫《现代殖民原理》,第一卷最后一章。后来萨义德写了《后殖民主义》,我们也讲过了。殖民主义、殖民主义就是新自由主义里边的核心内容啊,不要认为新自由主义是可以放诸四海而皆准的真理,在新自由主义的发明者眼里,它是有人种的、是有信仰的,你想搞新自由主义都没门儿。好多朋友说我们一定要用马克思主义吗?我们不可以用新自由主义吗?那么就请你再去听一遍那个海军司令在说什么。

我们要做的事情是要重新建立马克思主义的社会学,我们要对人生、对社会、对世界有一个完整的看法。这个完整的看法不仅仅是要针对我们自身做出解释,也要针对世界做出解释。当世界不接受我们的解释的时候,我们可能要用大炮和导弹来向他们做出解释。因为五百年的殖民历史,他们基本上是用刀子和子弹来向我们解释的。他说“十年之内必有一战“,我看未必需要一战,如果我们的学问做得好,可能能说服他们,如果学问做得不好,那么语言不能解释,那对不起了,只好用血与火做一个解释。

无论如何今天讲不了马克思主义中国化和马克思主义现代化了。中国化我刚才讲了有几个部分,因为现代化涉及到历史唯物主义的延伸,就是历史唯物主义要重新写,重新解什么叫新的历史唯物主义。另外,我们进入信息时代了,资本的载体变了,就是《资本论》里边所陈述的资本的载体变了。第三个,与产权对应的法权。如果信息、数据是产权,与它对应的法权是什么?第四个……,第四个先不说吧,我有点小小的压力。

好,切成两半儿吧,今天就讲马克思主义的中国化与现代化的第一个部分——什么是马克思主义?我以为能讲完,没想到一个多小时还说不清楚。我看看吧,我不能挤占二十四堂课的话,我就把它当成一次聊天,如果不行的话就再加一堂课吧。那我看一下吧。我原来就是大纲是安排的二十四讲,我们再看吧,反正是整个的《资本论》讲下来其实也是跳跃性的,因为二十四讲想把一个主义说清楚、一本书说清楚挺难的。我现在开始在准备《通论》,其实《通论》更难,因为里边数学太多了。

好吧,说几句。其实应该不能说卦象的哈,应该,因为这天机不可泄露,我点到为止,我也不说清楚,反正你们自己体会。因为我要不说呢就是谈不了经济,说了呢就把经济也谈了,就两件事情合并成一个事情,我用最简短的语言来陈述。你们都看到三十九象,三十九象是一座山,上面站了一只鸟,旁边有太阳。它有谶,谶言,然后它还有颂。谶言是这么说的:“鸟无足,山有月,旭日升,人都哭。”

鸟无足,好多人说那没足下边是个山,不就是个岛吗?瞎扯。还有一个字,它繁体字不是这么写的啊,简体字是鸟下边是个“几”,没那四个点,是个几,几块钱的几,那个字念fu。凫这个鸟是水鸟,类似于野鸭子,它是生长在水里边的,腿很短,然后有蹼。它划水的,它轻易不上山的,因为它是在水中生活,它体态比较大。现在可能已经见不到凫这个鸟了,至少我没见过凫这个鸟。它的特征是黑的、黑背。

凫,什么时候会上山的?通常凫上山顶,大体上是天崩地裂的时候,河水沸腾,江河沸腾,就是出灾难了,不管是泥石流也好,还是地震也好、海啸也好。凫为了逃命,笨鸟先飞,蹲到山顶上去了。它是一个象,好多人解释那只鸟是美国,或者好多人解释什么岛,瞎扯。山有月,山有月,其实左边一个山右边一个月也叫岄。他说的是什么样的山呢?是倾斜七十度角那种准备……

就是有点像比萨斜塔准备倾覆的那种山,叫岄。还有一个山有月,就是下边俩月——崩。这就跟“鸟无足”合上了,因为是凫——“鸟无足”,凫,“山有月”。第三句话,“旭日升,人都哭”。折腾完了,灾难结束了,太阳升起,新的时代,新的一天、新的开始、新的时代,人们已经失去了亲人和家园,哭。颂我就不解了,我就解这么多。

他说的是自然的景观,也可以理解为经济现象。其实有的时候经济危机也是沧海横流,资本横流、沧海横流,也是一片狼籍,也是痛失家园、痛失资产,也是“旭日升,人都哭”。嗯。至于会不会预示着其他的东西,比如说战争、或者是核战争呢,我不知道。我只是说,既济……

我只是说壬寅既济里边这个推背图的这幅像极为凶险,极为凶险,也算是我看的差不多是最凶险的啦。所以你懂得,我没那么大的本事,可能早了一年、一年半,短股长金。不是说其他资产不好,不是说债券不好、股票不好、楼不好,不是这个意思。请你做一只凫吧,请你做一只凫吧。

总有好多朋友有些批评、有些抱怨,其实每个人有每个人的命运、每个人处在不同的角度看待、在看不同的世界。我们大家凑在一起有缘分,一颗心对一颗心。不可能每一件事情都对,不可能每一个时间节点都对,但因为是一颗心对一颗心,我能把我知道的说出来,哪怕只是提个醒呢,好吧。新年不该说这么复杂的事情,但我又不敢不说,因为下周就开市了。

至于该怎么做,自己决定吧,我已经说清楚了。另外、另外,无论如何要继续警惕疫情,这件事情是否结束了,是否真的是超级免疫了,是否真的出现了所谓上帝抗体了,等等看吧,不急的。做好自身的防疫,不给家人添麻烦,不给国家添麻烦。好吧,我们今天就聊这么多,再次祝大家壬寅年进步、健康、快乐!谢谢大家!

\section{关于《资本论》的再思考、乌东问题}

大家好,我先试一下麦。现在是2022年的2月19号,壬寅年正月十九日,今天是第二个节气——雨水。今天我们是正式课,是《资本论》的第二十三讲,是关于《资本论》的再思考。原来是打算……就是原来大纲是写局限性的,后来我觉得我们没有资格来讨论《资本论》的局限性,所以我们把它改成《资本论》的再思考。这堂课极重、极重、非常重,因为发现一本书的问题,其实这个难度是非常高的,而且是非常危险的。讲完这个课之后,谈几句乌东的危机问题。好,一会儿见。

大家好,今天是2022年的2月19号,壬寅年正月十九日,是二十四个节气的第二个节气——雨水。今天香港还真下了雨,雨挺大的。气温骤降,然后我还是来到办公室,办公室安静一点,然后把资料准备的齐备一些。今天是讲《资本论》的第二十三讲——倒数第2讲,也是极为重要的一讲,这一讲的题目叫关于《资本论》的再思考。在原来的大纲里边,这一讲——第二十三讲,应该是讲《资本论》的局限性,或者是《资本论》批判,但后来思虑再三,

我们还是将这一讲的题目改为“关于《资本论》的再思考”。因为每一本书,它都有历史的局限性,每一个人都难免有自己的局限性,当然马克思也不例外。能不能讨论局限性呢?当然要讨论这个局限性啦,因为我们读一本书既要看到它的正面的意义,也要看到它可能存在的问题和局限性,甚至要找一找它负面的意义和影响,这样的话这本书才算是读完整了、读透了。另外,全世界对《资本论》的批判汗牛充栋,比赞扬多得多。

我这二十年在香港,去中央图书馆差不多把批评马克思和《资本论》的书,大部分的书我都看了吧。我还买了一部分有代表性的著作,在家里边。批评,其实批评马克思的最为刻薄和尖锐的人,你们可能真的不相信,是中国人,恰恰是我国改革开放之后的所谓的思想家和经济学家们在进行系统地批判。其中一部分人跑到了欧洲,跑到了美国,当然也有一些人在港澳台。他们写下了大量的东西,批评马克思、批评《资本论》。但我想今天跟大家说的是,他们批得不成,因为非常之浅薄。

西方的思想家特别是经济学家,也有一些人系统性地批评马克思和《资本论》。但我也必须实事求是地说,至今为止我未见一部真正有深度和有系统性的著作。也就是说,类似于像供给学派、像奥地利学派,马克思活着的时候他们就在攻击,马克思死了,他们还在攻击,但攻击有意义吧?有折损吗?没有的。那既然这样,那我们为什么要来讨论这个话题呢?其实历史必须向前发展,这是历史唯物主义;另外,马克思又是辩证唯物主义者,必须从正、反两个方向进行概述和总结。

请相信一个马克思主义者的真诚,真正的维护就是在不断地完善。完善的过程,它就需要进行再思考,建基于当下的时代,建基于目前的状况进行再思考,这个再思考其实极为重要。北京的朋友前一段时间发帖子,非常感慨。他说:“在这样一个混乱嘈杂的时代,还有一群人、数千人在一起读一部书,这部书的名字叫《资本论》。”他说:“你们真的很了不起,你们做的事情不仅有意义,而且可能真的算是一种文化现象。”

我说:“其实我要感谢的是你们,如不是你们跟我一起,我这个第五读是进行不下去的,因为确实是外部环境如此之乱,有的时候杂事又如此之多,不是逼到这个份儿上,一共这二十四讲,也不能静下心来进行这一次的完整地梳理。”所以我跟一个美国的朋友,这也是个极有水平的朋友,他前两天打电话说起此事,他说这应该算是个文化奇迹,我说不能算文化奇迹,因为现在已经没有正经的学校在教《资本论》了。

好,回到今天的主题——我们的第二十三讲,关于《资本论》的再思考,我将这个思考分成五个部分:第一个部分是关于价值论的部分,就是马克思的价值论是我认为整个《资本论》的基础框架性的分析,这里边包括了价值论和价值观。讨论这个问题之前,我想先构建一下子我们的框架,我的框架也是我们一起来构建的框架,这个框架是一个中国人的思维逻辑。通常我们把理分成五个层级,我们管它叫五理(道理的理),它是一个塔型的结构。最顶层的理,我们管它叫天理;第二层的理,我们管它叫伦理;

第三层的理,我们管它叫学理;第四层的理,我们管它叫法理;第五层的理,我们管它叫治理。第一层的理——天理,其实是神——由神来叙述或构成的,讲天理。其实能讲天理的就是神了。在中国有没有关于神的著作?有的,那就是那部《易经》,它讲的是天理。伦理是圣人之说,最经典的著作就是《论语》。中国很神奇,在天理与伦理之间有一部著作它叫《道德经》,它既是天理又是伦理。这个北宋的(赵普)……

赵匡胤的宰相,这个他说半部《论语》治天下,说的就是伦理,因为没有伦理就没有制定法理的逻辑基础。半部《论语》不是《论语》上面有治理天下的道理,而是《论语》中的伦理构成了大宋法律的——大宋律的逻辑基础。伦理非常重要,一个伟大的民族一定会有一整套的伦理体系的,这是圣人之说。在此之后是学理,中国的学理著作汗牛充栋,类似于像《管子》、像《盐铁论》。

中国在学理上面这5000年很牛很牛,上从天文,下从地理,中从社会,各种各样的学理方面的知识太多了,我就不举了。法理方面的东西就是中国也非常非常厉害,因为从《大秦律》一直到《大清律》,中国古代关于法理就是非常的系统的、成体系的。治理方面最经典的代表就是《资治通鉴》,我们中国好多人在治理国家的时候都会读《资治通鉴》。治理的成效比较好的政策,例如“一条鞭法”。所以神是管天理,圣人管伦理,贤者……

贤者来叙述学理,政治家、大政治家处理法理,能臣、干臣处理的是治理。在讨论价值观和价值论的时候,为什么要说这五个理呢?因为关于价值的部分介乎伦理与学理之间,就是马克思讨论的是伦理与学理之间的问题。马克思不是神,因为他没有办法从天理的角度来讨论价值问题,也许在神的眼里边,价值这个东西是另外一种的叙事体系。所以,关于价值的部分,特别是剩余价值的部分,

关于价值的部分,特别是剩余价值的部分,马克思在不小心之间、不小心的时候,可能在叙述方面出现了一点点的偏差。什么意思?因为即便是剩余价值理论,它也应该是中性的,应该不将剩余价值做贬义判断,它是中性的。也就是说剩余价值是有好处的、有意义的、有存在必要的,甚至不应该被消灭的。也许神会这样想、圣人会这样想,但是贤者可能不一定能这样想,但……

我们为什么要说剩余价值是中性的?因为我们既然知道资本积累的源泉是它,我们只能是将这样一个存在(存在的合理性)控制在一个合理的范畴之内,而非简单的将它进行某种的贬义,甚至在某种意义上消灭。消灭剩余价值,甚至消灭资本,就是后来列宁《国家与革命》里边提出来的事情,斯大林去落实的事情。后来显然它并非天理,也未必就合伦理,只是在学理上的一种认识。但是由此而建立的法理和治理,可能天生就存在着问题。如果我们不用这样一个……

如果我们不用这样一个逻辑框架来分析的话,就有可能会出现问题。在讨论剩余价值的时候,它就不简单是一个价值问题,它也有个价格问题,就是定价逻辑的问题。这个问题是被奥地利学派,特别供给学派攻得最多的,就是这个问题。他们认为剩余价值量是由谁决定的?他们认为不是由资本决定的,而是由市场决定的,是由供需来决定的。就是剥削工人没有、剥削多少是由市场决定的,不是资本家人好、人坏来决定的,是他们攻击的一个非常重要的点,就是市场决定论。

同时,我国当代的经济学家,我不点他的名字了,也反复在说,如果剩余价值不好,消灭了剩余价值,例如北朝鲜、古巴,他们的经济是不是应该比瑞士、瑞典更好一些呢?因为他们是资本主义,他们有剩余价值,而事实上并非如此。这个课题非常重要,因为我们讨论价值观、价值论的时候,它很快就涉及到资本增殖,资本增殖,资本的积累和资本的流转的问题。如果这个事情我们没想明白,后边的理论框架就不可能对。

所以在这个问题上,我必须说原《资本论》关于剩余价值的理论,要将之推回到中性的观点、中性的观点,甚至对资本的认识,也要推回到中性的观点,乃至于对资本家的看法也要推回到中性的观点,不作一般意义的褒贬。因为如果我们从天理的角度看、从伦理的角度看,可能会更清楚一点。如果我们从学理和法理和治理角度看,那是另外一回事情。所以第一个问题涉及到的我们今天讨论的第一个问题,就是关于价值论的部分。我们认为从价值论的角度出发,从价值观的角度出发,不要轻易……

不要轻易将某一种存在的事物或者是现象,例如剩余价值、例如资本,简单地将它进行道德判断。如果这个道德判断影响了我们对这个事物的理解和认识,可能会走向历史的反面。这个事情涉及到我们百余年的实践——社会主义实践,其中类似于供给学派关于北朝鲜和古巴的问题的质疑,关于瑞士和瑞典的质疑部分,其实不是剩余价值理论的错,而是我们将它偏向了某一个方面。我在读《资本论》的第一卷的时候,我认为马克思本人也并非要将它进行一般……

马克思本人也未必要将剩余价值和资本进行一般意义的贬义,尤其在他晚年的时候。其实马克思知道这不是一个一般意义的贬义可以解决的问题,因为我们没有办法超越历史发展的特殊的阶段,就是我们不能一步跨入共产主义,那我们必须面对资本主义,面对资本主义它就有资本、就有剩余价值,而资本和剩余价值我们只能管控它的边际,而不是消灭它们。如果你想消灭它们,其实你不可能消灭资本,只不过叫国家资本主义而已;不可能叫剩余价值,只不过是另外一种说法而已。我想说的是,在国家资本主义条件下,有没有剩余价值呢?在社会主义,比如说中国1949年之后的社会主义,有没有剩余价值?

有的,当然有。只不过这个剩余价值被我们集中起来,进行了伟大的资本积累,共和国的伟大的资本积累,我们进行了迅速的工业化。这个资本积累的过程相对而言比较的公平、正义,所以老百姓认为还是好的,虽然大家集体贫困,但是是好的。但我们由于没有办法去殖民,没有其他的办法去创造信用,我们只能进行原始积累。这个积累的过程非常痛苦,是我们集体将我们创造的剩余价值一起用于共和国的积累。你们知道这里边涉及到一个什么问题吗?涉及到对当下财富分配的一个看法。

当下,由于市场机制的建立过程中存在着这样或者是那样的问题,所以我们产生了全世界13\%的亿万富豪都是中国人,都是在中国,远远超过了美国、欧洲和日本的总和。这些富豪他们没有办法,就是中国当代的经济学家没有从《资本论》的角度来解释,就是他们占用的那些剩余价值是我们集体积累完成的。他们没有权利,无论是天理、伦理、学理、法理、治理都不应该是这个结构。但由于我们没有办法去阐释这五理,所以他们借助了一些空隙。

我上堂课就说过那个阿里的高管,在美国上个月前几天买了第四套,好像是四点几亿还是五点几亿美元。在地租上面获利是不对的,在数字租上获利就对了吗?那上面不是我们的剩余价值吗?当然是了。只不过是我们在解释上面缺少一个系统性的解释。我们要完善这个部分,就是想将它做天理、伦理、学理、法理和治理的一致性描述。其实这个工作量非常之大,也极为艰苦,我们今天也只是提出来而已。因为提出来就不简单,想把它做好那真的。

好。我们讲第二个部分,关于资本积累的秘密。这个资本积累的秘密里边,马克思将资本的积累的源泉说成是剩余价值。我个人同意这个说法,但我还愿意把它分成更细致的三个层级,就是原始资本积累有殖民的过程。这种中期的资本积累就是我或者是讲这个古典意义的这个资本积累其实是靠地租。现代的资本积累靠的是剩余价值。大概分成这样的几个不同的层级和阶段,而他们的构成也是不同的。为什么说马克思没有说错呢?

就是如果我们细致的来看,其实殖民掠夺的也是剩余价值,地租里包含的也是剩余价值,数据租里边包含的也是剩余价值。所以本质上还是剩余价值,就是马克思的资本积累的理论体系没有错,但表达的情况不太一样。这里边我想再阐释一点我或者是我们平台上的一个基本的判断吧,这个看法因为这是作为我们立论的一个基础。就是在说资本积累的秘密之前我们必须说一句话,什么是资本的本质?资本的本质是什么?是信用而非财富,这很重要。

资本的本质是信用,而非财富本身。有财富未必能转化成信用,有了信用才可以创造财富。资本的本质不是财富,是信用。这对于我们来理解资本积累是有意义的。在这个问题上,我个人认为马克思可以讲得更深刻、更系统一些。当然他也讲到了,今天我们把它扩展和完善一下子。资本的本质是信用,资本主义要做的事情就是信用扩张、就是信用扩张。如果你能理解了资本的本质,理解了资本主义的本质,你们就知道资本。

你们就能够理解这个资本主义的本质了,也能够理解资本主义国家的本质了。信用可以是一个人的信用,可以是一个机构的信用,也可以是一个国家的信用。其中一个国家的信用就可以用来发行纸币,发行货币了。那么这个国家的信用积累的过程是一个什么样的过程,貌似貌似是财富积累的过程,但也不完全是。

为什么中国在1995年要联系汇率?要联系汇率是因为我们自己创造的信用不足以在全世界流通或者被接受、被接纳。而我们要进行国际化,那么我们就必须借重他国的信用。信用的这个处理、信用的这个处理是对资本的积累和资本扩张的理解的一个非常重要的一个褃节儿。它并非简单的剩余价值积累,剩余价值积累的是财富而非信用,如不是信用它就不能……

它就不能简单化成资本。所以我们在很多时候,我们认为谁在创造信用呢?在很大意义上是国家在创造信用,而不是资本家在创造信用。为什么我们要搞社会主义呢?因为那是国家资本主义,是国家创造信用强制给社会创造出信用,才有可能进行资本积累,才能进行工业化。你们同意我的看法吗?为什么共产党解放了全中国,发行了人民币就有了信用、有了流转、有了资本运行、有了工业化的基础。今天我们……

今天我们讲的第二个部分,其实在理论上的意义是非常重大的。重复三遍资本的本质是信用、信用、信用,而不是财富。请一定要记住,将来你们做投资的时候要注意看信用,而不是财富。讨论到这儿你们就知道为什么资本会走向邪恶之路,为什么他们不愿意做实体经济,他们更愿意做赌场、做金融投机,甚至他们更愿意食利阶级,为什么他们会用杠杆的方法、用信用的方法进行韭菜的收割,一轮轮的收割,一会儿变成信用,一会儿变成财富;一会儿变成财富,一会儿变成信用。开始理解资本的本质。资本的本质在不同的地域和历史阶段表达为不同……

讨论完资本的本质,资本积累的部分我简单多说两句吧。资本积累的源泉我刚才讲了有殖民、有地租、有剩余价值。所以在理解资本积累的时候,国家治理的层面你必须解释土改和工商业改造的问题。就是我们的社会主义革命和社会主义建设为什么有合理性、合法性,我刚才说了天理、伦理、学理、法理和治理。为什么它是对的,要有一个解释。如何解释普遍的贫困和艰苦奋斗?普遍的贫困和艰苦奋斗使我们将剩余价值集体性的转移了,作为积累。如何解释要素价格的。

那么我们该如何解释所有要素价格的长期低迷?这个要素价格的长期低迷它是一个什么意义?就是我们长期处在一种通货紧缩,几十年都是2分钱一盒火柴。这个状况它是对还是不对?是好还是不好?我们做对了一些什么?做错了一些什么?如何解释农业对工业的这个补贴的问题?在资本积累里边,它有一系列的秘密,有一系列的问题,有一系列需要解释的。类似于最后一个问题,你如果能解释清楚,你就知道如何让资本下乡了,如何让资本完成区域间和产业间的正态分布。如此,使一个国家的经济更为健康、可持续地长期发展。资本积累的秘密非常重要——就是第二个部分非常重要,这一件事情马克思没来得及……。

不是马克思说不清楚,也不是马克思不想说清楚,是马克思没来得及将它说得很清楚。当然我能理解,就是马克思在写完第一卷之后,第二卷和第三卷不是他懒了不想写了,是他遇到了我刚才说的一些问题。就是第一卷上就已经有问题了。他到了晚年的时候是持一个对《资本论》的批判的态度,甚至他反对进行革命,而主张议会斗争了。他对资本和价值论的方法等诸多问题,他开始进行自我的批判和反思。只不过这些东西现在由于种种原因吧,可能它变成了非主流,而主流的东西我们看到了,它存在着,确实存在着历史的局限性。好吧,我们讲第三个部分——土地的资本化问题。

或者这个问题也可以叫做“寻租与主权资本化的问题”,这是一个大问题。这个有美国的朋友将这个发明了一个词叫“经济租”,就是他把这个所有的寻租的这个现象叫成“经济租”。我一直认为用“经济租”这个词不精确,地租是精确的,土地寻租嘛,那么权力寻租也是精确的。这个说到最后呢,实际上是一个关于这个产权的或者是权力的寻租问题,不管你是经济上的主权、房产的主权还是政权的主权。

实际上这涉及到一个非常复杂的概念,就是主权通过寻租来完成资本化的过程。也就是说在讨论到资本积累里边涉及到一个非常严重的问题了,就是不同的阶级、不同的人有不同的个人资本或者机构资本积累的源泉。这个源泉在很多的角度它就是个寻租。其实我们反腐败看到某人、某军,那一盒一盒的“小海鲜”,其实就是个寻租的过程。那么寻租实际上是在共产主义到来之前,我们广泛见识到的一个现象。

在西方较发达的资本主义国家里边,权力的寻租是受到制约和限制的。土地的寻租也有严格的直接税体系进行某种管控;数据寻租呢现在正在完善相关立法。但是寻租问题这里边就有一个我们来理解国家治理的问题了。这里边我刚才说了,就是我们尽量不把一个简单的问题进行非中性的归类,就是进行褒义或者贬义。土地财政,土改、土地财政、地租、超级地租都是特定历史时期出现的特定的现象,它有负面的意义,当然也有正面的意义。

我上个月写了篇长文,后来可能我不知道是因为我的原因,我这个名字可能现在是个麻烦,可能没有办法在某重要刊物上登载了,可能这个大家有压力吧就是。我里边讲述了这三者的关系,就是讲述了地租、讲述了土地财政、讲述了直接税三者的关系。其实是一个辩证关系,貌似它们三个不是直接联系,但是它是一个辩证的关系,它既是历史唯物主义对历史发展的一个必然进程,它也是个辩证唯物主义,就是它们之间的平衡和修正它有个关系。我这整篇文章用了一万五千字,后来做了某些删节吧。

如果不能直接发表,我想将此文送给大家。这里边就讨论了资本化的问题,因为土地的资本化是中国这二十年资本的重要源泉、资本积累的主要的源泉。就是如果没有土地的资本化,那么中国近二十年的资本源泉是不存在的。正是因为有了土地的资本化,导致中国出现了比较快速和猛烈的资本积累,这个资本积累才导致了我们的国家的整体的建设的现代化,甚至导致我们的国防的现代化等一系列的问题。这个土地的资本化也为我们创造了雄厚的国家信用。

这里边我有两件事情要提醒。第一件事情,还是讨论剩余价值。如果我们认为土地资本化它的源泉是剩余价值,那么这个剩余价值它和一般的剩余价值不一样,它是一个剩余价值的期权。因为我们预收了劳动者二三十年的地租,就是它的剩余价值提前二三十年。所谓的供楼款就是这个期权我们就是提前预征了,就是我们这个资本积累的速度加快了。加快了也不完全是一个大问题,只要是这个资本积累回到实体经济也可以,但是这里边相当一部分被操作者截取,截取并转移,向海外转移,这就是个很大很大的问题。

如果这个截取和转移不符合天理、不符合伦理、不符合学理,只是由于我们法理出了漏洞,治理有问题,那么它就是一个国家可能面对的一个陷阱,也可以叫它中等收入陷阱,叫什么名字不重要。另外呢,由于数据租的出现,我刚才说了,阿里的高管四套美国豪宅的问题。数据租出现,数据租也有同类的性质,它也是我们劳动者剩余的期权。那么它们也在被大量转移中,不转移哪有四套豪宅呢?所以它出现了问题。资本的载体原来是土地,现在是Data。

而我们用资本的载体创造的这个资本,或者是创造的信用,这个创造的国家信用,创造的人民的信用,在某种意义上它形成了一种结构性的扭曲,出现了诸多的问题。我们第三个部分用这样一句话来概述可能是有点点问题,就是寻租与主权资本化问题。其实在学理上这句话是不妥当的,但我又不想简单的用土地的资本化问题来概述,或者是又回到地租问题上去,但这是一个大问题。之所以成为大问题,是因为《资本论》关于地租的部分没讲清楚,特别是对主权资本化的问题,《资本论》没讲清楚,这个资本化的过程,

虽然在马克思《资本论》的第二卷和第三卷里边都有涉及到,但确实是留下了巨大的遗憾。这个事情也是应该我们来做的,我们不应该对老马有那么多的要求,因为他不可能解释完历史,还要解释当下的现实,也不可能完成对天理、伦理、学理、法理、治理的全面论述,那真的是要成神仙了。所以留有空缺恰恰是给我们留下了发展的空间或者是我们进步的机会。好吧,

我们来看我们今天讲的第四个问题。第四个问题实际上核心地涉及到这个《资本论》的第三卷了,就是资本流转的三大问题。其实马克思《资本论》在资本流转的时候,(怎么说呢)他侧重点是讲的是阶层流动。这个阶层流动,我个人认为是资本流转三个问题里边的一个问题,少了两个问题,一个是关于资本流转的区域性流动问题或者叫国家间流动的问题,这个没有讨论透。资本为什么会在国家间流动,如何流动?

例如鸦片战争,英国人从中国用枪炮加鸦片换取了白银;又例如甲午战争,中日签署的《马关条约》,为什么又是用英镑来支付呢?资本在国家间的流动,它有什么样的逻辑、规律和可以借鉴或者思考的内容呢?除了上个世纪野蛮的用枪炮和战争甚至用鸦片来处理国家间资本流动之外,当代,国家间资本流动是什么样的一个……

当代国家间资本流动,虽然没有使用枪炮、没有使用鸦片,但它符合天理、伦理、学理吗?它有法理逻辑吗?它难道不是在治理上出现了严重的问题吗?所谓的韭菜的收割、区域间流动——资本流转的区域间流动是我们必须、立刻、马上搞清楚的一个问题。到底资本为什么由西方流往中国,是供需的问题吗?是价值投资的问题吗?还是有其他的问题呢?

难道大清帝国累积中国两千年的财富用短短的时间,五十年时间全部流往西方,成全了欧洲、成全了北美洲的工业化、现代化改造,这个流动真的是合天理的吗?合伦理的吗?合学理的吗?合法理的吗?今日如果资本从西方流入中国,那它的天理逻辑、这个伦理逻辑、学理逻辑又在何处?我们如何制定相应的法理和治理呢?

简言之,如果美国经济出了严重的问题,如果真的美元出现严重的问题,全球资本将会迅速地、重新地、迅速地流转和再分配或者是再分布,我们现在能预见那可能出现的状况和它存在的逻辑吗?如果不能预见的话,那我们读《资本论》第三卷资本流转那不是又白读了吗?所以我觉得资本流转的三大问题里边,第一大问题就是区域间流转,主要是研究国家间流转。当然在中国,因为大嘛,区域间流转也很厉害,就是东北流往广东。为什么呢?

第二个部分也非常重要,这马克思关注的比较多的,就是阶层流转或者是叫阶级流转。阶层流转的原因就比较简单了,其实《资本论》也讲清楚了,这个我呢也想多说两句。就是阶层之间流转,在万恶的旧社会表达为土地兼并;在万恶的新社会它表达为资本垄断。土地兼并和资本垄断其实是一个事情。资本垄断了什么?垄断了价格;垄断了什么的价格?垄断了货币的价格。所以它控制了资产的价格,所以劳动者、劳动者剩余最后又统统回到了资本家手上,阶层流动。革命、

革命和造反使资本向下流动,财富向下流动。一旦革命和造反结束、造反和革命结束,那么资本又从下向上流动。就如同今日之美国又迅速地向少数人流动,99\%的人的财富向1\%流动,又严重的贫富分化。这个事情当然也不光止于美国了,到处都是如此,我们也大体如此。关于阶层流动的天理、伦理、学理、法理、治理逻辑,我写的那个一万五千字的长文就写这个的,为什么要上直接税?就是要遏止阶层流动、资本的阶层流动。

那上面把道理也讲了,到时候我给大家,大家看就是了。所以今天这个阶层流动就不说那么多,说一下产业流动吧。记着,资本流转的三个大问题:一个是区域流动,一个是阶层流动,一个是产业流动。产业流动是一个非常非常大的问题。我们注意到,西方发达资本主义国家在上个世纪八十年代之后出现了分化。产业流动做得好的,比如说美国,经济仍然维持高速增长,做得不好的,经济出现了问题。其中,有近一半的老牌的资本主义国家开始掉队,经济出现了严重的停滞——南欧、甚至包括日本。

为什么会出现产业流动的不均衡的问题呢?这里边非常复杂、非常复杂。有没有天理的逻辑呢?有的;有没有伦理的逻辑呢?有的;有没有学理的逻辑?当然有了,学理逻辑非常之重的。但是法理逻辑和治理逻辑也极为重要,甚至更为重要。就是美国在上个世纪八十年代开始,将军用的通讯技术转入民用,将一部分的生物科技转向民用,它极大地刺激了美国的整体的科技进步和科技产业的发展。而在这个浪潮里边,好多国家仍然固守于旧的传统产业。

最为经典的案例还不是南欧和日本,而是前苏联和东欧。前苏联和东欧国家的整体衰落不仅仅是由于国家治理的问题,而且也存在着严重的资本流转里边的在产业间流动的结构性的问题。就是他们由于产业不能升级而使得大部分的产业流失掉了。这件事情非常重要。这件事情重要到什么程度呢?重要到就是我们今天必须仔细地来思考,其实这三个问题都是我们要思考的。这事也不能指望马克思那时候就把资本流转的事情全说清楚。区域间流动,我们要不要思考?

阶层流动,要不要思考?产业流动,要不要思考?我们现在终于开始考虑全产业链了。考虑产业升级其实是三句话:第一,我们要拥有全产业链。为什么要拥有全产业链?因为世界在分裂。世界在分裂的时候,我们拥有全产业链才可以有独立的生态。我们有独立的生态、有全产业链,我们才能不被人敲诈、讹诈。第一位的是全产业链,我们产业流动要考虑这个问题哟,我们不能腾笼换鸟啊。第二个问题,产业升级的问题。你毕竟进入信息时代了,你能不用信息技术改造传统产业吗?

信息产业和信息技术改造传统产业,我们所谈的“两化”问题就变得极为的迫切、而且严重。如果这个事情处理得不好,你这个产业流动也是有问题的。第三个大的问题是,我们既然知道资本的载体由土地转向了数据Data,那么资本的载体在变,创造资本,(我刚才说了,资本的本质是信用,)创造信用和这个信用的分配就变成了当下非常现实的一个法理问题和治理问题了。当然这个法理要合天理、合伦理、合学理,但我们现在讨论的学理问题其实都讨论得极其困难,因为、

因为在很大程度上,我们都把马云、马化腾他们当成了个人努力的榜样、天才。事实上是吗?不是的。事实上是:这个历史、这个世界没变,只不过是收地租的,现在在收数据租而已;以前是土地兼并,现在是数据兼并而已;就这么简单。马克思其实把基本的东西教给我们了,我们要灵活运用。所以在讨论资本流转的三大问题的时候,希望大家能记住。时间过得太快,我也不能展开讲。其实今天五个问题,每一个问题都可以写一部书。好,涉及到最后一个问题,这个问题我已经说了无数遍了,就是国家与资本的关系。这是马克思没写的东西,他可能想了很久了,没写。

也是我一直认为《资本论》的第四卷,《资本论》的第四卷就应该是写国家与资本的关系。列宁写的《国家与革命》那部书写的是错的、不对,所以导致了苏维埃前期非常牛、高速发展。因为国家资本主义嘛,便于资本原始积累,将剩余价值平均化、整体化全部转入资本积累,所以苏维埃发展是快的、好的。但问题在于,我说了,资本的本质是信用,它最终要创造信用,创造这个时候,就是他将必要的主权资本化的过程中,他做不下去了,跟朝鲜、跟古巴的情况是类似的。最后,他们自己完结了自己。

我一直说,不要动不动就说赫鲁晓夫,动不动就说戈尔巴乔夫,他们有责任在一个特定历史时期改写历史,但是他们得有那个水平啊。所以,我一直说邓小平的存在是非常了不起的,当然邓小平不是孤案的、不是孤立的。在邓小平之前,毛泽东已经开始对国家资本主义提出疑问了,而解决这个问题的也不是小平一个人,其实刘少奇、陈云、周恩来他们都有一些贡献的。结论非常简单:国家资本主义不行,必须引入社会资本与国际资本形成混合资本主义,或者是德国人管它叫社会市场经济。邓小平做成了,中国活下来了。但这件事没完,国家与资本的关系到底是一个什么样的关系?一个什么样的结构?这个问题,《资本论》,不是《资本论》,《资本论》没有……

马克思没把它写出来,《资本论》也没有系统地回答这个问题。那么今天呢必须由我们来完成这部书。这部书可能有一部分会写进《新社会主义论》,另外一部分会写进《广义财政论》。但我自己对我自己的能力是很清楚的,就是我做不了像马克思这样的工作,做不了,这太了不起了!我发自内心地钦佩马克思、凯恩斯这种人,就是我们努力、十分的努力,但也只能做一小部分。也许呢,在平台上可能正在成长的年轻人里边有典型的高手,将来可以成为真正的大师,可以把这事儿做成了。

在讲国家与资本的关系的时候,我们要讲一个前提条件,前提就是国家在资本积累与资本流转中的角色。我们一直认为资本的积累和资本的流转是由资本家和市场自发形成的,我们一直这样认为。但我经过多年的观察和研究,包括我在香港工作这二十几年,没有的。资本积累与资本流转,无论是在任何一个时期,国家都是一个非常重要的角色,甚至是主要的角色。我不大同意某些阴谋论者所谈论的事情。

例如,有些人认为,资本的流转是由一些重要的犹太人,类似于共济会来控制的、来操纵的;我不完全认同,否则无法解释苏维埃,否则无法解释中华人民共和国。有没有操纵和控制?可能在个别的地方、个别的国家和个别的地区,有。而全球是不是有足够的能力控制呢?我想说的是,我们还活着,我们在,我们仍然在思考,这一件事情就不能那样。不然的话,我们为什么要写《国家与资本的关系》呢?对吧。

这里边有一个重要的事情,今天来不及细讲了。就是因为国家是创造信用的主体,国家发货币;国家是创造信用的主体,国家发行货币与创造信用形成了国家在资本积累中必然是主角、主要角色,而不是配角。如果这一点认识不清楚,被奥地利学派带偏了,带到沟里,必然是国家治理的失败。这里边呢,我们要决定进行历史辩证和唯物辩证。什么叫历史辩证呢?历史的看,无论是中国的古代,无论管她叫封建王朝,

还是近代资本主义和当代,其实国家都在资本积累的过程中,资本积累和国家资本流转中起到主要角色。请记住我的第一句话,这是历史的辩证;第二句话,这一种国家在资本积累与资本流转中的主要角色不可以越界。不能走列宁的国家资本主义道路,不可以高度集权垄断乃至于专制,形成了资本的效率递减、资本积累的效率递减、资本流转的效率的递减,以至于最后国家治理的失败。我说什么呢?我说的是两个层面的含义,我再重复一遍吧:

第一,国家必须是主角,资本积累和资本流转的主角。尤其是国家必须强力的管控资本积累的不合理的部分,比如说超级地租,比如说数据租。必须严厉的管控资本积累不合理的部分;必须严厉管控资本的非常态流转向海外转移。因为你是主角,这不是市场能决定的;如果你不相信你是主角,你要交给市场,你就是交给洋人主权。前苏联领导人就是个十八岁的少女,他上当了嘛;我国改革开放初期的领导人还是十八岁少女,差一点被骗了;我们现在不要再上当,不要再被骗了。

所以两句话,第一句话很重啊:你是主角,不要放弃责任,要管理积累啊,不能让他们在积累过程中占了大头啊,1\%占99\%,这合适吗?第二,你要管流转,不能动不动就跨境流动啊、就阶层流动啊、就产业流动啊,这不行的嘛。所以主角的含义是非常重要的。其次第二条,主角也不能太离谱啊,要允许社会资本和国际资本进入、融合、活化并形成历史性的大均衡。如果历史性的大均衡不能形成的话,要你主角做什么?其实这个辩证的关系非常复杂,能把这件事情论述清楚,其实国家的法律和……

其实国家的法律和治理也就说清楚了。国家与资本的关系里边需要处理五个大的问题,我念一下就算了,今天来不及,讲不了那么细了。第一个,国家如何处理资本积累;第二个,国家如何处理劳动所得与资本利得的关系;第三个,国家如何管理资本的跨境流动;第四个,国家如何处理劳动所得中的社会保障问题;第五个,社会保障与资本利得均衡如何得以实现。其实我知道,就是每一个题目都是一本书,都可以作为一篇博士论文的。先把这事情解释清楚了,从天理、伦理、学理、法理到治理讲清楚了,就了不起。

甚至,我个人认为、我个人认为,所有参与我国国家治理的公务员们都应该好好的去认真地读一遍《资本论》,把《资本论》读懂。当然了《资本论》上可能缺一些东西,那他就应该上我们的二十四讲,把这事搞清楚。如果这些事情都没想清楚,很容易走极端、走偏,很容易出严重的状况。实际上我们看到,中国之所以没走那么偏,其实是我们这个国家、我们这个民族读书人还是很厉害的。中国人是懂天理的,不要小看那本《易经》啊;中国人是懂伦理的,不要小看《论语》啊;也不要小看《道德经》,天理、伦理中间那部书;也不要小看学理啊,我们还有《盐铁论》呢。

好吧,今天累了,这个主课就讲这么多,第二十三讲,但我把条条框框全都说清楚了。这个你们可以在这个条条框框里边寻找自己的读书的点,寻找自己写作的点。如果有志青年,应该在这里边做发挥,我们这里边有好多人在做乡建,我一直在讨论:就是资本的区域流动问题、资本的阶层流动问题、资本的产业流动问题,就是三大流转,那么你们要如果把这个流转的原理搞清楚了,那么乡建其实就有了根基了。如何让资本下乡呢,它就有了逻辑过程了,对吧?

讲到这个地方,其实大家也知道,大体上知道“新马会”建立的意义了。就是为什么要有一个新马克思主义研究会?为什么我们一个人完不成,要一大堆人来一起做这件事情;而且这个“会”将来它会有重要的理论成果;而且这个“会”将来会形成一种确定,某种情况确定天理、伦理、学理、法理和治理的一个系统性的结构;甚至由此而延伸出一系列的生机勃勃的一些机构和事物和一些人物。我们多么期盼呐,多么期盼。

好,留点时间讲几句“乌东”问题,今天早上写完了,就给丢了,我也不知道为什么就丢。就是写完了,我准备向新浪的这个微博上转的时候,我这个华为手机上的备忘录不见了,然后找不到它,在控制里边看到它了,但是打不开了,转不了,打不开,所以也不知道发生什么情况。好吧,转不了、打不开,那意思就是让我把这个内容,今天在平台上跟大家汇报一下。那我就今天借平台上把这文章念叨几句,大概、梗概给大家听一下。我知道最近好多事情,都比较敏感,其中“乌东”的问题尤为敏感。

我们平台上的朋友,有些朋友喜欢听课、听《资本论》;有些朋友喜欢听聊天,讲讲心学、讲讲历史,没那么大压力,偶尔聊聊投资也行。谈“乌东”问题要回到历史,要回到二战结束时候的“雅尔塔体系”。你们将来有空去德国旅行的时候(它离柏林很近),要去一下那个小地方,很漂亮的一个小城镇。这个“雅尔塔体系”,实际上是、本质上是罗斯福和斯大林(可能设计者是马歇尔),罗斯福和斯大林重新规范这个世界的一套体系。这套体系被一个“笨蛋”叫里根的给破了,破解了。

里根破解的“雅尔塔体系”主要是西段,就是欧洲的部分,东段并没有完全破解。今天北朝鲜还是分裂着,日本和韩国依旧与中国和北朝鲜对立着,至少形式上这样,台湾仍然没有解放。就是东方的“雅尔塔体系”还在,虽然很快就会粉碎了,但他今天至少还在。但西部的“雅尔塔体系”破碎了,破碎了之后,按照当时英国人和美国人的想法,就是变成单极世界,就是俄罗斯就不存在了,苏联就不存在了,变成单极世界。想法很美好,现实很骨感。

所以在粉碎雅尔塔体系之后,我说了崛起了两个大家伙,比雅尔塔体系里边的两极世界还恐怖,一个大家伙就是中国,还有一个大家伙就是欧元区。就是我说的、我也常说的,就是德国人用马克代替坦克统一了欧洲,实现了他们德意志民族的伟大理想,但这件事情其实没彻底完成,要彻底完成这不就没事了吗?!今天的乌东问题不是俄罗斯在跟美国打架,大家又理解错了,不是,也不太像是美国人跟俄罗斯人达成了世界共治的这样一个协议。

当年罗斯福和斯大林是有条件对世界进行共治的,好,咱们成立两个阵营,你带一堆小弟,我带一堆小弟,我们搞点竞争,把世界分成两块,大家各玩各的,互相竞争、互相制约。这个好处是因为分成两个阵营,所以好治理。小弟就是小弟,没资格上桌吃饭,真正的盟友是苏联和美国两个大哥之间,这才是盟友,其他都是小弟,是在门口拎着棍子做打手的嘛,后来就都上了桌了。就是里根不明白,他听撒切尔夫人的,这坑死他,后来小弟们就都上桌了。上了桌以后,这个情况它就不是两极,也不是单极了。

大家很难理解拜登内心深处的东西,但我想,可能我这个一个是在境外,一个是学财政的,我看得很透彻,美国不行了。美国倒下之前替代他的那个人不是中国,是欧盟,主要是德法欧元区。因为美元出问题,欧元一统天下,可不要忘了奥运会开幕的时候,普京到北京,跟习主席签了一大摞的协议,里边结算货币是欧元。请想想拜登的内心深处的那种的恐惧,打掉欧元,只能依靠战争,其他手段来不及了。

上一次动了科索沃,克林顿干的,但是克林顿当时的想法是打残,没有下狠手打死,打残,从一块二打成八毛钱就结束了,南斯拉夫解体。这一次让乌克兰解体,打残,他其实是想打死,帮手是俄罗斯。但这回俄罗斯有大炮仗,但不是大笨蛋,不是大傻子,台湾那边大傻子,没大炮仗,普京会和美国严丝合缝的配合进行一场乌克兰战争,以至于将欧洲打得稀碎,欧元再一次重返八毛钱、七毛钱?

我个人认为俄罗斯没那么笨。虽然德法并不聪明,这两个小哥哥并不聪明,但普京没那么笨,乌克兰一定会粉碎,乌克兰必须粉碎。但这个结局既不是恢复到雅尔塔体制,因为中国的出现不可能再有雅尔塔了,连三极都不可能。那么其实真正的弃子不仅仅是乌克兰,也包括俄罗斯。那么普京看不到这一层吗?当然很清楚。那么普京不要那两个共和国吗?顿涅茨克和卢甘斯克,不要吗?当然要要。所以会是一场,

会是一场神奇的、巧妙的、有克制的战争,然后乌克兰粉碎,俄罗斯拿到他需要拿的东西,美国人可能部分拿到他需要拿的东西,最受伤的应该是德法。德法、欧洲以至于欧元会受到一次重创。这是一般意义的观察,但也不排除、也不排除德法有自己的小心思。虽然我看不出肖尔茨跟马克龙哪一个有大政治家的这个样子,实在不像大政治家,有可能他们跟普京达成某种共识,并且能够操纵乌克兰在这个地方完成一次分赃。

美国人处理区域性问题用的就是两手:一手我管它叫族群撕裂,他在我们新疆玩的就是这个,族群撕裂,挑拨离间不同的族群之间的矛盾让你打起来。你没有弄我就给你弄,我派NGO弄、派特种部队弄、派汉奸,派什么都行,反正我就给你弄族群撕裂;第二个是阶级对立、阶层对立,给你搞花儿革命。他就是两手,一个叫族群撕裂,一个叫阶层对立,在乌克兰这两手同时用,所以乌克兰几乎是万劫不复。在俄罗斯的周边用了一下子哈萨克斯坦不好使,在白俄罗斯也用了不好使,在俄罗斯现在还在用,未必有用,现在麻烦的是哪儿呢?

麻烦的是美国自己出现了族群撕裂和阶级对立,就是他到处挑这件事情,他自己开始有了,这个病反噬回去了,所以他的时间并不多。普京算是一个老道的政治家,我觉得他这一回能处理好这件事情,应该问题不大。德法我现在还看不出来,因为德法的,特别是马克龙去莫斯科,普京把那个桌子摇那么老远等他见面,普京表示了对马克龙的一个深切的不屑,不信任、不屑,马克龙也不像个样子,法国该做他该做的事情时候他做不出来,当然了肖尔茨不但远远比不上默克尔,甚至他离……

甚至他跟科尔、跟德国所有的伟大政治家比,怎么说好呢,不能说连一根脚趾头都不如,但至少确实是不像个样子。他在白宫翻了白眼儿,翻白眼有用吗?如果翻白眼有用的话,那还要政治家做什么呢?所以我觉得欧洲不管怎样,最后都要承载一次巨大的风险,甚至是灾难。留下来说两句我们自己,我们在处理乌东问题上是处理得非常好的,所以中国和俄国的联合声明里边,我们上面一个乌克兰的字样都没有,反而是俄国人在提台湾问题,我们没有谈乌克兰一个字,我们点到为止,

预留了所有的空间。当然,现在美国人非常急,就是美、英、澳、新、加非常急,希望将中国拖下水,最好拖到乌东战场上去,拖下水。但我们对这个事情,第一没什么兴趣,没有那么大的兴趣,而且这件事情本身对中国而言也不是完全的负面,它有它的正面意义,当然也不完全是正面意义,它也有负面的意义,但总之它对我们是一个间接的问题,所以中国可以采取一种观望的态度,可以采取一种和事佬的态度,可以采取……

可以在必要的时候成为调节事态的第三方,就是成为西方、俄罗斯之外的第三方——东方。我甚至认为乌东问题给中国政治家介入国际重大事务提供了一个重要的舞台,甚至是个支点,对我们在这个意义上来讲是好事。此外,无论是美元目前遇到的风险,也无论是欧元遇到的压力,对人民币都是极为正面的意义,所以人民币一直在升值,打不打仗人民币都升值,我们自己不想让它升值,我们在控制它,但是它都要升值。

美国政治家,特别是美国的议员,他们出现了一个严重的倾向,就是用立法代替司法和行政,尤其是在国际事务中,立法代替司法和行政这是非常糟糕的事情,非常残酷。比如说立法会议员竟然可以提出对他国元首乃至于他国的立法者——人大代表提出制裁的议案,这在天理、伦理、学理、法理上均不成立,这会搞得这个国家极为被动、极为被动。我们、我们……

我们需要抓住他们在国家治理过程中出现的严重问题,在合适的时候,予以非军事的、非经济的严厉的打击。我们自己必须完成我国国内的相关立法,对他们进行严厉的、出于司法的、出于严格法律法条的、符合国际法的严厉的打击。但无论如何,我们要避免出现在中国周边出现某种“乌东粉碎”的这样的事情和局面,我们不被卷入任何的局部的军事冲突,这是我们必须确保在2032年之前不发生的事情。

我甚至一直在说,好多人说:“啊!一定要解放台湾。”就是解放台湾,好多人就认为这个事情急迫、认真、很重要,行了,细节我不说。我今天再说一遍,我们要解放的不仅仅是台湾,我们要解放的是日本、韩国,我们是要解决东方的雅尔塔协议,我们必须彻底地粉碎东方的、强加给中国的那个旧的雅尔塔协议,抢走我们中国差不多将近三百平方公里的那个雅尔塔协议,我们根本不接受。另外关于中国周边的势力范围的划分,我们不接受,而不是台湾问题。我们把这些问题都解决完了,台湾问题就彻底解决了,不是解决一百年,是解决一千年,永久性的解决。好不好?

市场的事情我就不说了,因为大家都知道这个意思,耐心吧,今天就讲这么多。香港疫情如火,我尽量地小心谨慎,别出事儿,另外按照有关方面的要求,做好我们的责任、做好我的责任、做好公司的责任,尽可能地度过这段艰难岁月。好,今天就这么多,明天下午三点钟见。再见。

\subsection{乌克兰战争与恶性通货膨胀}

大家好,今天是2022年的2月26号,是壬寅年正月二十六日。今天是聊天,可是今天聊天儿也不轻松,今天我们聊乌克兰战争与恶性通货膨胀。这两个话题都比较沉重,但这两件事情都非常之重要。一会儿三点钟我们准时开始。这两天香港的疫情开始进入到极限状况了,就是昨天单日确诊人数上万了,估计下周可能会有一次爆炸性增长。谢谢大家、朋友们对我的关心。好,一会儿见。

大家好,今天是2022年2月26号,壬寅年正月二十六日。这个正月过得不平静啊,风起云涌。香港这两天的疫情真的是厉害,昨天确诊人数过万。原本想聊几句,后来也不能聊了;后来想写文章谈谈香港的疫情,后来又不能写了。现在这个事儿是不能说的事,其实不能说就不说吧。我们今天聊乌克兰的战争与恶性通胀,这两件事从中你可以看到香港的一些事情和它的关联。因为这个有的时候……

香港这边到了周五的时候,它的港币再已经上到,因为港币一般是在7.75到7.85之间徘徊,它已经远远越过了7.8的这个水平,也就是说有大量资金离港。大家也看到恒生指数的大跌,与之相对应是美股道琼斯涨了八百点。语言很苍白、身体很诚实,我们看到美国在这场战争中成为一个最大的赢家。今天不讨论这个事。

我们回到今天的乌克兰战争的这个事情上来。我在十多年前吧,我开始写这个,我算算时间,可能都不止十年了,快二十年了。我为了写这个《苏联解体十年祭》 ,我去拜访了一些朋友,其中有一个朋友现在八十多岁了。那个时候好像我印象里他应该还没全退,刚刚退休。他是我国在俄罗斯,后来是、最后他是离开乌克兰,是一个非常优秀的外交官,他是第一代领导的一个孩子。

其实讨论到当时他给我介绍乌克兰的时候,我已经很震惊了。他在做外交官的时候,乌克兰大概是两千多万人口,一个小小的国家竟然有七百多个机场,超过了我国改革开放,就是1978年之前的全国机场总和。生活水平极高,就是家家都有小汽车,有自己的房子,教育水准极高。用这个朋友的话,我不能说这个朋友的名字,因为好多人可能认得他,他说乌克兰太像我国的东北了,拥有黑土地,是农业生产的最好的地方。

同时,它几乎聚集了所有的苏联的最高端的工业,不光是重工业,包括了几乎全部门类的工业,特别是在航天、航空和造船领域极为发达。因为我们讨论、聊天的时候,那个时候乌克兰已经开始出问题,所以他感到非常的惋惜。我们在讨论苏联解体之后乌克兰的状况的时候,他说:“乌克兰的情况确实也很像改革开放之后的东北,没有伟大的思想家了,当然也更没有伟大的政治家了。如果你能熟悉什么……”他说:“那个走在舞台上的,可能有点像中国的赵本山这样的人吧。”

后来我因为写苏联解体嘛,讨论的都是一些历史问题,当时就没有关注乌克兰的问题。最近乌克兰出事儿,我们又开始调一些关于乌克兰的资料,我就想起了这位先生,他在境外的时候他是用的是一个假的姓和假的名字,但他挺有名的。我们今天不说这件事情,将来有时间聊天的时候,再聊一些他经历的细节,因为一个民族或者是一个地方它有它的特色。我今天想说一下子乌克兰的基本的数字,因为我们是搞财政的出身的,对数字敏感一些,有的时候数字比语言更能说明问题。

乌克兰的GDP大概是在1500亿左右,1500亿的GDP,他的财政支出大概在500亿左右,军费支出大概在100亿左右。有两件事情请大家注意,就是他的军费支出,因为他的军队不是很正规,他的军队有点杂。另外他这个军费因为乌克兰的腐败不光是政府腐败,军队更腐败,所以他这个100亿左右的军费里边可能用于装备、训练或者钱可能连一半都不到,就是50亿美元左右。而他的士兵的工资大概是,折合成美元大概每个月应该是200美元吧。

如果你看到这些数字,你应该知道它这个国家虽然在军力排名排世界第22位,台湾是第21位,但实际上那可就差远了。普京敢于动手的原因也是其实乌克兰这个就算是不打它,它也基本丧失了一个国家应有的国防能力。虽然他有20万的军队,但你知道,100亿军费还有一半可能被人贪掉了,拿50亿的经费养一个20万的军队,可能吃饭都困难,所以他才会有了亚速营,就是乌克兰军队里边的一个纳粹组织。

在很多时候,乌克兰的军队,其中的相当一部分军队沦落为地方武装或者私人武装,它就很不像个军队的样子,它就是……我相信现在的乌克兰总统可能很难指挥、调动海空军以外的陆军的部队,他能掌握的部队可能是非常有限的,而且那个部队的战斗力其实可想而知。但是他有些军队,就像亚速营,因为它是由22个国家的各色人等汇聚起来的一支很特殊的部队,又有人专门出资供养他们,所以它可能有着极为强悍的战斗力,可能目前俄军碰到的可能硬茬就这支部队。

普京谈了两个“去”:一个是去乌克兰的武装力量,一个是去纳粹化——去武装化、去纳粹化这两“去”。在整个战争爆发之后,我和朋友们——香港的朋友们和北京的朋友们,我们聊过几次。至于普京提出来的北约东扩的问题,我觉得那个不是普京的真实目的。因为其实北约扩不到乌克兰,因为北约不会接受乌克兰,所以北约东扩这是一个议题或者是一个借口。这两“去”呢是真实的要做的事情,但也不是最终目的。

去武装化,所有人都能理解,就是去除武装实际上就是解除了政权,因为这个枪杆子里边出政权嘛,解除了武装实际上是政权那就不再是由西方或者是西方集团或者是由特殊的西方势力所操纵。相当于部分的收回乌克兰的主权,也未必就是俄罗斯人收回乌克兰人的政治主权,可能也算是俄罗斯人替乌克兰人收回他们早已丧失的政治主权,“去武装化”的含义是这个意思。去纳粹化这一件事情呢,我想多聊几句,因为我知道多数的国内朋友可能听不明白。

这两天乌克兰的事情让我想起了香港的往事。1975年到2000年,在这个25年时间,香港曾经接纳并且处理了超过20万的越南的难民和船民。什么叫难民?什么叫船民?难民就是具有政治倾向,就是他是反共的、反华的、这个亲美的,后来没办法要离开……,因为西贡解放嘛就是南越被解放,他们政治避难。这个香港是第一收容所,所以收容了很多的政治难民。具体数字和最后的情况其实不详,因为这个不能对外公布。船民就是不纳入政治难民的……

不纳入政治难民的普通的逃亡者,其实里边有好多是属于……大部分都是华人,就是华侨、商人。当然还夹杂了一些其他的南越的所谓的贵族,或者是南越政府的高层或者是军队军官、一些……反正是各色人等,就是害怕越共的人逃到香港,一共是20多万。其中一部分被遣返——甄别和遣返,一部分就送往了其他国家,还有相当部分就地消化了。你们知道的黄之锋,就是这回2019黑衣的,

他就是难民的子弟——越南难民的子弟。我为什么想起难民的事情,因为香港是个大熔炉,它各种人都有。浮在水面上的,类似于像潮州帮、福建帮,潮州商会、福建商会,他们是浮在水面上的,而且他们那个头头脑脑获得了很高的地位,比如说人大、政协等等等等。但还有在水面下的,比如说这个难民们形成的,其实势力也非常庞大。因为有时候可能很难理解,难民很喜欢生孩子的,全世界的难民都是拼命生孩子的,所以他们也形成了一种很强大的势力和能力。

因为你在现场,你才能体会到,就是你突然发现黑衣暴徒如此之凶残,哪儿冒出来的?好好的文明人怎么会是这样一个状况?有时候你很难理解。如果你深入到最底层,了解了他们的社会结构,了解了他们的生活状况,他们确实是在社会上面,除了极个别人,其中有一个是我认识的,也是我的朋友,成为富豪,大部分的人是在最底层,很辛苦。而且他们生了那么多的孩子,至今仍是在底层挣扎着,所以他们又很团结,他们又有一定的结构、组织什么的,所以会形成一种很可怕的东西。

我不是想说香港,我是想说我的体会。因为在列宁十月革命之后,白军就是原来的沙皇军队分成了两部分。一部分在东线,东线的白军大部被消灭,剩余的几十万人逃进了中国的东北。我读书的时候在大连还经常看到一些非常漂亮的大鼻子女孩,就是算是俄裔的,具有俄裔血统的女孩,在大连、旅大都很多,哈尔滨也很多,这都大概到第五代了。这个不重要,重要的是在西面战线的白军。西面战线的白军呢,

起初退入了乌克兰,后来到1921年苏联红军就是斯大林倒过那口气来以后,就是一路打过去,他们可能一部分跑进波兰,但是波兰人不是很喜欢俄国人和乌克兰人,所以他们相当一部分人最后逃到了德国,逃到了欧洲其他的地方。其中去德国的比较多,大概的数量应该是百万级别吧。这些人心里边……难民嘛,心里边怀着非常深刻的仇恨,不光是对苏联的仇恨,对共产主义的仇恨,怀着一种愤怒。

其中有不少人其实在原来的沙皇俄国是贵族或者是有地位的人,失去了一切。当年来香港难民所的流离的部分里边也有好多的,实际上不一定完全是华侨了,也有就是西贡的原来的上流社会的一些人吧。他们充满了仇恨,这种仇恨不会在一代人身上消失。流散在德国和东欧的这部分的人,随着苏联的日渐强大,他们就在那些国家住了下来。

其中在德国的部分,有一种说不清楚的东西,就是纳粹这两个字是国家与社会主义的两个词汇抽出来其中的一个部分,组合成纳粹这个词,它是国家社会主义,其中相当部分的人最后成为纳粹。在后来苏联解体之后分出来乌克兰,流散在外边的这些人,已经第几代了,他们开始陆续返回,并且用他们的力量来控制了类似于像乌克兰这样的……

控制了乌克兰的教育、学术、传媒、政治、经济,控制了。他们还有了自己的武装,那个武装就叫“亚速营”,“亚速营”虽然叫一个营,它的编制肯定是在团以上规模,而且战斗力极强。乌东在此次俄罗斯进入之前死了1.4万人,大部分的人都是被这个“亚速营”干掉的,如果说他搞种族灭绝,搞这些事情,西方媒体睁一只眼闭一只眼,你也不可能看到有人详细地对这些事情做出一个报道,但这是真实的存在,这是真实的存在。历史,历久弥新。

北京的朋友说,知道我要聊天,说你要在平台上聊一聊还可以,可千万不要写文章,也不要在外边说这些事情,因为会引起不适的、会引起不适的。因为其实我挺关注难民问题的,我一直在看,因为这是个社会上的一个重要的结构。好在我国在这个问题上压力不大,我们没有比较严重的难民问题。当年香港那么点人口有20万难民,而且有难民营或者叫集中营,还产生过若干次暴动,还死了不少人。

我其实在某种意义上能理解乌克兰的现状。当我知道,当我知道普京要做事情的时候,我大约知道他的想法,另外他说出了这两个“去”,我知道他那种历史感,那种沉重的历史感,就是他说去武装化、去纳粹化。但普京其实没有说清楚,这是俄国人的一个特点,他就是不太愿意说清楚、或者是不太容易说清楚,因为这里边涉及到太漫长的历史的恩恩怨怨了。就像香港,你现在也不能说这件事,是你说这件事可能会引起极度不适。

但乌克兰这件事情,它具有经典特征,因为,一场伟大的社会主义革命是有遗留问题的,是有遗留问题的。中国在改革开放之后都有伤痕文学,你可想苏联解体之后那伤痕、那反扑、那算账,那就非常严重。因为在俄罗斯境内这个事情,它是有一个至少是在2000年之后有普京在,压得住。但像乌克兰这样的地方就不但压不住,而且可能被这种势力反噬了。所以我们看到的一些现象不要轻易……

在这个地方,我想加一段东西,我想再一次提一下“三断”:断是非、断大小、断远近。一个国家有没有思想家?如果有的话,这思想家就是要来完成“三断”。什么叫断是非啊?就是个路线选择的问题;什么叫断大小啊?就是进程,主要矛盾是什么?主要的是事情是什么?你要找着啊,你要理解事物的进程,就算是要去到这儿,也要分阶段慢慢走嘛;第三是断远近,跟谁关系好跟谁关系不好,实现一种关系的均衡或者平衡,使本国利益最大化。这“三断”全错,基本上是亡国的……

非常不好彩,乌克兰“三断”全错了。在乌克兰选择路线的时候,其实早在上个世纪80年代就开始讨论改革的问题了,就是苏联就在讨论改革,其中,乌克兰这个地方是最热闹的。走西方的路、还是走社会主义道路,在乌克兰应该是没有异议,要走西方的路。路线这个选择是一个大问题、大问题,但很遗憾,乌克兰原来治理乌克兰的思想家和政治家主要是俄罗斯人,解体之后,他们带着78万人的装备和一些大量的人就返回了俄罗斯,而留下来的人……

所以他们在处理的时候,平丘克(克拉夫丘克)这样的政治家就会犯浑。自弃武装,放下核武器,开启了门阀政治和财阀经济。你知道,这个门阀政治和财阀经济是德国人在一次世界大战之前玩的东西,就是容克地主模式。乌克兰在苏联解体之后,立刻,没有进入到社会主义,告别了社会主义,也没有进入到资本主义或者是德国的社会市场经济,直接就回归“容克地主”了,他就完成了向门阀政治和财阀经济的转换。至于在对外关系上,

至于在对外关系上简直是近乎弱智。他是脱俄——脱离俄国,去欧——他也不喜欢欧洲,而单独亲美。这是一个非常奇怪的现象,一个没有伟大思想家、没有伟大政治家的民族是非常可怜的,这要引起我国高度警觉。就我以前写文章批那个赛车手作家,批那个什么什么什么三人行,批那些东西,好多人不知道那里边的含义。其实这里边藏着的东西,决定的就是你这个断远近的问题。错得离谱啊!

做了,路线上错了。选哪一个都对:社会主义也行,资本主义也行,他选了封建主义,这什么情况?至于进程,他已经早早就完成工业化了,只是需要完成工业化升级。在欧元区成立之后,以德国为金字塔顶部的工业体系里边有乌克兰的位置,但乌克兰放弃了。到今天我不知道乌克兰人在想什么,我不知道他们为什么那么反感德国,不懂,就是乌克兰是脱俄、去欧,其实主要是去德国,就是离开德国,亲美。非常糟糕的一个选择。

在这里边,其实作为一个中等国家或者是中小型国家,在大国身边是要做出明智选择的。好,这一段我再插几句话、我再插几句话。我国的经济学家或者经济专家,我不能再批评他们,再批评他们,他们会集体跟我跟我翻脸的。就是在认定体量——经济体量的这个问题上是不用脑子的。比如说中、美、俄经济体量到底是个什么状况,按照美元计价的GDP来衡量,那么美国是20多万亿,中国是18万亿附近,俄国呢……

俄国是一万亿美元多一点,比广东省的体量还小。真实情况是这样吗?很扯很扯。按照购买力平价,中国应该大概在30万亿美元附近,美国绝没有20万亿,应该在15万亿附近,就是中国的一半。俄国的卢布被严重低估了,俄国的经济总量怎么都在5万亿这个水平,不然他撑不起如此庞大的军事。只不过是在用美元计价的这个逻辑基础上计算下来他很惨、很惨,其实事实完全不是这样。不要理解错,算错就会做错。

事实上,体量具有决定意义。中国的体量以购买力平价,不是超过了美国,是超过了奥库斯,超过了美国的现在盎格鲁人的联合。至于俄国和欧盟如果结体的话,它必然成为一个中心。也就是说美国为什么要挑动乌克兰问题,他很怕欧洲大陆一体化,就是德、法与俄形成某种连接——形成共同体。我说到这儿我就直接先把答案说了,其实普京打这场仗的目的就为这一件事儿——俄欧一体化,就为这一件事。

美国很清楚他处在迅速的衰落过程中,而且这种衰落目前没有伟大的政治家是遏制不住的。他比较可以依靠的或者比较相信的就是个奥库斯,就是这个五眼联盟加日本,五眼加日本。韩国和台湾也可以加入进去,就是五眼,再加东亚三眼,形成这么一个结构。但即便是形成这么一个结构,与一个中国也只是形成了一个平手的局面,并不具备(我说的经济体量)绝对优势。而中国现在有西亚、有中东,

有东南亚,甚至我们可以背靠背的还有俄罗斯。在总量上面其实已经构成了战略均衡,这是一个事实。中小国家要考验你的智慧了。我要说的是类似于像日、韩、台、新这些地区或者是国家,应该考虑如何与周边的大家伙相处。乌克兰的最好模式就是瑞士模式,中立,相安无事,两边好处我都要,坏处一点都不要,他就是一个最好的国家。一旦选边站,

一旦选边站,万劫不复。我要说的是我们今天看到的是乌克兰,其实我们周边的日、韩、台、新,这4个地区或者是国家面临的是同样的问题。如果不搞等距均衡,不这个长袖善舞,不在4个鸡蛋上跳舞。其实随着时间的推移,结局是非常悲催的。以台湾为例,现在靠的是护国神山——台积电。5年之后,2027年台积电或许还是神山,10年之后呢,2032年。

中国在迅速的进步,5年攻不下来,10年也攻下来了。10年之后,你这独门绝技没了,那怎么办呢?那怎么办呢?政治家有这么玩的吗?有这么短视的吗?有这么肤浅的吗?况且我从来不认为台湾是进入了西方发达国家的一个政治体,经济上是挺发达的,但是政治呢?日、韩、台、新4个地区和国家具有同质性——门阀政治、财阀经济,比乌克兰强不了多少。

我总对台湾的朋友说,你们那儿有政党政治吗?No,你们那儿不是政党政治。一个叫日裔,台独的;一个叫美系,也是台独的。日裔和美系,我们以前把美系,国民党的美系当成是统派,错误。错判了,错误的离谱,错误的离谱,没有的。有没有想统一的人?太多了。但他们呢非日裔非美系,能登上舞台吗?这回新党的选举其实是告诉我们了一切结果了,告诉我们该做什么了。不要寄望这些东西。

在一个日裔和美系控制的政权里边,我们非要去想其他的东西就没有意思了,我们只是心疼那2000多万的人民,他们是无辜的,虽然他们长时间地反复地被洗脑,他们确实有认知的问题。所以蔡蔡子说的认知作战是非常有意义的,对,因为是个认知问题。就像乌克兰的老百姓,反共、反俄这不是个问题,反德也不见得就是个问题。但你到底要什么呢?你要的是妻离子散吗?要的是国家的衰亡吗?

所以这两天在准备周六的这个聊天的时候,其实我心里边非常纠结,因为当我们穿越历史看回来的时候,对好多事情我们会产生一种非常沉重的感觉。我国算是比较的幸运,说到这儿我又忍不住想感谢一下子毛主席,他老人家太了不起了,不是他的土地革命就没有新中国,不是他那10年的那场革命就解决不了“乌克兰的问题”,我们就还有可能会乌克兰化,会台湾化,会日韩化。

不管别人怎么说那10年的事情,我都想辩证一点的说就是精神解放是第一位的。如果土地革命是完成肉体的解放,那么精神的解放是后边那10年。虽然他老人家很失落就是不知道后人怎么评价他。虽然离他离去的时间并不久远,但我们可以理解,中国人特别是中国人主体性复苏就是他给的。你不要小看造反有理这4个字啊,这是主体性复苏啊,他不迷信呐,他在质疑啊,质疑是成佛之路啊。

所以不管我国怎样,我国在体制内、在民间都有一大批的思考者,其中有一些是非常优秀的思想家,他们总能在这个国家需要的时候擎一把火炬站在高处照亮前行的路,这是我们这个国家、我们这个民族值得骄傲的地方。而乌克兰没有这样的人,在乌克兰最黑暗的时候,你看到的不是火炬,你看到的是更加的迷信。我们有时候可能也不是特别的,我们不是说我们不需要类似于像柴静、鲁豫这样的人,但我们这个民族怎么能把整个的方向交给他们这样的孩子呢?他们也不一定就是坏孩子。

有朋友送过我一些(我知道现在国内的人可能也读不到,也很少去读)俄罗斯和东欧的文学作品,有朋友送给我。我知道这个地方文化底蕴很深的,我去读一些文学作品,但我不喜欢那些类似于像俄国的那种伤痕文学那种东西,我很讨厌。我不但讨厌他们的伤痕文学,我也讨厌我们这边的没完没了的伤痕文学。他们不愿意写劳动者的伟大创造,那种精神的饱满和丰富,也不愿意让我国当代的思考者和思想家走上历史舞台,他们非要搞那种东西。

我国在很长一段时间沉浸在这种气氛里边,直到今年,今年的……好吧,我就不要再说了,再说又犯规了,直到今年一些作品仍然是这个东西,但它可以顺利的通过,我说的是那个赛车手作家的作品。我不知道发生了什么事情,但我希望乌克兰的事情可以叫醒我们有关方面,在教育、学术、传媒整个的领域里边,请记着,我们拒绝乌克兰化,少来吧,少给我们玩那些没有意义的东西,少来那套去除中华民族主体性的东西,我们不相信那些东西。

好,说点具体的。乌克兰战争的现况,我想大家每天看新闻应该知道,我说一点对战争进程的预期。我对普京是很钦佩的,他的判断、他的勇气,但我这两天也极度的担心、极度担心,因为如果做沙盘推演,如果我来指挥俄军的话,这场仗我不会打的这个样子,我觉得他可能还是70岁了,心慈手软呐,心慈手软。他采取的策略虽然军事策略是现代化的,

军事策略是现代化的,因为你拥有制天权、制空权、制海权、制电磁权,中枢神经的瘫痪,在技术上是没有问题的,现在大体上也做到了,但是他没有斩首,在这个地方看出他的硬汉的柔软的一面吧,他没有实施斩首。其实对乌克兰的,真正操纵乌克兰的政治与军事的特定的人,我想普京是知道的、清楚的,不应该给他们谈判的机会、不应该给他们谈判机会,应该在战争初始阶段,必须团灭,必须团灭,但我不知道为什么会这样。

所以我知道如果不能在短时间之内完成一面倒的局面,就可能陷入到一种僵持。毕竟乌克兰不是格鲁吉亚,毕竟乌克兰是一个四千万人口的大国,他又是在俄国与欧洲的交界区域,他又接近80\%的人口是乌克兰人,处理这个事情必须是快刀斩乱麻,不能胶着,绝不能变成持久战。我说一下我的意见,如果在一周之内能够彻底解决乌克兰的武装问题,就是去武装、去纳粹,

那么乌克兰的谈判可以进行。乌克兰的谈判展开之后,乌克兰可能会处在一种这个撕裂的状况,就是从基辅一直到敖德萨这一线以东的部分,大部分可能会成立亲俄的共和国,其中有部分可能会直接并入俄国版图,大部分应该是亲俄的共和国;至于西乌克兰的部分,可能在谈判之下,之后,鉴于混乱的情况,可能会波兰或者是其他的国家会出手代管,或者是形成另外的一些共和国。就是一周之内如能军事解决,那么最后的结局就是,

最后的结局就是乌克兰的分裂;如果一周之内军事解决不能完成,乌克兰部分的武装力量进入游击战状态,事情就会复杂化了、事情就会复杂化了。我说一下,最好的结局就是一周之内解决,然后分割成若干个共和国,如果不能在一周之内解决,可能要考虑俄军的治理能力和防卫能力,可能就必须以北顿涅茨河到敖德萨这一线划定一个类似于三八线的东西吧,最糟糕的结果可能就必须得退回到那两个共和国去。

看看这两天的战况,因为我想普京是一个有军事战略头脑的政治家,他注定成为二十一世纪的伟大政治家,是不是最伟大的,现在下结论尚早,他是一个重要的政治家;另外,乌克兰这场战争决定普京在俄国历史上的地位,也决定他在人类历史上的地位。如果,我这是说如果,普京做成了,那么欧洲大陆就是欧俄一体化的局面将会迅速展开,在普京的后十年执政,在2032年之前,

欧洲大陆将出现去美化的趋势,甚至可能北约会解散,形成新的欧洲安全框架,欧俄一体化的一个欧洲安全框架。并且这个安全框架由于俄国的存在,由于中亚和中东被中俄两国携手控制,形成欧亚大陆板块的相对的稳定,将进入一个十年的欧亚大陆的融合和高速发展阶段,真正的黄金十年,这是最好的结果。我想普京敢于冒险的目的就是要重新建立欧洲的安全新框架。奥布莱恩没有说错,他是准备改写欧洲地图的。

在这里边讲几句美国的盘算,美国再也没有伟大的政治家了。基辛格一再强调:不要触碰那条底线——就是基辛格是反对北约东扩的,应守住里根向戈尔巴乔夫的承诺,不要越过。因为越过之后,去美化将成为必然。因为让俄国独立长大,欧盟就必须在美国的控制之下,冷战还是隐形存在的。然而、然而,美国的政治家……

美国的政治家越来越赵本山,其实比泽伦斯基好不了多少,他们目光短浅、急功近利。所以北约的不断的东扩,在一再地激发俄国人,而北约的主要国家德、法又是能源和原材料极度稀缺的国家。你知道德、法与俄的匹配、一体化将是完美的一种结构,就是他的能源、原材料,他的核武装构成与德国的科技、法国的科技和工业的完美整合,所以他们更像一个国家,是更好的结构。

阻止这个结构发生呢,就是北约不东扩。北约东扩,美国人的想法是粉碎俄罗斯,太天真了,这也太不了解沙俄的历史了,他也太不了解俄国人了。所以东扩的结果,使那个沉睡的雄狮慢慢舔完伤口苏醒过来,重新振作,扑向欧洲。扑向欧洲不一定会吃掉欧洲,而是赶走美国。当美国离开中东的时候,他就应该思考他如何在欧洲的存在。离开中东之后,欧亚大陆板块的融合已经再也不可阻挡了,再也无法阻挡了。

在这个时候,在欧洲,北约东扩意味着德、法放弃北约,与俄国形成新的欧洲框架。在亚洲形成新的亚太安全体系——试图让日、韩、台、新形成一个防御链条,包括印度,形成一个新的防御链条。你知道这个链条绷太紧的时候会断掉,断掉的结果就是美国离开中东之后又离开欧洲,离开欧洲之后又必须离开亚洲。美国终将成为美洲的美国,貌似激进、貌似前卫的所有的外交活动,都违反了老子《道德经》上的基本原则。

在2月4号,我前两天写了篇东西,后来想了想还是别发了。2月4号,就是我们冬奥会开幕式的那天吧,普京和习主席签了《中俄联合声明》,其中经济的部分,特别是能源的部分是以欧元结算。好多人看不懂,说为什么不以人民币结算,或者是以俄国卢布结算,非要以欧元结算。你看得出来,中俄这两国的大政治家看的是多么的远——用欧元结算,欧元结算实际上是中、俄在支撑欧元,在支撑欧盟,非常简单。

美国不知道吗?美国非要在乌克兰再搞一次南斯拉夫式的内战,但他想普京会上他的当吗?会划定一个战线,在两个共和国和乌克兰之间,打得个稀烂,就像当年的南斯拉夫似的。但对手是俄罗斯,特别是普京,他再也不是塞尔维亚了,他对手完全变了;另外这个人他是有胆有识的,敢干,他不会给让你去打内战,陷入一场内战的。所以我在跟朋友交流的时候,我一早就说,不战则已,战则全战,战则势在必得,两去——“去武装、去纳粹化”,什么意思?就这个意思。

好,我们留点时间谈一下中国的思考。在这场乌克兰战争中,中国的角色是什么呢?我们必须承认,乌克兰战争为中国提供了……对中国的战略态势提供了一种缓解或者是减压。因为美国这个对中国的挤压非常的猛烈,这一次普京出手,极大地缓解了美国以及周边国家对我国的战略上的挤压,就是我们可以有一点空间、喘一口气了。在处理欧亚大陆的一体化问题上,中国要极具战略眼光,要勇敢的……

要勇敢地向西去,所以我当年写的……十几年前写的那篇文章《大路朝西》。在与俄国背靠背之后,我们迅速地要解决西亚和中东的发展问题,我们不用军队,但我们要迅速地解决这些国家经济发展问题。我想呢,在整理完我国的金融体系之后,我国领导也会认识到这个问题,要允许阿富汗、允许伊朗、允许土耳其到北京发行人民币国债,发行他们国家的人民币国债。然后展开大规模的建设,提供系统的财政、金融支持,让这些国家发展起来。让大陆桥从阿富汗、伊朗、土耳其延伸到欧洲,形成新的结构,要抓紧时间做。

但我国懂得断是非,我们有我们自己的路线;我们懂得断大小,什么事重要,什么事不重要;懂得断远近,我们知道该如何处理好关系,无论如何,我们要处理好与美国、与俄国、与欧洲的关系。我们实施一个基本的等距外交和战略均衡,我们不搞一边倒,不搞远近亲疏,但我们一定会在经济上做出非常好的努力。终于看到这个势头了,因为以前中国在某种意义上是一边倒的——就是亲美,救美国就是救中国嘛。

但我们终于在2012年之后,我们进入到一个新的时期,2014年我们央地税合并,2015年这个取消联汇,我们终于在经济上的主体性逐渐恢复和建立。政治上的主体性、经济上的主体性、文化上的主体性,全部建立。现在文化主体性是个大问题,春节怎么会有那些东西出来?文化的主体性建立起来之后,中国就可以来着手自己来规划整个的亚洲的未来,乃至于全球的未来。普京在考虑欧洲的安全框架,难道我们不需要考虑亚洲的安全框架吗?难道我们不需要跟欧洲一起来考虑欧亚大陆的安全框架吗?如果是用这样的胸怀和眼界……

如果是用这样的胸怀和眼界来理解问题的话,那么其实我们很清楚我们该做什么,我们不该做什么。在处理疫情方面,我们做得非常之好。另外呢到了2022年,我们看到了我们再也不会去提“腾笼换鸟”的事情了。双循环会建得非常之快,我们不再放弃中低端,只是把中低端做得更好。另外呢,全产业链、全经济生态、完整的经济生态、全产业链完整的经济生态一旦建立好,我们与周边的这种循环是没有问题的。中亚、中东、俄国、东南亚、印度、巴基斯坦都没有问题,甚至日、韩。

甚至包括日、韩、台、新都没有问题。奥库斯脱钩脱不了的嘛,奥库斯怎么脱呢?脱不了的。所以事情走到一定的程度的时候,他就转折出来了。不要忘了2022年是水火既济卦,既济它是一个上上签啊,它不是个坏的东西,尤其对中国而言。虽然卦象很凶险,虽然推背图那个象很凶险啊,但对中国来讲,它只是凶险而已嘛。2022年对我国是极为重要的。因为,你懂的。

我国了不起。在2012年之后,军改彻底的拿住了枪杆子;而随后在经济上,这两年经济上的改革,部分的拿住了秤杆子;2022年我们应该拿回笔杆子了,我们应该拿回笔杆子。文化这个事情说小也不小,说大也不大,但要拿回来了。因为我们在、我在香港这个地方看的,有时候觉得很憋屈、很憋屈,不像个样子、不像个样子。希望今年呢,我们大体上完成中华民族主体性、全方位主体性的建设。在这里边呢,要感谢一下子……

说这个话有点不妥啊,这个战争总是不好的,我们还是要希望这个尽快的恢复和平啊,这是为乌克兰的老百姓祈祷,希望他们少受一点损失和折磨。也希望普京、俄国能够处理好乌克兰的事情,处理好,能有一个最佳的、最妥善的处理方案。当然我国在必要的时候也会提供某种的支持。留下时间呢,谈几句通胀的问题。恶性通胀,确定的讲恶性通胀来了,到了。

美国人很不像样子,竟然修改通胀指标,这事应该是印度做啊,他们也改了。改了之后能解决问题吗?解决不了问题。剔除了租金,剔除了租金,剔除了房屋的这个租金等等之后呢,还是7.5\%。如果不剔除呢,大概其我们的估算啊,因为它这个租金的涨幅涨得低的在15\%左右,涨的高的话超过50\%啊。所以要不剔的话肯定是过10\%,会不会到15\%呢?都不一定。那么我们认为这是结构性通胀,这个不是周期性通胀,下不去了,回不去了。这个结构性通胀意味着什么呢?意味着美元的价值……

也就是说美元的购买力打问号啊,打问号啊。现在的中国不是2008年的中国,没有谁要救谁的问题,也没有救了谁就是救了谁的问题,没有这个问题了?我们也要减碳嘛,我们要节能减碳嘛,有些东西我们不一定卖给你的啦,因为我为什么不借钱给阿富汗的政府、阿富汗的人民,然后让阿富汗的人来买东西呢,我们借钱给他,他有那么多资源可以用来做抵押、做长线的安排。中亚像伊朗都面临着重建的问题,伊拉克。我们可以把我们的金融能力慢慢的强化,财政和金融能力强化。我们为什么一定要抬轿子呢?我看不需要了吧。

如果是这样的话,那么,其实美国说他要恢复制造业,他要重新启动恢复制造业、高端制造业。我看了美国的劳工数据,我大体上可以给结论,这是不可能的、这是不可能的。因为美国的劳工结构,因为他现在88\%的构成是服务业,服务业的人转工业,没有二十年时间是不行的,因为他教育水平决定了他转不了了、他做不了工程师的,连做熟练工人都做不了。所以一个国家去完工业化,想恢复工业的难度太高了。他不是一个资本和硬件的问题,他主要还是劳动力的问题。

如果美国不能恢复,那他就得找中国以外的替代。中国以外的替代主要是印度和东南亚,就是南亚和东南亚。非洲暂时不可能,拉丁美洲也不可能,就是南亚和东南亚。替代这个部分我估计还需要十年时间。因为中国的这个制造业的体系,它是一个完整的结构,所以它意味着它的成本可能综合成本并不高。而在南亚和东南亚,可能配套的综合性的成本并不低,而且质量也未必能满足需求。所以呢其实我觉得美国要思考的问题其实挺复杂的。但我有一个预判。

美国这个国家一向杀熟的,他连他的“亲爹”——英国人都干呐。所以弄不动中国,弄不动俄国,他只能从,因为欧洲也很麻烦,他也弄不动,因为德国人和法国人还是有点脑子的。那么我看只能从日、韩、台上下手了。所以我觉得东亚这几块材料可能会变成点心。我其实对他们挺忧虑的,但你知道我国说什么也没用,其实上一次平成战败,日本人已经吃了亏了,但他们吸取经验教训了吗?好像没有。因为日本的政治结构,它这个门阀政治、财阀经济决定了他,韩国一个样子。

那么湾湾呢?湾湾有个护国神山--台积电,湾湾的电子工业还是挺发达的。但是湾湾的问题在于当美国和欧洲都感到不安全的时候,他们会发展这个东西。因为是高端工业,下力气可以发展起来。另外,中国没闲着呀,中国也在努力啊。所以我在想,湾湾的时间可能只有五年的时间,十年想不到了,就是2027年之前它就结束了。如果在这之前它没有没有完成转型,湾湾其实很麻烦了,就它必须自己把自己解放了,送回来,它就不需要我们再去解放它了啊。

结构性通胀,这里边有一个区域性的平衡和产业性的变动的问题,就是因为我觉得我国大体上对经济的理解和对财政金融的理解还是到位的,所以我们的财政处在一个比较安全的状况。我国的金融在去杠杆之后,相对而言,也预留了震荡的空间,应该问题不大。那么人民币,大家都看到了,极强。这港币和人民币分叉了,港币极弱,人民币极强。我每天走在马路上看港币和人民币兑换的那个价格就觉得很有意思,它这个一天一变、一天一变。

我刚来香港的时候,是100块钱才能换80块钱港币,现在倒过来了,100块钱港币才能换80块钱人民币。沧海桑田、沧海桑田,这还没完呢、这还没完呢,照这个重置下去,很快,港币和人民币的比价关系应该是50比100吧,应该是这样的。因为太多了,印太多了。而香港联汇制度,没有人对这个事情感到有问题。所以香港联汇4500亿,现在还更多了,美元的外汇储备,全部是美元、美元资产,美元债券、美元资产。这事还不能说,我不知道为什么,这事不能说,很敏感。

通胀之下,我们会面临的一些结构性变动,我说区域性的变动就是有些货币会变得越来越强。我看好人民币,另外一个我极度看好的货币,得等一周之后才能说结论,就是,如果仗打赢了,一周之内解决所有问题,并且达到了战略目标,未来的卢布会极强,我说了卢布被严重严重低估,卢布被严重低估。卢布计价的资产太便宜了、太便宜了,包括俄铝、包括俄铝。太——就是严重被低估了。有些货币可能被严重高估了,会出现深刻的调整。所以我一再提醒香港人:注意了!

当然提醒这个事情,你知道好多人,他也不会感激你,有的时候还会抱怨。你们都记得我一次次提醒大家,离开比特币和虚拟货币,一次次的提醒和预警:“最后一次预警!”然后让大家先退出、先退出元宇宙,我说退出元宇宙不是说你尝试进去,而是在元宇宙里边的资产的交易先退出来。好多朋友不听话,我身边的这些年轻孩子很执着,有听话的,但是不听话的孩子超过了一半,听话的孩子一小半儿。不听话的孩子输懵了,这孩子们有的时候会:“卢老师,你当时没跟我说会跌到这个程度嘛!”

我能说出来什么时间跌到什么程度的话,那我就不是卢老师,好吗?这个,提醒了嘛,提醒了嘛?因为这是个大的趋势嘛,因为一个时代结束了嘛,这么点事情应该很好理解嘛!这一周我国又出台了相关的对数字货币的一系列的,是法律规范了;我估计美国那个拔插销的时间也快到了;虽说有好多朋友都输了一多半儿,还是要走的、还是要走的。别执着、别执着,恢复到正常的投资上来,做有意义的事情。

至于产业的变动大家都看到了,就是属于高科技“两化”的东西,掉得稀里哗啦、稀里哗啦的。好多朋友也是问我,我说这个形易势易,你得跟上。我说,我是鼓励大家投两化,但是我后来不是说短股长金了嘛?怎么就太执着,所以好多朋友损失是挺大的。不过也没关系,已经都淹到这儿了,就不要截肢了,就忍着。下回要听话,要跟上节奏,因为节奏非常重要。我说了,刚才我说产业结构就是我们一个是空间布局,就是你站的那个地方是对的;第二个是时间节奏。

两“jie”啊,一个是结构,一个是节奏,都不能错。你结构是对的,多待点时间不怕的,你忍一忍,它不会损失你,对吧?它可能是时间长一点。节奏非常重要,因为你结构是对的,等到比如说金,到了那个时候,我们出来,正好遍地都是好东西,我们就出去捡东西,捡完了以后可能10年,慢慢的不停地换一些,就是因为每一个产业也有它自己独立的节奏,我们跟着节奏走,有10年应该彻底解决财富问题了吧?应该解决、彻底解决啦?因为你只要结构是对的,节奏也是对的,成长的幅度应该是超过基金的平均水平吧?我都不敢说。

好,就聊这么多,我觉得今天该说的话都说完了。香港的疫情极为严重,严重到就是大家生活都有问题了,现在生活都出现严重的问题,好多公司因为发现了有人染疫之后,就都必须进行隔离了,我这层写字楼都没人了。公司呢,我就让他们休息,我每天来上班,我值班。我看这整个这个写字楼里边都空空荡荡了,因为我看可能有一多半的公司都有人会染病,因为你坐巴士、坐地铁,现在香港茶餐厅还在开呢,难免不被感染,所以情形真的非常的严峻。

国内也零星的有一些、有一些。我还是建议大家做好防护,第三针一定要打的,主要是为了活着,可能有附带作用,但是打针之后的这个重症率确实低,所以第三针还是要打的,这是第一句话。第二句话是,一定要戴口罩,最好戴N95,没有N95就戴两个口罩,双重,热一点就忍着。在南方会难受一些,北方还略好一些,一定要戴口罩。另外就是,不管从哪儿回到家,第一件事是洗手,盐水漱口、盐水漱口。最后,就是多吃点蛋白质的食物,保持自己的体能。

好,今天就聊这么多,明天下午三点见,有什么没说到的明天再做补充。今天的内容比较敏感,我们就内部交流,文字还是在内部,就是不往外传。因为虽然我知道我的看法,可能有些看法可能有价值、有意义,但我怕现在这个大家都非常敏感,另外这涉及到一个战争的问题,很复杂。好吧,明天见。

\section{新自由主义与新马克思主义、谈几句乌克兰战争、市场}

大家好,今天是2022年的3月5号,壬寅年,二月初三,惊蛰。理论上,惊蛰一启,可能这个瘟疫就会慢慢消退了,但愿如此吧。非常感谢大家对我的关心,我会尽量小心的。今天是《资本论》的最后一讲,第二十四讲,讲新自由主义与新马克思主义。今天这堂课重要,希望大家喜欢。如果有时间谈几句对市场的看法。好的,我们三点钟见。

大家好,今天是2022年的3月5号,壬寅年二月初三,惊蛰。今天是《资本论》第二十四讲,也是最后一讲,我们讲新自由主义与新马克思主义。最后一讲,我们将、特别是将新马克思主义以及新马会一些想法向大家做出汇报,并且我们将就此出发,开启壬寅年的一系列新的工作。其中一些工作可能、未来可能会将我们相当部分的人聚集在一起,有可能会形成一个……我现在还不敢多说,因为还没有想得非常的完整和好,而且外部压力也非常大。

然后最后如果有时间,我们就谈几句市场,因为市场的剧变已经开始了。我们似乎做对了一些什么,但是我们似乎还做得远远不够,所以我们今天再谈谈市场,其实主要是为了做一点点前瞻。我只是抛砖引玉,因为市场前瞻是一个难度极高的工作,我总怕我判断的不是十分的精准。就是两“jie”,一个是结构,一个是节奏,我们都不能错。另外如果还有时间的话,讲几句乌克兰战争的未来的走向。今天比较热闹。

好,开始我们《资本论》的最后一讲——新自由主义与新马克思主义。准备这一讲的时候,因为是最后一讲,我有一种淡淡的忧伤。我想到了鲁迅先生的作品《伤逝》,我在读中学的时候,第一次读到鲁迅的小说《伤逝》,我是被震撼到了,但我那时候并不是很懂,我只是觉得很忧伤,因为结尾不好。大学的时候我又读了这个《伤逝》,那个时候我已经开始懂一点爱情了、懂一点爱情了,所以我就更为忧伤,甚至,

甚至这个《伤逝》改变了我对爱情、对与人与人之间的相处,甚至改变了我的一些想法。我记得我在平台上说过,我在大学时候的一个演讲,就是关于这个愛这个字,我说的是愛这个字的繁体字,因为每个人都会写。一撇,你看那一撇,那一撇就是风,下边的三点是雨,我们在一个风风雨雨的世界,在风雨之下有一个秃宝盖挡住了外边的风雨。秃宝盖下边是一个友字,朋友的友,当然繁体字在友上面还有一颗心。

其实愛字的全部含义在这个秃宝盖上,它是道义感、责任感和牺牲。我其实由于《伤逝》的原因,所以我从来不敢用这个字,老实说我只敢用这个字对祖国、对母亲。因为心有多大爱就可以有多大,下边的友,可以是你的爱人,也可以是你的父母,也可以是你的孩子,当然也可以是你的祖国。我为什么讲《资本论》,要讲这么远的东西,要谈《伤逝》呢?是因为我觉得马克思在提出一个学说的时候,

马克思在提出一个学说的时候,特别是解放的学说的时候,要想更远。当然我没有抱怨马克思的意思。你知道我读鲁迅的《伤逝》的时候,我对涓生的意见是非常大的,《伤逝》的男主角叫涓生,女主角叫子君。这个《伤逝》写的背景是一百年前——五四运动。子君是生活在一个旧时代的青年女性,涓生是新青年或者是已经不是青年了,年纪稍大一些吧,他已经是拥有相当现代知识的现代性的知识分子。所以他跟子君的交往的时候,他把她的精神引入了现代,于是两个人非常的……

两个人热恋,并同居了。但随着时间的推移,涓生对子君的情感慢慢退却,后来他终于告给她说“我不爱你了”,涓生告给子君,然后涓生逃了。子君被父亲领回,死掉。后来涓生痛不欲生,就是这么一个故事。我想说的是两个人激情澎湃,为了反抗而私奔,私奔之后呢?私奔之后就是《伤逝》,大多私奔没有好结局。革命和解放,

革命和解放之后呢?那个第一个建立苏维埃的国家,原来的沙皇俄国后来私奔了,变成了苏联。一番“恋爱”之后,后来不爱了,又被他的父亲领回去,变成了今天的俄罗斯,好像回到了原点。其他很多的社会主义国家的革命和解放的之后的情况,也都未必如理想中的那样一个现实,很多时候涓生和子君都没有想到更远的地方。

马克思在写《资本论》的时候,这是一个无产阶级的圣经,是谋取自由与解放的思想武器。问题是革命、解放,然后呢?然后呢?然后呢?然后像苏联一样,在1991年圣诞节结束吗?结束一段历史吗?难道这个理论本身没有需要讨论的地方吗?抑或者他的后来人,

抑或者作为一个激情澎湃的哲学家、思想家,他其实并不能解决全部问题。而随后在马克思身后成长起来的哲学家和思想家,解决了问题了吗?抑或者他们并没有解决问题;抑或者解决了一些问题,还有一些问题需要解决。这就是我们为什么提出新马克思主义的原因。其实就马克思本人而言,也存在一个青年马克思主义,青年马克思、中年马克思和晚年马克思。你若问我,我喜欢晚年的马克思。

马克思写《资本论》,他写完第一卷其实就没有再怎么的往下写了。我想马克思在晚年的时候一直在思考以后的事情,就是私奔了、革命了、造反了以后的事情应该是什么样?但是马克思遇到了障碍。或者,抑或者,其实马克思已经想到了涓生和子君的结局,已经想到了。我并不认为涓生是个坏人,我只是觉得涓生没有负责任,愛这个字说出来是收不回去的。

因为它不是一个简单的表达,他后边儿跟着的是一个契约,有的这个字后边跟着的是一张结婚的纸,一生的厮守,还有一些产业;有的可能也没有那张纸,但仍然是一个契约。说的时候是情感,最后可能是资产,毁约可能毁掉的不是一个约定,而是一个人、一群人的生命。我们对哲学、对思想,必须得认真负责任。

《资本论》核心的部分,最有价值的部分是价值论。价值论的部分因为它通过价值论的讨论,讨论了这个价值的本体或者是本意,讨论了价值的构成,讨论了价值的创造,讨论了价值的积累。其中引申出剩余价值的理论,为大家来重新理解劳动、理解劳动所得提供了一个完整的分析框架。随后马克思又写了资本积累、写了资本流转。其中资本积累的部分我们看到了,看到了资本积累过程中的秘密。

在资本流转中呢,我们也看到了资本流转过程中所带来的意义和伤害,诸多的问题。在我们研读《资本论》二十四讲的时候,我们没有完全受限于19世纪马克思所处的这个环境或者是思维定式,我们将资本积累延伸出去。我特别提到了马克思资本积累里边的第一卷第七篇第二十五章——现代殖民原理。在资本积累里边,我们也讨论了后来在马克思的二卷和三卷均有提到的地租问题。

在资本流转的过程中,我们延伸了。因为马克思谈的资本流转实际上是在特定的结构中流转,我们将它延伸为地域的流转,就是国家之间的流转。现在我们看到的金融制裁SWIFT,其实就是国家之间的流转。我们讨论了阶层流转就是穷人向富人的流转,古代的所谓的兼并,现在所谓的资本兼并。我们讨论了阶层的流转或者是阶级的流转,我们也讨论了产业结构间的流转。比如说,现实向虚拟的流转,这是《资本论》所未能涉及的事情,因为时代不同了,我们做了理论的探讨的延伸。

那么,可能朋友问我,你谈新马克思主义,为什么要谈新自由主义呢?我必须说马克思主义就是为自由主义而生。因为在马克思主义产生之前,那个提供了资本主义的哲学的基础或者是理论的基础就是新自由主义,或者是,我把它分成三段吧:古典自由主义、英镑新自由主义和美元新自由主义,这是我的一个分法,可能不太好吧。但我现在没有办法把它想得更为——用更学术的语言来表达,我先把它分成三个部分,古典自由主义……

哦,我有可能又摁的时间长忘了,我是脑子走神了。古典自由主义对应的应该是古典殖民主义,或者是叫古典资本主义。这个对于理解今日之西方非常重要,因为古典殖民主义是消灭原住民而获取他们生产资料的那个殖民主义。它也构成了古典或者是原始资本主义资本积累的最主要的源泉,最主要的源泉。因为在原始资本主义或者古典资本主义里边,资本完成了帝国的扩张。

《资本论》涉及到的经济史的部分不是特别的丰富,或者是有论述到经济史,但没有非常系统的讨论。在马克思《资本论》的第三卷的一些资料里边,或者是在马恩列斯全集里边有涉及到一些经济史的问题,但在《资本论》本身里边讨论的并不多。在考虑资本积累,就是古典资本主义或者原始资本主义的资本积累的结构里边,我自己将殖民主义或者是殖民掠夺而形成的资本,将它放大到整个资本积累里边的最主要的源泉,其次才是剩余价值积累。我正在想办法,

我正在想办法提供详细的数据,经过整理,数据。比如说北美地区、拉丁美洲地区、澳大利亚新西兰地区,他这个资本主义形成的整个过程里边,大部分的可能是这样的一个形成的过程。其中类似于像美国的崛起里边,他们的资本源泉——白银的源泉,实际上是中国的,就是通过鸦片贸易来获取白银,或者直接通过战争来获取白银,形成了大规模的资本积累,而不是在本国的劳动者的剩余价值上完成的积累。这个研究是非常有意义的,这个和马克思《资本论》的讨论,可能它是个主次之间的差异吧。

就是马克思可能认为资本积累是剩余价值,但我认为剩余价值是第二位的,这是古典的部分。因为古典的部分,马克思处在的是,还不是,马克思接触到,他生命的一部分时间,他是接触到了古典的原始积累,因为他生长的时期包括了中国的鸦片战争,但是这个发现新大陆和开发新大陆的时间比马克思生长的年代还要久远。对整个自由主义、古典自由主义和新自由主义的解释,

对古典自由主义和新自由主义的解释是当代资本主义合理性、合法性的哲学概述或者是理论解释,马克思主义就是为了驳斥它们而产生的。新自由主义为什么我分成英镑新自由主义(我管它叫现代殖民主义,我刚才说的是古典殖民主义,现代殖民主义)、美元新自由主义呢(我管它叫后殖民主义),主要是为了划分三个区间。马克思接触的是一部分的古典殖民主义和现代殖民主义,所以马克思第二十五章写的是[现代殖民原理],就是第一卷第七篇资本积累过程中的第二十五章[现代殖民原理]。

现代殖民就是马克思写[现代殖民原理]那个时候的殖民,西方的殖民者不再,还屠杀殖民地的人民,但不像原始时期的那种种族、古典殖民主义里边的种族灭绝。就是把印第安人杀光,像澳大利亚是把土著基本杀光。现代殖民里边主要是掠夺生产资料,不杀光,不是他爱他们或者是仁慈,是他需要劳动力,他以劳动力的方式重新奴役他们,现代殖民。这个在中国表达得也比较清晰,在印度是纯殖民地,中国是半殖民地,表达得还是比较清晰的。

需要解释一条,也是《资本论》里边没有讲到的,就是现代殖民主义和现代殖民原理里边掠夺的不仅仅是生产资料,掠夺的最主要的是直接掠夺你的资本。鸦片战争,我国积累两千年的财富——黄金和白银就在短短的时间之内,以鸦片贸易的方式和战争赔款的方式基本上全部流往西方。其实是这个东西构成了西方的工业化的大跃进,就是这个资本,就是这个资本构成了工业化大跃进,特别是美国。

我知道我国的……我又开始批评了,我知道我国的学者、专家不用心,就是对当代西方的经济史和我国的经济史都不用心。因为我在整理香港的当代经济史,其实我是非常非常震撼的,我甚至可以用两个字来说,就是心痛,就是心疼。怎么就大家都不知道呢?怎么就看不明白呢?《资本论》解释了剥削,只是解释了资本家对工人的剥削,资本利得对劳动所得的剥削。这不能抱怨马克思,因为马克思他的出身、他的学术范畴,

使得他不能站在一个国家的角度、一个民族的角度来再次理解自由主义和新自由主义或者是理解殖民主义、古典殖民主义、现代殖民主义或者是后殖民主义。但作为我们这样的国家的马克思主义者,你就不能不理解;如果你不能理解后殖民主义,你说你能理解当日、当今的美国、当今的中美关系,我就觉得你有点夸大其词了,因为不理解后殖民主义原理就无法理解美元新自由主义,

就不能理解美元新自由主义。我今天开篇的时候说了《伤逝》,那么我想说的是当涓生给子君开启一面窗的时候,新社会、新生活是什么的时候,涓生可能并没有去仔细地审视原来子君的那个环境,就是她生存环境里边的那个东西。就是当提出马克思主义的时候,你可否真的理解自由主义呢?如果你并不是真的理解自由主义,而轻易地提出革命、造反,

轻易地提出革命、造反或者是私奔,那么就可能出问题。因为我主张,就是当你带她进入新时代的时候,这个“新”只是一定意义上的新,是旧制度上的新,而不是粉碎、割裂,不是。当你面对新的时候,第一个倒下的、第一个疲倦了的是涓生,不是子君,是涓生不行了,这就是问题。

好,我们来谈谈美元新自由主义,美元新自由主义。如果我们写新的《资本论》的话,就是美元新自由主义在经济上的表达主要是四化:第一化是私有化;第二化是市场化;第三化是资本化;第四化是全球化。你知道列宁带领着布尔什维克他们干了什么呢?他们干掉了私有化,他们搞公有制;他们干掉了市场化,搞计划经济;

他们干掉了资本化,他们搞产业发展;他们干掉了全球化,他们搞了局部的流通,搞了局部的流通。对还是不对呢?你必须看到四化,现在这个四化,我管它叫“新四化”,跟我们说的四个现代化是两码子事。私有化,对还是不对呀?那你苏联解体之后又私有化了;苏联解体之后又市场化了;苏联解体之后又重新资本化了;苏联解体之后又重新全球化了。普京否定了列宁,普京否定了列宁。

那个被涓生(马克思)启发之后出走的子君(俄罗斯)变成了苏联之后,最后子君又被她父亲领回去了,领回去的子君又回到了原来的样子。他们有一个非常糟糕的私有化;有一个非常糟糕的市场化;有一个非常糟糕的资本化;有一个非常糟糕的全球化。新马克思主义要回答的问题是什么?其实我认为新马克思主义也谈不上“新”,马克思晚年可能思考的就是我们今天在讨论的问题。换言之,马克思并不反对私有化,并不反对市场化,并不反对资本化,并不反对全球化。

马克思并不反对四化、新四化:私有化、市场化、资本化和全球化。为什么不反对?因为马克思是辩证唯物者、历史唯物者。在人类发展的这个阶段,是不能完全去私有化的,不能没有市场的,不能不进行资本化的,不能不进行全球化。如果你不懂这个东西,你的结局是北朝鲜。是不是马克思主义?我个人认为是青年马克思主义,而不是晚年的马克思主义或者是不是真正的马克思主义。

那么马克思主义或者是新马克思主义如何来理解“新四化”呢?要不要私有化?要的、要的;但是我们要社会化的私有化,这才是社会主义。社会化的私有化就是资产虽然是私有,但正态分布于整个社会的所有劳动者手上;它是私有的,但它是社会的,它集合起来是社会主义的,而不是用苏联的所谓的公有制,由国家代持形成国家资本主义,那不是晚年马克思的本意。

大家同意吗?因为晚年的马克思非常痛苦。你记着,马克思提出的公有制是大公无私的“公”,马克思没有说“共有”,是“公有”不是“共有”,共产主义才共有。这个“公”就是社会化的意思,社会化的私有化才是社会主义的本质。这是新马会里边谈经济的第一个根本点,如果没有这个根本点,我们就无法立足了。任何反对社会化而走向兼并的私有化才是我们的敌人,我们要的是社会化的私有化,而非集中的私有化。

第二,我们要不要市场化?当然要。越是社会主义越要市场化,但是这个市场必须是公平的。香港有平等机会委员会——“平机会”。其实大部分的资本主义国家、西方发达国家都有“平机会”,它其实是社会所有成员在市场里边的公平的身份。为什么反垄断呢?为什么有一系列的法律来保护市场的公平呢?如果我们不做市场化的公平化,我们叫社会主义吗?我们如果还像当今俄罗斯这样的从苏联解体以后回到最坏的资本主义,就是子君被父亲领回家了。

如,我,我不能说我,如我们是涓生,我们既然跟子君说了那个愛字,给了承诺,就应该给她一个新生活。你把她从黑暗中拉出来了,她看到光明,她回不去了,但你没有给她新的东西,因为涓生不知道什么是新的。虽然列宁、斯大林他们经历了痛苦的试验,认为那是对的、新的,后来发现不行,活不下去啊,疲倦了,又退出去了。可我们不行,我们不能那样做。

所以我们和他们不一样,我们决定提出新马克思主义,我们要在走出黑暗之后,在光明之处建立一个新的、完全是更好的、新的更美好的世界给子君、给我们的人民。好,我接着说完,那么马克思接受资本化吗?当然要接受资本化。不将重要的资产或者生产资料资本化,它就无法有效率地运行。但是资本化的过程中,必然有资本家,必须节制,有节制的资本化就是社会主义。

其实连孙中山都在反复强调节制资本的问题。但你知道,学了《资本论》的人或者是以为懂了马克思的人是比较主张消灭资产阶级、消灭资本家,乃至于消灭资本。而且再到今天(我有时候说话说得稍微重了一些)一些左翼或者是极端左翼或者极左或者是他们以毛左自称的一些人,完全不懂马克思,完全不懂《资本论》,所以他们才会强调公有制、计划经济,消灭资本家、消灭资本,然后反对全球化,表达为一种新民粹主义

或者是一种极端的民粹主义,其实是一种学问浅薄的标志。最后我们说一下子全球化。马克思主义、社会主义接受全球化的,必须要全球化的,不全球化、不实行全球化的社会分工怎么能有效率呢?怎么能获得伟大的发展呢?但是我们主导的全球化,我们讲的社会主义的全球化是由我们主导、我们的国家主导、我们的人民主导的全球化,而不是由资产阶级专政的国家主导的或者帝国主义主导的全球化。

要不要“新四化”?《资本论》的核心的要件,要不要?我看都得要。我们要不要走入新时代?当然要走入新时代了。但我们走入新时代是砸碎所有的东西吗?一件都不砸碎,赋予它新的内涵,给予它新的生机和活力。涓生不懂,子君不懂,我们懂。我们必须有一个完整的理论体系来阐释这一套逻辑,并在这套逻辑基础上建设我们的国家,她才真的会具有无与伦比的生命力。你说不私有化,公有能有积极性吗?能将社会资本做到最有效的分布吗?

能将个人的劳动积极性放到最大吗?你不进行市场化,要素如何流动?要素不流动,你那个生产的效率、生产力水平如何出现?你不进行资本化,你怎样进行产业发展和产业升级呢?你不进行全球化,有限的结构里边,你怎么能获得整个的资本收益最大化呢?在自由主义有他们的根本。有很多人说,比如说共济会,比如说犹太人,有他们的一个结构上的统一体,或者是有个结构上的最高的指挥,或者最高的一个权力的基础上。我们有吗?

新马克思主义将从理论上解决我们对500年殖民历史的最后的批判,也就是说新马克思主义将结束后殖民主义。这是我们这一代人、中国人的使命。就是我们新马克思主义的完善,将终结后殖民主义,同时新马克思主义将形成社会主义整体性的一种核心、核心力量和领导力量。我不大相信共济会这种事情,假设它有吧,与共济会这样的组织,或者是华尔街,或者是背后的犹太势力,或者是Deep State形成有力的抗衡对冲,否则我们是无法获取最后的胜利的。

我刚才简单说了一下子,新马克思主义的经济层面,经济基础决定上层建筑。那么新马克思主义不可能不提政治,但是今天我们这一堂课就不能谈政治,我只是想说晚年的马克思认识到三权的重要性,立法权、司法权和行政权的重要性。其实马克思到晚年的时候并不主张无产阶级专政。什么叫专政?专政就是三权合一。文革的时候我们试验过了,砸烂公检法,行政主导一切,革命委员会、革委会,解决得了问题吗?

那场实验非常惨烈,证明呢?不行。那场革命深刻地证明了马克思晚年的担心与忧虑。社会主义要不要三权?我刚才说了,社会主义要四化,社会主义要三权,只不过社会主义的三权与资本主义的三权是不一样的。今日资本主义的所谓的三权分立并未分立,后边有一只大手,它叫资本,在某种程度上依旧是资本专政。有些国家表达的非常充分,比如乌克兰,他的立法权在谁手上?他的司法权在谁手上?他的行政权在谁手上?他专政了。

比如今日之俄罗斯比乌克兰好一点,但有限、有限,他只是出了一个明君。但中国这样的国家不能简单依靠明君,因为我们说了,中国要300年不再有内战,500年不再被外侵。这就不是一个人的问题,是我们整体上、在思想上是一个民族觉醒的问题。新马会,新马克思主义研究必须在政治层面解决新三权的问题。如何让人民真正拥有立法权?我已经讲快20年了:“人民立法、人民立法。”

我们还可以再讲10年。人民立法,人民陪审员制度就是人民司法,集体诉讼制度就是人民司法。如果我们做到了,好吧……今天政治层面的这个三权问题不展开谈,因为我怕到时候文字版流入社会有人抓辫子。最近确实是又有很大压力,好多人——因为他们说你不能成立新马会、新马克思主义研究会,你要干什么?你这个是给添麻烦。其实就是讲这个《资本论》、讲马克思主义也确实触动了很大一部分人的利益。

我知道、我知道,马克思主义在中国是一碗饭,好多人吃这碗饭。另外,国家资本主义一切的恶借此而横行,他们不想改。你的新马一旦形成体系,我们提出的新马克思主义不仅仅是一个说法,我们有算法啊,就是广义财政论里边,我们一整套检测你是否现代、是否文明的方法。就是我们刚才说了,是私有化,但社会化水平在什么程度呢?是70\%完成社会化,还是90\%完成社会化?还是10\%的社会化,而90\%被1\%的人占……

我们有一整套的检测体系,它的这个体系比基尼系数更精确、更能说明问题。新马克思主义研究会将提供全球每一个国家的文明程度、社会化程度的一个鉴定、一个指标,它说的才是有道理的,当然它会让很多人胆寒的。我是想说什么呢?你让我不做我就不做了吗?不可能的嘛。我咬牙也还是要做,不让做可以暂时不做。但我可以在脑子里想,时机成熟了我们还是要做的。不然我们这代人没有办法面对历史。我已经说了,我们这一代人……

我们这一代人不做涓生,我们不会让子君被父亲领回去、惨死。这个我不接受,这个是不可以的。说远了,政治上的,但政治上的事情我们将来有时间再谈。其实我要说什么?好多平台上的朋友已经知道了,其实你们晚上躺在床上静静的自己思考和延伸就可以了。好,还有文化层级上的,新马还有文化层级上。文化层级上其实是将形成一种新马克思主义的审美,这个审美将形成伦理、形成法理,不是,形成伦理、学理、法理和治理,这是一个文化基础。

新马克思主义的这种文化将构成一个伦理、学理、法理和治理,它将形成一整套的这个审美体系,它更接近一种普世的宗教信仰,类似于当下新自由主义这样的东西吧。它在文化上更具有多样性和多元性,更具有包容性。我觉得是一种非常了不起、一种非常伟大的一种思考。

我大体上想将新马克思主义分成两个部分:一个部分是经济的部分,单独摘列出来,其中将以广义财政论的方式来表述;其余的部分将以《新社会主义通论——前言》的方式来表述。继续做、继续做好工作,然后争取把东西整理的稍微好一点,但是由于涵盖的内容稍微多了一些,所以整体上的这个结构的难度,就是大,不太好驾驭。我后来自己也想了,其实,我应该发动群众的,我应该和平台上的朋友们一起来完成这样的工作,而不是我一个人,

我一个人的努力,不在乎它的这个时间和速度,主要在于局限性。我一直害怕我自己的局限性,影响了、影响了这个事情的它更包容、更全面、更深刻,影响到这一点。所以在结束《资本论》二十四讲的时候,其实我这些日子辗转反侧。今天惊蛰,虽然没有那声惊雷,但我浑身上下仍然是龙抬头之后的那种复苏与惊醒的这种感受啊。我说这个《伤逝》的意思,我再重申一遍,没有对马克思本人的……

没有对马克思本人的不敬,更没有否定的含义,因为毕竟马克思开启了一个时代。在开启之后,我们进入现代性或者现代化,进入现代性和现代化往往是由知识分子来进入的,一个知识分子他可能会带他的身边的一个朋友,比如说涓生身边是子君,可能会带子君进入新时代,你们也可能会带你们身边的人进入新时代。那么,这个新时代的“新”,到底是一个怎样的新?到底可不可以成全子君对新的渴望,可以包容、容纳,或者是共荣和成长,这全赖我们的努力了。

在我去了好多次伦敦的马克思墓,其实我在跟马克思讨论问题的时候,里边涉及到新马会的一些内容,我们今天也讨论了。关于无产阶级的问题,就是子君离开了黑暗,无产阶级离开了黑暗,他的结局必定是有产阶级,除非我们不接受私有化,我们如果接受私有化,那么无产阶级就会变成有产阶级。那么有产阶级怎样呢?他的国家、他的制度怎样呢?他应该是进行一个什么样的情况呢?后来我提出了“共和”的新的含义,共和。

我提出了共和的三个共:社会共治、社会共享、社会共融,金融的融。当然这里边每一个都有它的更深层次的含义。我们将在一个合适的时候,我将这个大纲给大家,我想我把大纲给大家,我们大家一起来吧,一起来做这个事情,有兴趣和时间的朋友你也可以进行思考。将来一旦我们形成文字的时候,我们可以互相的批评、批判、砥砺、完善。其实原因非常简单,因为我怎么努力都永远,我们追不上马克思了。

其实在很大程度上新马会要完成《资本论》的,我说了,《资本论》的第四卷。第一卷是价值论,第二卷是资本流转,第三卷(现在的《资本论》第二卷、第三卷件合并为《资本论》第二卷:资本流转),第三卷是列宁的《国家与革命》。现在被证明这个《国家与革命》这本书存在问题,因为最后子君被爸爸领回去了,涓生走入绝望,这个不行的。那么我们要写《资本论》第四卷了。《资本论》第四卷的核心是国家与资本的关系,国家与资本的关系。那么一个社会主义国家到底应该长什么样?我们应该有一些想法。

我国一个重要的领导人反复向身边人推荐,读《旧制度与大革命》。其实《旧制度与大革命》里边看到的一些东西,其实也是马克思晚年的忧虑,马克思晚年的忧虑。我在法国协和广场的晚上遛弯儿的时候,我一直在寻找协和广场上路易老爷的头被砍下来的地方,那个广场上人头滚滚。我有时候在想、在想,对在哪,错在哪?我们走了这么长的路了,从1921年党的建立到1949年建国,走了这么长的路了,对在哪?错在哪?没错在哪?该怎样把它走下去?

国家一定要掌控资本的,但国家掌控资本与社会掌控资本到底是个什么关系?如何建立国家的强大的能力?如何由国家组建完整的经济生态,组建有效率的市场,组建强大的科研生产的能力、体系?国家与资本的关系是一个大的课题,但它又不是苏联式的国家资本主义,就是完成社会与国家的关系的描述,是新马非常重要的一个课题,而且可能是个主要的课题。完成了这个课题就算结束了吗?当然没有。

国家之间的关系、国家与社会的关系,国家之间的关系、社会、我国社会与他国社会资本的关系,等等吧,一系列的复杂的矛盾和问题和关系,都要有一个清晰的、完整的、系统的阐释。我想,今天我就先说这么多,因为,我总得在一个合适的时候把大纲先给大家,有一部分的初稿其实在平台上也可以拿出来了,到时候供大家来一起讨论、一起讨论。这个最后一堂课《资本论》的部分就先……

我想先讲几句,乌克兰战争打了十天,大体上,我说说我看到什么。我看到了两个大的民族的融合和崛起。一个是中华民族,或者是儒家文化圈。中华民族里边,我的认为是包含了朝鲜半岛和日本、东瀛,当然更包括台湾,甚至包括了东盟,整个的就是儒家文化圈的整合和崛起。第一个是我看到东西,将在战争中慢慢的显露出来。

第二个部分,我看到了大斯拉夫民族的崛起。普京在一个特定历史时期进行的这一场战斗,实际上是斯拉夫民族历史性的反攻或者是反扑。他如果成功了的话,将使得斯拉夫民族开始走向一体化,在未来的30年左右的时间走向一体化。好多朋友可能不知道我在说什么。我想说,欧洲一共是三条线,一条线是东正教——信东正教的斯拉夫民族。他们以前由于主义、由于种族诸多问题撕裂了,现在他们正在外部压力之下弥合。

第二条线,我把它说成是西罗马帝国,主要是在大陆上的以德、法、意为主体的斯拉夫民族以外的欧洲大陆上的西方国家,我管他叫西罗马帝国。他们曾经是在罗马人统治下,而且他们的语言文字实际上都脱胎于罗马文字。西罗马这个帝国欧洲的部分,由于长时期受第三条线的挤压,第三条线就是盎撒,由英国延伸出来到澳大利亚、新西兰、美国形成的盎撒圈——美国、加拿大形成的盎撒圈,广义的西方的盎撒。

这三条线随着中国的崛起,随着斯拉夫民族的复兴而产生巨大的变化。虽然眼前整体的斯拉夫民族处于弱势阶段,经济上处于弱势阶段,虽然有个政治强人,普京也七十多岁了,但是斯拉夫民族在整体上被盎撒和西罗马压迫的这个情况下走向一体化有他的历史的合理性,而且

他们必然与中华民族形成某种程度的合作,乃至于形成整个欧亚大陆板块的重新凝结,不再破碎。欧亚大陆板块中间还有一些非常重要的民族,中亚地区一共是四个大的板块——阿拉伯民族22国:波斯、伊朗、突厥、土耳其、犹太、以色列,当然还有库尔德什么的,但那个我们再说。它中间夹着的这几个民族这个历史性的争斗,随着盎撒人退出中亚的心脏地带而形成剧烈的变化。

我认为2022年初春的这场战争开启了一个新的时代,因为斯拉夫民族在未来30年将走向一体化,2050年左右他们的经济将获得极快的发展。而相对应,西罗马帝国衰落、盎撒衰落,西罗马帝国将被迫与斯拉夫民族形成某种结合或一体化,形成欧洲新的状况。而欧亚大陆板块凝结之后,将使得外缘的部分就是盎撒,甚至包括非洲和南美洲形成边缘地带。这是我对世界地缘政治和地缘经济的一个基本的结构性分析。

其实我也不是很介意这两天的这个战况,因为打得顺利和打得不顺利,其实对我国的影响不是特别大。就是,说这个好像有点自私、有点太本位主义。就是打得不好,我国的利益也很大;打得好,我国的利益也很大;只要打,我国的利益就很大。但是我也必须说明,就是打,对俄罗斯、对普京的利益真的是极大,也只有这样伟大的政治家才懂得在这样的时间、这样的关头做这样的事情。虽然我认为他打早了一年时间,但历史地看,也不在乎啊。

就着这个话题再说几句经济。这两天聊天,你知道我和香港的朋友们在聊天的时候,一直在触碰一个话题:就是我一直在问我认识的所有的人,难道美联储不知道印钞、QE会导致恶性通货膨胀吗?鲍威尔在国会上手摁《圣经》说的那番话,不是很可笑吗?他会看错吗?你设想一下子,一个拥有SWIFT的国家、一个可以看到所有交易的国家,

他难道看不清楚资产价格和商品价格的变动趋势吗?连我们小小的一个小平台都看清了一些事情,我们这平台多小啊,我们完全无法拿到SWIFT的数据,我们都看明白资产四矩阵的流转,我们都大体上知道他们要干什么,要发生什么事情,但美联储看不到,华尔街看不到,说什么呢这是?我不信呢!如果他们都看到了,这就是故意。他们要干什么呢?如果他们是故意,这个通胀到这儿就结束了吗?加息就结束了吗?

我们的讨论有的时候一直持续到夜里边三四点,就是大家提出一个问题,按照谁检控谁举证的原则。就是卢先生你提出问题你要拿证据啊,不然你的立论不成立啊,当然有人提出其他的想法也要拿证据出来。我说我先提出立论,然后我们再讨论证据。我提出一个看法,我说如果、如果我是美国的治理者,我会让通胀上升到全地球无法理解的高度,我会让通胀上升到一个无法理解的高度。

为什么呢?因为解决美元重置,一个货币就是个尺子,我这个尺子弄得太多了,我现在要处理这个问题,我只能是价值重置。价值重置的过程中,很多国家是没有办法适应价值重置的,因为所有的资产和商品价格都要进行重新的定义。在这个定义过程中是极其混乱的,会死很多人的,会很多破产的。但我们国家,如果一个国家治理水平足够高,可以在整个的重置过程中保持平静、平安、顺利,那么他就是赢者,赢者通吃嘛。

我说:新的证据现在无法提供,老的证据有啊!1971年,美元与黄金脱钩,实现了美元的价值重置。由黄金美元变成石油美元,石油怎么定义的呢?石油是3.5美元一桶,经协商涨到11美元一桶。当时好像是七个主要的石油交易商吧,与阿拉伯谈判,形成一个石油定价——美元定价,大类资产的定价就是这个。但美元和金的关系可不是这个,35美元变850美元——23倍。

货币重置的过程中,其实最先要解决的其实就是劳动者的收入问题。一个是劳动者的就业,一个是劳动者的收入,能否平衡。资产和商品价格上涨使劳动者不受太大伤害而形成革命,这就是水平吧,这是社会治理水平吧。如果在重置的过程中,老百姓工资涨得太慢,跟不上物价上涨,就上街了,就闹事了,我们1989年见过的。那不上涨,比如说你重置我不重置,行不行?那就变成日本了,对吧?1985年之后的情况不就是这个意思吗?所以面对如此纷繁复杂的形式,其实是需要测……

还是老习惯。就是我的朋友们说:就算你说的对,虽然你没有证据,那也得完成沙盘推演。比如说,设定三个不同的幅度(就是通胀)。你说涨到惊天黑地、惊天动地、天昏地暗,涨到妈都不认识。那是涨到哪儿啊?比如说油气涨到哪里?黄、小、玉涨到哪里?金银涨到哪里?为什么?能有个预见吧?幅度和时间?与之对应的,股市会成什么样?债市会成什么样?

与之对应的美元、欧元、日元、人民币、英镑、瑞士法郎,会是一个怎样的走向?怎样一个局面?我说:初步的想法有了,但全面的想法没有,原因是数据。今日之美元并非美国的商品流通所需要的通货,今日之美元实际上是为全球通货做服务的。多印出来的美元和全球通货流通的结构相对称,应可模拟或计算出资产或商品的价格来。

注意到数字货币和元宇宙的覆灭,不能叫覆灭,就是腰斩,覆灭还不会。我们注意到货币在重新回到商品和资产的流转过程中。但它是一个(我们今天讲资本流转)……它会如何在国家间——就是区域间的流转?如何进行阶层流转?如何进行产业流转?这里边其实今天根本就说不清楚,明天也未必能说清楚,我们今天只是把这个话题提出来。然后为了给他们一个答复,我也准备一个星期,我也会把这个思考的结果在下周聊天的时候给你们。

希望香港的疫情不要再进一步恶化,因为现在每天是五万多人,现在已经四十万人了。今天可能会到四十五万人感染,看来百万人是躲不过去了,死亡的水平也是挺高的。反正下个星期我们聊天嘛,我实在来不了办公室,我就在家里聊天,只要我别中招就行了。然后下下周,我看一下子,下下周要么就是聊个天儿,要么就是你们给我休息一周也行。然后我们开始准备《通论》。今天就先聊这么多,好多朋友说你这等于没聊,也没说金也没说银也没说股票。其实……

其实我都说了,最后提醒大家一下子。因为你们看到了,这个互联网的这个部分,包括阿里这些部分,都在创历史新低。还有一些类似于像俄铝这样的受特殊时期影响的东西,它不但创历史新低,它(每天都可以)单日之内就可以给你挺高的回报。我说了最后我们会留一小点钱做投机的嘛,这个机会在。另外,俄铝也有可能形成一个未来的、长远的高回报。我想说的是你手上那些功课做了吗?你在手机上设定了你那个心里的最低价买入点了吗?没设定,今天晚上就设好。

系紧安全带,迎接一个伟大的大时代,把你挑选的东西做一个结构性的安排。其实挺有意思的,因为我原来设定阿里的买入点是一百块钱,这两天穿了。那个手机在提醒我到了,但考虑到诸多因素,因为我们还在黄金屋里边嘛,所以我又重新设定了它的买入点。每一个人都应该有自己对一些重大东西的一个观察,长期的观察,可能我们不会去买,但它可能会历史性的出现一些机会,我们要提早把它做安排。短股长金不意味着金是永恒啊!我们在合适的时候还要出来的。

至于金到哪里?我今天不能说呀,因为我怕我说了我的预测呢,好多人操作上又出状况,我要不说我的预测呢,反正是你们自己决策。我们是灵活机动的战略战术啊,就不要太执着。但当一个大的风口来了的时候,我们要吃干喝净的嘛,我们要把该拿到的全部拿到。再者说也等一年多了,时间也不短了,我们还是有一个好的收割再结束。也请大家注意疫情、注意安全。我们明天下午三点钟见,再见。

\end{document}
