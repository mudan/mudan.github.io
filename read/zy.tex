\documentclass[hyperref, UTF8, 12pt, a4paper]{ctexrep}

\usepackage{fontspec}
\setmainfont{Menlo-Regular.ttf}
\setCJKmainfont{Source Han Serif SC}
\setCJKsansfont{Source Han Serif TC Heavy}
%\setCJKmainfont{HiraginoSansGB-W3}

\setlength{\parskip}{12pt}

\usepackage{ulem} % 下划线修正
%\uline{下划线}
\usepackage{CJKfntef} % 多种汉字样式下划线
%\CJKsout{汉字下划删除线}
%\CJKunderwave{汉字下划波浪线}
%\CJKunderdblline{汉字下划双线}
%\CJKunderdot{汉字下划加点}

\usepackage{changepage} % 段落的左侧和右侧缩进

\usepackage[bottom,marginal]{footmisc} % 脚注强制放到页面底部、首行不缩进

\setcounter{secnumdepth}{0} % section不带编号
\pagestyle{empty} % 停止文档编号
\ctexset{section={format+=\centering}}

%\begin{quotation}
%{\fangsong }
%\end{quotation}

\begin{document}

\begin{titlepage}
\centering
\vspace*{\stretch{1}}
{\sffamily\fontsize{30}{30}中庸註釋摘抄}\\
\vspace{\stretch{3}}

{\today}
\end{titlepage}

\section{中庸正文}



\subsection{第一章}

\subparagraph{$\bullet$ 中者,不偏不倚,無過不及之名。庸,平常也。} ~\\

程子曰:「不偏之謂中,不易之謂庸。中者,天下之正道;庸者,天下之定理。」

朱熹《中庸章句》:此篇乃孔門傳授心法,子思恐其久而差也,故筆之於書,以授孟子。其書始言一理,中散為萬事,末復合為一理,放之則彌六合,卷之則退藏於密,其味無窮,皆實學也。善讀者玩索而有得焉,則終身用之,有不能盡者矣。

\subparagraph{$\bullet$ 天命之謂性,率性之謂道,脩道之謂教。} ~\\

朱熹《中庸章句》:

\begin{adjustwidth}{2cm}{1cm}
\indent\indent 命,猶令也。性,即理也。天以陰陽五行化生萬物,氣以成形,而理亦賦焉,猶命令也。於是人物之生,因各得其所賦之理,以為健順五常之德,所謂性也。率,循也。道,猶路也。人物各循其性之自然,則其日用事物之間,莫不各有當行之路,是則所謂道也。脩,品節之也。性道雖同,而氣稟或異,故不能無過不及之差,聖人因人物之所當行者而品節之,以為法於天下,則謂之教,若禮、樂、刑、政之屬是也。蓋人之所以為人,道之所以為道,聖人之所以為教,原其所自,無一不本於天而備於我。學者知之,則其於學知所用力而自不能已矣。故子思於此首發明之,讀者所宜深體而默識也。

道也者,不可須臾離也,可離非道也。是故君子戒慎乎其所不睹,恐懼乎其所不聞。

離,去聲。者,日用事物當行之理,皆性之德而具於心,無物不有,無時不然,所以不可須臾離也。若其可離,則為外物而非道矣。是以君子之心常存敬畏,雖不見聞,亦不敢忽,所以存天理之本然,而不使離於須臾之頃也。
\end{adjustwidth}

\subparagraph{$\bullet$ 莫見乎隱,莫顯乎微,故君子慎其獨也。} ~\\

朱熹《中庸章句》:見,音現。隱,暗處也。微,細事也。獨者,人所不知而己所獨知之地也。言幽暗之中,細微之事,跡雖未形而幾則已動,人雖不知而己獨知之,則是天下之事無有著見明顯而過於此者。是以君子既常戒懼,而於此尤加謹焉,所以遏人欲於將萌,而不使其滋長於隱微之中,以至離道之遠也。

\subparagraph{$\bullet$ 喜怒哀樂之未發,謂之中;發而皆中節,謂之和。中也者,天下之大本也;和也者,天下之達道也。} ~\\

朱熹《中庸章句》:樂,音洛。中節之中,去聲。喜、怒、哀、樂,情也。其未發,則性也,無所偏倚,故謂之中。發皆中節,情之正也,無所乖戾,故謂之和。大本者,天命之性,天下之理皆由此出,道之體也。達道者,循性之謂,天下古今之所共由,道之用也。此言性情之德,以明道不可離之意。

\subparagraph{$\bullet$ 致中和,天地位焉,萬物育焉。} ~\\

朱熹《中庸章句》:

\begin{adjustwidth}{2cm}{1cm}
\indent\indent 致,推而極之也。位者,安其所也。育者,遂其生也。自戒懼而約之,以至於至靜之中,無少偏倚,而其守不失,則極其中而天地位矣。自謹獨而精之,以至於應物之處,無少差謬,而無適不然,則極其和而萬物育矣。蓋天地萬物本吾一體,吾之心正,則天地之心亦正矣,吾之氣順,則天地之氣亦順矣。故其效驗至於如此。此學問之極功、聖人之能事,初非有待於外,而修道之教亦在其中矣。是其一體一用雖有動靜之殊,然必其體立而後用有以行,則其實亦非有兩事也。故於此合而言之,以結上文之意。

第一章,子思述所傳之意以立言。首明道之本原出於天而不可易,其實體備於己而不可離,次言存養省察之要,終言聖神功化之極。蓋欲學者於此反求諸身而自得之,以去夫外誘之私,而充其本然之善,楊氏所謂一篇之體要是也。其下十章,蓋子思引夫子之言,以終此章之義。
\end{adjustwidth}

\newpage
\subsection{第二章}

\subparagraph{$\bullet$ 仲尼曰:「君子中庸,小人反中庸。} ~\\

朱熹《中庸章句》:中庸者,不偏不倚、無過不及,而平常之理,乃天命所當然,精微之極致也。惟君子為能體之,小人反是。

\subparagraph{$\bullet$ 君子之中庸也,君子而時中;小人之中庸也,小人而無忌憚也。」} ~\\

朱熹《中庸章句》:

\begin{adjustwidth}{2cm}{1cm}
\indent\indent 王肅本作「小人之反中庸也」,程子亦以為然。今從之。君子之所以為中庸者,以其有君子之德,而又能隨時以處中也。小人之所以反中庸者,以其有小人之心,而又無所忌憚也。蓋中無定體,隨時而在,是乃平常之理也。君子知其在我,故能戒謹不睹、恐懼不聞,而無時不中。小人不知有此,則肆欲妄行,而無所忌憚矣。

第二章,此下十章,皆論中庸以釋首章之義。文雖不屬,而意實相承也。變和言庸者,游氏曰:「以性情言之,則曰中和,以德行言之,則曰中庸是也。」然中庸之中,實兼中和之義。
\end{adjustwidth}

\newpage
\subsection{第三章}

\subparagraph{$\bullet$ 子曰:「中庸其至矣乎!民鮮能久矣!」} ~\\

朱熹《中庸章句》:鮮,上聲。下同。過則失中,不及則未至,故惟中庸之德為至。然亦人所同得,初無難事,但世教衰,民不興行,故鮮能之,今已久矣。論語無能字。

\newpage
\subsection{第四章}

\subparagraph{$\bullet$ 子曰:「道之不行也,我知之矣,知者過之,愚者不及也;道之不明也,我知之矣,賢者過之,不肖者不及也。} ~\\

朱熹《中庸章句》:知者之知,去聲。道者,天理之當然,中而已矣。知愚賢不肖之過不及,則生稟之異而失其中也。知者知之過,既以道為不足行;愚者不及知,又不知所以行,此道之所以常不行也。賢者行之過,既以道為不足知;不肖者不及行,又不求所以知,此道之所以常不明也。人莫不飲食也,鮮能知味也。道不可離,人自不察,是以有過不及之弊。

\newpage
\subsection{第五章}

\subparagraph{$\bullet$ 子曰:「道其不行矣夫!」} ~\\

朱熹《中庸章句》:夫,音扶。由不明,故不行。此章承上章而舉其不行之端,以起下章之意。

\newpage
\subsection{第六章}

\subparagraph{$\bullet$ 子曰:「舜其大知也與!舜好問而好察邇言,隱惡而揚善,執其兩端,用其中於民,其斯以為舜乎!」} ~\\

朱熹《中庸章句》:知,去聲。與,平聲。好,去聲。舜之所以為大知者,以其不自用而取諸人也。邇言者,淺近之言,猶必察焉,其無遺善可知。然於其言之未善者則隱而不宣,其善者則播而不匿,其廣大光明又如此,則人孰不樂告以善哉。兩端,謂眾論不同之極致。蓋凡物皆有兩端,如小大厚薄之類,於善之中又執其兩端,而量度以取中,然後用之,則其擇之審而行之至矣。然非在我之權度精切不差,何以與此。此知之所以無過不及,而道之所以行也。

\newpage
\subsection{第七章}

\subparagraph{$\bullet$ 子曰:「人皆曰予知,驅而納諸罟擭陷阱之中,而莫之知辟也。人皆曰予知,擇乎中庸而不能期月守也。」} ~\\

朱熹《中庸章句》:予知之知,去聲。罟,音古。擭,胡化反。阱,才性反。辟,避同。期,居之反。罟,網也;擭,機檻也;陷阱,坑坎也;皆所以掩取禽獸者也。擇乎中庸,辨別眾理,以求所謂中庸,即上章好問用中之事也。期月,匝一月也。言知禍而不知辟,以況能擇而不能守,皆不得為知也。承上章大知而言,又舉不明之端,以起下章也。

\newpage
\subsection{第八章}

\subparagraph{$\bullet$ 子曰:「回之為人也,擇乎中庸,得一善,則拳拳服膺而弗失之矣。」} ~\\

朱熹《中庸章句》:回,孔子弟子顏淵名。拳拳,奉持之貌。服,猶著也。膺,胸也。奉持而著之心胸之間,言能守也。顏子蓋真知之,故能擇能守如此,此行之所以無過不及,而道之所以明也。

\newpage
\subsection{第九章}

\subparagraph{$\bullet$ 子曰:「天下國家可均也,爵祿可辭也,白刃可蹈也,中庸不可能也。」} ~\\

朱熹《中庸章句》:均,平治也。三者亦知仁勇之事,天下之至難也,然不必其合於中庸,則質之近似者皆能以力為之。若中庸,則雖不必皆如三者之難,然非義精仁熟,而無一毫人欲之私者,不能及也。三者難而易,中庸易而難,此民之所以鮮能也。亦承上章以起下章,

\newpage
\subsection{第十章}

\subparagraph{$\bullet$ 子路問強。} ~\\

朱熹《中庸章句》:子路,孔子弟子仲由也。子路好勇,故問強。

\subparagraph{$\bullet$ 子曰:「南方之強與?北方之強與?抑而強與?} ~\\

朱熹《中庸章句》:與,平聲。抑,語辭。而,汝也。

\subparagraph{$\bullet$ 寬柔以教,不報無道,南方之強也,君子居之。} ~\\

朱熹《中庸章句》:寬柔以教,謂含容巽順以誨人之不及也。不報無道,謂橫逆之來,直受之而不報也。南方風氣柔弱,故以含忍之力勝人為強,君子之道也。

\subparagraph{$\bullet$ 衽金革,死而不厭,北方之強也,而強者居之。} ~\\

朱熹《中庸章句》:衽,席也。金,戈兵之屬。革,甲冑之屬。北方風氣剛勁,故以果敢之力勝人為強,強者之事也。

\subparagraph{$\bullet$ 故君子和而不流,強哉矯!中立而不倚,強哉矯!國有道,不變塞焉,強哉矯!國無道,至死不變,強哉矯!」} ~\\

朱熹《中庸章句》:此四者,汝之所當強也。矯,強貌。詩曰「矯矯虎臣」是也。倚,偏著也。塞,未達也。國有道,不變未達之所守;國無道,不變平生之所守也。此則所謂中庸之不可能者,非有以自勝其人欲之私,不能擇而守也。君子之強,孰大於是。夫子以是告子路者,所以抑其血氣之剛,而進之以德義之勇也。

\newpage
\subsection{第十一章}

\subparagraph{$\bullet$ 子曰:「素隱行怪,後世有述焉,吾弗為之矣。} ~\\

朱熹《中庸章句》:素,按漢書當作索,蓋字之誤也。索隱行怪,言深求隱僻之理,而過為詭異之行也。然以其足以欺世而盜名,故後世或有稱述之者。此知之過而不擇乎善,行之過而不用其中,不當強而強者也,聖人豈為之哉!

\subparagraph{$\bullet$ 君子遵道而行,半塗而廢,吾弗能已矣。} ~\\

朱熹《中庸章句》:遵道而行,則能擇乎善矣;半塗而廢,則力之不足也。此其知雖足以及之,而行有不逮,當強而不強者也。已,止也。聖人於此,非勉焉而不敢廢,蓋至誠無息,自有所不能止也。

\subparagraph{$\bullet$ 君子依乎中庸,遯世不見知而不悔,唯聖者能之。} ~\\

朱熹《中庸章句》:

\begin{adjustwidth}{2cm}{1cm}
\indent\indent 不為索隱行怪,則依乎中庸而已。不能半塗而廢,是以遯世不見知而不悔也。此中庸之成德,知之盡、仁之至、不賴勇而裕如者,正吾夫子之事,而猶不自居也。故曰唯聖者能之而已。

子思所引夫子之言,以明首章之義者止此。蓋此篇大旨,以知仁勇三達德為入道之門。故於篇首,即以大舜、顏淵、子路之事明之。舜,知也;顏淵,仁也;子路,勇也:三者廢其一,則無以造道而成德矣。餘見第二十章,
\end{adjustwidth}

\newpage
\subsection{第十二章}

\subparagraph{$\bullet$ 君子之道費而隱。} ~\\

朱熹《中庸章句》:費,符味反。費,用之廣也。隱,體之微也。

\subparagraph{$\bullet$ 夫婦之愚,可以與知焉,及其至也,雖聖人亦有所不知焉;夫婦之不肖,可以能行焉,及其至也,雖聖人亦有所不能焉。天地之大也,人猶有所憾。故君子語大,天下莫能載焉;語小,天下莫能破焉。} ~\\

朱熹《中庸章句》:與,去聲。君子之道,近自夫婦居室之間,遠而至於聖人天地之所不能盡,其大無外,其小無內,可謂費矣。然其理之所以然,則隱而莫之見也。蓋可知可能者,道中之一事,及其至而聖人不知不能。則舉全體而言,聖人固有所不能盡也。侯氏曰:「聖人所不知,如孔子問禮問官之類;所不能,如孔子不得位、堯舜病博施之類。」愚謂人所憾於天地,如覆載生成之偏,及寒暑災祥之不得其正者。

\subparagraph{$\bullet$ 詩云:「鳶飛戾天,魚躍于淵。」言其上下察也。} ~\\

朱熹《中庸章句》:鳶,余專反。詩大雅旱麓之篇。鳶,鴟類。戾,至也。察,著也。子思引此詩以明化育流行,上下昭著,莫非此理之用,所謂費也。然其所以然者,則非見聞所及,所謂隱也。故程子曰:「此一節,子思喫緊為人處,活潑潑地,讀者其致思焉。」

\subparagraph{$\bullet$ 君子之道,造端乎夫婦;及其至也,察乎天地。} ~\\

朱熹《中庸章句》:結上文。子思之言,蓋以申明首章道不可離之意也。其下八章,雜引孔子之言以明之。

\newpage
\subsection{第十三章}

\subparagraph{$\bullet$ 子曰:「道不遠人。人之為道而遠人,不可以為道。} ~\\

朱熹《中庸章句》:道者,率性而已,固眾人之所能知能行者也,故常不遠於人。若為道者,厭其卑近以為不足為,而反務為高遠難行之事,則非所以為道矣!

\subparagraph{$\bullet$ 詩云:『伐柯伐柯,其則不遠。』執柯以伐柯,睨而視之,猶以為遠。故君子以人治人,改而止。} ~\\

朱熹《中庸章句》:詩豳風伐柯之篇。柯,斧柄。則,法也。睨,邪視也。言人執柯伐木以為柯者,彼柯長短之法,在此柯耳。然猶有彼此之別,故伐者視之猶以為遠也。若以人治人,則所以為人之道,各在當人之身,初無彼此之別。故君子之治人也,即以其人之道,還治其人之身。其人能改,即止不治。蓋責之以其所能知能行,非欲其遠人以為道也。張子所謂「以眾人望人則易從」是也。

\subparagraph{$\bullet$ 忠恕違道不遠,施諸己而不願,亦勿施於人。} ~\\

朱熹《中庸章句》:盡己之心為忠,推己及人為恕。違,去也,如春秋傳「齊師違穀七里」之違。言自此至彼,相去不遠,非背而去之之謂也。道,即其不遠人者是也。施諸己而不願亦勿施於人,忠恕之事也。以己之心度人之心,未嘗不同,則道之不遠於人者可見。故己之所不欲,則勿以施之於人,亦不遠人以為道之事。張子所謂「以愛己之心愛人則盡仁」是也。

\subparagraph{$\bullet$ 君子之道四,丘未能一焉:所求乎子,以事父未能也;所求乎臣,以事君未能也;所求乎弟,以事兄未能也;所求乎朋友,先施之未能也。庸德之行,庸言之謹,有所不足,不敢不勉,有餘不敢盡;言顧行,行顧言,君子胡不慥慥爾!」} ~\\

朱熹《中庸章句》:

\begin{adjustwidth}{2cm}{1cm}
\indent\indent 子、臣、弟、友,四字絕句。求,猶責也。道不遠人,凡己之所以責人者,皆道之所當然也,故反之以自責而自修焉。庸,平常也。行者,踐其實。謹者,擇其可。德不足而勉,則行益力;言有餘而訒,則謹益至。謹之至則言顧行矣;行之力則行顧言矣。慥慥,篤實貌。言君子之言行如此,豈不慥慥乎,贊美之也。凡此皆不遠人以為道之事。張子所謂「以責人之心責己則盡道」是也。

第十三章,道不遠人者,夫婦所能,丘未能一者,聖人所不能,皆費也。而其所以然者,則至隱存焉。下章放此。
\end{adjustwidth}

\newpage
\subsection{第十四章}

\subparagraph{$\bullet$ 君子素其位而行,不願乎其外。} ~\\

朱熹《中庸章句》:素,猶見在也。言君子但因見在所居之位而為其所當為,無慕乎其外之心也。

\subparagraph{$\bullet$ 素富貴,行乎富貴;素貧賤,行乎貧賤;素夷狄,行乎夷狄;素患難,行乎患難;君子無入而不自得焉。} ~\\

朱熹《中庸章句》:難,去聲。此言素其位而行也。

\subparagraph{$\bullet$ 在上位不陵下,在下位不援上,正己而不求於人則無怨。上不怨天,下不尤人。} ~\\

朱熹《中庸章句》:援,平聲。此言不願乎其外也。

\subparagraph{$\bullet$ 故君子居易以俟命,小人行險以徼幸。} ~\\

朱熹《中庸章句》:易,去聲。易,平地也。居易,素位而行也。俟命,不願乎外也。徼,求也。幸,謂所不當得而得者。

\subparagraph{$\bullet$ 子曰:「射有似乎君子;失諸正鵠,反求諸其身。」} ~\\

朱熹《中庸章句》:正,音征。畫布曰正,棲皮曰鵠,皆侯之中,射之的也。子思引此孔子之言,以結上文之意。第十四章,子思之言也。凡章首無「子曰」字者放此。

\newpage
\subsection{第十五章}

\subparagraph{$\bullet$ 君子之道,辟如行遠必自邇,辟如登高必自卑。} ~\\

朱熹《中庸章句》:辟、譬同。

\subparagraph{$\bullet$ 詩曰:「妻子好合,如鼓瑟琴;兄弟既翕,和樂且耽;宜爾室家;樂爾妻帑。」} ~\\

朱熹《中庸章句》:好,去聲。耽,詩作湛,亦音耽。樂,音洛。詩小雅常棣之篇。鼓瑟琴,和也。翕,亦合也。耽,亦樂也。帑,子孫也。

\subparagraph{$\bullet$ 子曰:「父母其順矣乎!」} ~\\

朱熹《中庸章句》:夫子誦此詩而贊之曰:人能和於妻子,宜於兄弟如此,則父母其安樂之矣。子思引詩及此語,以明行遠自邇、登高自卑之意。

\newpage
\subsection{第十六章}

\subparagraph{$\bullet$ 子曰:「鬼神之為德,其盛矣乎!} ~\\

朱熹《中庸章句》:程子曰:「鬼神,天地之功用,而造化之跡也。」張子曰:「鬼神者,二氣之良能也。」愚謂以二氣言,則鬼者陰之靈也,神者陽之靈也。以一氣言,則至而伸者為神,反而歸者為鬼,其實一物而已。為德,猶言性情功效。

\subparagraph{$\bullet$ 視之而弗見,聽之而弗聞,體物而不可遺。} ~\\

朱熹《中庸章句》:鬼神無形與聲,然物之終始,莫非陰陽合散之所為,是其為物之體,而物所不能遺也。其言體物,猶易所謂幹事。

\subparagraph{$\bullet$ 使天下之人齊明盛服,以承祭祀。洋洋乎!如在其上,如在其左右。} ~\\

朱熹《中庸章句》:齊之為言齊也,所以齊不齊而致其齊也。明,猶潔也。洋洋,流動充滿之意。能使人畏敬奉承,而發見昭著如此,乃其體物而不可遺之驗也。孔子曰:「其氣發揚于上,為昭明焄蒿悽愴。此百物之精也,神之著也」,正謂此爾。

\subparagraph{$\bullet$ 詩曰:『神之格思,不可度思!矧可射思!』} ~\\

朱熹《中庸章句》:射,音亦,詩作斁。詩大雅抑之篇。格,來也。矧,況也。射,厭也,言厭怠而不敬也。思,語辭。

\subparagraph{$\bullet$ 夫微之顯,誠之不可揜如此夫。」} ~\\

朱熹《中庸章句》:

\begin{adjustwidth}{2cm}{1cm}
\indent\indent 夫,音扶。誠者,真實無妄之謂。陰陽合散,無非實者。故其發見之不可揜如此。

第十六章,不見不聞,隱也。體物如在,則亦費矣。此前三章,以其費之小者而言。此後三章,以其費之大者而言。此一章,兼費隱、包大小而言。
\end{adjustwidth}

\newpage
\subsection{第十七章}

\subparagraph{$\bullet$ 子曰:「舜其大孝也與!德為聖人,尊為天子,富有四海之內。宗廟饗之,子孫保之。} ~\\

朱熹《中庸章句》:與,平聲。子孫,謂虞思、陳胡公之屬。

\subparagraph{$\bullet$ 故大德必得其位,必得其祿,必得其名,必得其壽。} ~\\

朱熹《中庸章句》:舜年百有十歲。

\subparagraph{$\bullet$ 故天之生物,必因其材而篤焉。故栽者培之,傾者覆之。} ~\\

朱熹《中庸章句》:材,質也。篤,厚也。栽,植也。氣至而滋息為培。氣反而游散則覆。

\subparagraph{$\bullet$ 詩曰:『嘉樂君子,憲憲令德!宜民宜人;受祿于天;保佑命之,自天申之!』} ~\\

朱熹《中庸章句》:詩大雅假樂之篇。假,當依此作嘉。憲,當依詩作顯。申,重也。

\subparagraph{$\bullet$ 故大德者必受命。」} ~\\

朱熹《中庸章句》:受命者,受天命為天子也。第十七章,此由庸行之常,推之以極其至,見道之用廣也。而其所以然者,則為體微矣。後二章亦此意。

\newpage
\subsection{第十八章}

\subparagraph{$\bullet$ 子曰:「無憂者其惟文王乎!以王季為父,以武王為子,父作之,子述之。} ~\\

朱熹《中庸章句》:此言文王之事。書言「王季其勤王家」,蓋其所作,亦積功累仁之事也。

\subparagraph{$\bullet$ 武王纘大王、王季、文王之緒。壹戎衣而有天下,身不失天下之顯名。尊為天子,富有四海之內。宗廟饗之,子孫保之。} ~\\

朱熹《中庸章句》:大,音泰,下同。此言武王之事。纘,繼也。大王,王季之父也。書云:「大王肇基王跡。」詩云「至于大王,實始翦商。」緒,業也。戎衣,甲冑之屬。壹戎衣,武成文,言一著戎衣以伐紂也。

\subparagraph{$\bullet$ 武王末受命,周公成文武之德,追王大王、王季,上祀先公以天子之禮。斯禮也,達乎諸侯大夫,及士庶人。父為大夫,子為士;葬以大夫,祭以士。父為士,子為大夫;葬以士,祭以大夫。期之喪達乎大夫,三年之喪達乎天子,父母之喪無貴賤一也。」} ~\\

朱熹《中庸章句》:追王之王,去聲。此言周公之事。末,猶老也。追王,蓋推文武之意,以及乎王跡之所起也。先公,組紺以上至后稷也。上祀先公以天子之禮,又推大王、王季之意,以及於無窮也。制為禮法,以及天下,使葬用死者之爵,祭用生者之祿。喪服自期以下,諸侯絕;大夫降;而父母之喪,上下同之,推己以及人也。

\newpage
\subsection{第十九章}

\subparagraph{$\bullet$ 子曰:「武王、周公,其達孝矣乎!} ~\\

朱熹《中庸章句》:達,通也。承上章而言武王、周公之孝,乃天下之人通謂之孝,猶孟子之言達尊也。

\subparagraph{$\bullet$ 夫孝者:善繼人之志,善述人之事者也。} ~\\

朱熹《中庸章句》:上章言武王纘大王、王季、文王之緒以有天下,而周公成文武之德以追崇其先祖,此繼志述事之大者也。下文又以其所制祭祀之禮,通於上下者言之。

\subparagraph{$\bullet$ 春秋修其祖廟,陳其宗器,設其裳衣,薦其時食。} ~\\

朱熹《中庸章句》:祖廟:天子七,諸侯五,大夫三,適士二,官師一。宗器,先世所藏之重器;若周之赤刀、大訓、天球、河圖之屬也。裳衣,先祖之遺衣服,祭則設之以授尸也。時食,四時之食,各有其物,如春行羔、豚、膳、膏、香之類是也。

\subparagraph{$\bullet$ 宗廟之禮,所以序昭穆也;序爵,所以辨貴賤也;序事,所以辨賢也;旅酬下為上,所以逮賤也;燕毛,所以序齒也。} ~\\

朱熹《中庸章句》:昭,如字。為,去聲。宗廟之次:左為昭,右為穆,而子孫亦以為序。有事於太廟,則子姓、兄弟、群昭、群穆咸在而不失其倫焉。爵,公、侯、卿、大夫也。事,宗祝有司之職事也。旅,眾也。酬,導飲也。旅酬之禮,賓弟子、兄弟之子各舉觶於其長而眾相酬。蓋宗廟之中以有事為榮,故逮及賤者,使亦得以申其敬也。燕毛,祭畢而燕,則以毛髮之色別長幼,為坐次也。齒,年數也。

\subparagraph{$\bullet$ 踐其位,行其禮,奏其樂,敬其所尊,愛其所親,事死如事生,事亡如事存,孝之至也。} ~\\

朱熹《中庸章句》:踐,猶履也。其,指先王也。所尊所親,先王之祖考、子孫、臣庶也。始死謂之死,既葬則曰反而亡焉,皆指先王也。此結上文兩節,皆繼志述事之意也。

\subparagraph{$\bullet$ 郊社之禮,所以事上帝也,宗廟之禮,所以祀乎其先也。明乎郊社之禮、禘嘗之義,治國其如示諸掌乎。」} ~\\

朱熹《中庸章句》:郊,祀天。社,祭地。不言后土者,省文也。禘,天子宗廟之大祭,追祭太祖之所自出於太廟,而以太祖配之也。嘗,秋祭也。四時皆祭,舉其一耳。禮必有義,對舉之,互文也。示,與視同。視諸掌,言易見也。此與論語文意大同小異,記有詳略耳。

\newpage
\subsection{第二十章}

\subparagraph{$\bullet$ 哀公問政。} ~\\

朱熹《中庸章句》:哀公,魯君,名蔣。

\subparagraph{$\bullet$ 子曰:「文武之政,布在方策。其人存,則其政舉;其人亡,則其政息。} ~\\

朱熹《中庸章句》:方,版也。策,簡也。息,猶滅也。有是君,有是臣,則有是政矣。

\subparagraph{$\bullet$ 人道敏政,地道敏樹。夫政也者,蒲盧也。} ~\\

朱熹《中庸章句》:夫,音扶。敏,速也。蒲盧,沈括以為蒲葦是也。以人立政,猶以地種樹,其成速矣,而蒲葦又易生之物,其成尤速也。言人存政舉,其易如此。

\subparagraph{$\bullet$ 故為政在人,取人以身,脩身以道,脩道以仁。} ~\\

朱熹《中庸章句》:此承上文人道敏政而言也。為政在人,家語作「為政在於得人」,語意尤備。人,謂賢臣。身,指君身。道者,天下之達道。仁者,天地生物之心,而人得以生者,所謂元者善之長也。言人君為政在於得人,而取人之則又在脩身。能仁其身,則有君有臣,而政無不舉矣。

\subparagraph{$\bullet$ 仁者人也,親親為大;義者宜也,尊賢為大;親親之殺,尊賢之等,禮所生也。} ~\\

朱熹《中庸章句》:殺,去聲。人,指人身而言。具此生理,自然便有惻怛慈愛之意,深體味之可見。宜者,分別事理,各有所宜也。禮,則節文斯二者而已。

\subparagraph{$\bullet$ 在下位不獲乎上,民不可得而治矣!} ~\\

朱熹《中庸章句》:鄭氏曰:「此句在下,誤重在此。」

\subparagraph{$\bullet$ 故君子不可以不脩身;思脩身,不可以不事親;思事親,不可以不知人;思知人,不可以不知天。」} ~\\

朱熹《中庸章句》:為政在人,取人以身,故不可以不脩身。脩身以道,脩道以仁,故思脩身不可以不事親。欲盡親親之仁,必由尊賢之義,故又當知人。親親之殺,尊賢之等,皆天理也,故又當知天。

\subparagraph{$\bullet$ 天下之達道五,所以行之者三:曰君臣也,父子也,夫婦也,昆弟也,朋友之交也:五者天下之達道也。知、仁、勇三者,天下之達德也,所以行之者一也。} ~\\

朱熹《中庸章句》:知,去聲。達道者,天下古今所共由之路,即書所謂五典,孟子所謂「父子有親、君臣有義、夫婦有別、長幼有序、朋友有信」是也。知,所以知此也;仁,所以體此也;勇,所以強此也;謂之達德者,天下古今所同得之理也。一則誠而已矣。達道雖人所共由,然無是三德,則無以行之;達德雖人所同得,然一有不誠,則人欲間之,而德非其德矣。程子曰:「所謂誠者,止是誠實此三者。三者之外,更別無誠。」

\subparagraph{$\bullet$ 或生而知之,或學而知之,或困而知之,及其知之一也;或安而行之,或利而行之,或勉強而行之,及其成功一也。} ~\\

朱熹《中庸章句》:強,上聲。知之者之所知,行之者之所行,謂達道也。以其分而言:則所以知者知也,所以行者仁也,所以至於知之成功而一者勇也。以其等而言:則生知安行者知也,學知利行者仁也,困知勉行者勇也。蓋人性雖無不善,而氣稟有不同者,故聞道有蚤莫,行道有難易,然能自強不息,則其至一也。呂氏曰:「所入之塗雖異,而所至之域則同,此所以為中庸。若乃企生知安行之資為不可幾及,輕困知勉行謂不能有成,此道之所以不明不行也。」

\subparagraph{$\bullet$ 子曰:「好學近乎知,力行近乎仁,知恥近乎勇。} ~\\

朱熹《中庸章句》:『子曰』二字衍文。好近乎知之知,並去聲。此言未及乎達德而求以入德之事。通上文三知為知,三行為仁,則此三近者,勇之次也。呂氏曰:「愚者自是而不求,自私者殉人欲而忘反,懦者甘為人下而不辭。故好學非知,然足以破愚;力行非仁,然足以忘私;知恥非勇,然足以起懦。」

\subparagraph{$\bullet$ 知斯三者,則知所以脩身;知所以脩身,則知所以治人;知所以治人,則知所以治天下國家矣。」} ~\\

朱熹《中庸章句》:斯三者,指三近而言。人者,對己之稱。天下國家,則盡乎人矣。言此以結上文脩身之意,起下文九經之端也。

\subparagraph{$\bullet$ 凡為天下國家有九經,曰:脩身也,尊賢也,親親也,敬大臣也,體群臣也,子庶民也,來百工也,柔遠人也,懷諸侯也。} ~\\

朱熹《中庸章句》:經,常也。體,謂設以身處其地而察其心也。子,如父母之愛其子也。柔遠人,所謂無忘賓旅者也。此列九經之目也。呂氏曰:「天下國家之本在身,故脩身為九經之本。然必親師取友,然後脩身之道進,故尊賢次之。道之所進,莫先其家,故親親次之。由家以及朝廷,故敬大臣、體群臣次之。由朝廷以及其國,故子庶民、來百工次之。由其國以及天下,故柔遠人、懷諸侯次之。此九經之序也。」視群臣猶吾四體,視百姓猶吾子,此視臣視民之別也。

\subparagraph{$\bullet$ 脩身則道立,尊賢則不惑,親親則諸父昆弟不怨,敬大臣則不眩,體群臣則士之報禮重,子庶民則百姓勸,來百工則財用足,柔遠人則四方歸之,懷諸侯則天下畏之。} ~\\

朱熹《中庸章句》:此言九經之效也。道立,謂道成於己而可為民表,所謂皇建其有極是也。不惑,謂不疑於理。不眩,謂不迷於事。敬大臣則信任專,而小臣不得以間之,故臨事而不眩也。來百工則通功易事,農末相資,故財用足。柔遠人,則天下之旅皆悅而願出於其塗,故四方歸。懷諸侯,則德之所施者博,而威之所制者廣矣,故曰天下畏之。

\subparagraph{$\bullet$ 齊明盛服,非禮不動,所以脩身也;去讒遠色,賤貨而貴德,所以勸賢也;尊其位,重其祿,同其好惡,所以勸親親也;官盛任使,所以勸大臣也;忠信重祿,所以勸士也;時使薄斂,所以勸百姓也;日省月試,既稟稱事,所以勸百工也;送往迎來,嘉善而矜不能,所以柔遠人也;繼絕世,舉廢國,治亂持危,朝聘以時,厚往而薄來,所以懷諸侯也。} ~\\

朱熹《中庸章句》:去,上聲。遠、好、惡、斂,並去聲。既,許氣反。稟,彼錦、力錦二反。稱,去聲。朝,音潮。此言九經之事也。官盛任使,謂官屬眾盛,足任使令也,蓋大臣不當親細事,故所以優之者如此。忠信重祿,謂待之誠而養之厚,蓋以身體之,而知其所賴乎上者如此也。既,讀曰餼。餼稟,稍食也。稱事,如周禮稿人職,曰「考其弓弩,以上下其食」是也。往則為之授節以送之,來則豐其委積以迎之。朝,謂諸侯見於天子。聘,謂諸侯使大夫來獻。王制「比年一小聘,三年一大聘,五年一朝」。厚往薄來,謂燕賜厚而納貢薄。

\subparagraph{$\bullet$ 凡為天下國家有九經,所以行之者一也。} ~\\

朱熹《中庸章句》:一者,誠也。一有不誠,則是九者皆為虛文矣,此九經之實也。

\subparagraph{$\bullet$ 凡事豫則立,不豫則廢。言前定則不跲,事前定則不困,行前定則不疚,道前定則不窮。} ~\\

朱熹《中庸章句》:行,去聲。凡事,指達道達德九經之屬。豫,素定也。跲,躓也。疚,病也。此承上文,言凡事皆欲先立乎誠,如下文所推是也。

\subparagraph{$\bullet$ 在下位不獲乎上,民不可得而治矣;獲乎上有道:不信乎朋友,不獲乎上矣;信乎朋友有道:不順乎親,不信乎朋友矣;順乎親有道:反諸身不誠,不順乎親矣;誠身有道:不明乎善,不誠乎身矣。} ~\\

朱熹《中庸章句》:此又以在下位者,推言素定之意。反諸身不誠,謂反求諸身而所存所發,未能真實而無妄也。不明乎善,謂未能察於人心天命之本然,而真知至善之所在也。

\subparagraph{$\bullet$ 誠者,天之道也;誠之者,人之道也。誠者不勉而中,不思而得,從容中道,聖人也。誠之者,擇善而固執之者也。} ~\\

朱熹《中庸章句》:中,並去聲。此承上文誠身而言。誠者,真實無妄之謂,天理之本然也。誠之者,未能真實無妄,而欲其真實無妄之謂,人事之當然也。聖人之德,渾然天理,真實無妄,不待思勉而從容中道,則亦天之道也。未至於聖,則不能無人欲之私,而其為德不能皆實。故未能不思而得,則必擇善,然後可以明善;未能不勉而中,則必固執,然後可以誠身,此則所謂人之道也。不思而得,生知也。不勉而中,安行也。擇善,學知以下之事。固執,利行以下之事也。

\subparagraph{$\bullet$ 博學之,審問之,慎思之,明辨之,篤行之。} ~\\

朱熹《中庸章句》:此誠之之目也。學、問、思、辨,所以擇善而為知,學而知也。篤行,所以固執而為仁,利而行也。程子曰:「五者廢其一,非學也。」

\subparagraph{$\bullet$ 有弗學,學之弗能弗措也;有弗問,問之弗知弗措也;有弗思,思之弗得弗措也;有弗辨,辨之弗明弗措也;有弗行,行之弗篤弗措也;人一能之己百之,人十能之己千之。} ~\\

朱熹《中庸章句》:君子之學,不為則已,為則必要其成,故常百倍其功。此困而知,勉而行者也,勇之事也。

\subparagraph{$\bullet$ 果能此道矣,雖愚必明,雖柔必強。} ~\\

朱熹《中庸章句》:

\begin{adjustwidth}{2cm}{1cm}
\indent\indent 明者擇善之功,強者固執之效。呂氏曰:「君子所以學者,為能變化氣質而已。德勝氣質,則愚者可進於明,柔者可進於強。不能勝之,則雖有志於學,亦愚不能明,柔不能立而已矣。蓋均善而無惡者,性也,人所同也;昏明強弱之稟不齊者,才也,人所異也。誠之者所以反其同而變其異也。夫以不美之質,求變而美,非百倍其功,不足以致之。今以鹵莽滅裂之學,或作或輟,以變其不美之質,及不能變,則曰天質不美,非學所能變。是果於自棄,其為不仁甚矣!」

第二十章,此引孔子之言,以繼大舜、文、武、周公之緒,明其所傳之一致,舉而措之,亦猶是耳。蓋包費隱、兼小大,以終十二章之意。章內語誠始詳,而所謂誠者,實此篇之樞紐也。又按:孔子家語,亦載此章,而其文尤詳。「成功一也」之下,有「公曰:子之言美矣!至矣!寡人實固,不足以成之也」。故其下復以「子曰」起答辭。今無此問辭,而猶有「子曰」二字;蓋子思刪其繁文以附于篇,而所刪有不盡者,今當為衍文也。「博學之」以下,家語無之,意彼有闕文,抑此或子思所補也歟?
\end{adjustwidth}

\newpage
\subsection{第二十一章}

\subparagraph{$\bullet$ 自誠明,謂之性;自明誠,謂之教。誠則明矣,明則誠矣。} ~\\

朱熹《中庸章句》:

\begin{adjustwidth}{2cm}{1cm}
\indent\indent 自,由也。德無不實而明無不照者,聖人之德。所性而有者也,天道也。先明乎善,而後能實其善者,賢人之學。由教而入者也,人道也。誠則無不明矣,明則可以至於誠矣。

第二十一章,子思承上章夫子天道、人道之意而立言也。自此以下十二章,皆子思之言,以反覆推明此章之意。
\end{adjustwidth}

\newpage
\subsection{第二十二章}

\subparagraph{$\bullet$ 唯天下至誠,為能盡其性;能盡其性,則能盡人之性;能盡人之性,則能盡物之性;能盡物之性,則可以贊天地之化育;可以贊天地之化育,則可以與天地參矣。} ~\\

朱熹《中庸章句》:天下至誠,謂聖人之德之實,天下莫能加也。盡其性者德無不實,故無人欲之私,而天命之在我者,察之由之,巨細精粗,無毫髮之不盡也。人物之性,亦我之性,但以所賦形氣不同而有異耳。能盡之者,謂知之無不明而處之無不當也。贊,猶助也。與天地參,謂與天地並立為三也。此自誠而明者之事也。第二十二章,言天道也。

\newpage
\subsection{第二十三章}

\subparagraph{$\bullet$ 其次致曲,曲能有誠,誠則形,形則著,著則明,明則動,動則變,變則化,唯天下至誠為能化。} ~\\

朱熹《中庸章句》:其次,通大賢以下凡誠有未至者而言也。致,推致也。曲,一偏也。形者,積中而發外。著,則又加顯矣。明,則又有光輝發越之盛也。動者,誠能動物。變者,物從而變。化,則有不知其所以然者。蓋人之性無不同,而氣則有異,故惟聖人能舉其性之全體而盡之。其次則必自其善端發見之偏,而悉推致之,以各造其極也。曲無不致,則德無不實,而形、著、動、變之功自不能已。積而至於能化,則其至誠之妙,亦不異於聖人矣。第二十三章,言人道也。

\newpage
\subsection{第二十四章}

\subparagraph{$\bullet$ 至誠之道,可以前知。國家將興,必有禎祥;國家將亡,必有妖孽;見乎蓍龜,動乎四體。禍福將至:善,必先知之;不善,必先知之。故至誠如神。} ~\\

朱熹《中庸章句》:見,音現。禎祥者,福之兆。妖孽者,禍之萌。蓍,所以筮。龜,所以卜。四體,謂動作威儀之閒,如執玉高卑,其容俯仰之類。凡此皆理之先見者也。然惟誠之至極,而無一毫私偽留於心目之間者,乃能有以察其幾焉。神,謂鬼神。第二十四章,言天道也。

\newpage
\subsection{第二十五章}

\subparagraph{$\bullet$ 誠者自成也,而道自道也。} ~\\

朱熹《中庸章句》:道也之道,音導。言誠者物之所以自成,而道者人之所當自行也。誠以心言,本也;道以理言,用也。

\subparagraph{$\bullet$ 誠者物之終始,不誠無物。是故君子誠之為貴。} ~\\

朱熹《中庸章句》:天下之物,皆實理之所為,故必得是理,然後有是物。所得之理既盡,則是物亦盡而無有矣。故人之心一有不實,則雖有所為亦如無有,而君子必以誠為貴也。蓋人之心能無不實,乃為有以自成,而道之在我者亦無不行矣。

\subparagraph{$\bullet$ 誠者非自成己而已也,所以成物也。成己,仁也;成物,知也。性之德也,合外內之道也,故時措之宜也。} ~\\

朱熹《中庸章句》:知,去聲。誠雖所以成己,然既有以自成,則自然及物,而道亦行於彼矣。仁者體之存,知者用之發,是皆吾性之固有,而無內外之殊。既得於己,則見於事者,以時措之,而皆得其宜也。第二十五章,言人道也。

\newpage
\subsection{第二十六章}

\subparagraph{$\bullet$ 故至誠無息。} ~\\

朱熹《中庸章句》:既無虛假,自無間斷。

\subparagraph{$\bullet$ 不息則久,久則徵。} ~\\

朱熹《中庸章句》:久,常於中也。徵,驗於外也。

\subparagraph{$\bullet$ 徵則悠遠,悠遠則博厚,博厚則高明。} ~\\

朱熹《中庸章句》:此皆以其驗於外者言之。鄭氏所謂『至誠之德,著於四方』者是也。存諸中者既久,則驗於外者益悠遠而無窮矣。悠遠,故其積也廣博而深厚;博厚,故其發也高大而光明。

\subparagraph{$\bullet$ 博厚,所以載物也;高明,所以覆物也;悠久,所以成物也。} ~\\

朱熹《中庸章句》:悠久,即悠遠,兼內外而言之也。本以悠遠致高厚,而高厚又悠久也。此言聖人與天地同用。

\subparagraph{$\bullet$ 博厚配地,高明配天,悠久無疆。} ~\\

朱熹《中庸章句》:此言聖人與天地同體。

\subparagraph{$\bullet$ 如此者,不見而章,不動而變,無為而成。} ~\\

朱熹《中庸章句》:見,音現。見,猶示也。不見而章,以配地而言也。不動而變,以配天而言也。無為而成,以無疆而言也。

\subparagraph{$\bullet$ 天地之道,可一言而盡也:其為物不貳,則其生物不測。} ~\\

朱熹《中庸章句》:此以下,復以天地明至誠無息之功用。天地之道,可一言而盡,不過曰誠而已。不貳,所以誠也。誠故不息,而生物之多,有莫知其所以然者。

\subparagraph{$\bullet$ 天地之道:博也,厚也,高也,明也,悠也,久也。} ~\\

朱熹《中庸章句》:言天地之道,誠一不貳,故能各極所盛,而有下文生物之功。今夫天,斯昭昭之多,及其無窮也,日月星辰繫焉,萬物覆焉。

\subparagraph{$\bullet$ 今夫地,一撮土之多,及其廣厚,載華嶽而不重,振河海而不洩,萬物載焉。今夫山,一卷石之多,及其廣大,草木生之,禽獸居之,寶藏興焉。今夫水,一勺之多,及其不測,黿鼉、蛟龍、魚鱉生焉,貨財殖焉。} ~\\

朱熹《中庸章句》:夫,音扶。華、藏,並去聲。卷,平聲。勺,市若反。昭昭,猶耿耿,小明也。此指其一處而言之。及其無窮,猶十二章及其至也之意,蓋舉全體而言也。振,收也。卷,區也。此四條,皆以發明由其不貳不息以致盛大而能生物之意。然天、地、山、川,實非由積累而後大,讀者不以辭害意可也。

\subparagraph{$\bullet$ 詩云:「維天之命,於穆不已!」蓋曰天之所以為天也。「於乎不顯!文王之德之純!」蓋曰文王之所以為文也,純亦不已。} ~\\

朱熹《中庸章句》:於,音烏。乎,音呼。詩周頌維天之命篇。於,歎辭。穆,深遠也。不顯,猶言豈不顯也。純,純一不雜也。引此以明至誠無息之意。程子曰:「天道不已,文王純於天道,亦不已。純則無二無雜,不已則無間斷先後。」第二十六章,言天道也。

\newpage
\subsection{第二十七章}

\subparagraph{$\bullet$ 大哉聖人之道!} ~\\

朱熹《中庸章句》:包下文兩節而言。

\subparagraph{$\bullet$ 洋洋乎!發育萬物,峻極于天。} ~\\

朱熹《中庸章句》:峻,高大也。此言道之極於至大而無外也。

\subparagraph{$\bullet$ 優優大哉!禮儀三百,威儀三千。} ~\\

朱熹《中庸章句》:優優,充足有餘之意。禮儀,經禮也。威儀,曲禮也。此言道之入於至小而無閒也。

\subparagraph{$\bullet$ 待其人而後行。} ~\\

朱熹《中庸章句》:總結上兩節。

\subparagraph{$\bullet$ 故曰苟不至德,至道不凝焉。} ~\\

朱熹《中庸章句》:至德,謂其人。至道,指上兩節而言也。凝,聚也,成也。

\subparagraph{$\bullet$ 故君子尊德性而道問學,致廣大而盡精微,極高明而道中庸。溫故而知新,敦厚以崇禮。} ~\\

朱熹《中庸章句》:尊者,恭敬奉持之意。德性者,吾所受於天之正理。道,由也。溫,猶燖溫之溫,謂故學之矣,復時習之也。敦,加厚也。尊德性,所以存心而極乎道體之大也。道問學,所以致知而盡乎道體之細也。二者修德凝道之大端也。不以一毫私意自蔽,不以一毫私欲自累,涵泳乎其所已知。敦篤乎其所已能,此皆存心之屬也。析理則不使有毫釐之差,處事則不使有過不及之謬,理義則日知其所未知,節文則日謹其所未謹,此皆致知之屬也。蓋非存心無以致知,而存心者又不可以不致知。故此五句,大小相資,首尾相應,聖賢所示入德之方,莫詳於此,學者宜盡心焉。

\subparagraph{$\bullet$ 是故居上不驕,為下不倍,國有道其言足以興,國無道其默足以容。詩曰「既明且哲,以保其身」,其此之謂與!} ~\\

朱熹《中庸章句》:倍,與背同。與,平聲。興,謂興起在位也。詩大雅烝民之篇。第二十七章,言人道也。

\newpage
\subsection{第二十八章}

\subparagraph{$\bullet$ 子曰:「愚而好自用,賤而好自專,生乎今之世,反古之道。如此者,烖及其身者也。」} ~\\

朱熹《中庸章句》:好,去聲。烖,古灾字。以上孔子之言,子思引之。反,復也。

\subparagraph{$\bullet$ 非天子,不議禮,不制度,不考文。} ~\\

朱熹《中庸章句》:此以下,子思之言。禮,親疏貴賤相接之體也。度,品制。文,書名。

\subparagraph{$\bullet$ 今天下車同軌,書同文,行同倫。} ~\\

朱熹《中庸章句》:行,去聲。今,子思自謂當時也。軌,轍跡之度。倫,次序之體。三者皆同,言天下一統也。

\subparagraph{$\bullet$ 雖有其位,苟無其德,不敢作禮樂焉;雖有其德,苟無其位,亦不敢作禮樂焉。} ~\\

朱熹《中庸章句》:鄭氏曰:「言作禮樂者,必聖人在天子之位。」

\subparagraph{$\bullet$ 子曰:「吾說夏禮,杞不足徵也;吾學殷禮,有宋存焉;吾學周禮,今用之,吾從周。」} ~\\

朱熹《中庸章句》:此又引孔子之言。杞,夏之後。徵,證也。宋,殷之後。三代之禮,孔子皆嘗學之而能言其意;但夏禮既不可考證,殷禮雖存,又非當世之法,惟周禮乃時王之制,今日所用。孔子既不得位,則從周而已。第二十八章,承上章為下不倍而言,亦人道也。

\newpage
\subsection{第二十九章}

\subparagraph{$\bullet$ 王天下有三重焉,其寡過矣乎!} ~\\

朱熹《中庸章句》:王,去聲。呂氏曰:「三重,謂議禮、制度、考文。惟天子得以行之,則國不異政,家不殊俗,而人得寡過矣。」

\subparagraph{$\bullet$ 上焉者雖善無徵,無徵不信,不信民弗從;下焉者雖善不尊,不尊不信,不信民弗從。} ~\\

朱熹《中庸章句》:上焉者,謂時王以前,如夏、商之禮雖善,而皆不可考。下焉者,謂聖人在下,如孔子雖善於禮,而不在尊位也。

\subparagraph{$\bullet$ 故君子之道:本諸身,徵諸庶民,考諸三王而不繆,建諸天地而不悖,質諸鬼神而無疑,百世以俟聖人而不惑。} ~\\

朱熹《中庸章句》:此君子,指王天下者而言。其道,即議禮、制度、考文之事也。本諸身,有其德也。徵諸庶民,驗其所信從也。建,立也,立於此而參於彼也。天地者,道也。

\subparagraph{$\bullet$ 鬼神者,造化之跡也。百世以俟聖人而不惑,所謂聖人復起,不易吾言者也。} ~\\

朱熹《中庸章句》:質諸鬼神而無疑,知天也;百世以俟聖人而不惑,知人也。知天知人,知其理也。

\subparagraph{$\bullet$ 是故君子動而世為天下道,行而世為天下法,言而世為天下則。遠之則有望,近之則不厭。} ~\\

朱熹《中庸章句》:動,兼言行而言。道,兼法則而言。法,法度也。則,準則也。

\subparagraph{$\bullet$ 詩曰:「在彼無惡,在此無射;庶幾夙夜,以永終譽!」君子未有不如此而蚤有譽於天下者也。} ~\\

朱熹《中庸章句》:惡,去聲。射,音妒,詩作斁。詩周頌振鷺之篇。射,厭也。所謂此者,指本諸身以下六事而言。第二十九章,承上章居上不驕而言,亦人道也。

\newpage
\subsection{第三十章}

\subparagraph{$\bullet$ 仲尼祖述堯舜,憲章文武;上律天時,下襲水土。} ~\\

朱熹《中庸章句》:祖述者,遠宗其道。憲章者,近守其法。律天時者,法其自然之運。襲水土者,因其一定之理。皆兼內外該本末而言也。

\subparagraph{$\bullet$ 辟如天地之無不持載,無不覆幬,辟如四時之錯行,如日月之代明。} ~\\

朱熹《中庸章句》:辟,音譬。幬,徒報反。錯,猶迭也。此言聖人之德。

\subparagraph{$\bullet$ 萬物並育而不相害,道並行而不相悖,小德川流,大德敦化,此天地之所以為大也。} ~\\

朱熹《中庸章句》:悖,猶背也。天覆地載,萬物並育於其間而不相害;四時日月,錯行代明而不相悖。所以不害不悖者,小德之川流;所以並育並行者,大德之敦化。小德者,全體之分;大德者,萬殊之本。川流者,如川之流,脈絡分明而往不息也。敦化者,敦厚其化,根本盛大而出無窮也。此言天地之道,以見上文取辟之意也。第三十章,言天道也。

\newpage
\subsection{第三十一章}

\subparagraph{$\bullet$ 唯天下至聖,為能聰明睿知,足以有臨也;寬裕溫柔,足以有容也;發強剛毅,足以有執也;齊莊中正,足以有敬也;文理密察,足以有別也。} ~\\

朱熹《中庸章句》:知,去聲。齊,側皆反。別,彼列反。聰明睿知,生知之質。臨,謂居上而臨下也。其下四者,乃仁義禮知之德。文,文章也。理,條理也。密,詳細也。察,明辯也。

\subparagraph{$\bullet$ 溥博淵泉,而時出之。} ~\\

朱熹《中庸章句》:溥博,周遍而廣闊也。淵泉,靜深而有本也。出,發見也。言五者之德,充積於中,而以時發見於外也。

\subparagraph{$\bullet$ 溥博如天,淵泉如淵。見而民莫不敬,言而民莫不信,行而民莫不說。} ~\\

朱熹《中庸章句》:見,音現。說,音悅。言其充積極其盛,而發見當其可也。

\subparagraph{$\bullet$ 是以聲名洋溢乎中國,施及蠻貊;舟車所至,人力所通;天之所覆,地之所載,日月所照,霜露所隊;凡有血氣者,莫不尊親,故曰配天。} ~\\

朱熹《中庸章句》:施,去聲。隊,音墜。舟車所至以下,蓋極言之。配天,言其德之所及,廣大如天也。第三十一章,承上章而言小德之川流,亦天道也。

\newpage
\subsection{第三十二章}

\subparagraph{$\bullet$ 唯天下至誠,為能經綸天下之大經,立天下之大本,知天地之化育。夫焉有所倚?} ~\\

朱熹《中庸章句》:夫,音扶。焉,於虔反。經,綸,皆治絲之事。經者,理其緒而分之;綸者,比其類而合之也。經,常也。大經者,五品之人倫。大本者,所性之全體也。惟聖人之德極誠無妄,故於人倫各盡其當然之實,而皆可以為天下後世法,所謂經綸之也。其於所性之全體,無一毫人欲之偽以雜之,而天下之道千變萬化皆由此出,所謂立之也。其於天地之化育,則亦其極誠無妄者有默契焉,非但聞見之知而已。此皆至誠無妄,自然之功用,夫豈有所倚著於物而後能哉。

\subparagraph{$\bullet$ 肫肫其仁!淵淵其淵!浩浩其天!} ~\\

朱熹《中庸章句》:肫肫,懇至貌,以經綸而言也。淵淵,靜深貌,以立本而言也。浩浩,廣大貌,以知化而言也。其淵其天,則非特如之而已。

\subparagraph{$\bullet$ 苟不固聰明聖知達天德者,其孰能知之?} ~\\

朱熹《中庸章句》:

\begin{adjustwidth}{2cm}{1cm}
\indent\indent 聖知之知,去聲。固,猶實也。鄭氏曰:「惟聖人能知聖人也。」

第三十二章,承上章而言大德之敦化,亦天道也。前章言至聖之德,此章言至誠之道。然至誠之道,非至聖不能知;至聖之德,非至誠不能為,則亦非二物矣。此篇言聖人天道之極致,至此而無以加矣。
\end{adjustwidth}

\newpage
\subsection{第三十三章}

\subparagraph{$\bullet$ 詩曰「衣錦尚絅」,惡其文之著也。故君子之道,闇然而日章;小人之道,的然而日亡。君子之道:淡而不厭,簡而文,溫而理,知遠之近,知風之自,知微之顯,可與入德矣。} ~\\

朱熹《中庸章句》:衣,去聲。惡,去聲。前章言聖人之德,極其盛矣。此復自下學立心之始言之,而下文又推之以至其極也。詩國風衛碩人、鄭之丰,皆作「衣錦褧衣」。褧、絅同。襌衣也。尚,加也。古之學者為己,故其立心如此。尚絅故闇然,衣錦故有日章之實。淡、簡、溫,絅之襲於外也;不厭而文且理焉,錦之美在中也。小人反是,則暴於外而無實以繼之,是以的然而日亡也。遠之近,見於彼者由於此也。風之自,著乎外者本乎內也。微之顯,有諸內者形諸外也。有為己之心,而又知此三者,則知所謹而可入德矣。故下文引詩言謹獨之事。

\subparagraph{$\bullet$ 詩云:「潛雖伏矣,亦孔之昭!」故君子內省不疚,無惡於志。君子之所不可及者,其唯人之所不見乎。} ~\\

朱熹《中庸章句》:惡,去聲。詩小雅正月之篇。承上文言「莫見乎隱、莫顯乎微」也。疚,病也。無惡於志,猶言無愧於心,此君子謹獨之事也。

\subparagraph{$\bullet$ 詩云:「相在爾室,尚不愧于屋漏。」故君子不動而敬,不言而信。} ~\\

朱熹《中庸章句》:相,去聲。詩大雅抑之篇。相,視也。屋漏,室西北隅也。承上文又言君子之戒謹恐懼,無時不然,不待言動而後敬信,則其為己之功益加密矣。故下文引詩并言其效。

\subparagraph{$\bullet$ 詩曰:「奏假無言,時靡有爭。」是故君子不賞而民勸,不怒而民威於鈇鉞。} ~\\

朱熹《中庸章句》:鈇,音夫。詩商頌烈祖之篇。奏,進也。承上文而遂及其效,言進而感格於神明之際,極其誠敬,無有言說而人自化之也。威,畏也。鈇,莝斫刀也。鉞,斧也。

\subparagraph{$\bullet$ 詩曰:「不顯惟德!百辟其刑之。」是故君子篤恭而天下平。} ~\\

朱熹《中庸章句》:詩周頌烈文之篇。不顯,說見二十六章,此借引以為幽深玄遠之意。承上文言天子有不顯之德,而諸侯法之,則其德愈深而效愈遠矣。篤,厚也。篤恭,言不顯其敬也。篤恭而天下平,乃聖人至德淵微,自然之應,中庸之極功也。

\subparagraph{$\bullet$ 詩云:「予懷明德,不大聲以色。」子曰:「聲色之於以化民,末也。」詩曰:「德輶如毛」,毛猶有倫。「上天之載,無聲無臭」,至矣!} ~\\

朱熹《中庸章句》:

\begin{adjustwidth}{2cm}{1cm}
\indent\indent 詩大雅皇矣之篇。引之以明上文所謂不顯之德者,正以其不大聲與色也。又引孔子之言,以為聲色乃化民之末務,今但言不大之而已,則猶有聲色者存,是未足以形容不顯之妙。不若烝民之詩所言「德輶如毛」,則庶乎可以形容矣,而又自以為謂之毛,則猶有可比者,是亦未盡其妙。不若文王之詩所言「上天之事,無聲無臭」,然後乃為不顯之至耳。蓋聲臭有氣無形,在物最為微妙,而猶曰無之,故惟此可以形容不顯篤恭之妙。非此德之外,又別有是三等,然後為至也。

第三十三章,子思因前章極致之言,反求其本,復自下學為己謹獨之事,推而言之,以馴致乎篤恭而天下平之盛。又贊其妙,至於無聲無臭而後已焉。蓋舉一篇之要而約言之,其反復丁寧示人之意,至深切矣,學者其可不盡心乎!
\end{adjustwidth}

\newpage
\section{中庸章句序}

{\kaishu 胡炳文曰:「大學中不出性字,故朱子於序言性詳焉;中庸中不出心字,故此序言心詳焉。」}

中庸何為而作也?子思子憂道學之失其傳而作也。蓋自上古聖神繼天立極,而道統之傳有自來矣,其見於經則「允執厥中」者,堯之所以授舜也;「人心惟危,道心惟微,惟精惟一,允執厥中」\footnote{程若庸曰:「人生而靜氣,未用事未有人與道之分,但謂之心而已。感物而動,始有人心、道心之分焉,精一執中皆是動時工夫。」}者,舜之所以授禹也。堯之一言,至矣,盡矣!而舜復益之以三言者,則所以明夫堯之一言,必如是而後可庶幾也。\footnote{胡炳文曰:「執之工夫只在精一上,堯授舜曰『允執厥中』如夫子語曾子『以一貫』。舜必由精後而後執中,是猶曾子告門人必由忠恕而達於一貫也。」}

蓋嘗論之:心之虛靈知覺\footnote{程若庸曰:「虛靈心之体,知覺心之用。」},一而已矣,而以為有人心、道心之異者,則以其或生於形氣之私,或原於性命之正\footnote{朱熹曰:「有形氣之私方有人心故曰生,自賦命受性之初便有道心故曰原。」},而所以為知覺者不同,是以或危殆而不安,或微妙而難見耳\footnote{《朱子語錄》:「只是危險在欲墮未墮之間,易流於不好耳。微者難明,有時發見些子使自家見得,有時又不見了。」}。然人莫不有是形,故雖上智不能無人心,亦莫不有是性,故雖下愚不能無道心,二者雜於方寸之間,而不知所以治之,則危者愈危,微者愈微\footnote{陳櫟曰:「危愈危流於惡,微愈微幾於無。」},而天理之公卒無以勝夫人欲之私矣。精則察夫二者之間而不雜也,一則守其本心之正而不離也。從事於斯,無少閒斷,必使道心常為一身之主,而人心每聽命焉\footnote{《朱子語錄》:「問人心可無否?曰如何無得,但以道心為主而人心每聼道心之區處方可。」},則危者安、微者著,而動靜云為自無過不及之差矣。

夫堯、舜、禹,天下之大聖也。以天下相傳,天下之大事也。以天下之大聖,行天下之大事,而其授受之際,丁寧告戒,不過如此,則天下之理,豈有以加於此哉?自是以來,聖聖相承:若成湯、文、武之為君,皋陶、伊、傅、周、召之為臣,既皆以此而接夫道統之傳,若吾夫子,則雖不得其位,而所以繼往聖、開來學,其功反有賢於堯舜者。

然當是時,見而知之者,惟顏氏、曾氏之傳得其宗。及曾氏之再傳,而復得夫子之孫子思,則去聖遠而異端起矣。子思懼夫愈久而愈失其真也,於是推本堯舜以來相傳之意,質以平日所聞父師之言,更互演繹,作為此書,以詔後之學者。蓋其憂之也深,故其言之也切;其慮之也遠,故其說之也詳。其曰「天命率性」,則道心之謂也;其曰「擇善固執」,則精一之謂也;其曰「君子時中」,則執中之謂也。世之相後,千有餘年,而其言之不異,如合符節。歷選前聖之書,所以提挈綱維,開示蘊奧,未有若是之明且盡者也。

自是而又再傳,以得孟氏,為能推明是書,以承先聖之統,及其沒而遂失其傳焉。則吾道之所寄,不越乎言語文字之間,而異端之說,日新月盛,以至於老佛之徒出,則彌近理而大亂真矣。然而尚幸此書之不泯,故程夫子兄弟者出,得有所考,以續夫千載不傳之緒;得有所據,以斥夫二家似是之非。蓋子思之功於是為大,而微程夫子,則亦莫能因其語而得其心也。惜乎!其所以為說者不傳,而凡石氏之所輯錄,僅出於其門人之所記,是以大義雖明,而微言未析。至其門人所自為說,則雖頗詳盡而多所發明,然倍其師說而淫於老佛者,亦有之矣。

熹自蚤歲即嘗受讀而竊疑之,沈潛反覆,蓋亦有年,一旦恍然似有以得其要領者,然後乃敢會眾說而折其中,既為定著章句一篇,以俟後之君子,而一二同志。復取石氏書,刪其繁亂,名以輯略,且記所嘗論辯取捨之意,別為或問,以附其後。然後此書之旨,支分節解,脈絡貫通、詳略相因、巨細畢舉,而凡諸說之同異得失,亦得以曲暢旁通,而各極其趣。雖於道統之傳,不敢妄議,然初學之士,或有取焉,則亦庶乎升高行遠之一助云爾。

淳熙己酉春三月戊申\footnote{公時年六十。},新安朱熹序
\end{document}
